\documentclass[dvipdfmx]{jsarticle}
\setcounter{section}{3}
\setcounter{subsection}{8}
\usepackage{xr}
\externaldocument{2.2.1}
\externaldocument{2.2.2}
\externaldocument{2.2.3}
\externaldocument{2.2.4}
\externaldocument{2.3.6}
\externaldocument{2.3.7}
\externaldocument{2.3.8}
\usepackage{amsmath,amsfonts,amssymb,array,comment,mathtools,url,docmute}
\usepackage{longtable,booktabs,dcolumn,tabularx,mathtools,multirow,colortbl,xcolor}
\usepackage[dvipdfmx]{graphics}
\usepackage{bmpsize}
\usepackage{amsthm}
\usepackage{enumitem}
\setlistdepth{20}
\renewlist{itemize}{itemize}{20}
\setlist[itemize]{label=•}
\renewlist{enumerate}{enumerate}{20}
\setlist[enumerate]{label=\arabic*.}
\setcounter{MaxMatrixCols}{20}
\setcounter{tocdepth}{3}
\newcommand{\rotin}{\text{\rotatebox[origin=c]{90}{$\in $}}}
\renewcommand{\thesection}{第\arabic{section}部}
\renewcommand{\thesubsection}{\arabic{section}.\arabic{subsection}}
\renewcommand{\thesubsubsection}{\arabic{section}.\arabic{subsection}.\arabic{subsubsection}}
\everymath{\displaystyle}
\allowdisplaybreaks[4]
\usepackage{vtable}
\theoremstyle{definition}
\newtheorem{thm}{定理}[subsection]
\newtheorem*{thm*}{定理}
\newtheorem{dfn}{定義}[subsection]
\newtheorem*{dfn*}{定義}
\newtheorem{axs}[dfn]{公理}
\newtheorem*{axs*}{公理}
\renewcommand{\headfont}{\bfseries}
\makeatletter
  \renewcommand{\section}{%
    \@startsection{section}{1}{\z@}%
    {\Cvs}{\Cvs}%
    {\normalfont\huge\headfont\raggedright}}
\makeatother
\makeatletter
  \renewcommand{\subsection}{%
    \@startsection{subsection}{2}{\z@}%
    {0.5\Cvs}{0.5\Cvs}%
    {\normalfont\LARGE\headfont\raggedright}}
\makeatother
\makeatletter
  \renewcommand{\subsubsection}{%
    \@startsection{subsubsection}{3}{\z@}%
    {0.4\Cvs}{0.4\Cvs}%
    {\normalfont\Large\headfont\raggedright}}
\makeatother
\makeatletter
\renewenvironment{proof}[1][\proofname]{\par
  \pushQED{\qed}%
  \normalfont \topsep6\p@\@plus6\p@\relax
  \trivlist
  \item\relax
  {
  #1\@addpunct{.}}\hspace\labelsep\ignorespaces
}{%
  \popQED\endtrivlist\@endpefalse
}
\makeatother
\renewcommand{\proofname}{\textbf{証明}}
\usepackage{tikz,graphics}
\usepackage[dvipdfmx]{hyperref}
\usepackage{pxjahyper}
\hypersetup{
 setpagesize=false,
 bookmarks=true,
 bookmarksdepth=tocdepth,
 bookmarksnumbered=true,
 colorlinks=false,
 pdftitle={},
 pdfsubject={},
 pdfauthor={},
 pdfkeywords={}}
\begin{document}
%\hypertarget{spectrumux5206ux89e3}{%
\subsection{spectrum分解}%\label{spectrumux5206ux89e3}}
%\hypertarget{toeplitzux306eux5b9aux7406}{%
\subsubsection{Toeplitzの定理}%\label{toeplitzux306eux5b9aux7406}}
\begin{dfn*}[定義\ref{f-不変の定義}の再掲]
体$K$上のvector空間$V$の部分空間$W$、線形写像$f:V \rightarrow V$が与えられたとき、$V\left( f|W \right) \subseteq W$が成り立つなら、その部分空間$W$は$f$-不変であるという。
\end{dfn*}
\begin{thm}\label{2.3.9.1}
体$\mathbb{C}$上の$1 \leq \dim V = n$なる内積空間$(V,\varPhi)$が与えられたとき、$\forall f,g \in L(V,V)$に対し、$g \circ f = f \circ g$が成り立つなら、$\exists\mathbf{v} \in V$に対し、そのvector$\mathbf{v}$がそれらの線形写像たち$f$、$g$どちらともの固有vectorとなる。
\end{thm}
\begin{proof}
体$\mathbb{C}$上の$1 \leq \dim V = n$なる内積空間$(V,\varPhi)$が与えられたとき、$\forall f,g \in L(V,V)$に対し、$g \circ f = f \circ g$が成り立つなら、定理\ref{2.2.2.5}、定理\ref{2.2.2.6}よりそれらの線形写像たち$f$、$g$どちらとも固有値たちと固有vectorを必ずもつ。その線形写像$f$の固有値$\lambda$に対する固有空間を$W_{f}(\lambda)$とおくと、$\left\{ \mathbf{0} \right\} \subset W_{f}(\lambda)$が成り立ち、$\forall\mathbf{v} \in W_{f}(\lambda)$に対し、次のようになるので、
\begin{align*}
\left( \lambda I_{V} - f \right)\left( g\left( \mathbf{v} \right) \right) &= \left( \lambda I_{V} - f \right) \circ g\left( \mathbf{v} \right)\\
&= \left( \lambda I_{V} \circ g - f \circ g \right)\left( \mathbf{v} \right)\\
&= \lambda g\left( \mathbf{v} \right) - f \circ g\left( \mathbf{v} \right)\\
&= \lambda g\left( \mathbf{v} \right) - g \circ f\left( \mathbf{v} \right)\\
&= \lambda g\left( \mathbf{v} \right) - g \circ \left( - \lambda I_{V} + f + \lambda I_{V} \right)\left( \mathbf{v} \right)\\
&= \lambda g\left( \mathbf{v} \right) + g \circ \left( \lambda I_{V} - f \right)\left( \mathbf{v} \right) - g \circ \lambda I_{V}\left( \mathbf{v} \right)\\
&= \lambda g\left( \mathbf{v} \right) + g\left( \left( \lambda I_{V} - f \right)\left( \mathbf{v} \right) \right) - g\left( \lambda I_{V}\left( \mathbf{v} \right) \right)\\
&= \lambda g\left( \mathbf{v} \right) + g\left( \mathbf{0} \right) - \lambda g\left( \mathbf{v} \right)\\
&= g\left( \mathbf{0} \right) = \mathbf{0}
\end{align*}
$V\left( g|W_{f}(\lambda) \right) \subseteq W_{f}(\lambda)$が成り立つ。ゆえに、その固有空間$W_{f}(\lambda)$は$g$-不変である。\par
これにより、定理\ref{2.2.2.5}、定理\ref{2.2.2.6}よりその線形写像$g|W_{f}(\lambda)$の固有vector$\mathbf{w}$が存在する。もちろん、その固有vector$\mathbf{w}$はその線形写像$g$の固有vectorでもあるし、さらにいえば、$\mathbf{w} \in W_{f}(\lambda)$が成り立つので、その固有vector$\mathbf{w}$はその線形写像$f$の固有vectorでもある。
\end{proof}
\begin{thm}\label{2.3.9.2}
体$\mathbb{C}$上の$1 \leq \dim V = n$なる内積空間$(V,\varPhi)$が与えられたとき、$\forall f,g \in L(V,V)$に対し、$g \circ f = f \circ g$が成り立つなら、そのvector空間$V$のある正規直交基底$\mathcal{B}$が存在して、これに関するそれらの線形写像たち$f$、$g$の表現行列たち$[ f]_{\mathcal{B}}^{\mathcal{B}}$、$[ g]_{\mathcal{B}}^{\mathcal{B}}$が上三角行列となる。
\end{thm}
\begin{proof}
体$\mathbb{C}$上の$1 \leq \dim V = n$なる内積空間$(V,\varPhi)$が与えられたとき、$\forall f,g \in L(V,V)$に対し、$g \circ f = f \circ g$が成り立つなら、$n = 1$のときでは明らかであるので、$n = k$のときそのvector空間$V$のある正規直交基底が存在して、これに関するそれらの線形写像たち$f$、$g$の表現行列たちが上三角行列となると仮定しよう。\par
$n = k + 1$のとき、次のようになるので、
\begin{align*}
g^{*} \circ f^{*} = (f \circ g)^{*} = (g \circ f)^{*} = f^{*} \circ g^{*}
\end{align*}
定理\ref{2.3.9.1}よりそれらの線形写像たち$f^{*}$、$g^{*}$どちらともの固有vector$\widetilde{\mathbf{v}}$が存在する。このとき、定理\ref{2.3.7.10}より$V = {\mathrm{span}}\left\{ \widetilde{\mathbf{v}} \right\} \oplus {{\mathrm{span}}\left\{ \widetilde{\mathbf{v}} \right\}}^{\bot}$が成り立つので、次のようになる。
\begin{align*}
\dim{{\mathrm{span}}\left\{ \widetilde{\mathbf{v}} \right\}}^{\bot} &= \dim{{\mathrm{span}}\left\{ \widetilde{\mathbf{v}} \right\}} + \dim{{\mathrm{span}}\left\{ \widetilde{\mathbf{v}} \right\}}^{\bot} - \dim{{\mathrm{span}}\left\{ \widetilde{\mathbf{v}} \right\}}\\
&= \dim{{\mathrm{span}}\left\{ \widetilde{\mathbf{v}} \right\} \oplus {{\mathrm{span}}\left\{ \widetilde{\mathbf{v}} \right\}}^{\bot}} - \dim{{\mathrm{span}}\left\{ \widetilde{\mathbf{v}} \right\}}\\
&= \dim V - \dim{{\mathrm{span}}\left\{ \widetilde{\mathbf{v}} \right\}}\\
&= k + 1 - 1 = k
\end{align*}\par
さらに、$\forall\mathbf{v} \in {{\mathrm{span}}\left\{ \widetilde{\mathbf{v}} \right\}}^{\bot}$に対し、その固有vector$\widetilde{\mathbf{v}}$に対する固有値を$\lambda$とおくと、次のようになるので、
\begin{align*}
\varPhi\left( \widetilde{\mathbf{v}},f\left( \mathbf{v} \right) \right) &= \varPhi\left( f^{*}\left( \widetilde{\mathbf{v}} \right),\mathbf{v} \right)\\
&= \varPhi\left( f^{*}\left( \widetilde{\mathbf{v}} \right),\mathbf{v} \right)\\
&= \varPhi\left( \lambda\widetilde{\mathbf{v}},\mathbf{v} \right)\\
&= \overline{\lambda}\varPhi\left( \widetilde{\mathbf{v}},\mathbf{v} \right)\\
&= \overline{\lambda} \cdot 0 = 0
\end{align*}
$V\left( f|{{\mathrm{span}}\left\{ \widetilde{\mathbf{v}} \right\}}^{\bot} \right) \subseteq {{\mathrm{span}}\left\{ \widetilde{\mathbf{v}} \right\}}^{\bot}$が成り立つ。ゆえに、その直交空間${{\mathrm{span}}\left\{ \widetilde{\mathbf{v}} \right\}}^{\bot}$は$f$-不変である。同様にして、その直交空間${{\mathrm{span}}\left\{ \widetilde{\mathbf{v}} \right\}}^{\bot}$は$g$-不変であることも示される。\par
これにより、線形写像たち$f|{{\mathrm{span}}\left\{ \widetilde{\mathbf{v}} \right\}}^{\bot}$、$g|{{\mathrm{span}}\left\{ \widetilde{\mathbf{v}} \right\}}^{\bot}$が考えられれば、$g \circ f = f \circ g$が成り立つかつ、$\dim{{\mathrm{span}}\left\{ \widetilde{\mathbf{v}} \right\}}^{\bot} = k$が成り立つので、仮定よりその部分空間${{\mathrm{span}}\left\{ \widetilde{\mathbf{v}} \right\}}^{\bot}$のある正規直交基底$\widetilde{\mathcal{B}}$が存在して、これに関するそれらの線形写像たち$f|{{\mathrm{span}}\left\{ \widetilde{\mathbf{v}} \right\}}^{\bot}$、$g|{{\mathrm{span}}\left\{ \widetilde{\mathbf{v}} \right\}}^{\bot}$の表現行列たち$\left[ f|{{\mathrm{span}}\left\{ \widetilde{\mathbf{v}} \right\}}^{\bot} \right]_{\widetilde{\mathcal{B}}}^{\widetilde{\mathcal{B}}}$、$\left[ g|{{\mathrm{span}}\left\{ \widetilde{\mathbf{v}} \right\}}^{\bot} \right]_{\widetilde{\mathcal{B}}}^{\widetilde{\mathcal{B}}}$が上三角行列となる。\par
そこで、$\widetilde{\mathbf{v}} \neq \mathbf{0}$が成り立つことに注意すれば、その内積空間$(V,\varPhi)$から誘導されるnorm空間$\left( V,\varphi_{\varPhi} \right)$において、$\mathcal{B} =\left\langle \begin{matrix}
\widetilde{\mathcal{B}} & \frac{\widetilde{\mathbf{v}}}{\varphi_{\varPhi}\left( \widetilde{\mathbf{v}} \right)} \\
\end{matrix} \right\rangle$、$\widetilde{\mathcal{B}} = \left\langle \mathbf{o}_{i} \right\rangle_{i \in \varLambda_{k}}$とおかれると、その組$\mathcal{B}$はそのvector空間$V$の基底で、$\forall i \in \varLambda_{k}$に対し、$\mathbf{o}_{i} \in {{\mathrm{span}}\left\{ \widetilde{\mathbf{v}} \right\}}^{\bot}$が成り立つことから次のようになるので、
\begin{align*}
\varPhi\left( \frac{\widetilde{\mathbf{v}}}{\varphi_{\varPhi}\left( \widetilde{\mathbf{v}} \right)},\mathbf{o}_{i} \right) = \frac{\varPhi\left( \widetilde{\mathbf{v}},\mathbf{o}_{i} \right)}{\varphi_{\varPhi}\left( \widetilde{\mathbf{v}} \right)} = \frac{0}{\varphi_{\varPhi}\left( \widetilde{\mathbf{v}} \right)} = 0,\ \ \varphi_{\varPhi}\left( \frac{\widetilde{\mathbf{v}}}{\varphi_{\varPhi}\left( \widetilde{\mathbf{v}} \right)} \right) = \frac{\varphi_{\varPhi}\left( \widetilde{\mathbf{v}} \right)}{\varphi_{\varPhi}\left( \widetilde{\mathbf{v}} \right)} = 1
\end{align*}
その基底$\mathcal{B}$はそのvector空間$V$の正規直交基底をなす。\par
さらに、$A_{nn} \in M_{nn}\left( \mathbb{C} \right)$なる行列$A_{nn}$が対応する行列となっている線形写像が$L_{A_{nn}}:\mathbb{C}^{n} \rightarrow \mathbb{C}^{n};\mathbf{v} \mapsto A_{nn}\mathbf{v}$とおかれ、$\forall\mathbf{v} \in \mathbb{C}^{k + 1}$に対し、$\mathbf{v} = \begin{pmatrix}
\mathbf{w} \\
v \\
\end{pmatrix}$とおかれると、そのvector空間$\mathbb{C}^{k + 1}$の標準直交基底を$\left\langle \mathbf{e}_{i} \right\rangle_{i \in \varLambda_{k + 1}}$として次のようになるので、
\begin{align*}
L_{[ f]_{\mathcal{B}}^{\mathcal{B}}}\left( \mathbf{v} \right) &= \varphi_{\mathcal{B}}^{- 1} \circ f \circ \varphi_{\mathcal{B}}\left( \mathbf{v} \right)\\
&= \varphi_{\mathcal{B}}^{- 1} \circ f \circ \varphi_{\mathcal{B}}\begin{pmatrix}
\mathbf{w} \\
v \\
\end{pmatrix}\\
&= \varphi_{\mathcal{B}}^{- 1} \circ f \circ \varphi_{\mathcal{B}}\left( \begin{pmatrix}
\mathbf{w} \\
0 \\
\end{pmatrix} + \begin{pmatrix}
\mathbf{0} \\
v \\
\end{pmatrix} \right)\\
&= \varphi_{\mathcal{B}}^{- 1} \circ f \circ \varphi_{\mathcal{B}}\begin{pmatrix}
\mathbf{w} \\
0 \\
\end{pmatrix} + \varphi_{\mathcal{B}}^{- 1} \circ f \circ \varphi_{\mathcal{B}}\begin{pmatrix}
\mathbf{0} \\
v \\
\end{pmatrix}\\
&= \varphi_{\mathcal{B}}^{- 1} \circ f\left( \varphi_{\mathcal{B}}\begin{pmatrix}
\mathbf{w} \\
0 \\
\end{pmatrix} \right) + v\varphi_{\mathcal{B}}^{- 1} \circ f \circ \varphi_{\mathcal{B}}\begin{pmatrix}
\mathbf{0} \\
1 \\
\end{pmatrix}\\
&= \varphi_{\mathcal{B}}^{- 1} \circ f\left( \varphi_{\widetilde{\mathcal{B}}}\left( \mathbf{w} \right) \right) + v\varphi_{\mathcal{B}}^{- 1} \circ f \circ \varphi_{\mathcal{B}}\begin{pmatrix}
\mathbf{0} \\
1 \\
\end{pmatrix}\\
&= \varphi_{\mathcal{B}}^{- 1} \circ f|{{\mathrm{span}}\left\{ \widetilde{\mathbf{v}} \right\}}^{\bot} \circ \varphi_{\widetilde{\mathcal{B}}}\left( \mathbf{w} \right) + v\varphi_{\mathcal{B}}^{- 1} \circ f \circ \varphi_{\mathcal{B}}\begin{pmatrix}
\mathbf{0} \\
1 \\
\end{pmatrix}\\
&= \varphi_{\mathcal{B}}^{- 1}\left( f|{{\mathrm{span}}\left\{ \widetilde{\mathbf{v}} \right\}}^{\bot} \circ \varphi_{\widetilde{\mathcal{B}}}\left( \mathbf{w} \right) \right) + v\varphi_{\mathcal{B}}^{- 1} \circ f \circ \varphi_{\mathcal{B}}\begin{pmatrix}
\mathbf{0} \\
1 \\
\end{pmatrix}\\
&= \begin{pmatrix}
\varphi_{\widetilde{\mathcal{B}}}^{- 1}\left( f|{{\mathrm{span}}\left\{ \widetilde{\mathbf{v}} \right\}}^{\bot} \circ \varphi_{\widetilde{\mathcal{B}}}\left( \mathbf{w} \right) \right) \\
0 \\
\end{pmatrix} + v\varphi_{\mathcal{B}}^{- 1} \circ f \circ \varphi_{\mathcal{B}}\begin{pmatrix}
\mathbf{0} \\
1 \\
\end{pmatrix}\\
&= \begin{pmatrix}
\varphi_{\widetilde{\mathcal{B}}}^{- 1} \circ f|{{\mathrm{span}}\left\{ \widetilde{\mathbf{v}} \right\}}^{\bot} \circ \varphi_{\widetilde{\mathcal{B}}}\left( \mathbf{w} \right) \\
0 \\
\end{pmatrix} + v\varphi_{\mathcal{B}}^{- 1} \circ f \circ \varphi_{\mathcal{B}}\begin{pmatrix}
\mathbf{0} \\
1 \\
\end{pmatrix}\\
&= \begin{pmatrix}
\left[ f|{{\mathrm{span}}\left\{ \widetilde{\mathbf{v}} \right\}}^{\bot} \right]_{\widetilde{\mathcal{B}}}^{\widetilde{\mathcal{B}}}\mathbf{w} \\
0 \\
\end{pmatrix} + v[ f]_{\mathcal{B}}^{\mathcal{B}}\begin{pmatrix}
\mathbf{0} \\
1 \\
\end{pmatrix}\\
&= \begin{pmatrix}
\begin{matrix}
\left[ f|{{\mathrm{span}}\left\{ \widetilde{\mathbf{v}} \right\}}^{\bot} \right]_{\widetilde{\mathcal{B}}}^{\widetilde{\mathcal{B}}} \\
O \\
\end{matrix} \\
\end{pmatrix}\mathbf{w} + [ f]_{\mathcal{B}}^{\mathcal{B}}\begin{pmatrix}
\mathbf{0} \\
1 \\
\end{pmatrix} \cdot 0 + \begin{pmatrix}
\left[ f|{{\mathrm{span}}\left\{ \widetilde{\mathbf{v}} \right\}}^{\bot} \right]_{\widetilde{\mathcal{B}}}^{\widetilde{\mathcal{B}}} \\
O \\
\end{pmatrix}\mathbf{0} + [ f]_{\mathcal{B}}^{\mathcal{B}}\begin{pmatrix}
\mathbf{0} \\
1 \\
\end{pmatrix}v\\
&= \begin{pmatrix}
\begin{matrix}
\left[ f|{{\mathrm{span}}\left\{ \widetilde{\mathbf{v}} \right\}}^{\bot} \right]_{\widetilde{\mathcal{B}}}^{\widetilde{\mathcal{B}}} \\
O \\
\end{matrix} & [ f]_{\mathcal{B}}^{\mathcal{B}}\begin{pmatrix}
\mathbf{0} \\
1 \\
\end{pmatrix} \\
\end{pmatrix}\begin{pmatrix}
\mathbf{w} \\
0 \\
\end{pmatrix} + \begin{pmatrix}
\begin{matrix}
\left[ f|{{\mathrm{span}}\left\{ \widetilde{\mathbf{v}} \right\}}^{\bot} \right]_{\widetilde{\mathcal{B}}}^{\widetilde{\mathcal{B}}} \\
O \\
\end{matrix} & [ f]_{\mathcal{B}}^{\mathcal{B}}\begin{pmatrix}
\mathbf{0} \\
1 \\
\end{pmatrix} \\
\end{pmatrix}\begin{pmatrix}
\mathbf{0} \\
v \\
\end{pmatrix}\\
&= \begin{pmatrix}
\begin{matrix}
\left[ f|{{\mathrm{span}}\left\{ \widetilde{\mathbf{v}} \right\}}^{\bot} \right]_{\widetilde{\mathcal{B}}}^{\widetilde{\mathcal{B}}} \\
O \\
\end{matrix} & [ f]_{\mathcal{B}}^{\mathcal{B}}\begin{pmatrix}
\mathbf{0} \\
1 \\
\end{pmatrix} \\
\end{pmatrix}\left( \begin{pmatrix}
\mathbf{w} \\
0 \\
\end{pmatrix} + \begin{pmatrix}
\mathbf{0} \\
v \\
\end{pmatrix} \right)\\
&= \begin{pmatrix}
\begin{matrix}
\left[ f|{{\mathrm{span}}\left\{ \widetilde{\mathbf{v}} \right\}}^{\bot} \right]_{\widetilde{\mathcal{B}}}^{\widetilde{\mathcal{B}}} \\
O \\
\end{matrix} & [ f]_{\mathcal{B}}^{\mathcal{B}}\begin{pmatrix}
\mathbf{0} \\
1 \\
\end{pmatrix} \\
\end{pmatrix}\begin{pmatrix}
\mathbf{w} \\
v \\
\end{pmatrix}\\
&= \begin{pmatrix}
\begin{matrix}
\left[ f|{{\mathrm{span}}\left\{ \widetilde{\mathbf{v}} \right\}}^{\bot} \right]_{\widetilde{\mathcal{B}}}^{\widetilde{\mathcal{B}}} \\
O \\
\end{matrix} & [ f]_{\mathcal{B}}^{\mathcal{B}}\begin{pmatrix}
\mathbf{0} \\
1 \\
\end{pmatrix} \\
\end{pmatrix}\mathbf{v}\\
&= L_{\scriptsize \begin{pmatrix}
\begin{matrix}
\left[ f|{{\mathrm{span}}\left\{ \widetilde{\mathbf{v}} \right\}}^{\bot} \right]_{\widetilde{\mathcal{B}}}^{\widetilde{\mathcal{B}}} \\
O \\
\end{matrix} & [ f]_{\mathcal{B}}^{\mathcal{B}}\begin{pmatrix}
\mathbf{0} \\
1 \\
\end{pmatrix} \\
\end{pmatrix}}\left( \mathbf{v} \right)
\end{align*}
$[ f]_{\mathcal{B}}^{\mathcal{B}} = \begin{pmatrix}
\begin{matrix}
\left[ f|{{\mathrm{span}}\left\{ \widetilde{\mathbf{v}} \right\}}^{\bot} \right]_{\widetilde{\mathcal{B}}}^{\widetilde{\mathcal{B}}} \\
O \\
\end{matrix} & [ f]_{\mathcal{B}}^{\mathcal{B}}\begin{pmatrix}
\mathbf{0} \\
1 \\
\end{pmatrix} \\
\end{pmatrix}$が得られる。ここで、その行列$\left[ f|{{\mathrm{span}}\left\{ \widetilde{\mathbf{v}} \right\}}^{\bot} \right]_{\widetilde{\mathcal{B}}}^{\widetilde{\mathcal{B}}}$は上三角行列であるので、その行列$[ f]_{\mathcal{B}}^{\mathcal{B}}$も上三角行列である。その線形写像$g$についても同様にして示される。\par
以上数学的帰納法により、$g \circ f = f \circ g$が成り立つなら、そのvector空間$V$のある正規直交基底$\mathcal{B}$が存在して、これに関するそれらの線形写像たち$f$、$g$の表現行列たち$[ f]_{\mathcal{B}}^{\mathcal{B}}$、$[ g]_{\mathcal{B}}^{\mathcal{B}}$が上三角行列となることが示された。
\end{proof}
\begin{thm}\label{2.3.9.3}
体$\mathbb{C}$上の$1 \leq \dim V = n$なる内積空間$(V,\varPhi)$が与えられたとき、$\forall f \in L(V,V)$に対し、そのvector空間$V$のある正規直交基底$\mathcal{B}$が存在して、これに関するそれらの線形写像$f$の表現行列$[ f]_{\mathcal{B}}^{\mathcal{B}}$が上三角行列となる。
\end{thm}
\begin{proof} 定理\ref{2.3.9.2}における線形写像$g$を$g = f$または$g = I_{V}$とおかれればよい。
\end{proof}
\begin{thm}[Toeplitzの定理]\label{2.3.9.4}
体$\mathbb{C}$上の$1 \leq \dim V = n$なる内積空間$(V,\varPhi)$が与えられたとき、$\forall f \in L(V,V)$に対し、そのvector空間$V$のある正規直交基底$\mathcal{B}$が存在して、これに関するその線形写像$f$の表現行列$[ f]_{\mathcal{B}}^{\mathcal{B}}$が対角行列となるならそのときに限り、その線形写像$f$が正規変換である。この定理をToeplitzの定理という。\par
\end{thm}
この主張を行列で考えられれば、$\forall A_{nn} \in M_{nn}\left( \mathbb{C} \right)$に対し、ある$n$次unitary行列$U_{nn}$が存在して、その行列$U_{nn}^{*}A_{nn}U_{nn}$が対角行列となるならそのときに限り、$A_{nn}^{*}A_{nn} = A_{nn}A_{nn}^{*}$が成り立つことになる。
\begin{proof}
体$\mathbb{C}$上の$1 \leq \dim V = n$なる内積空間$(V,\varPhi)$が与えられたとき、$\forall f \in L(V,V)$に対し、そのvector空間$V$のある正規直交基底$\mathcal{B}$が存在して、これに関するその線形写像$f$の表現行列$[ f]_{\mathcal{B}}^{\mathcal{B}}$が対角行列となるなら、定義より${[ f]_{\mathcal{B}}^{\mathcal{B}}}^{*} = \left[ f^{*} \right]_{\mathcal{B}}^{\mathcal{B}}$が成り立つので、その行列$\left[ f^{*} \right]_{\mathcal{B}}^{\mathcal{B}}$も対角行列となって$\left[ f^{*} \right]_{\mathcal{B}}^{\mathcal{B}}[ f]_{\mathcal{B}}^{\mathcal{B}} = [ f]_{\mathcal{B}}^{\mathcal{B}}\left[ f^{*} \right]_{\mathcal{B}}^{\mathcal{B}}$、即ち、${[ f]_{\mathcal{B}}^{\mathcal{B}}}^{*}[ f]_{\mathcal{B}}^{\mathcal{B}} = [ f]_{\mathcal{B}}^{\mathcal{B}}{[ f]_{\mathcal{B}}^{\mathcal{B}}}^{*}$が成り立つ。定理\ref{2.3.8.8}よりよって、その線形写像$f$が正規変換である。\par
逆に、その線形写像$f$が正規変換であるなら、$f^{*} \circ f = f \circ f^{*}$が成り立つので、定理\ref{2.3.9.2}よりそのvector空間$V$のある正規直交基底$\mathcal{B}$が存在して、これに関するそれらの線形写像たち$f$、$f^{*}$の表現行列たち$[ f]_{\mathcal{B}}^{\mathcal{B}}$、$\left[ f^{*} \right]_{\mathcal{B}}^{\mathcal{B}}$が上三角行列となる。このとき、定理\ref{2.3.8.5}より$[ f]_{\mathcal{B}}^{\mathcal{B}} = \left[ f^{**} \right]_{\mathcal{B}}^{\mathcal{B}} = {\left[ f^{*} \right]_{\mathcal{B}}^{\mathcal{B}}}^{*}$が成り立つので、その行列$[ f]_{\mathcal{B}}^{\mathcal{B}}$は下三角行列でもある。ゆえに、その行列$[ f]_{\mathcal{B}}^{\mathcal{B}}$は対角行列となる。
\end{proof}
\begin{thm}\label{2.3.9.5}
体$\mathbb{C}$上の$1 \leq \dim V = n$なる内積空間$(V,\varPhi)$が与えられたとき、$\forall f \in L(V,V)$に対し、その線形写像$f$が正規変換であるとき、その線形写像$f$の固有値たち$\lambda$がすべて$|\lambda| = 1$を満たすなら、その線形写像$f$は等長変換であり、その線形写像$f$の固有値たち$\lambda$がすべて$\lambda = \pm 1$を満たすなら、その線形写像$f$はHermite変換である。
\end{thm}
\begin{proof}
体$\mathbb{C}$上の$1 \leq \dim V = n$なる内積空間$(V,\varPhi)$が与えられたとき、$\forall f \in L(V,V)$に対し、その線形写像$f$が正規変換であるとき、その線形写像$f$の固有値たち$\lambda$がすべて$|\lambda| = 1$を満たすなら、定理\ref{2.3.9.4}、即ち、Toeplitzの定理よりそのvector空間$V$のある正規直交基底$\mathcal{B}$が存在して、これに関するそれらの線形写像$f$の表現行列$[ f]_{\mathcal{B}}^{\mathcal{B}}$が対角行列となるようにすることができる。そこで、定理\ref{2.2.2.12}よりその表現行列$[ f]_{\mathcal{B}}^{\mathcal{B}}$の対角成分がその線形写像$f$の固有値であるので、次のようにおかれれば、
\begin{align*}
[ f]_{\mathcal{B}}^{\mathcal{B}} = \begin{pmatrix}
\lambda_{1} & \  & \  & O \\
\  & \lambda_{2} & \  & \  \\
\  & \  & \ddots & \  \\
O & \  & \  & \lambda_{n} \\
\end{pmatrix}
\end{align*}
次のようになるので、
\begin{align*}
{[ f]_{\mathcal{B}}^{\mathcal{B}}}^{*}[ f]_{\mathcal{B}}^{\mathcal{B}} &= \begin{pmatrix}
\overline{\lambda_{1}} & \  & \  & O \\
\  & \overline{\lambda_{2}} & \  & \  \\
\  & \  & \ddots & \  \\
O & \  & \  & \overline{\lambda_{n}} \\
\end{pmatrix}\begin{pmatrix}
\lambda_{1} & \  & \  & O \\
\  & \lambda_{2} & \  & \  \\
\  & \  & \ddots & \  \\
O & \  & \  & \lambda_{n} \\
\end{pmatrix}\\
&= \begin{pmatrix}
\left| \lambda_{1} \right|^{2} & \  & \  & O \\
\  & \left| \lambda_{2} \right|^{2} & \  & \  \\
\  & \  & \ddots & \  \\
O & \  & \  & \left| \lambda_{n} \right|^{2} \\
\end{pmatrix}\\
&= \begin{pmatrix}
1 & \  & \  & O \\
\  & 1 & \  & \  \\
\  & \  & \ddots & \  \\
O & \  & \  & 1 \\
\end{pmatrix} = I_{n}\\
[ f]_{\mathcal{B}}^{\mathcal{B}}{[ f]_{\mathcal{B}}^{\mathcal{B}}}^{*} &= \begin{pmatrix}
\lambda_{1} & \  & \  & O \\
\  & \lambda_{2} & \  & \  \\
\  & \  & \ddots & \  \\
O & \  & \  & \lambda_{n} \\
\end{pmatrix}\begin{pmatrix}
\overline{\lambda_{1}} & \  & \  & O \\
\  & \overline{\lambda_{2}} & \  & \  \\
\  & \  & \ddots & \  \\
O & \  & \  & \overline{\lambda_{n}} \\
\end{pmatrix}\\
&= \begin{pmatrix}
\left| \lambda_{1} \right|^{2} & \  & \  & O \\
\  & \left| \lambda_{2} \right|^{2} & \  & \  \\
\  & \  & \ddots & \  \\
O & \  & \  & \left| \lambda_{n} \right|^{2} \\
\end{pmatrix}\\
&= \begin{pmatrix}
1 & \  & \  & O \\
\  & 1 & \  & \  \\
\  & \  & \ddots & \  \\
O & \  & \  & 1 \\
\end{pmatrix} = I_{n}
\end{align*}
その線形写像$f$のその基底$\mathcal{B}$に関する表現行列$[ f]_{\mathcal{B}}^{\mathcal{B}}$がunitary行列である、即ち、${[ f]_{\mathcal{B}}^{\mathcal{B}}}^{- 1} = {[ f]_{\mathcal{B}}^{\mathcal{B}}}^{*}$が成り立ち、定理\ref{2.3.8.6}よりしたがって、その線形写像$f$は等長変換である。\par
その線形写像$f$の固有値たち$\lambda$がすべて$\lambda = \pm 1$を満たすなら、その線形写像$f$のその基底$\mathcal{B}$に関する表現行列$[ f]_{\mathcal{B}}^{\mathcal{B}}$がHermite行列である、即ち、${[ f]_{\mathcal{B}}^{\mathcal{B}}}^{*} = [ f]_{\mathcal{B}}^{\mathcal{B}}$が成り立つので、定理\ref{2.3.8.8}よりしたがって、その線形写像$f$はHermite変換である。
\end{proof}
%\hypertarget{ux5185ux7a4dux7a7aux9593ux3068ux5bfeux89d2ux5316}{%
\subsubsection{内積空間と対角化}%\label{ux5185ux7a4dux7a7aux9593ux3068ux5bfeux89d2ux5316}}
\begin{thm}\label{2.3.9.6}
体$\mathbb{C}$上の$1 \leq \dim V = n$なる内積空間$(V,\varPhi)$、$f \in L(V,V)$なる正規変換$f$が与えられたとき、定理\ref{2.2.3.1}よりその線形写像$f$の固有多項式$\varPhi_{f}$がその線形写像$f$の互いに異なる$s$つの固有値たち$\lambda_{i}$を用いて次式のように表されることができるので、そうするなら、
\begin{align*}
\varPhi_{f} = \prod_{i \in \varLambda_{s}} \left( X - \lambda_{i} \right)^{n_{i}},\ \ \sum_{i \in \varLambda_{s}} n_{i} = n
\end{align*}
次のことが成り立つ。
\begin{itemize}
\item
  そのvector空間$V$のある正規直交基底$\mathcal{B}$が存在して、これに含まれる任意のvectorはある固有空間$W_{f}\left( \lambda_{i} \right)$に属し、さらに、その基底$\mathcal{B}$に関するその線形写像$f$の表現行列$[ f]_{\mathcal{B}}^{\mathcal{B}}$が対角行列となる。
\item
  その固有値$\lambda_{i}$に対する固有空間たち$W_{f}\left( \lambda_{i} \right)$すべての和空間は直和空間でこれはそのvector空間$V$に等しい、即ち、そのvector空間$V$は次式を満たす。
\begin{align*}
V = \bigoplus_{i \in \varLambda_{s}} {W_{f}\left( \lambda_{i} \right)}
\end{align*}
\item
  $\forall i,j \in \varLambda_{s}$に対し、$i \neq j$が成り立つなら、$\varPhi|W_{f}\left( \lambda_{i} \right) \times W_{f}\left( \lambda_{j} \right) = 0$が成り立つ。
\item
  次式が成り立ち、
\begin{align*}
{W_{f}\left( \lambda_{i} \right)}^{\bot} = \bigoplus_{i' \in \varLambda_s \setminus \left\{ i \right\}} {W_{f}\left( \lambda_{i'} \right)}
\end{align*}
したがって、上の直和分解から定まるそのvector空間$V$からその固有空間$W_{f}\left( \lambda_{i} \right)$への射影はそのvector空間$V$からその固有空間$W_{f}\left( \lambda_{i} \right)$への正射影である。
\end{itemize}
\end{thm}
\begin{proof}
体$\mathbb{C}$上の$1 \leq \dim V = n$なる内積空間$(V,\varPhi)$、$f \in L(V,V)$なる正規変換$f$が与えられたとき、定理\ref{2.2.3.1}よりその線形写像$f$の固有多項式$\varPhi_{f}$がその線形写像$f$の互いに異なる$s$つの固有値たち$\lambda_{i}$を用いて次式のように表されることができるので、そうするなら、
\begin{align*}
\varPhi_{f} = \prod_{i \in \varLambda_{s}} \left( X - \lambda_{i} \right)^{n_{i}},\ \ \sum_{i \in \varLambda_{s}} n_{i} = n
\end{align*}
定理\ref{2.3.9.4}、即ち、Toeplitzの定理より$\forall f \in L(V,V)$に対し、その線形写像$f$が正規変換であるならそのときに限り、そのvector空間$V$のある正規直交基底$\mathcal{B}$が存在して、これに関するそれらの線形写像$f$の表現行列$[ f]_{\mathcal{B}}^{\mathcal{B}}$が対角行列となる。そこで、その基底$\mathcal{B}$に関する基底変換における線形同型写像$\varphi_{\mathcal{B}}$、そのvector空間$\mathbb{C}^{n}$の標準直交基底$\left\langle \mathbf{e}_{i} \right\rangle_{i \in \varLambda_{n}}$、線形写像$L_{[ f]_{\mathcal{B}}^{\mathcal{B}}}:\mathbb{C}^{n} \rightarrow \mathbb{C}^{n};\mathbf{v} \mapsto [ f]_{\mathcal{B}}^{\mathcal{B}}\mathbf{v}$について、定理\ref{2.2.4.15}より$\dim{W_{f}\left( \lambda_{i} \right)} = n_{i}$が成り立つので、次の表のように$\forall i' \in \varLambda_{n}$に対し、$\sum_{j' \in \varLambda_{i}} n_{j'} < i' \leq \sum_{j' \in \varLambda_{i + 1}} n_{j'}$が成り立つような自然数$i$を用いて$j = i' - \sum_{j' \in \varLambda_{i}} n_{j'}$とおかれることで、その正規直交基底$\mathcal{B}$が$\mathcal{B} =\left\langle \mathbf{o}_{ij} \right\rangle_{(i,j) \in \varLambda_{s} \times \varLambda_{n_{i}}}$とおかれることができる。
\begin{longtable}[c]{|c|c|}
  \hline
  $(i,j)$ & $i'$ \\
  \hline \hline
  $(1,1)$ & $1$ \\
  \hline
  $(1,2)$ & $2$ \\
  \hline
  $\vdots$ & $\vdots$ \\
  \hline
  $\left( 1,n_{1} \right)$ & $n_{1}$ \\
  \hline
  $(2,1)$ & $n_{1} + 1$ \\
  \hline
  $(2,2)$ & $n_{1} + 2$ \\
  \hline
  $\vdots$ & $\vdots$ \\
  \hline
  $\left( 2,n_{2} \right)$ & $n_{1} + n_{2}$ \\
  \hline
  $\vdots$ & $\vdots$ \\
  \hline
  $(s,1)$ & $n_{1} + n_{2} + \cdots + n_{s - 1} + 1$ \\
  \hline
  $(s,2)$ & $n_{1} + n_{2} + \cdots + n_{s - 1} + 1$ \\
  \hline
  $\vdots$ & $\vdots$ \\
  \hline
  $\left( s,n_{s} \right)$ & $n_{1} + n_{2} + \cdots + n_{s - 1} + n_{s}$ \\
  \hline
\end{longtable}
このとき、定理\ref{2.2.2.12}よりその表現行列$[ f]_{\mathcal{B}}^{\mathcal{B}}$の対角成分がその線形写像$f$の固有値であるので、次のようになる。
\begin{align*}
\left( \lambda_{i}I_{V} - f \right)\left( \mathbf{o}_{ij} \right) &= \lambda_{i}I_{V}\left( \mathbf{o}_{ij} \right) - f\left( \mathbf{o}_{ij} \right)\\
&= \lambda_{i}\mathbf{o}_{ij} - \varphi_{\mathcal{B}} \circ \varphi_{\mathcal{B}}^{- 1} \circ f \circ \varphi_{\mathcal{B}} \circ \varphi_{\mathcal{B}}^{- 1}\left( \mathbf{o}_{ij} \right)\\
&= \lambda_{i}\mathbf{o}_{ij} - \varphi_{\mathcal{B}} \circ L_{[ f]_{\mathcal{B}}^{\mathcal{B}}} \circ \varphi_{\mathcal{B}}^{- 1}\left( \mathbf{o}_{ij} \right)\\
&= \lambda_{i}\mathbf{o}_{ij} - \varphi_{\mathcal{B}}\left( [ f]_{\mathcal{B}}^{\mathcal{B}}\varphi_{\mathcal{B}}^{- 1}\left( \mathbf{o}_{ij} \right) \right)\\
&= \lambda_{i}\mathbf{o}_{ij} - \varphi_{\mathcal{B}}\left( \lambda_{i}\varphi_{\mathcal{B}}^{- 1}\left( \mathbf{o}_{ij} \right) \right)\\
&= \lambda_{i}\mathbf{o}_{ij} - \lambda_{i}\varphi_{\mathcal{B}} \circ \varphi_{\mathcal{B}}^{- 1}\left( \mathbf{o}_{ij} \right)\\
&= \lambda_{i}\mathbf{o}_{ij} - \lambda_{i}\mathbf{o}_{ij} = \mathbf{0}
\end{align*}
これにより、$\mathbf{o}_{ij} \in \ker\left( \lambda_{i}I_{V} - f \right) \setminus \left\{ \mathbf{0} \right\} = W_{f}\left( \lambda_{i} \right)$が得られる。\par
また、定理\ref{2.2.4.15}より次のことは同値である。
\begin{itemize}
\item
  その線形写像$f$は対角化可能である。
\item
  そのvector空間$V$はその固有値$\lambda_{i}$に対する固有空間$W_{f}\left( \lambda_{i} \right)$を用いて次式が成り立つ。
\begin{align*}
V = \bigoplus_{i \in \varLambda_{s}} {W_{f}\left( \lambda_{i} \right)}
\end{align*}
\end{itemize}
これにより、その固有値$\lambda_{i}$に対する固有空間たち$W_{f}\left( \lambda_{i} \right)$すべての和空間は直和空間でこれはそのvector空間$V$に等しい、即ち、そのvector空間$V$は次式を満たす。
\begin{align*}
V = \bigoplus_{i \in \varLambda_{s}} {W_{f}\left( \lambda_{i} \right)}
\end{align*}\par
$\forall i,j \in \varLambda_{s}$に対し、$i \neq j$が成り立つなら、$\forall\left( \mathbf{v}_{i},\mathbf{v}_{j} \right) \in W_{f}\left( \lambda_{i} \right) \times W_{f}\left( \lambda_{j} \right)$に対し、上記の議論により次のようにおかれることができて、
\begin{align*}
\mathbf{v}_{i} = \sum_{k \in \varLambda_{n_{i}}} {v_{ik}\mathbf{o}_{ik}},\ \ \mathbf{v}_{j} = \sum_{l \in \varLambda_{n_{j}}} {v_{jl}\mathbf{o}_{jl}}
\end{align*}
次のようになる。
\begin{align*}
\varPhi|W_{f}\left( \lambda_{i} \right) \times W_{f}\left( \lambda_{j} \right)\left( \mathbf{v}_{i},\mathbf{v}_{j} \right) &= \varPhi\left( \mathbf{v}_{i},\mathbf{v}_{j} \right)\\
&= \varPhi\left( \sum_{k \in \varLambda_{n_{i}}} {v_{ik}\mathbf{o}_{ik}},\sum_{l \in \varLambda_{n_{j}}} {v_{jl}\mathbf{o}_{jl}} \right)\\
&= \sum_{k \in \varLambda_{n_{i}}} {\sum_{l \in \varLambda_{n_{j}}} {\overline{v_{ik}}v_{jl}\varPhi\left( \mathbf{o}_{ik},\mathbf{o}_{jl} \right)}}\\
&= \sum_{k \in \varLambda_{n_{i}}} {\sum_{l \in \varLambda_{n_{j}}} {\overline{v_{ik}}v_{jl} \cdot 0}} = 0
\end{align*}
よって、$\varPhi|W_{f}\left( \lambda_{i} \right) \times W_{f}\left( \lambda_{j} \right) = 0$が成り立つ。\par
上の直和分解から定まるそのvector空間$V$からその固有空間$W_{f}\left( \lambda_{i} \right)$への射影$P_{i}$について、$\forall\mathbf{v} \in W_{f}\left( \lambda_{i} \right)\forall\mathbf{w} \in \bigoplus_{i' \in \varLambda_s \setminus \left\{ i \right\}} {W_{f}\left( \lambda_{i'} \right)}$に対し、次のようにおかれることができて、
\begin{align*}
\mathbf{v} = \sum_{j \in \varLambda_{n_{i}}} {v_{ij}\mathbf{o}_{ij}},\ \ \mathbf{w} = \sum_{i' \in \varLambda_s \setminus \left\{ i \right\}} {\sum_{j' \in \varLambda_{n_{i'}}} {v_{i'j'}\mathbf{o}_{i'j'}}}
\end{align*}
次のようになるので、
\begin{align*}
\varPhi\left( \mathbf{v},\mathbf{w} \right) &= \varPhi\left( \sum_{j \in \varLambda_{n_{i}}} {v_{ij}\mathbf{o}_{ij}},\sum_{i' \in \varLambda_s \setminus \left\{ i \right\}} {\sum_{j' \in \varLambda_{n_{i'}}} {v_{i'j'}\mathbf{o}_{i'j'}}} \right)\\
&= \sum_{i' \in \varLambda_s \setminus \left\{ i \right\}} {\sum_{j' \in \varLambda_{n_{i'}}} {\sum_{j \in \varLambda_{n_{i}}} {\overline{v_{ij}}v_{i'j'}\varPhi\left( \mathbf{o}_{ij},\mathbf{o}_{i'j'} \right)}}}\\
&= \sum_{i' \in \varLambda_s \setminus \left\{ i \right\}} {\sum_{j' \in \varLambda_{n_{i'}}} {\sum_{j \in \varLambda_{n_{i}}} {\overline{v_{ij}}v_{i'j'} \cdot 0}}} = 0
\end{align*}
$\bigoplus_{i' \in \varLambda_s \setminus \left\{ i \right\}} {W_{f}\left( \lambda_{i'} \right)} \subseteq {W_{f}\left( \lambda_{i} \right)}^{\bot}$が成り立つ。そこで、定理\ref{2.3.7.10}より次式が成り立つことにより、
\begin{align*}
V = \bigoplus_{i \in \varLambda_{s}} {W_{f}\left( \lambda_{i} \right)} = W_{f}\left( \lambda_{i} \right) \oplus \bigoplus_{i' \in \varLambda_s \setminus \left\{ i \right\}} {W_{f}\left( \lambda_{i'} \right)},\ \ V = W_{f}\left( \lambda_{i} \right) \oplus {W_{f}\left( \lambda_{i} \right)}^{\bot}
\end{align*}
次のようになるので、
\begin{align*}
\dim{\bigoplus_{i' \in \varLambda_s \setminus \left\{ i \right\}} {W_{f}\left( \lambda_{i'} \right)}} &= \dim{W_{f}\left( \lambda_{i} \right)} + \dim{\bigoplus_{i' \in \varLambda_s \setminus \left\{ i \right\}} {W_{f}\left( \lambda_{i'} \right)}} - \dim{W_{f}\left( \lambda_{i} \right)}\\
&= \dim{W_{f}\left( \lambda_{i} \right) \oplus \bigoplus_{i' \in \varLambda_s \setminus \left\{ i \right\}} {W_{f}\left( \lambda_{i'} \right)}} - \dim{W_{f}\left( \lambda_{i} \right)}\\
&= \dim V - \dim{W_{f}\left( \lambda_{i} \right)}\\
&= \dim{W_{f}\left( \lambda_{i} \right) \oplus {W_{f}\left( \lambda_{i} \right)}^{\bot}} - \dim{W_{f}\left( \lambda_{i} \right)}\\
&= \dim{W_{f}\left( \lambda_{i} \right)} + \dim{W_{f}\left( \lambda_{i} \right)}^{\bot} - \dim{W_{f}\left( \lambda_{i} \right)}\\
&= \dim{W_{f}\left( \lambda_{i} \right)}^{\bot}
\end{align*}
$\bigoplus_{i' \in \varLambda_s \setminus \left\{ i \right\}} {W_{f}\left( \lambda_{i'} \right)} = {W_{f}\left( \lambda_{i} \right)}^{\bot}$が得られる。したがって、上の直和分解から定まるそのvector空間$V$からその固有空間$W_{f}\left( \lambda_{i} \right)$への射影はそのvector空間$V$からその固有空間$W_{f}\left( \lambda_{i} \right)$への正射影である。
\end{proof}
%\hypertarget{spectrumux5206ux89e3-1}{%
\subsubsection{spectrum分解}%\label{spectrumux5206ux89e3-1}}
\begin{thm}\label{2.3.9.7}
体$\mathbb{C}$上の$1 \leq \dim V = n$なる内積空間$(V,\varPhi)$が与えられたとき、そのvector空間$V$の部分空間$W$について、定理\ref{2.2.1.11}より$V = U \oplus W$とおかれれば、その直和分解から定まるそのvector空間$V$からその部分空間$W$への射影$P$が正射影であるならそのときに限り、その射影$P$がHermite変換である、即ち、$P = P^{*}$が成り立つ。
\end{thm}
\begin{proof}
体$\mathbb{C}$上の$1 \leq \dim V = n$なる内積空間$(V,\varPhi)$が与えられたとき、そのvector空間$V$の部分空間$W$について、定理\ref{2.2.1.11}より$V = U \oplus W$とおかれれば、その直和分解から定まるそのvector空間$V$からその部分空間$W$への射影$P$が正射影であるなら、$U = W^{\bot}$が成り立つことになり、$\forall\mathbf{v},\mathbf{w} \in V$に対し、$\mathbf{u}_{\mathbf{v}},\mathbf{u}_{\mathbf{w}} \in U$、$\mathbf{w}_{\mathbf{v}},\mathbf{w}_{\mathbf{w}} \in W$なるvectorsを用いて$\mathbf{v} = \mathbf{u}_{\mathbf{v}} \oplus \mathbf{w}_{\mathbf{v}}$、$\mathbf{w} = \mathbf{u}_{\mathbf{w}} \oplus \mathbf{w}_{\mathbf{w}}$とおかれれば、次のようになるので
\begin{align*}
\varPhi\left( P\left( \mathbf{v} \right),\mathbf{w} \right) &= \varPhi\left( P\left( \mathbf{u}_{\mathbf{v}} \oplus \mathbf{w}_{\mathbf{v}} \right),\mathbf{u}_{\mathbf{w}} \oplus \mathbf{w}_{\mathbf{w}} \right)\\
&= \varPhi\left( \mathbf{w}_{\mathbf{v}},\mathbf{u}_{\mathbf{w}} \oplus \mathbf{w}_{\mathbf{w}} \right)\\
&= \varPhi\left( \mathbf{w}_{\mathbf{v}},\mathbf{u}_{\mathbf{w}} \right) + \varPhi\left( \mathbf{w}_{\mathbf{v}},\mathbf{w}_{\mathbf{w}} \right)\\
&= \varPhi\left( \mathbf{w}_{\mathbf{v}},\mathbf{w}_{\mathbf{w}} \right)\\
\varPhi\left( \mathbf{v},P\left( \mathbf{w} \right) \right) &= \varPhi\left( \mathbf{u}_{\mathbf{v}} \oplus \mathbf{w}_{\mathbf{v}},P\left( \mathbf{u}_{\mathbf{w}} \oplus \mathbf{w}_{\mathbf{w}} \right) \right)\\
&= \varPhi\left( \mathbf{u}_{\mathbf{v}} \oplus \mathbf{w}_{\mathbf{v}},\mathbf{w}_{\mathbf{w}} \right)\\
&= \varPhi\left( \mathbf{u}_{\mathbf{v}},\mathbf{w}_{\mathbf{w}} \right) + \varPhi\left( \mathbf{w}_{\mathbf{v}},\mathbf{w}_{\mathbf{w}} \right)\\
&= \varPhi\left( \mathbf{w}_{\mathbf{v}},\mathbf{w}_{\mathbf{w}} \right)
\end{align*}
$\varPhi\left( P\left( \mathbf{v} \right),\mathbf{w} \right) = \varPhi\left( \mathbf{v},P\left( \mathbf{w} \right) \right)$が得られる。したがって、$\forall\mathbf{v},\mathbf{w} \in V$に対し、次のようになるので
\begin{align*}
\varPhi\left( \mathbf{v},\left( P - P^{*} \right)\left( \mathbf{w} \right) \right) &= \varPhi\left( \mathbf{v},P\left( \mathbf{w} \right) - P^{*}\left( \mathbf{w} \right) \right)\\
&= \varPhi\left( \mathbf{v},P\left( \mathbf{w} \right) \right) - \varPhi\left( \mathbf{v},P^{*}\left( \mathbf{w} \right) \right)\\
&= \varPhi\left( P\left( \mathbf{v} \right),\mathbf{w} \right) - \varPhi\left( P\left( \mathbf{v} \right),\mathbf{w} \right) = 0
\end{align*}
$\left( P - P^{*} \right)\left( \mathbf{w} \right) = \mathbf{0}$が成り立つ。そのvector$\mathbf{w}$の任意性により$P - P^{*} = 0$、即ち、$P = P^{*}$が成り立つ。これにより、その射影$P$はHermite変換である。\par
逆に、その射影$P$がHermite変換であるなら、$\forall\mathbf{u} \in U\forall\mathbf{w} \in W$に対し、定理\ref{2.3.8.5}より次のようになるので、
\begin{align*}
\varPhi\left( \mathbf{u},\mathbf{w} \right) &= \varPhi\left( \mathbf{u},P\left( \mathbf{w} \right) \right)\\
&= \varPhi\left( \mathbf{u},P^{**}\left( \mathbf{w} \right) \right)\\
&= \varPhi\left( P^{*}\left( \mathbf{u} \right),\mathbf{w} \right)\\
&= \varPhi\left( P\left( \mathbf{u} \right),\mathbf{w} \right)\\
&= \varPhi\left( \mathbf{0},\mathbf{w} \right) = 0
\end{align*}
$\mathbf{u} \in W^{\bot}$が得られ、したがって、$U \subseteq W^{\bot}$が成り立つ。そこで、定理\ref{2.3.7.10}より$V = U \oplus W = W \oplus W^{\bot}$が成り立つので、次のようになる。
\begin{align*}
\dim U &= \dim W + \dim U - \dim W\\
&= \dim{U \oplus W} - \dim W\\
&= \dim V - \dim W\\
&= \dim{W \oplus W^{\bot}} - \dim W\\
&= \dim W + \dim W^{\bot} - \dim W\\
&= \dim W^{\bot}
\end{align*}
これにより、$U = W^{\bot}$が得られ、その射影$P$はその直和分解から定まるそのvector空間$V$からその部分空間$W$への射影$P$が正射影である。
\end{proof}\par
ここで、次の定理を再掲しておこう。
\begin{thm*}[定理\ref{2.2.1.8}の再掲]
体$K$上のvector空間$V$がこれの部分空間たちの添数集合$\varLambda_{n}$によって添数づけられた族$\left\{ W_{i} \right\}_{i \in \varLambda_{n}}$に直和分解されるとき、次のことが成り立つ。
\begin{itemize}
\item
  $\forall i \in \varLambda_{n}$に対し、そのvector空間$V$からその直和因子$W_{i}$への直和分解から定まる射影$P_{i}$はそのvector空間$V$の射影子である。
\item
  $\forall i \in \varLambda_{n}$に対し、そのvector空間$V$からその直和因子$W_{i}$への直和分解から定まる射影$P_{i}$について、$V\left( P_{i} \right) = W_{i}$が成り立つ。
\item
  $\forall i,j \in \varLambda_{n}$に対し、$i \neq j$が成り立つなら、そのvector空間$V$からその直和因子$W_{i}$、$W_{j}$への直和分解から定まる射影たちそれぞれ$P_{i}$、$P_{j}$について、$P_{j} \circ P_{i} = 0$が成り立つ。
\item
  $i \in \varLambda_{n}$なるそのvector空間$V$からその直和因子$W_{i}$への直和分解から定まる射影たち$P_{i}$について、そのvector空間$V$の恒等写像$I_{V}$を用いて$\sum_{i \in \varLambda_{n}} P_{i} = I_{V}$が成り立つ。
\item
  $\forall i' \in \varLambda_{n}$に対し、そのvector空間$V$からその直和因子$W_{i'}$への直和分解から定まる射影$P_{i'}$について、$\ker P_{i'} = \bigoplus_{i \in \varLambda_s \setminus \left\{ i' \right\}} W_{i}$が成り立つ。
\end{itemize}
\end{thm*}
\begin{thm}[spectrum分解]\label{2.3.9.8}
体$\mathbb{C}$上の$1 \leq \dim V = n$なる内積空間$(V,\varPhi)$、$f \in L(V,V)$なる正規変換$f$が与えられたとき、定理\ref{2.2.3.1}よりその線形写像$f$の固有多項式$\varPhi_{f}$がその線形写像$f$の互いに異なる$s$つの固有値たち$\lambda_{i}$を用いて次式のように表されることができるので、そうするなら、
\begin{align*}
\varPhi_{f} = \prod_{i \in \varLambda_{s}} \left( X - \lambda_{i} \right)^{n_{i}},\ \ \sum_{i \in \varLambda_{s}} n_{i} = n
\end{align*}
定理\ref{2.3.9.6}よりその固有値$\lambda_{i}$に対する固有空間たち$W_{f}\left( \lambda_{i} \right)$すべての和空間は直和空間でこれはそのvector空間$V$に等しいのであった。このとき、その直和分解から定まるそのvector空間$V$からその固有空間$W_{f}\left( \lambda_{i} \right)$への射影$P_{i}$は正射影子$P_{W_{f}\left( \lambda_{i} \right)}$であって次式が成り立つ。
\begin{align*}
f = \sum_{i \in \varLambda_{s}} {\lambda_{i}P_{W_{f}\left( \lambda_{i} \right)}}
\end{align*}
上の式をその正規変換$f$のspectrum分解という。
\end{thm}
\begin{proof}
体$\mathbb{C}$上の$1 \leq \dim V = n$なる内積空間$(V,\varPhi)$、$f \in L(V,V)$なる正規変換$f$が与えられたとき、定理\ref{2.2.3.1}よりその線形写像$f$の固有多項式$\varPhi_{f}$がその線形写像$f$の互いに異なる$s$つの固有値たち$\lambda_{i}$を用いて次式のように表されることができるので、そうするなら、
\begin{align*}
\varPhi_{f} = \prod_{i \in \varLambda_{s}} \left( X - \lambda_{i} \right)^{n_{i}},\ \ \sum_{i \in \varLambda_{s}} n_{i} = n
\end{align*}
定理\ref{2.3.9.6}より次式のようにその固有値$\lambda_{i}$に対する固有空間たち$W_{f}\left( \lambda_{i} \right)$すべての和空間は直和空間で
\begin{align*}
V = \bigoplus_{i \in \varLambda_{s}} {W_{f}\left( \lambda_{i} \right)}
\end{align*}
これはそのvector空間$V$に等しいのであった。このとき、定理\ref{2.3.9.6}より次式が成り立ち、
\begin{align*}
{W_{f}\left( \lambda_{i} \right)}^{\bot} = \bigoplus_{i' \in \varLambda_s \setminus \left\{ i \right\}} {W_{f}\left( \lambda_{i'} \right)}
\end{align*}
したがって、上の直和分解から定まるそのvector空間$V$からその固有空間$W_{f}\left( \lambda_{i} \right)$への射影$P_{i}$はそのvector空間$V$からその固有空間$W_{f}\left( \lambda_{i} \right)$への正射影$P_{W_{f}\left( \lambda_{i} \right)}$である。さらに、定理\ref{2.2.1.8}よりその射影$P_{i}$はそのvector空間$V$の射影子でもあるので、その直和分解から定まるそのvector空間$V$からその固有空間$W_{f}\left( \lambda_{i} \right)$への射影$P_{i}$は正射影子$P_{W_{f}\left( \lambda_{i} \right)}$である。\par
また、$\forall\mathbf{v} \in V$に対し、$\mathbf{v} = \sum_{i \in \varLambda_{s}} \mathbf{w}_{i}$、$\mathbf{w}_{i} \in W_{f}\left( \lambda_{i} \right)$とおかれれば、$\left( \lambda_{i}I_{V} - f \right)\left( \mathbf{w}_{i} \right) = \mathbf{0}$が成り立つので、次のようになる。
\begin{align*}
f\left( \mathbf{v} \right) &= f\left( \sum_{i \in \varLambda_{s}} \mathbf{w}_{i} \right)\\
&= \sum_{i \in \varLambda_{s}} {f\left( \mathbf{w}_{i} \right)}\\
&= \sum_{i \in \varLambda_{s}} \left( \lambda_{i}I_{V}\left( \mathbf{w}_{i} \right) - \lambda_{i}I_{V}\left( \mathbf{w}_{i} \right) + f\left( \mathbf{w}_{i} \right) \right)\\
&= \sum_{i \in \varLambda_{s}} \left( \lambda_{i}I_{V}\left( \mathbf{w}_{i} \right) - \left( \lambda_{i}I_{V} - f \right)\left( \mathbf{w}_{i} \right) \right)\\
&= \sum_{i \in \varLambda_{s}} \left( \lambda_{i}\mathbf{w}_{i} - \mathbf{0} \right)\\
&= \sum_{i \in \varLambda_{s}} {\lambda_{i}P_{W_{f}\left( \lambda_{i} \right)}\left( \sum_{i \in \varLambda_{s}} \mathbf{w}_{i} \right)}\\
&= \sum_{i \in \varLambda_{s}} {\lambda_{i}P_{W_{f}\left( \lambda_{i} \right)}\left( \mathbf{v} \right)}\\
&= \left( \sum_{i \in \varLambda_{s}} {\lambda_{i}P_{W_{f}\left( \lambda_{i} \right)}} \right)\left( \mathbf{v} \right)
\end{align*}
よって、次式が成り立つ。
\begin{align*}
f = \sum_{i \in \varLambda_{s}} {\lambda_{i}P_{W_{f}\left( \lambda_{i} \right)}}
\end{align*}
\end{proof}\par
例えば、次の正規変換$A$の固有値、これを対角化させるunitary行列$U$、spectrum分解を求めよう。
\begin{align*}
A = \begin{pmatrix}
0 & i & - 1 \\
 - i & 0 & - i \\
 - 1 & i & 0 \\
\end{pmatrix}
\end{align*}
その行列の固有多項式$\varPhi_{A}$を求めると、次のようになる。
\begin{align*}
\varPhi_{A} &= |XI - A| = \left| \begin{matrix}
X & - i & 1 \\
i & X & i \\
1 & - i & X \\
\end{matrix} \right|\\
&= X^{3} + 1 + 1 - X - X - X\\
&= X^{3} - 3X + 2\\
&= (X + 2)(X - 1)^{2}
\end{align*}
定理\ref{2.2.2.6}よりその固有多項式$\varPhi_{A}$の根がその行列$A$の固有値なので、その行列$A$の固有値が$- 2$、$1$と挙げられる。\par
このとき、これらの固有空間たち$W_{A}( - 2)$、$W_{A}(1)$について、次のようになる。
\begin{align*}
W_{A}( - 2) &= \ker( - 2I - A) = \ker\begin{pmatrix}
 - 2 & - i & 1 \\
i & - 2 & i \\
1 & - i & - 2 \\
\end{pmatrix}\\
W_{A}(1) &= \ker(I - A) = \ker\begin{pmatrix}
1 & - i & 1 \\
i & 1 & i \\
1 & - i & 1 \\
\end{pmatrix}
\end{align*}
ここで、次のようになることから、
\begin{align*}
\begin{pmatrix}
 - 2 & - i & 1 \\
i & - 2 & i \\
1 & - i & - 2 \\
\end{pmatrix} &\rightarrow \begin{pmatrix}
1 & - i & - 2 \\
i & - 2 & i \\
 - 2 & - i & 1 \\
\end{pmatrix}\\
&\rightarrow \begin{pmatrix}
1 & - i & - 2 \\
0 & - 3 & 3i \\
0 & - 3i & - 3 \\
\end{pmatrix}\\
&\rightarrow \begin{pmatrix}
1 & - i & - 2 \\
0 & 1 & - i \\
0 & i & 1 \\
\end{pmatrix}\\
&\rightarrow \begin{pmatrix}
1 & 0 & - 1 \\
0 & 1 & - i \\
0 & 0 & 0 \\
\end{pmatrix}\\
\begin{pmatrix}
1 & - i & 1 \\
i & 1 & i \\
1 & - i & 1 \\
\end{pmatrix} &\rightarrow \begin{pmatrix}
1 & - i & 1 \\
0 & 0 & 0 \\
0 & 0 & 0 \\
\end{pmatrix}
\end{align*}
次のようになる。
\begin{align*}
W_{A}( - 2) &= \ker\begin{pmatrix}
 - 2 & - i & 1 \\
i & - 2 & i \\
1 & - i & - 2 \\
\end{pmatrix} = {\mathrm{span}}\left\{ \begin{pmatrix}
1 \\
i \\
1 \\
\end{pmatrix} \right\}\\
W_{A}(1) &= \ker\begin{pmatrix}
1 & - i & 1 \\
i & 1 & i \\
1 & - i & 1 \\
\end{pmatrix} = {\mathrm{span}}\left\{ \begin{pmatrix}
i \\
1 \\
0 \\
\end{pmatrix},\begin{pmatrix}
 - 1 \\
0 \\
1 \\
\end{pmatrix} \right\}
\end{align*}\par
そこで、その固有空間$W_{A}( - 2)$の基底として$\left\langle \begin{pmatrix}
1 \\
i \\
1 \\
\end{pmatrix} \right\rangle$が与えられるので、Gram-Schmidtの正規直交化法により次のようになるので、
\begin{align*}
\frac{\begin{pmatrix}
1 \\
i \\
1 \\
\end{pmatrix}}{\left\| \begin{pmatrix}
1 \\
i \\
1 \\
\end{pmatrix} \right\|} &= \frac{\begin{pmatrix}
1 \\
i \\
1 \\
\end{pmatrix}}{\sqrt{\left\langle \begin{pmatrix}
1 \\
i \\
1 \\
\end{pmatrix} \middle| \begin{pmatrix}
1 \\
i \\
1 \\
\end{pmatrix} \right\rangle}} = \frac{\begin{pmatrix}
1 \\
i \\
1 \\
\end{pmatrix}}{\sqrt{\begin{matrix}
\  & 1 \cdot 1 \\
 + & - i \cdot i \\
 + & i \cdot 1 \\
\end{matrix}}} = \frac{1}{\sqrt{3}} \begin{pmatrix}
1 \\
i \\
1 \\
\end{pmatrix} = \begin{pmatrix}
{1}/{\sqrt{3}} \\
{i}/{\sqrt{3}} \\
{1}/{\sqrt{3}} \\
\end{pmatrix}
\end{align*}
その固有空間$W_{A}( - 2)$の正規直交基底として$\left\langle \begin{pmatrix}
{1}/{\sqrt{3}} \\
{i}/{\sqrt{3}} \\
{1}/{\sqrt{3}} \\
\end{pmatrix} \right\rangle$が与えられる。\par
一方で、その固有空間$W_{A}(1)$の基底として$\left\langle \begin{pmatrix}
i \\
1 \\
0 \\
\end{pmatrix},\begin{pmatrix}
 - 1 \\
0 \\
1 \\
\end{pmatrix} \right\rangle$が与えられるので、Gram-Schmidtの正規直交化法により次のようになるので、
\begin{align*}
\frac{\begin{pmatrix}
i \\
1 \\
0 \\
\end{pmatrix}}{\left\| \begin{pmatrix}
i \\
1 \\
0 \\
\end{pmatrix} \right\|} &= \frac{\begin{pmatrix}
i \\
1 \\
0 \\
\end{pmatrix}}{\sqrt{\left\langle \begin{pmatrix}
i \\
1 \\
0 \\
\end{pmatrix} \middle| \begin{pmatrix}
i \\
1 \\
0 \\
\end{pmatrix} \right\rangle}} = \frac{\begin{pmatrix}
i \\
1 \\
0 \\
\end{pmatrix}}{\sqrt{\begin{matrix}
\  & - i \cdot i \\
 + & 1 \cdot 1 \\
 + & 0 \cdot 0 \\
\end{matrix}}} \\
&= \frac{1}{\sqrt{2}}\begin{pmatrix}
i \\
1 \\
0 \\
\end{pmatrix} = \begin{pmatrix}
{i}/{\sqrt{2}} \\
{1}/{\sqrt{2}} \\
0 \\
\end{pmatrix}\\
\begin{pmatrix}
 - 1 \\
0 \\
1 \\
\end{pmatrix} - \left\langle \begin{pmatrix}
{i}/{\sqrt{2}} \\
{1}/{\sqrt{2}} \\
0 \\
\end{pmatrix} \middle| \begin{pmatrix}
 - 1 \\
0 \\
1 \\
\end{pmatrix} \right\rangle\begin{pmatrix}
{i}/{\sqrt{2}} \\
{1}/{\sqrt{2}} \\
0 \\
\end{pmatrix} &= \begin{pmatrix}
 - 1 \\
0 \\
1 \\
\end{pmatrix} - \begin{pmatrix}
\  & \left( - {i}/{\sqrt{2}} \right) \cdot ( - 1) \\
 + & 1/\sqrt{2} \cdot 0 \\
 + & 0 \cdot 1 \\
\end{pmatrix}\begin{pmatrix}
{i}/{\sqrt{2}} \\
{1}/{\sqrt{2}} \\
0 \\
\end{pmatrix}\\
&= \begin{pmatrix}
 - 1 \\
0 \\
1 \\
\end{pmatrix} - \frac{i}{\sqrt{2}}\begin{pmatrix}
{i}/{\sqrt{2}} \\
{1}/{\sqrt{2}} \\
0 \\
\end{pmatrix}\\
&= \begin{pmatrix}
 - 1 \\
0 \\
1 \\
\end{pmatrix} + \begin{pmatrix}
{1}/{2} \\
 - {i}/{2} \\
0 \\
\end{pmatrix} = \begin{pmatrix}
 - {1}/{2} \\
 - {i}/{2} \\
1 \\
\end{pmatrix}\\
\frac{\begin{pmatrix}
 - {1}/{2} \\
 - {i}/{2} \\
1 \\
\end{pmatrix}}{\left\| \begin{pmatrix}
 - {1}/{2} \\
 - {i}/{2} \\
1 \\
\end{pmatrix} \right\|} &= \frac{\begin{pmatrix}
 - {1}/{2} \\
 - {i}/{2} \\
1 \\
\end{pmatrix}}{\sqrt{\left\langle \begin{pmatrix}
 - {1}/{2} \\
 - {i}/{2} \\
1 \\
\end{pmatrix} \middle| \begin{pmatrix}
 - {1}/{2} \\
 - {i}/{2} \\
1 \\
\end{pmatrix} \right\rangle}}\\
&= \frac{\begin{pmatrix}
 - {1}/{2} \\
 - {i}/{2} \\
1 \\
\end{pmatrix}}{\sqrt{\begin{matrix}
\  & \left( - {1}/{2} \right) \cdot \left( - {1}/{2} \right) \\
 + &  {i}/{2} \cdot \left( - {i}/{2} \right) \\
 + & 1 \cdot 1 \\
\end{matrix}}}\\
&= \frac{2}{\sqrt{6}}\begin{pmatrix}
 - {1}/{2} \\
 - {i}/{2} \\
1 \\
\end{pmatrix} = \begin{pmatrix}
 - {1}/{\sqrt{6}} \\
 - {i}/{\sqrt{6}} \\
{2}/{\sqrt{6}} \\
\end{pmatrix}
\end{align*}
その固有空間$W_{A}(1)$の正規直交基底として$\left\langle \begin{pmatrix}
{i}/{\sqrt{2}} \\
{1}/{\sqrt{2}} \\
0 \\
\end{pmatrix},\begin{pmatrix}
 - {1}/{\sqrt{6}} \\
 - {i}/{\sqrt{6}} \\
{2}/{\sqrt{6}} \\
\end{pmatrix} \right\rangle$が与えられる。\par
定理\ref{2.3.9.6}より$\mathbb{C}^{3} = W_{A}( - 2) \oplus W_{A}(1)$が成り立ち、そのvector空間$\mathbb{C}^{3}$の正規直交基底として$\left\langle \begin{pmatrix}
{1}/{\sqrt{3}} \\
{i}/{\sqrt{3}} \\
{1}/{\sqrt{3}} \\
\end{pmatrix},\begin{pmatrix}
{i}/{\sqrt{2}} \\
{1}/{\sqrt{2}} \\
0 \\
\end{pmatrix},\begin{pmatrix}
 - {1}/{\sqrt{6}} \\
 - {i}/{\sqrt{6}} \\
{2}/{\sqrt{6}} \\
\end{pmatrix} \right\rangle$が与えられる。定理\ref{2.3.8.6}より行列$\begin{pmatrix}
{1}/{\sqrt{3}} & {i}/{\sqrt{2}} & - {1}/{\sqrt{6}} \\
{i}/{\sqrt{3}} & {1}/{\sqrt{2}} & - {i}/{\sqrt{6}} \\
{1}/{\sqrt{3}} & 0 & {2}/{\sqrt{6}} \\
\end{pmatrix}$がunitary行列となって次のようになる。
\begin{align*}
&\quad \begin{pmatrix}
{1}/{\sqrt{3}} & {i}/{\sqrt{2}} & - {1}/{\sqrt{6}} \\
{i}/{\sqrt{3}} & {1}/{\sqrt{2}} & - {i}/{\sqrt{6}} \\
{1}/{\sqrt{3}} & 0 & {2}/{\sqrt{6}} \\
\end{pmatrix}^{- 1}\begin{pmatrix}
0 & i & - 1 \\
 - i & 0 & - i \\
 - 1 & i & 0 \\
\end{pmatrix}\begin{pmatrix}
{1}/{\sqrt{3}} & {i}/{\sqrt{2}} & - {1}/{\sqrt{6}} \\
{i}/{\sqrt{3}} & {1}/{\sqrt{2}} & - {i}/{\sqrt{6}} \\
{1}/{\sqrt{3}} & 0 & {2}/{\sqrt{6}} \\
\end{pmatrix}\\
&= \begin{pmatrix}
{1}/{\sqrt{3}} & {i}/{\sqrt{2}} & - {1}/{\sqrt{6}} \\
{i}/{\sqrt{3}} & {1}/{\sqrt{2}} & - {i}/{\sqrt{6}} \\
{1}/{\sqrt{3}} & 0 & {2}/{\sqrt{6}} \\
\end{pmatrix}^{- 1} \left( \begin{matrix}
\begin{pmatrix}
0 & i & - 1 \\
 - i & 0 & - i \\
 - 1 & i & 0 \\
\end{pmatrix}\begin{pmatrix}
{1}/{\sqrt{3}} \\
{i}/{\sqrt{3}} \\
{1}/{\sqrt{3}} \\
\end{pmatrix} \end{matrix} \right. \\
&\quad \left. \begin{matrix} \begin{pmatrix}
0 & i & - 1 \\
 - i & 0 & - i \\
 - 1 & i & 0 \\
\end{pmatrix}\begin{pmatrix}
{i}/{\sqrt{2}} \\
{1}/{\sqrt{2}} \\
0 \\
\end{pmatrix} & \begin{pmatrix}
0 & i & - 1 \\
 - i & 0 & - i \\
 - 1 & i & 0 \\
\end{pmatrix}\begin{pmatrix}
 - {1}/{\sqrt{6}} \\
 - {i}/{\sqrt{6}} \\
{2}/{\sqrt{6}} \\
\end{pmatrix} \\
\end{matrix} \right) \\
&= \begin{pmatrix}
{1}/{\sqrt{3}} & {i}/{\sqrt{2}} & - {1}/{\sqrt{6}} \\
{i}/{\sqrt{3}} & {1}/{\sqrt{2}} & - {i}/{\sqrt{6}} \\
{1}/{\sqrt{3}} & 0 & {2}/{\sqrt{6}} \\
\end{pmatrix}^{- 1}\begin{pmatrix}
 - 2\begin{pmatrix}
{1}/{\sqrt{3}} \\
{i}/{\sqrt{3}} \\
{1}/{\sqrt{3}} \\
\end{pmatrix} & 1\begin{pmatrix}
{i}/{\sqrt{2}} \\
{1}/{\sqrt{2}} \\
0 \\
\end{pmatrix} & 1\begin{pmatrix}
 - {1}/{\sqrt{6}} \\
 - {i}/{\sqrt{6}} \\
{2}/{\sqrt{6}} \\
\end{pmatrix} \\
\end{pmatrix}\\
&= \begin{pmatrix}
{1}/{\sqrt{3}} & {i}/{\sqrt{2}} & - {1}/{\sqrt{6}} \\
{i}/{\sqrt{3}} & {1}/{\sqrt{2}} & - {i}/{\sqrt{6}} \\
{1}/{\sqrt{3}} & 0 & {2}/{\sqrt{6}} \\
\end{pmatrix}^{- 1} \left( \begin{matrix} - 2\begin{pmatrix}
{1}/{\sqrt{3}} \\
{i}/{\sqrt{3}} \\
{1}/{\sqrt{3}} \\
\end{pmatrix} + 0\begin{pmatrix}
{i}/{\sqrt{2}} \\
{1}/{\sqrt{2}} \\
0 \\
\end{pmatrix} + 0\begin{pmatrix}
 - {1}/{\sqrt{6}} \\
 - {i}/{\sqrt{6}} \\
{2}/{\sqrt{6}} \\
\end{pmatrix} \end{matrix} \right.\\
&\quad \left. \begin{matrix} 0\begin{pmatrix}
{1}/{\sqrt{3}} \\
{i}/{\sqrt{3}} \\
{1}/{\sqrt{3}} \\
\end{pmatrix} + 1\begin{pmatrix}
{i}/{\sqrt{2}} \\
{1}/{\sqrt{2}} \\
0 \\
\end{pmatrix} + 0\begin{pmatrix}
 - {1}/{\sqrt{6}} \\
 - {i}/{\sqrt{6}} \\
{2}/{\sqrt{6}} \\
\end{pmatrix} & 0\begin{pmatrix}
{1}/{\sqrt{3}} \\
{i}/{\sqrt{3}} \\
{1}/{\sqrt{3}} \\
\end{pmatrix} + 0\begin{pmatrix}
{i}/{\sqrt{2}} \\
{1}/{\sqrt{2}} \\
0 \\
\end{pmatrix} + 1\begin{pmatrix}
 - {1}/{\sqrt{6}} \\
 - {i}/{\sqrt{6}} \\
{2}/{\sqrt{6}} \\
\end{pmatrix} \end{matrix} \right)\\
&= \begin{pmatrix}
{1}/{\sqrt{3}} & {i}/{\sqrt{2}} & - {1}/{\sqrt{6}} \\
{i}/{\sqrt{3}} & {1}/{\sqrt{2}} & - {i}/{\sqrt{6}} \\
{1}/{\sqrt{3}} & 0 & {2}/{\sqrt{6}} \\
\end{pmatrix}^{- 1}\begin{pmatrix}
{1}/{\sqrt{3}} & {i}/{\sqrt{2}} & - {1}/{\sqrt{6}} \\
{i}/{\sqrt{3}} & {1}/{\sqrt{2}} & - {i}/{\sqrt{6}} \\
{1}/{\sqrt{3}} & 0 & {2}/{\sqrt{6}} \\
\end{pmatrix}\begin{pmatrix}
 - 2 & \  & O \\
\  & 1 & \  \\
O & \  & 1 \\
\end{pmatrix}\\
&= \begin{pmatrix}
 - 2 & \  & O \\
\  & 1 & \  \\
O & \  & 1 \\
\end{pmatrix}
\end{align*}
よって、$U = \begin{pmatrix}
{1}/{\sqrt{3}} & {i}/{\sqrt{2}} & - {1}/{\sqrt{6}} \\
{i}/{\sqrt{3}} & {1}/{\sqrt{2}} & - {i}/{\sqrt{6}} \\
{1}/{\sqrt{3}} & 0 & {2}/{\sqrt{6}} \\
\end{pmatrix}$が得られる。\par
$\mathbb{C}^{3} = W_{A}( - 2) \oplus W_{A}(1)$よりそのvector空間$\mathbb{C}^{3}$からそれらの固有空間たち$W_{A}( - 2)$、$W_{A}(1)$への正射影子たちそれぞれを$P_{W_{A}( - 2)}$、$P_{W_{A}(1)}$とおくと、次のようになるので、
\begin{align*}
&\quad \left\{ \begin{matrix}
P_{W_{A}( - 2)}\begin{pmatrix}
{1}/{\sqrt{3}} \\
{i}/{\sqrt{3}} \\
{1}/{\sqrt{3}} \\
\end{pmatrix} = \begin{pmatrix}
{1}/{\sqrt{3}} \\
{i}/{\sqrt{3}} \\
{1}/{\sqrt{3}} \\
\end{pmatrix} \\
P_{W_{A}( - 2)}\begin{pmatrix}
{i}/{\sqrt{2}} \\
{1}/{\sqrt{2}} \\
0 \\
\end{pmatrix} = \begin{pmatrix}
0 \\
0 \\
0 \\
\end{pmatrix} \\
P_{W_{A}( - 2)}\begin{pmatrix}
 - {1}/{\sqrt{6}} \\
 - {i}/{\sqrt{6}} \\
{2}/{\sqrt{6}} \\
\end{pmatrix} = \begin{pmatrix}
0 \\
0 \\
0 \\
\end{pmatrix} \\
\end{matrix} \right.\ \\
&\Leftrightarrow P_{W_{A}( - 2)}\begin{pmatrix}
{1}/{\sqrt{3}} & {i}/{\sqrt{2}} & - {1}/{\sqrt{6}} \\
{i}/{\sqrt{3}} & {1}/{\sqrt{2}} & - {i}/{\sqrt{6}} \\
{1}/{\sqrt{3}} & 0 & {2}/{\sqrt{6}} \\
\end{pmatrix} \\
&\quad = \begin{pmatrix}
P_{W_{A}( - 2)}\begin{pmatrix}
{1}/{\sqrt{3}} \\
{i}/{\sqrt{3}} \\
{1}/{\sqrt{3}} \\
\end{pmatrix} & P_{W_{A}( - 2)}\begin{pmatrix}
{i}/{\sqrt{2}} \\
{1}/{\sqrt{2}} \\
0 \\
\end{pmatrix} & P_{W_{A}( - 2)}\begin{pmatrix}
 - {1}/{\sqrt{6}} \\
 - {i}/{\sqrt{6}} \\
{2}/{\sqrt{6}} \\
\end{pmatrix} \\
\end{pmatrix} = \begin{pmatrix}
{1}/{\sqrt{3}} & 0 & 0 \\
{i}/{\sqrt{3}} & 0 & 0 \\
{1}/{\sqrt{3}} & 0 & 0 \\
\end{pmatrix}\\
&\Leftrightarrow P_{W_{A}( - 2)} = P_{W_{A}( - 2)}\begin{pmatrix}
{1}/{\sqrt{3}} & {i}/{\sqrt{2}} & - {1}/{\sqrt{6}} \\
{i}/{\sqrt{3}} & {1}/{\sqrt{2}} & - {i}/{\sqrt{6}} \\
{1}/{\sqrt{3}} & 0 & {2}/{\sqrt{6}} \\
\end{pmatrix}\begin{pmatrix}
{1}/{\sqrt{3}} & {i}/{\sqrt{2}} & - {1}/{\sqrt{6}} \\
{i}/{\sqrt{3}} & {1}/{\sqrt{2}} & - {i}/{\sqrt{6}} \\
{1}/{\sqrt{3}} & 0 & {2}/{\sqrt{6}} \\
\end{pmatrix}^{*} \\
&\quad = \begin{pmatrix}
{1}/{\sqrt{3}} & 0 & 0 \\
{i}/{\sqrt{3}} & 0 & 0 \\
{1}/{\sqrt{3}} & 0 & 0 \\
\end{pmatrix}\begin{pmatrix}
{1}/{\sqrt{3}} & - {i}/{\sqrt{3}} & {1}/{\sqrt{3}} \\
 - {i}/{\sqrt{2}} & {1}/{\sqrt{2}} & 0 \\
 - {1}/{\sqrt{6}} & {i}/{\sqrt{6}} & {2}/{\sqrt{6}} \\
\end{pmatrix} = \begin{pmatrix}
{1}/{3} & - {i}/{3} & {1}/{3} \\
{i}/{3} & {1}/{3} & {i}/{3} \\
{1}/{3} & - {i}/{3} & {1}/{3} \\
\end{pmatrix}\\
&\quad \left\{ \begin{matrix}
P_{W_{A}(1)}\begin{pmatrix}
{1}/{\sqrt{3}} \\
{i}/{\sqrt{3}} \\
{1}/{\sqrt{3}} \\
\end{pmatrix} = \begin{pmatrix}
0 \\
0 \\
0 \\
\end{pmatrix} \\
P_{W_{A}(1)}\begin{pmatrix}
{i}/{\sqrt{2}} \\
{1}/{\sqrt{2}} \\
0 \\
\end{pmatrix} = \begin{pmatrix}
{i}/{\sqrt{2}} \\
{1}/{\sqrt{2}} \\
0 \\
\end{pmatrix} \\
P_{W_{A}(1)}\begin{pmatrix}
 - {1}/{\sqrt{6}} \\
 - {i}/{\sqrt{6}} \\
{2}/{\sqrt{6}} \\
\end{pmatrix} = \begin{pmatrix}
 - {1}/{\sqrt{6}} \\
 - {i}/{\sqrt{6}} \\
{2}/{\sqrt{6}} \\
\end{pmatrix} \\
\end{matrix} \right.\ \\
&\Leftrightarrow P_{W_{A}(1)}\begin{pmatrix}
{1}/{\sqrt{3}} & {i}/{\sqrt{2}} & - {1}/{\sqrt{6}} \\
{i}/{\sqrt{3}} & {1}/{\sqrt{2}} & - {i}/{\sqrt{6}} \\
{1}/{\sqrt{3}} & 0 & {2}/{\sqrt{6}} \\
\end{pmatrix} \\
&\quad = \begin{pmatrix}
P_{W_{A}(1)}\begin{pmatrix}
{1}/{\sqrt{3}} \\
{i}/{\sqrt{3}} \\
{1}/{\sqrt{3}} \\
\end{pmatrix} & P_{W_{A}(1)}\begin{pmatrix}
{i}/{\sqrt{2}} \\
{1}/{\sqrt{2}} \\
0 \\
\end{pmatrix} & P_{W_{A}(1)}\begin{pmatrix}
 - {1}/{\sqrt{6}} \\
 - {i}/{\sqrt{6}} \\
{2}/{\sqrt{6}} \\
\end{pmatrix} \\
\end{pmatrix} = \begin{pmatrix}
0 & {i}/{\sqrt{2}} & - {1}/{\sqrt{6}} \\
0 & {1}/{\sqrt{2}} & - {i}/{\sqrt{6}} \\
0 & 0 & {2}/{\sqrt{6}} \\
\end{pmatrix}\\
&\Leftrightarrow P_{W_{A}(1)} = P_{W_{A}(1)}\begin{pmatrix}
{1}/{\sqrt{3}} & {i}/{\sqrt{2}} & - {1}/{\sqrt{6}} \\
{i}/{\sqrt{3}} & {1}/{\sqrt{2}} & - {i}/{\sqrt{6}} \\
{1}/{\sqrt{3}} & 0 & {2}/{\sqrt{6}} \\
\end{pmatrix}\begin{pmatrix}
{1}/{\sqrt{3}} & {i}/{\sqrt{2}} & - {1}/{\sqrt{6}} \\
{i}/{\sqrt{3}} & {1}/{\sqrt{2}} & - {i}/{\sqrt{6}} \\
{1}/{\sqrt{3}} & 0 & {2}/{\sqrt{6}} \\
\end{pmatrix}^{*} \\
&\quad = \begin{pmatrix}
0 & {i}/{\sqrt{2}} & - {1}/{\sqrt{6}} \\
0 & {1}/{\sqrt{2}} & - {i}/{\sqrt{6}} \\
0 & 0 & {2}/{\sqrt{6}} \\
\end{pmatrix}\begin{pmatrix}
{1}/{\sqrt{3}} & - {i}/{\sqrt{3}} & {1}/{\sqrt{3}} \\
 - {i}/{\sqrt{2}} & {1}/{\sqrt{2}} & 0 \\
 - {1}/{\sqrt{6}} & {i}/{\sqrt{6}} & {2}/{\sqrt{6}} \\
\end{pmatrix} = \begin{pmatrix}
{2}/{3} & {i}/{3} & - {1}/{3} \\
 - {i}/{3} & {2}/{3} & - {i}/{3} \\
 - {1}/{3} & {i}/{3} & {2}/{3} \\
\end{pmatrix}
\end{align*}
spectrum分解により次式が成り立つ。
\begin{align*}
\begin{pmatrix}
0 & i & - 1 \\
 - i & 0 & - i \\
 - 1 & i & 0 \\
\end{pmatrix} = - 2\begin{pmatrix}
{1}/{3} & - {i}/{3} & {1}/{3} \\
{i}/{3} & {1}/{3} & {i}/{3} \\
{1}/{3} & - {i}/{3} & {1}/{3} \\
\end{pmatrix} + \begin{pmatrix}
{2}/{3} & {i}/{3} & - {1}/{3} \\
 - {i}/{3} & {2}/{3} & - {i}/{3} \\
 - {1}/{3} & {i}/{3} & {2}/{3} \\
\end{pmatrix}
\end{align*}
%\hypertarget{ux6b63ux898fux5909ux63dbux3068ux6b63ux5c04ux5f71ux5b50}{%
\subsubsection{正規変換と正射影子}%\label{ux6b63ux898fux5909ux63dbux3068ux6b63ux5c04ux5f71ux5b50}}
\begin{thm}\label{2.3.9.9}
体$\mathbb{C}$上の$1 \leq \dim V = n$なる内積空間$(V,\varPhi)$、$f \in L(V,V)$なる正規変換$f$について、その正規変換$f$の固有値$\lambda$が与えられたとき、その複素数$\overline{\lambda}$はその随伴変換$f^{*}$の固有値である。さらに、固有空間について、$W_{f}(\lambda) = W_{f^{*}}\left( \overline{\lambda} \right)$が成り立つ。
\end{thm}
\begin{proof}
体$\mathbb{C}$上の$1 \leq \dim V = n$なる内積空間$(V,\varPhi)$、$f \in L(V,V)$なる正規変換$f$について、その正規変換$f$の固有値$\lambda$が与えられたとき、$\forall\mathbf{v} \in W_{f}(\lambda)$に対し、$A_{nn} \in M_{nn}\left( \mathbb{C} \right)$なる行列$A_{nn}$が対応する行列となっている線形写像が$L_{A_{nn}}:\mathbb{C}^{n} \rightarrow \mathbb{C}^{n};\mathbf{v} \mapsto A_{nn}\mathbf{v}$とおかれれば、次式が成り立つので、
\begin{align*}
L_{\left[ I_{V} \right]_{\mathcal{B}}^{\mathcal{B}}} = \varphi_{\mathcal{B}}^{- 1} \circ I_{V} \circ \varphi_{\mathcal{B}} = \varphi_{\mathcal{B}}^{- 1} \circ \varphi_{\mathcal{B}} = I_{K^{n}}
\end{align*}
$\left[ I_{V} \right]_{\mathcal{B}}^{\mathcal{B}} = I_{n}$が成り立つ。これにより、$\left[ I_{V}^{*} \right]_{\mathcal{B}}^{\mathcal{B}} = {\left[ I_{V} \right]_{\mathcal{B}}^{\mathcal{B}}}^{*} = \left[ I_{V} \right]_{\mathcal{B}}^{\mathcal{B}} = I_{n}$が成り立つことになり、したがって、定理\ref{2.3.8.5}より$\left( \lambda I_{V} - f \right)^{*} = \overline{\lambda}I_{V}^{*} - f^{*} = \overline{\lambda}I_{V} - f^{*}$が成り立つ。\par
また、次のようになるので、
\begin{align*}
\left( \lambda I_{V} - f \right)^{*} \circ \left( \lambda I_{V} - f \right) &= \left( \overline{\lambda}I_{V} - f^{*} \right) \circ \left( \lambda I_{V} - f \right)\\
&= \overline{\lambda}I_{V} \circ \lambda I_{V} - \overline{\lambda}I_{V} \circ f - f^{*} \circ \lambda I_{V} - f^{*} \circ f\\
&= |\lambda|^{2}\lambda I_{V} - \overline{\lambda}f - \lambda f^{*} - f^{*} \circ f\\
&= |\lambda|^{2}\lambda I_{V} - \lambda f^{*} - \overline{\lambda}f - f \circ f^{*}\\
&= \lambda I_{V} \circ \overline{\lambda}I_{V} - \lambda I_{V} \circ f^{*} - f \circ \lambda I_{V} - f \circ f^{*}\\
&= \left( \lambda I_{V} - f \right) \circ \left( \overline{\lambda}I_{V} - f^{*} \right)\\
&= \left( \lambda I_{V} - f \right) \circ \left( \lambda I_{V} - f \right)^{*}
\end{align*}
その線形写像$\lambda I_{V} - f$は正規変換でもある。\par
そこで、定理\ref{2.3.8.8}よりその内積空間$(V,\varPhi)$から誘導されるnorm空間$\left( V,\varphi_{\varPhi} \right)$について、$\varphi_{\varPhi} \circ \left( \lambda I_{V} - f \right) = \varphi_{\varPhi} \circ \left( \lambda I_{V} - f \right)^{*}$が成り立つことになる。これにより、$\forall\mathbf{v} \in W_{f}(\lambda)$に対し、$\left( \lambda I_{V} - f \right)\left( \mathbf{v} \right) = \mathbf{0}$が成り立つので、次のようになる。
\begin{align*}
\varphi_{\varPhi}\left( \left( \overline{\lambda}I_{V} - f^{*} \right)\left( \mathbf{v} \right) \right) &= \varphi_{\varPhi}\left( \left( \lambda I_{V} - f \right)^{*}\left( \mathbf{v} \right) \right)\\
&= \varphi_{\varPhi} \circ \left( \lambda I_{V} - f \right)^{*}\left( \mathbf{v} \right)\\
&= \varphi_{\varPhi} \circ \left( \lambda I_{V} - f \right)\left( \mathbf{v} \right)\\
&= \varphi_{\varPhi}\left( \left( \lambda I_{V} - f \right)\left( \mathbf{v} \right) \right)\\
&= \varphi_{\varPhi}\left( \mathbf{0} \right) = 0
\end{align*}
これにより、$\left( \overline{\lambda}I_{V} - f^{*} \right)\left( \mathbf{v} \right) = \mathbf{0}$が成り立つので、その複素数$\overline{\lambda}$はその随伴変換$f^{*}$の固有値である。なお、$W_{f}(\lambda) \subseteq W_{f^{*}}\left( \overline{\lambda} \right)$が成り立つ。\par
一方で、$\forall\mathbf{v} \in W_{f^{*}}\left( \overline{\lambda} \right)$に対し、$\left( \lambda I_{V} - f \right)^{*}\left( \mathbf{v} \right) = \mathbf{0}$が成り立つので、次のようになる。
\begin{align*}
\varphi_{\varPhi}\left( \left( \lambda I_{V} - f \right)\left( \mathbf{v} \right) \right) &= \varphi_{\varPhi} \circ \left( \lambda I_{V} - f \right)\left( \mathbf{v} \right)\\
&= \varphi_{\varPhi} \circ \left( \lambda I_{V} - f \right)^{*}\left( \mathbf{v} \right)\\
&= \varphi_{\varPhi}\left( \left( \lambda I_{V} - f \right)^{*}\left( \mathbf{v} \right) \right)\\
&= \varphi_{\varPhi}\left( \mathbf{0} \right) = 0
\end{align*}
これにより、$\left( \lambda I_{V} - f \right)\left( \mathbf{v} \right) = \mathbf{0}$が成り立つので、$W_{f}(\lambda) \supseteq W_{f^{*}}\left( \overline{\lambda} \right)$が成り立つ。以上の議論により、$W_{f}(\lambda) = W_{f^{*}}\left( \overline{\lambda} \right)$が成り立つ。
\end{proof}
\begin{thm}\label{2.3.9.10}
体$\mathbb{C}$上の$1 \leq \dim V = n$なる内積空間$(V,\varPhi)$が与えられたとき、そのvector空間$V$の部分空間$W$について、$P \in L(V,V)$なる線形写像$P$がその部分空間$W$への正射影子であるならそのときに限り、その線形写像$P$の固有値が$0$か$1$のHermite変換である。
\end{thm}
\begin{proof}
体$\mathbb{C}$上の$1 \leq \dim V = n$なる内積空間$(V,\varPhi)$が与えられたとき、そのvector空間$V$の部分空間$W$について、$P \in L(V,V)$なる線形写像$P$がその部分空間$W$への正射影子であるなら、その線形写像$P$は正射影であるかつ、射影子でもある。定理\ref{2.3.9.7}よりその線形写像$P$が正射影であるならそのときに限り、その射影$P$がHermite変換である。一方で、定理\ref{2.2.1.11}よりある補空間$U$が存在して、$V = U \oplus W$が成り立つので、$\dim U = r$としこれらの部分空間たち$U$、$W$の基底をそれぞれ$\alpha$、$\beta$とおき、$\mathcal{B} =\left\langle \begin{matrix}
\alpha & \beta \\
\end{matrix} \right\rangle$とおかれると、定理\ref{2.2.1.5}よりこれはそのvector空間$V$の基底をなす。そこで、$\mathcal{B} =\left\langle \mathbf{v}_{i} \right\rangle_{i \in \varLambda_{n}}$とおかれると、そのvector空間$\mathbb{C}^{n}$の標準直交基底$\left\langle \mathbf{e}_{i} \right\rangle_{i \in \varLambda_{n}}$を用いて$A_{nn} \in M_{nn}\left( \mathbb{C} \right)$なる行列$A_{nn}$が対応する行列となっている線形写像が$L_{A_{nn}}:\mathbb{C}^{n} \rightarrow \mathbb{C}^{n};\mathbf{v} \mapsto A_{nn}\mathbf{v}$とおかれれば、その射影$P$の固有多項式$\varPhi_{P}$は次のようになる。
\begin{align*}
\varPhi_{P} &= \det\left[ XI_{V} - P \right]_{\mathcal{B}}^{\mathcal{B}}\\
&= \det\begin{pmatrix}
L_{\left[ XI_{V} - P \right]_{\mathcal{B}}^{\mathcal{B}}}\left( \mathbf{e}_{1} \right) & \cdots & L_{\left[ XI_{V} - P \right]_{\mathcal{B}}^{\mathcal{B}}}\left( \mathbf{e}_{r} \right) & L_{\left[ XI_{V} - P \right]_{\mathcal{B}}^{\mathcal{B}}}\left( \mathbf{e}_{r + 1} \right) & \cdots & L_{\left[ XI_{V} - P \right]_{\mathcal{B}}^{\mathcal{B}}}\left( \mathbf{e}_{n} \right) \\
\end{pmatrix}\\
&= \det\left( \begin{matrix}
\varphi_{\mathcal{B}}^{- 1} \circ \left( XI_{V} - P \right) \circ \varphi_{\mathcal{B}}\left( \mathbf{e}_{1} \right) & \cdots & \varphi_{\mathcal{B}}^{- 1} \circ \left( XI_{V} - P \right) \circ \varphi_{\mathcal{B}}\left( \mathbf{e}_{r} \right) \\
\end{matrix} \right.\\
&\quad \left. \begin{matrix}
\varphi_{\mathcal{B}}^{- 1} \circ \left( XI_{V} - P \right) \circ \varphi_{\mathcal{B}}\left( \mathbf{e}_{r + 1} \right) & \cdots & \varphi_{\mathcal{B}}^{- 1} \circ \left( XI_{V} - P \right) \circ \varphi_{\mathcal{B}}\left( \mathbf{e}_{n} \right) \\
\end{matrix} \right)\\
&= \det\left( \begin{matrix}
\varphi_{\mathcal{B}}^{- 1} \circ XI_{V} \circ \varphi_{\mathcal{B}}\left( \mathbf{e}_{1} \right) - \varphi_{\mathcal{B}}^{- 1} \circ P \circ \varphi_{\mathcal{B}}\left( \mathbf{e}_{1} \right) & \cdots \\
\end{matrix} \right.\\
&\quad \begin{matrix} \varphi_{\mathcal{B}}^{- 1} \circ XI_{V} \circ \varphi_{\mathcal{B}}\left( \mathbf{e}_{r} \right) - \varphi_{\mathcal{B}}^{- 1} \circ P \circ \varphi_{\mathcal{B}}\left( \mathbf{e}_{r} \right) \end{matrix} \\
&\quad \begin{matrix} \varphi_{\mathcal{B}}^{- 1} \circ XI_{V} \circ \varphi_{\mathcal{B}}\left( \mathbf{e}_{r + 1} \right) - \varphi_{\mathcal{B}}^{- 1} \circ P \circ \varphi_{\mathcal{B}}\left( \mathbf{e}_{r + 1} \right) \end{matrix} \\
&\quad \left. \begin{matrix}
\cdots & \varphi_{\mathcal{B}}^{- 1} \circ XI_{V} \circ \varphi_{\mathcal{B}}\left( \mathbf{e}_{n} \right) - \varphi_{\mathcal{B}}^{- 1} \circ P \circ \varphi_{\mathcal{B}}\left( \mathbf{e}_{n} \right) \\
\end{matrix} \right)\\
&= \det\left( \begin{matrix}
X\mathbf{e}_{1} - \varphi_{\mathcal{B}}^{- 1}\left( P\left( \mathbf{o}_{1} \right) \right) & \cdots & X\mathbf{e}_{r} - \varphi_{\mathcal{B}}^{- 1}\left( P\left( \mathbf{o}_{r} \right) \right) \\
\end{matrix} \right.\\
&\quad \left. \begin{matrix}
X\mathbf{e}_{r + 1} - \varphi_{\mathcal{B}}^{- 1}\left( P\left( \mathbf{o}_{r + 1} \right) \right) & \cdots & X\mathbf{e}_{n} - \varphi_{\mathcal{B}}^{- 1}\left( P\left( \mathbf{o}_{n} \right) \right) \\
\end{matrix} \right)\\
&= \det\begin{pmatrix}
X\mathbf{e}_{1} - \varphi_{\mathcal{B}}^{- 1}\left( \mathbf{0} \right) & \cdots & X\mathbf{e}_{r} - \varphi_{\mathcal{B}}^{- 1}\left( \mathbf{0} \right) & X\mathbf{e}_{r + 1} - \varphi_{\mathcal{B}}^{- 1}\left( \mathbf{o}_{r + 1} \right) & \cdots & X\mathbf{e}_{n} - \varphi_{\mathcal{B}}^{- 1}\left( \mathbf{o}_{n} \right) \\
\end{pmatrix}\\
&= \det\begin{pmatrix}
X\mathbf{e}_{1} - \mathbf{0} & \cdots & X\mathbf{e}_{r} - \mathbf{0} & X\mathbf{e}_{r + 1} - \mathbf{e}_{r + 1} & \cdots & X\mathbf{e}_{n} - \mathbf{e}_{n} \\
\end{pmatrix}\\
&= \det\begin{pmatrix}
X\mathbf{e}_{1} & \cdots & X\mathbf{e}_{r} & (X - 1)\mathbf{e}_{r + 1} & \cdots & (X - 1)\mathbf{e}_{n} \\
\end{pmatrix}\\
&= \left| \begin{matrix}
X & \  & \  & \  & \  & O \\
\  & \ddots & \  & \  & \  & \  \\
\  & \  & X & \  & \  & \  \\
\  & \  & \  & X - 1 & \  & \  \\
\  & \  & \  & \  & \ddots & \  \\
O & \  & \  & \  & \  & X - 1 \\
\end{matrix} \right| = X^{r}(X - 1)^{n - r}
\end{align*}
定理\ref{2.2.2.6}よりその線形写像$P$の固有値が$0$か$1$のみとなる。\par
逆に、$P \in L(V,V)$なる線形写像$P$がこれの固有値が$0$か$1$のHermite変換であるなら、定理\ref{2.3.9.7}よりその線形写像$P$がHermite変換であるならそのときに限り、その線形写像$P$が正射影である。これにより、そのvector空間$V$がそれらの部分空間たち$W$、$W^{\bot}$に直和分解されたとき、そのvector空間$V$からその部分空間$W$への正射影$P_{W}$がその線形写像$P$そのものである。さらに、定理\ref{2.2.1.8}よりその射影$P_{W}$は射影子でもある。よって、その線形写像$P$はその部分空間$W$への正射影子である。
\end{proof}
\begin{thm}\label{2.3.9.11}
体$\mathbb{C}$上の$1 \leq \dim V = n$なる内積空間$(V,\varPhi)$が与えられたとき、そのvector空間$V$の2つの正射影子たち$P$、$Q$について、次のことは同値である。
\begin{itemize}
\item
  $\varPhi|V(P) \times V(Q) = 0$が成り立つ。
\item
  $P \circ Q = 0$が成り立つ。
\item
  $Q \circ P = 0$が成り立つ。
\end{itemize}
\end{thm}
\begin{proof}
体$\mathbb{C}$上の$1 \leq \dim V = n$なる内積空間$(V,\varPhi)$が与えられたとき、そのvector空間$V$の2つの正射影子たち$P$、$Q$について、$\varPhi|V(P) \times V(Q) = 0$が成り立つなら、定理\ref{2.3.9.10}より$\forall P\left( \mathbf{v} \right) \in V(P)\forall Q\left( \mathbf{w} \right) \in V(Q)$に対し、次のようになる。
\begin{align*}
\varPhi|V(P) \times V(Q)\left( P\left( \mathbf{v} \right),Q\left( \mathbf{w} \right) \right) &= \varPhi\left( P\left( \mathbf{v} \right),Q\left( \mathbf{w} \right) \right)\\
&= \varPhi\left( \mathbf{v},P^{*} \circ Q\left( \mathbf{w} \right) \right)\\
&= \varPhi\left( \mathbf{v},P \circ Q\left( \mathbf{w} \right) \right) = 0
\end{align*}
これが成り立つならそのときに限り、$\forall\mathbf{v},\mathbf{w} \in V$に対し、$\varPhi\left( \mathbf{v},P \circ Q\left( \mathbf{w} \right) \right) = 0$が成り立つことになり、定理\ref{2.3.6.5}より$P \circ Q = 0$が成り立つ。\par
逆に、$P \circ Q = 0$が成り立つなら、$\forall P\left( \mathbf{v} \right) \in V(P)\forall Q\left( \mathbf{w} \right) \in V(Q)$に対し、定理\ref{2.3.9.10}より次のようになる。
\begin{align*}
\varPhi|V(P) \times V(Q)\left( P\left( \mathbf{v} \right),Q\left( \mathbf{w} \right) \right) &= \varPhi\left( P\left( \mathbf{v} \right),Q\left( \mathbf{w} \right) \right)\\
&= \varPhi\left( \mathbf{v},P^{*} \circ Q\left( \mathbf{w} \right) \right)\\
&= \varPhi\left( \mathbf{v},P \circ Q\left( \mathbf{w} \right) \right)\\
&= \varPhi\left( \mathbf{v},0\left( \mathbf{w} \right) \right)\\
&= \varPhi\left( \mathbf{v},\mathbf{0} \right) = 0
\end{align*}
よって、$\varPhi|V(P) \times V(Q) = 0$が成り立つ。\par
同様にして、$\varPhi|V(P) \times V(Q) = 0$が成り立つならそのときに限り、$Q \circ P = 0$が成り立つことも示される。
\end{proof}
\begin{thm}\label{2.3.9.12}
体$\mathbb{C}$上の$1 \leq \dim V = n$なる内積空間$(V,\varPhi)$が与えられたとき、vector空間$V$の正射影子$P_{i}:V \rightarrow V$の添数集合$\varLambda_{s}$によって添数づけられた族$\left\{ P_{i} \right\}_{i \in \varLambda_{s}}$が与えられたとき、次のことを満たすなら、
\begin{itemize}
\item
  $\forall i \in \varLambda_{s}$に対し、$P_{i} \neq 0$が成り立つ。
\item
  $\forall i,j \in \varLambda_{s}$に対し、$i \neq j$が成り立つなら、$P_{j} \circ P_{i} = 0$が成り立つ。
\item
  $i \in \varLambda_{s}$なるそれらの線形写像たち$P_{i}$について、そのvector空間$V$の恒等写像$I_{V}$を用いて$\sum_{i \in \varLambda_{s}} P_{i} = I_{V}$が成り立つ。
\end{itemize}
複素数たち$a_{i}$を用いた線形写像$\sum_{i \in \varLambda_{s}} {a_{i}P_{i}}$は正規変換である。さらに、その線形写像$\sum_{i \in \varLambda_{s}} {a_{i}P_{i}}$のspectrum分解がまさに$\sum_{i \in \varLambda_{s}} {a_{i}P_{i}}$のことである、即ち、それらの複素数たち$a_{i}$がその線形写像$\sum_{i \in \varLambda_{s}} {a_{i}P_{i}}$の固有値たちでその射影$P_{i}$がそのvector空間$V$からその固有空間$W_{\sum_{i \in \varLambda_{s}} {a_{i}P_{i}}}\left( a_{i} \right)$への正射影子である。
\end{thm}
\begin{proof}
体$\mathbb{C}$上の$1 \leq \dim V = n$なる内積空間$(V,\varPhi)$が与えられたとき、vector空間$V$の正射影子$P_{i}:V \rightarrow V$の添数集合$\varLambda_{s}$によって添数づけられた族$\left\{ P_{i} \right\}_{i \in \varLambda_{s}}$が与えられたとき、次のことを満たすなら、
\begin{itemize}
\item
  $\forall i \in \varLambda_{s}$に対し、$P_{i} \neq 0$が成り立つ。
\item
  $\forall i,j \in \varLambda_{s}$に対し、$i \neq j$が成り立つなら、$P_{j} \circ P_{i} = 0$が成り立つ。
\item
  $i \in \varLambda_{s}$なるそれらの線形写像たち$P_{i}$について、そのvector空間$V$の恒等写像$I_{V}$を用いて$\sum_{i \in \varLambda_{s}} P_{i} = I_{V}$が成り立つ。
\end{itemize}
定理\ref{2.3.9.10}よりそれらの線形写像たち$P_{i}$はこれの固有値が$0$か$1$のHermite変換である。したがって、複素数たち$a_{i}$を用いた線形写像$\sum_{i \in \varLambda_{s}} {a_{i}P_{i}}$が$f$とおかれると、次のようになるので、
\begin{align*}
f^{*} \circ f &= \left( \sum_{i \in \varLambda_{s}} {a_{i}P_{i}} \right)^{*} \circ \left( \sum_{i \in \varLambda_{s}} {a_{i}P_{i}} \right)\\
&= \left( \sum_{j \in \varLambda_{s}} {\overline{a_{j}}P_{j}^{*}} \right) \circ \left( \sum_{i \in \varLambda_{s}} {a_{i}P_{i}} \right)\\
&= \sum_{i,j \in \varLambda_{s}} {\overline{a_{j}}a_{i}P_{j}^{*} \circ P_{i}}\\
&= \sum_{i,j \in \varLambda_{s}} {\overline{a_{j}}a_{i}P_{j} \circ P_{i}}\\
&= \sum_{\scriptsize \begin{matrix} i,j \in \varLambda_{s} \\i = j \\\end{matrix}} {\overline{a_{j}}a_{i}P_{j} \circ P_{i}} + \sum_{\scriptsize \begin{matrix} i,j \in \varLambda_{s} \\i \neq j \\\end{matrix}} {\overline{a_{j}}a_{i}P_{j} \circ P_{i}}\\
&= \sum_{i \in \varLambda_{s} } {\overline{a_{i}}a_{i}P_{i} \circ P_{i}} + \sum_{\scriptsize \begin{matrix} i,j \in \varLambda_{s} \\i \neq j \\\end{matrix}} {\overline{a_{j}}a_{i}0}\\
&= \sum_{i \in \varLambda_{s} } {\overline{a_{i}}a_{i}P_{i} \circ P_{i}}\\
&= \sum_{i \in \varLambda_{s} } {\overline{a_{i}}a_{i}P_{i} \circ P_{i}} + \sum_{\scriptsize \begin{matrix} i,j \in \varLambda_{s} \\i \neq j \\\end{matrix}} {\overline{a_{j}}a_{i}0}\\
&= \sum_{\scriptsize \begin{matrix} i,j \in \varLambda_{s} \\i = j \\\end{matrix}} {\overline{a_{j}}a_{i}P_{i} \circ P_{j}} + \sum_{\scriptsize \begin{matrix} i,j \in \varLambda_{s} \\i \neq j \\\end{matrix}} {\overline{a_{j}}a_{i}P_{i} \circ P_{j}}\\
&= \sum_{i,j \in \varLambda_{s}} {\overline{a_{j}}a_{i}P_{i} \circ P_{j}}\\
&= \sum_{i,j \in \varLambda_{s}} {\overline{a_{j}}a_{i}P_{i} \circ P_{j}^{*}}\\
&= \left( \sum_{i \in \varLambda_{s}} {a_{i}P_{i}} \right) \circ \left( \sum_{j \in \varLambda_{s}} {\overline{a_{j}}P_{j}^{*}} \right)\\
&= \left( \sum_{i \in \varLambda_{s}} {a_{i}P_{i}} \right) \circ \left( \sum_{i \in \varLambda_{s}} {a_{i}P_{i}} \right)^{*}\\
&= f \circ f^{*}
\end{align*}
その線形写像$f$は正規変換である。\par
定理\ref{2.2.1.10}より次式が成り立ち、
\begin{align*}
V = \bigoplus_{i \in \varLambda_{s}} {V\left( P_{i} \right)}
\end{align*}
さらに、$\forall i \in \varLambda_{s}$に対し、その線形写像$P_{i}$はいづれもそのvector空間$V$からその直和因子$V\left( P_{i} \right)$への直和分解から定まる射影でもある。これらの部分空間たち$V\left( P_{i} \right)$の次元を$n_{i}$としこれの基底がそれぞれ$\left\langle \mathbf{v}_{ij} \right\rangle_{j \in \varLambda_{n_{i}}}$とおかれると、定理\ref{2.2.1.5}よりその組$\left\langle \mathbf{v}_{ij} \right\rangle_{(i,j) \in \varLambda_{s} \times \varLambda_{n_{i}}}$はそのvector空間$V$の基底をなす。そこで、これが$\mathcal{B}$とおかれると、$\forall P\left( \mathbf{v} \right) \in V\left( P_{i} \right)$に対し、仮定と定理\ref{2.2.1.8}より次のようになるので、
\begin{align*}
\left( a_{i}I_{V} - f \right)\left( P_{i}\left( \mathbf{v} \right) \right) &= \left( a_{i}I_{V} - \sum_{j \in \varLambda_{s}} {a_{j}P_{j}} \right) \circ P_{i}\left( \mathbf{v} \right)\\
&= \left( a_{i}I_{V} \circ P_{i} - \sum_{j \in \varLambda_{s}} {a_{j}P_{j} \circ P_{i}} \right)\left( \mathbf{v} \right)\\
&= \left( a_{i}I_{V} \circ P_{i} - \sum_{\scriptsize \begin{matrix} j \in \varLambda_{s} \\i \neq j \\\end{matrix}} {a_{j}P_{j} \circ P_{i}} - a_{i}P_{i} \circ P_{i} \right)\left( \mathbf{v} \right)\\
&= \left( a_{i}P_{i} - \sum_{\scriptsize \begin{matrix} j \in \varLambda_{s} \\i \neq j \\\end{matrix}} {a_{j}0} - a_{i}P_{i} \right)\left( \mathbf{v} \right)\\
&= - \sum_{\scriptsize \begin{matrix} j \in \varLambda_{s} \\i \neq j \\\end{matrix}} {a_{j}0}\left( \mathbf{v} \right) = - 0\left( \mathbf{v} \right) = \mathbf{0}
\end{align*}
$P\left( \mathbf{v} \right) \in W_{f}\left( a_{i} \right)$が成り立つ。これにより、$V\left( P_{i} \right) \subseteq W_{f}\left( a_{i} \right)$が成り立つかつ、それらの複素数たち$a_{i}$がその線形写像$f$の固有値たちとなっている。\par
逆に、$\forall\mathbf{v} \in W_{f}\left( a_{i} \right)$に対し、上記の議論によりその線形写像$f$は正規変換であるので、定理\ref{2.3.9.6}よりその固有値$a_{i}$に対する固有空間たち$W_{f}\left( a_{i} \right)$すべての和空間は直和空間であるから、$\mathbf{v} = \mathbf{0}$のとき、もちろん、$\mathbf{v} \in V\left( P_{i} \right)$が成り立つ。$\mathbf{v} \neq \mathbf{0}$のとき、$\left( a_{i}I_{V} - f \right)\left( \mathbf{v} \right) = \mathbf{0}$より$f\left( \mathbf{v} \right) = a_{i}\mathbf{v}$が成り立つことになり、上記の議論により$V = \bigoplus_{i \in \varLambda_{s}} {V\left( P_{i} \right)}$が成り立つので、$\mathbf{v} = \sum_{j \in \varLambda_{s}} {P_{j}\left( \mathbf{v}_{j} \right)}$とおかれることができる。したがって、仮定と定理\ref{2.2.1.8}より次のようになる。
\begin{align*}
\sum_{j \in \varLambda_{s}} {a_{i}P_{j}\left( \mathbf{v}_{j} \right)} &= a_{i}\sum_{j \in \varLambda_{s}} {P_{j}\left( \mathbf{v}_{j} \right)}\\
&= a_{i}\mathbf{v}\\
&= f\left( \mathbf{v} \right)\\
&= \left( \sum_{k \in \varLambda_{s}} {a_{k}P_{k}} \right)\left( \sum_{j \in \varLambda_{s}} {P_{j}\left( \mathbf{v}_{j} \right)} \right)\\
&= \sum_{j \in \varLambda_{s}} {\sum_{k \in \varLambda_{s}} {a_{k}P_{k} \circ P_{j}\left( \mathbf{v}_{j} \right)}}\\
&= \sum_{j \in \varLambda_{s}} \left( \sum_{\scriptsize \begin{matrix} k \in \varLambda_{s} \\j \neq k \\\end{matrix}} {a_{k}P_{k} \circ P_{j}\left( \mathbf{v}_{j} \right)} + a_{k}P_{k} \circ P_{k}\left( \mathbf{v}_{k} \right) \right)\\
&= \sum_{j \in \varLambda_{s}} \left( \sum_{\scriptsize \begin{matrix} k \in \varLambda_{s} \\j \neq k \\\end{matrix}} {a_{k}0\left( \mathbf{v}_{j} \right)} + a_{k}P_{k}\left( \mathbf{v}_{k} \right) \right)\\
&= \sum_{j \in \varLambda_{s}} {a_{j}P_{j}}\left( \mathbf{v}_{j} \right)
\end{align*}
したがって、$\sum_{j \in \varLambda_{s}} {\left( a_{i} - a_{j} \right)P_{j}\left( \mathbf{v}_{j} \right)} = \mathbf{0}$が得られ、直和の定義より$\forall j \in \varLambda_{s}$に対し、$\left( a_{i} - a_{j} \right)P_{j}\left( \mathbf{v}_{j} \right) = \mathbf{0}$が得られる。そこで、$\mathbf{v} \neq \mathbf{0}$より$\exists j' \in \varLambda_{s}$に対し、$P_{j'}\left( \mathbf{v}_{j'} \right) \neq \mathbf{0}$が成り立つので、$a_{i} = a_{j'}$が成り立つ。そこで、仮定より$i = j'$が成り立つかつ、$\forall j \in \varLambda_{s}$に対し、$j \neq j'$が成り立つなら、$a_{j} \neq a_{j'} = a_{i}$が成り立つので、$a_{i} - a_{j} \neq 0$が成り立つ。これにより、$P_{j}\left( \mathbf{v}_{j} \right) = \mathbf{0}$が成り立つ。したがって、次のようになるので、
\begin{align*}
\mathbf{v} &= \sum_{j \in \varLambda_{s}} {P_{j}\left( \mathbf{v}_{j} \right)} \\
&= \sum_{\scriptsize \begin{matrix} j \in \varLambda_{s} \\i \neq j \\\end{matrix}} {P_{j}\left( \mathbf{v}_{j} \right)} + P_{i}\left( \mathbf{v}_{i} \right) \\
&= \sum_{\scriptsize \begin{matrix} j \in \varLambda_{s} \\i \neq j \\\end{matrix}} \mathbf{0} + P_{i}\left( \mathbf{v}_{i} \right) \\
&= P_{i}\left( \mathbf{v}_{i} \right) \in V\left( P_{i} \right)
\end{align*}
$W_{f}\left( a_{i} \right) \subseteq V\left( P_{i} \right)$が成り立つ。\par
以上の議論により、$V\left( P_{i} \right) = W_{f}\left( a_{i} \right)$が成り立つので、その射影$P_{i}$はまさしくその直和分解におけるそのvector空間$V$からその固有空間$W_{f}\left( a_{i} \right)$への射影であるので、定理\ref{2.3.9.8}よりその射影$P_{i}$は正射影子$P_{W_{f}\left( a_{i} \right)}$であって次式が成り立つ。
\begin{align*}
f = \sum_{i \in \varLambda_{s}} {a_{i}P_{W_{f}\left( a_{i} \right)}}
\end{align*}
よって、その線形写像$f$のspectrum分解がまさに$\sum_{i \in \varLambda_{s}} {a_{i}P_{i}}$のことである、即ち、それらの複素数たち$a_{i}$がその線形写像$f$の固有値たちでその射影$P_{i}$がそのvector空間$V$からその固有空間$W_{f}\left( a_{i} \right)$への正射影子である。
\end{proof}
\begin{thebibliography}{50}
  \bibitem{1}
    松坂和夫, 線型代数入門, 岩波書店, 1980. 新装版第2刷 p363-372 ISBN978-4-00-029872-8
  \bibitem{2}
    八起数学塾. "射影子とスペクトル分解". 八起数学塾. \url{https://www.yaokisj.com/doc/PrjSpcRsl.pdf} (2022-3-19 13:11 閲覧)
\end{thebibliography}
\end{document}
