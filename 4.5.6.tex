\documentclass[dvipdfmx]{jsarticle}
\setcounter{section}{5}
\setcounter{subsection}{5}
\usepackage{xr}
\externaldocument{4.5.2}
\externaldocument{4.5.3}
\usepackage{amsmath,amsfonts,amssymb,array,comment,mathtools,url,docmute}
\usepackage{longtable,booktabs,dcolumn,tabularx,mathtools,multirow,colortbl,xcolor}
\usepackage[dvipdfmx]{graphics}
\usepackage{bmpsize}
\usepackage{amsthm}
\usepackage{enumitem}
\setlistdepth{20}
\renewlist{itemize}{itemize}{20}
\setlist[itemize]{label=•}
\renewlist{enumerate}{enumerate}{20}
\setlist[enumerate]{label=\arabic*.}
\setcounter{MaxMatrixCols}{20}
\setcounter{tocdepth}{3}
\newcommand{\rotin}{\text{\rotatebox[origin=c]{90}{$\in $}}}
\renewcommand{\thesection}{第\arabic{section}部}
\renewcommand{\thesubsection}{\arabic{section}.\arabic{subsection}}
\renewcommand{\thesubsubsection}{\arabic{section}.\arabic{subsection}.\arabic{subsubsection}}
\everymath{\displaystyle}
\allowdisplaybreaks[4]
\usepackage{vtable}
\theoremstyle{definition}
\newtheorem{thm}{定理}[subsection]
\newtheorem*{thm*}{定理}
\newtheorem{dfn}{定義}[subsection]
\newtheorem*{dfn*}{定義}
\newtheorem{axs}[dfn]{公理}
\newtheorem*{axs*}{公理}
\renewcommand{\headfont}{\bfseries}
\makeatletter
  \renewcommand{\section}{%
    \@startsection{section}{1}{\z@}%
    {\Cvs}{\Cvs}%
    {\normalfont\huge\headfont\raggedright}}
\makeatother
\makeatletter
  \renewcommand{\subsection}{%
    \@startsection{subsection}{2}{\z@}%
    {0.5\Cvs}{0.5\Cvs}%
    {\normalfont\LARGE\headfont\raggedright}}
\makeatother
\makeatletter
  \renewcommand{\subsubsection}{%
    \@startsection{subsubsection}{3}{\z@}%
    {0.4\Cvs}{0.4\Cvs}%
    {\normalfont\Large\headfont\raggedright}}
\makeatother
\makeatletter
\renewenvironment{proof}[1][\proofname]{\par
  \pushQED{\qed}%
  \normalfont \topsep6\p@\@plus6\p@\relax
  \trivlist
  \item\relax
  {
  #1\@addpunct{.}}\hspace\labelsep\ignorespaces
}{%
  \popQED\endtrivlist\@endpefalse
}
\makeatother
\renewcommand{\proofname}{\textbf{証明}}
\usepackage{tikz,graphics}
\usepackage[dvipdfmx]{hyperref}
\usepackage{pxjahyper}
\hypersetup{
 setpagesize=false,
 bookmarks=true,
 bookmarksdepth=tocdepth,
 bookmarksnumbered=true,
 colorlinks=false,
 pdftitle={},
 pdfsubject={},
 pdfauthor={},
 pdfkeywords={}}
\begin{document}
%\hypertarget{ux5358ux8abfux65cf}{%
\subsection{単調族}%\label{ux5358ux8abfux65cf}}
%\hypertarget{ux5358ux8abfux65cf-1}{%
\subsubsection{単調族}%\label{ux5358ux8abfux65cf-1}}
\begin{axs}[単調族の公理]
空集合でない集合$X$の部分集合系$\mathfrak{M}$が次のことをいづれも満たすとき、その集合$\mathfrak{M}$をその集合$X$上の単調族という\footnote{定義をよくみると、その集合$X$が一意的にきまらないので、"その集合$X$上の"というのは不適切かもしれない。}。
\begin{itemize}
\item
  その集合$\mathfrak{M}$の単調増加する任意の元の列$\left( E_{n} \right)_{n \in \mathbb{N}}$が$\bigcup_{n \in \mathbb{N}} E_{n}\in \mathfrak{M}$を満たす。
\item
  その集合$\mathfrak{M}$の単調減少する任意の元の列$\left( E_{n} \right)_{n \in \mathbb{N}}$が$\bigcap_{n \in \mathbb{N}} E_{n}\in \mathfrak{M}$を満たす。
\end{itemize}
\end{axs}
\begin{thm}\label{4.5.6.1}
集合$X$上の有限加法族$\mathfrak{F}$が$\sigma$-加法族であるならそのときに限り、その有限加法族$\mathfrak{F}$がその集合$X$上の単調族である。
\end{thm}
\begin{proof}
集合$X$上の有限加法族$\mathfrak{F}$が$\sigma$-加法族であるなら、次のことを満たすので、
\begin{itemize}
\item
  その集合$\mathfrak{F}$の単調増加する任意の元の列$\left( E_{n} \right)_{n \in \mathbb{N}}$が$\bigcup_{n \in \mathbb{N}} E_{n}\in \mathfrak{F}$を満たす。
\item
  その集合$\mathfrak{F}$の単調減少する任意の元の列$\left( E_{n} \right)_{n \in \mathbb{N}}$が$\bigcap_{n \in \mathbb{N}} E_{n}\in \mathfrak{F}$を満たす。
\end{itemize}
その有限加法族$\mathfrak{F}$がその集合$X$上の単調族である。\par
逆に、その有限加法族$\mathfrak{F}$がその集合$X$上の単調族であるなら、その集合$\mathfrak{F}$の任意の元の列$\left( E_{n} \right)_{n \in \mathbb{N}}$に対し、元の列$\left\{ \bigcup_{i \in \varLambda_{n}} E_{i} \right\}_{n \in \mathbb{N}}$はその集合$\mathfrak{F}$の単調増加する元の列であるから、次のようになる。
\begin{align*}
\bigcup_{n \in \mathbb{N}} {\bigcup_{i \in \varLambda_{n}} E_{i}} = \bigcup_{n \in \mathbb{N}} E_{n}\in \mathfrak{F}
\end{align*}
よって、集合$X$上の有限加法族$\mathfrak{F}$が$\sigma$-加法族である。
\end{proof}
\begin{thm}\label{4.5.6.2}
集合$X$上の有限加法族$\mathfrak{F}$で定義されたJordan測度$m$が完全加法的であるならそのときに限り、その有限加法族$\mathfrak{F}$がその集合$X$上の単調族で次のことを満たす\footnote{証明がちょっと怪しいかも…。}。
\begin{itemize}
\item
  その集合$\mathfrak{F}$の単調増加する任意の元の列$\left( E_{n} \right)_{n \in \mathbb{N}}$が、$m\left( \bigcup_{n \in \mathbb{N}} E_{n} \right) = \infty$が成り立つなら、$\lim_{n \rightarrow \infty}{m\left( E_{n} \right)} = \infty$を満たす。
\item
  $m\left( E_{1} \right) < \infty$なるその集合$\mathfrak{F}$の単調減少する任意の元の列$\left( E_{n} \right)_{n \in \mathbb{N}}$が、$\bigcap_{n \in \mathbb{N}} E_{n} = \emptyset$が成り立つなら、$\lim_{n \rightarrow \infty}{m\left( E_{n} \right)} = 0$を満たす。
\end{itemize}
\end{thm}
\begin{proof}
集合$X$上の有限加法族$\mathfrak{F}$で定義されたJordan測度$m$が完全加法的であるとする。その集合$\mathfrak{F}$の単調増加する任意の元の列$\left( E_{n} \right)_{n \in \mathbb{N}}$が、$m\left( \bigcup_{n \in \mathbb{N}} E_{n} \right) = \infty$が成り立つなら、$E_{1} = F_{1}$かつ$F_{n + 1} = E_{n + 1} \setminus E_{n}$なる元の列$\left( F_{n} \right)_{n \in \mathbb{N}}$が与えられたとき、これらの項々は互いに素でその元の列$\left( F_{n} \right)_{n \in \mathbb{N}}$はその集合$\mathfrak{F}$の元の列であり、$\forall n \in \mathbb{N}$に対し、次のことを満たす。
\begin{align*}
E_{n} = \bigsqcup_{\nu \in \varLambda_{n}} F_{\nu},\ \ \bigcup_{n \in \mathbb{N}} E_{n} = \bigsqcup_{\nu \in \mathbb{N}} F_{\nu}
\end{align*}
したがって、次のようになる。
\begin{align*}
\lim_{n \rightarrow \infty}{m\left( E_{n} \right)} &= \lim_{n \rightarrow \infty}{m\left( \bigsqcup_{\nu \in \varLambda_{n}} F_{\nu} \right)} = \lim_{n \rightarrow \infty}{\sum_{\nu \in \varLambda_{n}} {m\left( F_{\nu} \right)}}\\
&= \sum_{\nu \in \mathbb{N}} {m\left( F_{\nu} \right)} = m\left( \bigsqcup_{\nu \in \mathbb{N}} F_{\nu} \right) = m\left( \bigcup_{n \in \mathbb{N}} E_{n} \right)
\end{align*}\par
一方で、$m\left( E_{1} \right) < \infty$なるその集合$\mathfrak{F}$の単調減少する任意の元の列$\left( E_{n} \right)_{n \in \mathbb{N}}$が、$\bigcap_{n \in \mathbb{N}} E_{n} = \emptyset$が成り立つなら、元の列$\left( E_{n} \setminus E_{n + 1} \right)_{n \in \mathbb{N}}$もその有限加法族$\mathfrak{F}$の元の列で、明らかに、$\forall n \in \mathbb{N}$に対し、$E_{n} = \bigsqcup_{\nu \in \mathbb{N} \setminus \varLambda_{n - 1}} F_{\nu}$が成り立つ。したがって、次のようになるので、
\begin{align*}
\sum_{n \in \mathbb{N}} {m\left( F_{n} \right)} = m\left( \bigsqcup_{n \in \mathbb{N}} F_{n} \right) = m\left( E_{1} \right) < \infty
\end{align*}
次のようになる。
\begin{align*}
\lim_{n \rightarrow \infty}{m\left( E_{n} \right)} &= \lim_{n \rightarrow \infty}{m\left( \bigsqcup_{\nu \in \mathbb{N} \setminus \varLambda_{n - 1}} F_{\nu} \right)} = \lim_{n \rightarrow \infty}{\sum_{\nu \in \mathbb{N} \setminus \varLambda_{n - 1}} {m\left( F_{\nu} \right)}}\\
&= \sum_{\nu \in \mathbb{N}} {m\left( F_{\nu} \right)} - \lim_{n \rightarrow \infty}{\sum_{\nu \in \varLambda_{n - 1}} {m\left( F_{\nu} \right)}} = m\left( E_{1} \right) - m\left( E_{1} \right) = 0
\end{align*}\par
逆に、その有限加法族$\mathfrak{F}$がその集合$X$上の単調族で次のことを満たすなら、
\begin{itemize}
\item
  その集合$\mathfrak{F}$の単調増加する任意の元の列$\left( E_{n} \right)_{n \in \mathbb{N}}$が、$m\left( \bigcup_{n \in \mathbb{N}} E_{n} \right) = \infty$が成り立つなら、$\lim_{n \rightarrow \infty}{m\left( E_{n} \right)} = \infty$を満たす。
\item
  $m\left( E_{1} \right) < \infty$なるその集合$\mathfrak{F}$の単調減少する任意の元の列$\left( E_{n} \right)_{n \in \mathbb{N}}$が、$\bigcap_{n \in \mathbb{N}} E_{n} = \emptyset$が成り立つなら、$\lim_{n \rightarrow \infty}{m\left( E_{n} \right)} = 0$を満たす。
\end{itemize}
その有限加法族$\mathfrak{F}$の元の列$\left( F_{n} \right)_{n \in \mathbb{N}}$のうち、互いに素で$\bigsqcup_{n \in \mathbb{N}} F_{n}$が成り立つとする。\par
$m\left( \bigsqcup_{n \in \mathbb{N}} F_{n} \right) < \infty$が成り立つなら、$\forall n \in \mathbb{N}$に対し、$E_{n} = \bigsqcup_{\nu \in \mathbb{N}} F_{\nu} \setminus \bigsqcup_{\nu \in \varLambda_{n}\mathbf{\ }} F_{\nu}$とおかれた元の列$\left( E_{n} \right)_{n \in \mathbb{N}}$は$m\left( E_{1} \right) < \infty$なるその集合$\mathfrak{F}$の単調減少する元の列であり$\bigcap_{n \in \mathbb{N}} E_{n} = \emptyset$を満たす。したがって、次のようになる。
\begin{align*}
m\left( \bigsqcup_{\nu \in \mathbb{N}} F_{\nu} \right) &= m\left( \bigsqcup_{\nu \in \mathbb{N}} F_{\nu} \right) - \sum_{\nu \in \varLambda_{n}} {m\left( F_{\nu} \right)} + \sum_{\nu \in \varLambda_{n}} {m\left( F_{\nu} \right)}\\
&= m\left( \bigsqcup_{\nu \in \mathbb{N}} F_{\nu} \right) - m\left( \bigsqcup_{\nu \in \varLambda_{n}\mathbf{\ }} F_{\nu} \right) + \sum_{\nu \in \varLambda_{n}} {m\left( F_{\nu} \right)}\\
&= m\left( \bigsqcup_{\nu \in \mathbb{N}} F_{\nu} \setminus \bigsqcup_{\nu \in \varLambda_{n}\mathbf{\ }} F_{\nu} \right) + \sum_{\nu \in \varLambda_{n}} {m\left( F_{\nu} \right)}\\
&= m\left( E_{n} \right) + \sum_{\nu \in \varLambda_{n}} {m\left( F_{\nu} \right)}
\end{align*}
したがって、$n \rightarrow \infty$とすれば、次式が成り立つ。
\begin{align*}
m\left( \bigsqcup_{\nu \in \mathbb{N}} F_{\nu} \right) = \sum_{\nu \in \mathbb{N}} {m\left( F_{\nu} \right)}
\end{align*}\par
$m\left( \bigsqcup_{n \in \mathbb{N}} F_{n} \right) = \infty$が成り立つなら、$\forall n \in \mathbb{N}$に対し、$E_{n} = \bigsqcup_{\nu \in \varLambda_{n}\mathbf{\ }} F_{\nu}$とおかれた元の列$\left( E_{n} \right)_{n \in \mathbb{N}}$はその集合$\mathfrak{F}$の単調増加する元の列$\left( E_{n} \right)_{n \in \mathbb{N}}$であり$m\left( \bigcup_{n \in \mathbb{N}} E_{n} \right) = \infty$を満たす。したがって、次のようになる。
\begin{align*}
m\left( \bigsqcup_{n \in \mathbb{N}} F_{n} \right) &= \infty = \lim_{n \rightarrow \infty}{m\left( E_{n} \right)} = \lim_{n \rightarrow \infty}{m\left( \bigsqcup_{\nu \in \varLambda_{n}\mathbf{\ }} F_{\nu} \right)}\\
&= \lim_{n \rightarrow \infty}{\sum_{\nu \in \varLambda_{n}} {m\left( F_{\nu} \right)}} = \sum_{n \in \mathbb{N}} {m\left( F_{n} \right)}
\end{align*}
\end{proof}
\begin{thm}\label{4.5.6.3}
集合$X$上の有限加法族$\mathfrak{F}$で定義された$\sigma$-有限なJordan測度$m$が完全加法的であるならそのときに限り、その有限加法族$\mathfrak{F}$がその集合$X$上の単調族で次のことを満たす。
\begin{itemize}
\item
  $X = \bigcup_{n \in \mathbb{N}} X_{n}$かつ$m\left( X_{n} \right) < \infty$なるその有限加法族$\mathfrak{F}$の単調増加する元の列$\left( X_{n} \right)_{n \in \mathbb{N}}$が存在して、$\forall F \in \mathfrak{F}$に対し、$m(F) = \infty$が成り立つなら、$\lim_{n \rightarrow \infty}{m\left( F \cap E_{n} \right)} = \infty$が成り立つ。
\item
  $m\left( E_{1} \right) < \infty$なるその集合$\mathfrak{F}$の単調減少する任意の元の列$\left( E_{n} \right)_{n \in \mathbb{N}}$が、$\bigcap_{n \in \mathbb{N}} E_{n} = \emptyset$が成り立つなら、$\lim_{n \rightarrow \infty}{m\left( E_{n} \right)} = 0$を満たす。
\end{itemize}
\end{thm}
\begin{proof}
集合$X$上の有限加法族$\mathfrak{F}$で定義された$\sigma$-有限なJordan測度$m$が完全加法的であるなら、定理\ref{4.5.3.22}より$X = \bigcup_{n \in \mathbb{N}} X_{n}$かつ$m\left( X_{n} \right) < \infty$なるその有限加法族$\mathfrak{F}$の単調増加する元の列$\left( X_{n} \right)_{n \in \mathbb{N}}$が存在して、$\forall F \in \mathfrak{F}$に対し、$m(F) = \infty$が成り立つなら、$F \subseteq X$より$m\left( \bigcup_{n \in \mathbb{N}} \left( F \cap X_{n} \right) \right) = m(X) = \infty$が成り立つので、定理\ref{4.5.6.2}より$\lim_{n \rightarrow \infty}{m\left( F \cap E_{n} \right)} = \infty$が成り立つ。したがって、次のことを満たす。
\begin{itemize}
\item
  $X = \bigcup_{n \in \mathbb{N}} X_{n}$かつ$m\left( X_{n} \right) < \infty$なるその有限加法族$\mathfrak{F}$の単調増加する元の列$\left( X_{n} \right)_{n \in \mathbb{N}}$が存在して、$\forall F \in \mathfrak{F}$に対し、$m(F) = \infty$が成り立つなら、$\lim_{n \rightarrow \infty}{m\left( F \cap E_{n} \right)} = \infty$が成り立つ。
\item
  $m\left( E_{1} \right) < \infty$なるその集合$\mathfrak{F}$の単調減少する任意の元の列$\left( E_{n} \right)_{n \in \mathbb{N}}$が、$\bigcap_{n \in \mathbb{N}} E_{n} = \emptyset$が成り立つなら、$\lim_{n \rightarrow \infty}{m\left( E_{n} \right)} = 0$を満たす。
\end{itemize}\par
逆に、これが成り立つとき、その有限加法族$\mathfrak{F}$の元の列$\left( F_{n} \right)_{n \in \mathbb{N}}$のうち、互いに素で$\bigsqcup_{n \in \mathbb{N}} F_{n}$が成り立つとする。$m\left( \bigsqcup_{n \in \mathbb{N}} F_{n} \right) < \infty$が成り立つなら、定理\ref{4.5.3.29}と同様にして示される。$m\left( \bigsqcup_{n \in \mathbb{N}} F_{n} \right) = \infty$が成り立つなら、このようなその集合$\mathfrak{F}$の元の列$\left( X_{n} \right)_{n \in \mathbb{N}}$が、$\forall n \in \mathbb{N}$に対し、$m\left( X_{n} \right) < \infty$を満たし、さらに、$\lim_{n \rightarrow \infty}X_{n} = X$が成り立つので、$\forall\varepsilon \in \mathbb{R}^{+}\exists\delta \in \mathbb{N}\forall n \in \mathbb{N}$に対し、$\delta \leq n$が成り立つなら、$\varepsilon < m\left( \bigsqcup_{\nu \in \mathbb{N}} F_{\nu} \cap X_{n} \right)$が成り立ち、したがって、次式が成り立つので、
\begin{align*}
\varepsilon < m\left( \bigsqcup_{\nu \in \mathbb{N}} F_{\nu} \cap X_{n} \right) \leq m\left( X_{n} \right) < \infty
\end{align*}
その元の列$\left( \bigsqcup_{\nu \in \mathbb{N}} F_{\nu} \cap X_{n} \right)_{n \in \mathbb{N}}$がその集合$\mathfrak{F}$の単調増加する元の列で、$m\left( \bigcup_{n \in \mathbb{N}} {\bigsqcup_{\nu \in \mathbb{N}} F_{\nu} \cap X_{n}} \right) = \infty$が成り立ち$\lim_{n \rightarrow \infty}{m\left( \bigsqcup_{\nu \in \mathbb{N}} F_{\nu} \cap X_{n} \right)} = \infty$を満たす。あとは定理\ref{4.5.3.29}と同様にして示される。
\end{proof}
\begin{thm}\label{4.5.6.4}
集合$X$上の有限加法族$\mathfrak{F}$で定義された$m(X) < \infty$なるJordan測度$m$が与えられたとき、その有限加法族$\mathfrak{F}$がその集合$X$上の単調族で次のことを満たすなら、そのJordan測度$m$は完全加法的である。
\begin{itemize}
\item
  その集合$\mathfrak{F}$の単調減少する任意の元の列$\left( E_{n} \right)_{n \in \mathbb{N}}$が、$\bigcap_{n \in \mathbb{N}} E_{n} = \emptyset$が成り立つなら、$\lim_{n \rightarrow \infty}{m\left( E_{n} \right)} = 0$を満たす。
\end{itemize}
\end{thm}
\begin{proof} 定理\ref{4.5.6.2}の証明により明らかである。
\end{proof}
%\hypertarget{ux751fux6210ux3055ux308cux308bux5358ux8abfux65cf}{%
\subsubsection{生成される単調族}%\label{ux751fux6210ux3055ux308cux308bux5358ux8abfux65cf}}
\begin{thm}\label{4.5.6.5}
集合$X$の部分集合系$\mathfrak{P}(X)$の任意の部分集合$\mathcal{I}$に対し、$\mathcal{I \subseteq}\mathfrak{M}$となるような単調族$\mathfrak{M}$全体の集合を$M\left( \mathcal{I} \right)$とおくと、順序集合$\left( M\left( \mathcal{I} \right), \subseteq \right)$において、最小元$\min{M\left( \mathcal{I} \right)}$が存在して$\min{M\left( \mathcal{I} \right)} = \bigcap_{} {M\left( \mathcal{I} \right)}$が成り立つ。
\end{thm}
\begin{proof}
集合$X$の部分集合系$\mathfrak{P}(X)$の任意の部分集合$\mathcal{I}$に対し、$\mathcal{I \subseteq}\mathfrak{M}$となるような単調族$\mathfrak{M}$全体の集合を$M\left( \mathcal{I} \right)$とおくと、$\mathfrak{P}(X) \in M\left( \mathcal{I} \right)$が成り立つので、$\mathcal{I \subseteq}\mathfrak{M}$となるような単調族$\mathfrak{M}$は明らかに存在する。あとは、積集合$\bigcap_{} {M\left( \mathcal{I} \right)}$を考えれば、$\mathcal{I \subseteq}\bigcap_{} {M\left( \mathcal{I} \right)}$が成り立つ。また、その集合$\bigcap_{} {M\left( \mathcal{I} \right)}$の単調増加する任意の元の列$\left( E_{n} \right)_{n \in \mathbb{N}}$について、これは、$\mathfrak{\forall M \in}M\left( \mathcal{I} \right)$に対し、その単調族$\mathfrak{M}$の元の列でもあるので、$\bigcup_{n \in \mathbb{N}} E_{n}\in \mathfrak{M}$が成り立つ。これにより、$\bigcup_{n \in \mathbb{N}} E_{n} \in \bigcap_{} {M\left( \mathcal{I} \right)}$が得られる。同様にして、その集合$\bigcap_{} {M\left( \mathcal{I} \right)}$の単調減少する任意の元の列$\left( E_{n} \right)_{n \in \mathbb{N}}$が$\bigcap_{n \in \mathbb{N}} E_{n} \in \bigcap_{} {M\left( \mathcal{I} \right)}$を満たすことが示される。以上より、$\bigcap_{} {M\left( \mathcal{I} \right)} \in M\left( \mathcal{I} \right)$が成り立つことになり、さらに、$\mathfrak{\forall M \in}M\left( \mathcal{I} \right)$に対し、$\bigcap_{} {M\left( \mathcal{I} \right)}\subseteq \mathfrak{M}$が成り立つので、この集合$\bigcap_{} {M\left( \mathcal{I} \right)}$が、順序集合$\left( M\left( \mathcal{I} \right), \subseteq \right)$において、最小元$\min{M\left( \mathcal{I} \right)}$となる。
\end{proof}
\begin{dfn}
この最小元$\min{M\left( \mathcal{I} \right)}$をその集合$\mathcal{I}$によって生成される単調族といい以下$\mathfrak{M}\left( \mathcal{I} \right)$と書く\footnote{なお、その集合$X$にはよらないことに注意されたい。}。
\end{dfn}
\begin{thm}\label{4.5.6.6}
集合$X$の部分集合系$\mathfrak{P}(X)$の任意の部分集合たち$\mathcal{I}$、$\mathcal{J}$に対し、$\mathcal{I \subseteq J}$が成り立つなら、$\mathfrak{M}\left( \mathcal{I} \right)\subseteq \mathfrak{M}\left( \mathcal{J} \right)$が成り立つ。
\end{thm}
\begin{proof}
集合$X$の部分集合系$\mathfrak{P}(X)$の任意の部分集合たち$\mathcal{I}$、$\mathcal{J}$に対し、$\mathcal{I \subseteq J}$が成り立つなら、$\mathcal{I \subseteq}\mathfrak{M}$、$\mathcal{J \subseteq}\mathfrak{M}$となるような$\sigma$-加法族$\mathfrak{M}$全体の集合をそれぞれ$M\left( \mathcal{I} \right)$、$M\left( \mathcal{J} \right)$とおくと、$M\left( \mathcal{I} \right) \subseteq M\left( \mathcal{J} \right)$が成り立つ。ここで、$\mathfrak{\forall M \in}M\left( \mathcal{I} \right)$に対し、$\min{M\left( \mathcal{I} \right)}\subseteq \mathfrak{M}$が成り立つので、$\mathfrak{\forall M \in}M\left( \mathcal{J} \right)$に対し、$\min{M\left( \mathcal{I} \right)}\subseteq \mathfrak{M}$が成り立つ。したがって、$\min{M\left( \mathcal{I} \right)} \subseteq \min{M\left( \mathcal{J} \right)}$が成り立ち、よって、$\mathfrak{M}\left( \mathcal{I} \right)\subseteq \mathfrak{M}\left( \mathcal{J} \right)$が成り立つ。
\end{proof}
\begin{thm}\label{4.5.6.7}
集合$X$上の有限加法族$\mathfrak{F}$によって生成される単調族$\mathfrak{M}\left( \mathfrak{F} \right)$もその集合$X$上の有限加法族である。
\end{thm}
\begin{proof}
集合$X$上の有限加法族$\mathfrak{F}$によって生成される単調族$\mathfrak{M}\left( \mathfrak{F} \right)$について、$\mathfrak{\emptyset \in F \subseteq M}\left( \mathfrak{F} \right)$が成り立つので、$\mathfrak{\emptyset \in M}\left( \mathfrak{F} \right)$が成り立つ。\par
$\mathfrak{M} =\left\{ A \in \mathfrak{P}(X) \middle| X \setminus A \in \mathfrak{M}\left( \mathfrak{F} \right) \right\}$とおかれると、$\forall A \in \mathfrak{F}$に対し、$X \setminus A \in \mathfrak{F \subseteq M}\left( \mathfrak{F} \right)$が成り立つので、$A \in \mathfrak{M}$が得られる。ゆえに、$\mathfrak{F \subseteq M}$が成り立つ。次に、その集合$\mathfrak{M}$が実は単調族であることを示そう。その集合$\mathfrak{M}$の単調増加する任意の元の列$\left( E_{n} \right)_{n \in \mathbb{N}}$に対し、$X \setminus E_{n}\in \mathfrak{M}\left( \mathfrak{F} \right)$が成り立つので、元の列$\left( X \setminus E_{n} \right)_{n \in \mathbb{N}}$はその集合$\mathfrak{M}\left( \mathfrak{F} \right)$の単調減少する元の列であり$\bigcap_{n \in \mathbb{N}} {X \setminus E_{n}}\in \mathfrak{M}\left( \mathfrak{F} \right)$が成り立つ。このとき、次式が成り立つので、
\begin{align*}
\bigcap_{n \in \mathbb{N}} {X \setminus E_{n}} = X \setminus \bigcup_{n \in \mathbb{N}} E_{n}\in \mathfrak{M}\left( \mathfrak{F} \right)
\end{align*}
$\bigcup_{n \in \mathbb{N}} E_{n}\in \mathfrak{M}$が得られる。同様にして、その集合$\mathfrak{M}$の単調減少する任意の元の列$\left( E_{n} \right)_{n \in \mathbb{N}}$が$\bigcap_{n \in \mathbb{N}} E_{n}\in \mathfrak{M}$を満たすことが示されるので、その集合$\mathfrak{M}$は単調族である。これにより、$\mathfrak{F \subseteq M}$となるような単調族$\mathfrak{M}$全体の集合を$M\left( \mathfrak{F} \right)$とおくと、$\mathfrak{M \in}M\left( \mathfrak{F} \right)$が得られ、したがって、次式が成り立つので、
\begin{align*}
\mathfrak{M}\left( \mathfrak{F} \right) = \min{M\left( \mathfrak{F} \right)}\subseteq \mathfrak{M}
\end{align*}
$A \in \mathfrak{M}\left( \mathfrak{F} \right)$が成り立つなら、$A \in \mathfrak{M}$が成り立ち、したがって、$X \setminus A \in \mathfrak{M}\left( \mathfrak{F} \right)$が成り立つ。\par
次に、$\mathfrak{N} = \left\{ A_{1}\in \mathfrak{P}(X) \middle| \forall A_{2}\in \mathfrak{F}\left[ A_{1} \cup A_{2}\in \mathfrak{M}\left( \mathfrak{F} \right) \right] \right\}$とおかれると、$\forall A_{1},A_{2}\in \mathfrak{F}$に対し、$A_{1} \cup A_{2}\in \mathfrak{F \subseteq M}\left( \mathfrak{F} \right)$が成り立つので、$A_{1}\in \mathfrak{N}$が得られる。ゆえに、$\mathfrak{F \subseteq N}$が成り立つ。次に、その集合$\mathfrak{N}$が実は単調族であることを示そう。その集合$\mathfrak{N}$の単調増加する任意の元の列$\left( E_{n} \right)_{n \in \mathbb{N}}$に対し、$\forall A \in \mathfrak{F}$に対し、$E_{n} \cup A \in \mathfrak{M}\left( \mathfrak{F} \right)$が成り立つので、元の列$\left( E_{n} \cup A \right)_{n \in \mathbb{N}}$はその集合$\mathfrak{M}\left( \mathfrak{F} \right)$の単調増加する元の列であり$\bigcup_{n \in \mathbb{N}} \left( E_{n} \cup A \right)\in \mathfrak{M}\left( \mathfrak{F} \right)$が成り立つ。このとき、次式が成り立つので、
\begin{align*}
\bigcup_{n \in \mathbb{N}} \left( E_{n} \cup A \right) = \bigcup_{n \in \mathbb{N}} E_{n} \cup A \in \mathfrak{M}\left( \mathfrak{F} \right)
\end{align*}
$\bigcup_{n \in \mathbb{N}} E_{n}\in \mathfrak{N}$が得られる。同様にして、その集合$\mathfrak{N}$の単調減少する任意の元の列$\left( E_{n} \right)_{n \in \mathbb{N}}$が$\bigcap_{n \in \mathbb{N}} E_{n}\in \mathfrak{N}$を満たすことが示されるので、その集合$\mathfrak{N}$は単調族である。これにより、$\mathfrak{N \in}M\left( \mathfrak{F} \right)$が得られ、したがって、次式が成り立つので、
\begin{align*}
\mathfrak{M}\left( \mathfrak{F} \right) = \min{M\left( \mathfrak{F} \right)}\subseteq \mathfrak{N}
\end{align*}
$A\in \mathfrak{M}\left( \mathfrak{F} \right)$が成り立つなら、$A\in \mathfrak{N}$が成り立ち、したがって、$\forall B\in \mathfrak{F}$に対し、$A \cup B\in \mathfrak{M}\left( \mathfrak{F} \right)$が成り立つ。\par
次に、$\mathfrak{O} = \left\{ A\in \mathfrak{P}(X) \middle| \forall B\in \mathfrak{M}\left( \mathfrak{F} \right)\left[ A \cup B\in \mathfrak{M}\left( \mathfrak{F} \right) \right] \right\}$とおかれると、$\forall A\in \mathfrak{F\forall}B\in \mathfrak{M}\left( \mathfrak{F} \right)$に対し、$\mathfrak{M}\left( \mathfrak{F} \right)\subseteq \mathfrak{N}$より$A \cup B\in \mathfrak{F \subseteq M}\left( \mathfrak{F} \right)$が成り立つので、$A\in \mathfrak{O}$が得られる。ゆえに、$\mathfrak{F \subseteq O}$が成り立つ。次に、上記の議論と同様にして、その集合$\mathfrak{O}$が単調族であることが示される。これにより、$\mathfrak{O \in}M\left( \mathfrak{F} \right)$が得られ、したがって、次式が成り立つので、
\begin{align*}
\mathfrak{M}\left( \mathfrak{F} \right) = \min{M\left( \mathfrak{F} \right)}\subseteq \mathfrak{O}
\end{align*}
$A\in \mathfrak{M}\left( \mathfrak{F} \right)$が成り立つなら、$A\in \mathfrak{O}$が成り立ち、したがって、$\forall B\in \mathfrak{M}\left( \mathfrak{F} \right)$に対し、$A \cup B \in \mathfrak{M}\left( \mathfrak{F} \right)$が成り立つ。\par
よって、集合$X$上の有限加法族$\mathfrak{F}$によって生成される単調族$\mathfrak{M}\left( \mathfrak{F} \right)$もその集合$X$上の有限加法族である。
\end{proof}
\begin{thm}[単調族定理]\label{4.5.6.8}
集合$X$上の有限加法族$\mathfrak{F}$によって生成される$\sigma$-加法族$\varSigma\left( \mathfrak{F} \right)$と単調族$\mathfrak{M}\left( \mathfrak{F} \right)$について、$\varSigma\left( \mathfrak{F} \right) = \mathfrak{M}\left( \mathfrak{F} \right)$が成り立つ。この定理を単調族定理という。
\end{thm}
\begin{proof}
集合$X$上の有限加法族$\mathfrak{F}$によって生成される$\sigma$-加法族$\varSigma\left( \mathfrak{F} \right)$と単調族$\mathfrak{M}\left( \mathfrak{F} \right)$について、その単調族$\mathfrak{M}\left( \mathfrak{F} \right)$は有限加法族であるので、定理\ref{4.5.6.1}よりその集合$\mathfrak{M}\left( \mathfrak{F} \right)$は$\sigma$-加法族でもある。$\mathfrak{F \subseteq M}\left( \mathfrak{F} \right)$よりしたがって、$\varSigma\left( \mathfrak{F} \right) \subseteq \varSigma\left( \mathfrak{M}\left( \mathfrak{F} \right) \right) = \mathfrak{M}\left( \mathfrak{F} \right)$が成り立つ。一方で、その$\sigma$-加法族$\varSigma\left( \mathfrak{F} \right)$は有限加法族であるので、定理\ref{4.5.6.1}よりその集合$\varSigma\left( \mathfrak{F} \right)$は単調族でもある。$\mathfrak{F \subseteq M}\left( \mathfrak{F} \right)$よりしたがって、$\mathfrak{M}\left( \mathfrak{F} \right)\subseteq \mathfrak{M}\left( \varSigma\left( \mathfrak{F} \right) \right) = \varSigma\left( \mathfrak{F} \right)$が成り立つ。
\end{proof}
%\hypertarget{dynkinux65cf}{%
\subsubsection{Dynkin族}%\label{dynkinux65cf}}
\begin{axs}[Dynkin族の公理]
空集合でない集合$X$の部分集合系$\mathfrak{D}$が次のことをいづれも満たすとき、その集合$\mathfrak{D}$をその集合$X$上のDynkin族、$\lambda$-系という。
\begin{itemize}
\item
  $\mathfrak{\emptyset \in D}$が成り立つ。
\item
  $\forall A,B\in \mathfrak{D}$に対し、$A \subseteq B$が成り立つなら、$B \setminus A\in \mathfrak{D}$が成り立つ。
\item
  その集合$\mathfrak{D}$の単調増加する任意の元の列$\left( E_{n} \right)_{n \in \mathbb{N}}$が$\bigcup_{n \in \mathbb{N}} E_{n} \in \mathfrak{D}$を満たす。
\end{itemize}
\end{axs}
\begin{thm}\label{4.5.6.9} 集合$X$上のDynkin族はその集合$X$上の単調族である。
\end{thm}
\begin{proof}
集合$X$上のDynkin族$\mathfrak{D}$が与えられたとき、定義より$\mathfrak{\emptyset \in D}$が成り立つので、その集合$\mathfrak{D}$は空集合ではない。定義から明らかにその集合$\mathfrak{D}$の単調増加する任意の元の列$\left( E_{n} \right)_{n \in \mathbb{N}}$が$\bigcup_{n \in \mathbb{N}} E_{n}\in \mathfrak{D}$を満たす。その集合$\mathfrak{D}$の単調減少する任意の元の列$\left( E_{n} \right)_{n \in \mathbb{N}}$について、$\forall n \in \mathbb{N}$に対し、$E_{n} \subseteq E_{1}$が成り立つので、$E_{1} \setminus E_{n}\in \mathfrak{D}$が成り立つ。これにより、その集合$\mathfrak{D}$の単調増加する元の列$\left( E_{1} \setminus E_{n} \right)_{n \in \mathbb{N}}$が得られるので、次のようになる。
\begin{align*}
\bigcup_{n \in \mathbb{N}} \left( E_{1} \setminus E_{n} \right) = E_{1} \setminus \bigcap_{n \in \mathbb{N}} E_{n}\in \mathfrak{D}
\end{align*}
もちろん、$E_{1} \setminus \bigcap_{n \in \mathbb{N}} E_{n} \subseteq E_{1}$が成り立つので、次のようになる。
\begin{align*}
E_{1} \setminus \left( E_{1} \setminus \bigcap_{n \in \mathbb{N}} E_{n} \right) = E_{1} \setminus E_{1} \cup \left( E_{1} \cap \bigcap_{n \in \mathbb{N}} E_{n} \right) = \emptyset \cup \bigcap_{n \in \mathbb{N}} E_{n} = \bigcap_{n \in \mathbb{N}} E_{n}\in \mathfrak{D}
\end{align*}
以上より、集合$X$上のDynkin族はその集合$X$上の単調族であることが示された。
\end{proof}
\begin{thm}\label{4.5.6.10}
集合$X$上の$\sigma$-加法族はその集合$X$上のDynkin族である。
\end{thm}
\begin{proof}
集合$X$上の$\sigma$-加法族$\varSigma$が与えられたとき、定義より明らかに$\mathfrak{\emptyset \in D}$が成り立つ。定理\ref{4.5.2.3}より集合$X$上の$\sigma$-加法族$\varSigma$はその集合$X$上の有限加法族であるので、定理\ref{4.5.2.1}より、$\forall A,B\in \mathfrak{D}$に対し、$A \subseteq B$が成り立つなら、$B \setminus A \in \varSigma$が成り立つ。定義から明らかにその集合$\varSigma$の単調増加する任意の元の列$\left( E_{n} \right)_{n \in \mathbb{N}}$が$\bigcup_{n \in \mathbb{N}} E_{n} \in \varSigma$を満たす。以上より、集合$X$上の$\sigma$-加法族はその集合$X$上のDynkin族であることが示された。
\end{proof}
%\hypertarget{ux751fux6210ux3055ux308cux308bdynkinux65cf}{%
\subsubsection{生成されるDynkin族}%\label{ux751fux6210ux3055ux308cux308bdynkinux65cf}}
\begin{thm}\label{4.5.6.11}
集合$X$の部分集合系$\mathfrak{P}(X)$の任意の部分集合$\mathcal{I}$に対し、$\mathcal{I \subseteq}\mathfrak{D}$となるようなDynkin族$\mathfrak{D}$全体の集合を$\varDelta\left( \mathcal{I} \right)$とおくと、順序集合$\left( \varDelta\left( \mathcal{I} \right), \subseteq \right)$において、最小元$\min{\varDelta\left( \mathcal{I} \right)}$が存在して$\min{\varDelta\left( \mathcal{I} \right)} = \bigcap_{} {\varDelta\left( \mathcal{I} \right)}$が成り立つ。
\end{thm}
\begin{proof}
集合$X$の部分集合系$\mathfrak{P}(X)$の任意の部分集合$\mathcal{I}$に対し、$\mathcal{I \subseteq}\mathfrak{D}$となるようなDynkin族$\mathfrak{D}$全体の集合を$\varDelta\left( \mathcal{I} \right)$とおくと、$\mathfrak{P}(X) \in \varDelta\left( \mathcal{I} \right)$が成り立つので、$\mathcal{I \subseteq}\mathfrak{D}$となるようなDynkin族は明らかに存在する。あとは、積集合$\bigcap_{} {\varDelta\left( \mathcal{I} \right)}$を考えれば、$\mathcal{I \subseteq}\bigcap_{} {\varDelta\left( \mathcal{I} \right)}$が成り立つ。また、$\mathfrak{\forall D \in}\varDelta\left( \mathcal{I} \right)$に対し、$\mathfrak{\emptyset \in D}$が成り立つので、$\emptyset \in \bigcap_{} {\varDelta\left( \mathcal{I} \right)}$が成り立つ。$\forall A,B \in \bigcap_{} {\varDelta\left( \mathcal{I} \right)}$に対し、$A \subseteq B$が成り立つなら、$\mathfrak{\forall D \in}\varDelta\left( \mathcal{I} \right)$に対し、$A,B\in \mathfrak{D}$が成り立ち、したがって、$B \setminus A\in \mathfrak{D}$が成り立つことにより、$B \setminus A \in \bigcap_{} {\varDelta\left( \mathcal{I} \right)}$が成り立つ。その集合$\bigcap_{} {\varDelta\left( \mathcal{I} \right)}$の単調増加する任意の元の列$\left( E_{n} \right)_{n \in \mathbb{N}}$が与えられたとき、$\mathfrak{\forall D \in}\varDelta\left( \mathcal{I} \right)$に対し、これ$\left( E_{n} \right)_{n \in \mathbb{N}}$はその集合$\mathfrak{D}$の単調増加する任意の元の列であり、したがって、$\bigcup_{n \in \mathbb{N}} E_{n}\in \mathfrak{D}$が成り立つことにより、$\bigcup_{n \in \mathbb{N}} E_{n} \in \bigcap_{} {\varDelta\left( \mathcal{I} \right)}$が成り立つ。これにより、この集合$\bigcap_{} {\varDelta\left( \mathcal{I} \right)}$が、順序集合$\left( \varDelta\left( \mathcal{I} \right), \subseteq \right)$において、最小元$\min{\varDelta\left( \mathcal{I} \right)}$となる。
\end{proof}
\begin{dfn}
この最小元$\min{\varDelta\left( \mathcal{I} \right)}$をその集合$\mathcal{I}$によって生成されるその集合$X$上のDynkin族といい以下$\mathfrak{D}\left( \mathcal{I} \right)$と書く。
\end{dfn}
\begin{thm}\label{4.5.6.12}
集合$X$の部分集合系$\mathfrak{P}(X)$の任意の部分集合たち$\mathcal{I}$、$\mathcal{J}$に対し、$\mathcal{I \subseteq J}$が成り立つなら、$\mathfrak{D}\left( \mathcal{I} \right) \subseteq \mathfrak{D}\left( \mathcal{J} \right)$が成り立つ。
\end{thm}
\begin{proof}
集合$X$の部分集合系$\mathfrak{P}(X)$の任意の部分集合たち$\mathcal{I}$、$\mathcal{J}$に対し、$\mathcal{I \subseteq J}$が成り立つなら、$\mathcal{I \subseteq}\mathfrak{D}$、$\mathcal{J \subseteq}\mathfrak{D}$となるようなDynkin族$\mathfrak{D}$全体の集合をそれぞれ$\varDelta\left( \mathcal{I} \right)$、$\varDelta\left( \mathcal{J} \right)$とおくと、$\varDelta\left( \mathcal{I} \right) \subseteq \varDelta\left( \mathcal{J} \right)$が成り立つ。ここで、$\mathfrak{\forall D \in}\varDelta\left( \mathcal{I} \right)$に対し、$\min{\varDelta\left( \mathcal{I} \right)}\subseteq \mathfrak{D}$が成り立つので、$\mathfrak{\forall D \in}\varDelta\left( \mathcal{J} \right)$に対し、$\min{\varDelta\left( \mathcal{I} \right)}\subseteq \mathfrak{D}$が成り立つ。したがって、$\min{\varDelta\left( \mathcal{I} \right)} \subseteq \min{\varDelta\left( \mathcal{J} \right)}$が成り立ち、よって、$\mathfrak{D}\left( \mathcal{I} \right)\subseteq \mathfrak{D}\left( \mathcal{J} \right)$が成り立つ。
\end{proof}
%\hypertarget{ux76f8ux5bfedynkinux65cf}{%
\subsubsection{相対Dynkin族}%\label{ux76f8ux5bfedynkinux65cf}}
\begin{thm}\label{4.5.6.13}
集合$X$上のDynkin族$\mathfrak{D}$が与えられたとき、$\forall A \in \mathfrak{P}(X)$に対し、次式のような集合$\mathfrak{D}_{A}$もDynkin族である。
\begin{align*}
\mathfrak{D}_{A} = \left\{ D \in \mathfrak{P}(X) \middle| A \cap D \in \mathfrak{D} \right\}
\end{align*}
\end{thm}
\begin{dfn}
このようにして定義された集合$\mathfrak{D}_{A}$をそのDynkin族$\mathfrak{D}$からその集合$A$によって誘導されるその集合$A$上の相対Dynkin族という。
\end{dfn}
\begin{proof}
集合$X$上のDynkin族$\mathfrak{D}$が与えられたとき、$\forall A \in \mathfrak{P}(X)$に対し、次式のような集合$\mathfrak{D}_{A}$について、
\begin{align*}
\mathfrak{D}_{A} = \left\{ D \in \mathfrak{P}(X) \middle| A \cap D \in \mathfrak{D} \right\}
\end{align*}
もちろん、$\emptyset \in \mathfrak{D}_{A}$が成り立つ。$\forall A',B' \in \mathfrak{D}_{A}$に対し、$A' \subseteq B'$が成り立つなら、$A \cap A',A \cap B' \in \mathfrak{D}_{A}$かつ$A \cap A' \subseteq A \cap B'$が成り立つことに注意すれば、次のようになるので、
\begin{align*}
A \cap \left( B' \setminus A' \right) = \left( A \cap B' \right) \setminus \left( A \cap A' \right) \in \mathfrak{D}_{A}
\end{align*}
$B' \setminus A' \in \mathfrak{D}_{A}$が成り立つ。その集合$\mathfrak{D}_{A}$の単調増加する任意の元の列$\left( E_{n} \right)_{n \in \mathbb{N}}$が与えられたとき、$\forall n \in \mathbb{N}$に対し、$A \cap E_{n}\in \mathfrak{D}$が成り立つので、その集合$\mathfrak{D}$の単調増加する元の列$\left( A \cap E_{n} \right)_{n \in \mathbb{N}}$が得られる。Dynkin族の定義より次のようになるので、
\begin{align*}
\bigcup_{n \in \mathbb{N}} \left( A \cap E_{n} \right) = A \cap \bigcup_{n \in \mathbb{N}} E_{n}\in \mathfrak{D}
\end{align*}
$\bigcup_{n \in \mathbb{N}} E_{n} \in \mathfrak{D}_{A}$が成り立つ。以上より、その集合$\mathfrak{D}_{A}$はその集合$X$上のDynkin族であることが示された。
\end{proof}
%\hypertarget{ux4e57ux6cd5ux65cf}{%
\subsubsection{乗法族}%\label{ux4e57ux6cd5ux65cf}}
\begin{axs}[乗法族の公理]
空集合でない集合$X$の部分集合系$\mathfrak{C}$が次のことをいづれも満たすとき、その集合$\mathfrak{C}$をその集合$X$上の乗法族、$\pi$-系という。
\begin{itemize}
\item
  $\forall A,B \in \mathfrak{C}$に対し、$A \cap B \in \mathfrak{C}$が成り立つ。
\end{itemize}
\end{axs}
\begin{thm}\label{4.5.6.14}
集合$X$上の$\sigma$-加法族はその集合$X$上の乗法族である。
\end{thm}
\begin{proof} 定理\ref{4.5.2.1}、定理\ref{4.5.2.3}より明らかである。
\end{proof}
%\hypertarget{ux751fux6210ux3055ux308cux308bux4e57ux6cd5ux65cf}{%
\subsubsection{生成される乗法族}%\label{ux751fux6210ux3055ux308cux308bux4e57ux6cd5ux65cf}}
\begin{thm}\label{4.5.6.15}
集合$X$の部分集合系$\mathfrak{P}(X)$の任意の部分集合$\mathcal{I}$に対し、$\mathcal{I \subseteq}\mathfrak{C}$となるような乗法族$\mathfrak{C}$全体の集合を$\varGamma\left( \mathcal{I} \right)$とおくと、順序集合$\left( \varGamma\left( \mathcal{I} \right), \subseteq \right)$において、最小元$\min{\varGamma\left( \mathcal{I} \right)}$が存在して$\min{\varGamma\left( \mathcal{I} \right)} = \bigcap_{} {\varGamma\left( \mathcal{I} \right)}$が成り立つ。
\end{thm}
\begin{proof}
集合$X$の部分集合系$\mathfrak{P}(X)$の任意の部分集合$\mathcal{I}$に対し、$\mathcal{I \subseteq}\mathfrak{C}$となるような乗法族$\mathfrak{C}$全体の集合を$\varGamma\left( \mathcal{I} \right)$とおくと、$\mathfrak{P}(X) \in \varGamma\left( \mathcal{I} \right)$が成り立つので、$\mathcal{I \subseteq}\mathfrak{C}$となるような乗法族は明らかに存在する。あとは、積集合$\bigcap_{} {\varGamma\left( \mathcal{I} \right)}$を考えれば、$\mathcal{I \subseteq}\bigcap_{} {\varGamma\left( \mathcal{I} \right)}$が成り立つ。また、$\forall A,B \in \bigcap_{} {\varGamma\left( \mathcal{I} \right)}$に対し、$\mathfrak{\forall C \in}\varGamma\left( \mathcal{I} \right)$に対し、$A,B\in \mathfrak{C}$が成り立ち、したがって、$A \cap B\in \mathfrak{C}$が成り立つことにより、$A \cap B \in \bigcap_{} {\varGamma\left( \mathcal{I} \right)}$が成り立つので、この集合$\bigcap_{} {\varGamma\left( \mathcal{I} \right)}$が、順序集合$\left( \varGamma\left( \mathcal{I} \right), \subseteq \right)$において、最小元$\min{\varGamma\left( \mathcal{I} \right)}$となる。
\end{proof}
\begin{dfn}
この最小元$\min{\varGamma\left( \mathcal{I} \right)}$をその集合$\mathcal{I}$によって生成されるその集合$X$上の乗法族といい以下$\mathfrak{C}\left( \mathcal{I} \right)$と書く。
\end{dfn}
\begin{thm}\label{4.5.6.16}
集合$X$の部分集合系$\mathfrak{P}(X)$の任意の部分集合たち$\mathcal{I}$、$\mathcal{J}$に対し、$\mathcal{I \subseteq J}$が成り立つなら、$\mathfrak{C}\left( \mathcal{I} \right) \subseteq \mathfrak{C}\left( \mathcal{J} \right)$が成り立つ。
\end{thm}
\begin{proof}
集合$X$の部分集合系$\mathfrak{P}(X)$の任意の部分集合たち$\mathcal{I}$、$\mathcal{J}$に対し、$\mathcal{I \subseteq J}$が成り立つなら、$\mathcal{I \subseteq}\mathfrak{C}$、$\mathcal{J}\subseteq \mathfrak{C}$となるような乗法族$\mathfrak{p}$全体の集合をそれぞれ$\varGamma\left( \mathcal{I} \right)$、$\varGamma\left( \mathcal{J} \right)$とおくと、$\varGamma\left( \mathcal{I} \right) \subseteq \varGamma\left( \mathcal{J} \right)$が成り立つ。ここで、$\mathfrak{\forall C \in}\varGamma\left( \mathcal{I} \right)$に対し、$\min{\varGamma\left( \mathcal{I} \right)}\subseteq \mathfrak{C}$が成り立つので、$\mathfrak{\forall C \in}\varGamma\left( \mathcal{J} \right)$に対し、$\min{\varGamma\left( \mathcal{I} \right)}\subseteq \mathfrak{C}$が成り立つ。したがって、$\min{\varGamma\left( \mathcal{I} \right)} \subseteq \min{\varGamma\left( \mathcal{J} \right)}$が成り立ち、よって、$\mathfrak{p}\left( \mathcal{I} \right)\subseteq \mathfrak{p}\left( \mathcal{J} \right)$が成り立つ。
\end{proof}
%\hypertarget{dynkinux65cfux5b9aux7406}{%
\subsubsection{Dynkin族定理}%\label{dynkinux65cfux5b9aux7406}}
\begin{thm}\label{4.5.6.17}
集合$X$上の乗法族$\mathfrak{C}$が与えられたとき、$\forall A \in \mathfrak{C}$に対し、$\mathfrak{D}\left( \mathfrak{C} \right) \subseteq {\mathfrak{D}\left( \mathfrak{C} \right)}_{A}$が成り立つ。
\end{thm}
\begin{proof}
集合$X$上の乗法族$\mathfrak{C}$が与えられたとき、$\forall A\in \mathfrak{C}$に対し、その集合$A$によって誘導される相対Dynkin族${\mathfrak{D}\left( \mathfrak{C} \right)}_{A}$について、$\forall P\in \mathfrak{C}$に対し、$A \cap P\in \mathfrak{C \subseteq D}\left( \mathfrak{C} \right)$が成り立つので、$P \in {\mathfrak{D}\left( \mathfrak{C} \right)}_{A}$が得られる。これにより、$\forall A\in \mathfrak{C}$に対し、$\mathfrak{C}\subseteq {\mathfrak{D}\left( \mathfrak{C} \right)}_{A}$が成り立つ。あとは、定理\ref{4.5.6.12}よりその集合${\mathfrak{D}\left( \mathfrak{C} \right)}_{A}$がDynkin族であることに注意すれば、$\mathfrak{D}\left( \mathfrak{C} \right)\subseteq \mathfrak{D}\left( {\mathfrak{D}\left( \mathfrak{C} \right)}_{A} \right) = {\mathfrak{D}\left( \mathfrak{C} \right)}_{A}$が成り立つ。
\end{proof}
\begin{thm}[Dynkin族の定理]\label{4.5.6.18}
集合$X$上の乗法族$\mathfrak{C}$が与えられたとき、その集合$\mathfrak{C}$から生成されるDynkin族$\mathfrak{D}\left( \mathfrak{C} \right)$は乗法族となる。この定理をDynkin族の定理、$\pi$-$\lambda$の定理などという。
\end{thm}
\begin{proof}
集合$X$上の乗法族$\mathfrak{C}$が与えられたとき、$\emptyset \subset \mathfrak{C} \subseteq \mathfrak{D}\left( \mathfrak{C} \right)$が成り立つので、その集合$\mathfrak{D}\left( \mathfrak{C} \right)$は空集合でない。$\forall A\in \mathfrak{C\forall}D \in \mathfrak{D}\left( \mathfrak{C} \right)$に対し、定理\ref{4.5.6.13}より$\mathfrak{D}\left( \mathfrak{C} \right) \subseteq {\mathfrak{D}\left( \mathfrak{C} \right)}_{A}$が成り立つので、$D \in {\mathfrak{D}\left( \mathfrak{C} \right)}_{A}$が成り立つ、即ち、$A \cap D \in \mathfrak{D}\left( \mathfrak{C} \right)$が成り立つ。これは$A \in {\mathfrak{D}\left( \mathfrak{C} \right)}_{D}$が成り立つことと同値であるから、$\forall D \in \mathfrak{D}\left( \mathfrak{C} \right)$に対し、$\mathfrak{C}\subseteq {\mathfrak{D}\left( \mathfrak{C} \right)}_{D}$が成り立つ。\par
$\forall A,B\in \mathfrak{D}\left( \mathfrak{C} \right)$に対し、上記の議論により$\mathfrak{C}\subseteq {\mathfrak{D}\left( \mathfrak{C} \right)}_{A}$が成り立つので、定理\ref{4.5.6.12}より$\mathfrak{D}\left( \mathfrak{C} \right)\subseteq \mathfrak{D}\left( {\mathfrak{D}\left( \mathfrak{C} \right)}_{A} \right)$が成り立つ。そこで、その集合${\mathfrak{D}\left( \mathfrak{C} \right)}_{A}$はDynkin族なので、$\mathfrak{D}\left( {\mathfrak{D}\left( \mathfrak{C} \right)}_{A} \right) = {\mathfrak{D}\left( \mathfrak{C} \right)}_{A}$が成り立つことになり、したがって、$\mathfrak{D}\left( \mathfrak{C} \right) \subseteq {\mathfrak{D}\left( \mathfrak{C} \right)}_{A}$が得られる。これにより、$B \in {\mathfrak{D}\left( \mathfrak{C} \right)}_{A}$が成り立つことになり、したがって、$A \cap B\in \mathfrak{D}\left( \mathfrak{C} \right)$が成り立つ。よって、その集合$\mathfrak{C}$から生成されるDynkin族$\mathfrak{D}\left( \mathfrak{C} \right)$は乗法族となる。
\end{proof}
\begin{thm}\label{4.5.6.19}
集合$X$上の乗法族$\mathfrak{C}$から生成されるDynkin族$\mathfrak{D}\left( \mathfrak{C} \right)$が$X \in \mathfrak{D}\left( \mathfrak{C} \right)$を満たすなら、$\mathfrak{D}\left( \mathfrak{C} \right) = \varSigma\left( \mathfrak{C} \right)$が成り立つ。
\end{thm}
\begin{proof}
集合$X$上の乗法族$\mathfrak{C}$から生成されるDynkin族$\mathfrak{D}\left( \mathfrak{C} \right)$が$X \in \mathfrak{D}\left( \mathfrak{C} \right)$を満たすとする。$\mathfrak{C}\subseteq \varSigma\left( \mathfrak{C} \right)$が成り立つかつ、定理\ref{4.5.6.10}よりその$\sigma$-加法族$\varSigma\left( \mathfrak{C} \right)$はその集合$X$上のDynkin族でもあるので、$\mathfrak{C} \subseteq \mathfrak{D}$となるようなDynkin族$\mathfrak{D}$全体の集合を$\varDelta\left( \mathfrak{C} \right)$とおくと、$\varSigma\left( \mathfrak{C} \right) \in \varDelta\left( \mathfrak{C} \right)$が得られ、したがって、$\mathfrak{D}\left( \mathfrak{C} \right) = \min{\varDelta\left( \mathfrak{C} \right)} \subseteq \varSigma\left( \mathfrak{C} \right)$が成り立つ。\par
一方で、$\forall A,B\in \mathfrak{D}\left( \mathfrak{C} \right)$に対し、$X \in \mathfrak{D}\left( \mathfrak{C} \right)$かつ$\mathfrak{D}\left( \mathfrak{C} \right)\subset \mathfrak{P}(X)$が成り立つので、$A \subseteq X$かつ$B \subseteq X$が得られる。Dynkin族の定義より$X \setminus A,X \setminus B\in \mathfrak{D}\left( \mathfrak{C} \right)$が得られ、定理\ref{4.5.6.14}よりその集合$\mathfrak{D}\left( \mathfrak{C} \right)$は乗法族をなすので、次のようになる。
\begin{align*}
X \setminus A \cap X \setminus B = X \setminus (A \cup B)\in \mathfrak{D}\left( \mathfrak{C} \right)
\end{align*}
そこで、$A \cup B \subseteq X$が成り立つので、Dynkin族の定義より$X \setminus \left( X \setminus (A \cup B) \right) = A \cup B\in \mathfrak{D}\left( \mathfrak{C} \right)$が得られる。そこで、その集合$\mathfrak{D}\left( \mathfrak{C} \right)$の元の列$\left( A_{n} \right)_{n \in \mathbb{N}}$が与えられたとき、$\forall n \in \mathbb{N}$に対し、上記の議論によりその元の列$\left( \bigcup_{i \in \varLambda_{n}} A_{i} \right)_{n \in \mathbb{N}}$もその集合$\mathfrak{D}\left( \mathfrak{C} \right)$の元の列でもあり、しかも、単調増加しているので、Dynkin族の定義より$\bigcup_{n \in \mathbb{N}} A_{n}\in \mathfrak{D}\left( \mathfrak{C} \right)$が成り立つ。\par
以上の議論により、次のことが成り立つので、
\begin{itemize}
\item
  $\mathfrak{\emptyset \in D}\left( \mathfrak{C} \right)$が成り立つ。
\item
  $A \in \mathfrak{D}\left( \mathfrak{C} \right)$が成り立つなら、$X \setminus A \in \mathfrak{D}\left( \mathfrak{C} \right)$も成り立つ。
\item
  その集合$\mathfrak{D}\left( \mathfrak{C} \right)$の元の列$\left( A_{n} \right)_{n \in \mathbb{N}}$が与えられたなら、$\bigcup_{n \in \mathbb{N}} A_{n}\in \mathfrak{D}\left( \mathfrak{C} \right)$が成り立つ。
\end{itemize}
その集合$\mathfrak{D}\left( \mathfrak{C} \right)$は$\sigma$-加法族でもある。$\mathfrak{C}\subseteq \varSigma$となるような$\sigma$-加法族$\varSigma$全体の集合を$\mathfrak{S}\left( \mathfrak{C} \right)$とおくと、$\varSigma\left( \mathfrak{C} \right)\in \mathfrak{S}\left( \mathfrak{C} \right)$が得られ、したがって、$\varSigma\left( \mathfrak{C} \right) = \min{\mathfrak{S}\left( \mathfrak{C} \right)}\subseteq \mathfrak{D}\left( \mathfrak{C} \right)$が成り立つ。以上の議論により、$\mathfrak{D}\left( \mathfrak{C} \right) = \varSigma\left( \mathfrak{C} \right)$が成り立つ。
\end{proof}
\begin{thebibliography}{50}
\bibitem{1}
  伊藤清三, ルベーグ積分入門, 裳華房, 1963. 新装第1版2刷 p53-61 ISBN978-4-7853-1318-0
\bibitem{2}
  Mathpedia. "測度と積分". Mathpedia. \url{https://math.jp/wiki/%E6%B8%AC%E5%BA%A6%E3%81%A8%E7%A9%8D%E5%88%86} (2021-7-12 9:20 閲覧)
\bibitem{3}
  岩田耕一郎, ルベーグ積分, 森北出版, 2015. 第1版第2刷 p119-122 ISBN978-4-627-05431-8
\bibitem{4}
  高信敏, 確率論, 共立出版, 2015. 初版1刷 p236-238 ISBN978-4-320-11159-2
\bibitem{5}
  数学の風景. "ディンキン族定理(π-λ定理)とその証明". 数学の風景. \url{https://mathlandscape.com/dynkin-system/} (2022-3-14 18:41 閲覧)
\end{thebibliography}
\end{document}
