\documentclass[10pt,a4paper]{jsarticle}
\usepackage{amsmath,amsfonts,amssymb,array,comment,mathtools,url,docmute}
\usepackage{longtable,booktabs,dcolumn,tabularx,mathtools,multirow,colortbl,xcolor}
\usepackage[dvipdfmx]{graphics}
\usepackage{bmpsize}
\usepackage{amsthm}
\usepackage{enumitem}
\setlistdepth{20}
\renewlist{itemize}{itemize}{20}
\setlist[itemize]{label=•}
\renewlist{enumerate}{enumerate}{20}
\setlist[enumerate]{label=\arabic*.}
\setcounter{MaxMatrixCols}{20}
\setcounter{tocdepth}{3}
\newcommand{\rotin}{\text{\rotatebox[origin=c]{90}{$\in $}}}
\newcommand{\amap}[6]{\text{\raisebox{-0.7cm}{\begin{tikzpicture} 
  \node (a) at (0, 1) {$\textstyle{#2}$};
  \node (b) at (#6, 1) {$\textstyle{#3}$};
  \node (c) at (0, 0) {$\textstyle{#4}$};
  \node (d) at (#6, 0) {$\textstyle{#5}$};
  \node (x) at (0, 0.5) {$\rotin $};
  \node (x) at (#6, 0.5) {$\rotin $};
  \draw[->] (a) to node[xshift=0pt, yshift=7pt] {$\textstyle{\scriptstyle{#1}}$} (b);
  \draw[|->] (c) to node[xshift=0pt, yshift=7pt] {$\textstyle{\scriptstyle{#1}}$} (d);
\end{tikzpicture}}}}
\newcommand{\twomaps}[9]{\text{\raisebox{-0.7cm}{\begin{tikzpicture} 
  \node (a) at (0, 1) {$\textstyle{#3}$};
  \node (b) at (#9, 1) {$\textstyle{#4}$};
  \node (c) at (#9+#9, 1) {$\textstyle{#5}$};
  \node (d) at (0, 0) {$\textstyle{#6}$};
  \node (e) at (#9, 0) {$\textstyle{#7}$};
  \node (f) at (#9+#9, 0) {$\textstyle{#8}$};
  \node (x) at (0, 0.5) {$\rotin $};
  \node (x) at (#9, 0.5) {$\rotin $};
  \node (x) at (#9+#9, 0.5) {$\rotin $};
  \draw[->] (a) to node[xshift=0pt, yshift=7pt] {$\textstyle{\scriptstyle{#1}}$} (b);
  \draw[|->] (d) to node[xshift=0pt, yshift=7pt] {$\textstyle{\scriptstyle{#2}}$} (e);
  \draw[->] (b) to node[xshift=0pt, yshift=7pt] {$\textstyle{\scriptstyle{#1}}$} (c);
  \draw[|->] (e) to node[xshift=0pt, yshift=7pt] {$\textstyle{\scriptstyle{#2}}$} (f);
\end{tikzpicture}}}}
\renewcommand{\thesection}{第\arabic{section}部}
\renewcommand{\thesubsection}{\arabic{section}.\arabic{subsection}}
\renewcommand{\thesubsubsection}{\arabic{section}.\arabic{subsection}.\arabic{subsubsection}}
\everymath{\displaystyle}
\allowdisplaybreaks[4]
\usepackage{vtable}
\theoremstyle{definition}
\newtheorem{thm}{定理}[subsection]
\newtheorem*{thm*}{定理}
\newtheorem{dfn}{定義}[subsection]
\newtheorem*{dfn*}{定義}
\newtheorem{axs}[dfn]{公理}
\newtheorem*{axs*}{公理}
\renewcommand{\headfont}{\bfseries}
\makeatletter
  \renewcommand{\section}{%
    \@startsection{section}{1}{\z@}%
    {\Cvs}{\Cvs}%
    {\normalfont\huge\headfont\raggedright}}
\makeatother
\makeatletter
  \renewcommand{\subsection}{%
    \@startsection{subsection}{2}{\z@}%
    {0.5\Cvs}{0.5\Cvs}%
    {\normalfont\LARGE\headfont\raggedright}}
\makeatother
\makeatletter
  \renewcommand{\subsubsection}{%
    \@startsection{subsubsection}{3}{\z@}%
    {0.4\Cvs}{0.4\Cvs}%
    {\normalfont\Large\headfont\raggedright}}
\makeatother
\makeatletter
\renewenvironment{proof}[1][\proofname]{\par
  \pushQED{\qed}%
  \normalfont \topsep6\p@\@plus6\p@\relax
  \trivlist
  \item\relax
  {
  #1\@addpunct{.}}\hspace\labelsep\ignorespaces
}{%
  \popQED\endtrivlist\@endpefalse
}
\makeatother
\renewcommand{\proofname}{\textbf{証明}}
\usepackage{tikz,graphics}
\usepackage[dvipdfmx]{hyperref}
\usepackage{pxjahyper}
\hypersetup{
 setpagesize=false,
 bookmarks=true,
 bookmarksdepth=tocdepth,
 bookmarksnumbered=true,
 colorlinks=false,
 pdftitle={},
 pdfsubject={},
 pdfauthor={},
 pdfkeywords={}}
\title{線形代数学の書籍のレビュー}
\author{@k74226197Y126}
\begin{document}
\maketitle
\section*{はじめに}
このPDF資料は一学生「江戸時代の農民(@k74226197Y126)」がもっている線形代数学の書籍のレビューです。正確性に努めておりますが、あくまでも個人的な感想ですので、完全には保証しきれません。なんなら、ぜひとも実際に現物に目を通しておいて自分の感想をもつといいかもしれません! 
\subsubsection*{対馬先生「線形代数学講義」}
対馬龍司, "線形代数学講義 改訂版", 共立出版, 2014. ISBN978-4-3201-1097-7 \url{https://www.kyoritsu-pub.co.jp/book/b10004120.html}
\begin{itemize}
    \item 基礎線形代数1,2の教科書だしおそらく明治大学理工学部生なら誰でも持っていると期待できる。なんならみんな一度は目を通したことはあるはず! 
    \item 行間は全体的にみれば狭く比較的難易度は低めで行列算数寄りである。ただし説明で誤魔化されてる箇所もありそうではある。
    \item ジョルダン標準形では単因子論の流儀をとってる。
    \item 直和の説明もあるがジョルダン標準形の内容は全体的にやや駆け足かもしれない? 
    \item テンソルについては一切言及がないのでテンソルもやりたいなら別の書籍にあたることになる。
\end{itemize}
\subsubsection*{齋藤先生「線型代数入門」}
斎藤正彦, "線型代数入門", 東京大学出版会, 1966. ISBN978-4-1306-2001-7 \url{https://www.utp.or.jp/book/b302039.html}
\begin{itemize}
    \item 線形代数学の教科書として定番なもののうち1つである。ちなみに有名な杉浦先生「解析入門I, II」と同じシリーズである。
    \item 上の対馬先生「線形代数学講義」と内容が似ていると思う。
    \item 難易度は人によるらしい? ごまかしは少なめだけど行間が広い箇所があると思われる。
    \item ジョルダン標準形は単因子論の流儀をとっている。
    \item 全体的に行列算数寄りだけど線形写像に関する記述もある。
    \item テンソルについては一切言及がないのでテンソルもやりたいなら別の書籍にあたることになる。
\end{itemize}
\subsubsection*{松坂先生「線型代数入門」}
松坂和夫, "線型代数入門", 岩波書店, 1980. ISBN978-4-0002-9872-8 (古いほう 978-4-0000-5556-7) \url{https://www.iwanami.co.jp/smp/book/b378348.html}
\begin{itemize}
    \item 線形代数学を行列算数のみにとどめず線形写像をフル活用してて抽象的に述べているもののうち個人的には読みやすいほうだと思う。行列算数でない線形代数学の側面が見られて線形代数学がいかに便利なのかが分かるかもしれない。
    \item ジョルダン標準形では冪零変換の流儀をとってる。ただ途中の議論を全部演習問題にまわすというところもある。特にヤングの図形に関する内容は後述する池田先生「テンソル代数と表現論」を参考にするといいかもしれない。
    \item それでも駆け足って書いてある書籍と比べたら遥かにジョルダン標準形の説明が結構ゆっくりだったと思う。
    \item それに内容を細かくみるような演習問題は難しくないし挑戦してみるのもいいと思う。明治大学理工学部物理学科2022年度統計力学1,2演習の講義資料の内容をきつく詰めてくるような感じの問題と似てるかもしれない。
    \item テンソルについては一切言及がないのでテンソルもやりたいなら別の書籍にあたることになる。
    \item 実は松坂先生「代数系入門」に単因子論の説明もある。さらに群論や環論の記述もある。参考までにどうぞ。
\end{itemize}
\subsubsection*{池田先生「テンソル代数と表現論」}
池田岳, "テンソル代数と表現論", 東京大学出版会, 2022. ISBN978-4-1306-2929-4 \url{https://www.utp.or.jp/book/b598957.html}
\begin{itemize}
    \item 線形代数学と代数学の内容が1つの本に一緒になったような感じが特徴である。
    \item 線形代数学パートでは、後述する佐武先生「線型代数学」に近い感じがするもののこれよりはブルバキスタイルになってて読みやすい。
    \item 線形代数学パートでは線形写像をフル活用している抽象的なほうである。
    \item 表現論では群論を既知としてるものの、一応簡潔ながらも、群論など代数学の初歩的な内容は巻末にまとめてあるといった配慮がみられる。
    \item テンソルも外積代数も扱っておりブルバキスタイルになってて結構分かりやすいと思う。
    \item ジョルダン標準形で少し判別で分かりづらいが実は冪零変換の流儀をとっている。しかしながら、ジョルダン標準形では商ベクトル空間を駆使してておりやや駆け足かもしれない? 
\end{itemize}
\subsubsection*{佐武先生「線型代数学」}
佐武一郎, "線型代数学", 裳華房, 1958. ISBN978-4-7853-1316-6 (古いほう978-4-7853-1301-2) \url{https://www.shokabo.co.jp/mybooks/ISBN978-4-7853-1316-6.htm}
\begin{itemize}
    \item 線形写像をフル活用している抽象的なほうである。
    \item 線形代数学に関してならなんでも載ってる。なんなら複素数の構成とか係数体の拡大とか表現論のいろはとかまでもある。線形代数学の辞書(?)みたいな感じで使うのがいいらしい。
    \item 私がB1のとき基礎線形代数1の先生から勧められたことがある\footnote{ただ、初学者に勧めると怒られるかもしれないけど、私は人によってはマセマを勧めるときもあれば、佐武先生「線型代数学」も勧めるときもあるかもしれないし、何らかの理由があって初学者に勧める人がいても怒らないと思う。}。
    \item ブルバキスタイルでない箇所がちらほらあって人によっては分かったふりになりやすくて結構難しいかもしれない。
    \item ジョルダン標準形では冪零変換の流儀をとっている。
    \item テンソルも外積代数も扱ってる。ただカノニカルな線形同型写像の記述が池田先生の「テンソル代数と表現論」と比べたら分かりにくいかもしれない。
    \item 表現論では斜め読みした限り上の池田先生の「テンソル代数と表現論」より群論に関する記述が少なめだとは思う。群論を別の書籍で慣れてから読まれるといいかもしれない。
\end{itemize}
\end{document}