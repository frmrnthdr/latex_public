\documentclass[dvipdfmx]{jsarticle}
\setcounter{section}{3}
\setcounter{subsection}{3}
\usepackage{xr}
\externaldocument{2.1.1}
\usepackage{amsmath,amsfonts,amssymb,array,comment,mathtools,url,docmute}
\usepackage{longtable,booktabs,dcolumn,tabularx,mathtools,multirow,colortbl,xcolor}
\usepackage[dvipdfmx]{graphics}
\usepackage{bmpsize}
\usepackage{amsthm}
\usepackage{enumitem}
\setlistdepth{20}
\renewlist{itemize}{itemize}{20}
\setlist[itemize]{label=•}
\renewlist{enumerate}{enumerate}{20}
\setlist[enumerate]{label=\arabic*.}
\setcounter{MaxMatrixCols}{20}
\setcounter{tocdepth}{3}
\newcommand{\rotin}{\text{\rotatebox[origin=c]{90}{$\in $}}}
\renewcommand{\thesection}{第\arabic{section}部}
\renewcommand{\thesubsection}{\arabic{section}.\arabic{subsection}}
\renewcommand{\thesubsubsection}{\arabic{section}.\arabic{subsection}.\arabic{subsubsection}}
\everymath{\displaystyle}
\allowdisplaybreaks[4]
\usepackage{vtable}
\theoremstyle{definition}
\newtheorem{thm}{定理}[subsection]
\newtheorem*{thm*}{定理}
\newtheorem{dfn}{定義}[subsection]
\newtheorem*{dfn*}{定義}
\newtheorem{axs}[dfn]{公理}
\newtheorem*{axs*}{公理}
\renewcommand{\headfont}{\bfseries}
\makeatletter
  \renewcommand{\section}{%
    \@startsection{section}{1}{\z@}%
    {\Cvs}{\Cvs}%
    {\normalfont\huge\headfont\raggedright}}
\makeatother
\makeatletter
  \renewcommand{\subsection}{%
    \@startsection{subsection}{2}{\z@}%
    {0.5\Cvs}{0.5\Cvs}%
    {\normalfont\LARGE\headfont\raggedright}}
\makeatother
\makeatletter
  \renewcommand{\subsubsection}{%
    \@startsection{subsubsection}{3}{\z@}%
    {0.4\Cvs}{0.4\Cvs}%
    {\normalfont\Large\headfont\raggedright}}
\makeatother
\makeatletter
\renewenvironment{proof}[1][\proofname]{\par
  \pushQED{\qed}%
  \normalfont \topsep6\p@\@plus6\p@\relax
  \trivlist
  \item\relax
  {
  #1\@addpunct{.}}\hspace\labelsep\ignorespaces
}{%
  \popQED\endtrivlist\@endpefalse
}
\makeatother
\renewcommand{\proofname}{\textbf{証明}}
\usepackage{tikz,graphics}
\usepackage[dvipdfmx]{hyperref}
\usepackage{pxjahyper}
\hypersetup{
 setpagesize=false,
 bookmarks=true,
 bookmarksdepth=tocdepth,
 bookmarksnumbered=true,
 colorlinks=false,
 pdftitle={},
 pdfsubject={},
 pdfauthor={},
 pdfkeywords={}}
\begin{document}
%\hypertarget{ux53ccux7ddaux5f62ux5f62ux5f0f}{%
\subsection{双線形形式}%\label{ux53ccux7ddaux5f62ux5f62ux5f0f}}
%\hypertarget{ux53ccux7ddaux5f62ux5f62ux5f0f-1}{%
\subsubsection{双線形形式}%\label{ux53ccux7ddaux5f62ux5f62ux5f0f-1}}
\begin{axs}[双線形形式の公理]
体$K$上のvector空間$V$が与えられたとき、次のことを満たすような写像$B:V \times V \rightarrow K$をそのvector空間$V$上の双線形形式、双1次形式という。
\begin{itemize}
\item
  $\forall a,b \in K\forall\mathbf{u},\mathbf{v},\mathbf{w} \in V$に対し、$B\left( a\mathbf{u} + b\mathbf{v},\mathbf{w} \right) = aB\left( \mathbf{u},\mathbf{w} \right) + bB\left( \mathbf{v},\mathbf{w} \right)$が成り立つ。
\item
  $\forall a,b \in K\forall\mathbf{u},\mathbf{v},\mathbf{w} \in V$に対し、$B\left( \mathbf{u},a\mathbf{v} + b\mathbf{w} \right) = aB\left( \mathbf{u},\mathbf{v} \right) + bB\left( \mathbf{u},\mathbf{w} \right)$が成り立つ。
\end{itemize}
\end{axs}
\begin{dfn}
体$K$上のvector空間$V$が与えられたとき、双線形形式$B:V \times V \rightarrow K$が次のことを満たすとき、その双線形形式$B$を対称双線形形式という。
\begin{itemize}
\item
  $\forall\mathbf{v},\mathbf{w} \in V$に対し、$B\left( \mathbf{v},\mathbf{w} \right) = B\left( \mathbf{w},\mathbf{v} \right)$が成り立つ。
\end{itemize}
\end{dfn}
\begin{dfn}
$K \subseteq \mathbb{C}$なる体$K$上のvector空間$V$が与えられたとき、次のことを満たすような写像$B:V \times V \rightarrow K$をそのvector空間$V$上の共役双線形形式、共役双1次形式という。
\begin{itemize}
\item
  $\forall a,b \in K\forall\mathbf{u},\mathbf{v},\mathbf{w} \in V$に対し、$B\left( a\mathbf{u} + b\mathbf{v},\mathbf{w} \right) = \overline{a}B\left( \mathbf{u},\mathbf{w} \right) + \overline{b}B\left( \mathbf{v},\mathbf{w} \right)$が成り立つ。
\item
  $\forall a,b \in K\forall\mathbf{u},\mathbf{v},\mathbf{w} \in V$に対し、$B\left( \mathbf{u},a\mathbf{v} + b\mathbf{w} \right) = aB\left( \mathbf{u},\mathbf{v} \right) + bB\left( \mathbf{u},\mathbf{w} \right)$が成り立つ。
\end{itemize}
\end{dfn}
\begin{dfn}
体$\mathbb{C}$上のvector空間$V$が与えられたとき、双線形形式$B:V \times V \rightarrow K$が次のことを満たすとき、その双線形形式$B$をHermite双線形形式、歪対称双線形形式という。
\begin{itemize}
\item
  $\forall\mathbf{v},\mathbf{w} \in V$に対し、$B\left( \mathbf{v},\mathbf{w} \right) = \overline{B\left( \mathbf{w},\mathbf{v} \right)}$が成り立つ。
\end{itemize}
\end{dfn}\par
ここで、体$K$が$K \subseteq \mathbb{R}$を満たすとき、双線形形式と共役双線形形式とは一致することに注意されたい。
\begin{thm}\label{2.3.4.1}
体$K$上のvector空間$V$上の任意の双線形形式$B$が与えられたとき、その体$K$の任意の族々$\left\{ a_{i} \right\}_{i \in \varLambda_{m}}$、$\left\{ b_{i} \right\}_{i \in \varLambda_{m}}$、そのvector空間の任意の族々$\left\{ \mathbf{v}_{i} \right\}_{i \in \varLambda_{m}}$、$\left\{ \mathbf{w}_{i} \right\}_{i \in \varLambda_{m}}$に対し、次式が成り立つ。
\begin{align*}
B\left( \sum_{i \in \varLambda_{m}} {a_{i}\mathbf{v}_{i}},\sum_{i \in \varLambda_{n}} {b_{i}\mathbf{w}_{i}} \right) = \sum_{(i,j) \in \varLambda_{m} \times \varLambda_{n}} {a_{i}b_{j}B\left( \mathbf{v}_{i},\mathbf{w}_{j} \right)}
\end{align*}
このとき、$\forall\mathbf{v} \in V$に対し、$B\left( \mathbf{v},\mathbf{0} \right) = B\left( \mathbf{0},\mathbf{v} \right) = 0$が成り立つ。
\end{thm}
\begin{proof}
体$K$上のvector空間$V$上の任意の双線形形式$B$が与えられたとき、これ上の任意の双線形形式$B$、その体$K$の任意の族々$\left\{ a_{i} \right\}_{i \in \varLambda_{m}}$、$\left\{ b_{i} \right\}_{i \in \varLambda_{n}}$、そのvector空間の任意の族々$\left\{ \mathbf{v}_{i} \right\}_{i \in \varLambda_{m}}$、$\left\{ \mathbf{w}_{i} \right\}_{i \in \varLambda_{n}}$に対し、数学的帰納法により直ちにわかるように次のようになる。
\begin{align*}
B\left( \sum_{i \in \varLambda_{m}} {a_{i}\mathbf{v}_{i}},\sum_{i \in \varLambda_{n}} {b_{i}\mathbf{w}_{i}} \right) &= \sum_{i \in \varLambda_{m}} {a_{i}B\left( \mathbf{v}_{i},\sum_{j \in \varLambda_{n}} {b_{j}\mathbf{w}_{j}} \right)}\\
&= \sum_{i \in \varLambda_{m}} {\sum_{j \in \varLambda_{n}} {a_{i}b_{j}B\left( \mathbf{v}_{i},\mathbf{w}_{j} \right)}}\\
&= \sum_{(i,j) \in \varLambda_{m} \times \varLambda_{n}} {a_{i}b_{j}B\left( \mathbf{v}_{i},\mathbf{w}_{j} \right)}
\end{align*}\par
このとき、$\forall\mathbf{v} \in V$に対し、次のようになるので、
\begin{align*}
B\left( \mathbf{v},\mathbf{0} \right) &= B\left( \mathbf{v},0\mathbf{v} \right) = 0B\left( \mathbf{v},\mathbf{v} \right) = 0\\
B\left( \mathbf{0},\mathbf{v} \right) &= B\left( 0\mathbf{v},\mathbf{v} \right) = 0B\left( \mathbf{v},\mathbf{v} \right) = 0
\end{align*}
$B\left( \mathbf{v},\mathbf{0} \right) = B\left( \mathbf{0},\mathbf{v} \right) = 0$が成り立つ
\end{proof}
\begin{thm}\label{2.3.4.2}
$K \subseteq \mathbb{C}$なる体$K$上のvector空間$V$上の任意の共役双線形形式$B$が与えられたとき、その体$K$の任意の族々$\left\{ a_{i} \right\}_{i \in \varLambda_{m}}$、$\left\{ b_{i} \right\}_{i \in \varLambda_{m}}$、そのvector空間の任意の族々$\left\{ \mathbf{v}_{i} \right\}_{i \in \varLambda_{m}}$、$\left\{ \mathbf{w}_{i} \right\}_{i \in \varLambda_{m}}$に対し、次式が成り立つ。
\begin{align*}
B\left( \sum_{i \in \varLambda_{m}} {a_{i}\mathbf{v}_{i}},\sum_{i \in \varLambda_{n}} {b_{i}\mathbf{w}_{i}} \right) = \sum_{(i,j) \in \varLambda_{m} \times \varLambda_{n}} {\overline{a_{i}}b_{j}B\left( \mathbf{v}_{i},\mathbf{w}_{j} \right)}
\end{align*}
このとき、$\forall\mathbf{v} \in V$に対し、$B\left( \mathbf{v},\mathbf{0} \right) = B\left( \mathbf{0},\mathbf{v} \right) = 0$が成り立つ。
\end{thm}
\begin{proof}
定理\ref{2.3.4.1}と同様にして示される。
\end{proof}
\begin{thm}\label{2.3.4.3}
$K \subseteq \mathbb{C}$なる体$K$上のvector空間$V$上の任意のHermite双線形形式$B$が与えられたとき、$\forall\mathbf{v} \in V$に対し、$B\left( \mathbf{v},\mathbf{v} \right) \in \mathbb{R}$が成り立つ。
\end{thm}
\begin{proof}
$K \subseteq \mathbb{C}$なる体$K$上のvector空間$V$上の任意のHermite双線形形式$B$が与えられたとき、$\forall\mathbf{v} \in V$に対し、定義より$B\left( \mathbf{v},\mathbf{v} \right) = \overline{B\left( \mathbf{v},\mathbf{v} \right)}$が成り立つので、$B\left( \mathbf{v},\mathbf{v} \right) = \alpha + \beta i$、$\alpha,\beta \in \mathbb{R}$とおくと、$\alpha + \beta i = \alpha - \beta i$が成り立つことになり、即ち、$\beta = - \beta$が成り立つ。ゆえに、$\beta = 0$が得られるので、$B\left( \mathbf{v},\mathbf{v} \right) = \alpha \in \mathbb{R}$が成り立つ。
\end{proof}
\begin{thm}\label{2.3.4.4}
体$K$上のvector空間$V$上の任意の双線形形式たち$B$、$C$が与えられたとき、$\forall k,l \in K$に対し、写像$kB + lC:V \times V \rightarrow K$もそのvector空間$V$の双線形形式である。
\end{thm}
\begin{proof}
体$K$上のvector空間$V$上の任意の双線形形式たち$B$、$C$が与えられたとき、$\forall k,l \in K$に対し、写像$kB + lC:V \times V \rightarrow K$において、$\forall a,b \in K\forall\mathbf{u},\mathbf{v},\mathbf{w} \in V$に対し、次のようになる。
\begin{align*}
(kB + lC)\left( a\mathbf{u} + b\mathbf{v},\mathbf{w} \right) &= kB\left( a\mathbf{u} + b\mathbf{v},\mathbf{w} \right) + lC\left( a\mathbf{u} + b\mathbf{v},\mathbf{w} \right)\\
&= k\left( aB\left( \mathbf{u},\mathbf{w} \right) + bB\left( \mathbf{v},\mathbf{w} \right) \right) + l\left( aC\left( \mathbf{u},\mathbf{w} \right) + bC\left( \mathbf{v},\mathbf{w} \right) \right)\\
&= akB\left( \mathbf{u},\mathbf{w} \right) + bkB\left( \mathbf{v},\mathbf{w} \right) + alC\left( \mathbf{u},\mathbf{w} \right) + blC\left( \mathbf{v},\mathbf{w} \right)\\
&= a\left( kB\left( \mathbf{u},\mathbf{w} \right) + lC\left( \mathbf{u},\mathbf{w} \right) \right) + b\left( kB\left( \mathbf{v},\mathbf{w} \right) + lC\left( \mathbf{v},\mathbf{w} \right) \right)\\
&= a(kB + lC)\left( \mathbf{u},\mathbf{w} \right) + b(kB + lC)\left( \mathbf{v},\mathbf{w} \right)\\
(kB + lC)\left( \mathbf{u},a\mathbf{v} + b\mathbf{w} \right) &= kB\left( \mathbf{u},a\mathbf{v} + b\mathbf{w} \right) + lC\left( \mathbf{u},a\mathbf{v} + b\mathbf{w} \right)\\
&= k\left( aB\left( \mathbf{u},\mathbf{v} \right) + bB\left( \mathbf{u},\mathbf{w} \right) \right) + l\left( aC\left( \mathbf{u},\mathbf{v} \right) + bC\left( \mathbf{u},\mathbf{w} \right) \right)\\
&= akB\left( \mathbf{u},\mathbf{v} \right) + bkB\left( \mathbf{u},\mathbf{w} \right) + alC\left( \mathbf{u},\mathbf{v} \right) + blC\left( \mathbf{u},\mathbf{w} \right)\\
&= a\left( kB\left( \mathbf{u},\mathbf{v} \right) + lC\left( \mathbf{u},\mathbf{v} \right) \right) + b\left( kB\left( \mathbf{u},\mathbf{w} \right) + lC\left( \mathbf{u},\mathbf{w} \right) \right)\\
&= a(kB + lC)\left( \mathbf{u},\mathbf{v} \right) + b(kB + lC)\left( \mathbf{u},\mathbf{w} \right)
\end{align*}
よって、その写像$kB + lC:V \times V \rightarrow K$もそのvector空間$V$の双線形形式である。
\end{proof}
\begin{thm}\label{2.3.4.5}
$K \subseteq \mathbb{C}$なる体$K$上のvector空間$V$上の任意の共役双線形形式たち$B$、$C$が与えられたとき、$\forall k,l \in K$に対し、写像$kB + lC:V \times V \rightarrow K$もそのvector空間$V$の共役双線形形式である。
\end{thm}
\begin{proof} 定理\ref{2.3.4.4}と同様にして示される。
\end{proof}
\begin{thm}\label{2.3.4.6}
体$K$上のvector空間$V$上の任意の双線形形式$B$が与えられたとき、次式のように写像$C$が定義されると、
\begin{align*}
C:V \times V \rightarrow K;\left( \mathbf{v},\mathbf{w} \right) \mapsto B\left( \mathbf{w},\mathbf{v} \right)
\end{align*}
その写像$C$もそのvector空間$V$上の双線形形式で、さらに、その写像$B + C$はそのvector空間$V$上の対称双線形形式である。
\end{thm}
\begin{proof}
体$K$上のvector空間$V$上の任意の双線形形式$B$が与えられたとき、次式のように写像$C$が定義されると、
\begin{align*}
C:V \times V \rightarrow K;\left( \mathbf{v},\mathbf{w} \right) \mapsto B\left( \mathbf{w},\mathbf{v} \right)
\end{align*}
$\forall a,b \in K\forall\mathbf{u},\mathbf{v},\mathbf{w} \in V$に対し、次のようになる。
\begin{align*}
C\left( a\mathbf{u} + b\mathbf{v},\mathbf{w} \right) &= B\left( \mathbf{w},a\mathbf{u} + b\mathbf{v} \right)\\
&= aB\left( \mathbf{w},\mathbf{u} \right) + bB\left( \mathbf{w},\mathbf{v} \right)\\
&= aC\left( \mathbf{u},\mathbf{w} \right) + bC\left( \mathbf{v},\mathbf{w} \right)\\
C\left( \mathbf{u},a\mathbf{v} + b\mathbf{w} \right) &= B\left( a\mathbf{v} + b\mathbf{w},\mathbf{u} \right)\\
&= aB\left( \mathbf{v},\mathbf{u} \right) + bB\left( \mathbf{w},\mathbf{u} \right)\\
&= aC\left( \mathbf{u},\mathbf{v} \right) + bC\left( \mathbf{u},\mathbf{w} \right)
\end{align*}
よって、その写像$C$もそのvector空間$V$の双線形形式である。\par
さらに、定理\ref{2.3.4.4}よりその写像$B + C$もそのvector空間$V$の双線形形式であり、さらに、$\forall\mathbf{v},\mathbf{w} \in V$に対し、次のようになる。
\begin{align*}
(B + C)\left( \mathbf{v},\mathbf{w} \right) &= B\left( \mathbf{v},\mathbf{w} \right) + C\left( \mathbf{v},\mathbf{w} \right)\\
&= C\left( \mathbf{w},\mathbf{v} \right) + B\left( \mathbf{w},\mathbf{v} \right)\\
&= B\left( \mathbf{w},\mathbf{v} \right) + C\left( \mathbf{w},\mathbf{v} \right)\\
&= (B + C)\left( \mathbf{w},\mathbf{v} \right)
\end{align*}
よって、その写像$B + C$はそのvector空間$V$上の対称双線形形式である。
\end{proof}
\begin{thm}\label{2.3.4.7}
$K \subseteq \mathbb{C}$なる体$K$上のvector空間$V$上の任意の共役双線形形式$B$が与えられたとき、次式のように写像$C$が定義されると、
\begin{align*}
C:V \times V \rightarrow K;\left( \mathbf{v},\mathbf{w} \right) \mapsto B\left( \mathbf{w},\mathbf{v} \right)
\end{align*}
その写像$C$もそのvector空間$V$上の共役双線形形式で、さらに、その写像$B + C$はそのvector空間$V$上のHermite双線形形式である。
\end{thm}
\begin{proof} 定理\ref{2.3.4.6}と同様にして示される。
\end{proof}
%\hypertarget{ux53ccux7ddaux5f62ux5f62ux5f0fux306eux8868ux73feux884cux5217}{%
\subsubsection{双線形形式の表現行列}%\label{ux53ccux7ddaux5f62ux5f62ux5f0fux306eux8868ux73feux884cux5217}}
\begin{dfn}
体$K$上の$n$次元vector空間$V$上の任意の双線形形式$B$が与えられたとき、$\alpha = \left\langle \mathbf{v}_{i} \right\rangle_{i \in \varLambda_{n}}$なるそのvector空間$V$の基底$\alpha$がとられれば、次式のように定義されるその集合$M_{nn}(K)$における行列$A_{nn}$をその双線形形式$B$のその基底$\alpha$に関する表現行列、表現などといい、
\begin{align*}
A_{nn} = \left( B\left( \mathbf{v}_{i},\mathbf{v}_{j} \right) \right)_{(i,j) \in \varLambda_{n}^{2}}
\end{align*}
以下、その行列$A_{nn}$を$[ B]_{\alpha}$とかく。
\end{dfn}
\begin{dfn}
$K \subseteq \mathbb{C}$なる体$K$上の$n$次元vector空間$V$上の任意の共役双線形形式$B$が与えられたとき、$\alpha = \left\langle \mathbf{v}_{i} \right\rangle_{i \in \varLambda_{n}}$なるそのvector空間$V$の基底$\alpha$がとられれば、次式のように定義されるその集合$M_{nn}(K)$における行列$A_{nn}$をその共役双線形形式$B$のその基底$\alpha$に関する表現行列、表現などといい、
\begin{align*}
A_{nn} = \left( B\left( \mathbf{v}_{i},\mathbf{v}_{j} \right) \right)_{(i,j) \in \varLambda_{n}^{2}}
\end{align*}
以下、その行列$A_{nn}$を$[ B]_{\alpha}$とかく。
\end{dfn}
\begin{thm}\label{2.3.4.8}
体$K$上の$n$次元vector空間$V$上の任意の双線形形式$B$が与えられたとき、$\alpha = \left\langle \mathbf{v}_{i} \right\rangle_{i \in \varLambda_{n}}$なるそのvector空間$V$の基底$\alpha$がとられ、さらに、その双線形形式$B$のその基底$\alpha$に関する表現行列$[ B]_{\alpha}$が$[ B]_{\alpha} = \left( B_{ij} \right)_{(i,j) \in \varLambda_{n}^{2}}$と成分表示されれば、$\forall\mathbf{v},\mathbf{w} \in V$に対し、その体$K$の族々$\left\{ a_{i} \right\}_{i \in \varLambda_{n}}$、$\left\{ b_{i} \right\}_{i \in \varLambda_{n}}$を用いて次のようにおくと、
\begin{align*}
\mathbf{v} = \sum_{i \in \varLambda_{n}} {a_{i}\mathbf{v}_{i}},\ \ \mathbf{w} = \sum_{i \in \varLambda_{n}} {b_{i}\mathbf{v}_{i}}
\end{align*}
その基底$\alpha$に関する基底変換における線形同型写像$\varphi_{\alpha}$を用いれば、次式が成り立つ。
\begin{align*}
B\left( \mathbf{v},\mathbf{w} \right) = \sum_{i,j \in \varLambda_{n}} {a_{i}b_{j}B_{ij}} ={}^{t}\varphi_{\alpha}^{- 1}\left( \mathbf{v} \right)[ B]_{\alpha}\varphi_{\alpha}^{- 1}\left( \mathbf{w} \right)
\end{align*}
\end{thm}
\begin{proof}
体$K$上の$n$次元vector空間$V$上の任意の双線形形式$B$が与えられたとき、$\alpha = \left\langle \mathbf{v}_{i} \right\rangle_{i \in \varLambda_{n}}$なるそのvector空間$V$の基底$\alpha$がとられ、さらに、その双線形形式$B$のその基底$\alpha$に関する表現行列$[ B]_{\alpha}$が$[ B]_{\alpha} = \left( B_{ij} \right)_{(i,j) \in \varLambda_{n}^{2}}$と成分表示されれば、$\forall\mathbf{v},\mathbf{w} \in V$に対し、その体$K$の族々$\left\{ a_{i} \right\}_{i \in \varLambda_{n}}$、$\left\{ b_{i} \right\}_{i \in \varLambda_{n}}$を用いて次のようにおくと、
\begin{align*}
\mathbf{v} = \sum_{i \in \varLambda_{n}} {a_{i}\mathbf{v}_{i}},\ \ \mathbf{w} = \sum_{i \in \varLambda_{n}} {b_{i}\mathbf{v}_{i}}
\end{align*}
定理\ref{2.3.4.1}より次のようになる。
\begin{align*}
B\left( \mathbf{v},\mathbf{w} \right) = B\left( \sum_{i \in \varLambda_{n}} {a_{i}\mathbf{v}_{i}},\sum_{i \in \varLambda_{n}} {b_{i}\mathbf{v}_{i}} \right) = \sum_{i,j \in \varLambda_{n}} {a_{i}b_{j}B\left( \mathbf{v}_{i},\mathbf{v}_{j} \right)}
\end{align*}
そこで、双線形形式の表現行列の定義より$B\left( \mathbf{v}_{i},\mathbf{v}_{j} \right) = B_{ij}$が成り立つので、次のようになる。
\begin{align*}
B\left( \mathbf{v},\mathbf{w} \right) = \sum_{i,j \in \varLambda_{n}} {a_{i}b_{j}B_{ij}}
\end{align*}\par
また、その基底$\alpha$に関する基底変換における線形同型写像$\varphi_{\alpha}$を用いれば、次のようになるので、
\begin{align*}
\varphi_{\alpha}^{- 1}\left( \mathbf{v} \right) = \begin{pmatrix}
a_{1} \\
a_{2} \\
 \vdots \\
a_{n} \\
\end{pmatrix},\ \ \varphi_{\alpha}^{- 1}\left( \mathbf{w} \right) = \begin{pmatrix}
b_{1} \\
b_{2} \\
 \vdots \\
b_{n} \\
\end{pmatrix}
\end{align*}
したがって、次のようになる。
\begin{align*}
{}^{t}\varphi_{\alpha}^{- 1}\left( \mathbf{v} \right)[ B]_{\alpha}\varphi_{\alpha}^{- 1}\left( \mathbf{w} \right) &={}^{t}\begin{pmatrix}
a_{1} \\
a_{2} \\
 \vdots \\
a_{n} \\
\end{pmatrix}\begin{pmatrix}
B_{11} & B_{12} & \cdots & B_{1n} \\
B_{21} & B_{22} & \cdots & B_{2n} \\
 \vdots & \vdots & \ddots & \vdots \\
B_{n1} & B_{n2} & \cdots & B_{nn} \\
\end{pmatrix}\begin{pmatrix}
b_{1} \\
b_{2} \\
 \vdots \\
b_{n} \\
\end{pmatrix}\\
&= \begin{pmatrix}
a_{1} & a_{2} & \cdots & a_{n} \\
\end{pmatrix}\begin{pmatrix}
B_{11} & B_{12} & \cdots & B_{1n} \\
B_{21} & B_{22} & \cdots & B_{2n} \\
 \vdots & \vdots & \ddots & \vdots \\
B_{n1} & B_{n2} & \cdots & B_{nn} \\
\end{pmatrix}\begin{pmatrix}
b_{1} \\
b_{2} \\
 \vdots \\
b_{n} \\
\end{pmatrix}\\
&= \begin{pmatrix}
a_{1} & a_{2} & \cdots & a_{n} \\
\end{pmatrix}\begin{pmatrix}
\sum_{j \in \varLambda_{n}} {b_{j}B_{1j}} \\
\sum_{j \in \varLambda_{n}} {b_{j}B_{2j}} \\
 \vdots \\
\sum_{j \in \varLambda_{n}} {b_{j}B_{nj}} \\
\end{pmatrix}\\
&= \sum_{i \in \varLambda_{n}} {\sum_{j \in \varLambda_{n}} {a_{i}b_{j}B_{ij}}} = \sum_{i,j \in \varLambda_{n}} {a_{i}b_{j}B_{ij}}
\end{align*}
よって、次式が成り立つ。
\begin{align*}
B\left( \mathbf{v},\mathbf{w} \right) = \sum_{i,j \in \varLambda_{n}} {a_{i}b_{j}B_{ij}} ={}^{t}\varphi_{\alpha}^{- 1}\left( \mathbf{v} \right)[ B]_{\alpha}\varphi_{\alpha}^{- 1}\left( \mathbf{w} \right)
\end{align*}
\end{proof}
\begin{thm}\label{2.3.4.9}
$K \subseteq \mathbb{C}$なる体$K$上の$n$次元vector空間$V$上の任意の共役双線形形式$B$が与えられたとき、$\alpha = \left\langle \mathbf{v}_{i} \right\rangle_{i \in \varLambda_{n}}$なるそのvector空間$V$の基底$\alpha$がとられ、さらに、その共役双線形形式$B$のその基底$\alpha$に関する表現行列$[ B]_{\alpha}$が$[ B]_{\alpha} = \left( B_{ij} \right)_{(i,j) \in \varLambda_{n}^{2}}$と成分表示されれば、$\forall\mathbf{v},\mathbf{w} \in V$に対し、その体$K$の族々$\left\{ a_{i} \right\}_{i \in \varLambda_{n}}$、$\left\{ b_{i} \right\}_{i \in \varLambda_{n}}$を用いて次のようにおくと、
\begin{align*}
\mathbf{v} = \sum_{i \in \varLambda_{n}} {a_{i}\mathbf{v}_{i}},\ \ \mathbf{w} = \sum_{i \in \varLambda_{n}} {b_{i}\mathbf{v}_{i}}
\end{align*}
その基底$\alpha$に関する基底変換における線形同型写像$\varphi_{\alpha}$を用いれば、次式が成り立つ。
\begin{align*}
B\left( \mathbf{v},\mathbf{w} \right) = \sum_{i,j \in \varLambda_{n}} {\overline{a_{i}}b_{j}B_{ij}} ={}^{t}\overline{\varphi_{\alpha}^{- 1}\left( \mathbf{v} \right)}[ B]_{\alpha}\varphi_{\alpha}^{- 1}\left( \mathbf{w} \right)
\end{align*}
\end{thm}
\begin{proof} 定理\ref{2.3.4.8}と同様にして示される。
\end{proof}
\begin{thm}\label{2.3.4.10}
体$K$上の$n$次元vector空間$V$が与えられたとき、$\alpha = \left\langle \mathbf{v}_{i} \right\rangle_{i \in \varLambda_{n}}$なるそのvector空間$V$の基底$\alpha$がとられて、$\forall A_{nn} \in M_{nn}(K)$に対し、その基底$\alpha$に関する基底変換における線形同型写像$\varphi_{\alpha}$を用いて次式のような写像$B_{A_{nn}}$を考えよう。
\begin{align*}
B_{A_{nn}}:V \times V \rightarrow K;\left( \mathbf{v},\mathbf{w} \right) \mapsto{}^{t}\varphi_{\alpha}^{- 1}\left( \mathbf{v} \right)A_{nn}\varphi_{\alpha}^{- 1}\left( \mathbf{w} \right)
\end{align*}
このとき、次のことが成り立つ。
\begin{itemize}
\item
  その写像$B_{A_{nn}}$はそのvector空間$V$上の双線形形式である。
\item
  $A_{nn} = \left( a_{ij} \right)_{(i,j) \in \varLambda_{n}^{2}}$とおかれれば、$\forall i,j \in \varLambda_{n}$に対し、$B_{A_{nn}}\left( \mathbf{v}_{i},\mathbf{v}_{j} \right) = a_{ij}$が成り立つ。
\item
  $\left[ B_{A_{nn}} \right]_{\alpha} = A_{nn}$が成り立つ。
\end{itemize}
\end{thm}
\begin{proof}
体$K$上の$n$次元vector空間$V$が与えられたとき、$\alpha = \left\langle \mathbf{v}_{i} \right\rangle_{i \in \varLambda_{n}}$なるそのvector空間$V$の基底$\alpha$がとられれば、$\forall A_{nn} \in M_{nn}(K)$に対し、その基底$\alpha$に関する基底変換における線形同型写像$\varphi_{\alpha}$を用いて次式のような写像$B_{A_{nn}}$を考えよう。
\begin{align*}
B_{A_{nn}}:V \times V \rightarrow K;\left( \mathbf{v},\mathbf{w} \right) \mapsto{}^{t}\varphi_{\alpha}^{- 1}\left( \mathbf{v} \right)A_{nn}\varphi_{\alpha}^{- 1}\left( \mathbf{w} \right)
\end{align*}
このとき、$\forall a,b \in K\forall\mathbf{u},\mathbf{v},\mathbf{w} \in V$に対し、次のようになる。
\begin{align*}
B_{A_{nn}}\left( a\mathbf{u} + b\mathbf{v},\mathbf{w} \right) &={}^{t}\varphi_{\alpha}^{- 1}\left( a\mathbf{u} + b\mathbf{v} \right)A_{nn}\varphi_{\alpha}^{- 1}\left( \mathbf{w} \right)\\
&= \left( a{}^{t}\varphi_{\alpha}^{- 1}\left( \mathbf{u} \right) + b{}^{t}\varphi_{\alpha}^{- 1}\left( \mathbf{v} \right) \right)A_{nn}\varphi_{\alpha}^{- 1}\left( \mathbf{w} \right)\\
&= a{}^{t}\varphi_{\alpha}^{- 1}\left( \mathbf{u} \right)A_{nn}\varphi_{\alpha}^{- 1}\left( \mathbf{w} \right) + b{}^{t}\varphi_{\alpha}^{- 1}\left( \mathbf{v} \right)A_{nn}\varphi_{\alpha}^{- 1}\left( \mathbf{w} \right)\\
&= aB_{A_{nn}}\left( \mathbf{u},\mathbf{w} \right) + bB_{A_{nn}}\left( \mathbf{v},\mathbf{w} \right)\\
B_{A_{nn}}\left( \mathbf{u},a\mathbf{v} + b\mathbf{w} \right) &={}^{t}\varphi_{\alpha}^{- 1}\left( \mathbf{u} \right)A_{nn}\varphi_{\alpha}^{- 1}\left( a\mathbf{v} + b\mathbf{w} \right)\\
&={}^{t}\varphi_{\alpha}^{- 1}\left( \mathbf{u} \right)A_{nn}\left( a\varphi_{\alpha}^{- 1}\left( \mathbf{v} \right) + b\varphi_{\alpha}^{- 1}\left( \mathbf{w} \right) \right)\\
&= a{}^{t}\varphi_{\alpha}^{- 1}\left( \mathbf{u} \right)A_{nn}\varphi_{\alpha}^{- 1}\left( \mathbf{v} \right) + b{}^{t}\varphi_{\alpha}^{- 1}\left( \mathbf{u} \right)A_{nn}\varphi_{\alpha}^{- 1}\left( \mathbf{w} \right)\\
&= aB_{A_{nn}}\left( \mathbf{u},\mathbf{v} \right) + bB_{A_{nn}}\left( \mathbf{u},\mathbf{w} \right)
\end{align*}
よって、その写像$B_{A_{nn}}$はそのvector空間$V$上の双線形形式である。\par
$A_{nn} = \left( a_{ij} \right)_{(i,j) \in \varLambda_{n}^{2}}$とおかれれば、$\forall i,j \in \varLambda_{n}$に対し、vector空間$K^{n}$の標準基底のうち第$i'$成分が$1$でこれ以外の成分が$0$であるようなvector$\mathbf{e}_{i'} = \left( \left\{ \begin{matrix}
1 & \mathrm{if} & i = i' \\
0 & \mathrm{if} & i \neq i' \\
\end{matrix} \right.\  \right)_{i \in \varLambda_{n}}$を用いて次のようになる。
\begin{align*}
B_{A_{nn}}\left( \mathbf{v}_{i},\mathbf{v}_{j} \right) &={}^{t}\varphi_{\alpha}^{- 1}\left( \mathbf{v}_{i} \right)A_{nn}\varphi_{\alpha}^{- 1}\left( \mathbf{v}_{j} \right)\\
&={}^{t}\mathbf{e}_{i}A_{nn}\mathbf{e}_{j}\\
&= \begin{pmatrix}
0 & \cdots & 1 & \cdots & 0 \\
\end{pmatrix}\begin{pmatrix}
a_{11} & \cdots & a_{1j} & \cdots & a_{1n} \\
 \vdots & \ddots & \vdots & \ddots & \vdots \\
a_{i1} & \cdots & a_{ij} & \cdots & a_{in} \\
 \vdots & \ddots & \vdots & \ddots & \vdots \\
a_{n1} & \cdots & a_{nj} & \cdots & a_{nn} \\
\end{pmatrix}\begin{pmatrix}
0 \\
 \vdots \\
1 \\
 \vdots \\
0 \\
\end{pmatrix}\\
&= \begin{pmatrix}
0 & \cdots & 1 & \cdots & 0 \\
\end{pmatrix}\begin{pmatrix}
a_{1j} \\
 \vdots \\
a_{ij} \\
 \vdots \\
a_{nj} \\
\end{pmatrix} = a_{ij}
\end{align*}\par
あとは、表現行列の定義より明らかに$\left[ B_{A_{nn}} \right]_{\alpha} = A_{nn}$が成り立つ。
\end{proof}
\begin{thm}\label{2.3.4.11}
$K \subseteq \mathbb{C}$なる体$K$上の$n$次元vector空間$V$が与えられたとき、$\alpha = \left\langle \mathbf{v}_{i} \right\rangle_{i \in \varLambda_{n}}$なるそのvector空間$V$の基底$\alpha$がとられて、$\forall A_{nn} \in M_{nn}(K)$に対し、その基底$\alpha$に関する基底変換における線形同型写像$\varphi_{\alpha}$を用いて次式のような写像$B_{A_{nn}}$を考えよう。
\begin{align*}
B_{A_{nn}}:V \times V \rightarrow K;\left( \mathbf{v},\mathbf{w} \right) \mapsto{}^{t}\overline{\varphi_{\alpha}^{- 1}\left( \mathbf{v} \right)}A_{nn}\varphi_{\alpha}^{- 1}\left( \mathbf{w} \right)
\end{align*}
このとき、次のことが成り立つ。
\begin{itemize}
\item
  その写像$B_{A_{nn}}$はそのvector空間$V$上の共役双線形形式である。
\item
  $A_{nn} = \left( a_{ij} \right)_{(i,j) \in \varLambda_{n}^{2}}$とおかれれば、$\forall i,j \in \varLambda_{n}$に対し、$B_{A_{nn}}\left( \mathbf{v}_{i},\mathbf{v}_{j} \right) = a_{ij}$が成り立つ。
\item
  $\left[ B_{A_{nn}} \right]_{\alpha} = A_{nn}$が成り立つ。
\end{itemize}
\end{thm}
\begin{proof} 定理\ref{2.3.4.10}と同様にして示される。
\end{proof}
\begin{thm}\label{2.3.4.12}
体$K$上の$n$次元vector空間$V$上の任意の双線形形式$B$が与えられたとき、$\alpha = \left\langle \mathbf{v}_{i} \right\rangle_{i \in \varLambda_{n}}$なるそのvector空間$V$の基底$\alpha$がとられれば、その双線形形式$B$が対称双線形形式となるならそのときに限り、その双線形形式$B$の基底$\alpha$に関する表現行列$[ B]_{\alpha}$が対称行列となる、即ち、$[ B]_{\alpha} ={}^{t}[ B]_{\alpha}$が成り立つ。
\end{thm}
\begin{proof}
体$K$上の$n$次元vector空間$V$上の任意の双線形形式$B$が与えられたとき、$\alpha = \left\langle \mathbf{v}_{i} \right\rangle_{i \in \varLambda_{n}}$なるそのvector空間$V$の基底$\alpha$がとられれば、その双線形形式$B$が対称双線形形式となるなら、$\forall i,j \in \varLambda_{n}$に対し、$B\left( \mathbf{v}_{i},\mathbf{v}_{j} \right) = B\left( \mathbf{v}_{j},\mathbf{v}_{i} \right)$が成り立つ。ここで、$[ B]_{\alpha} = \left( B_{ij} \right)_{(i,j) \in \varLambda_{n}^{2}}$と成分表示されれば、次のようになるので、
\begin{align*}
B_{ij} = B\left( \mathbf{v}_{i},\mathbf{v}_{j} \right) = B\left( \mathbf{v}_{j},\mathbf{v}_{i} \right) = B_{ji}
\end{align*}
次のようになる。
\begin{align*}
[ B]_{\alpha} &= \left( B_{ij} \right)_{(i,j) \in \varLambda_{n}^{2}}\\
&= \left( B_{ji} \right)_{(i,j) \in \varLambda_{n}^{2}}\\
&={}^{t}\left( B_{ji} \right)_{(i,j) \in \varLambda_{n}^{2}}\\
&={}^{t}[ B]_{\alpha}
\end{align*}\par
逆に、その双線形形式$B$の基底$\alpha$に関する表現行列$[ B]_{\alpha}$が対称行列となる、即ち、$[ B]_{\alpha} ={}^{t}[ B]_{\alpha}$が成り立つなら、その基底$\alpha$に関する基底変換における線形同型写像$\varphi_{\alpha}$を用いて、$\forall\mathbf{v},\mathbf{w} \in V$に対し、定理\ref{2.3.4.8}より次のようになる。
\begin{align*}
B\left( \mathbf{v},\mathbf{w} \right) &={}^{t}\varphi_{\alpha}^{- 1}\left( \mathbf{v} \right)[ B]_{\alpha}\varphi_{\alpha}^{- 1}\left( \mathbf{w} \right)\\
&={}^{t}\varphi_{\alpha}^{- 1}\left( \mathbf{v} \right){}^{t}{}^{t}[ B]_{\alpha}{}^{t}{}^{t}\varphi_{\alpha}^{- 1}\left( \mathbf{w} \right)\\
&={}^{t}\left({}^{t}\varphi_{\alpha}^{- 1}\left( \mathbf{w} \right){}^{t}[ B]_{\alpha}\varphi_{\alpha}^{- 1}\left( \mathbf{v} \right) \right)\\
&={}^{t}\left({}^{t}\varphi_{\alpha}^{- 1}\left( \mathbf{w} \right)[ B]_{\alpha}\varphi_{\alpha}^{- 1}\left( \mathbf{v} \right) \right)\\
&={}^{t}B\left( \mathbf{w},\mathbf{v} \right)\\
&= B\left( \mathbf{w},\mathbf{v} \right)
\end{align*}
\end{proof}
\begin{thm}\label{2.3.4.13}
$K \subseteq \mathbb{C}$なる体$K$上の$n$次元vector空間$V$上の任意の共役双線形形式$B$が与えられたとき、$\alpha = \left\langle \mathbf{v}_{i} \right\rangle_{i \in \varLambda_{n}}$なるそのvector空間$V$の基底$\alpha$がとられれば、その双線形形式$B$がHermite双線形形式となるならそのときに限り、その双線形形式$B$の基底$\alpha$に関する表現行列$[ B]_{\alpha}$がHermite行列となる、即ち、$[ B]_{\alpha} = [ B]_{\alpha}^{*}$が成り立つ。
\end{thm}
\begin{proof} 定理\ref{2.3.4.12}と同様にして示される。
\end{proof}
\begin{thm}\label{2.3.4.14}
体$K$上の$n$次元vector空間$V$上の任意の双線形形式$B$が与えられたとき、$\alpha = \left\langle \mathbf{v}_{i} \right\rangle_{i \in \varLambda_{n}}$、$\beta = \left\langle \mathbf{w}_{i} \right\rangle_{i \in \varLambda_{n}}$なるそのvector空間$V$の基底たち$\alpha$、$\beta$がとられれば、次式が成り立つ。
\begin{align*}
[ B]_{\alpha} ={}^{t}\left[ I_{V} \right]^{\beta}_{\alpha}[ B]_{\beta}\left[ I_{V} \right]^{\beta}_{\alpha}
\end{align*}
\end{thm}
\begin{proof}
体$K$上の$n$次元vector空間$V$上の任意の双線形形式$B$が与えられたとき、$\alpha = \left\langle \mathbf{v}_{i} \right\rangle_{i \in \varLambda_{n}}$、$\beta = \left\langle \mathbf{w}_{i} \right\rangle_{i \in \varLambda_{n}}$なるそのvector空間$V$の基底たち$\alpha$、$\beta$がとられれば、その基底$\alpha$に関する基底変換における線形同型写像$\varphi_{\alpha}$を用いて、$\forall\mathbf{v},\mathbf{w} \in V$に対し、定理\ref{2.3.4.8}より次のようになる。
\begin{align*}
B\left( \mathbf{v},\mathbf{w} \right) &={}^{t}\varphi_{\beta}^{- 1}\left( \mathbf{v} \right)[ B]_{\beta}\varphi_{\beta}^{- 1}\left( \mathbf{w} \right)\\
&={}^{t}\varphi_{\beta}^{- 1} \circ \varphi_{\alpha} \circ \varphi_{\alpha}^{- 1}\left( \mathbf{v} \right)[ B]_{\beta}\varphi_{\beta}^{- 1} \circ \varphi_{\alpha} \circ \varphi_{\alpha}^{- 1}\left( \mathbf{w} \right)\\
&={}^{t}\varphi_{\beta}^{- 1} \circ I_{V} \circ \varphi_{\alpha}\left( \varphi_{\alpha}^{- 1}\left( \mathbf{v} \right) \right)[ B]_{\beta}\varphi_{\beta}^{- 1} \circ I_{V} \circ \varphi_{\alpha}\left( \varphi_{\alpha}^{- 1}\left( \mathbf{w} \right) \right)\\
&={}^{t}\left( \left[ I_{V} \right]^{\beta}_{\alpha}\varphi_{\alpha}^{- 1}\left( \mathbf{v} \right) \right)[ B]_{\beta}\left( \left[ I_{V} \right]^{\beta}_{\alpha}\varphi_{\alpha}^{- 1}\left( \mathbf{w} \right) \right)\\
&={}^{t}\varphi_{\alpha}^{- 1}\left( \mathbf{v} \right){}^{t}\left[ I_{V} \right]^{\beta}_{\alpha}[ B]_{\beta}\left[ I_{V} \right]^{\beta}_{\alpha}\varphi_{\alpha}^{- 1}\left( \mathbf{w} \right)
\end{align*}
ここで、定理\ref{2.3.4.10}より次式が成り立つ。
\begin{align*}
[ B]_{\alpha} ={}^{t}\left[ I_{V} \right]^{\beta}_{\alpha}[ B]_{\beta}\left[ I_{V} \right]^{\beta}_{\alpha}
\end{align*}
\end{proof}
\begin{thm}\label{2.3.4.15}
$K \subseteq \mathbb{C}$なる体$K$上の$n$次元vector空間$V$上の任意の共役双線形形式$B$が与えられたとき、$\alpha = \left\langle \mathbf{v}_{i} \right\rangle_{i \in \varLambda_{n}}$、$\beta = \left\langle \mathbf{w}_{i} \right\rangle_{i \in \varLambda_{n}}$なるそのvector空間$V$の基底たち$\alpha$、$\beta$がとられれば、次式が成り立つ。
\begin{align*}
[ B]_{\alpha} = {\left[ I_{V} \right]^{\beta}_{\alpha}}^{*}[ B]_{\beta}\left[ I_{V} \right]^{\beta}_{\alpha}
\end{align*}
\end{thm}
\begin{proof} 定理\ref{2.3.4.14}と同様にして示される。
\end{proof}
\begin{thm}\label{2.3.4.16}
体$K$上のvector空間$V$上の任意の対称双線形形式$B$が与えられたとき、$\forall\mathbf{v},\mathbf{w} \in V$に対し、次式が成り立つ。
\begin{align*}
B\left( \mathbf{v},\mathbf{w} \right) = \frac{1}{2}\left( B\left( \mathbf{v} + \mathbf{w},\mathbf{v} + \mathbf{w} \right) - B\left( \mathbf{v},\mathbf{v} \right) - B\left( \mathbf{w},\mathbf{w} \right) \right)
\end{align*}
\end{thm}
\begin{proof}
体$K$上のvector空間$V$上の任意の対称双線形形式$B$が与えられたとき、$\forall\mathbf{v},\mathbf{w} \in V$に対し、次のようになることから、
\begin{align*}
B\left( \mathbf{v} + \mathbf{w},\mathbf{v} + \mathbf{w} \right) &= B\left( \mathbf{v},\mathbf{v} + \mathbf{w} \right) + B\left( \mathbf{w},\mathbf{v} + \mathbf{w} \right)\\
&= B\left( \mathbf{v},\mathbf{v} \right) + B\left( \mathbf{v},\mathbf{w} \right) + B\left( \mathbf{w},\mathbf{v} \right) + B\left( \mathbf{w},\mathbf{w} \right)\\
&= B\left( \mathbf{v},\mathbf{v} \right) + B\left( \mathbf{w},\mathbf{w} \right) + 2B\left( \mathbf{v},\mathbf{w} \right)
\end{align*}
次式が成り立つ。
\begin{align*}
B\left( \mathbf{v},\mathbf{w} \right) = \frac{1}{2}\left( B\left( \mathbf{v} + \mathbf{w},\mathbf{v} + \mathbf{w} \right) - B\left( \mathbf{v},\mathbf{v} \right) - B\left( \mathbf{w},\mathbf{w} \right) \right)
\end{align*}
\end{proof}
\begin{thm}\label{2.3.4.17}
$K \subseteq \mathbb{C}$なる体$K$上のvector空間$V$上の任意のHermite双線形形式$B$が与えられたとき、$\forall\mathbf{v},\mathbf{w} \in V$に対し、次式が成り立つ。
\begin{align*}
\mathrm{Re}{B\left( \mathbf{v},\mathbf{w} \right)} = \frac{1}{2}\left( B\left( \mathbf{v} + \mathbf{w},\mathbf{v} + \mathbf{w} \right) - B\left( \mathbf{v},\mathbf{v} \right) - B\left( \mathbf{w},\mathbf{w} \right) \right)
\end{align*}
\end{thm}
\begin{proof}
$\forall a \in \mathbb{C}$に対し、$a + \overline{a} = 2\mathrm{Re}a$が成り立つことに注意すれば、定理\ref{2.3.4.16}と同様にして示される。
\end{proof}
\subsubsection{2次形式とHermite形式}
\begin{dfn}
体$K$上のvector空間$V$上の任意の対称双線形形式$B$が与えられたとき、次式のように定義される写像$q$をその対称双線形形式$B$から定まるそのvector空間$V$上の2次形式という。
\begin{align*}
q:V \rightarrow K;\mathbf{v} \mapsto B\left( \mathbf{v},\mathbf{v} \right)
\end{align*}
逆に、その対称双線形形式$B$をその2次形式$q$の極形式という。
\end{dfn}
\begin{dfn}
$K \subseteq \mathbb{C}$なる体$K$上のvector空間$V$上の任意のHermite双線形形式$B$が与えられたとき、次式のように定義される写像$q$をそのHermite双線形形式$B$から定まるそのvector空間$V$上のHermite形式という。
\begin{align*}
q:V \rightarrow K;\mathbf{v} \mapsto B\left( \mathbf{v},\mathbf{v} \right)
\end{align*}
逆に、そのHermite双線形形式$B$をそのHermite形式$q$の極形式という。
\end{dfn}\par
もちろん、$K \subseteq \mathbb{R}$のときは2次形式とHermite形式とは一致する。
\begin{thm}\label{2.3.5.1}
体$K$上のvector空間$V$上の任意の2次形式$q$が与えられたとき、ある極形式、即ち、対称双線形形式$B$が一意的に存在して、その2次形式$q$がその極形式$B$から定まるそのvector空間$V$上の2次形式である。また、その極形式$B$は次のように与えられる。
\begin{align*}
B:V \times V \rightarrow K;\left( \mathbf{v},\mathbf{w} \right) \mapsto \frac{1}{2}\left( q\left( \mathbf{v} + \mathbf{w} \right) - q\left( \mathbf{v} \right) - q\left( \mathbf{w} \right) \right)
\end{align*}
\end{thm}
\begin{proof}
$K \subseteq \mathbb{C}$なる体$K$上のvector空間$V$上の任意の2次形式$q$が与えられたとき、定義よりそのvector空間$V$上のある対称双線形形式が存在して、次式が成り立つ。
\begin{align*}
q:V \rightarrow K;\mathbf{v} \mapsto B\left( \mathbf{v},\mathbf{v} \right)
\end{align*}
このとき、定理\ref{2.3.4.16}より$\forall\mathbf{v},\mathbf{w} \in V$に対し、次式が成り立つ。
\begin{align*}
B\left( \mathbf{v},\mathbf{w} \right) &= \frac{1}{2}\left( B\left( \mathbf{v} + \mathbf{w},\mathbf{v} + \mathbf{w} \right) - B\left( \mathbf{v},\mathbf{v} \right) - B\left( \mathbf{w},\mathbf{w} \right) \right)\\
&= \frac{1}{2}\left( q\left( \mathbf{v} + \mathbf{w} \right) - q\left( \mathbf{v} \right) - q\left( \mathbf{w} \right) \right)
\end{align*}
\end{proof}
\begin{thm}\label{2.3.5.2}
$K \subseteq \mathbb{C}$なる体$K$上のvector空間$V$上の任意のHermite形式$q$が与えられたとき、ある極形式、即ち、Hermite双線形形式$B$が一意的に存在して、そのHermite形式$q$がその極形式$B$から定まるそのvector空間$V$上のHermite形式である。また、その極形式$B$は次のように与えられる。
\begin{align*}
B:V \times V \rightarrow K;\left( \mathbf{v},\mathbf{w} \right) \mapsto \frac{1}{2}\left( q\left( \mathbf{v} + \mathbf{w} \right) - q\left( \mathbf{v} \right) - q\left( \mathbf{w} \right) + iq\left( i\mathbf{v} + \mathbf{w} \right) - iq\left( i\mathbf{v} \right) - iq\left( \mathbf{w} \right) \right)
\end{align*}
\end{thm}
\begin{proof}
$K \subseteq \mathbb{C}$なる体$K$上のvector空間$V$上の任意のHermite形式$q$が与えられたとき、定義よりそのvector空間$V$上のあるHermite双線形形式が存在して、次式が成り立つ。
\begin{align*}
q:V \rightarrow K;\mathbf{v} \mapsto B\left( \mathbf{v},\mathbf{v} \right)
\end{align*}
$K \subseteq \mathbb{R}$のとき、定理\ref{2.3.4.17}より$\forall\mathbf{v},\mathbf{w} \in V$に対し、次式が成り立つ。
\begin{align*}
B\left( \mathbf{v},\mathbf{w} \right) &= {\mathrm{Re}}{B\left( \mathbf{v},\mathbf{w} \right)}\\
&= \frac{1}{2}\left( B\left( \mathbf{v} + \mathbf{w},\mathbf{v} + \mathbf{w} \right) - B\left( \mathbf{v},\mathbf{v} \right) - B\left( \mathbf{w},\mathbf{w} \right) \right)\\
&= \frac{1}{2}\left( q\left( \mathbf{v} + \mathbf{w} \right) - q\left( \mathbf{v} \right) - q\left( \mathbf{w} \right) \right)
\end{align*}
一方で、$\mathbb{R} \subset K$のとき、定理\ref{2.3.4.17}より$\forall\mathbf{v},\mathbf{w} \in V$に対し、次式が成り立つかつ、
\begin{align*}
{\mathrm{Re}}{B\left( \mathbf{v},\mathbf{w} \right)} &= \frac{1}{2}\left( B\left( \mathbf{v} + \mathbf{w},\mathbf{v} + \mathbf{w} \right) - B\left( \mathbf{v},\mathbf{v} \right) - B\left( \mathbf{w},\mathbf{w} \right) \right)\\
&= \frac{1}{2}\left( q\left( \mathbf{v} + \mathbf{w} \right) - q\left( \mathbf{v} \right) - q\left( \mathbf{w} \right) \right)
\end{align*}
次のようになる。
\begin{align*}
{\mathrm{Im}}{B\left( \mathbf{v},\mathbf{w} \right)} &= {\mathrm{Re}}\left( - iB\left( \mathbf{v},\mathbf{w} \right) \right)\\
&= {\mathrm{Re}}{\overline{i}B\left( \mathbf{v},\mathbf{w} \right)}\\
&= {\mathrm{Re}}{B\left( i\mathbf{v},\mathbf{w} \right)}\\
&= \frac{1}{2}\left( B\left( i\mathbf{v} + \mathbf{w},i\mathbf{v} + \mathbf{w} \right) - B\left( i\mathbf{v},i\mathbf{v} \right) - B\left( \mathbf{w},\mathbf{w} \right) \right)\\
&= \frac{1}{2}\left( q\left( i\mathbf{v} + \mathbf{w} \right) - q\left( i\mathbf{v} \right) - q\left( \mathbf{w} \right) \right)
\end{align*}
以上より、$\forall\mathbf{v},\mathbf{w} \in V$に対し、次のようにおかれればよい。
\begin{align*}
B\left( \mathbf{v},\mathbf{w} \right) = \frac{1}{2}\left( q\left( \mathbf{v} + \mathbf{w} \right) - q\left( \mathbf{v} \right) - q\left( \mathbf{w} \right) + iq\left( i\mathbf{v} + \mathbf{w} \right) - iq\left( i\mathbf{v} \right) - iq\left( \mathbf{w} \right) \right)
\end{align*}
\end{proof}
%\hypertarget{ux6b21ux5f62ux5f0fux3068hermiteux5f62ux5f0fux306eux8868ux73feux884cux5217}{%
\subsubsection{2次形式とHermite形式の表現行列}%\label{ux6b21ux5f62ux5f0fux3068hermiteux5f62ux5f0fux306eux8868ux73feux884cux5217}}
\begin{dfn}
体$K$上の$n$次元vector空間$V$上の任意の2次形式$q$が与えられたとき、これの極形式を$B$とおくと、$\alpha = \left\langle \mathbf{v}_{i} \right\rangle_{i \in \varLambda_{n}}$なるそのvector空間$V$の基底$\alpha$がとられれば、次式のように定義されるその集合$M_{nn}(K)$における行列$A_{nn}$をその2次形式$q$のその基底$\alpha$に関する表現行列、表現などといい、
\begin{align*}
A_{nn} = \left( B\left( \mathbf{v}_{i},\mathbf{v}_{j} \right) \right)_{(i,j) \in \varLambda_{n}^{2}}
\end{align*}
以下、その行列$A_{nn}$を$[ q]_{\alpha}$とかく。
\end{dfn}
\begin{dfn}
$K \subseteq \mathbb{C}$なる体$K$上の$n$次元vector空間$V$上の任意のHermite形式$q$が与えられたとき、これの極形式を$B$とおくと、$\alpha = \left\langle \mathbf{v}_{i} \right\rangle_{i \in \varLambda_{n}}$なるそのvector空間$V$の基底$\alpha$がとられれば、次式のように定義されるその集合$M_{nn}(K)$における行列$A_{nn}$をそのHermite形式$q$のその基底$\alpha$に関する表現行列、表現などといい、
\begin{align*}
A_{nn} = \left( B\left( \mathbf{v}_{i},\mathbf{v}_{j} \right) \right)_{(i,j) \in \varLambda_{n}^{2}}
\end{align*}
以下、その行列$A_{nn}$を$[ q]_{\alpha}$とかく。
\end{dfn}
\begin{thm}\label{2.3.5.3}
体$K$上の$n$次元vector空間$V$上の任意の2次形式$q$が与えられたとき、$\alpha = \left\langle \mathbf{v}_{i} \right\rangle_{i \in \varLambda_{n}}$なるそのvector空間$V$の基底$\alpha$がとられ、さらに、その2次形式$q$のその基底$\alpha$に関する表現行列$[ q]_{\alpha}$が$[ q]_{\alpha} = \left( q_{ij} \right)_{(i,j) \in \varLambda_{n}^{2}}$と成分表示されれば、これは対称行列であり、$\forall\mathbf{v} \in V$に対し、その体$K$の族々$\left\{ a_{i} \right\}_{i \in \varLambda_{n}}$を用いて次のようにおくと、
\begin{align*}
\mathbf{v} = \sum_{i \in \varLambda_{n}} {a_{i}\mathbf{v}_{i}}
\end{align*}
その基底$\alpha$に関する基底変換における線形同型写像$\varphi_{\alpha}$を用いれば、次式が成り立つ。
\begin{align*}
q\left( \mathbf{v} \right) = \sum_{i,j \in \varLambda_{n}} {a_{i}a_{j}q_{ij}} =^{t}\varphi_{\alpha}^{- 1}\left( \mathbf{v} \right)[ q]_{\alpha}\varphi_{\alpha}^{- 1}\left( \mathbf{v} \right)
\end{align*}\par
逆に、$\forall A_{nn} \in M_{nn}(K)$に対し、その行列$A_{nn}$が対称行列であるとき、次式のような写像$q_{A_{nn}}$を考えよう。
\begin{align*}
q_{A_{nn}}:V \rightarrow K;\mathbf{v} \mapsto^{t}\varphi_{\alpha}^{- 1}\left( \mathbf{v} \right)A_{nn}\varphi_{\alpha}^{- 1}\left( \mathbf{v} \right)
\end{align*}
このとき、その写像$q_{A_{nn}}$はそのvector空間$V$上の2次形式である。
\end{thm}
\begin{proof}
体$K$上の$n$次元vector空間$V$上の任意の2次形式$q$が与えられたとき、$\alpha = \left\langle \mathbf{v}_{i} \right\rangle_{i \in \varLambda_{n}}$なるそのvector空間$V$の基底$\alpha$がとられ、さらに、その2次形式$q$のその基底$\alpha$に関する表現行列$[ q]_{\alpha}$が$[ q]_{\alpha} = \left( q_{ij} \right)_{(i,j) \in \varLambda_{n}^{2}}$と成分表示されれば、$\forall\mathbf{v} \in V$に対し、その体$K$の族々$\left\{ a_{i} \right\}_{i \in \varLambda_{n}}$を用いて次のようにおくと、
\begin{align*}
\mathbf{v} = \sum_{i \in \varLambda_{n}} {a_{i}\mathbf{v}_{i}}
\end{align*}
その基底$\alpha$に関する基底変換における線形同型写像$\varphi_{\alpha}$を用いれば、定義よりその2次形式$q$の極形式$B$を用いて定義より$[ q]_{\alpha} = [ B]_{\alpha}$が成り立つので、定理\ref{2.3.4.8}より次のようになる。
\begin{align*}
q\left( \mathbf{v} \right) &= B\left( \mathbf{v},\mathbf{v} \right)\\
&= \sum_{i,j \in \varLambda_{n}} {a_{i}a_{j}q_{ij}}\\
&=^{t}\varphi_{\alpha}^{- 1}\left( \mathbf{v} \right)[ q]_{\alpha}\varphi_{\alpha}^{- 1}\left( \mathbf{v} \right)
\end{align*}
さらに、定理\ref{2.3.4.12}よりその2次形式$q$のその基底$\alpha$に関する表現行列$[ q]_{\alpha}$は対称行列である。\par
逆に、$\forall A_{nn} \in M_{nn}(K)$に対し、その行列$A_{nn}$が対称行列であるとき、次式のような写像$q_{A_{nn}}$を考えよう。
\begin{align*}
q_{A_{nn}}:V \rightarrow K;\mathbf{v} \mapsto^{t}\varphi_{\alpha}^{- 1}\left( \mathbf{v} \right)A_{nn}\varphi_{\alpha}^{- 1}\left( \mathbf{v} \right)
\end{align*}
これの代わりに次式のような写像$B_{A_{nn}}$が考えられれば、
\begin{align*}
B_{A_{nn}}:V \times V \rightarrow K;\left( \mathbf{v},\mathbf{w} \right) \mapsto^{t}\varphi_{\alpha}^{- 1}\left( \mathbf{v} \right)A_{nn}\varphi_{\alpha}^{- 1}\left( \mathbf{w} \right)
\end{align*}
定理\ref{2.3.4.10}より次のことが成り立つ。
\begin{itemize}
\item
  その写像$B_{A_{nn}}$はそのvector空間$V$上の双線形形式である。
\item
  $A_{nn} = \left( a_{ij} \right)_{(i,j) \in \varLambda_{n}^{2}}$とおかれれば、$\forall i,j \in \varLambda_{n}$に対し、$B_{A_{nn}}\left( \mathbf{v}_{i},\mathbf{v}_{j} \right) = a_{ij}$が成り立つ。
\item
  $\left[ B_{A_{nn}} \right]_{\alpha} = A_{nn}$が成り立つ。
\end{itemize}
さらに、定理\ref{2.3.4.12}よりその双線形形式$B_{A_{nn}}$は対称双線形形式でもある。以上より、定義から直ちにその対称双線形形式$B_{A_{nn}}$から定まる2次形式がまさしく先ほどの写像$q_{A_{nn}}$となる。
\end{proof}
\begin{thm}\label{2.3.5.4}
$K \subseteq \mathbb{C}$なる体$K$上の$n$次元vector空間$V$上の任意のHermite形式$q$が与えられたとき、$\alpha = \left\langle \mathbf{v}_{i} \right\rangle_{i \in \varLambda_{n}}$なるそのvector空間$V$の基底$\alpha$がとられ、さらに、そのHermite形式$q$のその基底$\alpha$に関する表現行列$[ q]_{\alpha}$が$[ q]_{\alpha} = \left( q_{ij} \right)_{(i,j) \in \varLambda_{n}^{2}}$と成分表示されれば、これはHermite行列であり、$\forall\mathbf{v} \in V$に対し、その体$K$の族々$\left\{ a_{i} \right\}_{i \in \varLambda_{n}}$を用いて次のようにおくと、
\begin{align*}
\mathbf{v} = \sum_{i \in \varLambda_{n}} {a_{i}\mathbf{v}_{i}}
\end{align*}
その基底$\alpha$に関する基底変換における線形同型写像$\varphi_{\alpha}$を用いれば、次式が成り立つ。
\begin{align*}
q\left( \mathbf{v} \right) = \sum_{i,j \in \varLambda_{n}} {\overline{a_{i}}a_{j}q_{ij}} =^{t}\overline{\varphi_{\alpha}^{- 1}\left( \mathbf{v} \right)}[ q]_{\alpha}\varphi_{\alpha}^{- 1}\left( \mathbf{v} \right)
\end{align*}\par
逆に、$\forall A_{nn} \in M_{nn}(K)$に対し、その行列$A_{nn}$がHermite行列であるとき、次式のような写像$q_{A_{nn}}$を考えよう。
\begin{align*}
q_{A_{nn}}:V \rightarrow K;\mathbf{v} \mapsto^{t}\overline{\varphi_{\alpha}^{- 1}\left( \mathbf{v} \right)}A_{nn}\varphi_{\alpha}^{- 1}\left( \mathbf{v} \right)
\end{align*}
このとき、その写像$q_{A_{nn}}$はそのvector空間$V$上のHermite形式である。
\end{thm}
\begin{proof} 定理\ref{2.3.5.3}と同様にして示される。
\end{proof}
\begin{thm}\label{2.3.5.5}
体$K$上の$n$次元vector空間$V$が与えられたとき、$\alpha = \left\langle \mathbf{v}_{i} \right\rangle_{i \in \varLambda_{n}}$なるそのvector空間$V$の基底$\alpha$がとられ、さらに、$\forall A_{nn} \in M_{nn}(K)$に対し、その行列$A_{nn}$が対称行列であるとき、その基底$\alpha$に関する基底変換における線形同型写像$\varphi_{\alpha}$を用いた次式のような写像たち$q_{A_{nn}}$、$q_{B_{nn}}$が考えられれば、
\begin{align*}
q_{A_{nn}}&:V \rightarrow K;\mathbf{v} \mapsto^{t}\varphi_{\alpha}^{- 1}\left( \mathbf{v} \right)A_{nn}\varphi_{\alpha}^{- 1}\left( \mathbf{v} \right)\\
q_{B_{nn}}&:V \rightarrow K;\mathbf{v} \mapsto^{t}\varphi_{\alpha}^{- 1}\left( \mathbf{v} \right)B_{nn}\varphi_{\alpha}^{- 1}\left( \mathbf{v} \right)
\end{align*}
$q_{A_{nn}} = q_{B_{nn}}$が成り立つなら、$A_{nn} = B_{nn}$が成り立つ。
\end{thm}
\begin{proof}
体$K$上の$n$次元vector空間$V$が与えられたとき、$\alpha = \left\langle \mathbf{v}_{i} \right\rangle_{i \in \varLambda_{n}}$なるそのvector空間$V$の基底$\alpha$がとられ、さらに、$\forall A_{nn} \in M_{nn}(K)$に対し、その行列$A_{nn}$が対称行列であるとき、その基底$\alpha$に関する基底変換における線形同型写像$\varphi_{\alpha}$を用いた次式のような写像たち$q_{A_{nn}}$、$q_{B_{nn}}$が考えられれば、
\begin{align*}
q_{A_{nn}}&:V \rightarrow K;\mathbf{v} \mapsto^{t}\varphi_{\alpha}^{- 1}\left( \mathbf{v} \right)A_{nn}\varphi_{\alpha}^{- 1}\left( \mathbf{v} \right)\\
q_{B_{nn}}&:V \rightarrow K;\mathbf{v} \mapsto^{t}\varphi_{\alpha}^{- 1}\left( \mathbf{v} \right)B_{nn}\varphi_{\alpha}^{- 1}\left( \mathbf{v} \right)
\end{align*}
これらは定理\ref{2.3.5.3}よりそのvector空間$V$上の2次形式となり、$q_{A_{nn}} = q_{B_{nn}}$が成り立つなら、$\forall\mathbf{v} \in V$に対し、$q_{A_{nn}}\left( \mathbf{v} \right) = q_{B_{nn}}\left( \mathbf{v} \right)$が成り立つことになり、定理\ref{2.3.5.1}よりそれらの2次形式たち$q_{A_{nn}}$、$q_{B_{nn}}$の極形式$B_{A_{nn}}$、$B_{B_{nn}}$について、$\forall\left( \mathbf{v},\mathbf{w} \right) \in V \times V$に対し、次式が成り立つことから、
\begin{align*}
B_{A_{nn}}\left( \mathbf{v},\mathbf{w} \right) &= \frac{1}{2}\left( q_{A_{nn}}\left( \mathbf{v} + \mathbf{w} \right) - q_{A_{nn}}\left( \mathbf{v} \right) - q_{A_{nn}}\left( \mathbf{w} \right) \right)\\
&= \frac{1}{2}\left( q_{B_{nn}}\left( \mathbf{v} + \mathbf{w} \right) - q_{B_{nn}}\left( \mathbf{v} \right) - q_{B_{nn}}\left( \mathbf{w} \right) \right)\\
&= B_{B_{nn}}\left( \mathbf{v},\mathbf{w} \right)
\end{align*}
$B_{A_{nn}} = B_{B_{nn}}$が成り立つ。したがって、定義より次のようになるので、
\begin{align*}
A_{nn} &= \left( B_{A_{nn}}\left( \mathbf{v}_{i},\mathbf{v}_{j} \right) \right)_{(i,j) \in \varLambda_{n}^{2}}\\
&= \left( B_{B_{nn}}\left( \mathbf{v}_{i},\mathbf{v}_{j} \right) \right)_{(i,j) \in \varLambda_{n}^{2}}\\
&= B_{nn}
\end{align*}
$A_{nn} = B_{nn}$が得られる。
\end{proof}
\begin{thm}\label{2.3.5.6}
$K \subseteq \mathbb{C}$なる体$K$上の$n$次元vector空間$V$が与えられたとき、$\alpha = \left\langle \mathbf{v}_{i} \right\rangle_{i \in \varLambda_{n}}$なるそのvector空間$V$の基底$\alpha$がとられ、さらに、$\forall A_{nn} \in M_{nn}(K)$に対し、その行列$A_{nn}$がHermite行列であるとき、その基底$\alpha$に関する基底変換における線形同型写像$\varphi_{\alpha}$を用いた次式のような写像たち$q_{A_{nn}}$、$q_{B_{nn}}$が考えられれば、
\begin{align*}
q_{A_{nn}}&:V \rightarrow K;\mathbf{v} \mapsto^{t}\overline{\varphi_{\alpha}^{- 1}\left( \mathbf{v} \right)}A_{nn}\varphi_{\alpha}^{- 1}\left( \mathbf{v} \right)\\
q_{B_{nn}}&:V \rightarrow K;\mathbf{v} \mapsto^{t}\overline{\varphi_{\alpha}^{- 1}\left( \mathbf{v} \right)}B_{nn}\varphi_{\alpha}^{- 1}\left( \mathbf{v} \right)
\end{align*}
$q_{A_{nn}} = q_{B_{nn}}$が成り立つなら、$A_{nn} = B_{nn}$が成り立つ。
\end{thm}
\begin{proof} 定理\ref{2.3.5.5}と同様にして示される。
\end{proof}
\begin{thm}\label{2.3.5.7}
$K \subseteq \mathbb{C}$なる体$K$上のvector空間$V$が与えられたとき、これ上の任意の双線形形式$B$を用いて次式のように写像$q$が定義されれば、
\begin{align*}
q:V \rightarrow K;\mathbf{v} \mapsto B\left( \mathbf{v},\mathbf{v} \right)
\end{align*}
その写像$q$は2次形式であり、さらに、その極形式$C$は次式のように与えられる。
\begin{align*}
C:V \times V \rightarrow K;\left( \mathbf{v},\mathbf{w} \right) \mapsto \frac{1}{2}\left( B\left( \mathbf{v} + \mathbf{w},\mathbf{v} + \mathbf{w} \right) - B\left( \mathbf{v},\mathbf{v} \right) - B\left( \mathbf{w},\mathbf{w} \right) \right)
\end{align*}
\end{thm}
\begin{proof}
$K \subseteq \mathbb{C}$なる体$K$上のvector空間$V$が与えられたとき、これ上の任意の双線形形式$B$を用いて次式のように写像$q$が定義されよう。
\begin{align*}
q:V \rightarrow K;\mathbf{v} \mapsto B\left( \mathbf{v},\mathbf{v} \right)
\end{align*}
このとき、次式のような写像$C$が考えられれば、
\begin{align*}
C:V \times V \rightarrow K;\left( \mathbf{v},\mathbf{w} \right) \mapsto \frac{1}{2}\left( B\left( \mathbf{v} + \mathbf{w},\mathbf{v} + \mathbf{w} \right) - B\left( \mathbf{v},\mathbf{v} \right) - B\left( \mathbf{w},\mathbf{w} \right) \right)
\end{align*}
$\forall a,b \in K\forall\mathbf{u},\mathbf{v},\mathbf{w} \in V$に対し、次のようになるかつ、
\begin{align*}
C\left( a\mathbf{u} + b\mathbf{v},\mathbf{w} \right) &= \frac{1}{2}\left( B\left( a\mathbf{u} + b\mathbf{v} + \mathbf{w},a\mathbf{u} + b\mathbf{v} + \mathbf{w} \right) \right. \\
&\quad \left.- B\left( a\mathbf{u} + b\mathbf{v},a\mathbf{u} + b\mathbf{v} \right) - B\left( \mathbf{w},\mathbf{w} \right) \right)\\
&= \frac{1}{2}\left( a^{2}B\left( \mathbf{u},\mathbf{u} \right) + abB\left( \mathbf{u},\mathbf{v} \right) + aB\left( \mathbf{u},\mathbf{w} \right) + abB\left( \mathbf{v},\mathbf{u} \right) + b^{2}B\left( \mathbf{v},\mathbf{v} \right) \right. \\
&\quad + bB\left( \mathbf{v},\mathbf{w} \right) + aB\left( \mathbf{w},\mathbf{u} \right) + bB\left( \mathbf{w},\mathbf{v} \right) + B\left( \mathbf{w},\mathbf{w} \right) \\
&\quad \left. - a^{2}B\left( \mathbf{u},\mathbf{u} \right) - abB\left( \mathbf{u},\mathbf{v} \right) - abB\left( \mathbf{v},\mathbf{u} \right) - b^{2}B\left( \mathbf{v},\mathbf{v} \right) - B\left( \mathbf{w},\mathbf{w} \right) \right)\\
&= \frac{1}{2}\left( aB\left( \mathbf{u},\mathbf{w} \right) + bB\left( \mathbf{v},\mathbf{w} \right) + aB\left( \mathbf{w},\mathbf{u} \right) + bB\left( \mathbf{w},\mathbf{v} \right) \right)\\
&= \frac{1}{2}\left( aB\left( \mathbf{u},\mathbf{u} \right) + aB\left( \mathbf{u},\mathbf{w} \right) + aB\left( \mathbf{w},\mathbf{u} \right) + aB\left( \mathbf{w},\mathbf{w} \right) - aB\left( \mathbf{u},\mathbf{u} \right) - aB\left( \mathbf{w},\mathbf{w} \right) \right. \\
&\quad \left. + bB\left( \mathbf{v},\mathbf{v} \right) + bB\left( \mathbf{v},\mathbf{w} \right) + bB\left( \mathbf{w},\mathbf{v} \right) + bB\left( \mathbf{w},\mathbf{w} \right) - bB\left( \mathbf{v},\mathbf{v} \right) - bB\left( \mathbf{w},\mathbf{w} \right) \right)\\
&= \frac{1}{2}\left( aB\left( \mathbf{u} + \mathbf{w},\mathbf{u} + \mathbf{w} \right) - aB\left( \mathbf{u},\mathbf{u} \right) - aB\left( \mathbf{w},\mathbf{w} \right) \right. \\
&\quad \left. + bB\left( \mathbf{v} + \mathbf{w},\mathbf{v} + \mathbf{w} \right) - bB\left( \mathbf{v},\mathbf{v} \right) - bB\left( \mathbf{w},\mathbf{w} \right) \right)\\
&= a \cdot \frac{1}{2}\left( B\left( \mathbf{u} + \mathbf{w},\mathbf{u} + \mathbf{w} \right) - B\left( \mathbf{u},\mathbf{u} \right) - B\left( \mathbf{w},\mathbf{w} \right) \right) \\
&\quad + b \cdot \frac{1}{2}\left( B\left( \mathbf{v} + \mathbf{w},\mathbf{v} + \mathbf{w} \right) - B\left( \mathbf{v},\mathbf{v} \right) - B\left( \mathbf{w},\mathbf{w} \right) \right)\\
&= aC\left( \mathbf{u},\mathbf{w} \right) + bC\left( \mathbf{v},\mathbf{w} \right)
\end{align*}
$\forall a,b \in K\forall\mathbf{u},\mathbf{v},\mathbf{w} \in V$に対し、次のようになるので、
\begin{align*}
C\left( \mathbf{u},a\mathbf{v} + b\mathbf{w} \right) &= \frac{1}{2}\left( B\left( \mathbf{u} + a\mathbf{v} + b\mathbf{w},\mathbf{u} + a\mathbf{v} + b\mathbf{w} \right) - B\left( \mathbf{u},\mathbf{u} \right) - B\left( a\mathbf{v} + b\mathbf{w},a\mathbf{v} + b\mathbf{w} \right) \right)\\
&= \frac{1}{2}\left( B\left( \mathbf{u},\mathbf{u} \right) + aB\left( \mathbf{u},\mathbf{v} \right) + bB\left( \mathbf{u},\mathbf{w} \right) + aB\left( \mathbf{v},\mathbf{u} \right) \right. \\
&\quad + a^{2}B\left( \mathbf{v},\mathbf{v} \right) + abB\left( \mathbf{v},\mathbf{w} \right) + bB\left( \mathbf{w},\mathbf{u} \right) + abB\left( \mathbf{w},\mathbf{v} \right) + b^{2}B\left( \mathbf{w},\mathbf{w} \right) \\
&\quad \left. - B\left( \mathbf{u},\mathbf{u} \right) - a^{2}B\left( \mathbf{v},\mathbf{v} \right) - abB\left( \mathbf{v},\mathbf{w} \right) - abB\left( \mathbf{w},\mathbf{v} \right) - b^{2}B\left( \mathbf{w},\mathbf{w} \right) \right)\\
&= \frac{1}{2}\left( aB\left( \mathbf{u},\mathbf{v} \right) + bB\left( \mathbf{u},\mathbf{w} \right) + aB\left( \mathbf{v},\mathbf{u} \right) + bB\left( \mathbf{w},\mathbf{u} \right) \right)\\
&= \frac{1}{2}\left( aB\left( \mathbf{u},\mathbf{u} \right) + aB\left( \mathbf{u},\mathbf{v} \right) + aB\left( \mathbf{v},\mathbf{u} \right) + aB\left( \mathbf{v},\mathbf{v} \right) - aB\left( \mathbf{u},\mathbf{u} \right) - aB\left( \mathbf{v},\mathbf{v} \right) \right. \\
&\quad \left. + bB\left( \mathbf{u},\mathbf{u} \right) + bB\left( \mathbf{u},\mathbf{w} \right) + bB\left( \mathbf{w},\mathbf{u} \right) + bB\left( \mathbf{w},\mathbf{w} \right) - bB\left( \mathbf{u},\mathbf{u} \right) - bB\left( \mathbf{w},\mathbf{w} \right) \right)\\
&= \frac{1}{2}\left( aB\left( \mathbf{u} + \mathbf{v},\mathbf{u} + \mathbf{v} \right) - aB\left( \mathbf{u},\mathbf{u} \right) - aB\left( \mathbf{v},\mathbf{v} \right) \right. \\
&\quad \left. + bB\left( \mathbf{u} + \mathbf{w},\mathbf{u} + \mathbf{w} \right) - bB\left( \mathbf{u},\mathbf{u} \right) - bB\left( \mathbf{w},\mathbf{w} \right) \right)\\
&= a \cdot \frac{1}{2}\left( B\left( \mathbf{u} + \mathbf{v},\mathbf{u} + \mathbf{v} \right) - B\left( \mathbf{u},\mathbf{u} \right) - B\left( \mathbf{v},\mathbf{v} \right) \right) \\
&\quad + b \cdot \frac{1}{2}\left( B\left( \mathbf{u} + \mathbf{w},\mathbf{u} + \mathbf{w} \right) - B\left( \mathbf{u},\mathbf{u} \right) - B\left( \mathbf{w},\mathbf{w} \right) \right)\\
&= aC\left( \mathbf{u},\mathbf{v} \right) + bC\left( \mathbf{u},\mathbf{w} \right)
\end{align*}
その写像$C$は双線形形式である。\par
さらに、$\forall\mathbf{v},\mathbf{w} \in V$に対し、次のようになることから、
\begin{align*}
C\left( \mathbf{v},\mathbf{w} \right) &= \frac{1}{2}\left( B\left( \mathbf{v} + \mathbf{w},\mathbf{v} + \mathbf{w} \right) - B\left( \mathbf{v},\mathbf{v} \right) - B\left( \mathbf{w},\mathbf{w} \right) \right)\\
&= \frac{1}{2}\left( B\left( \mathbf{w} + \mathbf{v},\mathbf{w} + \mathbf{v} \right) - B\left( \mathbf{w},\mathbf{w} \right) - B\left( \mathbf{v},\mathbf{v} \right) \right)\\
&= C\left( \mathbf{w},\mathbf{v} \right)
\end{align*}
その双線形形式$C$は対称双線形形式でもある。\par
そこで、$\forall\mathbf{v} \in V$に対し、次のようになることから、
\begin{align*}
q\left( \mathbf{v} \right) &= B\left( \mathbf{v},\mathbf{v} \right)\\
&= \frac{1}{2} \cdot 2B\left( \mathbf{v},\mathbf{v} \right)\\
&= \frac{1}{2}\left( 4B\left( \mathbf{v},\mathbf{v} \right) - 2B\left( \mathbf{v},\mathbf{v} \right) \right)\\
&= \frac{1}{2}\left( B\left( \mathbf{v} + \mathbf{v},\mathbf{v} + \mathbf{v} \right) - B\left( \mathbf{v},\mathbf{v} \right) - B\left( \mathbf{v},\mathbf{v} \right) \right) = C\left( \mathbf{v},\mathbf{v} \right)
\end{align*}
その写像$q$はそのvector空間$V$上の2次形式となり、その対称双線形形式$C$がその2次形式の極形式である。
\end{proof}
\begin{thm}\label{2.3.5.8}
$K \subseteq \mathbb{C}$なる体$K$上のvector空間$V$が与えられたとき、これ上の任意の共役双線形形式$B$を用いて次式のように写像$q$が定義されれば、
\begin{align*}
q:V \rightarrow K;\mathbf{v} \mapsto B\left( \mathbf{v},\mathbf{v} \right)
\end{align*}
その写像$q$はHermite形式であり、さらに、その極形式$C$は次式のように与えられる。
\begin{align*}
C:V \times V \rightarrow K;\left( \mathbf{v},\mathbf{w} \right) \mapsto \frac{1}{2}\left( B\left( \mathbf{v} + \mathbf{w},\mathbf{v} + \mathbf{w} \right) - B\left( \mathbf{v},\mathbf{v} \right) - B\left( \mathbf{w},\mathbf{w} \right) + iB\left( i\mathbf{v} + \mathbf{w},i\mathbf{v} + \mathbf{w} \right) - iB\left( i\mathbf{v},i\mathbf{v} \right) - iB\left( \mathbf{w},\mathbf{w} \right) \right)
\end{align*}
\end{thm}
\begin{proof} 定理\ref{2.3.5.7}と同様にして示される。
\end{proof}
%\hypertarget{ux76f4ux4ea4ux57faux5e95}{%
\subsubsection{直交基底}%\label{ux76f4ux4ea4ux57faux5e95}}
\begin{dfn}
体$K$上のvector空間$V$上の対称双線形形式$B$が与えられたとき、$\mathbf{v},\mathbf{w} \in V$なるvectors$\mathbf{v}$、$\mathbf{w}$が$B\left( \mathbf{v},\mathbf{w} \right) = 0$を満たすとき、そのvecotrs$\mathbf{v}$、$\mathbf{w}$はその対称双線形形式$B$に関して直交するという。
\end{dfn}
\begin{dfn}
体$K$上のvector空間$V$上の対称双線形形式$B$が与えられたとき、そのvector空間$V$の基底$\mathcal{B}$に属するどの2つのvectorsがその対称双線形形式$B$に関して直交しているとき、その基底$\mathcal{B}$をそのvector空間$V$のその対称双線形形式$B$に関する直交基底という。
\end{dfn}
\begin{dfn}
$K \subseteq \mathbb{C}$なる体$K$上のvector空間$V$上のHermite双線形形式$B$が与えられたとき、$\mathbf{v},\mathbf{w} \in V$なるvectors$\mathbf{v}$、$\mathbf{w}$が$B\left( \mathbf{v},\mathbf{w} \right) = 0$を満たすとき、そのvecotrs$\mathbf{v}$、$\mathbf{w}$はそのHermite双線形形式$B$に関して直交するという。
\end{dfn}
\begin{dfn}
$K \subseteq \mathbb{C}$なる体$K$上のvector空間$V$上のHermite双線形形式$B$が与えられたとき、そのvector空間$V$の基底$\mathcal{B}$に属するどの2つのvectorsがそのHermite双線形形式$B$に関して直交しているとき、その基底$\mathcal{B}$をそのvector空間$V$のそのHermite双線形形式$B$に関する直交基底という。
\end{dfn}
\begin{thm}\label{2.3.5.9}
体$K$上の$n$次元vector空間$V$上の対称双線形形式$B$が与えられたとき、そのvector空間$V$の基底$\alpha$がその対称双線形形式$B$に関する直交基底であるならそのときに限り、$\alpha = \left\langle \mathbf{v}_{i} \right\rangle_{i \in \varLambda_{n}}$とおかれれば、その基底$\alpha$に関するその対称双線形形式$B$の表現行列$[ B]_{\alpha}$が次式のような対角行列である。
\begin{align*}
[ B]_{\alpha} = \begin{pmatrix}
B\left( \mathbf{v}_{1},\mathbf{v}_{1} \right) & \  & \  & O \\
\  & B\left( \mathbf{v}_{2},\mathbf{v}_{2} \right) & \  & \  \\
\  & \  & \ddots & \  \\
O & \  & \  & B\left( \mathbf{v}_{n},\mathbf{v}_{n} \right) \\
\end{pmatrix}
\end{align*}
\end{thm}
\begin{proof}
体$K$上の$n$次元vector空間$V$上の対称双線形形式$B$が与えられたとき、そのvector空間$V$の基底$\left\langle \mathbf{v}_{i} \right\rangle_{i \in \varLambda_{n}}$がその対称双線形形式$B$に関する直交基底であるなら、$\alpha = \left\langle \mathbf{v}_{i} \right\rangle_{i \in \varLambda_{n}}$とおかれ定理\ref{2.3.4.8}、定理\ref{2.3.4.10}よりその基底$\alpha$に関するその対称双線形形式$B$の表現行列$[ B]_{\alpha}$が$[ B]_{\alpha} = \left( B_{ij} \right)_{(i,j) \in \varLambda_{n}^{2}}$とおかれれば、$\forall i,j \in \varLambda_{n}$に対し、$B\left( \mathbf{v}_{i},\mathbf{v}_{j} \right) = B_{ij}$が成り立つ。そこで、$i \neq j$が成り立つなら、$B\left( \mathbf{v}_{i},\mathbf{v}_{j} \right) = 0$が成り立つかつ、$i = j$が成り立つなら、$B_{ij} = B\left( \mathbf{v}_{i},\mathbf{v}_{j} \right)$が成り立つので、その基底$\alpha$に関するその対称双線形形式$B$の表現行列$[ B]_{\alpha}$が次式のような対角行列である。
\begin{align*}
[ B]_{\alpha} = \begin{pmatrix}
B\left( \mathbf{v}_{1},\mathbf{v}_{1} \right) & \  & \  & O \\
\  & B\left( \mathbf{v}_{2},\mathbf{v}_{2} \right) & \  & \  \\
\  & \  & \ddots & \  \\
O & \  & \  & B\left( \mathbf{v}_{n},\mathbf{v}_{n} \right) \\
\end{pmatrix}
\end{align*}\par
逆に、その基底$\alpha$に関するその対称双線形形式$B$の表現行列$[ B]_{\alpha}$が次式のような対角行列であるなら、
\begin{align*}
[ B]_{\alpha} = \begin{pmatrix}
B\left( \mathbf{v}_{1},\mathbf{v}_{1} \right) & \  & \  & O \\
\  & B\left( \mathbf{v}_{2},\mathbf{v}_{2} \right) & \  & \  \\
\  & \  & \ddots & \  \\
O & \  & \  & B\left( \mathbf{v}_{n},\mathbf{v}_{n} \right) \\
\end{pmatrix}
\end{align*}
定理\ref{2.3.4.8}、定理\ref{2.3.4.10}より$\forall i,j \in \varLambda_{n}$に対し、$B\left( \mathbf{v}_{i},\mathbf{v}_{j} \right) = B_{ij}$が成り立ち、そこで、$i \neq j$が成り立つなら、$B\left( \mathbf{v}_{i},\mathbf{v}_{j} \right) = 0$が成り立つので、そのvector空間$V$の基底$\alpha$がその対称双線形形式$B$に関する直交基底である。
\end{proof}
\begin{thm}\label{2.3.5.10}
$K \subseteq \mathbb{C}$なる体$K$上の$n$次元vector空間$V$上のHermite双線形形式$B$が与えられたとき、そのvector空間$V$の基底$\alpha$がそのHermite双線形形式$B$に関する直交基底であるならそのときに限り、$\alpha = \left\langle \mathbf{v}_{i} \right\rangle_{i \in \varLambda_{n}}$とおかれれば、その基底$\alpha$に関するそのHermite双線形形式$B$の表現行列$[ B]_{\alpha}$が次式のような対角行列である。
\begin{align*}
[ B]_{\alpha} = \begin{pmatrix}
B\left( \mathbf{v}_{1},\mathbf{v}_{1} \right) & \  & \  & O \\
\  & B\left( \mathbf{v}_{2},\mathbf{v}_{2} \right) & \  & \  \\
\  & \  & \ddots & \  \\
O & \  & \  & B\left( \mathbf{v}_{n},\mathbf{v}_{n} \right) \\
\end{pmatrix}
\end{align*}
\end{thm}\par
なお、定理\ref{2.3.4.3}よりその対角行列の対角成分いづれも実数であることに注意しておこう。
\begin{proof} 定理\ref{2.3.5.9}と同様にして示される。
\end{proof}
\begin{thm}\label{2.3.5.11}
体$K$上の$n$次元vector空間$V$が与えられたとき、そのvector空間$V$上の任意の対称双線形形式$B$に関する直交基底が存在する。
\end{thm}
\begin{proof}
体$K$上の$n$次元vector空間$V$が与えられたとき、そのvector空間上の任意の対称双線形形式$B$に対し、$n = 1$のときは明らかである。$n = k$のとき、そのvector空間$V$上の任意の対称双線形形式$B$に関する直交基底が存在すると仮定しよう。$n = k + 1$のとき、$\forall\mathbf{v} \in V$に対し、$B\left( \mathbf{v},\mathbf{v} \right) = 0$が成り立つなら、$\forall\mathbf{v},\mathbf{w} \in V$に対し、定理\ref{2.3.4.16}より次のようになる。
\begin{align*}
B\left( \mathbf{v},\mathbf{w} \right) &= \frac{1}{2}\left( B\left( \mathbf{v} + \mathbf{w},\mathbf{v} + \mathbf{w} \right) - B\left( \mathbf{v},\mathbf{v} \right) - B\left( \mathbf{w},\mathbf{w} \right) \right)\\
&= \frac{1}{2}(0 - 0 - 0) = 0
\end{align*}
これにより、もちろん、そのvector空間$V$の基底がどのようにとられてもこれは直交基底となる。\par
$\exists\mathbf{v} \in V$に対し、$B\left( \mathbf{v},\mathbf{v} \right) \neq 0$が成り立つなら、その元が$\mathbf{v}'$とおかれ、さらに、次式のようにおかれれば、
\begin{align*}
W = \left\{ \mathbf{w} \in V \middle| B\left( \mathbf{v}',\mathbf{w} \right) = 0 \right\}
\end{align*}
もちろん、$\mathbf{0} \in W$が成り立つかつ、$\forall k,l \in K\forall\mathbf{v},\mathbf{w} \in W$に対し、次のようになることから、
\begin{align*}
B\left( \mathbf{v}',k\mathbf{v} + l\mathbf{w} \right) = kB\left( \mathbf{v}',\mathbf{v} \right) + lB\left( \mathbf{v}',\mathbf{w} \right) = 0
\end{align*}
$k\mathbf{v} + l\mathbf{w} \in W$が成り立つので、定理\ref{2.1.1.9}よりその集合$W$はそのvector空間$V$の部分空間である。そこで、$B\left( \mathbf{v}',\mathbf{v}' \right) \neq 0$が成り立つことに注意すれば、$\forall\mathbf{v} \in V$に対し、次のようになる。
\begin{align*}
B\left( \mathbf{v}',\mathbf{v} - \frac{B\left( \mathbf{v}',\mathbf{v} \right)}{B\left( \mathbf{v}',\mathbf{v}' \right)}\mathbf{v}' \right) &= B\left( \mathbf{v}',\mathbf{v} \right) - \frac{B\left( \mathbf{v}',\mathbf{v} \right)}{B\left( \mathbf{v}',\mathbf{v}' \right)}B\left( \mathbf{v}',\mathbf{v}' \right)\\
&= B\left( \mathbf{v}',\mathbf{v} \right) - B\left( \mathbf{v}',\mathbf{v} \right) = 0
\end{align*}
これにより、$\mathbf{v} - \frac{B\left( \mathbf{v}',\mathbf{v} \right)}{B\left( \mathbf{v}',\mathbf{v}' \right)}\mathbf{v}' \in W$が成り立つので、$\mathbf{v} \in {\mathrm{span}}\left\{ \mathbf{v}' \right\} + W$が得られる。これにより、$V \subseteq {\mathrm{span}}\left\{ \mathbf{v}' \right\} + W$が成り立つ。もちろん、${\mathrm{span}}\left\{ \mathbf{v}' \right\} + W \subseteq V$が成り立つので、$V = {\mathrm{span}}\left\{ \mathbf{v}' \right\} + W$が得られる。そこで、$B\left( \mathbf{v}',\mathbf{v}' \right) \neq 0$が成り立つので、$\mathbf{v}' \notin W$が成り立つことから、${\mathrm{span}}\left\{ \mathbf{v}' \right\} \cap W = \left\{ \mathbf{0} \right\}$が得られ、したがって、$V = {\mathrm{span}}\left\{ \mathbf{v}' \right\} \oplus W$が成り立つ。ゆえに、次のようになるので、
\begin{align*}
\dim W &= \dim{{\mathrm{span}}\left\{ \mathbf{v}' \right\}} + \dim W - \dim{{\mathrm{span}}\left\{ \mathbf{v}' \right\}}\\
&= \dim{{\mathrm{span}}\left\{ \mathbf{v}' \right\} \oplus W} - 1\\
&= \dim V - 1\\
&= k + 1 - 1 = k
\end{align*}
仮定よりそのvector空間$W$のその対称双線形形式$B$に関する直交基底$\mathcal{B}$が存在する。しかも、その基底に含まれる任意のvectorsはそのvector$\mathbf{v}'$とその対称双線形形式$B$と直交するので、その基底$\mathcal{B}$にそのvector$\mathbf{v}'$を付け加えたものがそのvector空間$V$の直交基底となる。\par
以上、いかなる場合でも、その$k + 1$次元vector空間$V$上の任意の対称双線形形式$B$に関する直交基底が存在するので、数学的帰納法により$n$次元vector空間$V$上の任意の対称双線形形式$B$に関する直交基底が存在することが示された。
\end{proof}
\begin{thm}\label{2.3.5.12}
$K \subseteq \mathbb{C}$なる体$K$上の$n$次元vector空間$V$が与えられたとき、そのvector空間$V$上の任意のHermite双線形形式$B$に関する直交基底が存在する。
\end{thm}
\begin{proof}
$K \subseteq \mathbb{C}$なる体$K$上の$n$次元vector空間$V$が与えられたとき、そのvector空間上の任意のHermite双線形形式$B$に対し、$n = 1$のときは明らかである。$n = k$のとき、そのvector空間$V$上の任意のHermite双線形形式$B$に関する直交基底が存在すると仮定しよう。$n = k + 1$のとき、$\forall\mathbf{v} \in V$に対し、$B\left( \mathbf{v},\mathbf{v} \right) = 0$が成り立つなら、$\forall\mathbf{v},\mathbf{w} \in V$に対し、定理\ref{2.3.4.17}より次のようになることから、
\begin{align*}
{\mathrm{Re}}{B\left( \mathbf{v},\mathbf{w} \right)} = \frac{1}{2}\left( B\left( \mathbf{v} + \mathbf{w},\mathbf{v} + \mathbf{w} \right) - B\left( \mathbf{v},\mathbf{v} \right) - B\left( \mathbf{w},\mathbf{w} \right) \right)
\end{align*}
次のようになる。
\begin{align*}
B\left( \mathbf{v},\mathbf{w} \right) &= {\mathrm{Re}}{B\left( \mathbf{v},\mathbf{w} \right)} + i{\mathrm{Im}}{B\left( \mathbf{v},\mathbf{w} \right)}\\
&= {\mathrm{Re}}{B\left( \mathbf{v},\mathbf{w} \right)} + i{\mathrm{Re}}\left( - iB\left( \mathbf{v},\mathbf{w} \right) \right)\\
&= {\mathrm{Re}}{B\left( \mathbf{v},\mathbf{w} \right)} + i{\mathrm{Re}}{B\left( i\mathbf{v},\mathbf{w} \right)}\\
&= \frac{1}{2}\left( B\left( \mathbf{v} + \mathbf{w},\mathbf{v} + \mathbf{w} \right) - B\left( \mathbf{v},\mathbf{v} \right) - B\left( \mathbf{w},\mathbf{w} \right) \right) \\
&\quad + \frac{i}{2}\left( B\left( i\mathbf{v} + \mathbf{w},i\mathbf{v} + \mathbf{w} \right) - B\left( i\mathbf{v},i\mathbf{v} \right) - B\left( \mathbf{w},\mathbf{w} \right) \right)\\
&= \frac{1}{2}(0 - 0 - 0) + \frac{i}{2}(0 - 0 - 0) = 0
\end{align*}
これにより、もちろん、そのvector空間$V$の基底がどのようにとられてもこれは直交基底となる。\par
$\exists\mathbf{v} \in V$に対し、$B\left( \mathbf{v},\mathbf{v} \right) \neq 0$が成り立つなら、その元が$\mathbf{v}'$とおかれ、さらに、次式のようにおかれれば、
\begin{align*}
W = \left\{ \mathbf{w} \in V \middle| B\left( \mathbf{v}',\mathbf{w} \right) = 0 \right\}
\end{align*}
もちろん、$\mathbf{0} \in W$が成り立つかつ、$\forall k,l \in K\forall\mathbf{v},\mathbf{w} \in W$に対し、次のようになることから、
\begin{align*}
B\left( \mathbf{v}',k\mathbf{v} + l\mathbf{w} \right) = kB\left( \mathbf{v}',\mathbf{v} \right) + lB\left( \mathbf{v}',\mathbf{w} \right) = 0
\end{align*}
$k\mathbf{v} + l\mathbf{w} \in W$が成り立つので、その集合$W$はそのvector空間$V$の部分空間である。そこで、$B\left( \mathbf{v}',\mathbf{v}' \right) \neq 0$が成り立つことに注意すれば、$\forall\mathbf{v} \in V$に対し、次のようになる。
\begin{align*}
B\left( \mathbf{v}',\mathbf{v} - \frac{B\left( \mathbf{v}',\mathbf{v} \right)}{B\left( \mathbf{v}',\mathbf{v}' \right)}\mathbf{v}' \right) &= B\left( \mathbf{v}',\mathbf{v} \right) - \frac{B\left( \mathbf{v}',\mathbf{v} \right)}{B\left( \mathbf{v}',\mathbf{v}' \right)}B\left( \mathbf{v}',\mathbf{v}' \right)\\
&= B\left( \mathbf{v}',\mathbf{v} \right) - B\left( \mathbf{v}',\mathbf{v} \right) = 0
\end{align*}
これにより、$\mathbf{v} - \frac{B\left( \mathbf{v}',\mathbf{v} \right)}{B\left( \mathbf{v}',\mathbf{v}' \right)}\mathbf{v}' \in W$が成り立つので、$\mathbf{v} \in {\mathrm{span}}\left\{ \mathbf{v}' \right\} + W$が得られる。これにより、$V \subseteq {\mathrm{span}}\left\{ \mathbf{v}' \right\} + W$が成り立つ。もちろん、${\mathrm{span}}\left\{ \mathbf{v}' \right\} + W \subseteq V$が成り立つので、$V = {\mathrm{span}}\left\{ \mathbf{v}' \right\} + W$が得られる。そこで、$B\left( \mathbf{v}',\mathbf{v}' \right) \neq 0$が成り立つので、$\mathbf{v}' \notin W$が成り立つことから、${\mathrm{span}}\left\{ \mathbf{v}' \right\} \cap W = \left\{ \mathbf{0} \right\}$が得られ、したがって、$V = {\mathrm{span}}\left\{ \mathbf{v}' \right\} \oplus W$が成り立つ。ゆえに、次のようになるので、
\begin{align*}
\dim W &= \dim{{\mathrm{span}}\left\{ \mathbf{v}' \right\}} + \dim W - \dim{{\mathrm{span}}\left\{ \mathbf{v}' \right\}}\\
&= \dim{{\mathrm{span}}\left\{ \mathbf{v}' \right\} \oplus W} - 1\\
&= \dim V - 1\\
&= k + 1 - 1 = k
\end{align*}
仮定よりそのvector空間$W$のそのHermite双線形形式$B$に関する直交基底$\mathcal{B}$が存在する。しかも、その基底に含まれる任意のvectorsはそのvector$\mathbf{v}'$とそのHermite双線形形式$B$と直交するので、その基底$\mathcal{B}$にそのvector$\mathbf{v}'$を付け加えたものがそのvector空間$V$の直交基底となる。\par
以上、いかなる場合でも、その$k + 1$次元vector空間$V$上の任意のHermite双線形形式$B$に関する直交基底が存在するので、数学的帰納法により$n$次元vector空間$V$上の任意のHermite双線形形式$B$に関する直交基底が存在することが示された。
\end{proof}
\begin{thm}\label{2.3.5.13}
体$K$上で$\forall A_{nn} \in M_{nn}(K)$に対し、その行列$A_{nn}$が対称行列であるとき、$\exists P \in {\mathrm{GL}}_{n}(K)$に対し、行列$^{t}PA_{nn}P$は対角行列となる。
\end{thm}
\begin{proof}
体$K$上で$\forall A_{nn} \in M_{nn}(K)$に対し、その行列$A_{nn}$が対称行列であるとき、次式のような写像$B_{A_{nn}}$を考えよう。
\begin{align*}
B_{A_{nn}}:K^{n} \times K^{n} \rightarrow K;\left( \mathbf{v},\mathbf{w} \right) \mapsto^{t}\mathbf{v}A_{nn}\mathbf{w}
\end{align*}
このとき、定理\ref{2.3.4.10}より次のことが成り立つ。
\begin{itemize}
\item
  その写像$B_{A_{nn}}$はそのvector空間$K^{n}$上の双線形形式である。
\item
  標準直交基底$\varepsilon$を用いた$\left[ B_{A_{nn}} \right]_{\varepsilon} = A_{nn}$が成り立つ。
\end{itemize}
そこで、定理\ref{2.3.4.12}よりその双線形形式$B_{A_{nn}}$が対称双線形形式となるならそのときに限り、その双線形形式$B_{A_{nn}}$の表現行列$\left[ B_{A_{nn}} \right]_{\varepsilon}$が対称行列となるのであったので、その双線形形式$B_{A_{nn}}$は対称双線形形式となる。そこで、定理\ref{2.3.5.11}よりそのvector空間$K^{n}$上のその対称双線形形式$B_{A_{nn}}$に関する直交基底が存在するので、これが$\mathcal{B}$とおかれると、定理\ref{2.3.5.9}よりその基底$\mathcal{B}$がその対称双線形形式$B_{A_{nn}}$に関する直交基底であるならそのときに限り、$\mathcal{B} =\left\langle \mathbf{v}_{i} \right\rangle_{i \in \varLambda_{n}}$とおかれれば、その基底$\mathcal{B}$に関するその対称双線形形式$B_{A_{nn}}$の表現行列$\left[ B_{A_{nn}} \right]_{\mathcal{B}}$が次式のような対角行列である。
\begin{align*}
\left[ B_{A_{nn}} \right]_{\mathcal{B}} = \begin{pmatrix}
B_{A_{nn}}\left( \mathbf{v}_{1},\mathbf{v}_{1} \right) & \  & \  & O \\
\  & B_{A_{nn}}\left( \mathbf{v}_{2},\mathbf{v}_{2} \right) & \  & \  \\
\  & \  & \ddots & \  \\
O & \  & \  & B_{A_{nn}}\left( \mathbf{v}_{n},\mathbf{v}_{n} \right) \\
\end{pmatrix}
\end{align*}
定理\ref{2.3.4.14}より次式が成り立つので、
\begin{align*}
\left[ B_{A_{nn}} \right]_{\mathcal{B}} =^{t}\left[ I_{K^{n}} \right]^{\varepsilon}_{\mathcal{B}}\left[ B_{A_{nn}} \right]_{\varepsilon}\left[ I_{K^{n}} \right]^{\varepsilon}_{\mathcal{B}}
\end{align*}
その行列$\left[ I_{K^{n}} \right]^{\varepsilon}_{\mathcal{B}}$が$\left[ I_{K^{n}} \right]^{\varepsilon}_{\mathcal{B}} \in {\mathrm{GL}}_{n}(K)$を満たす、即ち、正則行列であることに注意すれば、これを$P$とおくことで、次式が得られる。
\begin{align*}
^{t}PA_{nn}P = \begin{pmatrix}
B_{A_{nn}}\left( \mathbf{v}_{1},\mathbf{v}_{1} \right) & \  & \  & O \\
\  & B_{A_{nn}}\left( \mathbf{v}_{2},\mathbf{v}_{2} \right) & \  & \  \\
\  & \  & \ddots & \  \\
O & \  & \  & B_{A_{nn}}\left( \mathbf{v}_{n},\mathbf{v}_{n} \right) \\
\end{pmatrix}
\end{align*}
よって、$\exists P \in {\mathrm{GL}}_{n}(K)$に対し、その行列$^{t}PA_{nn}P$は対角行列となる。
\end{proof}
\begin{thm}\label{2.3.5.14}
$K \subseteq \mathbb{C}$なる体$K$上で$\forall A_{nn} \in M_{nn}(K)$に対し、その行列$A_{nn}$がHermite行列であるとき、$\exists P \in {\mathrm{GL}}_{n}(K)$に対し、行列$P^{*}A_{nn}P$は対角行列となる。
\end{thm}
\begin{proof} 定理\ref{2.3.5.13}と同様にして示される。
\end{proof}
\begin{thm}\label{2.3.5.15}
体$K$上の$n$次元vector空間$V$が与えられたとき、そのvector空間$V$上の任意の対称双線形形式$B$に関する直交基底$\mathcal{B}$が$\mathcal{B} =\left\langle \mathbf{v}_{i} \right\rangle_{i \in \varLambda_{n}}$とおかれると、$\forall\mathbf{v},\mathbf{w} \in V$に対し、次のようにおかれれば、
\begin{align*}
\mathbf{v} = \sum_{i \in \varLambda_{n}} {a_{i}\mathbf{v}_{i}},\ \ \mathbf{w} = \sum_{i \in \varLambda_{n}} {b_{i}\mathbf{v}_{i}}
\end{align*}
次式が成り立つ。
\begin{align*}
B\left( \mathbf{v},\mathbf{w} \right) = \sum_{i \in \varLambda_{n}} {a_{i}b_{i}B\left( \mathbf{v}_{i},\mathbf{v}_{i} \right)}
\end{align*}
特に、次式が成り立つ。
\begin{align*}
B\left( \mathbf{v},\mathbf{v} \right) = \sum_{i \in \varLambda_{n}} {a_{i}^{2}B\left( \mathbf{v}_{i},\mathbf{v}_{i} \right)}
\end{align*}
\end{thm}
\begin{proof}
体$K$上の$n$次元vector空間$V$が与えられたとき、そのvector空間$V$上の任意の対称双線形形式$B$に関する直交基底$\mathcal{B}$が$\mathcal{B} =\left\langle \mathbf{v}_{i} \right\rangle_{i \in \varLambda_{n}}$とおかれると、$\forall\mathbf{v},\mathbf{w} \in V$に対し、次のようにおかれれば、
\begin{align*}
\mathbf{v} = \sum_{i \in \varLambda_{n}} {a_{i}\mathbf{v}_{i}},\ \ \mathbf{w} = \sum_{i \in \varLambda_{n}} {b_{i}\mathbf{v}_{i}}
\end{align*}
$\forall(i,j) \in \varLambda_{n}^{2}$に対し、$i \neq j$が成り立つなら、$B\left( \mathbf{v}_{i},\mathbf{v}_{j} \right) = 0$が成り立つことに注意すれば、次のようになる。
\begin{align*}
B\left( \mathbf{v},\mathbf{w} \right) &= B\left( \sum_{i \in \varLambda_{n}} {a_{i}\mathbf{v}_{i}},\sum_{i \in \varLambda_{n}} {b_{i}\mathbf{v}_{i}} \right)\\
&= \sum_{i \in \varLambda_{n}} {\sum_{j \in \varLambda_{n}} {a_{i}b_{j}B\left( \mathbf{v}_{i},\mathbf{v}_{j} \right)}}\\
&= \sum_{(i,j) \in \varLambda_{n}^{2} ,i = j } {a_{i}b_{j}B\left( \mathbf{v}_{i},\mathbf{v}_{j} \right)} + \sum_{(i,j) \in \varLambda_{n}^{2} ,i \neq j } {a_{i}b_{j}B\left( \mathbf{v}_{i},\mathbf{v}_{j} \right)}\\
&= \sum_{(i,j) \in \varLambda_{n}^{2} ,i = j } {a_{i}b_{j}B\left( \mathbf{v}_{i},\mathbf{v}_{j} \right)} + \sum_{(i,j) \in \varLambda_{n}^{2} ,i \neq j } {a_{i}b_{j}0}\\
&= \sum_{i \in \varLambda_{n}} {a_{i}b_{i}B\left( \mathbf{v}_{i},\mathbf{v}_{i} \right)}
\end{align*}\par
特に、次式が成り立つ。
\begin{align*}
B\left( \mathbf{v},\mathbf{v} \right) = \sum_{i \in \varLambda_{n}} {a_{i}^{2}B\left( \mathbf{v}_{i},\mathbf{v}_{i} \right)}
\end{align*}
\end{proof}
\begin{thm}\label{2.3.5.16}
$K \subseteq \mathbb{C}$なる体$K$上の$n$次元vector空間$V$が与えられたとき、そのvector空間$V$上の任意のHermite双線形形式$B$に関する直交基底$\mathcal{B}$が$\mathcal{B} =\left\langle \mathbf{v}_{i} \right\rangle_{i \in \varLambda_{n}}$とおかれると、$\forall\mathbf{v},\mathbf{w} \in V$に対し、次のようにおかれれば、
\begin{align*}
\mathbf{v} = \sum_{i \in \varLambda_{n}} {a_{i}\mathbf{v}_{i}},\ \ \mathbf{w} = \sum_{i \in \varLambda_{n}} {b_{i}\mathbf{v}_{i}}
\end{align*}
次式が成り立つ。
\begin{align*}
B\left( \mathbf{v},\mathbf{w} \right) = \sum_{i \in \varLambda_{n}} {\overline{a_{i}}b_{i}B\left( \mathbf{v}_{i},\mathbf{v}_{i} \right)}
\end{align*}
特に、次式が成り立つ。
\begin{align*}
B\left( \mathbf{v},\mathbf{v} \right) = \sum_{i \in \varLambda_{n}} {\left| v_{i} \right|^{2}B\left( \mathbf{v}_{i},\mathbf{v}_{i} \right)}
\end{align*}
\end{thm}
\begin{proof} 定理\ref{2.3.5.15}と同様にして示される。
\end{proof}
\begin{thebibliography}{50}
  \bibitem{1}
  松坂和夫, 線型代数入門, 岩波書店, 1980. 新装版第2刷 p319-333 ISBN978-4-00-029872-8
\end{thebibliography}
\end{document}
