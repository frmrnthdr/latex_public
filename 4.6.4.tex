\documentclass[dvipdfmx]{jsarticle}
\setcounter{section}{6}
\setcounter{subsection}{3}
\usepackage{xr}
\externaldocument{4.5.2}
\externaldocument{4.5.3}
\externaldocument{4.5.5}
\externaldocument{4.5.6}
\externaldocument{4.5.7}
\externaldocument{4.6.1}
\externaldocument{4.6.2}
\externaldocument{4.6.4}
\usepackage{amsmath,amsfonts,amssymb,array,comment,mathtools,url,docmute}
\usepackage{longtable,booktabs,dcolumn,tabularx,mathtools,multirow,colortbl,xcolor}
\usepackage[dvipdfmx]{graphics}
\usepackage{bmpsize}
\usepackage{amsthm}
\usepackage{enumitem}
\setlistdepth{20}
\renewlist{itemize}{itemize}{20}
\setlist[itemize]{label=•}
\renewlist{enumerate}{enumerate}{20}
\setlist[enumerate]{label=\arabic*.}
\setcounter{MaxMatrixCols}{20}
\setcounter{tocdepth}{3}
\newcommand{\rotin}{\text{\rotatebox[origin=c]{90}{$\in $}}}
\renewcommand{\thesection}{第\arabic{section}部}
\renewcommand{\thesubsection}{\arabic{section}.\arabic{subsection}}
\renewcommand{\thesubsubsection}{\arabic{section}.\arabic{subsection}.\arabic{subsubsection}}
\everymath{\displaystyle}
\allowdisplaybreaks[4]
\usepackage{vtable}
\theoremstyle{definition}
\newtheorem{thm}{定理}[subsection]
\newtheorem*{thm*}{定理}
\newtheorem{dfn}{定義}[subsection]
\newtheorem*{dfn*}{定義}
\newtheorem{axs}[dfn]{公理}
\newtheorem*{axs*}{公理}
\renewcommand{\headfont}{\bfseries}
\makeatletter
  \renewcommand{\section}{%
    \@startsection{section}{1}{\z@}%
    {\Cvs}{\Cvs}%
    {\normalfont\huge\headfont\raggedright}}
\makeatother
\makeatletter
  \renewcommand{\subsection}{%
    \@startsection{subsection}{2}{\z@}%
    {0.5\Cvs}{0.5\Cvs}%
    {\normalfont\LARGE\headfont\raggedright}}
\makeatother
\makeatletter
  \renewcommand{\subsubsection}{%
    \@startsection{subsubsection}{3}{\z@}%
    {0.4\Cvs}{0.4\Cvs}%
    {\normalfont\Large\headfont\raggedright}}
\makeatother
\makeatletter
\renewenvironment{proof}[1][\proofname]{\par
  \pushQED{\qed}%
  \normalfont \topsep6\p@\@plus6\p@\relax
  \trivlist
  \item\relax
  {
  #1\@addpunct{.}}\hspace\labelsep\ignorespaces
}{%
  \popQED\endtrivlist\@endpefalse
}
\makeatother
\renewcommand{\proofname}{\textbf{証明}}
\usepackage{tikz,graphics}
\usepackage[dvipdfmx]{hyperref}
\usepackage{pxjahyper}
\hypersetup{
 setpagesize=false,
 bookmarks=true,
 bookmarksdepth=tocdepth,
 bookmarksnumbered=true,
 colorlinks=false,
 pdftitle={},
 pdfsubject={},
 pdfauthor={},
 pdfkeywords={}}
\begin{document}
%\hypertarget{fubini-tonelliux306eux5b9aux7406}{%
\subsection{Fubini-Tonelliの定理}%\label{fubini-tonelliux306eux5b9aux7406}}
%\hypertarget{fubini-tonelliux306eux5b9aux7406ux306eux6e96ux5099}{%
\subsubsection{Fubini-Tonelliの定理の準備}%\label{fubini-tonelliux306eux5b9aux7406ux306eux6e96ux5099}}
\begin{thm}\label{4.6.4.1}
2つの$\sigma$-有限な測度空間たち$(X,\varSigma,\mu)$、$(Y,T,\nu)$が与えられたとき、直積測度空間$(X \times Y,\ \ \varSigma \otimes T,\ \ \mu \otimes \nu)$が定義される。$\forall E \in \varSigma \otimes T\forall x \in X\forall y \in Y$に対し、次のように集合$P_{E,x \in X}$、$P_{E,y \in Y}$が定義されよう\footnote{まぁ、ホントのことをいうと、記法でまずいところはある。さぁ、探してみよう! hintは$X = Y$のときを考えることです。}。
\begin{align*}
P_{E,x \in X} = \left\{ y \in Y \middle| (x,y) \in E \right\},\ \ P_{E,y \in Y} = \left\{ x \in X \middle| (x,y) \in E \right\}
\end{align*}
このとき、次のことが成り立つ。
\begin{itemize}
\item
  $\forall E,F \in \varSigma \otimes T$に対し、$E \subseteq F$が成り立つなら、$\forall x \in X\forall y \in Y$に対し、$P_{F \setminus E,x \in X} = P_{F,x \in X} \setminus P_{E,x \in X}$かつ$P_{F \setminus E,y \in Y} = P_{F,y \in Y} \setminus P_{E,y \in Y}$が成り立つ。
\item
  その集合$\varSigma \otimes T$の元の列$\left( E_{n} \right)_{n \in \mathbb{N}}$が与えられたなら、$\forall x \in X\forall y \in Y$に対し、$P_{\bigcup_{n \in \mathbb{N}} E_{n},x \in X} = \bigcup_{n \in \mathbb{N}} P_{E_{n},x \in X}$かつ$P_{\bigcup_{n \in \mathbb{N}} E_{n},y \in Y} = \bigcup_{n \in \mathbb{N}} P_{E_{n},y \in Y}$が成り立つ。
\item
  その集合$\varSigma \otimes T$の元の列$\left( E_{n} \right)_{n \in \mathbb{N}}$が与えられたなら、$\forall x \in X\forall y \in Y$に対し、$P_{\bigcap_{n \in \mathbb{N}} E_{n},x \in X} = \bigcap_{n \in \mathbb{N}} P_{E_{n},x \in X}$かつ$P_{\bigcap_{n \in \mathbb{N}} E_{n},y \in Y} = \bigcap_{n \in \mathbb{N}} P_{E_{n},y \in Y}$が成り立つ。
\end{itemize}
\end{thm}
\begin{proof}
2つの$\sigma$-有限な測度空間たち$(X,\varSigma,\mu)$、$(Y,T,\nu)$が与えられたとき、直積測度空間$(X \times Y,\ \ \varSigma \otimes T,\ \ \mu \otimes \nu)$が定義される。$\forall E \in \varSigma \otimes T\forall x \in X\forall y \in Y$に対し、次のように集合$P_{E,x \in X}$、$P_{E,y \in Y}$が定義されよう。
\begin{align*}
P_{E,x \in X} = \left\{ y \in Y \middle| (x,y) \in E \right\},\ \ P_{E,y \in Y} = \left\{ x \in X \middle| (x,y) \in E \right\}
\end{align*}
このとき、$\forall E,F \in \varSigma \otimes T$に対し、$E \subseteq F$が成り立つなら、$\forall x \in X$に対し、集合$P_{F \setminus E,x \in X}$が定義されることができて、$\forall y \in Y$に対し、次のようになる。
\begin{align*}
y \in P_{F \setminus E,x \in X} &\Leftrightarrow y \in \left\{ y \in Y \middle| (x,y) \in F \setminus E \right\}\\
&\Leftrightarrow y \in Y \land (x,y) \in F \setminus E\\
&\Leftrightarrow y \in Y \land (x,y) \in F \land (x,y) \notin E\\
&\Leftrightarrow \left( y \in Y \land (x,y) \in F \right) \land \left( y \notin Y \vee (x,y) \notin E \right)\\
&\Leftrightarrow y \in \left\{ y \in Y \middle| (x,y) \in F \right\} \land y \notin \left\{ y \in Y \middle| (x,y) \in E \right\}\\
&\Leftrightarrow y \in P_{F,y \in Y} \land y \notin P_{E,y \in Y}\\
&\Leftrightarrow y \in P_{F,y \in Y} \setminus P_{E,y \in Y}
\end{align*}
よって、$\forall x \in X$に対し、$P_{F \setminus E,x \in X} = P_{F,x \in X} \setminus P_{E,x \in X}$が成り立つ。同様にして、$\forall y \in Y$に対し、$P_{F \setminus E,y \in Y} = P_{F,y \in Y} \setminus P_{E,y \in Y}$が成り立つことが示される。\par
その集合$\varSigma \otimes T$の元の列$\left( E_{n} \right)_{n \in \mathbb{N}}$が与えられたなら、$\forall x \in X$に対し、集合$P_{\bigcup_{n \in \mathbb{N}} E_{n},x \in X}$が定義されることができて、$\forall y \in Y$に対し、次のようになる。
\begin{align*}
y \in P_{\bigcup_{n \in \mathbb{N}} E_{n},x \in X} &\Leftrightarrow y \in \left\{ y \in Y \middle| (x,y) \in \bigcup_{n \in \mathbb{N}} E_{n} \right\}\\
&\Leftrightarrow y \in Y \land (x,y) \in \bigcup_{n \in \mathbb{N}} E_{n}\\
&\Leftrightarrow y \in Y \land \exists n \in \mathbb{N}\left[ (x,y) \in E_{n} \right]\\
&\Leftrightarrow \exists n \in \mathbb{N}\left[ y \in Y \land (x,y) \in E_{n} \right]\\
&\Leftrightarrow \exists n \in \mathbb{N}\left[ y \in \left\{ y \in Y \middle| (x,y) \in E_{n} \right\} \right]\\
&\Leftrightarrow \exists n \in \mathbb{N}\left[ y \in P_{E_{n},x \in X} \right]\\
&\Leftrightarrow y \in \bigcup_{n \in \mathbb{N}} P_{E_{n},x \in X}
\end{align*}
よって、$\forall x \in X$に対し、$P_{\bigcup_{n \in \mathbb{N}} E_{n},x \in X} = \bigcup_{n \in \mathbb{N}} P_{E_{n},x \in X}$が成り立つ。同様にして、$\forall y \in Y$に対し、$P_{\bigcup_{n \in \mathbb{N}} E_{n},y \in Y} = \bigcup_{n \in \mathbb{N}} P_{E_{n},y \in Y}$が成り立つことが示される。\par
その集合$\varSigma \otimes T$の元の列$\left( E_{n} \right)_{n \in \mathbb{N}}$が与えられたなら、$\forall x \in X$に対し、集合$P_{\bigcap_{n \in \mathbb{N}} E_{n},x \in X}$が定義されることができて、$\forall y \in Y$に対し、次のようになる。
\begin{align*}
y \in P_{\bigcap_{n \in \mathbb{N}} E_{n},x \in X} &\Leftrightarrow y \in \left\{ y \in Y \middle| (x,y) \in \bigcap_{n \in \mathbb{N}} E_{n} \right\}\\
&\Leftrightarrow y \in Y \land (x,y) \in \bigcap_{n \in \mathbb{N}} E_{n}\\
&\Leftrightarrow y \in Y \land \forall n \in \mathbb{N}\left[ (x,y) \in E_{n} \right]\\
&\Leftrightarrow \forall n \in \mathbb{N}\left[ y \in Y \land (x,y) \in E_{n} \right]\\
&\Leftrightarrow \forall n \in \mathbb{N}\left[ y \in \left\{ y \in Y \middle| (x,y) \in E_{n} \right\} \right]\\
&\Leftrightarrow \forall n \in \mathbb{N}\left[ y \in P_{E_{n},x \in X} \right]\\
&\Leftrightarrow y \in \bigcap_{n \in \mathbb{N}} P_{E_{n},x \in X}
\end{align*}
よって、$\forall x \in X$に対し、$P_{\bigcap_{n \in \mathbb{N}} E_{n},x \in X} = \bigcap_{n \in \mathbb{N}} P_{E_{n},x \in X}$が成り立つ。\par
同様にして、$\forall y \in Y$に対し、$P_{\bigcap_{n \in \mathbb{N}} E_{n},y \in Y} = \bigcap_{n \in \mathbb{N}} P_{E_{n},y \in Y}$が成り立つことが示される。
\end{proof}
%\hypertarget{fubini-tonelliux306eux5b9aux7406-1}{%
\subsubsection{Fubini-Tonelliの定理}%\label{fubini-tonelliux306eux5b9aux7406-1}}
\begin{thm}\label{4.6.4.2}
2つの$\sigma$-有限な測度空間たち$(X,\varSigma,\mu)$、$(Y,T,\nu)$が与えられたとき、直積測度空間$(X \times Y,\ \ \varSigma \otimes T,\ \ \mu \otimes \nu)$が定義される。$\forall E \in \varSigma \otimes T\forall x \in X\forall y \in Y$に対し、次のように集合$P_{E,x \in X}$、$P_{E,y \in Y}$が定義されよう。
\begin{align*}
P_{E,x \in X} = \left\{ y \in Y \middle| (x,y) \in E \right\},\ \ P_{E,y \in Y} = \left\{ x \in X \middle| (x,y) \in E \right\}
\end{align*}
このとき、次のことが成り立つ。
\begin{itemize}
\item
  $\forall x \in X$に対し、$P_{E,x \in X} \in T$が成り立つかつ、$\forall y \in Y$に対し、$P_{E,y \in Y} \in \varSigma$が成り立つ。
\item
  次のような写像たち$p_{X}$、$p_{Y}$はいづれもそれぞれそれらの測度空間たち$(X,\varSigma,\mu)$、$(Y,T,\nu)$で可測である。
\begin{align*}
p_{X}:X \rightarrow \mathrm{cl}\mathbb{R}^{+};x \mapsto \nu\left( P_{E,x \in X} \right),\ \ p_{Y}:Y \rightarrow \mathrm{cl}\mathbb{R}^{+};y \mapsto \mu\left( P_{E,y \in Y} \right)
\end{align*}
\item
  次式が成り立つ。
\begin{align*}
\int_{X} {p_{X}\mu} = \int_{Y} {p_{Y}\nu} = \mu \otimes \nu(E)
\end{align*}
\item
  $\mu \otimes \nu(E) < \infty$が成り立つなら、$\nu\left( P_{E,x \in X} \right) < \infty\ (X,\varSigma,\mu) \ \text{-} \ \mathrm{a.e.}x \in X$かつ$\mu\left( P_{E,y \in Y} \right) < \infty\ (Y,T,\nu) \ \text{-} \ \mathrm{a.e.}y \in Y$が成り立つ。
\end{itemize}
\end{thm}
\begin{proof}
2つの$\sigma$-有限な測度空間たち$(X,\varSigma,\mu)$、$(Y,T,\nu)$が与えられたとき、直積測度空間$(X \times Y,\ \ \varSigma \otimes T,\ \ \mu \otimes \nu)$が定義される。$\forall E \in \varSigma \otimes T\forall x \in X\forall y \in Y$に対し、次のように集合$P_{E,x \in X}$、$P_{E,y \in Y}$が定義されよう。
\begin{align*}
P_{E,x \in X} = \left\{ y \in Y \middle| (x,y) \in E \right\},\ \ P_{E,y \in Y} = \left\{ x \in X \middle| (x,y) \in E \right\}
\end{align*}\par
単調増加する$\sigma$-加法族たち$\varSigma$、$T$の集合列たち$\left( A_{n} \right)_{n \in \mathbb{N}}$、$\left( B_{n} \right)_{n \in \mathbb{N}}$のうち$\mu\left( A_{n} \right) < \infty$、$\nu\left( Y_{n} \right) < \infty$が成り立つかつ、$\lim_{n \rightarrow \infty}A_{n} = X$、$\lim_{n \rightarrow \infty}B_{n} = Y$が成り立つようなものが存在するのであった。このとき、もちろん、集合列$\left( A_{n} \times B_{n} \right)_{n \in \mathbb{N}}$は単調増加しており、直積$\sigma$-加法族の定義から直ちに、$A_{n} \times B_{n} \in \varSigma \otimes T$が成り立つかつ、直積測度の定義より次のようになる。
\begin{align*}
\mu \otimes \nu\left( A_{n} \times B_{n} \right) = \mu\left( X_{n} \right)\nu\left( B_{n} \right) < \infty
\end{align*}
さらに、もちろん、極限$\lim_{n \rightarrow \infty}{A_{n} \times B_{n}}$も存在して$\lim_{n \rightarrow \infty}{A_{n} \times B_{n}} = X \times Y$が成り立つ。そこで、定理\ref{4.5.2.2}より有限集合である添数集合$\varLambda$を用いて、$\forall i \in \varLambda$に対し、$E_{i} \in \varSigma$かつ$F_{i} \in T$なる直積$E_{i} \times F_{i}$の直和$\bigsqcup_{\scriptsize \begin{matrix}
i \in \varLambda \\
\#\varLambda < \aleph_{0} \\
\end{matrix}} \left( E_{i} \times F_{i} \right)$全体の集合$\mathfrak{K}$もその集合$X \times Y$上の有限加法族であることになる。さらに、定理\ref{4.5.7.7}より$\varSigma\left( \mathfrak{K} \right) = \varSigma \otimes T$が成り立つ。\par
ここで、次のように集合$\mathfrak{M}$をおくと、
\begin{align*}
\mathfrak{M}=\left\{ E \in \mathfrak{P}(X \times Y) \middle| \forall x \in X\left[ P_{E,x \in X} \in T \right] \right\}
\end{align*}
$\forall K \in \mathfrak{K}$に対し、その集合$K$はある有限個の矩形集合の直和$\bigsqcup_{\scriptsize \begin{matrix}
i \in \varLambda \\
\#\varLambda < \aleph_{0} \\
\end{matrix}} \left( E_{i} \times F_{i} \right)$の形で表されることができるので、$\forall x \in X$に対し、集合$P_{K,x \in X}$は、それらの集合たち$F_{i}$のうちいづれかの直和か空集合であるから、$F_{i} \in T$が成り立つことに注意すれば、$P_{K,x \in X} \in T$が成り立つ\footnote{図を描くとわかりやすいです…!}。これにより、$K \in \mathfrak{M}$が得られるので、$\mathfrak{K \subseteq M}$が成り立つ。また、その集合$\mathfrak{M}$の元の列$\left( K_{n} \right)_{n \in \mathbb{N}}$が単調増加する、または、単調減少するなら、その極限$\lim_{n \rightarrow \infty}K_{n}$についても上と同様にして考えれば、$\lim_{n \rightarrow \infty}K_{n}\in \mathfrak{M}$が成り立つので、その集合$\mathfrak{M}$は単調族である。以上、定理\ref{4.5.6.8}より次式が成り立つので、
\begin{align*}
\varSigma \otimes T = \varSigma\left( \mathfrak{K} \right)=\mathfrak{M}\left( \mathfrak{K} \right)\subseteq \mathfrak{M}\left( \mathfrak{M} \right)=\mathfrak{M}
\end{align*}
$\forall E \in \varSigma \otimes T$に対し、$E \in \varSigma \otimes T$が成り立つなら、$\forall x \in X$に対し、$P_{E,x \in X} \in T$が成り立つ。同様にして、$\forall E \in \varSigma \otimes T$に対し、$E \in \varSigma \otimes T$が成り立つなら、$\forall y \in Y$に対し、$P_{E,y \in Y} \in \varSigma$が成り立つことも示される。\par
次のような写像たち$p_{X}$、$p_{Y}$が考えられれば、
\begin{align*}
p_{X}:X \rightarrow \mathrm{cl}\mathbb{R}^{+};x \mapsto \nu\left( P_{E,x \in X} \right),\ \ p_{Y}:Y \rightarrow \mathrm{cl}\mathbb{R}^{+};y \mapsto \mu\left( P_{E,y \in Y} \right)
\end{align*}
単調増加するその集合$\varSigma \otimes T$の元の列$\left( A_{n} \times B_{n} \right)_{n \in \mathbb{N}}$について、次のように集合$\mathfrak{M}_{n}$が定義されれば、
\begin{align*}
\mathfrak{M}_{n} = \left\{ E \in \varSigma \otimes T \middle| X \rightarrow \mathrm{cl}\mathbb{R}^{+};x \mapsto \nu\left( P_{E \cap \left( A_{n} \times B_{n} \right),x \in X} \right) \in \mathcal{M}_{(X,\varSigma,\mu)}^{+} \right\}
\end{align*}
$\forall K \in \mathfrak{K}$に対し、その集合$K$はある有限個の矩形集合の直和$\bigsqcup_{\scriptsize \begin{matrix}
i \in \varLambda \\
\#\varLambda < \aleph_{0} \\
\end{matrix}} \left( E_{i} \times F_{i} \right)$の形で表されることができるので、次式が成り立つことになる。
\begin{align*}
K \cap \left( A_{n} \times B_{n} \right) &= \left( \bigsqcup_{\scriptsize \begin{matrix}
i \in \varLambda \\
\#\varLambda < \aleph_{0} \\
\end{matrix}} \left( E_{i} \times F_{i} \right) \right) \cap \left( A_{n} \times B_{n} \right)\\
&= \bigsqcup_{\scriptsize \begin{matrix}
i \in \varLambda \\
\#\varLambda < \aleph_{0} \\
\end{matrix}} \left( \left( E_{i} \times F_{i} \right) \cap \left( A_{n} \times B_{n} \right) \right)\\
&= \bigsqcup_{\scriptsize \begin{matrix}
i \in \varLambda \\
\#\varLambda < \aleph_{0} \\
\end{matrix}} \left( \left( E_{i} \cap A_{n} \right) \times \left( F_{i} \cap B_{n} \right) \right)
\end{align*}
したがって、$\forall x \in X$に対し、集合$P_{K \cap \left( A_{n} \times B_{n} \right),x \in X}$は、それらの集合たち$F_{i} \cap B_{n}$のうちいづれかの直和か空集合であるから、$P_{K \cap \left( A_{n} \times B_{n} \right),x \in X} = \bigsqcup_{i \in \varLambda_{x} \subseteq \varLambda} \left( F_{i} \cap B_{n} \right)$とおかれると、次のようになる。
\begin{align*}
\nu\left( P_{E \cap \left( A_{n} \times B_{n} \right),x \in X} \right) &= \nu\left( \bigsqcup_{i \in \varLambda_{x} \subseteq \varLambda} \left( F_{i} \cap B_{n} \right) \right)\\
&= \sum_{i \in \varLambda_{x} \subseteq \varLambda} {\nu\left( F_{i} \cap B_{n} \right)}\\
&= \sum_{i \in \varLambda} {\nu\left( F_{i} \cap B_{n} \right)\chi_{\left\{ x \in X|i \in \varLambda_{x} \right\}}}
\end{align*}
これは単関数となっており可測である。ゆえに、$K \in \mathfrak{M}_{n}$が成り立つことになるので、$\mathfrak{K} \subseteq \mathfrak{M}_{n}$が成り立つ。また、その集合$\mathfrak{M}_{n}$の元の列$\left( K_{k} \right)_{k \in \mathbb{N}}$が単調増加する、または、単調減少するなら、その極限$\lim_{k \rightarrow \infty}K_{k}$についても上と同様にして考えれば、$\mathfrak{M}_{k} \subseteq \varSigma \otimes T$が成り立つことにより$\sigma$-加法族は単調族でもあるので、$\lim_{k \rightarrow \infty}K_{k} \in \varSigma \otimes T$が成り立つ。このとき、$\forall x \in X$に対し、元の列$\left( P_{K_{k} \cap \left( A_{n} \times B_{n} \right),x \in X} \right)_{k \in \mathbb{N}}$は単調増加する、または、単調減少するので、次のようになる。
\begin{align*}
\lim_{k \rightarrow \infty}{\nu\left( P_{K_{k} \cap \left( A_{n} \times B_{n} \right),x \in X} \right)} = \nu\left( \lim_{k \rightarrow \infty}P_{K_{k} \cap \left( A_{n} \times B_{n} \right),x \in X} \right)
\end{align*}
あとは上と同様にして考えれば、写像$X \rightarrow \mathrm{cl}\mathbb{R}^{+};x \mapsto \lim_{k \rightarrow \infty}{\nu\left( P_{K_{k} \cap \left( A_{n} \times B_{n} \right),x \in X} \right)}$は単関数となるので、その写像は可測である。これにより、$\lim_{k \rightarrow \infty}K_{k} \in \mathfrak{M}_{n}$が成り立つので、その集合$\mathfrak{M}_{n}$は単調族である。以上、定理\ref{4.5.6.8}より次式が成り立つので、
\begin{align*}
\varSigma \otimes T = \varSigma\left( \mathfrak{K} \right) = \mathfrak{M}\left( \mathfrak{K} \right)\subseteq \mathfrak{M}\left( \mathfrak{M}_{n} \right) = \mathfrak{M}_{n}
\end{align*}
$\forall E \in \varSigma \otimes T$に対し、$E \in \varSigma \otimes T$が成り立つなら、写像$X \rightarrow \mathrm{cl}\mathbb{R}^{+};x \mapsto \nu\left( P_{E \cap \left( A_{n} \times B_{n} \right),x \in X} \right)$が可測である。\par
ところが、$\forall x \in X$に対し、元の列$\left( P_{E \cap \left( A_{n} \times B_{n} \right),x \in X} \right)_{n \in \mathbb{N}}$は単調増加して$\lim_{n \rightarrow \infty}{A_{n} \times B_{n}} = X \times Y$が成り立つことにより次のようになるので、
\begin{align*}
\lim_{n \rightarrow \infty}{E \cap \left( A_{n} \times B_{n} \right)} &= \bigcup_{n \in \mathbb{N}} \left( E \cap \left( A_{n} \times B_{n} \right) \right)\\
&= E \cap \bigcup_{n \in \mathbb{N}} \left( A_{n} \times B_{n} \right)\\
&= E \cap \lim_{n \rightarrow \infty}{A_{n} \times B_{n}}\\
&= E \cap X \times Y = E
\end{align*}
$\lim_{n \rightarrow \infty}P_{E \cap \left( A_{n} \times B_{n} \right),x \in X} = P_{\lim_{n \rightarrow \infty}{E \cap \left( A_{n} \times B_{n} \right)},x \in X} = P_{E,x \in X}$が成り立つ。これにより、次式が成り立つことになり、
\begin{align*}
\nu\left( P_{E,x \in X} \right) &= \nu\left( \lim_{n \rightarrow \infty}P_{E \cap \left( A_{n} \times B_{n} \right),x \in X} \right)\\
&= \lim_{n \rightarrow \infty}{\nu\left( P_{E \cap \left( A_{n} \times B_{n} \right),x \in X} \right)}
\end{align*}
上記の議論により、その写像$X \rightarrow \mathrm{cl}\mathbb{R}^{+};x \mapsto \nu\left( P_{E \cap \left( A_{n} \times B_{n} \right),x \in X} \right)$が可測であるので、写像$X \rightarrow \mathrm{cl}\mathbb{R}^{+};x \mapsto \nu\left( P_{E,x \in X} \right)$、即ち、その写像$p_{X}$も可測である。同様にして、$\forall E \in \varSigma \otimes T$に対し、$E \in \varSigma \otimes T$が成り立つなら、写像$Y \rightarrow \mathrm{cl}\mathbb{R}^{+};y \mapsto \mu\left( P_{E,y \in Y} \right)$、即ち、その写像$p_{Y}$も可測であることも示される。\par
次に、次のように集合$\mathfrak{N}_{n}$が定義されれば、
\begin{align*}
\mathfrak{N}_{n} = \left\{ E \in \varSigma \otimes T;\int_{X} {\begin{matrix}
X & \rightarrow & \mathrm{cl}\mathbb{R}^{+} \\
\text{\rotatebox[origin=c]{90}{$\in$}} & & \text{\rotatebox[origin=c]{90}{$\in$}} \\
x & \mapsto & \nu\left( P_{E \cap \left( A_{n} \times B_{n} \right),x \in X} \right) \\
\end{matrix}\mu} = \mu \otimes \nu\left( E \cap \left( A_{n} \times B_{n} \right) \right) \right\}
\end{align*}
上記の議論と同様にして考えれば、$\mathfrak{K \subseteq}\mathfrak{N}_{n}$が成り立つ。さらに、その集合$\mathfrak{N}_{n}$の元の列$\left( K_{k} \right)_{k \in \mathbb{N}}$が単調増加する、または、単調減少するなら、その極限$\lim_{k \rightarrow \infty}K_{k}$についても上と同様にして考えれば、$\mathfrak{N}_{n} \subseteq \varSigma \otimes T$が成り立つことにより$\sigma$-加法族は単調族でもあるので、$\lim_{k \rightarrow \infty}K_{k} \in \varSigma \otimes T$が成り立つ。このとき、$\forall x \in X$に対し、元の列$\left( P_{K_{k} \cap \left( A_{n} \times B_{n} \right),x \in X} \right)_{k \in \mathbb{N}}$は単調増加する、または、単調減少するので、その元の列$\left( \nu\left( P_{K_{k} \cap \left( A_{n} \times B_{n} \right),x \in X} \right) \right)_{k \in \mathbb{N}}$も単調増加する、または、単調減少する。これにより、単調増加する場合には、定理\ref{4.6.1.26}、即ち、単調収束定理が用いられ、単調減少する場合には、次式が成り立つことから、
\begin{align*}
\int_{X} {\begin{matrix}
X & \rightarrow & \mathrm{cl}\mathbb{R}^{+} \\
\text{\rotatebox[origin=c]{90}{$\in$}} & & \text{\rotatebox[origin=c]{90}{$\in$}} \\
x & \mapsto & \nu\left( P_{E \cap \left( A_{1} \times B_{1} \right),x \in X} \right) \\
\end{matrix}\mu} < \infty
\end{align*}
定理\ref{4.6.2.11}より極限と積分の順序が交換できるようになって次のようになる。
\begin{align*}
\int_{X} {\begin{matrix}
X & \rightarrow & \mathrm{cl}\mathbb{R}^{+} \\
\text{\rotatebox[origin=c]{90}{$\in$}} & & \text{\rotatebox[origin=c]{90}{$\in$}} \\
x & \mapsto & \nu\left( P_{\left( \lim_{k \rightarrow \infty}K_{k} \right) \cap \left( A_{n} \times B_{n} \right),x \in X} \right) \\
\end{matrix}\mu} &= \int_{X} {\begin{matrix}
X & \rightarrow & \mathrm{cl}\mathbb{R}^{+} \\
\text{\rotatebox[origin=c]{90}{$\in$}} & & \text{\rotatebox[origin=c]{90}{$\in$}} \\
x & \mapsto & \nu\left( P_{\lim_{k \rightarrow \infty}\left( K_{k} \cap \left( A_{n} \times B_{n} \right) \right),x \in X} \right) \\
\end{matrix}\mu}\\
&= \int_{X} {\begin{matrix}
X & \rightarrow & \mathrm{cl}\mathbb{R}^{+} \\
\text{\rotatebox[origin=c]{90}{$\in$}} & & \text{\rotatebox[origin=c]{90}{$\in$}} \\
x & \mapsto & \lim_{k \rightarrow \infty}{\nu\left( P_{K_{k} \cap \left( A_{n} \times B_{n} \right),x \in X} \right)} \\
\end{matrix}\mu}\\
&= \int_{X} {\lim_{k \rightarrow \infty}\begin{matrix}
X & \rightarrow & \mathrm{cl}\mathbb{R}^{+} \\
\text{\rotatebox[origin=c]{90}{$\in$}} & & \text{\rotatebox[origin=c]{90}{$\in$}} \\
x & \mapsto & \nu\left( P_{K_{k} \cap \left( A_{n} \times B_{n} \right),x \in X} \right) \\
\end{matrix}\mu}\\
&= \lim_{k \rightarrow \infty}{\int_{X} {\begin{matrix}
X & \rightarrow & \mathrm{cl}\mathbb{R}^{+} \\
\text{\rotatebox[origin=c]{90}{$\in$}} & & \text{\rotatebox[origin=c]{90}{$\in$}} \\
x & \mapsto & \nu\left( P_{K_{k} \cap \left( A_{n} \times B_{n} \right),x \in X} \right) \\
\end{matrix}\mu}}\\
&= \lim_{k \rightarrow \infty}{\mu \otimes \nu\left( K_{k} \cap \left( A_{n} \times B_{n} \right) \right)}\\
&= \mu \otimes \nu\left( \lim_{k \rightarrow \infty}\left( K_{k} \cap \left( A_{n} \times B_{n} \right) \right) \right)\\
&= \mu \otimes \nu\left( \left( \lim_{k \rightarrow \infty}K_{k} \right) \cap \left( A_{n} \times B_{n} \right) \right)
\end{align*}
これにより、$\lim_{k \rightarrow \infty}K_{k} \in \mathfrak{N}_{n}$が成り立つので、その集合$\mathfrak{N}_{n}$は単調族である。以上、定理\ref{4.5.6.8}より次式が成り立つので、
\begin{align*}
\varSigma \otimes T = \varSigma\left( \mathfrak{K} \right) = \mathfrak{M}\left( \mathfrak{K} \right)\subseteq \mathfrak{M}\left( \mathfrak{N}_{n} \right) = \mathfrak{N}_{n}
\end{align*}
$\forall E \in \varSigma \otimes T$に対し、$E \in \varSigma \otimes T$が成り立つなら、次式が成り立つ。
\begin{align*}
\int_{X} {\begin{matrix}
X & \rightarrow & \mathrm{cl}\mathbb{R}^{+} \\
\text{\rotatebox[origin=c]{90}{$\in$}} & & \text{\rotatebox[origin=c]{90}{$\in$}} \\
x & \mapsto & \nu\left( P_{E \cap \left( A_{n} \times B_{n} \right),x \in X} \right) \\
\end{matrix}\mu} = \mu \otimes \nu\left( E \cap \left( A_{n} \times B_{n} \right) \right)
\end{align*}\par
ところが、上記の議論と同様にして、$\nu\left( P_{E,x \in X} \right) = \lim_{n \rightarrow \infty}{\nu\left( P_{E \cap \left( A_{n} \times B_{n} \right),x \in X} \right)}$が成り立つことになり、したがって、定理\ref{4.6.1.26}、即ち、単調収束定理より次のようになる。
\begin{align*}
\int_{X} {p_{X}\mu} &= \int_{X} {\begin{matrix}
X & \rightarrow & \mathrm{cl}\mathbb{R}^{+} \\
\text{\rotatebox[origin=c]{90}{$\in$}} & & \text{\rotatebox[origin=c]{90}{$\in$}} \\
x & \mapsto & \nu\left( P_{E,x \in X} \right) \\
\end{matrix}\mu}\\
&= \int_{X} {\begin{matrix}
X & \rightarrow & \mathrm{cl}\mathbb{R}^{+} \\
\text{\rotatebox[origin=c]{90}{$\in$}} & & \text{\rotatebox[origin=c]{90}{$\in$}} \\
x & \mapsto & \lim_{n \rightarrow \infty}{\nu\left( P_{E \cap \left( A_{n} \times B_{n} \right),x \in X} \right)} \\
\end{matrix}\mu}\\
&= \int_{X} {\lim_{n \rightarrow \infty}\begin{matrix}
X & \rightarrow & \mathrm{cl}\mathbb{R}^{+} \\
\text{\rotatebox[origin=c]{90}{$\in$}} & & \text{\rotatebox[origin=c]{90}{$\in$}} \\
x & \mapsto & \nu\left( P_{E \cap \left( A_{n} \times B_{n} \right),x \in X} \right) \\
\end{matrix}\mu}\\
&= \lim_{n \rightarrow \infty}{\int_{X} {\begin{matrix}
X & \rightarrow & \mathrm{cl}\mathbb{R}^{+} \\
\text{\rotatebox[origin=c]{90}{$\in$}} & & \text{\rotatebox[origin=c]{90}{$\in$}} \\
x & \mapsto & \nu\left( P_{E \cap \left( A_{n} \times B_{n} \right),x \in X} \right) \\
\end{matrix}\mu}}\\
&= \lim_{n \rightarrow \infty}{\mu \otimes \nu\left( E \cap \left( A_{n} \times B_{n} \right) \right)}\\
&= \mu \otimes \nu\left( \lim_{n \rightarrow \infty}\left( E \cap \left( A_{n} \times B_{n} \right) \right) \right)\\
&= \mu \otimes \nu\left( E \cap \lim_{k \rightarrow \infty}\left( A_{n} \times B_{n} \right) \right)\\
&= \mu \otimes \nu\left( E \cap (X \times Y) \right)\\
&= \mu \otimes \nu(E)
\end{align*}
上記の議論により、$\int_{X} {p_{X}\mu} = \mu \otimes \nu(E)$が成り立つ。同様にして、$\forall E \in \varSigma \otimes T$に対し、$E \in \varSigma \otimes T$が成り立つなら、$\int_{Y} {p_{Y}\mu} = \mu \otimes \nu(E)$が成り立つことも示される。\par
したがって、$\mu \otimes \nu(E) < \infty$が成り立つなら、$p_{X}(x) < \infty\ (X,\varSigma,\mu) \ \text{-} \ \mathrm{a.e.}x \in X$かつ$p_{Y}(y) < \infty\ (Y,T,\nu) \ \text{-} \ \mathrm{a.e.}y \in Y$が成り立つ、即ち、$\nu\left( P_{E,x \in X} \right) < \infty\ (X,\varSigma,\mu) \ \text{-} \ \mathrm{a.e.}x \in X$かつ$\mu\left( P_{E,y \in Y} \right) < \infty\ (Y,T,\nu) \ \text{-} \ \mathrm{a.e.}y \in Y$が成り立つ。
\end{proof}
\begin{thm}[Tonelliの定理]\label{4.6.4.3}
2つの$\sigma$-有限な測度空間たち$(X,\varSigma,\mu)$、$(Y,T,\nu)$が与えられたとき、直積測度空間$(X \times Y,\ \ \varSigma \otimes T,\ \ \mu \otimes \nu)$が定義される。$\forall E \in \varSigma \otimes T\forall x \in X\forall y \in Y$に対し、次のように集合$P_{E,x \in X}$、$P_{E,y \in Y}$が定義されよう。
\begin{align*}
P_{E,x \in X} = \left\{ y \in Y \middle| (x,y) \in E \right\},\ \ P_{E,y \in Y} = \left\{ x \in X \middle| (x,y) \in E \right\}
\end{align*}
このとき、$\forall f:X \times Y \rightarrow \mathrm{cl}\mathbb{R}^{+} \in \mathcal{M}_{(X \times Y,\ \ \varSigma \otimes T,\ \ \mu \otimes \nu)}^{+}$に対し、次のことが成り立つ。
\begin{itemize}
\item
  $\forall x \in X$に対し、その写像$f$は、変数$y$の写像とみたとき、その測度空間$(Y,T,\nu)$で可測であるかつ、$\forall y \in Y$に対し、その写像$f$は、変数$x$の写像とみたとき、その測度空間$(X,\varSigma,\mu)$で可測である。
\item
  次のような写像たち$I_{f,X}$、$I_{f,Y}$はいづれもそれぞれそれらの測度空間たち$(X,\varSigma,\mu)$、$(Y,T,\nu)$で可測である。
\begin{align*}
I_{f,X}:X \rightarrow \mathrm{cl}\mathbb{R}^{+};x \mapsto \int_{Y} {f\nu},\ \ I_{f,Y}:Y \rightarrow \mathrm{cl}\mathbb{R}^{+};y \mapsto \int_{X} {f\mu}
\end{align*}
\item
  次式が成り立つ。
\begin{align*}
\int_{X} {\int_{Y} {f\nu}\mu} = \int_{Y} {\int_{X} {f\mu}\nu} = \int_{X \times Y} {f\mu \otimes \nu}
\end{align*}
\end{itemize}
この定理をTonelliの定理という。
\end{thm}
\begin{proof}
2つの$\sigma$-有限な測度空間たち$(X,\varSigma,\mu)$、$(Y,T,\nu)$が与えられたとき、直積測度空間$(X \times Y,\ \ \varSigma \otimes T,\ \ \mu \otimes \nu)$が定義される。$\forall E \in \varSigma \otimes T\forall x \in X\forall y \in Y$に対し、次のように集合$P_{E,x \in X}$、$P_{E,y \in Y}$が定義されよう。
\begin{align*}
P_{E,x \in X} = \left\{ y \in Y \middle| (x,y) \in E \right\},\ \ P_{E,y \in Y} = \left\{ x \in X \middle| (x,y) \in E \right\}
\end{align*}\par
$\forall f:X \times Y \rightarrow \mathrm{cl}\left( \mathbb{R}^{\mathbf{+}} \right) \in \mathcal{M}_{(X \times Y,\ \ \varSigma \otimes T,\ \ \mu \otimes \nu)}^{+}$に対し、定理\ref{4.5.5.18}、即ち、非負可測関数の非負単関数の列による近似により$0 \leq (f)_{n}$なる単関数の列$\left( (f)_{n} \right)_{n \in \mathbb{N}}$が存在して、$\lim_{n \rightarrow \infty}(f)_{n} = f$が成り立つ。そこで、$(f)_{n} = \sum_{i \in \varLambda_{m_{n}}} {\alpha_{n,i}\chi_{E_{n,i}}}$かつ$X = \bigsqcup_{i \in \varLambda_{m_{n}}} E_{n,i}$とおかれれば、$\forall x \in X$に対し、その写像$(f)_{n}$が、変数$y$の写像とみたとき、$(f)_{n} = \sum_{i \in \varLambda_{m_{n}}} {\alpha_{n,i}\chi_{P_{E_{n,i},x \in X}}}$が成り立つことになり、定理\ref{4.6.4.2}より$P_{E_{n,i},x \in X} \in T$が成り立つので、その測度空間$(Y,T,\nu)$で可測である。ゆえに、その写像$f$は、変数$y$の写像とみたとき、その測度空間$(Y,T,\nu)$で可測である。同様にして、$\forall y \in Y$に対し、その写像$f$は、変数$x$の写像とみたとき、その測度空間$(X,\varSigma,\mu)$で可測であることが示される。\par
次のような写像たち$I_{f,X}$、$I_{f,Y}$が考えられれば、
\begin{align*}
I_{f,X}:X \rightarrow \mathrm{cl}\mathbb{R}^{+};x \mapsto \int_{Y} {f\nu},\ \ I_{f,Y}:Y \rightarrow \mathrm{cl}\mathbb{R}^{+};y \mapsto \int_{X} {f\mu}
\end{align*}
$\forall E \in \varSigma \otimes T$に対し、$f = \chi_{E}$とおかれれば、定理\ref{4.5.5.18}、即ち、非負可測関数の非負単関数の列による近似により$0 \leq (f)_{n}$なる単関数の列$\left( (f)_{n} \right)_{n \in \mathbb{N}}$が存在して、$\lim_{n \rightarrow \infty}(f)_{n} = f$が成り立つ。そこで、$(f)_{n} = \sum_{i \in \varLambda_{m_{n}}} {\alpha_{n,i}\chi_{E_{n,i}}}$かつ$X = \bigsqcup_{i \in \varLambda_{m_{n}}} E_{n,i}$とおかれれば、$\forall x \in X$に対し、その写像$(f)_{n}$が、変数$y$の写像とみたとき、$(f)_{n} = \sum_{i \in \varLambda_{m_{n}}} {\alpha_{n,i}\chi_{P_{E_{n,i},x \in X}}}$が成り立つことになり、定理\ref{4.6.4.2}より$P_{E_{n,i},x \in X} \in T$が成り立つので、その測度空間$(Y,T,\nu)$で可測である。ゆえに、$\forall x \in X$に対し、定理\ref{4.6.1.26}、即ち、単調収束定理より次のようになるので、
\begin{align*}
I_{f,X}(x) &= \int_{Y} {f\nu}\\
&= \int_{Y} {\lim_{n \rightarrow \infty}(f)_{n}\nu}\\
&= \lim_{n \rightarrow \infty}{\int_{Y} {(f)_{n}\nu}}\\
&= \lim_{n \rightarrow \infty}{\int_{Y} {\sum_{i \in \varLambda_{m_{n}}} {\alpha_{n,i}\chi_{P_{E_{n,i},x \in X}}}\nu}}\\
&= \lim_{n \rightarrow \infty}{\sum_{i \in \varLambda_{m_{n}}} {\alpha_{n,i}\int_{Y} {\chi_{P_{E_{n,i},x \in X}}\nu}}}\\
&= \lim_{n \rightarrow \infty}{\sum_{i \in \varLambda_{m_{n}}} {\alpha_{n,i}\nu\left( P_{E_{n,i},x \in X} \right)}}
\end{align*}
定理\ref{4.6.4.2}より写像$X \rightarrow \mathrm{cl}\mathbb{R}^{+};x \mapsto \nu\left( P_{E_{n,i},x \in X} \right)$は可測で写像$X \rightarrow \mathrm{cl}\mathbb{R}^{+};x \mapsto \sum_{i \in \varLambda_{m_{n}}} {\alpha_{n,i}\nu\left( P_{E_{n,i},x \in X} \right)}$も可測であるので、その写像$I_{f,X}$、即ち、写像$X \rightarrow \mathrm{cl}\mathbb{R}^{+};x \mapsto \lim_{n \rightarrow \infty}{\sum_{i \in \varLambda_{m_{n}}} {\alpha_{n,i}\nu\left( P_{E_{n,i},x \in X} \right)}}$もその測度空間$(X,\varSigma,\mu)$で可測である。同様にして、その写像$I_{f,Y}$もその測度空間$(Y,T,\nu)$で可測であることが示される。\par
さらに、元の列$\left( \int_{Y} {(f)_{n}\nu} \right)_{n \in \mathbb{N}}$も単調増加していることに注意すれば、上と同様にして定理\ref{4.6.1.26}、即ち、単調収束定理と定理\ref{4.6.4.2}より次式が成り立つことから、
\begin{align*}
\int_{X} {\nu\left( \chi_{P_{E_{n,i},x \in X}} \right)\mu} = \mu \otimes \nu\left( E_{in} \right)
\end{align*}
次のようになる。
\begin{align*}
\int_{X} {\int_{Y} {f\nu}\mu} &= \int_{X} {\int_{Y} {\lim_{n \rightarrow \infty}(f)_{n}\nu}\mu}\\
&= \int_{X} {\lim_{n \rightarrow \infty}{\int_{Y} {(f)_{n}\nu}}\mu}\\
&= \lim_{n \rightarrow \infty}{\int_{X} {\int_{Y} {(f)_{n}\nu}\mu}}\\
&= \lim_{n \rightarrow \infty}{\int_{X} {\int_{Y} {\sum_{i \in \varLambda_{m_{n}}} {\alpha_{n,i}\chi_{P_{E_{n,i},x \in X}}}\nu}\mu}}\\
&= \lim_{n \rightarrow \infty}{\sum_{i \in \varLambda_{m_{n}}} {\alpha_{n,i}\int_{X} {\int_{Y} {\chi_{P_{E_{n,i},x \in X}}\nu}\mu}}}\\
&= \lim_{n \rightarrow \infty}{\sum_{i \in \varLambda_{m_{n}}} {\alpha_{n,i}\int_{X} {\nu\left( \chi_{P_{E_{n,i},x \in X}} \right)\mu}}}\\
&= \lim_{n \rightarrow \infty}{\sum_{i \in \varLambda_{m_{n}}} {\alpha_{n,i}\mu \otimes \nu\left( E_{n,i} \right)}}\\
&= \lim_{n \rightarrow \infty}{\sum_{i \in \varLambda_{m_{n}}} {\alpha_{n,i}\int_{X \times Y} {\chi_{E_{n,i}}\mu \otimes \nu}}}\\
&= \lim_{n \rightarrow \infty}{\int_{X \times Y} {\sum_{i \in \varLambda_{m_{n}}} {\alpha_{n,i}\chi_{E_{n,i}}}\mu \otimes \nu}}\\
&= \lim_{n \rightarrow \infty}{\int_{X \times Y} {(f)_{n}\mu \otimes \nu}}\\
&= \int_{X \times Y} {\lim_{n \rightarrow \infty}(f)_{n}\mu \otimes \nu}\\
&= \int_{X \times Y} {f\mu \otimes \nu}
\end{align*}
同様にして、次式が成り立つことが示される。
\begin{align*}
\int_{Y} {\int_{X} {f\mu}\nu} = \int_{X \times Y} {f\mu \otimes \nu}
\end{align*}
\end{proof}
\begin{thm}[Fubiniの定理]\label{4.6.4.4}
2つの$\sigma$-有限な測度空間たち$(X,\varSigma,\mu)$、$(Y,T,\nu)$が与えられたとき、直積測度空間$(X \times Y, \varSigma \otimes T, \mu \otimes \nu)$が定義される。$\forall E \in \varSigma \otimes T\forall x \in X\forall y \in Y$に対し、次のように集合$P_{E,x \in X}$、$P_{E,y \in Y}$が定義されよう。
\begin{align*}
P_{E,x \in X} = \left\{ y \in Y \middle| (x,y) \in E \right\}, P_{E,y \in Y} = \left\{ x \in X \middle| (x,y) \in E \right\}
\end{align*}
$\forall f:X \times Y \rightarrow{}^{*}\mathbb{R} \in \mathcal{M}_{(X \times Y, \varSigma \otimes T, \mu \otimes \nu)}$に対し、その写像$f$が定積分可能であるなら、次のことが成り立つ。
\begin{itemize}
\item
  $(X,\varSigma,\mu) \ \text{-} \ \mathrm{a.e.}x \in X$に対し、その写像$f$は、変数$y$の写像とみたとき、その集合$Y$で定積分可能であるかつ、$(Y,T,\nu) \ \text{-} \ \mathrm{a.e.}y \in Y$に対し、その写像$f$は、変数$x$の写像とみたとき、その集合$X$で定積分可能である。
\item
  次のような写像たち$I_{f,X}$、$I_{f,Y}$はいづれもそれぞれそれらの集合たち$X$、$Y$で定積分可能である。
\begin{align*}
I_{f,X}:X \rightarrow \mathrm{cl}\mathbb{R}^{+};x \mapsto \int_{Y} {f\nu},\ \ I_{f,Y}:Y \rightarrow \mathrm{cl}\mathbb{R}^{+};y \mapsto \int_{X} {f\mu}
\end{align*}
\item
  次式が成り立つ。
\begin{align*}
\int_{X} {\int_{Y} {f\nu}\mu} = \int_{Y} {\int_{X} {f\mu}\nu} = \int_{X \times Y} {f\mu \otimes \nu}
\end{align*}
\end{itemize}
この定理をFubiniの定理という。
\end{thm}
\begin{proof}
2つの$\sigma$-有限な測度空間たち$(X,\varSigma,\mu)$、$(Y,T,\nu)$が与えられたとき、直積測度空間$(X \times Y,\ \ \varSigma \otimes T,\ \ \mu \otimes \nu)$が定義される。$\forall E \in \varSigma \otimes T\forall x \in X\forall y \in Y$に対し、次のように集合$P_{E,x \in X}$、$P_{E,y \in Y}$が定義されよう。
\begin{align*}
P_{E,x \in X} = \left\{ y \in Y \middle| (x,y) \in E \right\},\ \ P_{E,y \in Y} = \left\{ x \in X \middle| (x,y) \in E \right\}
\end{align*}\par
$\forall f:X \times Y \rightarrow{}^{*}\mathbb{R} \in \mathcal{M}_{(X \times Y,\ \ \varSigma \otimes T,\ \ \mu \otimes \nu)}$に対し、その写像$f$が定積分可能であるなら、$f = (f)_{+} - (f)_{-}$が成り立つので、定理\ref{4.6.4.3}、即ち、Tonelliの定理より次のことが成り立つ。
\begin{itemize}
\item
  $\forall x \in X$に対し、その写像たち$(f)_{+}$、$(f)_{-}$は、変数$y$の写像とみたとき、その測度空間$(Y,T,\nu)$で可測であるかつ、$\forall y \in Y$に対し、その写像たち$(f)_{+}$、$(f)_{-}$は、変数$x$の写像とみたとき、その測度空間$(X,\varSigma,\mu)$で可測である。
\item
  次のような写像たち$I_{(f)_{+},X}$、$I_{(f)_{+},Y}$、$I_{(f)_{-},X}$、$I_{(f)_{-},Y}$はいづれもそれぞれそれらの測度空間たち$(X,\varSigma,\mu)$、$(Y,T,\nu)$で可測である。
\begin{align*}
I_{(f)_{+},X}&:X \rightarrow \mathrm{cl}\mathbb{R}^{+};x \mapsto \int_{Y} {(f)_{+}\nu},\ \ I_{(f)_{+},Y}:Y \rightarrow \mathrm{cl}\mathbb{R}^{+};y \mapsto \int_{X} {(f)_{+}\mu},\\
I_{(f)_{-},X}&:X \rightarrow \mathrm{cl}\mathbb{R}^{+};x \mapsto \int_{Y} {(f)_{-}\nu},\ \ I_{(f)_{-},Y}:Y \rightarrow \mathrm{cl}\mathbb{R}^{+};y \mapsto \int_{X} {(f)_{-}\mu}
\end{align*}
\item
  次式が成り立つ。
\begin{align*}
\int_{X} {\int_{Y} {(f)_{+}\nu}\mu} &= \int_{Y} {\int_{X} {(f)_{+}\mu}\nu} = \int_{X \times Y} {(f)_{+}\mu \otimes \nu} < \infty\\
\int_{X} {\int_{Y} {(f)_{-}\nu}\mu} &= \int_{Y} {\int_{X} {(f)_{-}\mu}\nu} = \int_{X \times Y} {(f)_{-}\mu \otimes \nu} < \infty
\end{align*}
\end{itemize}
したがって、$(X,\varSigma,\mu) \ \text{-} \ \mathrm{a.e.}x \in X$に対し、その写像$f$は、変数$y$の写像とみたとき、その集合$Y$で定積分可能であるかつ、$(Y,T,\nu) \ \text{-} \ \mathrm{a.e.}y \in Y$に対し、その写像$f$は、変数$x$の写像とみたとき、その集合$X$で定積分可能である。\par
さらに、積分の単調性により次式が成り立つので、
\begin{align*}
\int_{X} {\int_{Y} {(f)_{+}\nu}\mu} &= \int_{X} {\int_{Y} {\begin{matrix}
X & \overset{(f)_{+}}{\rightarrow} & \mathrm{cl}\mathbb{R}^{+} \\
\text{\rotatebox[origin=c]{90}{$\in$}} & & \text{\rotatebox[origin=c]{90}{$\in$}} \\
x & \mapsto & \max\left\{ f(x),0 \right\} \\
\end{matrix}\nu}\mu}\\
&= \int_{X} {\max\left\{ \int_{Y} {\begin{matrix}
X & \rightarrow & \mathrm{cl}\mathbb{R}^{+} \\
\text{\rotatebox[origin=c]{90}{$\in$}} & & \text{\rotatebox[origin=c]{90}{$\in$}} \\
x & \mapsto & f(x) \\
\end{matrix}\nu},\int_{Y} {\begin{matrix}
X & \rightarrow & \mathrm{cl}\mathbb{R}^{+} \\
\text{\rotatebox[origin=c]{90}{$\in$}} & & \text{\rotatebox[origin=c]{90}{$\in$}} \\
x & \mapsto & 0 \\
\end{matrix}\nu} \right\}\mu}\\
&= \int_{X} {\max\left\{ \begin{matrix}
X & \rightarrow & \mathrm{cl}\mathbb{R}^{+} \\
\text{\rotatebox[origin=c]{90}{$\in$}} & & \text{\rotatebox[origin=c]{90}{$\in$}} \\
x & \mapsto & \int_{Y} {f\nu} \\
\end{matrix},\begin{matrix}
X & \rightarrow & \mathrm{cl}\mathbb{R}^{+} \\
\text{\rotatebox[origin=c]{90}{$\in$}} & & \text{\rotatebox[origin=c]{90}{$\in$}} \\
x & \mapsto & \int_{Y} {0\nu} \\
\end{matrix} \right\}\mu}\\
&= \int_{X} {\max\left\{ \begin{matrix}
X & \overset{I_{f,X}}{\rightarrow} & \mathrm{cl}\mathbb{R}^{+} \\
\text{\rotatebox[origin=c]{90}{$\in$}} & & \text{\rotatebox[origin=c]{90}{$\in$}} \\
x & \mapsto & \int_{Y} {f\nu} \\
\end{matrix},\begin{matrix}
X & \overset{0}{\rightarrow} & \mathrm{cl}\mathbb{R}^{+} \\
\text{\rotatebox[origin=c]{90}{$\in$}} & & \text{\rotatebox[origin=c]{90}{$\in$}} \\
x & \mapsto & 0 \\
\end{matrix} \right\}\mu}\\
&= \int_{X} {\max\left\{ I_{f,X},0 \right\}\mu}\\
&= \int_{X} {\left( I_{f,X} \right)_{+}\mu} < \infty\\
\int_{X} {\int_{Y} {(f)_{-}\nu}\mu} &= \int_{X} {\int_{Y} {\begin{matrix}
X & \overset{(f)_{-}}{\rightarrow} & \mathrm{cl}\mathbb{R}^{+} \\
\text{\rotatebox[origin=c]{90}{$\in$}} & & \text{\rotatebox[origin=c]{90}{$\in$}} \\
x & \mapsto & \max\left\{ - f(x),0 \right\} \\
\end{matrix}\nu}\mu}\\
&= \int_{X} {\max\left\{ \int_{Y} {\begin{matrix}
X & \rightarrow & \mathrm{cl}\mathbb{R}^{+} \\
\text{\rotatebox[origin=c]{90}{$\in$}} & & \text{\rotatebox[origin=c]{90}{$\in$}} \\
x & \mapsto & - f(x) \\
\end{matrix}\nu},\int_{Y} {\begin{matrix}
X & \rightarrow & \mathrm{cl}\mathbb{R}^{+} \\
\text{\rotatebox[origin=c]{90}{$\in$}} & & \text{\rotatebox[origin=c]{90}{$\in$}} \\
x & \mapsto & 0 \\
\end{matrix}\nu} \right\}\mu}\\
&= \int_{X} {\max\left\{ \begin{matrix}
X & \rightarrow & \mathrm{cl}\mathbb{R}^{+} \\
\text{\rotatebox[origin=c]{90}{$\in$}} & & \text{\rotatebox[origin=c]{90}{$\in$}} \\
x & \mapsto & \int_{Y} {( - f)\nu} \\
\end{matrix},\begin{matrix}
X & \rightarrow & \mathrm{cl}\mathbb{R}^{+} \\
\text{\rotatebox[origin=c]{90}{$\in$}} & & \text{\rotatebox[origin=c]{90}{$\in$}} \\
x & \mapsto & \int_{Y} {0\nu} \\
\end{matrix} \right\}\mu}\\
&= \int_{X} {\max\left\{ \begin{matrix}
X & \overset{- I_{f,X}}{\rightarrow} & \mathrm{cl}\mathbb{R}^{+} \\
\text{\rotatebox[origin=c]{90}{$\in$}} & & \text{\rotatebox[origin=c]{90}{$\in$}} \\
x & \mapsto & - \int_{Y} {f\nu} \\
\end{matrix},\begin{matrix}
X & \overset{0}{\rightarrow} & \mathrm{cl}\mathbb{R}^{+} \\
\text{\rotatebox[origin=c]{90}{$\in$}} & & \text{\rotatebox[origin=c]{90}{$\in$}} \\
x & \mapsto & 0 \\
\end{matrix} \right\}\mu}\\
&= \int_{X} {\max\left\{ - I_{f,X},0 \right\}\mu}\\
&= \int_{X} {\left( I_{f,X} \right)_{-}\mu} < \infty
\end{align*}
次のような写像たち$I_{f,X}$、$I_{f,Y}$はいづれもそれぞれそれらの集合たち$X$、$Y$で定積分可能である。
\begin{align*}
I_{f,X}:X \rightarrow \mathrm{cl}\mathbb{R}^{+};x \mapsto \int_{Y} {f\nu},\ \ I_{f,Y}:Y \rightarrow \mathrm{cl}\mathbb{R}^{+};y \mapsto \int_{X} {f\mu}
\end{align*}\par
このとき、定理\ref{4.6.4.3}より次式が成り立つ。
\begin{align*}
\int_{X} {\int_{Y} {f\nu}\mu} &= \int_{X} {\int_{Y} {\left( (f)_{+} - (f)_{-} \right)\nu}\mu}\\
&= \int_{X} {\int_{Y} {(f)_{+}\nu}\mu} - \int_{X} {\int_{Y} {(f)_{-}\nu}\mu}\\
&= \int_{X \times Y} {(f)_{+}\mu \otimes \nu} - \int_{X \times Y} {(f)_{-}\mu \otimes \nu}\\
&= \int_{X \times Y} {\left( (f)_{+} - (f)_{-} \right)\mu \otimes \nu}\\
&= \int_{X \times Y} {f\mu \otimes \nu}\\
\int_{Y} {\int_{X} {f\mu}\nu} &= \int_{Y} {\int_{X} {\left( (f)_{+} - (f)_{-} \right)\mu}\nu}\\
&= \int_{Y} {\int_{X} {(f)_{+}\mu}\nu} - \int_{Y} {\int_{X} {(f)_{-}\mu}\nu}\\
&= \int_{X \times Y} {(f)_{+}\mu \otimes \nu} - \int_{X \times Y} {(f)_{-}\mu \otimes \nu}\\
&= \int_{X \times Y} {\left( (f)_{+} - (f)_{-} \right)\mu \otimes \nu}\\
&= \int_{X \times Y} {f\mu \otimes \nu}
\end{align*}
\end{proof}
\begin{thm}[Fubini-Tonelliの定理]\label{4.6.4.5}
2つの$\sigma$-有限な測度空間たち$(X,\varSigma,\mu)$、$(Y,T,\nu)$が与えられたとき、直積測度空間$(X \times Y,\varSigma \otimes T,\mu \otimes \nu)$が定義される。$\forall E \in \varSigma \otimes T\forall x \in X\forall y \in Y$に対し、次のように集合$P_{E,x \in X}$、$P_{E,y \in Y}$が定義されよう。
\begin{align*}
P_{E,x \in X} = \left\{ y \in Y \middle| (x,y) \in E \right\},\ \ P_{E,y \in Y} = \left\{ x \in X \middle| (x,y) \in E \right\}
\end{align*}
$\forall f:X \times Y \rightarrow{}^{*}\mathbb{R} \in \mathcal{M}_{(X \times Y,\varSigma \otimes T,\mu \otimes \nu)}$に対し、次のうちどれか1つでも成り立つなら、
\begin{align*}
\int_{X} {\int_{Y} {|f|\nu}\mu} < \infty,\int_{Y} {\int_{X} {|f|\mu}\nu} < \infty,\int_{X \times Y} {|f|\mu \otimes \nu} < \infty
\end{align*}
次のことが成り立つ。
\begin{itemize}
\item
  次式が成り立つ。
\begin{align*}
\int_{X} {\int_{Y} {|f|\nu}\mu} = \int_{Y} {\int_{X} {|f|\mu}\nu} = \int_{X \times Y} {|f|\mu \otimes \nu}
\end{align*}
\item
  $(X,\varSigma,\mu) \ \text{-} \ \mathrm{a.e.}x \in X$に対し、その写像$f$は、変数$y$の写像とみたとき、その集合$Y$で定積分可能であるかつ、$(Y,T,\nu) \ \text{-} \ \mathrm{a.e.}y \in Y$に対し、その写像$f$は、変数$x$の写像とみたとき、その集合$X$で定積分可能である。
\item
  次のような写像たち$I_{f,X}$、$I_{f,Y}$はいづれもそれぞれそれらの集合たち$X$、$Y$で定積分可能である。
\begin{align*}
I_{f,X}:X \rightarrow \mathrm{cl}\mathbb{R}^{+};x \mapsto \int_{Y} {f\nu},\ \ I_{f,Y}:Y \rightarrow \mathrm{cl}\mathbb{R}^{+};y \mapsto \int_{X} {f\mu}
\end{align*}
\item
  次式が成り立つ。
\begin{align*}
\int_{X} {\int_{Y} {f\nu}\mu} = \int_{Y} {\int_{X} {f\mu}\nu} = \int_{X \times Y} {f\mu \otimes \nu}
\end{align*}
\end{itemize}
この定理をFubini-Tonelliの定理という。
\end{thm}
\begin{proof}
2つの$\sigma$-有限な測度空間たち$(X,\varSigma,\mu)$、$(Y,T,\nu)$が与えられたとき、直積測度空間$(X \times Y,\ \ \varSigma \otimes T,\ \ \mu \otimes \nu)$が定義される。$\forall E \in \varSigma \otimes T\forall x \in X\forall y \in Y$に対し、次のように集合$P_{E,x \in X}$、$P_{E,y \in Y}$が定義されよう。
\begin{align*}
P_{E,x \in X} = \left\{ y \in Y \middle| (x,y) \in E \right\},\ \ P_{E,y \in Y} = \left\{ x \in X \middle| (x,y) \in E \right\}
\end{align*}\par
また、$\forall f:X \times Y \rightarrow{}^{*}\mathbb{R} \in \mathcal{M}_{(X \times Y,\ \ \varSigma \otimes T,\ \ \mu \otimes \nu)}$に対し、もちろん、$|f|:X \times Y \rightarrow \mathrm{cl}\mathbb{R}^{+} \in \mathcal{M}_{(X \times Y,\ \ \varSigma \otimes T,\ \ \mu \otimes \nu)}^{+}$が成り立つので、次のうちどれか1つでも成り立つなら、
\begin{align*}
\int_{X} {\int_{Y} {|f|\nu}\mu} < \infty,\ \ \int_{Y} {\int_{X} {|f|\mu}\nu} < \infty,\ \ \int_{X \times Y} {|f|\mu \otimes \nu} < \infty
\end{align*}
定理\ref{4.6.4.3}、即ち、Tonelliの定理より次式が成り立つ。
\begin{align*}
\int_{X} {\int_{Y} {|f|\nu}\mu} = \int_{Y} {\int_{X} {|f|\mu}\nu} = \int_{X \times Y} {|f|\mu \otimes \nu}
\end{align*}\par
そこで、次のようになるので、
\begin{align*}
\left\{ \begin{matrix}
|f| = (f)_{+} + (f)_{-} \\
0 \leq (f)_{+} \\
0 \leq (f)_{-} \\
\end{matrix} \right.\  \Leftrightarrow \left\{ \begin{matrix}
|f| = (f)_{+} + (f)_{-} \\
0 \leq (f)_{+} \leq (f)_{+} + (f)_{-} \\
0 \leq (f)_{-} \leq (f)_{+} + (f)_{-} \\
\end{matrix} \right.\  \Leftrightarrow \left\{ \begin{matrix}
|f| = (f)_{+} + (f)_{-} \\
0 \leq (f)_{+} \leq |f| \\
0 \leq (f)_{-} \leq |f| \\
\end{matrix} \right.\ 
\end{align*}
積分の単調性により次のようになる。
\begin{align*}
\int_{X \times Y} {(f)_{+}\mu \otimes \nu} \leq \int_{X \times Y} {|f|\mu \otimes \nu} < \infty,\ \ \int_{X \times Y} {(f)_{-}\mu \otimes \nu} \leq \int_{X \times Y} {|f|\mu \otimes \nu} < \infty
\end{align*}
したがって、その写像$f$が定積分可能であるので、定理\ref{4.6.4.4}、即ち、Fubiniの定理より次のことが成り立つ。
\begin{itemize}
\item
  $(X,\varSigma,\mu) \ \text{-} \ \mathrm{a.e.}x \in X$に対し、その写像$f$は、変数$y$の写像とみたとき、その集合$Y$で定積分可能であるかつ、$(Y,T,\nu) \ \text{-} \ \mathrm{a.e.}y \in Y$に対し、その写像$f$は、変数$x$の写像とみたとき、その集合$X$で定積分可能である。
\item
  次のような写像たち$I_{f,X}$、$I_{f,Y}$はいづれもそれぞれそれらの集合たち$X$、$Y$で定積分可能である。
\begin{align*}
I_{f,X}:X \rightarrow \mathrm{cl}\mathbb{R}^{+};x \mapsto \int_{Y} {f\nu},\ \ I_{f,Y}:Y \rightarrow \mathrm{cl}\mathbb{R}^{+};y \mapsto \int_{X} {f\mu}
\end{align*}
\item
  次式が成り立つ。
\begin{align*}
\int_{X} {\int_{Y} {f\nu}\mu} = \int_{Y} {\int_{X} {f\mu}\nu} = \int_{X \times Y} {f\mu \otimes \nu}
\end{align*}
\end{itemize}
\end{proof}
%\hypertarget{ux5b8cux5099ux6e2cux5ea6ux306bux95a2ux3059ux308bfubini-tonelliux306eux5b9aux7406}{%
\subsubsection{完備測度に関するFubini-Tonelliの定理}%\label{ux5b8cux5099ux6e2cux5ea6ux306bux95a2ux3059ux308bfubini-tonelliux306eux5b9aux7406}}
\begin{thm}\label{4.6.4.6}
2つの$\sigma$-有限で完備な測度空間たち$(X,\varSigma,\mu)$、$(Y,T,\nu)$が与えられたとき、直積測度空間$(X \times Y, \varSigma \otimes T, \mu \otimes \nu)$が定義される。これが完備化されたものを$\left( X \times Y, \overline{\varSigma \otimes T},\overline{\mu \otimes \nu} \right)$とおくと、$\forall E \in \overline{\varSigma \otimes T}\forall x \in X\forall y \in Y$に対し、次のように集合$P_{E,x \in X}$、$P_{E,y \in Y}$が定義されよう。
\begin{align*}
P_{E,x \in X} = \left\{ y \in Y \middle| (x,y) \in E \right\},\ \ P_{E,y \in Y} = \left\{ x \in X \middle| (x,y) \in E \right\}
\end{align*}
このとき、$\overline{\mu \otimes \nu}(E) = 0$が成り立つなら、$\nu\left( P_{E,x \in X} \right) = 0\ (X,\varSigma,\mu) \ \text{-} \ \mathrm{a.e.}x \in X$かつ$\mu\left( P_{E,y \in Y} \right) = 0\ (Y,T,\nu) \ \text{-} \ \mathrm{a.e.}y \in Y$が成り立つ\footnote{なんでそんなまわりくどいことをやっているのかといいますと、完備化された直積測度空間では定理\ref{4.6.4.5}に適用できないからです…。}。
\end{thm}
\begin{proof}
2つの$\sigma$-有限で完備な測度空間たち$(X,\varSigma,\mu)$、$(Y,T,\nu)$が与えられたとき、直積測度空間$(X \times Y, \varSigma \otimes T, \mu \otimes \nu)$が定義される。これが完備化されたものを$\left( X \times Y, \overline{\varSigma \otimes T},\overline{\mu \otimes \nu} \right)$とおくと、$\forall E \in \overline{\varSigma \otimes T}\forall x \in X\forall y \in Y$に対し、次のように集合$P_{E,x \in X}$、$P_{E,y \in Y}$が定義されよう。
\begin{align*}
P_{E,x \in X} = \left\{ y \in Y \middle| (x,y) \in E \right\},\ \ P_{E,y \in Y} = \left\{ x \in X \middle| (x,y) \in E \right\}
\end{align*}\par
$\forall E \in \overline{\varSigma \otimes T}$に対し、$\overline{\mu \otimes \nu}(E) = 0$が成り立つなら、$\sigma$-有限な直積測度空間$(X \times Y, \varSigma \otimes T, \mu \otimes \nu)$から導かれた測度空間$\left( X \times Y, \mathfrak{M}_{C}\left( \gamma_{\mu \otimes \nu} \right), \gamma_{\mu \otimes \nu}|\mathfrak{M}_{C}\left( \gamma_{\mu \otimes \nu} \right) \right)$について、定理\ref{4.5.3.26}よりその測度空間$\left( X \times Y, \mathfrak{M}_{C}\left( \gamma_{\mu \otimes \nu} \right), \gamma_{\mu \otimes \nu}|\mathfrak{M}_{C}\left( \gamma_{\mu \otimes \nu} \right) \right)$はその測度空間$(X \times Y, \varSigma \otimes T, \mu \otimes \nu)$から完備化された測度空間$\left( X \times Y,\ \ \overline{\varSigma \otimes T}, \overline{\mu \otimes \nu} \right)$に等しく、定理\ref{4.5.3.25}より次式が成り立つような集合たち$F$、$G$がその$\sigma$-加法族$\varSigma \otimes T$に存在する。
\begin{align*}
F \subseteq E \subseteq G,\ \ \mu \otimes \nu(G \setminus F) = 0
\end{align*}
したがって、定理\ref{4.5.3.1}、定理\ref{4.5.3.24}より次式が成り立つので、
\begin{align*}
\left\{ \begin{matrix}
\varSigma \otimes T \subseteq \mathfrak{M}_{C}\left( \gamma_{\mu \otimes \nu} \right) \\
\gamma_{\mu \otimes \nu}|\varSigma \otimes T = \mu \otimes \nu \\
\end{matrix} \right. &\Leftrightarrow \left\{ \begin{matrix}
\varSigma \otimes T \subseteq \mathfrak{M}_{C}\left( \gamma_{\mu \otimes \nu} \right) \\
\left( \gamma_{\mu \otimes \nu}|\mathfrak{M}_{C}\left( \gamma_{\mu \otimes \nu} \right) \right)|\varSigma \otimes T = \mu \otimes \nu \\
\end{matrix} \right.\ \\
&\Leftrightarrow \left\{ \begin{matrix}
\varSigma \otimes T \subseteq \overline{\varSigma \otimes T} \\
\overline{\mu \otimes \nu}|\varSigma \otimes T = \mu \otimes \nu \\
\end{matrix} \right.\ \\
&\Rightarrow \overline{\mu \otimes \nu}|\varSigma \otimes T = \mu \otimes \nu
\end{align*}
次のようになる。
\begin{align*}
\left\{ \begin{matrix}
F \subseteq E \subseteq G \\
\mu \otimes \nu(G \setminus F) = 0 \\
\end{matrix} \right. &\Leftrightarrow \left\{ \begin{matrix}
F \subseteq E \subseteq G \\
\overline{\mu \otimes \nu}(F) \leq \overline{\mu \otimes \nu}(E) = 0 \leq \overline{\mu \otimes \nu}(G) \\
\mu \otimes \nu(G \setminus F) = 0 \\
\end{matrix} \right.\ \\
&\Leftrightarrow \left\{ \begin{matrix}
F \subseteq E \subseteq G \\
\mu \otimes \nu(F) \leq \overline{\mu \otimes \nu}(E) = 0 \leq \mu \otimes \nu(G) \\
\mu \otimes \nu(G \setminus F) = \mu \otimes \nu(G) - \mu \otimes \nu(G \cap F) = 0 \\
\end{matrix} \right.\ \\
&\Leftrightarrow \left\{ \begin{matrix}
F \subseteq E \subseteq G \\
\mu \otimes \nu(F) \leq \overline{\mu \otimes \nu}(E) = 0 \leq \mu \otimes \nu(G) \\
\mu \otimes \nu(G) = \mu \otimes \nu(F) \\
\end{matrix} \right.\ \\
&\Leftrightarrow \left\{ \begin{matrix}
F \subseteq E \subseteq G \\
\mu \otimes \nu(F) = \overline{\mu \otimes \nu}(E) = \mu \otimes \nu(G) = 0 \\
\end{matrix} \right.\ \\
&\Rightarrow \left\{ \begin{matrix}
E \subseteq G \\
\overline{\mu \otimes \nu}(E) = \mu \otimes \nu(G) = 0 \\
\end{matrix} \right.
\end{align*}
もちろん、$\forall x \in X$に対し、$P_{E,x \in X} \subseteq P_{G,x \in X}$が成り立つので、次式が成り立ち、
\begin{align*}
0 \leq \left( X \rightarrow \mathrm{cl}\mathbb{R}^{+};x \mapsto \nu\left( P_{E,x \in X} \right) \right) \leq \left( X \rightarrow \mathrm{cl}\mathbb{R}^{+};x \mapsto \nu\left( P_{G,x \in X} \right) \right)
\end{align*}
積分の単調性と定理\ref{4.6.4.2}よりしたがって、次のようになる。
\begin{align*}
0 \leq \int_{X} {\begin{matrix}
X & \rightarrow & \mathrm{cl}\mathbb{R}^{+} \\
\text{\rotatebox[origin=c]{90}{$\in$}} & & \text{\rotatebox[origin=c]{90}{$\in$}} \\
x & \mapsto & \nu\left( P_{E,x \in X} \right) \\
\end{matrix}\mu} \leq \int_{X} {\begin{matrix}
X & \rightarrow & \mathrm{cl}\mathbb{R}^{+} \\
\text{\rotatebox[origin=c]{90}{$\in$}} & & \text{\rotatebox[origin=c]{90}{$\in$}} \\
x & \mapsto & \nu\left( P_{G,x \in X} \right) \\
\end{matrix}\mu} = \mu \otimes \nu(G) = 0
\end{align*}
これにより、次式が成り立つので、
\begin{align*}
\int_{X} {\begin{matrix}
X & \rightarrow & \mathrm{cl}\mathbb{R}^{+} \\
\text{\rotatebox[origin=c]{90}{$\in$}} & & \text{\rotatebox[origin=c]{90}{$\in$}} \\
x & \mapsto & \nu\left( P_{E,x \in X} \right) \\
\end{matrix}\mu} = 0
\end{align*}
$\nu\left( P_{E,x \in X} \right) = 0\ (X,\varSigma,\mu) \ \text{-} \ \mathrm{a.e.}x \in X$が成り立つ。同様にして、$\mu\left( P_{E,y \in Y} \right) = 0\ (Y,T,\nu) \ \text{-} \ \mathrm{a.e.}y \in Y$が成り立つことが示される。
\end{proof}
\begin{thm}\label{4.6.4.7}
2つの$\sigma$-有限で完備な測度空間たち$(X,\varSigma,\mu)$、$(Y,T,\nu)$が与えられたとき、直積測度空間$(X \times Y,\ \ \varSigma \otimes T,\ \ \mu \otimes \nu)$が定義される。これが完備化されたものを$\left( X \times Y,\ \ \overline{\varSigma \otimes T},\ \ \overline{\mu \otimes \nu} \right)$とおくと、$\forall E \in \overline{\varSigma \otimes T}\forall x \in X\forall y \in Y$に対し、次のように集合$P_{E,x \in X}$、$P_{E,y \in Y}$が定義されよう。
\begin{align*}
P_{E,x \in X} = \left\{ y \in Y \middle| (x,y) \in E \right\},\ \ P_{E,y \in Y} = \left\{ x \in X \middle| (x,y) \in E \right\}
\end{align*}
このとき、次のことが成り立つ。
\begin{itemize}
\item
  $P_{E,x \in X} \in T\ (X,\varSigma,\mu) \ \text{-} \ \mathrm{a.e.}x \in X$が成り立つかつ、$P_{E,y \in Y} \in \varSigma\ (Y,T,\nu) \ \text{-} \ \mathrm{a.e.}y \in Y$が成り立つ。
\item
  次のような写像たち$p_{X}$、$p_{Y}$はいづれもそれぞれそれらの測度空間たち$(X,\varSigma,\mu)$、$(Y,T,\nu)$で可測である。
\begin{align*}
p_{X}:X \rightarrow \mathrm{cl}\mathbb{R}^{+};x \mapsto \nu\left( P_{E,x \in X} \right),\ \ p_{Y}:Y \rightarrow \mathrm{cl}\mathbb{R}^{+};y \mapsto \mu\left( P_{E,y \in Y} \right)
\end{align*}
\item
  次式が成り立つ。
\begin{align*}
\int_{X} {p_{X}\mu} = \int_{Y} {p_{Y}\nu} = \overline{\mu \otimes \nu}(E)
\end{align*}
\item
  $\overline{\mu \otimes \nu}(E) < \infty$が成り立つなら、$\nu\left( P_{E,x \in X} \right) < \infty\ (X,\varSigma,\mu) \ \text{-} \ \mathrm{a.e.}x \in X$かつ$\mu\left( P_{E,y \in Y} \right) < \infty\ (Y,T,\nu) \ \text{-} \ \mathrm{a.e.}y \in Y$が成り立つ。
\end{itemize}
\end{thm}
\begin{proof}
2つの$\sigma$-有限で完備な測度空間たち$(X,\varSigma,\mu)$、$(Y,T,\nu)$が与えられたとき、直積測度空間$(X \times Y,\ \ \varSigma \otimes T,\ \ \mu \otimes \nu)$が定義される。これが完備化されたものを$\left( X \times Y,\ \ \overline{\varSigma \otimes T},\ \ \overline{\mu \otimes \nu} \right)$とおくと、$\forall E \in \overline{\varSigma \otimes T}\forall x \in X\forall y \in Y$に対し、次のように集合$P_{E,x \in X}$、$P_{E,y \in Y}$が定義されよう。
\begin{align*}
P_{E,x \in X} = \left\{ y \in Y \middle| (x,y) \in E \right\},\ \ P_{E,y \in Y} = \left\{ x \in X \middle| (x,y) \in E \right\}
\end{align*}
このとき、$\sigma$-有限な直積測度空間$(X \times Y,\ \ \varSigma \otimes T,\ \ \mu \otimes \nu)$から導かれた測度空間$\left( X \times Y,\ \ \mathfrak{M}_{C}\left( \gamma_{\mu \otimes \nu} \right),\ \ \gamma_{\mu \otimes \nu}|\mathfrak{M}_{C}\left( \gamma_{\mu \otimes \nu} \right) \right)$について、定理\ref{4.5.3.26}よりその測度空間$\left( X \times Y,\ \ \mathfrak{M}_{C}\left( \gamma_{\mu \otimes \nu} \right),\ \ \gamma_{\mu \otimes \nu}|\mathfrak{M}_{C}\left( \gamma_{\mu \otimes \nu} \right) \right)$はその測度空間$(X \times Y,\ \ \varSigma \otimes T,\ \ \mu \otimes \nu)$から完備化された測度空間$\left( X \times Y,\ \ \overline{\varSigma \otimes T},\ \ \overline{\mu \otimes \nu} \right)$に等しく、定理\ref{4.5.3.25}より次式が成り立つような集合たち$F$、$G$がその$\sigma$-加法族$\varSigma \otimes T$に存在する。
\begin{align*}
F \subseteq E \subseteq G,\ \ \mu \otimes \nu(G \setminus F) = 0
\end{align*}
したがって、定理\ref{4.5.3.1}、定理\ref{4.5.3.24}より次のようになるので、
\begin{align*}
\left\{ \begin{matrix}
\varSigma \otimes T \subseteq \mathfrak{M}_{C}\left( \gamma_{\mu \otimes \nu} \right) \\
\gamma_{\mu \otimes \nu}|\varSigma \otimes T = \mu \otimes \nu \\
\end{matrix} \right. &\Leftrightarrow \left\{ \begin{matrix}
\varSigma \otimes T \subseteq \mathfrak{M}_{C}\left( \gamma_{\mu \otimes \nu} \right) \\
\left( \gamma_{\mu \otimes \nu}|\mathfrak{M}_{C}\left( \gamma_{\mu \otimes \nu} \right) \right)|\varSigma \otimes T = \mu \otimes \nu \\
\end{matrix} \right.\ \\
&\Leftrightarrow \left\{ \begin{matrix}
\varSigma \otimes T \subseteq \overline{\varSigma \otimes T} \\
\overline{\mu \otimes \nu}|\varSigma \otimes T = \mu \otimes \nu \\
\end{matrix} \right.\ \\
&\Rightarrow \overline{\mu \otimes \nu}|\varSigma \otimes T = \mu \otimes \nu
\end{align*}
次のようになる。
\begin{align*}
\mu \otimes \nu(G \setminus F) = \overline{\mu \otimes \nu}|\varSigma \otimes T(G \setminus F) = \overline{\mu \otimes \nu}(G \setminus F) = 0
\end{align*}
定理\ref{4.6.4.6}より$(X,\varSigma,\mu) \ \text{-} \ \mathrm{a.e.}x \in X$に対し、$\nu\left( P_{G \setminus F,x \in X} \right) = 0$が成り立つ。そこで、$E \setminus F \subseteq G \setminus F$が成り立ち、したがって、$P_{E \setminus F,x \in X} \subseteq P_{G \setminus F,x \in X}$が成り立つので、その測度空間$(Y,T,\nu)$が完備であることに注意すれば、$P_{E \setminus F,x \in X} \in T$が成り立つ。定理\ref{4.6.4.2}より$P_{F,x \in X} \in T$が成り立つので、次のようになる。
\begin{align*}
P_{E \setminus F,x \in X} \cup P_{F,x \in X} = P_{E \setminus F \cup F,x \in X} = P_{E,x \in X} \in T
\end{align*}
ゆえに、$P_{E,x \in X} \in T\ (X,\varSigma,\mu) \ \text{-} \ \mathrm{a.e.}x \in X$が成り立つ。同様にして、$P_{E,y \in Y} \in \varSigma\ (Y,T,\nu) \ \text{-} \ \mathrm{a.e.}y \in Y$が成り立つ。\par
次のような写像$p_{X}$について、
\begin{align*}
p_{X}:X \rightarrow \mathrm{cl}\mathbb{R}^{+};x \mapsto \nu\left( P_{E,x \in X} \right)
\end{align*}
その測度空間$(Y,T,\nu)$が完備であることに注意すれば、上記の議論と定理\ref{4.6.4.6}より次のようになるので、
\begin{align*}
\left\{ \begin{matrix}
F \subseteq E \subseteq G \\
\mu \otimes \nu(G \setminus F) = 0 \\
\end{matrix} \right. &\Rightarrow \forall x \in X\left[ \left\{ \begin{matrix}
P_{F,x \in X} \subseteq P_{E,x \in X} \subseteq P_{G,x \in X} \\
\overline{\mu \otimes \nu}(G \setminus F) = 0 \\
\end{matrix} \right.\  \right]\\
&\Leftrightarrow \forall x \in X\left[ \left\{ \begin{matrix}
P_{F,x \in X} \subseteq P_{E,x \in X} \subseteq P_{G,x \in X} \\
\emptyset \subseteq P_{E,x \in X} \setminus P_{F,x \in X} \subseteq P_{G,x \in X} \setminus P_{F,x \in X} \\
\nu\left( P_{G \setminus F,x \in X} \right) = 0 \\
\end{matrix} \right.\  \right]\\
&\Leftrightarrow \forall x \in X\left[ \left\{ \begin{matrix}
P_{F,x \in X} \subseteq P_{E,x \in X} \subseteq P_{G,x \in X} \\
\emptyset \subseteq P_{E,x \in X} \setminus P_{F,x \in X} \subseteq P_{G,x \in X} \setminus P_{F,x \in X} \\
\nu\left( P_{G,x \in X} \setminus P_{F,x \in X} \right) = 0 \\
\end{matrix} \right.\  \right]\\
&\Leftrightarrow \forall x \in X\left[ \left\{ \begin{matrix}
P_{F,x \in X} \subseteq P_{E,x \in X} \subseteq P_{G,x \in X} \\
0 \leq \nu\left( P_{E,x \in X} \setminus P_{F,x \in X} \right) \leq \nu\left( P_{G,x \in X} \setminus P_{F,x \in X} \right) \\
\nu\left( P_{G,x \in X} \setminus P_{F,x \in X} \right) = 0 \\
\end{matrix} \right.\  \right]\\
&\Rightarrow \forall x \in X\left[ \left\{ \begin{matrix}
P_{F,x \in X} \subseteq P_{E,x \in X} \subseteq P_{G,x \in X} \\
\nu\left( P_{E,x \in X} \setminus P_{F,x \in X} \right) = 0 \\
\end{matrix} \right.\  \right]\\
&\Rightarrow \forall x \in X\left[ \nu\left( P_{F,x \in X} \right) = \nu\left( P_{E,x \in X} \right) \right]\\
&\Leftrightarrow \forall x \in X\left[ \nu\left( P_{F,x \in X} \right) = p_{X}(x) \right]
\end{align*}
定理\ref{4.6.4.2}よりその写像$p_{X}$はその測度空間$(X,\varSigma,\mu)$で可測である。同様にして、次のような写像$p_{Y}$もその測度空間$(Y,T,\nu)$で可測であることが示される。
\begin{align*}
p_{Y}:Y \rightarrow \mathrm{cl}\mathbb{R}^{+};y \mapsto \mu\left( P_{E,y \in Y} \right)
\end{align*}\par
上記の議論により次のようになることから、
\begin{align*}
\left\{ \begin{matrix}
F \subseteq E \subseteq G \\
\mu \otimes \nu(G \setminus F) = 0 \\
\end{matrix} \right. &\Leftrightarrow \left\{ \begin{matrix}
F \subseteq E \subseteq G \\
\emptyset \subseteq E \setminus F \subseteq G \setminus F \\
\overline{\mu \otimes \nu}(G \setminus F) = 0 \\
\end{matrix} \right.\ \\
&\Rightarrow \left\{ \begin{matrix}
F \subseteq E \subseteq G \\
0 \leq \overline{\mu \otimes \nu}(E \setminus F) \leq \overline{\mu \otimes \nu}(G \setminus F) \\
\overline{\mu \otimes \nu}(G \setminus F) = 0 \\
\end{matrix} \right.\ \\
&\Rightarrow \left\{ \begin{matrix}
F \subseteq E \subseteq G \\
\overline{\mu \otimes \nu}(E \setminus F) = 0 \\
\end{matrix} \right.\ \\
&\Rightarrow \overline{\mu \otimes \nu}(F) = \overline{\mu \otimes \nu}(E)
\end{align*}
上記の議論と定理\ref{4.6.4.2}より次のようになる。
\begin{align*}
\int_{X} {p_{X}\mu} &= \int_{X} {\begin{matrix}
X & \rightarrow & \mathrm{cl}\left( \mathbb{R}^{+} \right) \\
\text{\rotatebox[origin=c]{90}{$\in$}} & & \text{\rotatebox[origin=c]{90}{$\in$}} \\
x & \mapsto & p_{X}(x) \\
\end{matrix}\mu}\\
&= \int_{X} {\begin{matrix}
X & \rightarrow & \mathrm{cl}\left( \mathbb{R}^{+} \right) \\
\text{\rotatebox[origin=c]{90}{$\in$}} & & \text{\rotatebox[origin=c]{90}{$\in$}} \\
x & \mapsto & \nu\left( P_{F,x \in X} \right) \\
\end{matrix}\mu}\\
&= \mu \otimes \nu(F)\\
&= \overline{\mu \otimes \nu}(F)\\
&= \overline{\mu \otimes \nu}(E)
\end{align*}
同様にして、次式が成り立つことが示される。
\begin{align*}
\int_{Y} {p_{Y}\nu} = \overline{\mu \otimes \nu}(E)
\end{align*}\par
したがって、$\overline{\mu \otimes \nu}(E) < \infty$が成り立つなら、$p_{X}(x) < \infty\ (X,\varSigma,\mu) \ \text{-} \ \mathrm{a.e.}x \in X$かつ$p_{Y}(y) < \infty\ (Y,T,\nu) \ \text{-} \ \mathrm{a.e.}y \in Y$が成り立つ、即ち、$\nu\left( P_{E,x \in X} \right) < \infty\ (X,\varSigma,\mu) \ \text{-} \ \mathrm{a.e.}x \in X$かつ$\mu\left( P_{E,y \in Y} \right) < \infty\ (Y,T,\nu) \ \text{-} \ \mathrm{a.e.}y \in Y$が成り立つ。
\end{proof}\par
また、$\forall f \in \mathcal{M}_{\left( X \times Y,\ \ \overline{\varSigma \otimes T},\ \ \overline{\mu \otimes \nu} \right)}^{+}\exists g \in \mathcal{M}_{(X \times Y,\ \ \varSigma \otimes T,\ \ \mu \otimes \nu)}^{+}$に対し、$0 \leq g \leq f$かつ$\overline{\mu \otimes \nu}\left( \left\{ f \neq g \right\} \right) = 0$が成り立つ。実際、このことは定理\ref{4.5.5.22}そのものである。
\begin{thm}[完備測度に関するTonelliの定理]\label{4.6.4.8}
2つの$\sigma$-有限で完備な測度空間たち$(X,\varSigma,\mu)$、$(Y,T,\nu)$が与えられたとき、直積測度空間$(X \times Y,\ \ \varSigma \otimes T,\ \ \mu \otimes \nu)$が定義される。これが完備化されたものを$\left( X \times Y,\ \ \overline{\varSigma \otimes T},\ \ \overline{\mu \otimes \nu} \right)$とおくと、$\forall E \subseteq \overline{\varSigma \otimes T}\forall x \in X\forall y \in Y$に対し、次のように集合$P_{E,x \in X}$、$P_{E,y \in Y}$が定義されよう。
\begin{align*}
P_{E,x \in X} = \left\{ y \in Y \middle| (x,y) \in E \right\},\ \ P_{E,y \in Y} = \left\{ x \in X \middle| (x,y) \in E \right\}
\end{align*}
$\forall f:X \times Y \rightarrow \mathrm{cl}\mathbb{R}^{+} \in \mathcal{M}_{\left( X \times Y,\ \ \overline{\varSigma \otimes T},\ \ \overline{\mu \otimes \nu} \right)}$に対し、$0 \leq f(x,y)\ \left( X \times Y,\ \ \overline{\varSigma \otimes T},\ \ \overline{\mu \otimes \nu} \right) \ \text{-} \ \mathrm{a.e.}(x,y) \in X \times Y$が成り立つなら、次のことが成り立つ。
\begin{itemize}
\item
  $(X,\varSigma,\mu) \ \text{-} \ \mathrm{a.e.}x \in X$に対し、その写像$f$は、変数$y$の写像とみたとき、その測度空間$(Y,T,\nu)$で可測であるかつ、$(Y,T,\nu) \ \text{-} \ \mathrm{a.e.}y \in Y$に対し、その写像$f$は、変数$x$の写像とみたとき、その測度空間$(X,\varSigma,\mu)$で可測である。
\item
  次のような写像たち$I_{f,X}$、$I_{f,Y}$はいづれもそれぞれそれらの測度空間たち$(X,\varSigma,\mu)$、$(Y,T,\nu)$で可測である。
\begin{align*}
I_{f,X}:X \rightarrow \mathrm{cl}\mathbb{R}^{+};x \mapsto \int_{Y} {f\nu},\ \ I_{f,Y}:Y \rightarrow \mathrm{cl}\mathbb{R}^{+};y \mapsto \int_{X} {f\mu}
\end{align*}
\item
  次式が成り立つ。
\begin{align*}
\int_{X} {\int_{Y} {f\nu}\mu} = \int_{Y} {\int_{X} {f\mu}\nu} = \int_{X \times Y} {f\overline{\mu \otimes \nu}}
\end{align*}
\end{itemize}
この定理を完備測度に関するTonelliの定理という。
\end{thm}
\begin{proof}
2つの$\sigma$-有限で完備な測度空間たち$(X,\varSigma,\mu)$、$(Y,T,\nu)$が与えられたとき、直積測度空間$(X \times Y,\ \ \varSigma \otimes T,\ \ \mu \otimes \nu)$が定義される。これが完備化されたものを$\left( X \times Y,\ \ \overline{\varSigma \otimes T},\ \ \overline{\mu \otimes \nu} \right)$とおくと、$\forall E \subseteq \overline{\varSigma \otimes T}\forall x \in X\forall y \in Y$に対し、次のように集合$P_{E,x \in X}$、$P_{E,y \in Y}$が定義されよう。
\begin{align*}
P_{E,x \in X} = \left\{ y \in Y \middle| (x,y) \in E \right\},\ \ P_{E,y \in Y} = \left\{ x \in X \middle| (x,y) \in E \right\}
\end{align*}
$\forall f:X \times Y \rightarrow \mathrm{cl}\mathbb{R}^{+} \in \mathcal{M}_{\left( X \times Y,\ \ \overline{\varSigma \otimes T},\ \ \overline{\mu \otimes \nu} \right)}$に対し、$0 \leq f(x,y)\ \left( X \times Y,\ \ \overline{\varSigma \otimes T},\ \ \overline{\mu \otimes \nu} \right) \ \text{-} \ \mathrm{a.e.}(x,y) \in X \times Y$が成り立つなら、定理\ref{4.5.5.22}より$\exists g \in \mathcal{M}_{(X \times Y,\ \ \varSigma \otimes T,\ \ \mu \otimes \nu)}^{+}$に対し、$0 \leq g(x,y) \leq f(x,y)\ \left( X \times Y,\ \ \overline{\varSigma \otimes T},\ \ \overline{\mu \otimes \nu} \right) \ \text{-} \ \mathrm{a.e.}(x,y) \in X \times Y$かつ$\overline{\mu \otimes \nu}\left( \left\{ f \neq g \right\} \right) = 0$が成り立つ。そこで、その写像$g$が定理\ref{4.6.4.3}、即ち、Tonelliの定理に適用されることで、$\forall x \in X$に対し、その写像$g$は、変数$y$の写像とみたとき、その測度空間$(Y,T,\nu)$で可測である。そこで、定理\ref{4.6.4.6}より$\overline{\mu \otimes \nu}\left( \left\{ f \neq g \right\} \right) = 0$が成り立つなら、$\nu\left( P_{\left\{ f \neq g \right\},x \in X} \right) = 0\ (X,\varSigma,\mu) \ \text{-} \ \mathrm{a.e.}x \in X$が成り立つ。そこで、$\forall y \in Y$に対し、次のようになるので、
\begin{align*}
y \in P_{\left\{ f \neq g \right\},x \in X} &\Leftrightarrow y \in \left\{ y \in Y \middle| (x,y) \in \left\{ f \neq g \right\} \right\}\\
&\Leftrightarrow y \in \left\{ y \in Y \middle| (x,y) \in \left\{ (x,y) \in X \times Y \middle| f(x,y) \neq g(x,y) \right\} \right\}\\
&\Leftrightarrow y \in Y \land (x,y) \in X \times Y \land f(x,y) \neq g(x,y)\\
&\Leftrightarrow y \in Y \land f(x,y) \neq g(x,y)\\
&\Leftrightarrow y \in \left\{ y \in Y \middle| f(x,y) \neq g(x,y) \right\}\\
&\Leftrightarrow y \in \left\{ \begin{matrix}
Y & \rightarrow & \mathrm{cl}\mathbb{R}^{+} \\
\text{\rotatebox[origin=c]{90}{$\in$}} & & \text{\rotatebox[origin=c]{90}{$\in$}} \\
y & \mapsto & f(x,y) \\
\end{matrix} \neq \begin{matrix}
Y & \rightarrow & \mathrm{cl}\mathbb{R}^{+} \\
\text{\rotatebox[origin=c]{90}{$\in$}} & & \text{\rotatebox[origin=c]{90}{$\in$}} \\
y & \mapsto & g(x,y) \\
\end{matrix} \right\}
\end{align*}
$P_{\left\{ f \neq g \right\},x \in X} = \left\{ \begin{matrix}
Y & \rightarrow & \mathrm{cl}\mathbb{R}^{+} \\
\text{\rotatebox[origin=c]{90}{$\in$}} & & \text{\rotatebox[origin=c]{90}{$\in$}} \\
y & \mapsto & f(x,y) \\
\end{matrix} \neq \begin{matrix}
Y & \rightarrow & \mathrm{cl}\mathbb{R}^{+} \\
\text{\rotatebox[origin=c]{90}{$\in$}} & & \text{\rotatebox[origin=c]{90}{$\in$}} \\
y & \mapsto & g(x,y) \\
\end{matrix} \right\}$が得られ、したがって、次式が成り立つ。
\begin{align*}
\nu\left( \left\{ \begin{matrix}
Y & \rightarrow & \mathrm{cl}\mathbb{R}^{+} \\
\text{\rotatebox[origin=c]{90}{$\in$}} & & \text{\rotatebox[origin=c]{90}{$\in$}} \\
y & \mapsto & f(x,y) \\
\end{matrix} \neq \begin{matrix}
Y & \rightarrow & \mathrm{cl}\mathbb{R}^{+} \\
\text{\rotatebox[origin=c]{90}{$\in$}} & & \text{\rotatebox[origin=c]{90}{$\in$}} \\
y & \mapsto & g(x,y) \\
\end{matrix} \right\} \right) = 0\ (X,\varSigma,\mu) \ \text{-} \ \mathrm{a.e.}x \in X
\end{align*}
これにより、$(X,\varSigma,\mu) \ \text{-} \ \mathrm{a.e.}x \in X$に対し、$\begin{matrix}
Y & \rightarrow & \mathrm{cl}\mathbb{R}^{+} \\
\text{\rotatebox[origin=c]{90}{$\in$}} & & \text{\rotatebox[origin=c]{90}{$\in$}} \\
y & \mapsto & f(x,y) \\
\end{matrix} = \begin{matrix}
Y & \rightarrow & \mathrm{cl}\mathbb{R}^{+} \\
\text{\rotatebox[origin=c]{90}{$\in$}} & & \text{\rotatebox[origin=c]{90}{$\in$}} \\
y & \mapsto & g(x,y) \\
\end{matrix}$が成り立つので、$(X,\varSigma,\mu) \ \text{-} \ \mathrm{a.e.}x \in X$に対し、その写像$f$は、変数$y$の写像とみたとき、その測度空間$(Y,T,\nu)$で可測である。同様にして、$(Y,T,\nu) \ \text{-} \ \mathrm{a.e.}y \in Y$に対し、その写像$f$は、変数$x$の写像とみたとき、その測度空間$(X,\varSigma,\mu)$で可測であることが示される。\par
また、上記の議論により次のような写像たち$I_{f,X}$について、
\begin{align*}
I_{f,X}:X \rightarrow \mathrm{cl}\mathbb{R}^{+};x \mapsto \int_{Y} {f\nu}
\end{align*}
次のようになることから、
\begin{align*}
I_{f,X} = \begin{matrix}
X & \overset{I_{f,X}}{\rightarrow} & \mathrm{cl}\mathbb{R}^{+} \\
\text{\rotatebox[origin=c]{90}{$\in$}} & & \text{\rotatebox[origin=c]{90}{$\in$}} \\
x & \mapsto & \int_{Y} {f\nu} \\
\end{matrix} = \begin{matrix}
X & \rightarrow & \mathrm{cl}\mathbb{R}^{+} \\
\text{\rotatebox[origin=c]{90}{$\in$}} & & \text{\rotatebox[origin=c]{90}{$\in$}} \\
x & \mapsto & \int_{Y} {g\nu} \\
\end{matrix}
\end{align*}
定理\ref{4.6.4.3}、即ち、Tonelliの定理よりその写像$I_{f,Y}$はその測度空間$(X,\varSigma,\mu)$で可測である。同様にして、次のような写像$I_{f,Y}$はその測度空間$(Y,T,\nu)$で可測である。
\begin{align*}
I_{f,Y}:Y \rightarrow \mathrm{cl}\mathbb{R}^{+};y \mapsto \int_{X} {f\mu}
\end{align*}\par
さらに、Tonelliの定理より次のようになる。
\begin{align*}
\int_{X} {\int_{Y} {f\nu}\mu} &= \int_{X} {I_{f,X}\mu}\\
&= \int_{X} {\begin{matrix}
X & \overset{I_{f,X}}{\rightarrow} & \mathrm{cl}\mathbb{R}^{+} \\
\text{\rotatebox[origin=c]{90}{$\in$}} & & \text{\rotatebox[origin=c]{90}{$\in$}} \\
x & \mapsto & \int_{Y} {f\nu} \\
\end{matrix}\mu}\\
&= \int_{X} {\begin{matrix}
X & \rightarrow & \mathrm{cl}\mathbb{R}^{+} \\
\text{\rotatebox[origin=c]{90}{$\in$}} & & \text{\rotatebox[origin=c]{90}{$\in$}} \\
x & \mapsto & \int_{Y} {g\nu} \\
\end{matrix}\mu}\\
&= \int_{X} {\int_{Y} {g\nu}\mu}\\
&= \int_{X \times Y} {g\mu \otimes \nu}\\
&= \int_{X \times Y} {g\overline{\mu \otimes \nu}}\\
&= \int_{X \times Y} {f\overline{\mu \otimes \nu}}
\end{align*}
同様にして、次式が成り立つことが示される。
\begin{align*}
\int_{Y} {\int_{X} {f\mu}\nu} = \int_{X \times Y} {f\overline{\mu \otimes \nu}}
\end{align*}
\end{proof}
\begin{thm}[完備測度に関するFubiniの定理]\label{4.6.4.7}
2つの$\sigma$-有限で完備な測度空間たち$(X,\varSigma,\mu)$、$(Y,T,\nu)$が与えられたとき、直積測度空間$(X \times Y,\ \ \varSigma \otimes T,\ \ \mu \otimes \nu)$が定義される。これが完備化されたものを$\left( X \times Y,\ \ \overline{\varSigma \otimes T},\ \ \overline{\mu \otimes \nu} \right)$とおくと、$\forall E \subseteq \overline{\varSigma \otimes T}\forall x \in X\forall y \in Y$に対し、次のように集合$P_{E,x \in X}$、$P_{E,y \in Y}$が定義されよう。
\begin{align*}
P_{E,x \in X} = \left\{ y \in Y \middle| (x,y) \in E \right\},\ \ P_{E,y \in Y} = \left\{ x \in X \middle| (x,y) \in E \right\}
\end{align*}
$\forall f:X \times Y \rightarrow \mathrm{cl}\mathbb{R} \in \mathcal{M}_{\left( X \times Y,\ \ \overline{\varSigma \otimes T},\ \ \overline{\mu \otimes \nu} \right)}$に対し、その写像$f$が定積分可能であるなら、次のことが成り立つ。
\begin{itemize}
\item
  $(X,\varSigma,\mu) \ \text{-} \ \mathrm{a.e.}x \in X$に対し、その写像$f$は、変数$y$の写像とみたとき、その集合$Y$で定積分可能であるかつ、$(Y,T,\nu) \ \text{-} \ \mathrm{a.e.}y \in Y$に対し、その写像$f$は、変数$x$の写像とみたとき、その集合$X$で定積分可能である。
\item
  次のような写像たち$I_{f,X}$、$I_{f,Y}$はいづれもそれぞれそれらの集合たち$X$、$Y$で定積分可能である。
\begin{align*}
I_{f,X}:X \rightarrow \mathrm{cl}\mathbb{R}^{+};x \mapsto \int_{Y} {f\nu},\ \ I_{f,Y}:Y \rightarrow \mathrm{cl}\mathbb{R}^{+};y \mapsto \int_{X} {f\mu}
\end{align*}
\item
  次式が成り立つ。
\begin{align*}
\int_{X} {\int_{Y} {f\nu}\mu} = \int_{Y} {\int_{X} {f\mu}\nu} = \int_{X \times Y} {f\mu \otimes \nu}
\end{align*}
\end{itemize}
この定理を完備測度に関するFubiniの定理という。
\end{thm}
\begin{proof}
2つの$\sigma$-有限で完備な測度空間たち$(X,\varSigma,\mu)$、$(Y,T,\nu)$が与えられたとき、直積測度空間$(X \times Y,\ \ \varSigma \otimes T,\ \ \mu \otimes \nu)$が定義される。これが完備化されたものを$\left( X \times Y,\ \ \overline{\varSigma \otimes T},\ \ \overline{\mu \otimes \nu} \right)$とおくと、$\forall E \subseteq \overline{\varSigma \otimes T}\forall x \in X\forall y \in Y$に対し、次のように集合$P_{E,x \in X}$、$P_{E,y \in Y}$が定義されよう。
\begin{align*}
P_{E,x \in X} = \left\{ y \in Y \middle| (x,y) \in E \right\},\ \ P_{E,y \in Y} = \left\{ x \in X \middle| (x,y) \in E \right\}
\end{align*}
$\forall f:X \times Y \rightarrow \mathrm{cl}\mathbb{R} \in \mathcal{M}_{\left( X \times Y,\ \ \overline{\varSigma \otimes T},\ \ \overline{\mu \otimes \nu} \right)}$に対し、その写像$f$が定積分可能であるなら、
$f = (f)_{+} - (f)_{-}$が成り立つので、定理\ref{4.6.4.3}、即ち、Tonelliの定理より次のことが成り立つ。
\begin{itemize}
\item
  $\forall x \in X$に対し、その写像たち$(f)_{+}$、$(f)_{-}$は、変数$y$の写像とみたとき、その測度空間$(Y,T,\nu)$で可測であるかつ、$\forall y \in Y$に対し、その写像たち$(f)_{+}$、$(f)_{-}$は、変数$x$の写像とみたとき、その測度空間$(X,\varSigma,\mu)$で可測である。
\item
  次のような写像たち$I_{(f)_{+},X}$、$I_{(f)_{+},Y}$、$I_{(f)_{-},X}$、$I_{(f)_{-},Y}$はいづれもそれぞれそれらの測度空間たち$(X,\varSigma,\mu)$、$(Y,T,\nu)$で可測である。
\begin{align*}
I_{(f)_{+},X}&:X \rightarrow \mathrm{cl}\mathbb{R}^{+};x \mapsto \int_{Y} {(f)_{+}\nu},\ \ I_{(f)_{+},Y}:Y \rightarrow \mathrm{cl}\mathbb{R}^{+};y \mapsto \int_{X} {(f)_{+}\mu},\\
I_{(f)_{-},X}&:X \rightarrow \mathrm{cl}\mathbb{R}^{+};x \mapsto \int_{Y} {(f)_{-}\nu},\ \ I_{(f)_{-},Y}:Y \rightarrow \mathrm{cl}\mathbb{R}^{+};y \mapsto \int_{X} {(f)_{-}\mu}
\end{align*}
\item
  次式が成り立つ。
\begin{align*}
\int_{X} {\int_{Y} {(f)_{+}\nu}\mu} &= \int_{Y} {\int_{X} {(f)_{+}\mu}\nu} = \int_{X \times Y} {(f)_{+}\overline{\mu \otimes \nu}} < \infty\\
\int_{X} {\int_{Y} {(f)_{-}\nu}\mu} &= \int_{Y} {\int_{X} {(f)_{-}\mu}\nu} = \int_{X \times Y} {(f)_{-}\overline{\mu \otimes \nu}} < \infty
\end{align*}
\end{itemize}
したがって、$(X,\varSigma,\mu) \ \text{-} \ \mathrm{a.e.}x \in X$に対し、その写像$f$は、変数$y$の写像とみたとき、その集合$Y$で定積分可能であるかつ、$(Y,T,\nu) \ \text{-} \ \mathrm{a.e.}y \in Y$に対し、その写像$f$は、変数$x$の写像とみたとき、その集合$X$で定積分可能である。\par
さらに、定理\ref{4.6.4.4}、即ち、Fubiniの定理の証明と同様にして、次のような写像たち$I_{f,X}$、$I_{f,Y}$はいづれもそれぞれそれらの集合たち$X$、$Y$で定積分可能であることが示される。
\begin{align*}
I_{f,X}:X \rightarrow \mathrm{cl}\mathbb{R}^{+};x \mapsto \int_{Y} {f\nu},\ \ I_{f,Y}:Y \rightarrow \mathrm{cl}\mathbb{R}^{+};y \mapsto \int_{X} {f\mu}
\end{align*}\par
また、ここでも定理\ref{4.6.4.4}、即ち、Fubiniの定理の証明と同様にして、次式が成り立つことが示される。
\begin{align*}
\int_{X} {\int_{Y} {f\nu}\mu} = \int_{Y} {\int_{X} {f\mu}\nu} = \int_{X \times Y} {f\overline{\mu \otimes \nu}}
\end{align*}
\end{proof}
\begin{thm}[完備測度に関するFubini-Tonelliの定理]\label{4.6.4.8}
2つの$\sigma$-有限で完備な測度空間たち$(X,\varSigma,\mu)$、$(Y,T,\nu)$が与えられたとき、直積測度空間$(X \times Y, \varSigma \otimes T, \mu \otimes \nu)$が定義される。これが完備化されたものを$\left( X \times Y, \overline{\varSigma \otimes T}, \overline{\mu \otimes \nu} \right)$とおくと、$\forall E \subseteq \overline{\varSigma \otimes T}\forall x \in X\forall y \in Y$に対し、次のように集合$P_{E,x \in X}$、$P_{E,y \in Y}$が定義されよう。
\begin{align*}
P_{E,x \in X} = \left\{ y \in Y \middle| (x,y) \in E \right\},\ \ P_{E,y \in Y} = \left\{ x \in X \middle| (x,y) \in E \right\}
\end{align*}
$\forall f:X \times Y \rightarrow{}^{*}\mathbb{R} \in \mathcal{M}_{\left( X \times Y, \overline{\varSigma \otimes T}, \overline{\mu \otimes \nu} \right)}$に対し、次のうちどれか1つでも成り立つなら、
\begin{align*}
\int_{X} {\int_{Y} {|f|\nu}\mu} < \infty,\ \ \int_{Y} {\int_{X} {|f|\mu}\nu} < \infty,\ \ \int_{X \times Y} {|f|\mu \otimes \nu} < \infty
\end{align*}
次のことが成り立つ。
\begin{itemize}
\item
  次式が成り立つ。
\begin{align*}
\int_{X} {\int_{Y} {|f|\nu}\mu} = \int_{Y} {\int_{X} {|f|\mu}\nu} = \int_{X \times Y} {|f|\mu \otimes \nu}
\end{align*}
\item
  $(X,\varSigma,\mu) \ \text{-} \ \mathrm{a.e.}x \in X$に対し、その写像$f$は、変数$y$の写像とみたとき、その集合$Y$で定積分可能であるかつ、$(Y,T,\nu) \ \text{-} \ \mathrm{a.e.}y \in Y$に対し、その写像$f$は、変数$x$の写像とみたとき、その集合$X$で定積分可能である。
\item
  次のような写像たち$I_{f,X}$、$I_{f,Y}$はいづれもそれぞれそれらの集合たち$X$、$Y$で定積分可能である。
\begin{align*}
I_{f,X}:X \rightarrow \mathrm{cl}\mathbb{R}^{+};x \mapsto \int_{Y} {f\nu},\ \ I_{f,Y}:Y \rightarrow \mathrm{cl}\mathbb{R}^{+};y \mapsto \int_{X} {f\mu}
\end{align*}
\item
  次式が成り立つ。
\begin{align*}
\int_{X} {\int_{Y} {f\nu}\mu} = \int_{Y} {\int_{X} {f\mu}\nu} = \int_{X \times Y} {f\mu \otimes \nu}
\end{align*}
\end{itemize}
この定理を完備測度に関するFubini-Tonelliの定理という。
\end{thm}
\begin{proof} 定理\ref{4.6.4.5}、即ち、Fubini-Tonelliの定理と同様にして示される。
\end{proof}
\begin{thebibliography}{50}
\bibitem{1}
  伊藤清三, ルベーグ積分入門, 裳華房, 1963. 新装第1版2刷 p53-61 ISBN978-4-7853-1318-0
\bibitem{2}
  Mathpedia. "測度と積分". Mathpedia. \url{https://math.jp/wiki/%E6%B8%AC%E5%BA%A6%E3%81%A8%E7%A9%8D%E5%88%86} (2021-7-12 9:20 閲覧)
\bibitem{3}
  岩田耕一郎. "測度と積分". 広島大学. \url{https://home.hiroshima-u.ac.jp/iwatakch/analysisA/lecturenote/analysisA2006.pdf} (2022-11-6 15:50 閲覧)
\bibitem{4}
  山本拓人. "フビニの定理,トネリの定理,フビニ・トネリの定理のまとめ". あーるえぬ. \url{https://math-note.xyz/analysis/measure-theory/fubini-theorem-and-tonelli-theorem/} (2022-3-15 13:22 閲覧)
\end{thebibliography}
\end{document}