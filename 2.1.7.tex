\documentclass[dvipdfmx]{jsarticle}
\setcounter{section}{1}
\setcounter{subsection}{6}
\usepackage{xr}
\externaldocument{2.1.4}
\externaldocument{2.1.6}
\usepackage{amsmath,amsfonts,amssymb,array,comment,mathtools,url,docmute}
\usepackage{longtable,booktabs,dcolumn,tabularx,mathtools,multirow,colortbl,xcolor}
\usepackage[dvipdfmx]{graphics}
\usepackage{bmpsize}
\usepackage{amsthm}
\usepackage{enumitem}
\setlistdepth{20}
\renewlist{itemize}{itemize}{20}
\setlist[itemize]{label=•}
\renewlist{enumerate}{enumerate}{20}
\setlist[enumerate]{label=\arabic*.}
\setcounter{MaxMatrixCols}{20}
\setcounter{tocdepth}{3}
\newcommand{\rotin}{\text{\rotatebox[origin=c]{90}{$\in $}}}
\newcommand{\amap}[6]{\text{\raisebox{-0.7cm}{\begin{tikzpicture} 
  \node (a) at (0, 1) {$\textstyle{#2}$};
  \node (b) at (#6, 1) {$\textstyle{#3}$};
  \node (c) at (0, 0) {$\textstyle{#4}$};
  \node (d) at (#6, 0) {$\textstyle{#5}$};
  \node (x) at (0, 0.5) {$\rotin $};
  \node (x) at (#6, 0.5) {$\rotin $};
  \draw[->] (a) to node[xshift=0pt, yshift=7pt] {$\textstyle{\scriptstyle{#1}}$} (b);
  \draw[|->] (c) to node[xshift=0pt, yshift=7pt] {$\textstyle{\scriptstyle{#1}}$} (d);
\end{tikzpicture}}}}
\newcommand{\twomaps}[9]{\text{\raisebox{-0.7cm}{\begin{tikzpicture} 
  \node (a) at (0, 1) {$\textstyle{#3}$};
  \node (b) at (#9, 1) {$\textstyle{#4}$};
  \node (c) at (#9+#9, 1) {$\textstyle{#5}$};
  \node (d) at (0, 0) {$\textstyle{#6}$};
  \node (e) at (#9, 0) {$\textstyle{#7}$};
  \node (f) at (#9+#9, 0) {$\textstyle{#8}$};
  \node (x) at (0, 0.5) {$\rotin $};
  \node (x) at (#9, 0.5) {$\rotin $};
  \node (x) at (#9+#9, 0.5) {$\rotin $};
  \draw[->] (a) to node[xshift=0pt, yshift=7pt] {$\textstyle{\scriptstyle{#1}}$} (b);
  \draw[|->] (d) to node[xshift=0pt, yshift=7pt] {$\textstyle{\scriptstyle{#2}}$} (e);
  \draw[->] (b) to node[xshift=0pt, yshift=7pt] {$\textstyle{\scriptstyle{#1}}$} (c);
  \draw[|->] (e) to node[xshift=0pt, yshift=7pt] {$\textstyle{\scriptstyle{#2}}$} (f);
\end{tikzpicture}}}}
\renewcommand{\thesection}{第\arabic{section}部}
\renewcommand{\thesubsection}{\arabic{section}.\arabic{subsection}}
\renewcommand{\thesubsubsection}{\arabic{section}.\arabic{subsection}.\arabic{subsubsection}}
\everymath{\displaystyle}
\allowdisplaybreaks[4]
\usepackage{vtable}
\theoremstyle{definition}
\newtheorem{thm}{定理}[subsection]
\newtheorem*{thm*}{定理}
\newtheorem{dfn}{定義}[subsection]
\newtheorem*{dfn*}{定義}
\newtheorem{axs}[dfn]{公理}
\newtheorem*{axs*}{公理}
\renewcommand{\headfont}{\bfseries}
\makeatletter
  \renewcommand{\section}{%
    \@startsection{section}{1}{\z@}%
    {\Cvs}{\Cvs}%
    {\normalfont\huge\headfont\raggedright}}
\makeatother
\makeatletter
  \renewcommand{\subsection}{%
    \@startsection{subsection}{2}{\z@}%
    {0.5\Cvs}{0.5\Cvs}%
    {\normalfont\LARGE\headfont\raggedright}}
\makeatother
\makeatletter
  \renewcommand{\subsubsection}{%
    \@startsection{subsubsection}{3}{\z@}%
    {0.4\Cvs}{0.4\Cvs}%
    {\normalfont\Large\headfont\raggedright}}
\makeatother
\makeatletter
\renewenvironment{proof}[1][\proofname]{\par
  \pushQED{\qed}%
  \normalfont \topsep6\p@\@plus6\p@\relax
  \trivlist
  \item\relax
  {
  #1\@addpunct{.}}\hspace\labelsep\ignorespaces
}{%
  \popQED\endtrivlist\@endpefalse
}
\makeatother
\renewcommand{\proofname}{\textbf{証明}}
\usepackage{tikz,graphics}
\usepackage[dvipdfmx]{hyperref}
\usepackage{pxjahyper}
\hypersetup{
 setpagesize=false,
 bookmarks=true,
 bookmarksdepth=tocdepth,
 bookmarksnumbered=true,
 colorlinks=false,
 pdftitle={},
 pdfsubject={},
 pdfauthor={},
 pdfkeywords={}}
\begin{document}
%\hypertarget{ux884cux5217ux306eux57faux672cux5909ux5f62}{%
\subsection{行列の基本変形}%\label{ux884cux5217ux306eux57faux672cux5909ux5f62}}
%\hypertarget{ux57faux672cux884cux5217}{%
\subsubsection{基本行列}%\label{ux57faux672cux884cux5217}}
\begin{dfn}
可換環$R$上で$\forall i',j' \in \varLambda_{n}\forall k \in R$に対し、$i' \neq j'$が成り立つとき、集合$M_{nn}(R)$に属する3つの行列たち$P_{n}\left( i',k \right)$、$Q_{n}\left( i',j',k \right)$、$R_{n}\left( i',j' \right)$を次式のように定義する。このような行列たちを基本行列などという\footnote{block行列や行列の成分表示では$i' < j'$と仮定されている。以下同様である。}。
\begin{align*}
P_{n}\left( i',k \right) &= \begin{pmatrix}
I_{i' - 1} & \  & O \\
\  & k & \  \\
O & \  & I_{n - i'} \\
\end{pmatrix}\ \mathrm{if}\ \exists\frac{1}{k} \in R\left[ k\frac{1}{k} = \frac{1}{k}k = 1 \right]\\
Q_{n}\left( i',j',k \right) &= \begin{pmatrix}
I_{i' - 1} & \  & \  & \  & O \\
\  & 1 & \  & k & \  \\
\  & \  & I_{j' - i' - 1} & \  & \  \\
\  & \  & \  & 1 & \  \\
O & \  & \  & \  & I_{n - j'} \\
\end{pmatrix}\\
R_{n}\left( i',j' \right) &= \begin{pmatrix}
I_{i' - 1} & \  & \  & \  & O \\
\  & 0 & \  & 1 & \  \\
\  & \  & I_{j' - i' - 1} & \  & \  \\
\  & 1 & \  & 0 & \  \\
O & \  & \  & \  & I_{n - j'} \\
\end{pmatrix}
\end{align*}
\end{dfn}
\begin{thm}\label{2.1.7.1}
基本行列について、次式が成り立つ。
\begin{align*}
{P_{n}\left( i',k \right)}^{- 1} &= P_{n}\left( i',\frac{1}{k} \right)\\
{Q_{n}\left( i',j',k \right)}^{- 1} &= Q_{n}\left( i',j', - k \right)\\
{R_{n}\left( i',j' \right)}^{- 1} &= R_{n}\left( i',j' \right)
\end{align*}
\end{thm}\par
これにより、基本行列は可逆行列である。
\begin{proof}
可換環$R$上で$\forall i',j' \in \varLambda_{n}\forall k \in R$に対し、$i' \neq j'$が成り立つとき、集合$M_{nn}(R)$に属する3つの基本行列たち$P_{n}\left( i',k \right)$、$Q_{n}\left( i',j',k \right)$、$R_{n}\left( i',j' \right)$が与えられ、行列$P_{n}\left( i',k \right)$について、次のようになる。
\begin{align*}
P_{n}\left( i',k \right)P_{n}\left( i',\frac{1}{k} \right) &= \begin{pmatrix}
I_{i' - 1} & \  & O \\
\  & k & \  \\
O & \  & I_{n - i'} \\
\end{pmatrix}\begin{pmatrix}
I_{i' - 1} & \  & O \\
\  & \frac{1}{k} & \  \\
O & \  & I_{n - i'} \\
\end{pmatrix}\\
&= \begin{pmatrix}
I_{i' - 1}I_{i' - 1} & O & O \\
O & k\frac{1}{k} & O \\
O & O & I_{n - i'}I_{n - i'} \\
\end{pmatrix}\\
&= \begin{pmatrix}
I_{i' - 1} & \  & O \\
\  & 1 & \  \\
O & \  & I_{n - i'} \\
\end{pmatrix} = I_{n}
\end{align*}
同様にして、次式が得られる。
\begin{align*}
P_{n}\left( i',\frac{1}{k} \right)P_{n}\left( i',k \right) = I_{n}
\end{align*}
以上より次式が成り立ち、その行列$P_{n}\left( i',k \right)$の逆行列は行列$P_{n}\left( i',\frac{1}{k} \right)$である。
\begin{align*}
P_{n}\left( i',k \right)P_{n}\left( i',\frac{1}{k} \right) = P_{n}\left( i',\frac{1}{k} \right)P_{n}\left( i',k \right) = I_{n}
\end{align*}\par
また、行列$Q_{n}\left( i',j',k \right)$について、次のようになる。
\begin{align*}
Q_{n}\left( i',j',k \right)Q_{n}\left( i',j', - k \right) &= \begin{pmatrix}
I_{i' - 1} & \  & \  & \  & O \\
\  & 1 & \  & k & \  \\
\  & \  & I_{j' - i' - 1} & \  & \  \\
\  & \  & \  & 1 & \  \\
O & \  & \  & \  & I_{n - j'} \\
\end{pmatrix}\begin{pmatrix}
I_{i' - 1} & \  & \  & \  & O \\
\  & 1 & \  & - k & \  \\
\  & \  & I_{j' - i' - 1} & \  & \  \\
\  & \  & \  & 1 & \  \\
O & \  & \  & \  & I_{n - j'} \\
\end{pmatrix}\\
&= \begin{pmatrix}
I_{i' - 1}I_{i' - 1} & O & O & O & O \\
O & 1 & O & - k + k & O \\
O & O & I_{j' - i' - 1}I_{j' - i' - 1} & O & O \\
O & O & O & - 0k + 1 & O \\
O & O & O & O & I_{n - j'}I_{n - j'} \\
\end{pmatrix}\\
&= \begin{pmatrix}
I_{i' - 1} & \  & \  & \  & \  \\
\  & 1 & \  & 0 & \  \\
\  & \  & I_{j' - i' - 1} & \  & \  \\
\  & \  & \  & 1 & \  \\
O & \  & \  & \  & I_{n - j'} \\
\end{pmatrix} = I_{n}
\end{align*}
同様にして、次式が得られる。
\begin{align*}
Q_{n}\left( i',j', - k \right)Q_{n}\left( i',j',k \right) = I_{n}
\end{align*}
以上より次式が成り立ち、その行列$Q_{n}\left( i',j' \right)$の逆行列は行列$Q_{n}\left( i',j', - k \right)$である。
\begin{align*}
Q_{n}\left( i',j',k \right)Q_{n}\left( i',j', - k \right) = Q_{n}\left( i',j', - k \right)Q_{n}\left( i',j',k \right) = I_{n}
\end{align*}\par
また、行列$R_{n}\left( i',j' \right)$について、次のようになる。
\begin{align*}
R_{n}\left( i',j' \right)R_{n}\left( i',j' \right) &= \begin{pmatrix}
I_{i' - 1} & \  & \  & \  & O \\
\  & 0 & \  & 1 & \  \\
\  & \  & I_{j' - i' - 1} & \  & \  \\
\  & 1 & \  & 0 & \  \\
O & \  & \  & \  & I_{n - j'} \\
\end{pmatrix} \begin{pmatrix}
I_{i' - 1} & \  & \  & \  & O \\
\  & 0 & \  & 1 & \  \\
\  & \  & I_{j' - i' - 1} & \  & \  \\
\  & 1 & \  & 0 & \  \\
O & \  & \  & \  & I_{n - j'} \\
\end{pmatrix}\\
&= \begin{pmatrix}
I_{i' - 1}I_{i' - 1} & O & O & O & O \\
O & 1 & O & O & O \\
O & O & I_{j' - i' - 1}I_{j' - i' - 1} & O & O \\
O & O & O & 1 & O \\
O & O & O & O & I_{n - j'}I_{n - j'} \\
\end{pmatrix}\\
&= \begin{pmatrix}
I_{i' - 1} & \  & \  & \  & O \\
\  & 1 & \  & \  & \  \\
\  & \  & I_{j' - i' - 1} & \  & \  \\
\  & \  & \  & 1 & \  \\
O & \  & \  & \  & I_{n - j'} \\
\end{pmatrix} = I_{n}
\end{align*}
以上より次式が成り立ち、その行列$R_{n}\left( i',j' \right)$の逆行列はその行列$R_{n}\left( i',j' \right)$自身である。
\begin{align*}
R_{n}\left( i',j' \right)R_{n}\left( i',j' \right) = I_{n}
\end{align*}
\end{proof}
%\hypertarget{ux884cux5217ux306eux57faux672cux5909ux5f62-1}{%
\subsubsection{行列の基本変形}%\label{ux884cux5217ux306eux57faux672cux5909ux5f62-1}}
\begin{dfn}
可換環$R$上で$\forall i',j' \in \varLambda_{n}\forall k \in R$に対し、$i' \neq j'$が成り立つとき、次のような写像たちを考えられることができる。
\begin{align*}
(\mathrm{C1})\left( n,i',k \right)&:M_{mn}(R) \rightarrow M_{mn}(R);A_{mn} \mapsto A_{mn}P_{n}\left( i',k \right)\ \mathrm{if}\ \exists\frac{1}{k} \in R\left[ k\frac{1}{k} = \frac{1}{k}k = 1 \right]\\
(\mathrm{C2})\left( n,i',j',k \right)&:M_{mn}(R) \rightarrow M_{mn}(R);A_{mn} \mapsto A_{mn}Q_{n}\left( i',j',k \right)\\
(\mathrm{C3})\left( n,i',j' \right)&:M_{mn}(R) \rightarrow M_{mn}(R);A_{mn} \mapsto A_{mn}R_{n}\left( i',j' \right)\\
(\mathrm{R1})\left( m,i',k \right)&:M_{mn}(R) \rightarrow M_{mn}(R);A_{mn} \mapsto P_{m}\left( i',k \right)A_{mn}\ \mathrm{if}\ \exists\frac{1}{k} \in R\left[ k\frac{1}{k} = \frac{1}{k}k = 1 \right]\\
(\mathrm{R2})\left( m,i',j',k \right)&:M_{mn}(R) \rightarrow M_{mn}(R);A_{mn} \mapsto Q_{m}\left( i',j',k \right)A_{mn}\\
(\mathrm{R3})\left( m,i',j' \right)&:M_{mn}(R) \rightarrow M_{mn}(R);A_{mn} \mapsto R_{m}\left( i',j' \right)A_{mn}
\end{align*}
\end{dfn}
\begin{thm}\label{2.1.7.2}
可換環$R$上でこれらの写像たちは全単射で次のような逆写像たちが存在する。
\begin{align*}
{(\mathrm{C1})\left( n,i',k \right)}^{- 1}&:M_{mn}(R) \rightarrow M_{mn}(R);A_{mn} \mapsto A_{mn}P_{n}\left( i',\frac{1}{k} \right)\\
{(\mathrm{C2})\left( n,i',j',k \right)}^{- 1}&:M_{mn}(R) \rightarrow M_{mn}(R);A_{mn} \mapsto A_{mn}Q_{n}\left( i',j', - k \right)\\
{(\mathrm{C3})\left( n,i',j' \right)}^{- 1}&:M_{mn}(R) \rightarrow M_{mn}(R);A_{mn} \mapsto A_{mn}R_{n}\left( i',j' \right)\\
{(\mathrm{R1})\left( m,i',k \right)}^{- 1}&:M_{mn}(R) \rightarrow M_{mn}(R);A_{mn} \mapsto P_{m}\left( i',\frac{1}{k} \right)A_{mn}\\
{(\mathrm{R2})\left( m,i',j',k \right)}^{- 1}&:M_{mn}(R) \rightarrow M_{mn}(R);A_{mn} \mapsto Q_{m}\left( i',j', - k \right)A_{mn}\\
{(\mathrm{R3})\left( m,i',j' \right)}^{- 1}&:M_{mn}(R) \rightarrow M_{mn}(R);A_{mn} \mapsto R_{m}\left( i',j' \right)A_{mn}
\end{align*}
\end{thm}
\begin{proof} ほとんど明らかである。
\end{proof}
\begin{thm}\label{2.1.7.3}
可換環$R$上で、$\forall A_{mn} \in M_{mn}(R)$に対し、$A_{mn} = \left( a_{ij} \right)_{(i,j) \in \varLambda_{m} \times \varLambda_{n}}$とおかれれば、次式が成り立つ。
\begin{align*}
(\mathrm{C1})\left( n,i',k \right)\left( A_{mn} \right) &= A_{mn}P_{n}\left( i',k \right) = \begin{pmatrix}
a_{11} & \cdots & ka_{1i'} & \cdots & a_{1n} \\
 \vdots & \ddots & \vdots & \ddots & \vdots \\
a_{m1} & \cdots & ka_{mi'} & \cdots & a_{mn} \\
\end{pmatrix}\\
(\mathrm{C2})\left( n,i',j',k \right)\left( A_{mn} \right) &= A_{mn}Q_{n}\left( i',j',k \right) = \begin{pmatrix}
a_{11} & \cdots & a_{1i'} & \cdots & a_{1j'} + ka_{1i'} & \cdots & a_{1n} \\
 \vdots & \ddots & \vdots & \ddots & \vdots & \ddots & \vdots \\
a_{m1} & \cdots & a_{mi'} & \cdots & a_{mj'} + ka_{mi'} & \cdots & a_{mn} \\
\end{pmatrix}\\
(\mathrm{C3})\left( n,i',j' \right)\left( A_{mn} \right) &= A_{mn}R_{n}\left( i',j' \right) = \begin{pmatrix}
a_{11} & \cdots & a_{1j'} & \cdots & a_{1i'} & \cdots & a_{1n} \\
 \vdots & \ddots & \vdots & \ddots & \vdots & \ddots & \vdots \\
a_{m1} & \cdots & a_{mj'} & \cdots & a_{mi'} & \cdots & a_{mn} \\
\end{pmatrix}\\
(\mathrm{R1})\left( m,i',k \right)\left( A_{mn} \right) &= P_{m}\left( i',k \right)A_{mn} = \begin{pmatrix}
a_{11} & \cdots & a_{1n} \\
 \vdots & \ddots & \vdots \\
ka_{i'1} & \cdots & ka_{i'n} \\
 \vdots & \ddots & \vdots \\
a_{m1} & \cdots & a_{mn} \\
\end{pmatrix}\\
(\mathrm{R2})\left( m,i',j',k \right)\left( A_{mn} \right) &= Q_{m}\left( i',j',k \right)A_{mn} = \begin{pmatrix}
a_{11} & \cdots & a_{1n} \\
 \vdots & \ddots & \vdots \\
a_{i'1} & \cdots & a_{i'n} \\
 \vdots & \ddots & \vdots \\
a_{j'1} + ka_{i'1} & \cdots & a_{j'n} + ka_{i'n} \\
 \vdots & \ddots & \vdots \\
a_{m1} & \cdots & a_{mn} \\
\end{pmatrix}\\
(\mathrm{R3})\left( m,i',j' \right)\left( A_{mn} \right) &= R_{m}\left( i',j' \right)A_{mn} = \begin{pmatrix}
a_{11} & \cdots & a_{1n} \\
 \vdots & \ddots & \vdots \\
a_{j'1} & \cdots & a_{j'n} \\
 \vdots & \ddots & \vdots \\
a_{i'1} & \cdots & a_{i'n} \\
 \vdots & \ddots & \vdots \\
a_{m1} & \cdots & a_{mn} \\
\end{pmatrix}
\end{align*}
\end{thm}
\begin{proof}
成分表示して計算すればよい。
\end{proof}\par
これにより、次のことがわかる。
\begin{itemize}
\item
  行列$A_{mn}$が写像$(\mathrm{C1})\left( n,i',k \right)$によってうつされると、即ち、その行列$A_{mn}$の右に行列$P_{n}\left( i',k \right)$との積をとると、その行列$A_{mn}$の第$i'$列の各成分が$k$倍される。
\item
  行列$A_{mn}$が写像$(\mathrm{C2})\left( n,i',j',k \right)$によってうつされると、即ち、その行列$A_{mn}$の右に行列$Q_{n}\left( i',j' \right)$との積をとると、その行列$A_{mn}$の第$j'$列の各成分に$k$倍されたその第$i'$列の同じ行ごとの各成分が加わる。
\item
  行列$A_{mn}$が写像$(\mathrm{C3})\left( n,i',j' \right)$によってうつされると、即ち、その行列$A_{mn}$の右に行列$R_{n}\left( i',j' \right)$との積をとると、その行列$A_{mn}$の第$i'列と第j'$列が互いに入れ替わる。
\item
  行列$A_{mn}$が写像$(\mathrm{R1})\left( m,i',k \right)$によってうつされると、即ち、その行列$A_{mn}$の左に行列$P_{m}\left( i',k \right)$との積をとると、その行列$A_{mn}$の第$i'$行の各成分が$k$倍される。
\item
  行列$A_{mn}$が写像$(\mathrm{R2})\left( m,i',j',k \right)$によってうつされると、即ち、その行列$A_{mn}$の左に行列$Q_{m}\left( i',j' \right)$との積をとると、その行列$A_{mn}$の第$j'$行の各成分に$k$倍されたその第$i'$行の同じ列ごとの各成分が加わる。
\item
  行列$A_{mn}$が写像$(\mathrm{R3})\left( m,i',j' \right)$によってうつされると、即ち、その行列$A_{mn}$の左に行列$R_{m}\left( i',j' \right)$とのをとると、その行列$A_{mn}$の第$i'行と第j'$行が互いに入れ替わる。
\end{itemize}
\begin{dfn}
可換環$R$上の行列に関して、写像たち$(\mathrm{C1})$、$(\mathrm{C2})$、$(\mathrm{C3})$を用いた次の操作を列基本変形といい、これらを組み合わせたものを列変形という。
\begin{itemize}
\item
  ある2つの列々を入れかえる。
\item
  $\forall k \in R$に対し、その元$k$が可逆元であるなら、ある1つの列の成分全体を$k$倍する。
\item
  $\forall k \in R$に対し、ある1つの列の各成分の$k$倍を別の1つの列の対応する各成分に加える。
\end{itemize}
同様にして、写像たち$(\mathrm{R1})$、$(\mathrm{R2})$、$(\mathrm{R3})$を用いた次の操作を行基本変形といい、これらを組み合わせたものを行変形という
\begin{itemize}
\item
  ある2つの行々を入れかえる。
\item
  $\forall k \in R$に対し、その元$k$が可逆元であるなら、ある1つの行の成分全体を$k$倍する。
\item
  $\forall k \in R$に対し、ある1つの行の各成分の$k$倍を別の1つの行の対応する各成分に加える。
\end{itemize}
以上の列基本変形、行基本変形合わせて行列の基本変形、または単に、基本変形といい、これらを組み合わせたものを行列の変形という。\par
$A_{mn} \in M_{mn}(R)$なる行列$A_{mn}$に写像$f$による行列の変形を行った後の行列を$A_{mn}'$とおくと、この関係は次式のように書かれることが多い。
\begin{align*}
A_{mn} \rightarrow A_{mn}',\ \ A_{mn}\overset{f}{\rightarrow}A_{mn}'
\end{align*}
\end{dfn}
%\hypertarget{ux4f53kux4e0aux306eux884cux5217ux306eux6a19ux6e96ux5f62}{%
\subsubsection{体$K$上の行列の標準形}%\label{ux4f53kux4e0aux306eux884cux5217ux306eux6a19ux6e96ux5f62}}
\begin{thm}\label{2.1.7.3}
体$K$上で、$\forall A_{mn} \in M_{mn}(K)$に対し、その行列$A_{mn}$が行列の基本変形をされたとしても、その行列の階数は一定である。\par
このことは行列の基本変形が右あるいは左から基本行列をかけているだけにすぎないことと定理\ref{2.1.6.7}より明らかであるが、別の証明も与えておこう。
\end{thm}
\begin{proof}
まず、列基本変形の場合を示そう。体$K$上で、$\forall A_{mn} \in M_{mn}(K)$に対し、$A_{mn} = \left( \mathbf{a}_{j} \right)_{j \in \varLambda_{n}}$とおかれると、その行列の列空間${\mathrm{span}}\left\{ \mathbf{a}_{j} \right\}_{j \in \varLambda_{n}}$は$\forall j \in \varLambda_{n}$に対するvectors$\mathbf{a}_{j}$によって生成される集合$K^{m}$の部分空間である。したがって、次式が成り立つ。
\begin{align*}
{\mathrm{span}}\left\{ \mathbf{a}_{j} \right\}_{j \in \varLambda_{n}} = \left\{ \mathbf{v} \in K^{n} \middle| \mathbf{v} = \sum_{j \in \varLambda_{n}} {k_{j}\mathbf{a}_{j}} \right\}
\end{align*}\par
ここで、$\forall k \in K \setminus \left\{ 0 \right\}$なる元$k$を用いてある第$j'$列の成分全体を$k$倍したとしても、されたあとの行列の列空間${\mathrm{span}}\left\{ \mathbf{a}_{j}' \right\}_{j \in \varLambda_{n}}$について、次のようになり
\begin{align*}
\mathbf{v} &= \sum_{j \in \varLambda_{n}} {k_{j}\mathbf{a}_{j}'}\\
&= \sum_{j \in \varLambda_{n} \setminus \left\{ j' \right\}} {k_{j}\mathbf{a}_{j}'} + k_{j'}\mathbf{a}_{j'}'\\
&= \sum_{j \in \varLambda_{n} \setminus \left\{ j' \right\}} {k_{j}\mathbf{a}_{j}} + k_{j'}k\mathbf{a}_{j'}
\end{align*}
${\mathrm{span}}\left\{ \mathbf{a}_{j}' \right\}_{j \in \varLambda_{n}} = {\mathrm{span}}\left\{ \mathbf{a}_{j} \right\}_{j \in \varLambda_{n}}$が成り立つので、その行列の階数は一定である。\par
また、ある第$i'$列の各成分の$k$倍を別の第$j'$行の対応する各成分に加えたとしても、されたあとの行列の列空間${\mathrm{span}}\left\{ \mathbf{a}_{j}' \right\}_{j \in \varLambda_{n}}$について、次のようになり
\begin{align*}
\mathbf{v} &= \sum_{j \in \varLambda_{n}} {k_{j}\mathbf{a}_{j}'}\\
&= \sum_{j \in \varLambda_{n} \setminus \left\{ i',j' \right\}} {k_{j}\mathbf{a}_{j}'} + k_{i'}\mathbf{a}_{i'}' + k_{j'}\mathbf{a}_{j'}'\\
&= \sum_{j \in \varLambda_{n} \setminus \left\{ i',j' \right\}} {k_{j}\mathbf{a}_{j}} + k_{i'}\mathbf{a}_{i'} + k_{j'}\left( \mathbf{a}_{j'} + k\mathbf{a}_{i'} \right)\\
&= \sum_{j \in \varLambda_{n} \setminus \left\{ i',j' \right\}} {k_{j}\mathbf{a}_{j}} + k_{i'}\mathbf{a}_{i'} + k_{j'}\mathbf{a}_{j'} + k_{j'}k\mathbf{a}_{i'}\\
&= \sum_{j \in \varLambda_{n} \setminus \left\{ i',j' \right\}} {k_{j}\mathbf{a}_{j}} + \left( k_{i'} + k_{j'}k \right)\mathbf{a}_{i'} + k_{j'}\mathbf{a}_{j'}
\end{align*}
${\mathrm{span}}\left\{ \mathbf{a}_{j}' \right\}_{j \in \varLambda_{n}} = {\mathrm{span}}\left\{ \mathbf{a}_{j} \right\}_{j \in \varLambda_{n}}$が成り立つので、その行列の階数は一定である。\par
その行列$A_{mn}$のある2つの列々を入れかえたとしても、されたあとの行列の列空間は${\mathrm{span}}\left\{ \mathbf{a}_{j} \right\}_{j \in \varLambda_{n}}$に等しいので、その行列の階数は一定である。\par
以上より、その行列$A_{mn}$に列基本変形を行ったとしても、その列空間は一定であるので、その行列$A_{mn}$の階数も一定である。\par
同様にして、その行列$A_{mn}$に行基本変形を行ったとしても、その行空間は一定であるので、その行列$A_{mn}$の階数も一定である。
\end{proof}
\begin{thm}\label{2.1.7.4}
体$K$上で、$\forall A_{mn} \in M_{mn}(K)$に対し、その行列$A_{mn}$は行列の基本変形を有限回くり返すと、${\mathrm{rank}}A_{mn} = r$として次のその行列$A_{mn}$の標準形に変形できる。
\begin{align*}
A_{mn} \rightarrow \begin{pmatrix}
I_{r} & O \\
O & O \\
\end{pmatrix} = \begin{pmatrix}
1 & \cdots & 0 & 0 & \cdots & 0 \\
 \vdots & \ddots & \vdots & \vdots & \ddots & \vdots \\
0 & \cdots & 1 & 0 & \cdots & 0 \\
0 & \cdots & 0 & 0 & \cdots & 0 \\
 \vdots & \ddots & \vdots & \vdots & \ddots & \vdots \\
0 & \cdots & 0 & 0 & \cdots & 0 \\
\end{pmatrix}
\end{align*}
後述する証明の議論から分かるように、この変形は次のalgorithmに従う。
\begin{enumerate}
\item
  $\forall k \in \varLambda_{\min\left\{ m,n \right\}}$に対し次式のように表されたとする。
\begin{align*}
A_{mn} \rightarrow \begin{pmatrix}
1 & \cdots & 0 & 0 & \cdots & 0 \\
 \vdots & \ddots & \vdots & \vdots & \ddots & \vdots \\
0 & \cdots & 1 & 0 & \cdots & 0 \\
0 & \cdots & 0 & a_{kk} & \cdots & a_{kn} \\
 \vdots & \ddots & \vdots & \vdots & \ddots & \vdots \\
0 & \cdots & 0 & a_{mk} & \cdots & a_{mn} \\
\end{pmatrix}
\end{align*}
\item
  行列$\begin{pmatrix}
  a_{kk} & \cdots & a_{kn} \\
   \vdots & \ddots & \vdots \\
  a_{mk} & \cdots & a_{mn} \\
  \end{pmatrix}$の各成分がすべて0であれば、9. へ進む。
\item
  そうでないならば、0でない成分$a_{i'j'}$が存在することになりある2つの行々を入れかえる操作とある2つの列々を入れかえる操作でその成分$a_{i'j'}$を第$(k,k)$成分とすることができる。
\item
  このときの第$(i,j)$成分を$a_{ij}'$とおき$\forall i \in \varLambda_{m} \setminus \left\{ k \right\} に対し第k$行の成分全体を$- \frac{a_{ij'}'}{a_{i'j'}}$倍しその各成分を第$i$行の対応する各成分に加える。
\item
  このときの第$(i,j)$成分を$a_{ij}''$とおき$\forall j \in \varLambda_{n} \setminus \left\{ k \right\} に対し第k$列の成分全体を$- \frac{a_{i'j}''}{a_{i'j'}}$倍しその各成分を第$j$列の対応する各成分に加える。
\item
  第$k$行の成分全体を$\frac{1}{a_{i'j'}}$倍する。そうすると、$\forall i \in \varLambda_{m} \setminus \left\{ k \right\} に対し第(i,k)$成分は0と、$\forall j \in \varLambda_{n} \setminus \left\{ k \right\}$に対し第$(k,j)$成分は0と、第$(k,k)$成分は1となる。
\item
  $k = \min\left\{ m,n \right\}$が成り立つなら、9. へ進む。
\item
  そうでないならば、1. へ戻る。
\item
  このalgorithmは終了する。
\end{enumerate}
\end{thm}
\begin{proof}
体$K$上で、$\forall A_{mn} \in M_{mn}(K)$に対し、$A_{mn} = \left( a_{ij} \right)_{(i,j) \in \varLambda_{m} \times \varLambda_{n}}$とおかれると、その各成分が0であれば、この行列$A_{mn}$は零行列$O$でありその行列$A_{mn}$の階数${\mathrm{rank}}A_{mn}$は0であるから、明らかである。\par
0でないその行列$A_{mn}$の成分$a_{i'j'}$が存在すれば、ある2つの行々を入れかえる操作とある2つの列々を入れかえる操作でその成分を第$(1,1)$成分とすることができる。
\begin{align*}
\quad A_{mn} = \begin{pmatrix}
a_{11} & \cdots & a_{1j'} & \cdots & a_{1n} \\
 \vdots & \ddots & \vdots & \ddots & \vdots \\
a_{i'1} & \cdots & a_{i'j'} & \cdots & a_{i'n} \\
 \vdots & \ddots & \vdots & \ddots & \vdots \\
a_{m1} & \cdots & a_{mj'} & \cdots & a_{mn} \\
\end{pmatrix} &\rightarrow \begin{pmatrix}
a_{i'1} & \cdots & a_{i'j'} & \cdots & a_{i'n} \\
 \vdots & \ddots & \vdots & \ddots & \vdots \\
a_{11} & \cdots & a_{1j'} & \cdots & a_{1n} \\
 \vdots & \ddots & \vdots & \ddots & \vdots \\
a_{m1} & \cdots & a_{mj'} & \cdots & a_{mn} \\
\end{pmatrix}\\
&\rightarrow \begin{pmatrix}
a_{i'j'} & \cdots & a_{i'1} & \cdots & a_{i'n} \\
 \vdots & \ddots & \vdots & \ddots & \vdots \\
a_{1j'} & \cdots & a_{11} & \cdots & a_{1n} \\
 \vdots & \ddots & \vdots & \ddots & \vdots \\
a_{mj'} & \cdots & a_{m1} & \cdots & a_{mn} \\
\end{pmatrix}
\end{align*}
このときの第$(i,j)$成分を$a_{ij}'$とおき$\forall i \in \varLambda_{m} \setminus \left\{ 1 \right\}$に対し第1行の成分全体を$- \frac{a_{ij'}'}{a_{i'j'}}$倍しその各成分を第$i$行の対応する各成分に加えると、$\forall i \in \varLambda_{m} \setminus \left\{ 1 \right\}$に対し第$(i,1)$成分は0となる。
\begin{align*}
A_{mn} &\rightarrow \begin{pmatrix}
a_{i'j'} & \cdots & a_{i'1} & \cdots & a_{i'n} \\
 \vdots & \ddots & \vdots & \ddots & \vdots \\
a_{1j'} - \frac{a_{1j'}}{a_{i'j'}}a_{i'j'} & \cdots & a_{11} - \frac{a_{kj'}}{a_{i'j'}}a_{i'1} & \cdots & a_{1n} - \frac{a_{1j'}}{a_{i'j'}}a_{i'n} \\
 \vdots & \ddots & \vdots & \ddots & \vdots \\
a_{mj'} - \frac{a_{mj'}}{a_{i'j'}}a_{i'j'} & \cdots & a_{m1} - \frac{a_{mj'}}{a_{i'j'}}a_{i'1} & \cdots & a_{mn} - \frac{a_{mj'}}{a_{i'j'}}a_{i'n} \\
\end{pmatrix}\\
&\rightarrow \begin{pmatrix}
a_{i'j'} & \cdots & a_{i'1} & \cdots & a_{i'n} \\
 \vdots & \ddots & \vdots & \ddots & \vdots \\
0 & \cdots & a_{11} - \frac{a_{kj'}}{a_{i'j'}}a_{i'1} & \cdots & a_{1n} - \frac{a_{1j'}}{a_{i'j'}}a_{i'n} \\
 \vdots & \ddots & \vdots & \ddots & \vdots \\
0 & \cdots & a_{m1} - \frac{a_{mj'}}{a_{i'j'}}a_{i'1} & \cdots & a_{mn} - \frac{a_{mj'}}{a_{i'j'}}a_{i'n} \\
\end{pmatrix}
\end{align*}
このときの第$(i,j)$成分を$a_{i,j}''$とおき$\forall j \in \varLambda_{n} \setminus \left\{ 1 \right\}$に対し第1列の成分全体を$- \frac{a_{i'j}''}{a_{i'j'}}$倍しその各成分を第$j$列の対応する各成分に加えると、$\forall j \in \varLambda_{n} \setminus \left\{ 1 \right\}$に対し第$(1,j)$成分は0となる。
\begin{align*}
A_{mn} &\rightarrow \begin{pmatrix}
a_{i'j'} & \cdots & a_{i'1} - \frac{a_{i'1}}{a_{i'j'}}a_{i'j'} & \cdots & a_{i'n} - \frac{a_{i'n}}{a_{i'j'}}a_{i'j'} \\
 \vdots & \ddots & \vdots & \ddots & \vdots \\
0 & \cdots & a_{11} - \frac{a_{kj'}}{a_{i'j'}}a_{i'1} & \cdots & a_{1n} - \frac{a_{1j'}}{a_{i'j'}}a_{i'n} \\
 \vdots & \ddots & \vdots & \ddots & \vdots \\
0 & \cdots & a_{m1} - \frac{a_{mj'}}{a_{i'j'}}a_{i'1} & \cdots & a_{mn} - \frac{a_{mj'}}{a_{i'j'}}a_{i'n} \\
\end{pmatrix}\\
&\rightarrow \begin{pmatrix}
a_{i'j'} & \cdots & 0 & \cdots & 0 \\
 \vdots & \ddots & \vdots & \ddots & \vdots \\
0 & \cdots & a_{11} - \frac{a_{kj'}}{a_{i'j'}}a_{i'1} & \cdots & a_{1n} - \frac{a_{1j'}}{a_{i'j'}}a_{i'n} \\
 \vdots & \ddots & \vdots & \ddots & \vdots \\
0 & \cdots & a_{m1} - \frac{a_{mj'}}{a_{i'j'}}a_{i'1} & \cdots & a_{mn} - \frac{a_{mj'}}{a_{i'j'}}a_{i'n} \\
\end{pmatrix}
\end{align*}
第1行の成分全体を$\frac{1}{a_{i'j'}}$倍すると、次の行列が得られる。
\begin{align*}
A_{mn} \rightarrow \begin{pmatrix}
1 & \cdots & 0 & \cdots & 0 \\
 \vdots & \ddots & \vdots & \ddots & \vdots \\
0 & \cdots & a_{11} - \frac{a_{kj'}}{a_{i'j'}}a_{i'1} & \cdots & a_{1n} - \frac{a_{1j'}}{a_{i'j'}}a_{i'n} \\
 \vdots & \ddots & \vdots & \ddots & \vdots \\
0 & \cdots & a_{m1} - \frac{a_{mj'}}{a_{i'j'}}a_{i'1} & \cdots & a_{mn} - \frac{a_{mj'}}{a_{i'j'}}a_{i'n} \\
\end{pmatrix}
\end{align*}
$k \in \varLambda_{\min\left\{ m,n \right\}}$に対し次式のように表されたとする。このとき、行列$\begin{pmatrix}
a_{kk} & \cdots & a_{kn} \\
 \vdots & \ddots & \vdots \\
a_{mk} & \cdots & a_{mn} \\
\end{pmatrix}$の各成分がすべて$0$であるなら、その行列$A_{mn}$は$\begin{pmatrix}
I_{k} & O \\
O & O \\
\end{pmatrix}$と変形されており、その行列$\begin{pmatrix}
I_{k} & O_{k,n - k} \\
\end{pmatrix}$の行vectorsはその行列$\begin{pmatrix}
I_{k} & O \\
O & O \\
\end{pmatrix}$の行空間を張り線形独立でありその行空間の基底をなしそれらの行vectorsが$k$つあるから、その行列$\begin{pmatrix}
I_{k} & O \\
O & O \\
\end{pmatrix}$の階数${\mathrm{rank}}\begin{pmatrix}
I_{k} & O \\
O & O \\
\end{pmatrix}$は$k$に等しい。ここで、ある行列に行列の基本変形を行ったとしても、その行列の階数は一定であったので、その行列$A_{mn}$の階数${\mathrm{rank}}A_{mn}$は$k$に等しい。\par
逆に、0でないその行列$\begin{pmatrix}
a_{kk} & \cdots & a_{kn} \\
 \vdots & \ddots & \vdots \\
a_{mk} & \cdots & a_{mn} \\
\end{pmatrix}$の成分$a_{i'j'}$が存在すれば、ある2つの行々を入れかえる操作とある2つの列々を入れかえる操作でその成分を第$(k,k)$成分とすることができる。
\begin{align*}
A_{mn} &\rightarrow \begin{pmatrix}
1 & \cdots & 0 & 0 & \cdots & 0 & \cdots & 0 \\
 \vdots & \ddots & \vdots & \vdots & \ddots & \vdots & \ddots & \vdots \\
0 & \cdots & 1 & 0 & \cdots & 0 & \cdots & 0 \\
0 & \cdots & 0 & a_{kk} & \cdots & a_{kj'} & \cdots & a_{kn} \\
 \vdots & \ddots & \vdots & \vdots & \ddots & \vdots & \ddots & \vdots \\
0 & \cdots & 0 & a_{i'k} & \cdots & a_{i'j'} & \cdots & a_{i'n} \\
 \vdots & \ddots & \vdots & \vdots & \ddots & \vdots & \ddots & \vdots \\
0 & \cdots & 0 & a_{mk} & \cdots & a_{mj'} & \cdots & a_{mn} \\
\end{pmatrix}\\
&\rightarrow \begin{pmatrix}
1 & \cdots & 0 & 0 & \cdots & 0 & \cdots & 0 \\
 \vdots & \ddots & \vdots & \vdots & \ddots & \vdots & \ddots & \vdots \\
0 & \cdots & 1 & 0 & \cdots & 0 & \cdots & 0 \\
0 & \cdots & 0 & a_{i'k} & \cdots & a_{i'j'} & \cdots & a_{i'n} \\
 \vdots & \ddots & \vdots & \vdots & \ddots & \vdots & \ddots & \vdots \\
0 & \cdots & 0 & a_{kk} & \cdots & a_{kj'} & \cdots & a_{kn} \\
 \vdots & \ddots & \vdots & \vdots & \ddots & \vdots & \ddots & \vdots \\
0 & \cdots & 0 & a_{mk} & \cdots & a_{mj'} & \cdots & a_{mn} \\
\end{pmatrix}\\
&\rightarrow \begin{pmatrix}
1 & \cdots & 0 & 0 & \cdots & 0 & \cdots & 0 \\
 \vdots & \ddots & \vdots & \vdots & \ddots & \vdots & \ddots & \vdots \\
0 & \cdots & 1 & 0 & \cdots & 0 & \cdots & 0 \\
0 & \cdots & 0 & a_{i'j'} & \cdots & a_{i'k} & \cdots & a_{i'n} \\
 \vdots & \ddots & \vdots & \vdots & \ddots & \vdots & \ddots & \vdots \\
0 & \cdots & 0 & a_{kj'} & \cdots & a_{kk} & \cdots & a_{kn} \\
 \vdots & \ddots & \vdots & \vdots & \ddots & \vdots & \ddots & \vdots \\
0 & \cdots & 0 & a_{mj'} & \cdots & a_{mk} & \cdots & a_{mn} \\
\end{pmatrix}
\end{align*}
このときの第$(i,j)$成分を$a_{ij}'$とおき$\forall i \in \varLambda_{m} \setminus \left\{ k \right\}$に対し第$k$行の成分全体を$- \frac{a_{ij'}'}{a_{i'j'}}$倍しその各成分を第$i$行の対応する各成分に加えると、$\forall i \in \varLambda_{m} \setminus \left\{ k \right\}$に対し第$(i,k)$成分は0となる。
\begin{align*}
A_{mn} &\rightarrow \begin{pmatrix}
1 & \cdots & 0 & 0 & \cdots & 0 & \cdots & 0 \\
 \vdots & \ddots & \vdots & \vdots & \ddots & \vdots & \ddots & \vdots \\
0 & \cdots & 1 & 0 & \cdots & 0 & \cdots & 0 \\
0 & \cdots & 0 & a_{i'j'} & \cdots & a_{i'k} & \cdots & a_{i'n} \\
 \vdots & \ddots & \vdots & \vdots & \ddots & \vdots & \ddots & \vdots \\
0 & \cdots & 0 & a_{kj'} - \frac{a_{kj'}}{a_{i'j'}}a_{i'j'} & \cdots & a_{kk} - \frac{a_{kj'}}{a_{i'j'}}a_{i'k} & \cdots & a_{kn} - \frac{a_{kj'}}{a_{i'j'}}a_{i'n} \\
 \vdots & \ddots & \vdots & \vdots & \ddots & \vdots & \ddots & \vdots \\
0 & \cdots & 0 & a_{mj'} - \frac{a_{mj'}}{a_{i'j'}}a_{i'j'} & \cdots & a_{mk} - \frac{a_{mj'}}{a_{i'j'}}a_{i'k} & \cdots & a_{mn} - \frac{a_{mj'}}{a_{i'j'}}a_{i'n} \\
\end{pmatrix}\\
&\rightarrow \begin{pmatrix}
1 & \cdots & 0 & 0 & \cdots & 0 & \cdots & 0 \\
 \vdots & \ddots & \vdots & \vdots & \ddots & \vdots & \ddots & \vdots \\
0 & \cdots & 1 & 0 & \cdots & 0 & \cdots & 0 \\
0 & \cdots & 0 & a_{i'j'} & \cdots & a_{i'k} & \cdots & a_{i'n} \\
 \vdots & \ddots & \vdots & \vdots & \ddots & \vdots & \ddots & \vdots \\
0 & \cdots & 0 & 0 & \cdots & a_{kk} - \frac{a_{kj'}}{a_{i'j'}}a_{i'k} & \cdots & a_{kn} - \frac{a_{kj'}}{a_{i'j'}}a_{i'n} \\
 \vdots & \ddots & \vdots & \vdots & \ddots & \vdots & \ddots & \vdots \\
0 & \cdots & 0 & 0 & \cdots & a_{mk} - \frac{a_{mj'}}{a_{i'j'}}a_{i'k} & \cdots & a_{mn} - \frac{a_{mj'}}{a_{i'j'}}a_{i'n} \\
\end{pmatrix}
\end{align*}
このときの第$(i,j)$成分を$a_{i,j}'$とおき$\forall j \in \varLambda_{n} \setminus \left\{ k \right\}$に対し第$k$列の成分全体を$- \frac{a_{i',j}'}{a_{i',j'}}$倍しその各成分を第$j$列の対応する各成分に加えると、$\forall j \in \varLambda_{n} \setminus \left\{ k \right\}$に対し第$(k,j)$成分は0となる。
\begin{align*}
A_{mn} &\rightarrow \begin{pmatrix}
1 & \cdots & 0 & 0 & \cdots & 0 & \cdots & 0 \\
 \vdots & \ddots & \vdots & \vdots & \ddots & \vdots & \ddots & \vdots \\
0 & \cdots & 1 & 0 & \cdots & 0 & \cdots & 0 \\
0 & \cdots & 0 & a_{i'j'} & \cdots & a_{i'k} - \frac{a_{i'k}}{a_{i'j'}}a_{i'j'} & \cdots & a_{i'n} - \frac{a_{i'n}}{a_{i'j'}}a_{i'j'} \\
 \vdots & \ddots & \vdots & \vdots & \ddots & \vdots & \ddots & \vdots \\
0 & \cdots & 0 & 0 & \cdots & a_{kk} - \frac{a_{kj'}}{a_{i'j'}}a_{i'k} & \cdots & a_{kn} - \frac{a_{kj'}}{a_{i'j'}}a_{i'n} \\
 \vdots & \ddots & \vdots & \vdots & \ddots & \vdots & \ddots & \vdots \\
0 & \cdots & 0 & 0 & \cdots & a_{mk} - \frac{a_{mj'}}{a_{i'j'}}a_{i'k} & \cdots & a_{mn} - \frac{a_{mj'}}{a_{i'j'}}a_{i'n} \\
\end{pmatrix}\\
&\rightarrow \begin{pmatrix}
1 & \cdots & 0 & 0 & \cdots & 0 & \cdots & 0 \\
 \vdots & \ddots & \vdots & \vdots & \ddots & \vdots & \ddots & \vdots \\
0 & \cdots & 1 & 0 & \cdots & 0 & \cdots & 0 \\
0 & \cdots & 0 & a_{i'j'} & \cdots & 0 & \cdots & 0 \\
 \vdots & \ddots & \vdots & \vdots & \ddots & \vdots & \ddots & \vdots \\
0 & \cdots & 0 & 0 & \cdots & a_{kk} - \frac{a_{kj'}}{a_{i'j'}}a_{i'k} & \cdots & a_{kn} - \frac{a_{kj'}}{a_{i'j'}}a_{i'n} \\
 \vdots & \ddots & \vdots & \vdots & \ddots & \vdots & \ddots & \vdots \\
0 & \cdots & 0 & 0 & \cdots & a_{mk} - \frac{a_{mj'}}{a_{i'j'}}a_{i'k} & \cdots & a_{mn} - \frac{a_{mj'}}{a_{i'j'}}a_{i'n} \\
\end{pmatrix}
\end{align*}
第$k$行の成分全体を$\frac{1}{a_{i',j'}}$倍すると、次の行列が得られる。
\begin{align*}
A_{mn} \rightarrow \begin{pmatrix}
1 & \cdots & 0 & 0 & \cdots & 0 & \cdots & 0 \\
 \vdots & \ddots & \vdots & \vdots & \ddots & \vdots & \ddots & \vdots \\
0 & \cdots & 1 & 0 & \cdots & 0 & \cdots & 0 \\
0 & \cdots & 0 & 1 & \cdots & 0 & \cdots & 0 \\
 \vdots & \ddots & \vdots & \vdots & \ddots & \vdots & \ddots & \vdots \\
0 & \cdots & 0 & 0 & \cdots & a_{kk} - \frac{a_{kj'}}{a_{i'j'}}a_{i'k} & \cdots & a_{kn} - \frac{a_{kj'}}{a_{i'j'}}a_{i'n} \\
 \vdots & \ddots & \vdots & \vdots & \ddots & \vdots & \ddots & \vdots \\
0 & \cdots & 0 & 0 & \cdots & a_{mk} - \frac{a_{mj'}}{a_{i'j'}}a_{i'k} & \cdots & a_{mn} - \frac{a_{mj'}}{a_{i'j'}}a_{i'n} \\
\end{pmatrix}
\end{align*}\par
以上より数学的帰納法によって示すべきことは示された。
\end{proof}
\begin{thm}\label{2.1.7.5}
$A_{mn} \in M_{mn}(K)$なる行列$A_{mn}$は、列基本変形とある2つの行々を入れかえる操作を有限回くり返すと、${\mathrm{rank}}A_{mn} = r$として次の形に変形できる。この形をその行列$A_{mn}$の列標準形という。
\begin{align*}
\begin{pmatrix}
I_{r} & O \\
* & O \\
\end{pmatrix} = \begin{pmatrix}
1 & \cdots & 0 & 0 & \cdots & 0 \\
 \vdots & \ddots & \vdots & \vdots & \ddots & \vdots \\
0 & \cdots & 1 & 0 & \cdots & 0 \\
a_{r + 1,1}' & \cdots & a_{r + 1,r}' & 0 & \cdots & 0 \\
 \vdots & \ddots & \vdots & \vdots & \ddots & \vdots \\
a_{m1}' & \cdots & a_{mr}' & 0 & \cdots & 0 \\
\end{pmatrix}
\end{align*}\par
同様に、行基本変形とある2つの列々を入れかえる操作を有限回くり返すと、次の形に変形できる。この形をその行列$A_{mn}$の行標準形という。
\begin{align*}
\begin{pmatrix}
I_{r} & * \\
O & O \\
\end{pmatrix} = \begin{pmatrix}
1 & \cdots & 0 & a_{1,r + 1}' & \cdots & a_{1n}' \\
 \vdots & \ddots & \vdots & \vdots & \ddots & \vdots \\
0 & \cdots & 1 & a_{r,r + 1}' & \cdots & a_{rn}' \\
0 & \cdots & 0 & 0 & \cdots & 0 \\
 \vdots & \ddots & \vdots & \vdots & \ddots & \vdots \\
0 & \cdots & 0 & 0 & \cdots & 0 \\
\end{pmatrix}
\end{align*}
これらの操作は第$(i,j)$成分と第$(j,i)$成分とを逆にすればよいだけなので、行標準形のほうのみで議論することにしよう。後述する証明の議論から分かるように、この変形は次のalgorithmに従う。
\begin{enumerate}
\item
  $\forall k \in \varLambda_{\min\left\{ m,n \right\}}$に対し次式のように表されたとする。
\begin{align*}
A_{mn} \rightarrow \begin{pmatrix}
1 & \cdots & 0 & 0 & \cdots & 0 \\
 \vdots & \ddots & \vdots & \vdots & \ddots & \vdots \\
0 & \cdots & 1 & 0 & \cdots & 0 \\
0 & \cdots & 0 & a_{kk} & \cdots & a_{kn} \\
 \vdots & \ddots & \vdots & \vdots & \ddots & \vdots \\
0 & \cdots & 0 & a_{mk} & \cdots & a_{mn} \\
\end{pmatrix}
\end{align*}
\item
  行列$\begin{pmatrix}
  a_{kk} & \cdots & a_{kn} \\
   \vdots & \ddots & \vdots \\
  a_{mk} & \cdots & a_{mn} \\
  \end{pmatrix}$の各成分がすべて0であれば、9. へ進む。
\item
  そうでないならば、0でない成分$a_{i'j'}$が存在することになりある2つの行々を入れかえる操作とある2つの列々を入れかえる操作でその成分$a_{i'j'}$を第$(k,k)$成分とすることができる。
\item
  このときの第$(i,j)$成分を$a_{ij}'$とおき$\forall i \in \varLambda_{m} \setminus \left\{ k \right\} に対し第k$行の成分全体を$- \frac{a_{ij'}'}{a_{i'j'}}$倍しその各成分を第$i$行の対応する各成分に加える。
\item
  第$k$行の成分全体を$\frac{1}{a_{i'j'}}$倍する。
\item
  そうすると、$\forall i \in \varLambda_{m} \setminus \left\{ k \right\} に対し第(i,k)$成分は0と、第$(k,k)$成分は1となる。
\item
  $k = \min\left\{ m,n \right\}$が成り立つなら、9. へ進む。
\item
  そうでないならば、1. へ戻る。
\item
  このalgorithmは終了する。
\end{enumerate}
\end{thm}
\begin{proof}
$A_{mn} \in M_{mn}(K)$なる行列$A_{mn} = \left( a_{ij} \right)_{(i,j) \in \varLambda_{m} \times \varLambda_{n}}$を考える。その各成分が0であれば、この行列$A_{mn}$は零行列でありその行列$A_{mn}$の階数${\mathrm{rank}}A_{mn}$は0であるから、明らかである。\par
0でないその行列$A_{mn}$の成分$a_{i'j'}$が存在すれば、ある2つの行々を入れかえる操作とある2つの列々を入れかえる操作でその成分を第$(1,1)$成分とすることができる。
\begin{align*}
A_{mn} = \begin{pmatrix}
a_{11} & \cdots & a_{1j'} & \cdots & a_{1n} \\
 \vdots & \ddots & \vdots & \ddots & \vdots \\
a_{i'1} & \cdots & a_{i'j'} & \cdots & a_{i'n} \\
 \vdots & \ddots & \vdots & \ddots & \vdots \\
a_{m1} & \cdots & a_{mj'} & \cdots & a_{mn} \\
\end{pmatrix} &\rightarrow \begin{pmatrix}
a_{i'1} & \cdots & a_{i'j'} & \cdots & a_{i'n} \\
 \vdots & \ddots & \vdots & \ddots & \vdots \\
a_{11} & \cdots & a_{1j'} & \cdots & a_{1n} \\
 \vdots & \ddots & \vdots & \ddots & \vdots \\
a_{m1} & \cdots & a_{mj'} & \cdots & a_{mn} \\
\end{pmatrix}\\
&\rightarrow \begin{pmatrix}
a_{i'j'} & \cdots & a_{i'1} & \cdots & a_{i'n} \\
 \vdots & \ddots & \vdots & \ddots & \vdots \\
a_{1j'} & \cdots & a_{11} & \cdots & a_{1n} \\
 \vdots & \ddots & \vdots & \ddots & \vdots \\
a_{mj'} & \cdots & a_{m1} & \cdots & a_{mn} \\
\end{pmatrix}
\end{align*}
このときの第$(i,j)$成分を$a_{ij}'$とおき$\forall i \in \varLambda_{m} \setminus \left\{ 1 \right\}$に対し第1行の成分全体を$- \frac{a_{ij'}'}{a_{i'j'}}$倍しその各成分を第$i$行の対応する各成分に加えると、$\forall i \in \varLambda_{m} \setminus \left\{ 1 \right\}$に対し第$(i,1)$成分は0となる。
\begin{align*}
A_{mn} &\rightarrow \begin{pmatrix}
a_{i'j'} & \cdots & a_{i'1} & \cdots & a_{i'n} \\
 \vdots & \ddots & \vdots & \ddots & \vdots \\
a_{1j'} - \frac{a_{1j'}}{a_{i'j'}}a_{i'j'} & \cdots & a_{11} - \frac{a_{kj'}}{a_{i'j'}}a_{i'1} & \cdots & a_{1n} - \frac{a_{1j'}}{a_{i'j'}}a_{i'n} \\
 \vdots & \ddots & \vdots & \ddots & \vdots \\
a_{mj'} - \frac{a_{mj'}}{a_{i'j'}}a_{i'j'} & \cdots & a_{m1} - \frac{a_{mj'}}{a_{i'j'}}a_{i'1} & \cdots & a_{mn} - \frac{a_{mj'}}{a_{i'j'}}a_{i'n} \\
\end{pmatrix}\\
&\rightarrow \begin{pmatrix}
a_{i'j'} & \cdots & a_{i'1} & \cdots & a_{i'n} \\
 \vdots & \ddots & \vdots & \ddots & \vdots \\
0 & \cdots & a_{11} - \frac{a_{kj'}}{a_{i'j'}}a_{i'1} & \cdots & a_{1n} - \frac{a_{1j'}}{a_{i'j'}}a_{i'n} \\
 \vdots & \ddots & \vdots & \ddots & \vdots \\
0 & \cdots & a_{m1} - \frac{a_{mj'}}{a_{i'j'}}a_{i'1} & \cdots & a_{mn} - \frac{a_{mj'}}{a_{i'j'}}a_{i'n} \\
\end{pmatrix}
\end{align*}
第1行の成分全体を$\frac{1}{a_{i'j'}}$倍すると、次の行列が得られる。
\begin{align*}
A_{mn} \rightarrow \begin{pmatrix}
1 & \cdots & \frac{a_{i'1}}{a_{i'j'}} & \cdots & \frac{a_{i'n}}{a_{i'j'}} \\
 \vdots & \ddots & \vdots & \ddots & \vdots \\
0 & \cdots & a_{11} - \frac{a_{kj'}}{a_{i'j'}}a_{i'1} & \cdots & a_{1n} - \frac{a_{1j'}}{a_{i'j'}}a_{i'n} \\
 \vdots & \ddots & \vdots & \ddots & \vdots \\
0 & \cdots & a_{m1} - \frac{a_{mj'}}{a_{i'j'}}a_{i'1} & \cdots & a_{mn} - \frac{a_{mj'}}{a_{i'j'}}a_{i'n} \\
\end{pmatrix}
\end{align*}
$k \in \varLambda_{\min\left\{ m,n \right\}}$に対し次式のように表されたとする。このとき、行列$\begin{pmatrix}
a_{kk} & \cdots & a_{kn} \\
 \vdots & \ddots & \vdots \\
a_{mk} & \cdots & a_{mn} \\
\end{pmatrix}$の各成分がすべて0であるなら、その行列$A_{mn}$は$\begin{pmatrix}
I_{k} & * \\
O & O \\
\end{pmatrix}$と変形されており、その行列$\begin{pmatrix}
I_{k} & * \\
\end{pmatrix}$の行vectorsはその行列$\begin{pmatrix}
I_{k} & * \\
O & O \\
\end{pmatrix}$の行空間を張り線形独立でありその行空間の基底をなしそれらの行vectorsが$k$つあるから、その行列$\begin{pmatrix}
I_{k} & * \\
O & O \\
\end{pmatrix}$の階数${\mathrm{rank}} \begin{pmatrix}
I_{k} & * \\
O & O \\
\end{pmatrix}$は$k$に等しい。ここで、ある行列に行列の基本変形を行ったとしても、その行列の階数は一定であったので、その行列$A_{mn}$の階数${\mathrm{rank}}A_{mn}$は$k$に等しい。\par
逆に、0でないその行列$\begin{pmatrix}
a_{kk} & \cdots & a_{kn} \\
 \vdots & \ddots & \vdots \\
a_{mk} & \cdots & a_{mn} \\
\end{pmatrix}$の成分$a_{i'j'}$が存在すれば、ある2つの行々を入れかえる操作とある2つの列々を入れかえる操作でその成分を第$(k,k)$成分とすることができる。
\begin{align*}
A_{mn} &\rightarrow \begin{pmatrix}
1 & \cdots & 0 & a_{1k} & \cdots & a_{1j'} & \cdots & a_{1n} \\
 \vdots & \ddots & \vdots & \vdots & \ddots & \vdots & \ddots & \vdots \\
0 & \cdots & 1 & a_{k - 1,k} & \cdots & a_{k - 1,j'} & \cdots & a_{k - 1,n} \\
0 & \cdots & 0 & a_{kk} & \cdots & a_{kj'} & \cdots & a_{kn} \\
 \vdots & \ddots & \vdots & \vdots & \ddots & \vdots & \ddots & \vdots \\
0 & \cdots & 0 & a_{i'k} & \cdots & a_{i'j'} & \cdots & a_{i'n} \\
 \vdots & \ddots & \vdots & \vdots & \ddots & \vdots & \ddots & \vdots \\
0 & \cdots & 0 & a_{mk} & \cdots & a_{mj'} & \cdots & a_{mn} \\
\end{pmatrix}\\
&\rightarrow \begin{pmatrix}
1 & \cdots & 0 & a_{1k} & \cdots & a_{1j'} & \cdots & a_{1n} \\
 \vdots & \ddots & \vdots & \vdots & \ddots & \vdots & \ddots & \vdots \\
0 & \cdots & 1 & a_{k - 1,k} & \cdots & a_{k - 1,j'} & \cdots & a_{k - 1,n} \\
0 & \cdots & 0 & a_{i'k} & \cdots & a_{i'j'} & \cdots & a_{i'n} \\
 \vdots & \ddots & \vdots & \vdots & \ddots & \vdots & \ddots & \vdots \\
0 & \cdots & 0 & a_{kk} & \cdots & a_{kj'} & \cdots & a_{kn} \\
 \vdots & \ddots & \vdots & \vdots & \ddots & \vdots & \ddots & \vdots \\
0 & \cdots & 0 & a_{mk} & \cdots & a_{mj'} & \cdots & a_{mn} \\
\end{pmatrix}\\
&\rightarrow \begin{pmatrix}
1 & \cdots & 0 & a_{1j'} & \cdots & a_{1k} & \cdots & a_{1n} \\
 \vdots & \ddots & \vdots & \vdots & \ddots & \vdots & \ddots & \vdots \\
0 & \cdots & 1 & a_{k - 1,j'} & \cdots & a_{k - 1,k} & \cdots & a_{k - 1,n} \\
0 & \cdots & 0 & a_{i'j'} & \cdots & a_{i'k} & \cdots & a_{i'n} \\
 \vdots & \ddots & \vdots & \vdots & \ddots & \vdots & \ddots & \vdots \\
0 & \cdots & 0 & a_{kj'} & \cdots & a_{kk} & \cdots & a_{kn} \\
 \vdots & \ddots & \vdots & \vdots & \ddots & \vdots & \ddots & \vdots \\
0 & \cdots & 0 & a_{mj'} & \cdots & a_{mk} & \cdots & a_{mn} \\
\end{pmatrix}
\end{align*}
このときの第$(i,j)$成分を$a_{ij}'$とおき$\forall i \in \varLambda_{m} \setminus \left\{ k \right\}$に対し第$k$行の成分全体を$- \frac{a_{ij'}'}{a_{i'j'}}$倍しその各成分を第$i$行の対応する各成分に加えると、$\forall i \in \varLambda_{m} \setminus \left\{ k \right\}$に対し第$(i,k)$成分は0となる。
\begin{align*}
A_{mn} &\rightarrow \begin{pmatrix}
1 & \cdots & 0 & a_{1j'} & \cdots & a_{1k} & \cdots & a_{1n} \\
 \vdots & \ddots & \vdots & \vdots & \ddots & \vdots & \ddots & \vdots \\
0 & \cdots & 1 & a_{k - 1,j'} & \cdots & a_{k - 1,k} & \cdots & a_{k - 1,n} \\
0 & \cdots & 0 & a_{i'j'} & \cdots & a_{i'k} & \cdots & a_{i'n} \\
 \vdots & \ddots & \vdots & \vdots & \ddots & \vdots & \ddots & \vdots \\
0 & \cdots & 0 & a_{kj'} - \frac{a_{kj'}}{a_{i'j'}}a_{i'j'} & \cdots & a_{kk} - \frac{a_{kj'}}{a_{i'j'}}a_{i'k} & \cdots & a_{kn} - \frac{a_{kj'}}{a_{i'j'}}a_{i'n} \\
 \vdots & \ddots & \vdots & \vdots & \ddots & \vdots & \ddots & \vdots \\
0 & \cdots & 0 & a_{mj'} - \frac{a_{mj'}}{a_{i'j'}}a_{i'j'} & \cdots & a_{mk} - \frac{a_{mj'}}{a_{i'j'}}a_{i'k} & \cdots & a_{mn} - \frac{a_{mj'}}{a_{i'j'}}a_{i'n} \\
\end{pmatrix}\\
&\rightarrow \begin{pmatrix}
1 & \cdots & 0 & a_{1j'} & \cdots & a_{1k} & \cdots & a_{1n} \\
 \vdots & \ddots & \vdots & \vdots & \ddots & \vdots & \ddots & \vdots \\
0 & \cdots & 1 & a_{k - 1,j'} & \cdots & a_{k - 1,k} & \cdots & a_{k - 1,n} \\
0 & \cdots & 0 & a_{i'j'} & \cdots & a_{i'k} & \cdots & a_{i'n} \\
 \vdots & \ddots & \vdots & \vdots & \ddots & \vdots & \ddots & \vdots \\
0 & \cdots & 0 & 0 & \cdots & a_{kk} - \frac{a_{kj'}}{a_{i'j'}}a_{i'k} & \cdots & a_{kn} - \frac{a_{kj'}}{a_{i'j'}}a_{i'n} \\
 \vdots & \ddots & \vdots & \vdots & \ddots & \vdots & \ddots & \vdots \\
0 & \cdots & 0 & 0 & \cdots & a_{mk} - \frac{a_{mj'}}{a_{i'j'}}a_{i'k} & \cdots & a_{mn} - \frac{a_{mj'}}{a_{i'j'}}a_{i'n} \\
\end{pmatrix}
\end{align*}
第$k$行の成分全体を$\frac{1}{a_{i'j'}}$倍すると、次の行列が得られる。
\begin{align*}
A_{mn} \rightarrow \begin{pmatrix}
1 & \cdots & 0 & a_{1j'} & \cdots & a_{1k} & \cdots & a_{1n} \\
 \vdots & \ddots & \vdots & \vdots & \ddots & \vdots & \ddots & \vdots \\
0 & \cdots & 1 & a_{k - 1,j'} & \cdots & a_{k - 1,k} & \cdots & a_{k - 1,n} \\
0 & \cdots & 0 & a_{i'j'} & \cdots & \frac{a_{i'k}}{a_{i'j'}} & \cdots & \frac{a_{i'n}}{a_{i'j'}} \\
 \vdots & \ddots & \vdots & \vdots & \ddots & \vdots & \ddots & \vdots \\
0 & \cdots & 0 & 0 & \cdots & a_{kk} - \frac{a_{kj'}}{a_{i'j'}}a_{i'k} & \cdots & a_{kn} - \frac{a_{kj'}}{a_{i'j'}}a_{i'n} \\
 \vdots & \ddots & \vdots & \vdots & \ddots & \vdots & \ddots & \vdots \\
0 & \cdots & 0 & 0 & \cdots & a_{mk} - \frac{a_{mj'}}{a_{i'j'}}a_{i'k} & \cdots & a_{mn} - \frac{a_{mj'}}{a_{i'j'}}a_{i'n} \\
\end{pmatrix}
\end{align*}\par
以上より数学的帰納法によって示すべきことは示された。\par
同様にして、列基本変形とある2つの行々を入れかえる操作を有限回くり返すと、列標準形に変形できる。
\end{proof}
\begin{thm}\label{2.1.7.6}
体$K$上で$\forall A_{nn} \in M_{nn}(K)$に対し、その行列$A_{nn}$が正則行列であるならそのときに限り、その標準形が単位行列である。
\end{thm}
\begin{proof}
体$K$上で$\forall A_{nn} \in M_{nn}(K)$に対し、その行列$A_{nn}$が正則行列であるなら、定理\ref{2.1.4.14}、定理\ref{2.1.7.4}よりその標準形$I$が与えられたとき、その標準形$I$は単位行列そのものである。\par
逆に、その標準形が単位行列であるなら、$\exists P_{nn},Q_{nn} \in {\mathrm{GL}}_{n}(K)$に対し、$P_{nn}A_{nn}Q_{nn} = I_{n}$が成り立つので、次のようになる。
\begin{align*}
A_{nn}Q_{nn}P_{nn} &= P_{nn}^{- 1}P_{nn}A_{nn}Q_{nn}P_{nn}\\
&= P_{nn}^{- 1}I_{n}P_{nn}\\
&= P_{nn}^{- 1}P_{nn} = I_{n}\\
Q_{nn}P_{nn}A_{nn} &= Q_{nn}P_{nn}A_{nn}Q_{nn}Q_{nn}^{- 1}\\
&= Q_{nn}I_{n}Q_{nn}^{- 1}\\
&= Q_{nn}Q_{nn}^{- 1} = I_{n}
\end{align*}
以上より、$A_{nn}^{- 1} = Q_{nn}P_{nn}$が成り立つので、その行列$A_{nn}$は正則行列である。
\end{proof}
\begin{thm}\label{2.1.7.7}
体$K$上で$\forall A_{nn} \in {\mathrm{GL}}_{n}(K)$に対し、その行列$A_{nn}$は基本行列の積で表されることができる。
\end{thm}
\begin{proof} 体$K$上で$\forall A_{nn} \in {\mathrm{GL}}_{n}(K)$に対し、定理\ref{2.1.7.6}より$\exists P_{nn},Q_{nn} \in {\mathrm{GL}}_{n}(K)$に対し、$P_{nn}A_{nn}Q_{nn} = I_{n}$が成り立つので、次のようになる。
\begin{align*}
A_{nn}Q_{nn}P_{nn} &= P_{nn}^{- 1}P_{nn}A_{nn}Q_{nn}P_{nn}\\
&= P_{nn}^{- 1}I_{n}P_{nn}\\
&= P_{nn}^{- 1}P_{nn} = I_{n}\\
Q_{nn}P_{nn}A_{nn} &= Q_{nn}P_{nn}A_{nn}Q_{nn}Q_{nn}^{- 1}\\
&= Q_{nn}I_{n}Q_{nn}^{- 1}\\
&= Q_{nn}Q_{nn}^{- 1} = I_{n}
\end{align*}
以上より、$A_{nn}^{- 1} = Q_{nn}P_{nn}$が成り立つので、$A_{nn} = P_{nn}^{- 1}Q_{nn}^{- 1}$が成り立つ。これにより、その行列$A_{nn}$は基本行列の積で表されることができる。
\end{proof}
\begin{thm}\label{2.1.7.8}
体$K$上で$\forall A_{nn} \in {\mathrm{GL}}_{n}(K)$に対し、その行列$A_{nn}$は列変形あるいは行変形だけで単位行列に変形されることができる。
\end{thm}
\begin{proof} 体$K$上で$\forall A_{nn} \in {\mathrm{GL}}_{n}(K)$に対し、定理\ref{2.1.7.6}より$\exists P_{nn},Q_{nn} \in {\mathrm{GL}}_{n}(K)$に対し、$P_{nn}A_{nn}Q_{nn} = I_{n}$が成り立つので、次のようになる。
\begin{align*}
A_{nn}Q_{nn}P_{nn} &= P_{nn}^{- 1}P_{nn}A_{nn}Q_{nn}P_{nn}\\
&= P_{nn}^{- 1}I_{n}P_{nn}\\
&= P_{nn}^{- 1}P_{nn} = I_{n}\\
Q_{nn}P_{nn}A_{nn} &= Q_{nn}P_{nn}A_{nn}Q_{nn}Q_{nn}^{- 1}\\
&= Q_{nn}I_{n}Q_{nn}^{- 1}\\
&= Q_{nn}Q_{nn}^{- 1} = I_{n}
\end{align*}
よって、その行列$A_{nn}$は列変形あるいは行変形だけで単位行列に変形されることができる。
\end{proof}
\begin{thm}\label{2.1.7.9}
体$K$上で$\forall A_{mn},B_{mn} \in M_{mn}(K)$に対し、その行列$A_{nn}$が行列の変形でその行列$B_{nn}$に変形されることができるならそのときに限り、これらの行列たち$A_{mn}$、$B_{mn}$は対等である。
\end{thm}
\begin{proof}
体$K$上で$\forall A_{mn},B_{mn} \in M_{mn}(K)$に対し、その行列$A_{mn}$が行列の変形でその行列$B_{mn}$に変形されることができるなら、$\exists P_{mm},Q_{nn} \in {\mathrm{GL}}_{n}(K)$に対し、$A_{mn} = P_{mm}B_{mn}Q_{nn}$が成り立つので、$P_{mm}^{- 1}A_{mn} = B_{mn}Q_{nn}$が得られる。したがって、これらの行列たち$A_{mn}$、$B_{mn}$は対等である。\par
逆に、これが成り立つなら、$\exists P_{mm},Q_{nn} \in {\mathrm{GL}}_{n}(K)$に対し、$P_{mm}A_{mn} = B_{mn}Q_{nn}$が成り立つので、$A_{mn} = P_{mm}^{- 1}B_{mn}Q_{nn}$が得られる。定理\ref{2.1.7.7}より、これらの行列たち$P_{mm}$、$Q_{nn}$は基本行列の積で表されることができることにより、その行列$A_{nn}$が行列の変形でその行列$B_{nn}$に変形されることができる。
\end{proof}
\begin{thebibliography}{50}
  \bibitem{1}
    松坂和夫, 線型代数入門, 岩波書店, 1980. 新装版第2刷 p113-p118,196-215 ISBN978-4-00-029872-8
  \bibitem{2}
    対馬龍司, 線形代数学講義, 共立出版, 2007. 改訂版8刷 p104-113 ISBN978-4-320-11097-7
\end{thebibliography}
\end{document}
