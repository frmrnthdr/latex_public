\documentclass[dvipdfmx]{jsarticle}
\setcounter{section}{1}
\setcounter{subsection}{3}
\usepackage{xr}
\externaldocument{2.1.1}
\externaldocument{2.1.2}
\usepackage{amsmath,amsfonts,amssymb,array,comment,mathtools,url,docmute}
\usepackage{longtable,booktabs,dcolumn,tabularx,mathtools,multirow,colortbl,xcolor}
\usepackage[dvipdfmx]{graphics}
\usepackage{bmpsize}
\usepackage{amsthm}
\usepackage{enumitem}
\setlistdepth{20}
\renewlist{itemize}{itemize}{20}
\setlist[itemize]{label=•}
\renewlist{enumerate}{enumerate}{20}
\setlist[enumerate]{label=\arabic*.}
\setcounter{MaxMatrixCols}{20}
\setcounter{tocdepth}{3}
\newcommand{\rotin}{\text{\rotatebox[origin=c]{90}{$\in $}}}
\newcommand{\amap}[6]{\text{\raisebox{-0.7cm}{\begin{tikzpicture} 
  \node (a) at (0, 1) {$\textstyle{#2}$};
  \node (b) at (#6, 1) {$\textstyle{#3}$};
  \node (c) at (0, 0) {$\textstyle{#4}$};
  \node (d) at (#6, 0) {$\textstyle{#5}$};
  \node (x) at (0, 0.5) {$\rotin $};
  \node (x) at (#6, 0.5) {$\rotin $};
  \draw[->] (a) to node[xshift=0pt, yshift=7pt] {$\textstyle{\scriptstyle{#1}}$} (b);
  \draw[|->] (c) to node[xshift=0pt, yshift=7pt] {$\textstyle{\scriptstyle{#1}}$} (d);
\end{tikzpicture}}}}
\newcommand{\twomaps}[9]{\text{\raisebox{-0.7cm}{\begin{tikzpicture} 
  \node (a) at (0, 1) {$\textstyle{#3}$};
  \node (b) at (#9, 1) {$\textstyle{#4}$};
  \node (c) at (#9+#9, 1) {$\textstyle{#5}$};
  \node (d) at (0, 0) {$\textstyle{#6}$};
  \node (e) at (#9, 0) {$\textstyle{#7}$};
  \node (f) at (#9+#9, 0) {$\textstyle{#8}$};
  \node (x) at (0, 0.5) {$\rotin $};
  \node (x) at (#9, 0.5) {$\rotin $};
  \node (x) at (#9+#9, 0.5) {$\rotin $};
  \draw[->] (a) to node[xshift=0pt, yshift=7pt] {$\textstyle{\scriptstyle{#1}}$} (b);
  \draw[|->] (d) to node[xshift=0pt, yshift=7pt] {$\textstyle{\scriptstyle{#2}}$} (e);
  \draw[->] (b) to node[xshift=0pt, yshift=7pt] {$\textstyle{\scriptstyle{#1}}$} (c);
  \draw[|->] (e) to node[xshift=0pt, yshift=7pt] {$\textstyle{\scriptstyle{#2}}$} (f);
\end{tikzpicture}}}}
\renewcommand{\thesection}{第\arabic{section}部}
\renewcommand{\thesubsection}{\arabic{section}.\arabic{subsection}}
\renewcommand{\thesubsubsection}{\arabic{section}.\arabic{subsection}.\arabic{subsubsection}}
\everymath{\displaystyle}
\allowdisplaybreaks[4]
\usepackage{vtable}
\theoremstyle{definition}
\newtheorem{thm}{定理}[subsection]
\newtheorem*{thm*}{定理}
\newtheorem{dfn}{定義}[subsection]
\newtheorem*{dfn*}{定義}
\newtheorem{axs}[dfn]{公理}
\newtheorem*{axs*}{公理}
\renewcommand{\headfont}{\bfseries}
\makeatletter
  \renewcommand{\section}{%
    \@startsection{section}{1}{\z@}%
    {\Cvs}{\Cvs}%
    {\normalfont\huge\headfont\raggedright}}
\makeatother
\makeatletter
  \renewcommand{\subsection}{%
    \@startsection{subsection}{2}{\z@}%
    {0.5\Cvs}{0.5\Cvs}%
    {\normalfont\LARGE\headfont\raggedright}}
\makeatother
\makeatletter
  \renewcommand{\subsubsection}{%
    \@startsection{subsubsection}{3}{\z@}%
    {0.4\Cvs}{0.4\Cvs}%
    {\normalfont\Large\headfont\raggedright}}
\makeatother
\makeatletter
\renewenvironment{proof}[1][\proofname]{\par
  \pushQED{\qed}%
  \normalfont \topsep6\p@\@plus6\p@\relax
  \trivlist
  \item\relax
  {
  #1\@addpunct{.}}\hspace\labelsep\ignorespaces
}{%
  \popQED\endtrivlist\@endpefalse
}
\makeatother
\renewcommand{\proofname}{\textbf{証明}}
\usepackage{tikz,graphics}
\usepackage[dvipdfmx]{hyperref}
\usepackage{pxjahyper}
\hypersetup{
 setpagesize=false,
 bookmarks=true,
 bookmarksdepth=tocdepth,
 bookmarksnumbered=true,
 colorlinks=false,
 pdftitle={},
 pdfsubject={},
 pdfauthor={},
 pdfkeywords={}}
\begin{document}
%\hypertarget{n-vector}{%
\subsection{$n$-vector}%\label{n-vector}}
%\hypertarget{n-vector-1}{%
\subsubsection{$n$-vector}%\label{n-vector-1}}
\begin{dfn}
体$K$の$n$つの積$K^{n}$の元、即ち、体$K$の元$a_{i}$の順序付けられた組$\left( a_{i} \right)_{i \in \varLambda_{n}} = \begin{pmatrix}
a_{1} \\
a_{2} \\
 \vdots \\
a_{n} \\
\end{pmatrix}$を考え集合$K^{n}$について、$\forall k,l \in K\forall\left( a_{i} \right)_{i \in \varLambda_{n}},\left( b_{i} \right)_{i \in \varLambda_{n}} \in K^{n}$に対し、$k\left( a_{i} \right)_{i \in \varLambda_{n}} + l\left( b_{i} \right)_{i \in \varLambda_{n}} = \left( ka_{i} + lb_{i} \right)_{i \in \varLambda_{n}}$が成り立つと定義する。
\end{dfn}
\begin{thm}\label{2.1.4.1}
集合$K^{n}$はvector空間である。
\end{thm}
\begin{dfn}
体$K$上のvector空間$K^{n}$の元$\left( a_{i} \right)_{i \in \varLambda_{n}}$はその体$K$上の$n$-vectorといい、$a_{i}$をこのvectorの第$i$成分、第$i$座標という。
\end{dfn}
\begin{proof}
体$K$の$n$つの積$K^{n}$の元、即ち体$K$の元$a_{i}$の順序付けられた組$\left( a_{i} \right)_{i \in \varLambda_{n}}$が与えられたとき、集合$K^{n}$について定義より$\forall\left( a_{i} \right)_{i \in \varLambda_{n}},\left( b_{i} \right)_{i \in \varLambda_{n}} \in K^{n}$に対し、$\left( a_{i} \right)_{i \in \varLambda_{n}} + \left( b_{i} \right)_{i \in \varLambda_{n}} = \left( a_{i} + b_{i} \right)_{i \in \varLambda_{n}}$が成り立つ。このとき、体$K$が加法について可換群$(K, + )$をなし、$a_{i} + b_{i}$のみ着眼して考えると、各成分も加法について可換群$(K, + )$をなしているので、$K^{n}$は加法について可換群$\left( K^{n}, + \right)$をなす。\par
また、定義より$\forall k \in K\forall\left( a_{i} \right)_{i \in \varLambda_{n}} \in K^{n}$に対し、$k\left( a_{i} \right)_{i \in \varLambda_{n}} = \left( {ka}_{i} \right)_{i \in \varLambda_{n}}$が成り立ち${ka}_{i} \in K$が成り立つので、$\left( {ka}_{i} \right)_{i \in \varLambda_{n}} \in K^{n}$が成り立ち写像$\mu_{2}:K \times K^{n} \rightarrow K^{n};\left( k,\left( a_{i} \right)_{i \in \varLambda_{n}} \right) \mapsto \left( {ka}_{i} \right)_{i \in \varLambda_{n}}$が定義される。\par
さらに、$\forall k \in K\forall\left( a_{i} \right)_{i \in \varLambda_{n}},\left( b_{i} \right)_{i \in \varLambda_{n}} \in K^{n}$に対し、次のようになる。
\begin{align*}
k\left( \left( a_{i} \right)_{i \in \varLambda_{n}} + \left( b_{i} \right)_{i \in \varLambda_{n}} \right) &= k\left( a_{i} + b_{i} \right)_{i \in \varLambda_{n}}\\
&= \left( k\left( a_{i} + b_{i} \right) \right)_{i \in \varLambda_{n}}\\
&= \left( ka_{i} + kb_{i} \right)_{i \in \varLambda_{n}}\\
&= \left( ka_{i} \right)_{i \in \varLambda_{n}}\mathbf{+}\left( kb_{i} \right)_{i \in \varLambda_{n}}\\
&= k\left( a_{i} \right)_{i \in \varLambda_{n}}\mathbf{+}k\left( b_{i} \right)_{i \in \varLambda_{n}}
\end{align*}
$\forall k,l \in K\forall\left( a_{i} \right)_{i \in \varLambda_{n}} \in K^{n}$に対し、次のようになる。
\begin{align*}
(k + l)\left( a_{i} \right)_{i \in \varLambda_{n}} &= \left( (k + l)a_{i} \right)_{i \in \varLambda_{n}}\\
&= \left( ka_{i} + la_{i} \right)_{i \in \varLambda_{n}}\\
&= \left( ka_{i} \right)_{i \in \varLambda_{n}}\mathbf{+}\left( la_{i} \right)_{i \in \varLambda_{n}}\\
&= k\left( a_{i} \right)_{i \in \varLambda_{n}}\mathbf{+}l\left( a_{i} \right)_{i \in \varLambda_{n}}
\end{align*}
$\forall k,l \in K\forall\left( a_{i} \right)_{i \in \varLambda_{n}} \in K^{n}$に対し、次のようになる。
\begin{align*}
(kl)\left( a_{i} \right)_{i \in \varLambda_{n}} &= \left( (kl)a_{i} \right)_{i \in \varLambda_{n}}\\
&= \left( k\left( la_{i} \right) \right)_{i \in \varLambda_{n}}\\
&= {k\left( la_{i} \right)}_{i \in \varLambda_{n}} = k\left( l\left( a_{i} \right)_{i \in \varLambda_{n}} \right)
\end{align*}
$\exists 1 \in K\forall\left( a_{i} \right)_{i \in \varLambda_{n}} \in K^{n}$に対し、次のようになる。
\begin{align*}
1\left( a_{i} \right)_{i \in \varLambda_{n}} = \left( 1a_{i} \right)_{i \in \varLambda_{n}} = \left( a_{i} \right)_{i \in \varLambda_{n}}
\end{align*}\par
以上より、その集合$K^{n}$はvector空間の定義を満たしているので、vector空間である。
\end{proof}
\begin{thm}\label{2.1.4.2}
$j \in \varLambda_{n}$なる添数$j$を用いて$\left( a_{ij} \right)_{i \in \varLambda_{m}} \in K^{m}$なるvectors$\left( a_{ij} \right)_{i \in \varLambda_{m}}$が与えられたとする。$m < n$が成り立つなら、族$\left\{ \left( a_{ij} \right)_{i \in \varLambda_{m}}\right\}_{j \in \varLambda_{n} } $は体$K$上のvector空間$K^{m}$で線形従属である。
\end{thm}
\begin{proof}
$j \in \varLambda_{n}$なる添数$j$を用いて$\left( a_{ij} \right)_{i \in \varLambda_{m}} \in K^{m}$なるvectors$\left( a_{ij} \right)_{i \in \varLambda_{m}}$が与えられ、$m < n$が成り立つとする。このとき、$\sum_{j \in \varLambda_{n}} {c_{j}\left( a_{ij} \right)_{i \in \varLambda_{m}}} = (0)_{i \in \varLambda_{m}}$の成分を比較すれば、$\forall i \in \varLambda_{m}$に対し、$\sum_{j \in \varLambda_{n}} {c_{j}a_{ij}} = 0$が成り立つことになる。定理\ref{2.1.1.17}より$\exists j \in \varLambda_{n}$に対し、$c_{j} \neq 0$が成り立つので、示すべきことは示された。
\end{proof}
%\hypertarget{ux6a19ux6e96ux76f4ux4ea4ux57faux5e95}{%
\subsubsection{標準直交基底}%\label{ux6a19ux6e96ux76f4ux4ea4ux57faux5e95}}
\begin{dfn}
次式で定義される写像$\delta$の組$(i,j)$による像$\delta(i,j)$をKroneckerのdeltaといい$\delta_{ij}$と書く。
\begin{align*}
\delta:\varLambda_{m} \times \varLambda_{n} \rightarrow R;(i,j) \mapsto \left\{ \begin{matrix}
1 & \mathrm{if} & i = j \\
0 & \mathrm{if} & i \neq j \\
\end{matrix} \right.\ 
\end{align*}
\end{dfn}
\begin{thm}\label{2.1.4.3}
体$K$上のvector空間$K^{n}$において、vectors$\mathbf{e}_{i}$を$\mathbf{e}_{i} = \left( \delta_{ij} \right)_{j \in \varLambda_{n}}$のようにおく。このとき、次のことが成り立つ。
\begin{itemize}
\item
  $\forall\mathbf{v} = \left( a_{i} \right)_{i \in \varLambda_{n}} \in K^{n}$に対し、$\mathbf{v} = \sum_{i \in \varLambda_{n}} {a_{i}\mathbf{e}_{i}}$が成り立つ。
\item
  組$\left\langle \mathbf{e}_{i} \right\rangle_{i \in \varLambda_{n}}$はそのvector空間$K^{n}$の基底である。
\item
  $\dim K^{n} = n$が成り立つ。
\end{itemize}
\end{thm}
\begin{dfn}
体$K$上のvector空間$K^{n}$のvector$\mathbf{v}$が$\sum_{i \in \varLambda_{n}} {a_{i}\mathbf{e}_{i}}$と書かれることができるとき、そのvector$\mathbf{v}$をそのvector$\left( a_{i} \right)_{i \in \varLambda_{n}}$に書きかえることをそのvector$\mathbf{v}$をそのvector$\left( a_{i} \right)_{i \in \varLambda_{n}}$と成分表示するといい、$\forall i \in \varLambda_{n}$に対し、それらのvectors$\mathbf{e}_{i}$からなる基底$\left\langle \mathbf{e}_{i} \right\rangle_{i \in \varLambda_{n}}$を標準基底、標準直交基底などという。基底を言及せずに単に座標といわれたら、多くの場合、その基底は標準直交基底を指す。以下ここでは、標準直交基底を$\varepsilon$とおくことにする。
\end{dfn}
\begin{proof}
体$K$上のvector空間$K^{n}$において、$\forall i \in \varLambda_{n}$に対しvectors$\mathbf{e}_{i}$を$\mathbf{e}_{i} = \left( \delta_{ij} \right)_{j \in \varLambda_{n}}$のようにおく。このとき、$\forall\mathbf{v} = \left( a_{i} \right)_{i \in \varLambda_{n}} \in K^{n}$に対し、次のようになる。
\begin{align*}
\mathbf{v} &= \left( a_{i} \right)_{i \in \varLambda_{n}}\\
&= \left( \sum_{i \in \varLambda_{n} \setminus \left\{ j \right\}} {a_{i}0} + a_{j}1 \right)_{j \in \varLambda_{n}}\\
&= \left( \sum_{i \in \varLambda_{n} \setminus \left\{ j \right\}} {a_{i}\delta_{ij}} + a_{j}\delta_{jj} \right)_{j \in \varLambda_{n}}\\
&= \left( \sum_{i \in \varLambda_{n}} {a_{i}\delta_{ij}} \right)_{j \in \varLambda_{n}}\\
&= \sum_{i \in \varLambda_{n}} \left( a_{i}\delta_{ij} \right)_{j \in \varLambda_{n}}\\
&= \sum_{i \in \varLambda_{n}} {a_{i}\left( \delta_{ij} \right)_{j \in \varLambda_{n}}}\\
&= \sum_{i \in \varLambda_{n}} {a_{i}\mathbf{e}_{i}}
\end{align*}\par
次に、式$\sum_{i \in \varLambda_{n}} {c_{i}\mathbf{e}_{i}} = \mathbf{0}$について考えよう。このとき、上と同様にして次のようになるので、
\begin{align*}
(0)_{j \in \varLambda_{n}} = \mathbf{0} = \sum_{i \in \varLambda_{n}} {c_{i}\mathbf{e}_{i}} = \left( c_{j} \right)_{j \in \varLambda_{n}}
\end{align*}
$\forall i \in \varLambda_{n}$に対し、$c_{i} = 0$が成り立つ。また、上記より$\forall\mathbf{v} = \left( a_{i} \right)_{i \in \varLambda_{n}} \in K^{n}$に対し、$\mathbf{v} = \sum_{i \in \varLambda_{n}} {a_{i}\mathbf{e}_{i}}$が成り立つのであったので、$\forall\mathbf{v} \in K^{n}$に対しvector$\mathbf{v}$はその族$\left\{ \mathbf{e}_i \right\}_{i \in \varLambda_{n} } $の線形結合である。以上より、$\forall i \in \varLambda_{n}$に対しそれらのvectors$\mathbf{e}_{i}$は線形独立で、$\forall\left( a_{i} \right)_{i \in \varLambda_{n}} \in K^{n}$に対し、その族$\left\{ \mathbf{e}_i \right\}_{i \in \varLambda_{n} } $の線形結合で表されそのvector空間$K^{n}$は、$\forall i \in \varLambda_{n}$に対し、それらのvectors$\mathbf{e}_{i}$によって生成されるので、vectors$\mathbf{e}_{i}$は体$K$上のvector空間$K^{n}$の基底である。\par
このとき、$\forall i \in \varLambda_{n}$に対しそれらのvectors$\mathbf{e}_{j}$は全て$n$つあるので、次元の定義より$\dim K^{n} = n$が成り立つ。
\end{proof}
\begin{dfn}
集合$\mathbb{R}^{n}$の元を点といい$\forall\mathbf{r} \in \mathbb{R}^{n}$なる点$\mathbf{r}$が$\forall i \in \varLambda_{n}$に対しそれらのvectors$\mathbf{e}_{i}$を用いて$\sum_{i \in \varLambda_{n}} {x_{i}\mathbf{e}_{i}}$と書かれることができるとき、標準直交基底$\left\langle \mathbf{e}_{i} \right\rangle_{i \in \varLambda_{n}}$が与えられたときのその集合$\mathbb{R}^{n}$を$n$次元直交座標系、または、単に直交座標系などという。$\forall\mathbf{r} \in \mathbb{R}^{n}$に対し、その点$\mathbf{r}$のその基底が標準直交基底である座標$\left( x_{i} \right)_{i \in \varLambda_{n}}$をその点$\mathbf{r}$の$n$次元直交座標、または単に、直交座標などという。特に、2次元直交座標系なら$x_{1}$、$x_{2}$をそれぞれ$x$、$y$などと表しこの座標系を$xy$平面ともいい、3次元直交座標系なら$x_{1}$、$x_{2}$、$x_{3}$をそれぞれ$x$、$y$、$z$などと表しこの座標系を$xyz$空間ともいい、任意の実数$x_{j}$を用いて、次式で表される集合$P$を$j = 1$なら$yz$平面、$j = 2$なら$zx$平面、$j = 3$なら$xy$平面という。
\begin{align*}
P = \left\{ \left( x_{i} \right)_{i \in \varLambda_{3}} \in \mathbb{R}^{3} \middle| \left( x_{i} \right)_{i \in \varLambda_{3}} = \left( \left\{ \begin{matrix}
0 & \mathrm{if} & i = j \\
x_{i} & \mathrm{if} & i \neq j \\
\end{matrix} \right.\  \right)_{i \in \varLambda_{3}} \right\}
\end{align*}
\end{dfn}
%\hypertarget{ux5f62ux5f0fux7684ux306aux5185ux7a4d}{%
\subsubsection{形式的な内積}%\label{ux5f62ux5f0fux7684ux306aux5185ux7a4d}}
\begin{dfn}
体$K$上で$\forall\mathbf{v},\mathbf{w} \in K^{n}$に対し、$\mathbf{v} = \left( a_{i} \right)_{i \in \varLambda_{n}}$、$\mathbf{w} = \left( b_{i} \right)_{i \in \varLambda_{n}}$として次式のような写像$\left( \bullet \middle| \bullet \right)$を考えよう。
\begin{align*}
\left( \bullet \middle| \bullet \right):K^{n} \times K^{n} \rightarrow K;\left( \mathbf{v},\mathbf{w} \right) \mapsto \sum_{i \in \varLambda_{n}} {a_{i}b_{i}}
\end{align*}
この写像$\left( \bullet \middle| \bullet \right)$を形式的な内積、または単に、内積、$\sum_{i \in \varLambda_{n}} {a_{i}b_{i}}$を2つのvectors$\mathbf{v}$と$\mathbf{w}$との形式的な内積、または単に、内積といい、$\left( \mathbf{v} \middle| \mathbf{w} \right)$、$\mathbf{v} \cdot \mathbf{w}$、$\left( \mathbf{v},\mathbf{w} \right)$などとかく。
\end{dfn}
\begin{thm}\label{2.1.4.4}
形式的な内積について、体$K$上で次のことが成り立つ。
\begin{itemize}
\item
  $\forall\mathbf{v},\mathbf{w} \in K^{n}$に対し、$\left( \mathbf{v} \middle| \mathbf{w} \right) = \left( \mathbf{w} \middle| \mathbf{v} \right)$が成り立つ。
\item
  $\forall k,l \in K\forall\mathbf{u,v,w} \in K^{n}$に対し、$\left( k\mathbf{u} + l\mathbf{v} \middle| \mathbf{w} \right) = k\left( \mathbf{u} \middle| \mathbf{w} \right) + l\left( \mathbf{v} \middle| \mathbf{w} \right)$が成り立つ。
\end{itemize}
\end{thm}
\begin{proof}
体$K$上で$\forall\mathbf{v},\mathbf{w} \in K^{n}$に対し、$\mathbf{v} = \left( a_{i} \right)_{i \in \varLambda_{n}}$、$\mathbf{w} = \left( b_{i} \right)_{i \in \varLambda_{n}}$とすれば、次のようになる。
\begin{align*}
\left( \mathbf{v} \middle| \mathbf{w} \right) &= \sum_{i \in \varLambda_{n}} {a_{i}b_{i}}\\
&= \sum_{i \in \varLambda_{n}} {b_{i}a_{i}}\\
&= \left( \mathbf{w} \middle| \mathbf{v} \right)
\end{align*}
また、$\forall k,l \in K\forall\mathbf{u,v,w} \in K^{n}$に対し、$\mathbf{u} = \left( a_{i} \right)_{i \in \varLambda_{n}}$、$\mathbf{v} = \left( b_{i} \right)_{i \in \varLambda_{n}}$、$\mathbf{w} = \left( c_{i} \right)_{i \in \varLambda_{n}}$とすれば、次のようになる。
\begin{align*}
\left( k\mathbf{u} + l\mathbf{v} \middle| \mathbf{w} \right) &= \left( \left( ka_{i} + lb_{i} \right)_{i \in \varLambda_{n}} \middle| \mathbf{w} \right)\\
&= \sum_{i \in \varLambda_{n}} {\left( ka_{i} + lb_{i} \right)c_{i}}\\
&= \sum_{i \in \varLambda_{n}} \left( ka_{i}c_{i} + lb_{i}c_{i} \right)\\
&= k\sum_{i \in \varLambda_{n}} {a_{i}c_{i}} + l\sum_{i \in \varLambda_{n}} {b_{i}c_{i}}\\
&= k\left( \mathbf{u} \middle| \mathbf{w} \right) + l\left( \mathbf{v} \middle| \mathbf{w} \right)
\end{align*}
\end{proof}
\begin{dfn}
体$K$上で$\forall A_{mn} \in M_{mn}(K)\forall\mathbf{v} \in K^{n}$に対し、$A_{mn} = \left(^{t}\mathbf{a}_{i} \right)_{i \in \varLambda_{m}}$、$\mathbf{v} = \left( x_{i} \right)_{i \in \varLambda_{n}}$として、次式のように積$A_{mn}\mathbf{v}$を定義する。
\begin{align*}
M_{mn}(K) \times K^{n} \rightarrow K^{m};\left( A_{mn},\mathbf{v} \right) \mapsto A_{mn}\mathbf{v} = \left( \left( \mathbf{a}_{i} \middle| \mathbf{v} \right) \right)_{i \in \varLambda_{m}}
\end{align*}
\end{dfn}\par
即ち、その行列$A_{mn}$が次のようにおかれれば、
\begin{align*}
A_{mn} = \begin{pmatrix}
^{t}\mathbf{a}_{1} \\
^{t}\mathbf{a}_{2} \\
 \vdots \\
{^{t}\mathbf{a}}_{m} \\
\end{pmatrix}
\end{align*}
次式が成り立つことを意味する。
\begin{align*}
\begin{pmatrix}
^{t}\mathbf{a}_{1} \\
^{t}\mathbf{a}_{2} \\
 \vdots \\
{^{t}\mathbf{a}}_{m} \\
\end{pmatrix}\mathbf{v} = \begin{pmatrix}
\left( \mathbf{a}_{1} \middle| \mathbf{v} \right) \\
\left( \mathbf{a}_{2} \middle| \mathbf{v} \right) \\
 \vdots \\
\left( \mathbf{a}_{m} \middle| \mathbf{v} \right) \\
\end{pmatrix}
\end{align*}
\begin{thm}\label{2.1.4.5}
体$K$上で$\forall A_{mn} \in M_{mn}(K)\forall\mathbf{v} \in K^{n}$に対し、$A_{mn} = \left( \mathbf{a}_{i} \right)_{i \in \varLambda_{m}}$、$\mathbf{v} = \left( x_{i} \right)_{i \in \varLambda_{n}}$とすれば、次式が成り立つ。
\begin{align*}
A_{mn}\mathbf{v} = \sum_{i \in \varLambda_{n}} {x_{i}\mathbf{a}_{i}}
\end{align*}
\end{thm}\par
即ち、その行列$A_{mn}$が次のようにおかれれば、
\begin{align*}
A_{mn} = \begin{pmatrix}
\mathbf{a}_{1} & \mathbf{a}_{2} & \cdots & \mathbf{a}_{n} \\
\end{pmatrix}
\end{align*}
次式が成り立つことを意味する。
\begin{align*}
\begin{pmatrix}
\mathbf{a}_{1} & \mathbf{a}_{2} & \cdots & \mathbf{a}_{n} \\
\end{pmatrix}\begin{pmatrix}
x_{1} \\
x_{2} \\
 \vdots \\
x_{m} \\
\end{pmatrix} = x_{1}\mathbf{a}_{1} + x_{2}\mathbf{a}_{2} + \cdots + x_{n}\mathbf{a}_{n}
\end{align*}
\begin{proof}
体$K$上で$\forall A_{mn} \in M_{mn}(K)\forall\mathbf{v} \in K^{n}$に対し、$A_{mn} = \left( a_{ij} \right)_{(i,j) \in \varLambda_{m} \times \varLambda_{n}} = \left( \mathbf{a}_{i} \right)_{i \in \varLambda_{m}}$、$\mathbf{v} = \left( x_{i} \right)_{i \in \varLambda_{n}}$とすれば、次のようになる。
\begin{align*}
A_{mn}\mathbf{v} &= \left( \left( \mathbf{a}_{k} \middle| \mathbf{v} \right) \right)_{k \in \varLambda_{m}}\\
&= \left( \sum_{i \in \varLambda_{n}} {a_{ki}x_{i}} \right)_{k \in \varLambda_{m}}\\
&= \left( \sum_{i \in \varLambda_{n}} {x_{i}a_{ki}} \right)_{k \in \varLambda_{m}}\\
&= \sum_{i \in \varLambda_{n}} {x_{i}\left( a_{ki} \right)_{k \in \varLambda_{m}}}\\
&= \sum_{i \in \varLambda_{n}} {x_{i}\mathbf{a}_{i}}
\end{align*}
\end{proof}
\begin{thm}\label{2.1.4.6}
体$K$上で$\forall k,l \in K\forall\mathbf{v},\mathbf{w} \in K^{n}\forall A_{mn} \in M_{mn}(K)$に対し、次式が成り立つ。
\begin{align*}
A_{mn}\left( k\mathbf{v} + l\mathbf{w} \right) = \left( kA_{mn} \right)\mathbf{v} + \left( lA_{mn} \right)\mathbf{w} = k\left( A_{mn}\mathbf{v} \right) + l\left( A_{mn}\mathbf{w} \right)
\end{align*}
\end{thm}\par
これにより、$A_{mn}\left( k\mathbf{v} + l\mathbf{w} \right) = kA_{mn}\mathbf{v} + lA_{mn}\mathbf{w}$と書かれることができるようになる。
\begin{proof}
体$K$上で$\forall\mathbf{v},\mathbf{w} \in K^{n}\forall A_{mn} \in M_{mn}(K)$に対し、次のようにおかれれば、
\begin{align*}
\mathbf{v} = \left( x_{i} \right)_{i \in \varLambda_{n}},\ \ \mathbf{w} = \left( y_{i} \right)_{i \in \varLambda_{n}},\ \ A_{mn} = \left( a_{ij} \right)_{(i,j) \in \varLambda_{m} \times \varLambda_{n}}
\end{align*}
次のようになる。
\begin{align*}
A_{mn}\left( k\mathbf{v} + l\mathbf{w} \right) &= \left( a_{ij} \right)_{(i,j) \in \varLambda_{m} \times \varLambda_{n}}\left( k\left( x_{i} \right)_{i \in \varLambda_{n}} + l\left( y_{i} \right)_{i \in \varLambda_{n}} \right)\\
&= \left( a_{ij} \right)_{(i,j) \in \varLambda_{m} \times \varLambda_{n}}\left( kx_{i} + ly_{i} \right)_{i \in \varLambda_{n}}\\
&= \left( \sum_{j \in \varLambda_{n}} {a_{ij}\left( kx_{j} + ly_{j} \right)} \right)_{i \in \varLambda_{m}}\\
&= \left( \sum_{j \in \varLambda_{n}} \left( ka_{ij}x_{j} + la_{ij}y_{j} \right) \right)_{i \in \varLambda_{m}}\\
&= \left( \sum_{j \in \varLambda_{n}} {ka_{ij}x_{j}} + \sum_{j \in \varLambda_{n}} {la_{ij}y_{j}} \right)_{i \in \varLambda_{m}}\\
&= \left( \sum_{j \in \varLambda_{n}} {ka_{ij}x_{j}} \right)_{i \in \varLambda_{m}} + \left( \sum_{j \in \varLambda_{n}} {la_{ij}y_{j}} \right)_{i \in \varLambda_{m}}\\
&= \left( ka_{ij} \right)_{(i,j) \in \varLambda_{m} \times \varLambda_{n}}\left( x_{i} \right)_{i \in \varLambda_{n}} + \left( la_{ij} \right)_{(i,j) \in \varLambda_{m} \times \varLambda_{n}}\left( y_{i} \right)_{i \in \varLambda_{n}}\\
&= \left( kA_{mn} \right)\mathbf{v} + \left( lA_{mn} \right)\mathbf{w}
\end{align*}\par
また、同様にして次のようになる。
\begin{align*}
A_{mn}\left( k\mathbf{v} + l\mathbf{w} \right) &= \left( \sum_{j \in \varLambda_{n}} {ka_{ij}x_{j}} \right)_{i \in \varLambda_{m}} + \left( \sum_{j \in \varLambda_{n}} {la_{ij}y_{j}} \right)_{i \in \varLambda_{m}}\\
&= k\left( \sum_{j \in \varLambda_{n}} {a_{ij}x_{j}} \right)_{i \in \varLambda_{m}} + l\left( \sum_{j \in \varLambda_{n}} {a_{ij}y_{j}} \right)_{i \in \varLambda_{m}}\\
&= k\left( \left( a_{ij} \right)_{(i,j) \in \varLambda_{m} \times \varLambda_{n}}\left( x_{i} \right)_{i \in \varLambda_{n}} \right) + l\left( \left( a_{ij} \right)_{(i,j) \in \varLambda_{m} \times \varLambda_{n}}\left( y_{i} \right)_{i \in \varLambda_{n}} \right)\\
&= k\left( A_{mn}\mathbf{v} \right) + l\left( A_{mn}\mathbf{w} \right)
\end{align*}
\end{proof}
%\hypertarget{ux884cux5217ux3068ux7ddaux5f62ux5199ux50cf}{%
\subsubsection{行列と線形写像}%\label{ux884cux5217ux3068ux7ddaux5f62ux5199ux50cf}}
\begin{thm}\label{2.1.4.7}
体$K$上で$\forall\mathbf{v} \in K^{n}\forall A_{mn} \in M_{mn}(K)$を用いて次式のような写像$L_{A_{mn}}$について考えよう。
\begin{align*}
L_{A_{mn}}:K^{n} \rightarrow K^{m};\mathbf{v} \mapsto A_{mn}\mathbf{v}
\end{align*}
このとき、次のことが成り立つ。
\begin{itemize}
\item
  その写像$L_{A_{mn}}$は線形的である。
\item
  vector空間$K^{n}$の標準基底のうち第$j$成分が$1$でこれ以外の成分が$0$であるようなvector$\mathbf{e}_{j}$において、$A_{mn} = \begin{pmatrix}
  \mathbf{a}_{1} & \mathbf{a}_{2} & \cdots & \mathbf{a}_{n} \\
  \end{pmatrix}$とおかれると、次式が成り立つ、
\begin{align*}
L_{A_{mn}}\left( \mathbf{e}_{j} \right) = \mathbf{a}_{j}
\end{align*}
即ち、次式が成り立つ。
\begin{align*}
A_{mn} = \begin{pmatrix}
L_{A_{mn}}\left( \mathbf{e}_{1} \right) & L_{A_{mn}}\left( \mathbf{e}_{2} \right) & \cdots & L_{A_{mn}}\left( \mathbf{e}_{n} \right) \\
\end{pmatrix}
\end{align*}
\item
  任意の線形写像$L:K^{n} \rightarrow K^{m}$は、$\exists A_{mn} \in M_{mn}(K)$に対し、次式のように表される。
\begin{align*}
L:K^{n} \rightarrow K^{m};\mathbf{v} \mapsto A_{mn}\mathbf{v}
\end{align*}
\end{itemize}
\end{thm}
\begin{dfn}
ここで、この行列$A_{mn}$をその線形写像$L_{A_{mn}}$の行列、対応する行列という。
\end{dfn}
\begin{proof}
体$K$上で$\forall\mathbf{v} \in K^{n}\forall A_{mn} \in M_{mn}(K)$を用いて次式のような写像$L_{A_{mn}}$について考えよう。
\begin{align*}
L_{A_{mn}}:K^{n} \rightarrow K^{m};\mathbf{v} \mapsto A_{mn}\mathbf{v}
\end{align*}
$\forall k,l \in K\forall\mathbf{v},\mathbf{w} \in K^{n}$に対し、定理\ref{2.1.4.6}より次のようになる。
\begin{align*}
L_{A_{mn}}\left( k\mathbf{v} + l\mathbf{w} \right) &= A_{mn}\left( k\mathbf{v} + l\mathbf{w} \right)\\
&= kA_{mn}\mathbf{v} + lA_{mn}\mathbf{w}\\
&= kL_{A_{mn}}\left( \mathbf{v} \right) + lL_{A_{mn}}\left( \mathbf{w} \right)
\end{align*}
ゆえに、その写像$L_{A_{mn}}$は線形的である。\par
vector空間$K^{n}$の標準基底のうち第$j$成分が$1$でこれ以外の成分が$0$であるようなvector$\mathbf{e}_{j}$において、$A_{mn} = \begin{pmatrix}
\mathbf{a}_{1} & \mathbf{a}_{2} & \cdots & \mathbf{a}_{n} \\
\end{pmatrix}$とおかれると、定理\ref{2.1.4.5}より次のようになる。
\begin{align*}
L_{A_{mn}}\left( \mathbf{e}_{j} \right) &= A_{mn}\mathbf{e}_{j}\\
&= \begin{pmatrix}
\mathbf{a}_{1} & \mathbf{a}_{2} & \cdots & \mathbf{a}_{n} \\
\end{pmatrix}\begin{pmatrix}
0 \\
 \vdots \\
1 \\
 \vdots \\
0 \\
\end{pmatrix}\\
&= 0\mathbf{a}_{1} + \cdots + 1\mathbf{a}_{j} + \cdots + 0\mathbf{a}_{n} = \mathbf{a}_{j}
\end{align*}
これにより、次式が得られる。
\begin{align*}
A_{mn} = \begin{pmatrix}
\mathbf{a}_{1} & \mathbf{a}_{2} & \cdots & \mathbf{a}_{n} \\
\end{pmatrix} = \begin{pmatrix}
L_{A_{mn}}\left( \mathbf{e}_{1} \right) & L_{A_{mn}}\left( \mathbf{e}_{2} \right) & \cdots & L_{A_{mn}}\left( \mathbf{e}_{n} \right) \\
\end{pmatrix}
\end{align*}\par
任意の線形写像$L:K^{n} \rightarrow K^{m}$が与えられたとき、$\forall\mathbf{v} \in K^{n}$に対し、次のようにおかれれば、
\begin{align*}
\mathbf{v} = \left( x_{i} \right)_{i \in \varLambda_{n}}
\end{align*}
直ちに次式が成り立つ。
\begin{align*}
\mathbf{v} = \left( x_{i} \right)_{i \in \varLambda_{n}} = \sum_{i \in \varLambda_{n}} {x_{i}\mathbf{e}_{i}}
\end{align*}
したがって、次のようになるので、
\begin{align*}
L\left( \mathbf{v} \right) = L\left( \sum_{i \in \varLambda_{n}} {x_{i}\mathbf{e}_{i}} \right) = \sum_{i \in \varLambda_{n}} {x_{i}L\left( \mathbf{e}_{i} \right)}
\end{align*}
定理\ref{2.1.4.7}より次のようになる。
\begin{align*}
L\left( \mathbf{v} \right) &= \sum_{i \in \varLambda_{n}} {x_{i}L\left( \mathbf{e}_{i} \right)}\\
&= \begin{pmatrix}
L\left( \mathbf{e}_{1} \right) & L\left( \mathbf{e}_{2} \right) & \cdots & L\left( \mathbf{e}_{n} \right) \\
\end{pmatrix}\begin{pmatrix}
x_{1} \\
x_{2} \\
 \vdots \\
x_{m} \\
\end{pmatrix}\\
&= \begin{pmatrix}
L\left( \mathbf{e}_{1} \right) & L\left( \mathbf{e}_{2} \right) & \cdots & L\left( \mathbf{e}_{n} \right) \\
\end{pmatrix}\mathbf{v}
\end{align*}
あとは、次式のようにおかれれば、
\begin{align*}
A_{mn} = \begin{pmatrix}
L\left( \mathbf{e}_{1} \right) & L\left( \mathbf{e}_{2} \right) & \cdots & L\left( \mathbf{e}_{n} \right) \\
\end{pmatrix}
\end{align*}
$\exists A_{mn} \in M_{mn}(K)$に対し、次式のように表される。
\begin{align*}
L:K^{n} \rightarrow K^{m};\mathbf{v} \mapsto A_{mn}\mathbf{v}
\end{align*}
\end{proof}
\begin{thm}\label{2.1.4.8}
体$K$上で$\forall A_{lm} \in M_{lm}(K)\forall B_{mn} \in M_{mn}(K)$に対し、2つの線形写像$L_{A_{lm}}:K^{m} \rightarrow K^{l}$、$L_{B_{mn}}:K^{n} \rightarrow K^{m}$が与えられたとき、その合成写像$L_{A_{lm}} \circ L_{B_{mn}}:K^{n} \rightarrow K^{l}$も線形写像であり、さらに、定理\ref{2.1.4.7}より対応する行列$C_{ln}$が存在する。このとき、$C_{ln} = A_{lm}B_{mn}$が成り立つ。
\end{thm}
\begin{proof}
体$K$上で$\forall A_{lm} \in M_{lm}(K)\forall B_{mn} \in M_{mn}(K)$に対し、2つの線形写像$L_{A_{lm}}:K^{m} \rightarrow K^{l}$、$L_{B_{mn}}:K^{n} \rightarrow K^{m}$が与えられたとき、その合成写像$L_{A_{lm}} \circ L_{B_{mn}}:K^{n} \rightarrow K^{l}$も線形写像であり、さらに、定理\ref{2.1.4.7}より対応する行列$C_{ln}$が存在する。\par
ここで、行列たち$A_{lm}$、$B_{mn}$、$C_{ln}$を$A_{lm} = \left( \mathbf{a}_{j} \right)_{j \in \varLambda_{m}} = \left( a_{ij} \right)_{(i,j) \in \varLambda_{l} \times \varLambda_{m}}$、$B_{mn} = \left( \mathbf{b}_{j} \right)_{j \in \varLambda_{n}} = \left( b_{ij} \right)_{(i,j) \in \varLambda_{m} \times \varLambda_{n}}$、$C_{ln} = \left( \mathbf{c}_{j} \right)_{j \in \varLambda_{n}} = \left( c_{ij} \right)_{(i,j) \in \varLambda_{l} \times \varLambda_{n}}$とおき、$K^{n}$の標準直交基底を$\left\langle \mathbf{e}_{i} \right\rangle_{i \in \varLambda_{n}}$とおくと、定理\ref{2.1.4.7}より次式が成り立つので、
\begin{align*}
L_{B_{mn}}\left( \mathbf{e}_{j} \right) = \mathbf{b}_{j},\ \ L_{C_{ln}}\left( \mathbf{e}_{j} \right) = \mathbf{c}_{j}
\end{align*}
したがって、次のようになる。
\begin{align*}
\mathbf{c}_{j} &= L_{C_{ln}}\left( \mathbf{e}_{j} \right)\\
&= L_{A_{lm}} \circ L_{B_{mn}}\left( \mathbf{e}_{j} \right)\\
&= L_{A_{lm}}\left( L_{B_{mn}}\left( \mathbf{e}_{j} \right) \right)\\
&= L_{A_{lm}}\left( \mathbf{b}_{j} \right)\\
&= A_{lm}\mathbf{b}_{j}\\
&= \sum_{k \in \varLambda_{m}} {b_{kj}\mathbf{a}_{j}}
\end{align*}
順序付けられた組の定義より次のようになる。
\begin{align*}
C_{ln} &= \left( \mathbf{c}_{j} \right)_{j \in \varLambda_{n}}\\
&= \left( \sum_{k \in \varLambda_{m}} {b_{kj}\mathbf{a}_{j}} \right)_{j \in \varLambda_{n}}\\
&= \left( \sum_{k \in \varLambda_{m}} {a_{ik}b_{kj}} \right)_{(i,j) \in \varLambda_{l} \times \varLambda_{n}}\\
&= \left( a_{ij} \right)_{(i,j) \in \varLambda_{l} \times \varLambda_{m}}\left( b_{ij} \right)_{(i,j) \in \varLambda_{m} \times \varLambda_{n}}\\
&= A_{lm}B_{mn}
\end{align*}
\end{proof}
\begin{thm}\label{2.1.4.9}
体$K$上で次式のような写像$L_{\bullet}$は線形同型写像である。
\begin{align*}
L_{\bullet}:M_{mn}(K) \rightarrow L\left( K^{n},K^{m} \right);A_{mn} \mapsto L_{A_{mn}}
\end{align*}
\end{thm}
\begin{proof}体$K$上で次式のような写像$L_{\bullet}$を考える。
\begin{align*}
L_{\bullet}:M_{mn}(K) \rightarrow L\left( K^{n},K^{m} \right);A_{mn} \mapsto L_{A_{mn}}
\end{align*}\par
このとき、任意の線形写像$L:K^{n} \rightarrow K^{m}$は$A_{mn} \in M_{m,n}(K)$なる行列$A_{mn}$を用いて次式のように表され、
\begin{align*}
L:K^{n} \rightarrow K^{m};\mathbf{v} \mapsto L\left( \mathbf{v} \right) = A_{mn}\mathbf{v}
\end{align*}
これ$L$からその行列$A_{mn}$へうつす写像が写像$L_{\bullet}$の逆写像$L_{\bullet}^{- 1}$となるので、写像$L_{\bullet}$は全単射$L_{\bullet}:M_{mn}(K)\overset{\sim}{\rightarrow}L\left( K^{n},K^{m} \right)$である。\par
また、$\forall k,l \in K\forall A_{mn},B_{mn} \in M_{mn}(K)\forall\mathbf{v} \in K^{n}$に対し、次のようになる。
\begin{align*}
\left( L_{\bullet}\left( kA_{mn} + lB_{mn} \right) \right)\left( \mathbf{v} \right) &= L_{kA_{mn} + lB_{mn}}\left( \mathbf{v} \right)\\
&= \left( kA_{mn} + lB_{mn} \right)\left( \mathbf{v} \right)\\
&= \left( lA_{mn} \right)\left( \mathbf{v} \right) + \left( lB_{mn} \right)\left( \mathbf{v} \right)\\
&= kA_{mn}\left( \mathbf{v} \right) + lB_{mn}\left( \mathbf{v} \right)\\
&= kL_{A_{mn}}\left( \mathbf{v} \right) + lL_{B_{mn}}\left( \mathbf{v} \right)\\
&= k\left( L_{\bullet}\left( A_{mn} \right) \right)\left( \mathbf{v} \right) + l\left( L_{\bullet}\left( B_{mn} \right) \right)\left( \mathbf{v} \right)\\
&= \left( kL_{\bullet}\left( A_{mn} \right) + lL_{\bullet}\left( B_{mn} \right) \right)\left( \mathbf{v} \right)
\end{align*}
ゆえに、$L_{\bullet}\left( kA_{mn} + lB_{mn} \right) = kL_{\bullet}\left( A_{mn} \right) + lL_{\bullet}\left( B_{mn} \right)$が得られる。
\end{proof}
%\hypertarget{ux884cux5217ux306eux968eux6570}{%
\subsubsection{行列の階数}%\label{ux884cux5217ux306eux968eux6570}}
\begin{thm}\label{2.1.4.10}
体$K$上で$\forall A_{mn} \in M_{mn}(K)$に対し、線形写像$L_{A_{mn}}:K^{n} \rightarrow K^{m};\mathbf{v} \mapsto A_{mn}\mathbf{v}$が与えられたとき、次のようにおかれれば、
\begin{align*}
A_{mn} = \left( \mathbf{a}_{j} \right)_{j \in \varLambda_{n}} = \begin{pmatrix}
\mathbf{a}_{1} & \mathbf{a}_{2} & \cdots & \mathbf{a}_{n} \\
\end{pmatrix}
\end{align*}
その値域$V\left( L_{A_{mn}} \right)$は$n$つのvectors$\mathbf{a}_{j}$によって張られるvector空間$K^{m}$の部分空間${\mathrm{span} }\left\{ \mathbf{a}_{j} \right\}_{j \in \varLambda_{n}}$に等しい、即ち、次式が成り立つ。
\begin{align*}
V\left( L_{A_{mn}} \right) = {\mathrm{span} }\left\{ \mathbf{a}_{j} \right\}_{j \in \varLambda_{n}}
\end{align*}
\end{thm}
\begin{proof}
体$K$上で$\forall A_{mn} \in M_{mn}(K)$に対し、線形写像$L_{A_{mn}}:K^{n} \rightarrow K^{m};\mathbf{v} \mapsto A_{mn}\mathbf{v}$が与えられたとき、次のようにおかれよう。
\begin{align*}
A_{mn} = \left( \mathbf{a}_{j} \right)_{j \in \varLambda_{n}} = \begin{pmatrix}
\mathbf{a}_{1} & \mathbf{a}_{2} & \cdots & \mathbf{a}_{n} \\
\end{pmatrix}
\end{align*}
その値域$V\left( L_{A_{mn}} \right)$について、$\mathbf{v} = \left( x_{i} \right)_{i \in \varLambda_{n}}$とすれば、定理\ref{2.1.4.5}より次のようになる。
\begin{align*}
V\left( L_{A_{mn}} \right) &= \left\{ L_{A_{mn}}\left( \mathbf{v} \right) \in K^{m}|\mathbf{v} \in K^{n} \right\}\\
&= \left\{ A_{mn}\mathbf{v} \in K^{m}|\mathbf{v} \in K^{n} \right\}\\
&= \left\{ \sum_{j \in \varLambda_{n}} {x_{j}\mathbf{a}_{j}} \in K^{m} \middle| \mathbf{v} \in K^{n} \right\}\\
&= \left\{ \mathbf{w} \in K^{m} \middle| \mathbf{w} = \sum_{j \in \varLambda_{n}} {x_{j}\mathbf{a}_{j}} \land \mathbf{v} \in K^{n} \right\}\\
&= {\mathrm{span} }\left\{ \mathbf{a}_{j} \right\}_{j \in \varLambda_{n}} \subseteq K^{m}
\end{align*}
\end{proof}
\begin{dfn}
体$K$上で$\forall A_{mn} = \left( \mathbf{a}_{j} \right)_{j \in \varLambda_{n}} \in M_{mn}(K)$に対し、その行列$A_{mn}$の$n$つの列vectors$\mathbf{a}_{j}$によって張られるvector空間$K^{n}$の部分空間${\mathrm{span} }\left\{ \mathbf{a}_{j} \right\}_{j \in \varLambda_{n}}$を行列$A_{mn}$の列空間といいその次元$\dim{{\mathrm{span} }\left\{ \mathbf{a}_{j} \right\}_{j \in \varLambda_{n}}}$をその行列$A_{mn}$の列階数という。
\end{dfn}\par
ここで、線形写像$L_{A_{mn}}:K^{n} \rightarrow K^{m};\mathbf{v} \mapsto A_{mn}\mathbf{v}$が与えられたとき、定理\ref{2.1.4.10}よりその値域$V\left( L_{A_{mn}} \right)$は$n$つの$vectors\mathbf{a}_{j}$によって張られるvector空間$K^{m}$の部分空間${\mathrm{span} }\left\{ \mathbf{a}_{j} \right\}_{j \in \varLambda_{n}}$に等しいのであった。
\begin{dfn}
体$K$上で$\forall A_{mn} = \left(^{t}\mathbf{b}_{i} \right)_{i \in \varLambda_{m}} \in M_{mn}(K)$に対し、その行列$A_{mn}$の$m$つの行vectors$\mathbf{b}_{i}$によって張られるvector空間$K^{n}$の部分空間${\mathrm{span} }\left\{ \mathbf{b}_{i} \right\}_{i \in \varLambda_{m}}$を行列$A_{mn}$の行空間といいその次元$\dim{{\mathrm{span} }\left\{ \mathbf{b}_{i} \right\}_{i \in \varLambda_{m}}}$をその行列$A_{mn}$の行階数という。
\end{dfn}\par
線形写像$L_{A_{mn}}:K^{n} \rightarrow K^{m};\mathbf{v} \mapsto A_{mn}\mathbf{v}$の核$\ker L_{A_{mn}}$において、その写像$L_{A_{mn}}$を用いたvector$\mathbf{v}$の式$L_{A_{mn}}\left( \mathbf{v} \right) = A_{mn}\mathbf{v} = \mathbf{0}$が与えられたとき、その式$L_{A_{mn}}\left( \mathbf{v} \right) = A_{mn}\mathbf{v} = \mathbf{0}$を方程式とみなせば、このような$\mathbf{v}$はその方程式$L_{A_{mn}}\left( \mathbf{v} \right) = A_{mn}\mathbf{v} = \mathbf{0}$の解にあたることから、その写像$L_{A_{mn}}$の核$\ker L_{A_{mn}}$はその方程式$L_{A_{mn}}\left( \mathbf{v} \right) = A_{mn}\mathbf{v} = \mathbf{0}$の解空間ともいうのであった。ここで、$A_{mn}\mathbf{v} = \mathbf{0}$は次のように変形されることができるので、
\begin{align*}
\left\{ \begin{matrix}
a_{11}x_{1} + a_{12}x_{2} + \cdots + a_{1n}x_{n} = 0 \\
a_{21}x_{1} + a_{22}x_{2} + \cdots + a_{2n}x_{n} = 0 \\
 \vdots \\
a_{m1}x_{1} + a_{m2}x_{2} + \cdots + a_{mn}x_{n} = 0 \\
\end{matrix} \right.\ 
\end{align*}
これより、その核$\ker L_{A_{mn}}$は後述する上の$n$元連立$1$次方程式の解全体の集合でもある。
\begin{thm}\label{2.1.4.11}
体$K$上で$\forall A_{mn} \in M_{mn}(K)$に対し、$A_{mn} = \left( \mathbf{a}_{j} \right)_{j \in \varLambda_{n}} = \left(^{t}\mathbf{b}_{i} \right)_{i \in \varLambda_{m}}$とすれば、次式が成り立つ。
\begin{align*}
\dim{{\mathrm{span} }\left\{ \mathbf{a}_{j} \right\}_{j \in \varLambda_{n}}} = \dim{{\mathrm{span} }\left\{ \mathbf{b}_{i} \right\}_{i \in \varLambda_{m}}}
\end{align*}
\end{thm}
\begin{dfn}
体$K$上で$\forall A_{mn} = \left( \mathbf{a}_{j} \right)_{j \in \varLambda_{n}} \in M_{mn}(K)$に対し、その行列$A_{mn}$の列階数$\dim{{\mathrm{span} }\left\{ \mathbf{a}_{j} \right\}_{j \in \varLambda_{n}}}$をその行列$A_{mn}$の階数といい${\mathrm{rank} }A_{mn}$と書く。
\end{dfn}\par
この定理は次のようにして示される。
\begin{enumerate}
\item
  $\dim{{\mathrm{span} }\left\{ \mathbf{b}_{i} \right\}_{i \in \varLambda_{m}}} = s$とおきその行空間${\mathrm{span} }\left\{ \mathbf{b}_{i} \right\}_{i \in \varLambda_{m}}$の基底を$\left\langle \mathbf{b}_{i} \right\rangle_{i \in \varLambda_{s}}$とおく。
\item
  $\forall i \in \varLambda_{s}$に対し、$\left( \mathbf{b}_{i} \middle| \mathbf{v} \right) = 0$なるvector空間$K^{n}$の元$\mathbf{v}$を考えたとき、$\forall j \in \varLambda_{m} \setminus \varLambda_{s}$に対し、$\left( \mathbf{b}_{j} \middle| \mathbf{v} \right) = 0$が成り立つことを示す。
\item
  $A_{sn}' = \left(^{t}\mathbf{b}_{i} \right)_{i \in \varLambda_{s}} \in M_{sn}(K)$なる行列$A_{sn}'$が考えられたとき、2. より、$\forall\mathbf{v} \in K^{n}$に対し、$A_{mn}\mathbf{v} = \mathbf{0}$が成り立つならそのときに限り、$A_{sn}'\mathbf{v} = \mathbf{0}$が成り立つことを示す。
\item
  線形写像たち$L_{A_{mn}}:K^{n} \rightarrow K^{m};\mathbf{v} \mapsto A_{mn}\mathbf{v}$、$L_{A_{sn}'}:K^{n} \rightarrow K^{s};\mathbf{v} \mapsto A_{sn}'\mathbf{v}$が考えられ${\mathrm{nullity}}L_{A_{mn}} = {\mathrm{nullity}}L_{A_{sn}'}$が成り立つことを示す。
\item
  $A_{sn}' = \left( \mathbf{a}_{j}' \right)_{j \in \varLambda_{n}}$として次元公式を用いて次式を示す。
\begin{align*}
\dim{{\mathrm{span} }\left\{ \mathbf{a}_{j} \right\}_{j \in \varLambda_{n}}} = \dim{V\left( L_{A_{sn}'} \right)}
\end{align*}
\item
  $V\left( L_{A_{sn}'} \right) \subseteq K^{s}$を用いて次式を示す。
\begin{align*}
\dim{{\mathrm{span} }\left\{ \mathbf{a}_{j} \right\}_{j \in \varLambda_{n}}} \leq \dim{{\mathrm{span} }\left\{ \mathbf{b}_{i} \right\}_{i \in \varLambda_{m}}}
\end{align*}
\item
  転置行列${}^{t}A_{mn}$のときを考えることで示すべきことが示される。
\end{enumerate}
\begin{proof}
体$K$上で$\forall A_{mn} \in M_{mn}(K)$に対し、$A_{mn} = \left( \mathbf{a}_{j} \right)_{j \in \varLambda_{n}} = \left(^{t}\mathbf{b}_{i} \right)_{i \in \varLambda_{m}}$とおかれよう。$\dim{{\mathrm{span} }\left\{ \mathbf{b}_{i} \right\}_{i \in \varLambda_{m}}} = s$とおくと、その族$\left\{ \mathbf{b}_i \right\}_{i \in \varLambda_{m} } $がその行空間${\mathrm{span} }\left\{ \mathbf{b}_{i} \right\}_{i \in \varLambda_{m}}$を生成するので、その族$\left\{ \mathbf{b}_i \right\}_{i \in \varLambda_{m} } $のうちどれかがその行空間${\mathrm{span} }\left\{ \mathbf{b}_{i} \right\}_{i \in \varLambda_{m}}$の基底をなすvectorとなる。したがって、その行列$A_{mn}$の行空間${\mathrm{span} }\left\{ \mathbf{b}_{i} \right\}_{i \in \varLambda_{m}}$の基底を$\left\langle \mathbf{b}_{i} \right\rangle_{i \in \varLambda_{s}}$としてもよい。\par
このとき、$\forall j \in \varLambda_{m} \setminus \varLambda_{s}$に対し、$\mathbf{a}_{j} = \sum_{i \in \varLambda_{s}} {k_{ij}\mathbf{b}_{i}}$が成り立つ。もし、vector空間$K^{n}$の元$\mathbf{v}$が、$\forall i \in \varLambda_{s}$に対し、$\left( \mathbf{b}_{i} \middle| \mathbf{v} \right) = 0$を満たせば、$\forall j \in \varLambda_{m} \setminus \varLambda_{s}$に対し、次のようになる。
\begin{align*}
\left( \mathbf{a}_{j} \middle| \mathbf{v} \right) &= \left( \sum_{i \in \varLambda_{s}} {k_{ij}\mathbf{b}_{i}} \middle| \mathbf{v} \right)\\
&= \sum_{i \in \varLambda_{s}} {k_{ij}\left( \mathbf{b}_{i} \middle| \mathbf{v} \right)}\\
&= \sum_{i \in \varLambda_{s}} {k_{ij}0} = 0
\end{align*}
ゆえに、このvector$\mathbf{v}$は、$\forall i \in \varLambda_{m}$に対し、$\left( \mathbf{b}_{i} \middle| \mathbf{v} \right) = 0$を満たす。\par
そこで、$A_{sn}' = \left(^{t}\mathbf{b}_{i} \right)_{i \in \varLambda_{s}} \in M_{sn}(K)$なる行列$A_{sn}'$が考えられたとき、$\forall\mathbf{v} \in K^{n}$に対し、$A_{mn}\mathbf{v} = \mathbf{0}$が成り立つなら、$\forall i \in \varLambda_{m}$に対し、$\left( \mathbf{b}_{i} \middle| \mathbf{v} \right) = 0$が成り立つことになる。したがって、$\forall i \in \varLambda_{s}$に対し、$\left( \mathbf{b}_{i} \middle| \mathbf{v} \right) = 0$も成り立つので、$A_{sn}'\mathbf{v} = \mathbf{0}$も成り立つ。逆に、$A_{sn}'\mathbf{v} = \mathbf{0}$が成り立つなら、$\forall i \in \varLambda_{s}$に対し、$\left( \mathbf{b}_{i} \middle| \mathbf{v} \right) = 0$が成り立ち、上記の議論により、$\forall i \in \varLambda_{m}$に対し、$\left( \mathbf{b}_{i} \middle| \mathbf{v} \right) = 0$も成り立つので、$A_{mn}\mathbf{v} = \mathbf{0}$も成り立つ。ゆえに、$A_{mn}\mathbf{v} = \mathbf{0}$が成り立つならそのときに限り、$A_{sn}'\mathbf{v} = \mathbf{0}$が成り立つ。\par
線形写像たち$L_{A_{mn}}:K^{n} \rightarrow K^{m};\mathbf{v} \mapsto A_{mn}\mathbf{v}$、$L_{A_{sn}'}:K^{n} \rightarrow K^{s};\mathbf{v} \mapsto A_{sn}'\mathbf{v}$が考えられれば、$\forall\mathbf{v} \in K^{n}$に対し、$\mathbf{v} \in \ker L_{A_{mn}}$が成り立つならそのときに限り、$L_{A_{mn}}\left( \mathbf{v} \right) = A_{mn}\mathbf{v} = \mathbf{0}$が成り立つ。上記の議論により、これが成り立つならそのときに限り、$L_{A_{sn}'}\left( \mathbf{v} \right) = A_{sn}'\mathbf{v} = \mathbf{0}$が成り立つので、これが成り立つならそのときに限り、$\mathbf{v} \in \ker L_{A_{sn}'}$が成り立つ。したがって、$\ker L_{A_{mn}} = \ker L_{A_{sn}'}$が得られる。よって、${\mathrm{nullity}}L_{A_{mn}} = {\mathrm{nullity}}L_{A_{sn}'}$が成り立つ。\par
$A_{sn}' = \left( \mathbf{a}_{j}' \right)_{j \in \varLambda_{n}}$とすれば、定理\ref{2.1.4.10}と次元公式より次のようになる。
\begin{align*}
\dim{{\mathrm{span} }\left\{ \mathbf{a}_{j} \right\}_{j \in \varLambda_{n}}} &= \dim{V\left( L_{A_{mn}} \right)}\\
&= {\mathrm{rank} }L_{A_{mn}}\\
&= n - {\mathrm{nullity}}L_{A_{mn}}\\
&= n - {\mathrm{nullity}}L_{A_{sn}'}\\
&= {\mathrm{rank} }L_{A_{sn}'}\\
&= \dim{V\left( L_{A_{sn}'} \right)}
\end{align*}
ここで、$V\left( L_{A_{sn}} \right) \subseteq K^{s}$が成り立つので、次のようになる。
\begin{align*}
\dim{{\mathrm{span} }\left\{ \mathbf{a}_{j} \right\}_{j \in \varLambda_{n}}} &= \dim{V\left( L_{A_{sn}'} \right)}\\
&\leq \dim K^{s} = s\\
&= \dim{{\mathrm{span} }\left\{ \mathbf{b}_{i} \right\}_{i \in \varLambda_{m}}}
\end{align*}\par
その行列$A_{mn}$の転置行列${}^{t}A_{mn}$についても同様にして考えることで次のようになることから、
\begin{align*}
\dim{{\mathrm{span} }\left\{ \mathbf{a}_{j} \right\}_{j \in \varLambda_{n}}} \geq \dim{{\mathrm{span} }\left\{ \mathbf{b}_{i} \right\}_{i \in \varLambda_{m}}}
\end{align*}
よって、次式が成り立つ。
\begin{align*}
\dim{{\mathrm{span} }\left\{ \mathbf{a}_{j} \right\}_{j \in \varLambda_{n}}} = \dim{{\mathrm{span} }\left\{ \mathbf{b}_{i} \right\}_{i \in \varLambda_{m}}}
\end{align*}
\end{proof}
\begin{thm}\label{2.1.4.12}
体$K$上で$\forall A_{mn} \in M_{mn}(K)$に対し、線形写像$L_{A_{mn}}:K^{n} \rightarrow K^{m};\mathbf{v} \mapsto A_{mn}\mathbf{v}$が与えられたとき、$A_{mn} = \left( \mathbf{a}_{j} \right)_{j \in \varLambda_{n}} = \left(^{t}\mathbf{b}_{i} \right)_{i \in \varLambda_{m}}$とし$n$つのvectors$\mathbf{a}_{j}$のうち線形独立なものの最大個数を$r$、$m$つのvectors$\mathbf{b}_{i}$のうち線形独立なものの最大個数を$s$とすれば、次式が成り立つ。
\begin{align*}
{\mathrm{rank} }A_{mn} &= {\mathrm{rank} }L_{A_{mn}} = \dim{V\left( L_{A_{mn}} \right)}\\
&= \dim{{\mathrm{span} }\left\{ \mathbf{a}_{j} \right\}_{j \in \varLambda_{n}}} = r\\
&= \dim{{\mathrm{span} }\left\{ \mathbf{b}_{i} \right\}_{i \in \varLambda_{m}}} = s
\end{align*}
\end{thm}
\begin{proof}
体$K$上で$\forall A_{mn} \in M_{mn}(K)$に対し、線形写像$L_{A_{mn}}:K^{n} \rightarrow K^{m};\mathbf{v} \mapsto A_{mn}\mathbf{v}$が与えられたとき、$A_{mn} = \left( \mathbf{a}_{j} \right)_{j \in \varLambda_{n}} = \left(^{t}\mathbf{b}_{i} \right)_{i \in \varLambda_{m}}$とし$n$つのvectors$\mathbf{a}_{j}$のうち線形独立なものの最大個数を$r$、$m$つのvectors$\mathbf{b}_{i}$のうち線形独立なものの最大個数を$s$とすれば、定義より明らかに${\mathrm{rank} }A_{mn} = \dim{{\mathrm{span} }\left\{ \mathbf{a}_{j} \right\}_{j \in \varLambda_{n}}}$が成り立つ。\par
定理\ref{2.1.4.10}より$V\left( L_{A_{mn}} \right) = {\mathrm{span} }\left\{ \mathbf{a}_{j} \right\}_{j \in \varLambda_{n}}$が成り立つので、次式が成り立つ。
\begin{align*}
{\mathrm{rank} }L_{A_{mn}} = \dim{V\left( L_{A_{mn}} \right)} = \dim{{\mathrm{span} }\left\{ \mathbf{a}_{j} \right\}_{j \in \varLambda_{n}}}
\end{align*}\par
また、その列空間${\mathrm{span} }\left\{ \mathbf{a}_{j} \right\}_{j \in \varLambda_{n}}$は$n$個のvectors$\mathbf{a}_{j}$によって張られているので、その列空間${\mathrm{span} }\left\{ \mathbf{a}_{j} \right\}_{j \in \varLambda_{n}}$の基底は$n$個のvectors$\mathbf{a}_{j}$のうちどれかである。そこで、$r$つの線形独立なvectors$\mathbf{a}_{j}$の添数全体の集合を$\varLambda'$とすれば、その族$\left\{ \mathbf{a}_{j} \right\}_{j \in \varLambda'} $がその列空間${\mathrm{span} }\left\{ \mathbf{a}_{j} \right\}_{j \in \varLambda_{n}}$の基底となるので、${card}\varLambda' = r$より$\dim{{\mathrm{span} }\left\{ \mathbf{a}_{j} \right\}_{j \in \varLambda_{n}}} = r$が成り立つ。同様にして、$\dim{{\mathrm{span} }\left\{ \mathbf{b}_{i} \right\}_{i \in \varLambda_{m}}} = s$が成り立つことが示される。\par
最後に、定理\ref{2.1.4.11}より$\dim{{\mathrm{span} }\left\{ \mathbf{a}_{j} \right\}_{j \in \varLambda_{n}}} = \dim{{\mathrm{span} }\left\{ \mathbf{b}_{i} \right\}_{i \in \varLambda_{m}}}$が成り立つ。
\end{proof}
\begin{thm}\label{2.1.4.13}
体$K$上で$\forall A_{nn} \in M_{nn}(K)$に対し、線形写像$L_{A_{nn}}:K^{n} \rightarrow K^{n};\mathbf{v} \mapsto A_{nn}\mathbf{v}$について、次のことは同値である。
\begin{itemize}
\item
  その線形写像$L_{A_{nn}}:K^{n} \rightarrow K^{n}$が全単射である。
\item
  その線形写像$L_{A_{nn}}:K^{n} \rightarrow K^{n}$が全射である。
\item
  その線形写像$L_{A_{nn}}:K^{n} \rightarrow K^{n}$が単射である。
\end{itemize}
\end{thm}
\begin{proof}
定理\ref{2.1.2.15}より明らかである。
\end{proof}
\begin{thm}\label{2.1.4.14}
体$K$上で$\forall A_{nn} \in M_{nn}(K)$に対し、次のことは同値である。
\begin{itemize}
\item
  線形写像$L_{A_{nn}}:K^{n} \rightarrow K^{n};\mathbf{v} \mapsto A_{nn}\mathbf{v}$が全単射である。
\item
  ${\mathrm{rank} }A_{nn} = n$が成り立つ。
\item
  $A_{nn} = \left( \mathbf{a}_{j} \right)_{j \in \varLambda_{n}}$とおくと、族$\left\{ \mathbf{a}_j \right\}_{j \in \varLambda_{n} } $が線形独立である。
\item
  $A_{nn} = \left(^{t}\mathbf{b}_{i} \right)_{i \in \varLambda_{n}}$とおくと、族$\left\{ \mathbf{b}_i \right\}_{i \in \varLambda_{n} } $が線形独立である。
\item
  $\exists X_{nn} \in M_{nn}(K)$に対し、$A_{nn}X_{nn} = X_{nn}A_{nn} = I_{n}$が成り立つ。
\item
  $\ker L_{A_{mn}} = \left\{ \mathbf{0} \right\}$が成り立つ。
\item
  $\forall\mathbf{v} \in K^{n}$に対し、$A_{nn}\mathbf{v} = \mathbf{0}$が成り立つならそのときに限り、$\mathbf{v} = \mathbf{0}$が成り立つ。
\end{itemize}
\end{thm}
\begin{dfn}\label{正則行列}
体$K$上の$n$次正方行列$A_{nn}$が上の条件たちのうち少なくとも1つ満たせば、その他全ての条件たちを満たす。このような行列$A_{nn}$を正則行列などといいこのような行列$A_{nn}$全体の集合を$\mathrm{GL}_{n}(K)$と書く場合がある。このような行列$A_{nn}$に対応する上から3つ目の式での行列$X_{nn}$をその行列$A_{nn}$の逆行列といい、既に述べられたようにこれが存在するなら一意的なので、これを$A_{nn}^{-1}$と書く。
\end{dfn}
\begin{proof}
体$K$上で$\forall A_{nn} \in M_{nn}(K)$に対し、線形写像$L_{A_{nn}}:K^{n} \rightarrow K^{n};\mathbf{v} \mapsto A_{nn}\mathbf{v}$が全単射であるとき、明らかに、この写像$L_{A_{nn}}$は全射であるかつ、単射でもあり定理\ref{2.1.2.15}より次のようになる。
\begin{align*}
{\mathrm{rank} }A_{nn} = {\mathrm{rank} }L_{A_{nn}} = \dim K^{n} = n
\end{align*}
逆に、これが成り立つなら、次のようになるので、
\begin{align*}
\dim K^{n} = n = {\mathrm{rank} }A_{nn} = {\mathrm{rank} }L_{A_{nn}}
\end{align*}
定理\ref{2.1.2.15}よりその写像$L_{A_{nn}}$は全射であるかつ、単射でもあるので、その線形写像$L_{A_{nn}}:K^{n} \rightarrow K^{n};\mathbf{v} \mapsto A_{nn}\mathbf{v}$は全単射である。ゆえに、次のことは同値である。
\begin{itemize}
\item
  線形写像$L_{A_{nn}}:K^{n} \rightarrow K^{n};\mathbf{v} \mapsto A_{nn}\mathbf{v}$が全単射である。
\item
  ${\mathrm{rank} }A_{nn} = n$が成り立つ。
\end{itemize}\par
${\mathrm{rank} }A_{nn} = n$が成り立つとき、$A_{nn} = \left( \mathbf{a}_{j} \right)_{j \in \varLambda_{n}}$とおくと、定理\ref{2.1.4.12}より${\mathrm{rank} }A_{nn} = n = \dim{{\mathrm{span} }\left\{ \mathbf{a}_{j} \right\}_{j \in \varLambda_{n}}}$が成り立つので、組$\left\langle \mathbf{a}_{j} \right\rangle_{j \in \varLambda_{n}}$がその列空間${\mathrm{span} }\left\{ \mathbf{a}_{j} \right\}_{j \in \varLambda_{n}}$の基底をなす。したがって、その族$\left\{ \mathbf{a}_j \right\}_{j \in \varLambda_{n} } $が線形独立である。逆に、これが成り立つなら、組$\left\langle \mathbf{a}_{j} \right\rangle_{j \in \varLambda_{n}}$がその列空間${\mathrm{span} }\left\{ \mathbf{a}_{j} \right\}_{j \in \varLambda_{n}}$の基底をなすので、定理\ref{2.1.4.12}より${\mathrm{rank} }A_{nn} = n$が成り立つ。行空間についても同様にして示されるので、以上の議論により、次のことは同値である。
\begin{itemize}
\item
  ${\mathrm{rank} }A_{nn} = n$が成り立つ。
\item
  $A_{nn} = \left( \mathbf{a}_{j} \right)_{j \in \varLambda_{n}}$とおくと、族$\left\{ \mathbf{a}_j \right\}_{j \in \varLambda_{n} } $が線形独立である。
\item
  $A_{nn} = \left(^{t}\mathbf{b}_{i} \right)_{i \in \varLambda_{n}}$とおくと、族$\left\{ \mathbf{b}_i \right\}_{i \in \varLambda_{n} } $が線形独立である。
\end{itemize}\par
その線形写像$L_{A_{nn}}:K^{n} \rightarrow K^{n};\mathbf{v} \mapsto A_{nn}\mathbf{v}$が全単射であるとき、これの逆写像$L_{A_{nn}}^{- 1}:K^{n} \rightarrow K^{n}$が存在するので、これの行列を$X_{nn}$とおくと、次式が成り立つ。
\begin{align*}
L_{A_{nn}} \circ L_{A_{nn}}^{- 1} = L_{A_{nn}}^{- 1} \circ L_{A_{nn}} = I_{K^{n}}
\end{align*}
ここで、$I_{K^{n}}:K^{n} \rightarrow K^{n};\mathbf{v} \mapsto \mathbf{v} = I_{n}\mathbf{v}$が成り立つかつ、線形写像たち$L_{A_{nn}} \circ L_{A_{nn}}^{- 1}$、$L_{A_{nn}}^{- 1} \circ L_{A_{nn}}$の行列がそれぞれ$A_{nn}X_{nn}$、$X_{nn}A_{nn}$と与えられるので、次式が成り立つ。
\begin{align*}
A_{nn}X_{nn} = X_{nn}A_{nn} = I_{n}
\end{align*}
一方で、$\exists A_{nn}^{- 1} \in M_{nn}(K)$に対し、次式が成り立つなら、
\begin{align*}
A_{nn}A_{nn}^{- 1} = A_{nn}^{- 1}A_{nn} = I_{n}
\end{align*}
線形写像$L_{A_{nn}^{- 1}}:K^{n} \rightarrow K^{n};\mathbf{v} \mapsto A_{nn}^{- 1}\mathbf{v}$が考えられれば、次式が成り立つので、
\begin{align*}
L_{A_{nn}} \circ L_{A_{nn}^{- 1}} = L_{A_{nn}^{- 1}} \circ L_{A_{nn}} = I_{K^{n}}
\end{align*}
その写像$L_{A_{nn}}$の逆写像$L_{A_{nn}}^{- 1}$が存在し、したがって、その線形写像$L_{A_{nn}}:K^{n} \rightarrow K^{n};\mathbf{v} \mapsto A_{nn}\mathbf{v}$は全単射である。ゆえに、次のことは同値である。
\begin{itemize}
\item
  線形写像$L_{A_{nn}}:K^{n} \rightarrow K^{n};\mathbf{v} \mapsto A_{nn}\mathbf{v}$が全単射である。
\item
  $\exists A_{nn}^{- 1} \in M_{nn}(K)$に対し、$A_{nn}A_{nn}^{- 1} = A_{nn}^{- 1}A_{nn} = I_{n}$が成り立つ。
\end{itemize}\par
その線形写像$L_{A_{nn}}:K^{n} \rightarrow K^{n};\mathbf{v} \mapsto A_{nn}\mathbf{v}$が全単射であるとき、明らかに、この写像$L_{A_{nn}}$は全射であるかつ、単射でもあり定理\ref{2.1.2.12}より$\ker L_{A_{mn}} = \left\{ \mathbf{0} \right\}$が成り立つ。逆に、これが成り立つなら、${\mathrm{nullity}}L_{A_{mn}} = \dim{\ker L_{A_{mn}}} = 0$が成り立つので、定理\ref{2.1.2.15}よりその線形写像$L_{A_{nn}}$は全射であるかつ、単射でもあるので、その線形写像$L_{A_{nn}}:K^{n} \rightarrow K^{n};\mathbf{v} \mapsto A_{nn}\mathbf{v}$は全単射である。ゆえに、次のことは同値である。
\begin{itemize}
\item
  線形写像$L_{A_{nn}}:K^{n} \rightarrow K^{n};\mathbf{v} \mapsto A_{nn}\mathbf{v}$が全単射である。
\item
  $\ker L_{A_{mn}} = \left\{ \mathbf{0} \right\}$が成り立つ。
\end{itemize}\par
$\ker L_{A_{mn}} = \left\{ \mathbf{0} \right\}$が成り立つならそのときに限り、$\forall\mathbf{v} \in K^{n}$に対し、$\mathbf{v} \in \ker L_{A_{nn}}$が成り立つならそのときに限り、$\mathbf{v} = \mathbf{0}$が成り立つ。ここで、$\mathbf{v} \in \ker L_{A_{nn}}$が成り立つならそのときに限り、$A_{nn}\mathbf{v} = \mathbf{0}$が成り立つことに注意すれば、これが成り立つならそのときに限り、$\forall\mathbf{v} \in K^{n}$に対し、$A_{nn}\mathbf{v} = \mathbf{0}$が成り立つならそのときに限り、$\mathbf{v} = \mathbf{0}$が成り立つことになる。ゆえに、次のことは同値である。
\begin{itemize}
\item
  $\ker L_{A_{nn}} = \left\{ \mathbf{0} \right\}$が成り立つ。
\item
  $\forall\mathbf{v} \in K^{n}$に対し、$A_{nn}\mathbf{v} = \mathbf{0}$が成り立つならそのときに限り、$\mathbf{v} = \mathbf{0}$が成り立つ。
\end{itemize}
\end{proof}
\begin{thm}\label{2.1.4.15}
体$K$上で$\forall A_{lm} \in M_{lm}(K)\forall B_{mn} \in M_{mn}(K)$に対し、次式が成り立つ。
\begin{align*}
{\mathrm{rank} }{A_{lm}B_{mn}} \leq {\mathrm{rank} }A_{lm},\ \ {\mathrm{rank} }{A_{lm}B_{mn}} \leq {\mathrm{rank} }B_{mn}
\end{align*}
\end{thm}
\begin{proof}
体$K$上で$\forall A_{mn} \in M_{mn}(K)\forall B_{lm} \in M_{lm}(K)$に対し、次のような線形写像たち$L_{A_{mn}}$、$L_{B_{lm}}$、$L_{A_{lm}B_{mn}}$が考えられれば、
\begin{align*}
L_{A_{mn}}&:K^{n} \rightarrow K^{m};\mathbf{v} \mapsto A_{mn}\mathbf{v}\\
L_{B_{lm}}&:K^{m} \rightarrow K^{l};\mathbf{v} \mapsto B_{lm}\mathbf{v}\\
L_{A_{lm}B_{mn}}&:K^{n} \rightarrow K^{l};\mathbf{v} \mapsto A_{lm}B_{mn}\mathbf{v}
\end{align*}
定理\ref{2.1.2.16} 、定理\ref{2.1.4.8}、定理\ref{2.1.4.12}より次のようになる。
\begin{align*}
{\mathrm{rank} }{A_{lm}B_{mn}} &= {\mathrm{rank} }L_{A_{lm}B_{mn}}\\
&= {\mathrm{rank} }{L_{A_{lm}} \circ L_{B_{mn}}}\\
&\leq {\mathrm{rank} }L_{A_{lm}}\\
&= {\mathrm{rank} }A_{lm}\\
{\mathrm{rank} }{A_{lm}B_{mn}} &= {\mathrm{rank} }L_{A_{lm}B_{mn}}\\
&= {\mathrm{rank} }{L_{A_{lm}} \circ L_{B_{mn}}}\\
&\leq {\mathrm{rank} }B_{mn}\\
&= {\mathrm{rank} }B_{mn}
\end{align*}
\end{proof}
\begin{thm}\label{2.1.4.16}
体$K$上で$\forall A_{lm} \in M_{lm}(K)\forall B_{mn} \in M_{mn}(K)$に対し、$A_{lm}B_{mn} = O_{ln}$が成り立つなら、${\mathrm{rank} }A_{lm} + {\mathrm{rank} }B_{mn} \leq m$が成り立つ。
\end{thm}
\begin{proof}
体$K$上で$\forall A_{lm} \in M_{lm}(K)\forall B_{mn} \in M_{mn}(K)$に対し、$A_{lm}B_{mn} = O_{ln}$が成り立つなら、次のような線形写像たち$L_{A_{mn}}$、$L_{B_{lm}}$、$L_{A_{lm}B_{mn}}$が考えられれば、
\begin{align*}
L_{A_{lm}}&:K^{m} \rightarrow K^{l};\mathbf{v} \mapsto A_{lm}\mathbf{v}\\
L_{B_{lm}}&:K^{n} \rightarrow K^{m};\mathbf{v} \mapsto B_{mn}\mathbf{v}\\
L_{A_{lm}B_{mn}}&:K^{n} \rightarrow K^{l};\mathbf{v} \mapsto A_{lm}B_{mn}\mathbf{v}
\end{align*}
$\forall\mathbf{v} \in K^{m}$に対し、$\mathbf{v} \in V\left( L_{B_{lm}} \right)$が成り立つなら、$\exists\mathbf{w} \in K^{n}$に対し、$\mathbf{v} = B_{mn}\mathbf{w}$が成り立つ。このとき、次のようになるので、
\begin{align*}
L_{A_{lm}}\left( \mathbf{v} \right) = A_{lm}\mathbf{v} = A_{lm}B_{mn}\mathbf{w} = O_{ln}\mathbf{w} = \mathbf{0}
\end{align*}
$\mathbf{v} \in \ker L_{A_{lm}}$が成り立つ。これにより、$V\left( L_{B_{mn}} \right) \subseteq \ker L_{A_{lm}}$が成り立つので、次元公式と定理\ref{2.1.4.1}よりしたがって、次のようになる。
\begin{align*}
{\mathrm{rank} }A_{lm} + {\mathrm{rank} }B_{mn} &= {\mathrm{rank} }L_{A_{lm}} + {\mathrm{rank} }L_{B_{mn}}\\
&= m - {\mathrm{nullity}}L_{A_{mn}} + \dim{V\left( L_{B_{mn}} \right)}\\
&\leq m - \dim{\ker L_{A_{lm}}} + \dim{\ker L_{A_{lm}}} = m
\end{align*}
\end{proof}
\begin{thebibliography}{50}
  \bibitem{1}
    松坂和夫, 線型代数入門, 岩波書店, 1980. 新装版第2刷 p41-65,92-113 ISBN978-4-00-029872-8
  \bibitem{2}
    松坂和夫, 代数系入門, 岩波書店, 1976. 新装版第2刷 p45-51,107-112,170-199 ISBN978-4-00-029873-5
  \bibitem{3}
    中西敏浩. "数学基礎 IV (線形代数学) 講義ノート". 島根大学. \url{http:// www.math.shimane-u.ac.jp/~tosihiro/skks4main.pdf} (2020-9-7 取得)
\end{thebibliography}
\end{document}
