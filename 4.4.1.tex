\documentclass[dvipdfmx]{jsarticle}
\setcounter{section}{4}
\setcounter{subsection}{0}
\usepackage{xr}
\externaldocument{4.1.1}
\externaldocument{4.1.5}
\externaldocument{4.1.10}
\externaldocument{4.2.2}
\externaldocument{4.2.5}
\externaldocument{4.2.6}
\usepackage{amsmath,amsfonts,amssymb,array,comment,mathtools,url,docmute}
\usepackage{longtable,booktabs,dcolumn,tabularx,mathtools,multirow,colortbl,xcolor}
\usepackage[dvipdfmx]{graphics}
\usepackage{bmpsize}
\usepackage{amsthm}
\usepackage{enumitem}
\setlistdepth{20}
\renewlist{itemize}{itemize}{20}
\setlist[itemize]{label=•}
\renewlist{enumerate}{enumerate}{20}
\setlist[enumerate]{label=\arabic*.}
\setcounter{MaxMatrixCols}{20}
\setcounter{tocdepth}{3}
\newcommand{\rotin}{\text{\rotatebox[origin=c]{90}{$\in $}}}
\renewcommand{\thesection}{第\arabic{section}部}
\renewcommand{\thesubsection}{\arabic{section}.\arabic{subsection}}
\renewcommand{\thesubsubsection}{\arabic{section}.\arabic{subsection}.\arabic{subsubsection}}
\everymath{\displaystyle}
\allowdisplaybreaks[4]
\usepackage{vtable}
\theoremstyle{definition}
\newtheorem{thm}{定理}[subsection]
\newtheorem*{thm*}{定理}
\newtheorem{dfn}{定義}[subsection]
\newtheorem*{dfn*}{定義}
\newtheorem{axs}[dfn]{公理}
\newtheorem*{axs*}{公理}
\renewcommand{\headfont}{\bfseries}
\makeatletter
  \renewcommand{\section}{%
    \@startsection{section}{1}{\z@}%
    {\Cvs}{\Cvs}%
    {\normalfont\huge\headfont\raggedright}}
\makeatother
\makeatletter
  \renewcommand{\subsection}{%
    \@startsection{subsection}{2}{\z@}%
    {0.5\Cvs}{0.5\Cvs}%
    {\normalfont\LARGE\headfont\raggedright}}
\makeatother
\makeatletter
  \renewcommand{\subsubsection}{%
    \@startsection{subsubsection}{3}{\z@}%
    {0.4\Cvs}{0.4\Cvs}%
    {\normalfont\Large\headfont\raggedright}}
\makeatother
\makeatletter
\renewenvironment{proof}[1][\proofname]{\par
  \pushQED{\qed}%
  \normalfont \topsep6\p@\@plus6\p@\relax
  \trivlist
  \item\relax
  {
  #1\@addpunct{.}}\hspace\labelsep\ignorespaces
}{%
  \popQED\endtrivlist\@endpefalse
}
\makeatother
\renewcommand{\proofname}{\textbf{証明}}
\usepackage{tikz,graphics}
\usepackage[dvipdfmx]{hyperref}
\usepackage{pxjahyper}
\hypersetup{
 setpagesize=false,
 bookmarks=true,
 bookmarksdepth=tocdepth,
 bookmarksnumbered=true,
 colorlinks=false,
 pdftitle={},
 pdfsubject={},
 pdfauthor={},
 pdfkeywords={}}
\begin{document}
%\hypertarget{ux9670ux95a2ux6570ux5b9aux7406}{%
\subsection{陰関数定理}%\label{ux9670ux95a2ux6570ux5b9aux7406}}
%\hypertarget{ux9670ux95a2ux6570}{%
\subsubsection{陰関数}%\label{ux9670ux95a2ux6570}}
\begin{dfn}
関数$f:\mathbb{R}^{l} \times \mathbb{R}^{m} \rightarrow \mathbb{R}^{n}$が与えられたとき、$\exists\left( \mathbf{a},\mathbf{b} \right) \in \mathbb{R}^{l} \times \mathbb{R}^{m}$に対し、$f\left( \mathbf{a},\mathbf{b} \right) = \mathbf{0}$が成り立つとする。このとき、その点$\mathbf{a}$のある開近傍$U$と関数$g:\mathbb{R}^{l} \rightarrow \mathbb{R}^{m}$が存在して、$\forall\mathbf{x} \in U$に対し、$f\begin{pmatrix}
\mathbf{x} \\
g\left( \mathbf{x} \right) \\
\end{pmatrix} = \mathbf{0}$が成り立つとき、その関数$g$をその関数$f$によって陰に定められた関数、陰関数という。
\end{dfn}\par
これはただ1つ決まるとは限らない。例えば、関数$f:\mathbb{R}^{2} \rightarrow \mathbb{R};(x,y) \mapsto x^{2} + y^{2} - 1$から定まる陰関数$g$は$g:\mathbb{R} \rightarrow \mathbb{R};x \mapsto \pm \sqrt{1 - x^{2}}$と2通り決まる。
%\hypertarget{ux9670ux95a2ux6570ux5b9aux7406-1}{%
\subsubsection{陰関数定理}%\label{ux9670ux95a2ux6570ux5b9aux7406-1}}
\begin{thm}[陰関数定理]\label{4.4.1.1}
$U \subseteq \mathbb{R}^{n} \times \mathbb{R}$なる開集合$U$を用いた$C^{1}$級関数$f:U \rightarrow \mathbb{R};\begin{pmatrix}
\mathbf{x} \\
y \\
\end{pmatrix} \mapsto f\begin{pmatrix}
\mathbf{x} \\
y \\
\end{pmatrix}$が、$\exists\begin{pmatrix}
\mathbf{a} \\
b \\
\end{pmatrix} \in \mathbb{R}^{n} \times \mathbb{R}$に対し、$f\begin{pmatrix}
\mathbf{a} \\
b \\
\end{pmatrix} = 0$かつ$\partial_{n + 1}f\begin{pmatrix}
\mathbf{a} \\
b \\
\end{pmatrix} \neq 0$が成り立つとき、$V \times W \subseteq U$かつ$V \subseteq \mathbb{R}^{n}$かつ$W \subseteq \mathbb{R}$なるそれらの点々$\mathbf{a}$、$b$の開近傍たちそれぞれ$V$、$W$と関数$g:V \rightarrow W$が存在して、$g_{\downarrow}:V \rightarrow \mathbb{R}^{n} \times W;\mathbf{x} \mapsto \begin{pmatrix}
\mathbf{x} \\
g\left( \mathbf{x} \right) \\
\end{pmatrix}$とすれば、次のことが成り立つ。
\begin{itemize}
\item
  その開近傍$V$上で次式が成り立つ。
\begin{align*}
\partial_{n + 1}f \circ g_{\downarrow}:V \rightarrow \mathbb{R} \neq 0
\end{align*}
\item
  $g\left( \mathbf{a} \right) = b$が成り立つ。
\item
  その関数$g$はその開近傍$V$で連続である。
\item
  $\forall\begin{pmatrix}
  \mathbf{x} \\
  y \\
  \end{pmatrix} \in V \times W$に対し、$f\begin{pmatrix}
  \mathbf{x} \\
  y \\
  \end{pmatrix} = 0$が成り立つならそのときに限り、$y = g\left( \mathbf{x} \right)$が成り立つ。
\end{itemize}
さらに、その陰関数$g$について、次のことが成り立つ。
\begin{itemize}
\item
  その陰関数$g$はその開近傍$V$上で$C^{1}$級である。
\item
  その開近傍$V$上で次式が成り立つ。
\begin{align*}
\begin{pmatrix}
\mathrm{grad}g \\
 - 1 \\
\end{pmatrix} = - \frac{\mathrm{grad}f}{\partial_{n + 1}f} \circ g_{\downarrow}
\end{align*}
\item
  その開近傍$V$上で、$\forall i \in \varLambda_{n}$に対し、次式が成り立つ。
\begin{align*}
\partial_{i}g = - \frac{\partial_{i}f}{\partial_{n + 1}f} \circ g_{\downarrow}
\end{align*}
\end{itemize}
\end{thm}
\begin{proof}
$U \subseteq \mathbb{R}^{n} \times \mathbb{R}$なる開集合$U$を用いた$C^{1}$級関数$f:U \rightarrow \mathbb{R};\begin{pmatrix}
\mathbf{x} \\
y \\
\end{pmatrix} \mapsto f\begin{pmatrix}
\mathbf{x} \\
y \\
\end{pmatrix}$が、$\exists\begin{pmatrix}
\mathbf{a} \\
b \\
\end{pmatrix} \in \mathbb{R}^{n} \times \mathbb{R}$に対し、$f\begin{pmatrix}
\mathbf{a} \\
b \\
\end{pmatrix} = 0$かつ$\partial_{n + 1}f\begin{pmatrix}
\mathbf{a} \\
b \\
\end{pmatrix} \neq 0$が成り立つとき、関数$- f$も考えることで$\partial_{n + 1}f\begin{pmatrix}
\mathbf{a} \\
b \\
\end{pmatrix} > 0$が成り立つとしてもよい。その関数$f$は$C^{1}$級であることから偏導関数$\partial_{n + 1}f$も連続であるので、定理\ref{4.1.10.16}より$\forall\varepsilon \in \mathbb{R}^{+}\exists\delta \in \mathbb{R}^{+}$に対し、$V\left( \partial_{n + 1}f|U\left( \begin{pmatrix}
\mathbf{a} \\
b \\
\end{pmatrix},\delta \right) \cap U \right) \subseteq U\left( \partial_{n + 1}f\begin{pmatrix}
\mathbf{a} \\
b \\
\end{pmatrix},\varepsilon \right)$が成り立つ、即ち、$V_{0} \times W \subseteq U$かつ$V_{0} \subseteq \mathbb{R}^{n}$かつ$W \subseteq \mathbb{R}$なるそれらの点々$\mathbf{a}$、$b$の開近傍たちそれぞれ$V_{0}$、$W$が存在して、$\forall\begin{pmatrix}
\mathbf{x} \\
y \\
\end{pmatrix} \in V_{0} \times W$に対し、$\partial_{n + 1}f\begin{pmatrix}
\mathbf{a} \\
b \\
\end{pmatrix} - \varepsilon < \partial_{n + 1}f\begin{pmatrix}
\mathbf{x} \\
y \\
\end{pmatrix} < \partial_{n + 1}f\begin{pmatrix}
\mathbf{a} \\
b \\
\end{pmatrix} + \varepsilon$が成り立つ。ここで、その正の実数$\varepsilon$を十分小さくとれば、$0 < \partial_{n + 1}f\begin{pmatrix}
\mathbf{x} \\
y \\
\end{pmatrix}$が成り立つこともできる。\par
ここで、$\mathbf{x} = \mathbf{a}$として、関数$f^{\mathbf{a}}:W \rightarrow \mathbb{R};y \mapsto f\begin{pmatrix}
\mathbf{a} \\
y \\
\end{pmatrix}$が考えられると、その関数$f^{\mathbf{a}}$は上記の議論により狭義単調増加するので、$\forall y \in W$に対し、次のことが成り立つ。
\begin{align*}
b < y \Rightarrow 0 = f\begin{pmatrix}
\mathbf{a} \\
b \\
\end{pmatrix} = f^{\mathbf{a}}(b) < f^{\mathbf{a}}(y) = f\begin{pmatrix}
\mathbf{a} \\
y \\
\end{pmatrix},\ \ y < b \Rightarrow f\begin{pmatrix}
\mathbf{a} \\
y \\
\end{pmatrix} = f^{\mathbf{a}}(y) < f^{\mathbf{a}}(b) = f\begin{pmatrix}
\mathbf{a} \\
b \\
\end{pmatrix} = 0
\end{align*}
そこで、$\forall y_{-},y_{+} \in W$に対し、$y_{-} < b < y_{+}$が成り立つなら、$f\begin{pmatrix}
\mathbf{a} \\
y_{-} \\
\end{pmatrix} < 0 < f\begin{pmatrix}
\mathbf{a} \\
y_{+} \\
\end{pmatrix}$が成り立ち、その関数$f$は定理\ref{4.2.6.3}より連続であるから、定理\ref{4.1.10.16}より$\forall y \in W\forall\varepsilon \in \mathbb{R}^{+}\exists\delta \in \mathbb{R}^{+}$に対し、$V\left( f|U\left( \begin{pmatrix}
\mathbf{a} \\
y \\
\end{pmatrix},\delta \right) \cap U \right) \subseteq U\left( f\begin{pmatrix}
\mathbf{a} \\
y \\
\end{pmatrix},\varepsilon \right)$が成り立つ、即ち、$V \times W \subseteq U$かつ$V \subseteq V_{0}$かつ$W \subseteq \mathbb{R}$なるそれらの点々$\mathbf{a}$、$b$の開近傍たちそれぞれ$V$、$W$が存在して、$\forall\begin{pmatrix}
\mathbf{x} \\
y \\
\end{pmatrix} \in V \times W$に対し、$f\begin{pmatrix}
\mathbf{a} \\
y \\
\end{pmatrix} - \varepsilon < f\begin{pmatrix}
\mathbf{x} \\
y \\
\end{pmatrix} < f\begin{pmatrix}
\mathbf{a} \\
y \\
\end{pmatrix} + \varepsilon$が成り立つ。これにより、その正の実数$\varepsilon$を十分小さくとれば、定理\ref{4.1.1.8}より$f\begin{pmatrix}
\mathbf{x} \\
y_{-} \\
\end{pmatrix} < 0 < f\begin{pmatrix}
\mathbf{x} \\
y_{+} \\
\end{pmatrix}$が成り立つこともできる。\par
ここで、$V \subseteq V_{0}$が成り立つことから、もちろん、$\forall\begin{pmatrix}
\mathbf{x} \\
y \\
\end{pmatrix} \in V \times W$に対し、$0 < \partial_{n + 1}f\begin{pmatrix}
\mathbf{x} \\
y \\
\end{pmatrix}$が成り立つので、関数$f^{\mathbf{x}}:W \rightarrow \mathbb{R};y \mapsto f\begin{pmatrix}
\mathbf{x} \\
y \\
\end{pmatrix}$が考えられると、その関数$f^{\mathbf{x}}$は狭義単調増加するので、$f\begin{pmatrix}
\mathbf{x} \\
y_{-} \\
\end{pmatrix} < 0 < f\begin{pmatrix}
\mathbf{x} \\
y_{+} \\
\end{pmatrix}$が成り立つことによって中間値の定理より、$\forall\mathbf{x} \in V\exists!y \in \left( y_{-},y_{+} \right) \subseteq W$に対し、$f\begin{pmatrix}
\mathbf{x} \\
y \\
\end{pmatrix} = 0$が成り立つ。その点$x$からこのような点$y$へ写す関数$g:V \rightarrow W$と関数$g_{\downarrow}:V \rightarrow \mathbb{R}^{n} \times W;\mathbf{x} \mapsto \begin{pmatrix}
\mathbf{x} \\
g\left( \mathbf{x} \right) \\
\end{pmatrix}$が考えられれば、上記の議論により、$\forall\mathbf{x} \in V$に対し、$g_{\downarrow}\left( \mathbf{x} \right) = \begin{pmatrix}
\mathbf{x} \\
g\left( \mathbf{x} \right) \\
\end{pmatrix} \in V_{0} \times W$が成り立つので、$\partial_{n + 1}f \circ g_{\downarrow}\left( \mathbf{x} \right) = \partial_{n + 1}f\begin{pmatrix}
\mathbf{x} \\
g\left( \mathbf{x} \right) \\
\end{pmatrix} \neq 0$が成り立ち、したがって、その開近傍$V$上で次式が成り立つ。
\begin{align*}
\partial_{n + 1}f \circ g_{\downarrow}:V \rightarrow \mathbb{R} \neq 0
\end{align*}
また、定義より$f\begin{pmatrix}
\mathbf{a} \\
b \\
\end{pmatrix} = 0$が成り立つので、$g\left( \mathbf{a} \right) = b$が成り立ち、さらに、$\forall\begin{pmatrix}
\mathbf{x} \\
y \\
\end{pmatrix} \in V \times W$に対し、$y = g\left( \mathbf{x} \right)$が成り立つなら、$f\begin{pmatrix}
\mathbf{x} \\
y \\
\end{pmatrix} = 0$が成り立つ。逆に、$f\begin{pmatrix}
\mathbf{x} \\
y \\
\end{pmatrix} = 0$が成り立つなら、先ほどの議論でその点$y$が一意的に存在することにより、$y = g\left( \mathbf{x} \right)$が成り立つ。\par
上記の開近傍$V$に属する任意の点$\alpha$に収束するその開近傍$V$の任意の点列$\left( \mathbf{a}_{n} \right)_{n \in \mathbb{N}}$に対し、$V(g) \subseteq \left( y_{-},y_{+} \right)$が成り立つので、その開近傍$W$の元の列$\left( g\left( \mathbf{a}_{n} \right) \right)_{n \in \mathbb{N}}$は有界である。ここで、定理\ref{4.1.5.7}、即ち、Bolzano-Weierstrassの定理の定理よりその元の列$\left( g\left( \mathbf{a}_{n} \right) \right)_{n \in \mathbb{N}}$には収束する部分列$\left( g\left( \mathbf{a}_{n_{k}} \right) \right)_{k \in \mathbb{N}}$が存在して、これの極限を$\eta$とおく。もちろん、$\eta \in \mathrm{cl}\left( y_{-},y_{+} \right) = \left[ y_{-},y_{+} \right]$が成り立つ。ここで、$\forall k \in \mathbb{N}$に対し、$f\begin{pmatrix}
\mathbf{a}_{n_{k}} \\
g\left( \mathbf{a}_{n_{k}} \right) \\
\end{pmatrix} = 0$が成り立つので、$k \rightarrow \infty$とすれば、定理\ref{4.1.5.6}より$f\begin{pmatrix}
\alpha \\
\eta \\
\end{pmatrix} = 0$が得られるので、その関数$g$の定義より$\eta = g(\alpha)$が成り立つ。ここで、$\lim_{n \rightarrow \infty}{g\left( \mathbf{a}_{n} \right)} = g(\alpha)$が成り立たないと仮定すると、$\exists\varepsilon \in \mathbb{R}^{+}\forall\delta \in \mathbb{N}\exists n \in \mathbb{N}$に対し、$\delta \leq n$が成り立つかつ、$\left\| g\left( \mathbf{a}_{n} \right) - g(\alpha) \right\| \geq \varepsilon$が成り立つので、このような自然数たち$n$を取り出すことで、$\left\{ g\left( \mathbf{a}_{n_{k}} \right) \right\}_{k \in \mathbb{N}} \subseteq U\left( g(\alpha),\varepsilon \right)$が成り立たないようなその元の列$\left( g\left( \mathbf{a}_{n} \right) \right)_{n \in \mathbb{N}}$の部分列が存在する。ここで、$V(g) \subseteq \left[ y_{-},y_{+} \right]$が成り立つので、その部分列は定理\ref{4.1.5.7}、即ち、Bolzano-Weierstrassの定理の定理よりその元の列$\left( g\left( \mathbf{a}_{n} \right) \right)_{n \in \mathbb{N}}$には収束する部分列$\left( g\left( \mathbf{a}_{n_{k}} \right) \right)_{k \in \mathbb{N}}$をもち、上記の議論により、これは点$g(\alpha)$に収束する。一方で、$\forall k \in \mathbb{N}$に対し、仮定より$\varepsilon \leq \left\| g\left( \mathbf{a}_{n_{k}} \right) - g(\alpha) \right\|$が成り立つので、$k \rightarrow \infty$とすれば、$0 < \varepsilon \leq \left\| g(\alpha) - g(\alpha) \right\| = 0$が成り立つことになるが、これは矛盾している。よって、$\lim_{n \rightarrow \infty}{g\left( \mathbf{a}_{n} \right)} = g(\alpha)$が成り立ちその関数$g$はその開近傍$V$で連続であることが示された。\par
さらに、上の陰関数$g:V \rightarrow W$について、そのvector空間$\mathbb{R}^{n}$の標準直交基底$\left\langle \mathbf{e}_{i} \right\rangle_{i \in \varLambda_{n}}$がとられて、絶対値が十分小さい任意の$0$でない実数$h$を用いて次式のようにおく。
\begin{align*}
k = g\left( \mathbf{x} + h\mathbf{e}_{i} \right) - g\left( \mathbf{x} \right)
\end{align*}
さらに、関数$\varphi_{i}$が次式のように定義されれば、
\begin{align*}
\varphi_{i} = \left( \varphi_{ij} \right)_{j \in \varLambda_{n + 1}}:[ 0,1] \rightarrow \mathbb{R}^{n} \times \mathbb{R};t \mapsto \begin{pmatrix}
\mathbf{x} + th\mathbf{e}_{i} \\
g\left( \mathbf{x} \right) + tk \\
\end{pmatrix}
\end{align*}
次のようになる。
\begin{align*}
f \circ \varphi_{i}(0) &= f\begin{pmatrix}
\mathbf{x} + 0h\mathbf{e}_{i} \\
g\left( \mathbf{x} \right) + 0k \\
\end{pmatrix} \\
&= f\begin{pmatrix}
\mathbf{x} \\
g\left( \mathbf{x} \right) \\
\end{pmatrix} = 0\\
f \circ \varphi_{i}(1) &= f\begin{pmatrix}
\mathbf{x} + h\mathbf{e}_{i} \\
g\left( \mathbf{x} \right) + k \\
\end{pmatrix} \\
&= f\begin{pmatrix}
\mathbf{x} + h\mathbf{e}_{i} \\
g\left( \mathbf{x} \right) + g\left( \mathbf{x} + h\mathbf{e}_{i} \right) - g\left( \mathbf{x} \right) \\
\end{pmatrix}\\
&= f\begin{pmatrix}
\mathbf{x} + h\mathbf{e}_{i} \\
g\left( \mathbf{x} + h\mathbf{e}_{i} \right) \\
\end{pmatrix} = 0
\end{align*}
ここで、定理\ref{4.2.2.2}、即ち、Rolleの定理よりその関数$f \circ \varphi_{i}$が有界閉区間$[ 0,1]$で連続であるかつ、その開区間$(0,1)$で微分可能であり、さらに、$f \circ \varphi_{i}(0) = f \circ \varphi_{i}(1)$が成り立つので、次式のようになる実数$\theta$がその開区間$(0,1)$で存在する。
\begin{align*}
\partial(f \circ \varphi_{i})(\theta) = 0
\end{align*}
したがって、$\mathbf{x} = \left( x_{j} \right)_{j \in \varLambda_{n}}$、$\mathbf{e}_{i} = \left( \delta_{ij} \right)_{j \in \varLambda_{n}}$とすれば、次のようになる\footnote{ここで、Einstein縮約記法に則ってみることにしよう。これは和をとるとき、予め約束した文字が2回あらわれたらば、その2文字を添字とする和の記号を省略するものである。$\partial \varphi_{ij} =h\delta_{ij}$、$\partial \varphi_{i,n+1} =k$に注意すれば、$j\in \varLambda_{n}$として次のようになる。
\begin{align*}
0 &= \partial \left( f\circ \varphi_{i} \right) \\
&= \left( \partial_{j} f \circ \varphi_{i} \right) \partial \varphi_{ij} + \left( \partial_{n+1} f \circ \varphi_{i} \right) \partial \varphi_{i,n+1} \\
&= \left( \partial_{j} f \circ \varphi_{i} \right) h \delta_{ij} + \left( \partial_{n+1} f \circ \varphi_{i} \right) k\\
&= h \left( \partial_{i} f \circ \varphi_{i} \right) + k \left( \partial_{n+1} f \circ \varphi_{i} \right) 
\end{align*} 
したがって、次式が得られる。
\begin{align*}
\frac{k}{h} = - \frac{\partial_{i} f \circ \varphi_{i} }{\partial_{n+1} f \circ \varphi_{i} }
\end{align*} }。
\begin{align*}
0 &= \partial(f \circ \varphi_{i})(\theta) = J_{f \circ \varphi_{i}}(\theta) = \left( J_{f} \circ \varphi_{i} \right)J_{\varphi_{i}}(\theta)\\
&= \left({}^{t}\mathrm{grad}f \circ \varphi_{i} \right)\partial\varphi_{i}(\theta)\\
&= \sum_{j \in \varLambda_{n + 1}} {\left( \partial_{j}f \circ \varphi_{i} \right)\partial\varphi_{ij}(\theta)}\\
&= \sum_{j \in \varLambda_{n}} {\partial_{j}f \circ \varphi_{i}(\theta)\partial\varphi_{ij}(\theta)} + \partial_{n + 1}f \circ \varphi_{i}(\theta)\partial\varphi_{i,n + 1}(\theta)\\
&= \sum_{j \in \varLambda_{n}} {\partial_{j}f \circ \varphi_{i}(\theta)\left. \ \frac{\partial}{\partial t}\left( x_{j} + th\delta_{ij} \right) \right|_{t = \theta}} + \partial_{n + 1}f \circ \varphi_{i}(\theta)\left. \ \frac{\partial}{\partial t}\left( g\left( \mathbf{x} \right) + tk \right) \right|_{t = \theta}\\
&= \partial_{i}f \circ \varphi_{i}(\theta)\left. \ \frac{\partial}{\partial t}\left( x_{j} + th \right) \right|_{t = \theta} + \partial_{n + 1}f \circ \varphi_{i}(\theta)\left. \ \frac{\partial}{\partial t}\left( g\left( \mathbf{x} \right) + tk \right) \right|_{t = \theta}\\
&= h\partial_{i}f \circ \varphi_{i}(\theta) + k\partial_{n + 1}f \circ \varphi_{i}(\theta)\\
&= \left( h\partial_{i}f \circ \varphi_{i} + k\partial_{n + 1}f \circ \varphi_{i} \right)(\theta)
\end{align*}
ここで、$h \neq 0$に注意すれば、次のようになる。
\begin{align*}
- \frac{\partial_{i}f \circ \varphi_{i}}{\partial_{n + 1}f \circ \varphi_{i}}(\theta) = \frac{k}{h} = \frac{g\left( \mathbf{x} + h\mathbf{e}_{i} \right) - g\left( \mathbf{x} \right)}{h}
\end{align*}
ここで、上で述べられているようにその関数$g$はその開近傍$V$で連続であるから、次のようになり、
\begin{align*}
\lim_{h \rightarrow 0}k &= \lim_{h \rightarrow 0}\left( g\left( \mathbf{x} + h\mathbf{e}_{i} \right) - g\left( \mathbf{x} \right) \right)\\
&= \lim_{h \rightarrow 0}{g\left( \mathbf{x} + h\mathbf{e}_{i} \right)} - g\left( \mathbf{x} \right)\\
&= g\left( \lim_{h \rightarrow 0}\left( \mathbf{x} + h\mathbf{e}_{i} \right) \right) - g\left( \mathbf{x} \right)\\
&= g\left( \mathbf{x} \right) - g\left( \mathbf{x} \right) = 0
\end{align*}
したがって、その関数$f$が$C^{1}$級であることから、$\partial_{n + 1}f\begin{pmatrix}
\mathbf{a} \\
b \\
\end{pmatrix} > 0$に注意して、次のようになる。
\begin{align*}
\lim_{h \rightarrow 0}\left( - \frac{\partial_{i}f \circ \varphi_{i}}{\partial_{n + 1}f \circ \varphi_{i}}(\theta) \right) &= - \lim_{h \rightarrow 0}{\frac{\partial_{i}f}{\partial_{n + 1}f} \circ \varphi_{i}(\theta)}\\
&= - \frac{\partial_{i}f}{\partial_{n + 1}f}\left( \lim_{h \rightarrow 0}{\varphi_{i}(\theta)} \right)\\
&= - \frac{\partial_{i}f}{\partial_{n + 1}f}\begin{pmatrix}
\lim_{h \rightarrow 0}\left( \mathbf{x} + \theta h\mathbf{e}_{i} \right) \\
\lim_{h \rightarrow 0}\left( g\left( \mathbf{x} \right) + \theta k \right) \\
\end{pmatrix}\\
&= - \frac{\partial_{i}f}{\partial_{n + 1}f}\begin{pmatrix}
\mathbf{x} \\
g\left( \mathbf{x} \right) \\
\end{pmatrix}\\
&= - \frac{\partial_{i}f}{\partial_{n + 1}f}\left( g_{\downarrow}\left( \mathbf{x} \right) \right)\\
&= - \frac{\partial_{i}f}{\partial_{n + 1}f} \circ g_{\downarrow}\left( \mathbf{x} \right)
\end{align*}
一方で、次のようになるので、
\begin{align*}
\lim_{h \rightarrow 0}\frac{g\left( \mathbf{x} + h\mathbf{e}_{i} \right) - g\left( \mathbf{x} \right)}{h} = \partial_{i}g\left( \mathbf{x} \right)
\end{align*}
次式が得られる。
\begin{align*}
\partial_{i}g\left( \mathbf{x} \right) = \lim_{h \rightarrow 0}\frac{k}{h} = - \frac{\partial_{i}f}{\partial_{n + 1}f} \circ g_{\downarrow}\left( \mathbf{x} \right)
\end{align*}\par
ここで、その関数$g$が連続であるので、上の式の右辺も連続となり、したがって、偏導関数$\partial_{i}g$もその開近傍$V$上で連続である。ゆえに、その関数$g$は$C^{1}$級であるから、次のことに注意すれば、
\begin{align*}
- \frac{\partial_{n+1}f}{\partial_{n + 1}f} \circ g_{\downarrow}\left( \mathbf{x} \right) = - \frac{\partial_{n+1}f \circ g_{\downarrow}\left( \mathbf{x} \right) }{\partial_{n + 1}f \circ g_{\downarrow}\left( \mathbf{x} \right) } =-1
\end{align*}
\begin{comment}
\begin{align*}
\mathrm{grad}g\left( \mathbf{x} \right) &= \begin{pmatrix}
\partial_{1}g\left( \mathbf{x} \right) \\
\partial_{2}g\left( \mathbf{x} \right) \\
 \vdots \\
\partial_{n}g\left( \mathbf{x} \right) \\
\end{pmatrix} = \begin{pmatrix}
 - \frac{\partial_{1}f}{\partial_{n + 1}f} \circ g_{\downarrow}\left( \mathbf{x} \right) \\
 - \frac{\partial_{2}f}{\partial_{n + 1}f} \circ g_{\downarrow}\left( \mathbf{x} \right) \\
 \vdots \\
 - \frac{\partial_{n}f}{\partial_{n + 1}f} \circ g_{\downarrow}\left( \mathbf{x} \right) \\
\end{pmatrix}\\
&= - \left( \frac{1}{\partial_{n + 1}f}\begin{pmatrix}
\partial_{1}f \\
\partial_{2}f \\
 \vdots \\
\partial_{n}f \\
\end{pmatrix} \right) \circ g_{\downarrow}\left( \mathbf{x} \right)
\end{align*}
\end{comment}
よって、次式が得られる。
\begin{align*}
\begin{pmatrix}
\mathrm{grad}g\left( \mathbf{x} \right) \\
 - 1 \\
\end{pmatrix} = - \frac{\mathrm{grad}f}{\partial_{n + 1}f} \circ g_{\downarrow}\left( \mathbf{x} \right)
\end{align*}
\begin{comment}
\begin{align*}
\begin{pmatrix}
\mathrm{grad}g\left( \mathbf{x} \right) \\
 - 1 \\
\end{pmatrix} &= \begin{pmatrix}
 - \left( \frac{1}{\partial_{n + 1}f}\begin{pmatrix}
\partial_{1}f \\
\partial_{2}f \\
 \vdots \\
\partial_{n}f \\
\end{pmatrix} \right) \circ g_{\downarrow}\left( \mathbf{x} \right) \\
 - \frac{\partial_{n + 1}f}{\partial_{n + 1}f} \circ g_{\downarrow}\left( \mathbf{x} \right) \\
\end{pmatrix}\\
&= - \left( \frac{1}{\partial_{n + 1}f}\begin{pmatrix}
\partial_{1}f \\
\partial_{2}f \\
 \vdots \\
\partial_{n + 1}f \\
\end{pmatrix} \right) \circ g_{\downarrow}\left( \mathbf{x} \right)\\
&= - \frac{\mathrm{grad}f}{\partial_{n + 1}f} \circ g_{\downarrow}\left( \mathbf{x} \right)
\end{align*}
\end{comment}
\par
あとは明らかであろう。
\end{proof}
\begin{thm}[陰関数定理]\label{4.4.1.2}
$U \subseteq \mathbb{R}^{n} \times \mathbb{R}$なる開集合$U$を用いた$C^{r}$級関数$f:U \rightarrow \mathbb{R};\begin{pmatrix}
\mathbf{x} \\
y \\
\end{pmatrix} \mapsto f\begin{pmatrix}
\mathbf{x} \\
y \\
\end{pmatrix}$が、$\exists\begin{pmatrix}
\mathbf{a} \\
b \\
\end{pmatrix} \in \mathbb{R}^{n} \times \mathbb{R}$に対し、$f\begin{pmatrix}
\mathbf{a} \\
b \\
\end{pmatrix} = 0$かつ$\partial_{n + 1}f\begin{pmatrix}
\mathbf{a} \\
b \\
\end{pmatrix} \neq 0$が成り立つとき、定理\ref{4.4.1.1}の陰関数$g:V \rightarrow W$もその開近傍$V$上で$C^{r}$級である。\par
定理\ref{4.4.1.1}、定理\ref{4.4.1.2}をまとめて陰関数定理という。
\end{thm}
\begin{proof} 定理\ref{4.4.1.1}と数学的帰納法により直ちに示される。実際、$r = 1$のときでは、定理\ref{4.4.1.1}そのものであるから、$r = k$のとき、仮定が成り立つものとすれば、$r = k + 1$のとき、その関数$f$は$C^{k}$級であるから、その陰関数$g$はその開近傍$V$上で$C^{k}$級であり、定理\ref{4.4.1.1}の次の式から
\begin{align*}
\partial_{i}g = - \frac{\partial_{i}f}{\partial_{n + 1}f} \circ g_{\downarrow}
\end{align*}
その関数$\partial_{i}g$も$C^{k - 1}$級で、ゆえに、その陰関数$g$もその開近傍$V$上で$C^{k + 1}$級である。
\end{proof}
%\hypertarget{ux3088ux308aux3088ux3044ux9670ux95a2ux6570ux5b9aux7406}{%
\subsubsection{よりよい陰関数定理}%\label{ux3088ux308aux3088ux3044ux9670ux95a2ux6570ux5b9aux7406}}\par
以下、記法的な煩雑さをさけるために形式的な議論をしよう。単射な写像$p:\varLambda_{n} \rightarrow \varLambda_{m}$を用いて$\nabla = \left( \partial_{p(i)} \right)_{i \in \varLambda_{n} }$と与えられるvector$\nabla$をnablaとよび、これを用いて次のように定義されよう\footnote{例えば、$p:\varLambda_{3} \rightarrow \varLambda_{5}$として、$p(1) = 3,\ \ p(2) = 1,\ \ p(3) = 5$などが挙げられる。}\footnote{$\varLambda_{n}$から$\varLambda_{m}$への単射な写像が存在するという仮定より$n \leq m$が成り立っていることになる。}。
\begin{itemize}
\item
  $D(f) \subseteq \mathbb{R}^{m}$なる関数$f = \left( f_{i} \right)_{i \in \varLambda_{o}}:D(f) \rightarrow \mathbb{R}^{o}$に対し、次のようにする。
\begin{align*}
\nabla{}^{t}f = \begin{pmatrix}
\partial_{p(1)}f_{1} & \partial_{p(1)}f_{2} & \cdots & \partial_{p(1)}f_{o} \\
\partial_{p(2)}f_{1} & \partial_{p(2)}f_{2} & \cdots & \partial_{p(2)}f_{o} \\
 \vdots & \vdots & \ddots & \vdots \\
\partial_{p(m)}f_{1} & \partial_{p(m)}f_{2} & \cdots & \partial_{p(m)}f_{o} \\
\end{pmatrix}
\end{align*}
\item
  特に、$D(f) \subseteq \mathbb{R}^{m}$なる関数$f:D(f) \rightarrow \mathbb{R}$に対し、次のようにする。
\begin{align*}
\nabla f = \begin{pmatrix}
\partial_{p(1)}f \\
\partial_{p(2)}f \\
 \vdots \\
\partial_{p(m)}f \\
\end{pmatrix}
\end{align*}
\item
  $D(f) \subseteq \mathbb{R}^{m}$なる関数$f = \left( f_{i} \right)_{i \in \varLambda_{n}}:D(f) \rightarrow \mathbb{R}^{n}$に対し、次のようにする。
\begin{align*}
{}^{t}\nabla f = \partial_{p(1)}f_{1} + \partial_{p(2)}f_{2} + \cdots + \partial_{p(n)}f_{n}
\end{align*}
\end{itemize}\par
例えば、$\nabla = \left( \partial_{i} \right)_{i \in \varLambda_{m} }$としたとき、$D(f) \subseteq \mathbb{R}^{m}$なる関数$f = \left( f_{i} \right)_{i \in \varLambda_{n}}:D(f) \rightarrow \mathbb{R}^{n}$に対し、$J_{f} ={}^{t}\left( \nabla{}^{t}f \right)$が成り立つし、$D(f) \subseteq \mathbb{R}^{m}$なる関数$f:D(f) \rightarrow \mathbb{R}$に対し、$\mathrm{grad}f = \nabla f$が成り立つ。
\begin{thm}[よりよい陰関数定理]\label{4.4.1.3}
$U \subseteq \mathbb{R}^{m} \times \mathbb{R}^{n}$なる開集合$U$を用いた$C^{1}$級関数$f:U \rightarrow \mathbb{R}^{n};\begin{pmatrix}
\mathbf{x} \\
\mathbf{y} \\
\end{pmatrix} \mapsto f\begin{pmatrix}
\mathbf{x} \\
\mathbf{y} \\
\end{pmatrix}$が、$\exists\begin{pmatrix}
\mathbf{a} \\
\mathbf{b} \\
\end{pmatrix} \in \mathbb{R}^{m} \times \mathbb{R}^{n}$に対し、$f\begin{pmatrix}
\mathbf{a} \\
\mathbf{b} \\
\end{pmatrix} = \mathbf{0}$かつ$f = \left( f_{i} \right)_{i \in \varLambda_{n}}$、$\nabla^{*} = \left( \partial_{i} \right)_{i \in \varLambda_{m}}$、$\nabla_{*} = \left( \partial_{i} \right)_{i \in \varLambda_{m + n} \setminus \varLambda_{m}}$とおいて次式が成り立つとき\footnote{つまり次式が成り立つときである。
\begin{align*}
\left| \begin{matrix}
  \partial_{m + 1}f_{1} \begin{pmatrix} \mathbf{a} \\ \mathbf{b} \\ \end{pmatrix} & \partial_{m + 2}f_{1} \begin{pmatrix} \mathbf{a} \\ \mathbf{b} \\ \end{pmatrix} & \cdots & \partial_{m + n}f_{1} \begin{pmatrix} \mathbf{a} \\ \mathbf{b} \\ \end{pmatrix} \\
  \partial_{m + 1}f_{2} \begin{pmatrix} \mathbf{a} \\ \mathbf{b} \\ \end{pmatrix} & \partial_{m + 2}f_{2} \begin{pmatrix} \mathbf{a} \\ \mathbf{b} \\ \end{pmatrix} & \cdots & \partial_{m + n}f_{2} \begin{pmatrix} \mathbf{a} \\ \mathbf{b} \\ \end{pmatrix} \\
  \vdots & \vdots & \ddots & \vdots \\
  \partial_{m + 1}f_{n} \begin{pmatrix} \mathbf{a} \\ \mathbf{b} \\ \end{pmatrix} & \partial_{m + 2}f_{n} \begin{pmatrix} \mathbf{a} \\ \mathbf{b} \\ \end{pmatrix} & \cdots & \partial_{m + n}f_{n} \begin{pmatrix} \mathbf{a} \\ \mathbf{b} \\ \end{pmatrix} \\
\end{matrix} \right| 
\neq 0
\end{align*} }、
\begin{align*}
\det{{}^{t}\left( \nabla_{*}{}^{t}f \right)}\begin{pmatrix}
\mathbf{a} \\
\mathbf{b} \\
\end{pmatrix} \neq 0
\end{align*}
$V \times W \subseteq U$かつ$V \subseteq \mathbb{R}^{m}$かつ$W \subseteq \mathbb{R}^{n}$なるそれらの点々$\mathbf{a}$、$\mathbf{b}$の開近傍たちそれぞれ$V$、$W$と関数$g:V \rightarrow W$が存在して、$g_{\downarrow}:V \rightarrow \mathbb{R}^{n} \times W;\mathbf{x} \mapsto \begin{pmatrix}
\mathbf{x} \\
g\left( \mathbf{x} \right) \\
\end{pmatrix}$とすれば、次のことが成り立つ。
\begin{itemize}
\item
  その開近傍$V$上で次式が成り立つ\footnote{つまり、次式が成り立つ。
\begin{align*}
\left| 
  \begin{matrix}
    \partial_{m + 1}f_{1} \circ g_{\downarrow} & \partial_{m + 2}f_{1} \circ g_{\downarrow} & \cdots & \partial_{m + n}f_{1} \circ g_{\downarrow} \\
    \partial_{m + 1}f_{2} \circ g_{\downarrow} & \partial_{m + 2}f_{2} \circ g_{\downarrow} & \cdots & \partial_{m + n}f_{2} \circ g_{\downarrow} \\
    \vdots & \vdots & \ddots & \vdots \\
    \partial_{m + 1}f_{n} \circ g_{\downarrow} & \partial_{m + 2}f_{n} \circ g_{\downarrow} & \cdots & \partial_{m + n}f_{n} \circ g_{\downarrow} \\
  \end{matrix}
\right| \neq 0
\end{align*} }。
\begin{align*}
\left( \det{{}^{t}\left( \nabla_{*}{}^{t}f \right)} \circ g_{\downarrow}:V \rightarrow \mathbb{R} \right) \neq 0
\end{align*}
\item
  $g\left( \mathbf{a} \right) = \mathbf{b}$が成り立つ。
\item
  その関数$g$はその開近傍$V$で連続である。
\item
  $\forall\begin{pmatrix}
  \mathbf{x} \\
  \mathbf{y} \\
  \end{pmatrix} \in V \times W$に対し、$f\begin{pmatrix}
  \mathbf{x} \\
  \mathbf{y} \\
  \end{pmatrix} = \mathbf{0}$が成り立つならそのときに限り、$\mathbf{y} = g\left( \mathbf{x} \right)$が成り立つ。
\end{itemize}
さらに、その陰関数$g$について、次のことが成り立つ。
\begin{itemize}
\item
  その陰関数$g$はその開近傍$V$上で$C^{1}$級である。
\item
  その開近傍$V$上で次式が成り立つ\footnote{つまり、次式が成り立つ。
\begin{align*}
\begin{pmatrix}
  \partial_{1}g_{1} & \partial_{2}g_{1} & \cdots & \partial_{m}g_{1} \\
  \partial_{1}g_{2} & \partial_{2}g_{2} & \cdots & \partial_{m}g_{2} \\
  \vdots & \vdots & \ddots & \vdots \\
  \partial_{1}g_{n} & \partial_{2}g_{n} & \cdots & \partial_{m}g_{n} \\
\end{pmatrix} = - \begin{pmatrix}
  \partial_{m + 1}f_{1} \circ g_{\downarrow} & \partial_{m + 2}f_{1} \circ g_{\downarrow} & \cdots & \partial_{m + n}f_{1} \circ g_{\downarrow} \\
  \partial_{m + 1}f_{2} \circ g_{\downarrow} & \partial_{m + 2}f_{2} \circ g_{\downarrow} & \cdots & \partial_{m + n}f_{2} \circ g_{\downarrow} \\
  \vdots & \vdots & \ddots & \vdots \\
  \partial_{m + 1}f_{n} \circ g_{\downarrow} & \partial_{m + 2}f_{n} \circ g_{\downarrow} & \cdots & \partial_{m + n}f_{n} \circ g_{\downarrow} \\
\end{pmatrix}^{- 1} \begin{pmatrix}
  \partial_{1}f_{1} \circ g_{\downarrow} & \partial_{2}f_{1} \circ g_{\downarrow} & \cdots & \partial_{m}f_{1} \circ g_{\downarrow} \\
  \partial_{1}f_{2} \circ g_{\downarrow} & \partial_{2}f_{2} \circ g_{\downarrow} & \cdots & \partial_{m}f_{2} \circ g_{\downarrow} \\
  \vdots & \vdots & \ddots & \vdots \\
  \partial_{1}f_{n} \circ g_{\downarrow} & \partial_{2}f_{n} \circ g_{\downarrow} & \cdots & \partial_{m}f_{n} \circ g_{\downarrow} \\
\end{pmatrix}
\end{align*}}。
\begin{align*}
J_{g} = - \left({}^{t}\left( \nabla_{*}{}^{t}f \right)^{- 1}{}^{t}\left( \nabla^{*}{}^{t}f \right) \right) \circ g_{\downarrow}
\end{align*}
\end{itemize}
\end{thm}
\begin{proof}
$U \subseteq \mathbb{R}^{m} \times \mathbb{R}^{n}$なる開集合$U$を用いた$C^{1}$級関数$f:U \rightarrow \mathbb{R}^{n};\begin{pmatrix}
\mathbf{x} \\
\mathbf{y} \\
\end{pmatrix} \mapsto f\begin{pmatrix}
\mathbf{x} \\
\mathbf{y} \\
\end{pmatrix}$が、$\exists\begin{pmatrix}
\mathbf{a} \\
\mathbf{b} \\
\end{pmatrix} \in \mathbb{R}^{m} \times \mathbb{R}^{n}$に対し、$f\begin{pmatrix}
\mathbf{a} \\
\mathbf{b} \\
\end{pmatrix} = \mathbf{0}$かつ$f = \left( f_{i} \right)_{i \in \varLambda_{n}}$、$\nabla^{*} = \left( \partial_{i} \right)_{i \in \varLambda_{m}}$、$\nabla_{*} = \left( \partial_{i} \right)_{i \in \varLambda_{m + n} \setminus \varLambda_{m}}$とおいて次式が成り立つとき、
\begin{align*}
\det{{}^{t}\left( \nabla_{*}{}^{t}f \right)}\begin{pmatrix}
\mathbf{a} \\
\mathbf{b} \\
\end{pmatrix} = \left| \begin{matrix}
\partial_{m + 1}f_{1} & \partial_{m + 2}f_{1} & \cdots & \partial_{m + n}f_{1} \\
\partial_{m + 1}f_{2} & \partial_{m + 2}f_{2} & \cdots & \partial_{m + n}f_{2} \\
 \vdots & \vdots & \ddots & \vdots \\
\partial_{m + 1}f_{n} & \partial_{m + 2}f_{n} & \cdots & \partial_{m + n}f_{n} \\
\end{matrix} \right|\begin{pmatrix}
\mathbf{a} \\
\mathbf{b} \\
\end{pmatrix} \neq 0
\end{align*}
$n = 1$のときはまさしく陰関数定理そのものであるから、$n = k$のとき、示すべきことが成り立つと仮定すると、行列${}^{t}\left( \nabla_{*}{}^{t}f \right)\begin{pmatrix}
\mathbf{a} \\
\mathbf{b} \\
\end{pmatrix}$の第$n$行の成分はすべて$0$であることはないので、添数を付け替えて$\partial_{m + n}f_{n}\begin{pmatrix}
\mathbf{a} \\
\mathbf{b} \\
\end{pmatrix} \neq 0$が成り立つとしてもよい。ここで、陰関数定理より$\mathbf{b} = \left( b_{i} \right)_{i \in \varLambda_{n}}$、$\mathbf{b}^{*} = \left( b_{i} \right)_{i \in \varLambda_{n - 1}}$とおかれれば、$V_{1} \times W_{1} \subseteq U$かつ$V_{1} \subseteq \mathbb{R}^{m + n - 1}$かつ$W_{1} \subseteq \mathbb{R}$なるそれらの点々$\begin{pmatrix}
\mathbf{a} \\
\mathbf{b}^{*} \\
\end{pmatrix}$、$b_{n}$の開近傍たちそれぞれ$V_{1}$、$W_{1}$と関数$h:V_{1} \rightarrow W_{1}$が存在して、$\mathbf{y} = \begin{pmatrix}
\mathbf{y}^{*} \\
y_{n} \\
\end{pmatrix}$として、$h_{\downarrow}:V_{1} \rightarrow \mathbb{R}^{m + n - 1} \times W_{1};\begin{pmatrix}
\mathbf{x} \\
\mathbf{y}^{*} \\
\end{pmatrix} \mapsto \begin{pmatrix}
\begin{pmatrix}
\mathbf{x} \\
\mathbf{y}^{*} \\
\end{pmatrix} \\
h\begin{pmatrix}
\mathbf{x} \\
\mathbf{y}^{*} \\
\end{pmatrix} \\
\end{pmatrix}$とすれば、次のことが成り立つ。
\begin{itemize}
\item
  $h\begin{pmatrix}
  \mathbf{a} \\
  \mathbf{b}^{*} \\
  \end{pmatrix} = b_{n}$が成り立つ。
\item
  その関数$h$はその開近傍$V_{1}$で連続である。
\item
  $\forall\begin{pmatrix}
  \mathbf{x} \\
  \mathbf{y} \\
  \end{pmatrix} \in V_{1} \times W_{1}$に対し、$f_{n}\begin{pmatrix}
  \mathbf{x} \\
  \mathbf{y} \\
  \end{pmatrix} = 0$が成り立つならそのときに限り、$y_{n} = h\begin{pmatrix}
  \mathbf{x} \\
  \mathbf{y}^{*} \\
  \end{pmatrix}$が成り立つ。
\item
  その開近傍$V_{1}$上で、$\forall i \in \varLambda_{n - 1}$に対し、次式が成り立つ。
\begin{align*}
\partial_{m + i}h = - \frac{\partial_{m + i}f_{n}}{\partial_{m + n}f_{n}} \circ h_{\downarrow}
\end{align*}
\end{itemize}\par
ここで、$f^{*} = \left( f_{i} \right)_{i \in \varLambda_{n - 1}}$、$h_{\downarrow} = \left( h_{i}^{\downarrow} \right)_{i \in \varLambda_{m + n}}$とし関数$F = \left( F_{i} \right)_{i \in \varLambda_{n - 1}} = f^{*} \circ h_{\downarrow}:V_{1} \rightarrow \mathbb{R}^{n - 1}$が定義されると、その関数$f$とその陰関数$h$は$C_{1}$級であるから、その関数$F$も$C^{1}$級であり、$h\begin{pmatrix}
\mathbf{a} \\
\mathbf{b}^{*} \\
\end{pmatrix} = b_{n}$が成り立つので、次のようになる。
\begin{align*}
h_{\downarrow}\begin{pmatrix}
\mathbf{a} \\
\mathbf{b}^{*} \\
\end{pmatrix} &= \begin{pmatrix}
\mathbf{a} \\
\mathbf{b}^{*} \\
h\begin{pmatrix}
\mathbf{a} \\
\mathbf{b}^{*} \\
\end{pmatrix} \\
\end{pmatrix} = \begin{pmatrix}
\mathbf{a} \\
\mathbf{b}^{*} \\
b_{n} \\
\end{pmatrix} = \begin{pmatrix}
\mathbf{a} \\
\mathbf{b} \\
\end{pmatrix}\\
F\begin{pmatrix}
\mathbf{a} \\
\mathbf{b}^{*} \\
\end{pmatrix} &= f^{*} \circ h_{\downarrow}\begin{pmatrix}
\mathbf{a} \\
\mathbf{b}^{*} \\
\end{pmatrix} = f^{*}\begin{pmatrix}
\mathbf{a} \\
\mathbf{b} \\
\end{pmatrix} = 0\\
\partial_{m + i}h\begin{pmatrix}
\mathbf{x} \\
\mathbf{y}^{*} \\
\end{pmatrix} &= - \frac{\partial_{m + i}f_{n}}{\partial_{m + n}f_{n}}\begin{pmatrix}
\mathbf{x} \\
\mathbf{y}^{*} \\
h\begin{pmatrix}
\mathbf{x} \\
\mathbf{y}^{*} \\
\end{pmatrix} \\
\end{pmatrix} = - \frac{\partial_{m + i}f_{n}}{\partial_{m + n}f_{n}} \circ h_{\downarrow}\begin{pmatrix}
\mathbf{x} \\
\mathbf{y}^{*} \\
\end{pmatrix}
\end{align*}\par
さらに、関数$\det{{}^{t}\left( \nabla_{*}{}^{t}F \right)}:V_{1} \rightarrow \mathbb{R}$が次のように定義されると、
\begin{align*}
\det{{}^{t}\left( \nabla_{*}{}^{t}F \right)}:V_{1} \rightarrow \mathbb{R};\begin{pmatrix}
\mathbf{x} \\
\mathbf{y}^{*} \\
\end{pmatrix} \mapsto \left| \begin{matrix}
\partial_{m + 1}F_{1} \begin{pmatrix} \mathbf{x} \\ \mathbf{y}^{*} \\ \end{pmatrix} & \partial_{m + 2}F_{1} \begin{pmatrix} \mathbf{x} \\ \mathbf{y}^{*} \\ \end{pmatrix} & \cdots & \partial_{m + n - 1}F_{1} \begin{pmatrix} \mathbf{x} \\ \mathbf{y}^{*} \\ \end{pmatrix} \\
\partial_{m + 1}F_{2} \begin{pmatrix} \mathbf{x} \\ \mathbf{y}^{*} \\ \end{pmatrix} & \partial_{m + 2}F_{2} \begin{pmatrix} \mathbf{x} \\ \mathbf{y}^{*} \\ \end{pmatrix} & \cdots & \partial_{m + n - 1}F_{2} \begin{pmatrix} \mathbf{x} \\ \mathbf{y}^{*} \\ \end{pmatrix} \\
 \vdots & \vdots & \ddots & \vdots \\
\partial_{m + 1}F_{n - 1} \begin{pmatrix} \mathbf{x} \\ \mathbf{y}^{*} \\ \end{pmatrix} & \partial_{m + 2}F_{n - 1} \begin{pmatrix} \mathbf{x} \\ \mathbf{y}^{*} \\ \end{pmatrix} & \cdots & \partial_{m + n - 1}F_{n - 1} \begin{pmatrix} \mathbf{x} \\ \mathbf{y}^{*} \\ \end{pmatrix} \\
\end{matrix} \right|
\end{align*}
$\det{{}^{t}\left( \nabla_{*}{}^{t}F \right)}\begin{pmatrix}
\mathbf{a} \\
\mathbf{b}^{*} \\
\end{pmatrix} \neq 0$が成り立つ。実際、その関数$F$の定義より、次のようにおいて、$\forall(i,j) \in \varLambda_{n - 1}^{2}$に対し、連鎖律より次のようになるので\footnote{ここもEinstein縮約記法を用いれば、$k\in \varLambda_{m+n-1} $として$\forall i,j\in \varLambda_{n-1} $に対し、$\partial_{m+j} h_{k}^{\downarrow} =\delta_{m+j,k} $、$\partial_{m+n} h_{m+n}^{\downarrow} =\partial_{m+n} h$に注意すれば、次のようになる。
\begin{align*}
\partial_{m+j} F_{i} &= \partial_{m+j} \left( f_{i} \circ h_{\downarrow} \right) \\
&= \left( \partial_{k} f_{i} \circ h_{\downarrow} \right) \partial_{m+j} h_{k}^{\downarrow} + \left( \partial_{m+n} f_{i} \circ h_{\downarrow} \right) \partial_{m+j} h_{m+n}^{\downarrow} \\
&= \left( \partial_{k} f_{i} \circ h_{\downarrow} \right) \delta_{m+j,k} + \left( \partial_{m+n} f_{i} \circ h_{\downarrow} \right) \partial_{m+n} h \\
&= \partial_{m+j} f_{i} \circ h_{\downarrow} + \left( \partial_{m+n} f_{i} \circ h_{\downarrow} \right) \partial_{m+n} h
\end{align*} }、
\begin{align*}
\partial_{m + j}F_{i} &= \partial_{m + j}\left( f_{i} \circ h_{\downarrow} \right)\\
&= \sum_{k \in \varLambda_{m + n}} {\left( \partial_{k}f_{i} \circ h_{\downarrow} \right)\partial_{m + j}h_{k}^{\downarrow}}\\
&= \sum_{k \in \varLambda_{m + n} \setminus \left\{ m + j,m + n \right\}} {\left( \partial_{k}f_{i} \circ h_{\downarrow} \right)\partial_{m + j}h_{k}^{\downarrow}} \\
&\quad + \left( \partial_{m + j}f_{i} \circ h_{\downarrow} \right)\partial_{m + j}h_{m + j}^{\downarrow} + \left( \partial_{m + n}f_{i} \circ h_{\downarrow} \right)\partial_{m + j}h_{m + n}^{\downarrow}\\
&= \partial_{m + j}f_{i} \circ h_{\downarrow} + \left( \partial_{m + n}f_{i} \circ h_{\downarrow} \right)\partial_{m + j}h
\end{align*}
したがって、行列$A$の第$j$列vectorを$\mathbf{a}$と置き換えたものを$\mathfrak{s}_{j,\mathbf{a}}(A)$とおくことにすると、次のようになる。
\begin{align*}
\det{{}^{t}\left( \nabla_{*}{}^{t}F \right)} &= \left| \begin{matrix}
\partial_{m + 1}F_{1} & \partial_{m + 2}F_{1} & \cdots & \partial_{m + n - 1}F_{1} \\
\partial_{m + 1}F_{2} & \partial_{m + 2}F_{2} & \cdots & \partial_{m + n - 1}F_{2} \\
 \vdots & \vdots & \ddots & \vdots \\
\partial_{m + 1}F_{n - 1} & \partial_{m + 2}F_{n - 1} & \cdots & \partial_{m + n - 1}F_{n - 1} \\
\end{matrix} \right|\\
&= \left| \begin{matrix}
\partial_{m + 1}f_{1} \circ h_{\downarrow} + \left( \partial_{m + n}f_{1} \circ h_{\downarrow} \right)\partial_{m + 1}h \\
\partial_{m + 1}f_{2} \circ h_{\downarrow} + \left( \partial_{m + n}f_{2} \circ h_{\downarrow} \right)\partial_{m + 1}h \\
 \vdots \\
\partial_{m + 1}f_{n - 1} \circ h_{\downarrow} + \left( \partial_{m + n}f_{n - 1} \circ h_{\downarrow} \right)\partial_{m + 1}h \\
\end{matrix} \right. \\
&\quad \begin{matrix}
\partial_{m + 2}f_{1} \circ h_{\downarrow} + \left( \partial_{m + n}f_{1} \circ h_{\downarrow} \right)\partial_{m + 2}h \\
\partial_{m + 2}f_{2} \circ h_{\downarrow} + \left( \partial_{m + n}f_{2} \circ h_{\downarrow} \right)\partial_{m + 2}h \\
 \vdots \\
\partial_{m + 2}f_{n - 1} \circ h_{\downarrow} + \left( \partial_{m + n}f_{n - 1} \circ h_{\downarrow} \right)\partial_{m + 2}h \\
\end{matrix} \\
&\quad \left. \ \begin{matrix}
\cdots & \partial_{m + n - 1}f_{1} \circ h_{\downarrow} + \left( \partial_{m + n}f_{1} \circ h_{\downarrow} \right)\partial_{m + n - 1}h \\
\cdots & \partial_{m + n - 1}f_{2} \circ h_{\downarrow} + \left( \partial_{m + n}f_{2} \circ h_{\downarrow} \right)\partial_{m + n - 1}h \\
 \ddots & \vdots \\
\cdots & \partial_{m + n - 1}f_{n - 1} \circ h_{\downarrow} + \left( \partial_{m + n}f_{n - 1} \circ h_{\downarrow} \right)\partial_{m + n - 1}h \\
\end{matrix} \right|\\
&= \left| \begin{matrix}
\partial_{m + 1}f_{1} \circ h_{\downarrow} & \partial_{m + 2}f_{1} \circ h_{\downarrow} & \cdots & \partial_{m + n - 1}f_{1} \circ h_{\downarrow} \\
\partial_{m + 1}f_{2} \circ h_{\downarrow} & \partial_{m + 2}f_{2} \circ h_{\downarrow} & \cdots & \partial_{m + n - 1}f_{2} \circ h_{\downarrow} \\
 \vdots & \vdots & \ddots & \vdots \\
\partial_{m + 1}f_{n - 1} \circ h_{\downarrow} & \partial_{m + 2}f_{n - 1} \circ h_{\downarrow} & \cdots & \partial_{m + n - 1}f_{n - 1} \circ h_{\downarrow} \\
\end{matrix} \right| \\
&\quad + \partial_{m + 1}h\left| \begin{matrix}
\partial_{m + n}f_{1} \circ h_{\downarrow} & \partial_{m + 2}f_{1} \circ h_{\downarrow} & \cdots & \partial_{m + n - 1}f_{1} \circ h_{\downarrow} \\
\partial_{m + n}f_{2} \circ h_{\downarrow} & \partial_{m + 2}f_{2} \circ h_{\downarrow} & \cdots & \partial_{m + n - 1}f_{2} \circ h_{\downarrow} \\
 \vdots & \vdots & \ddots & \vdots \\
\partial_{m + n}f_{n - 1} \circ h_{\downarrow} & \partial_{m + 2}f_{n - 1} \circ h_{\downarrow} & \cdots & \partial_{m + n - 1}f_{n - 1} \circ h_{\downarrow} \\
\end{matrix} \right| \\
&\quad + \partial_{m + 2}h\left| \begin{matrix}
\partial_{m + 1}f_{1} \circ h_{\downarrow} & \partial_{m + n}f_{1} \circ h_{\downarrow} & \cdots & \partial_{m + n - 1}f_{1} \circ h_{\downarrow} \\
\partial_{m + 1}f_{2} \circ h_{\downarrow} & \partial_{m + n}f_{2} \circ h_{\downarrow} & \cdots & \partial_{m + n - 1}f_{2} \circ h_{\downarrow} \\
 \vdots & \vdots & \ddots & \vdots \\
\partial_{m + 1}f_{n - 1} \circ h_{\downarrow} & \partial_{m + n}f_{n - 1} \circ h_{\downarrow} & \cdots & \partial_{m + n - 1}f_{n - 1} \circ h_{\downarrow} \\
\end{matrix} \right| \\
&\quad + \cdots + \partial_{m + n - 1}h\left| \begin{matrix}
\partial_{m + 1}f_{1} \circ h_{\downarrow} & \partial_{m + 2}f_{1} \circ h_{\downarrow} & \cdots & \partial_{m + n}f_{1} \circ h_{\downarrow} \\
\partial_{m + 1}f_{2} \circ h_{\downarrow} & \partial_{m + 2}f_{2} \circ h_{\downarrow} & \cdots & \partial_{m + n}f_{2} \circ h_{\downarrow} \\
 \vdots & \vdots & \ddots & \vdots \\
\partial_{m + 1}f_{n - 1} \circ h_{\downarrow} & \partial_{m + 2}f_{n - 1} \circ h_{\downarrow} & \cdots & \partial_{m + n}f_{n - 1} \circ h_{\downarrow} \\
\end{matrix} \right| \\
&\quad + \partial_{m + 1}h\partial_{m + 2}h\cdots\partial_{m + n - 1}h\left| \begin{matrix}
\partial_{m + n}f_{1} \circ h_{\downarrow} & \partial_{m + n}f_{1} \circ h_{\downarrow} & \cdots & \partial_{m + n}f_{1} \circ h_{\downarrow} \\
\partial_{m + n}f_{2} \circ h_{\downarrow} & \partial_{m + n}f_{2} \circ h_{\downarrow} & \cdots & \partial_{m + n}f_{2} \circ h_{\downarrow} \\
 \vdots & \vdots & \ddots & \vdots \\
\partial_{m + n}f_{n - 1} \circ h_{\downarrow} & \partial_{m + n}f_{n - 1} \circ h_{\downarrow} & \cdots & \partial_{m + n}f_{n - 1} \circ h_{\downarrow} \\
\end{matrix} \right|\\
&= \left| \begin{matrix}
\partial_{m + 1}f_{1} \circ h_{\downarrow} & \partial_{m + 2}f_{1} \circ h_{\downarrow} & \cdots & \partial_{m + n - 1}f_{1} \circ h_{\downarrow} \\
\partial_{m + 1}f_{2} \circ h_{\downarrow} & \partial_{m + 2}f_{2} \circ h_{\downarrow} & \cdots & \partial_{m + n - 1}f_{2} \circ h_{\downarrow} \\
 \vdots & \vdots & \ddots & \vdots \\
\partial_{m + 1}f_{n - 1} \circ h_{\downarrow} & \partial_{m + 2}f_{n - 1} \circ h_{\downarrow} & \cdots & \partial_{m + n - 1}f_{n - 1} \circ h_{\downarrow} \\
\end{matrix} \right| \\
&\quad + \sum_{j \in \varLambda_{n - 1}} {\partial_{m + j}h\left| \begin{matrix}
\partial_{m + 1}f_{1} \circ h_{\downarrow} & \cdots & \partial_{m + n}f_{1} \circ h_{\downarrow} & \cdots & \partial_{m + n - 1}f_{1} \circ h_{\downarrow} \\
\partial_{m + 1}f_{2} \circ h_{\downarrow} & \cdots & \partial_{m + n}f_{2} \circ h_{\downarrow} & \cdots & \partial_{m + n - 1}f_{2} \circ h_{\downarrow} \\
 \vdots & \ddots & \vdots & \ddots & \vdots \\
\partial_{m + 1}f_{n - 1} \circ h_{\downarrow} & \cdots & \partial_{m + n}f_{n - 1} \circ h_{\downarrow} & \cdots & \partial_{m + n - 1}f_{n - 1} \circ h_{\downarrow} \\
\end{matrix} \right|}
\end{align*}
ここで、上により次式が成り立つので、
\begin{align*}
\partial_{m + i}h = - \frac{\partial_{m + i}f_{n}}{\partial_{m + n}f_{n}} \circ h_{\downarrow}
\end{align*}
余因子展開に気を付ければ、次のようになり、
\begin{align*}
\det{{}^{t}\left( \nabla_{*}{}^{t}F \right)} &=
\left| 
  \begin{matrix}
    \partial_{m + 1}f_{1} \circ h_{\downarrow} & \partial_{m + 2}f_{1} \circ h_{\downarrow} & \cdots & \partial_{m + n - 1}f_{1} \circ h_{\downarrow} \\
    \partial_{m + 1}f_{2} \circ h_{\downarrow} & \partial_{m + 2}f_{2} \circ h_{\downarrow} & \cdots & \partial_{m + n - 1}f_{2} \circ h_{\downarrow} \\
    \vdots & \vdots & \ddots & \vdots \\
    \partial_{m + 1}f_{n - 1} \circ u & \partial_{m + 2}f_{n - 1} \circ h_{\downarrow} & \cdots & \partial_{m + n - 1}f_{n - 1} \circ h_{\downarrow} \\
  \end{matrix} 
\right| \\
&\quad - \sum_{j \in \varLambda_{n - 1}} \left( \frac{\partial_{m + j}f_{n}}{\partial_{m + n}f_{n}} \circ h_{\downarrow} \right)
\left| 
  \begin{matrix}
    \partial_{m + 1}f_{1} \circ h_{\downarrow} \\
    \partial_{m + 1}f_{2} \circ h_{\downarrow} \\
    \vdots \\
    \partial_{m + 1}f_{n - 1} \circ h_{\downarrow} \\
  \end{matrix} 
\right. \\
&\quad \left. \begin{matrix}
  \cdots & \partial_{m + n}f_{1} \circ h_{\downarrow} & \cdots & \partial_{m + n - 1}f_{1} \circ h_{\downarrow} \\
  \cdots & \partial_{m + n}f_{2} \circ h_{\downarrow} & \cdots & \partial_{m + n - 1}f_{2} \circ h_{\downarrow} \\
  \ddots & \vdots & \ddots & \vdots \\
  \cdots & \partial_{m + n}f_{n - 1} \circ h_{\downarrow} & \cdots & \partial_{m + n - 1}f_{n - 1} \circ h_{\downarrow} \\
  \end{matrix} 
\right| \\
&=\frac{1}{\partial_{m+n} f_{n} \circ h_{\downarrow} } 
\left( \partial_{m+n} f_{n} \circ h_{\downarrow} 
  \left| 
    \begin{matrix}
      \partial_{m + 1}f_{1} \circ h_{\downarrow} & \partial_{m + 2}f_{1} \circ h_{\downarrow} & \cdots & \partial_{m + n - 1}f_{1} \circ h_{\downarrow} \\
      \partial_{m + 1}f_{2} \circ h_{\downarrow} & \partial_{m + 2}f_{2} \circ h_{\downarrow} & \cdots & \partial_{m + n - 1}f_{2} \circ h_{\downarrow} \\
      \vdots & \vdots & \ddots & \vdots \\
      \partial_{m + 1}f_{n - 1} \circ u & \partial_{m + 2}f_{n - 1} \circ h_{\downarrow} & \cdots & \partial_{m + n - 1}f_{n - 1} \circ h_{\downarrow} \\
    \end{matrix} 
  \right| 
\right. \\
&\quad - \sum_{j \in \varLambda_{n - 1}}  {\partial_{m + j}f_{n}} \circ h_{\downarrow} 
\left| 
  \begin{matrix}
    \partial_{m + 1}f_{1} \circ h_{\downarrow} \\
    \partial_{m + 1}f_{2} \circ h_{\downarrow} \\
    \vdots \\
    \partial_{m + 1}f_{n - 1} \circ h_{\downarrow} \\
  \end{matrix} 
\right. \\
&\quad 
\left. 
  \left. 
    \begin{matrix}
      \cdots & \partial_{m + n}f_{1} \circ h_{\downarrow} & \cdots & \partial_{m + n - 1}f_{1} \circ h_{\downarrow} \\
      \cdots & \partial_{m + n}f_{2} \circ h_{\downarrow} & \cdots & \partial_{m + n - 1}f_{2} \circ h_{\downarrow} \\
      \ddots & \vdots & \ddots & \vdots \\
      \cdots & \partial_{m + n}f_{n - 1} \circ h_{\downarrow} & \cdots & \partial_{m + n - 1}f_{n - 1} \circ h_{\downarrow} \\
    \end{matrix} 
  \right| 
\right) \\
&=\frac{1}{\partial_{m+n} f_{n} \circ h_{\downarrow} } 
\left( \partial_{m+n} f_{n} \circ h_{\downarrow} 
  \left| 
    \begin{matrix}
      \partial_{m + 1}f_{1} \circ h_{\downarrow} & \partial_{m + 2}f_{1} \circ h_{\downarrow} & \cdots & \partial_{m + n - 1}f_{1} \circ h_{\downarrow} \\
      \partial_{m + 1}f_{2} \circ h_{\downarrow} & \partial_{m + 2}f_{2} \circ h_{\downarrow} & \cdots & \partial_{m + n - 1}f_{2} \circ h_{\downarrow} \\
      \vdots & \vdots & \ddots & \vdots \\
      \partial_{m + 1}f_{n - 1} \circ u & \partial_{m + 2}f_{n - 1} \circ h_{\downarrow} & \cdots & \partial_{m + n - 1}f_{n - 1} \circ h_{\downarrow} \\
    \end{matrix} 
  \right| 
\right. \\
&\quad + \sum_{j \in \varLambda_{n - 1}}  \left( - 1 \right)^{n + j} \partial_{m + j}f_{n} \circ h_{\downarrow} 
\left| 
  \begin{matrix}
    \partial_{m + 1}f_{1} \circ h_{\downarrow} & \cdots & \partial_{m + j - 1}f_{1} \circ h_{\downarrow} \\
    \partial_{m + 1}f_{2} \circ h_{\downarrow} & \cdots & \partial_{m + j - 1}f_{2} \circ h_{\downarrow} \\
    \vdots & \ddots & \vdots \\
    \partial_{m + 1}f_{n - 1} \circ h_{\downarrow} & \cdots & \partial_{m + j - 1}f_{n - 1} \circ h_{\downarrow} \\
  \end{matrix} 
\right. \\
&\quad 
\left. 
  \left. 
    \begin{matrix}
      \partial_{m + j + 1}f_{1} \circ h_{\downarrow} & \cdots & \partial_{m + n}f_{1} \circ h_{\downarrow} \\
      \partial_{m + j + 1}f_{2} \circ h_{\downarrow} & \cdots & \partial_{m + n}f_{2} \circ h_{\downarrow} \\
      \vdots & \ddots & \vdots \\
      \partial_{m + j + 1}f_{n - 1} \circ h_{\downarrow} & \cdots & \partial_{m + n}f_{n - 1} \circ h_{\downarrow} \\
    \end{matrix} 
  \right| 
\right) \\
&=\frac{1}{\partial_{m+n} f_{n} \circ h_{\downarrow} } 
\left( 
  \sum_{j \in \varLambda_{n}}  \left( - 1 \right)^{n + j} \partial_{m + j}f_{n} \circ h_{\downarrow} 
  \left| 
    \begin{matrix}
      \partial_{m + 1}f_{1} \circ h_{\downarrow} & \cdots & \partial_{m + j - 1}f_{1} \circ h_{\downarrow} \\
      \partial_{m + 1}f_{2} \circ h_{\downarrow} & \cdots & \partial_{m + j - 1}f_{2} \circ h_  {\downarrow} \\
      \vdots & \ddots & \vdots \\
      \partial_{m + 1}f_{n - 1} \circ h_{\downarrow} & \cdots & \partial_{m + j - 1}f_{n - 1} \circ h_{\downarrow} \\
    \end{matrix} 
  \right.
\right. \\
&\quad 
\left. 
  \left. 
    \begin{matrix}
      \partial_{m + j + 1}f_{1} \circ h_{\downarrow} & \cdots & \partial_{m + n}f_{1} \circ h_{\downarrow} \\
      \partial_{m + j + 1}f_{2} \circ h_{\downarrow} & \cdots & \partial_{m + n}f_{2} \circ h_{\downarrow} \\
      \vdots & \ddots & \vdots \\
      \partial_{m + j + 1}f_{n - 1} \circ h_{\downarrow} & \cdots & \partial_{m + n}f_{n - 1} \circ h_{\downarrow} \\
    \end{matrix} 
  \right| 
\right) \\
&= \frac{1}{\partial_{m+n} f_{n} \circ h_{\downarrow} } 
\left| 
  \begin{matrix}
    \partial_{m + 1}f_{1} \circ h_{\downarrow} & \partial_{m + 2}f_{1} \circ h_{\downarrow} & \cdots & \partial_{m + n}f_{1} \circ h_{\downarrow} \\
    \partial_{m + 1}f_{2} \circ h_{\downarrow} & \partial_{m + 2}f_{2} \circ h_{\downarrow} & \cdots & \partial_{m + n}f_{2} \circ h_{\downarrow} \\
    \vdots & \vdots & \ddots & \vdots \\
    \partial_{m + 1}f_{n} \circ h_{\downarrow} & \partial_{m + 2}f_{n} \circ h_{\downarrow} & \cdots & \partial_{m + n}f_{n} \circ h_{\downarrow} \\
  \end{matrix} 
\right| 
\end{align*}
したがって、$h_{\downarrow}\begin{pmatrix} \mathbf{a} \\ \mathbf{b}^{*} \\ \end{pmatrix} = \begin{pmatrix} \mathbf{a} \\ \mathbf{b} \\ \end{pmatrix}$より次のようになる。
\begin{align*}
\det{{}^{t}\left( \nabla_{*}{}^{t}F \right)} \begin{pmatrix} \mathbf{a} \\ \mathbf{b}^{*} \\ \end{pmatrix} &= \frac{1}{\partial_{m+n} f_{n} \circ h_{\downarrow} \begin{pmatrix} \mathbf{a} \\ \mathbf{b}^{*} \\ \end{pmatrix} }
\left| 
  \begin{matrix}
    \partial_{m + 1}f_{1} \circ h_{\downarrow} \begin{pmatrix} \mathbf{a} \\ \mathbf{b}^{*} \\ \end{pmatrix} & \partial_{m + 2}f_{1} \circ h_{\downarrow} \begin{pmatrix} \mathbf{a} \\ \mathbf{b}^{*} \\ \end{pmatrix} & \cdots & \partial_{m + n}f_{1} \circ h_{\downarrow} \begin{pmatrix} \mathbf{a} \\ \mathbf{b}^{*} \\ \end{pmatrix} \\
    \partial_{m + 1}f_{2} \circ h_{\downarrow} \begin{pmatrix} \mathbf{a} \\ \mathbf{b}^{*} \\ \end{pmatrix} & \partial_{m + 2}f_{2} \circ h_{\downarrow} \begin{pmatrix} \mathbf{a} \\ \mathbf{b}^{*} \\ \end{pmatrix} & \cdots & \partial_{m + n}f_{2} \circ h_{\downarrow} \begin{pmatrix} \mathbf{a} \\ \mathbf{b}^{*} \\ \end{pmatrix} \\
    \vdots & \vdots & \ddots & \vdots \\
    \partial_{m + 1}f_{n} \circ h_{\downarrow} \begin{pmatrix} \mathbf{a} \\ \mathbf{b}^{*} \\ \end{pmatrix} & \partial_{m + 2}f_{n} \circ h_{\downarrow} \begin{pmatrix} \mathbf{a} \\ \mathbf{b}^{*} \\ \end{pmatrix} & \cdots & \partial_{m + n}f_{n} \circ h_{\downarrow} \begin{pmatrix} \mathbf{a} \\ \mathbf{b}^{*} \\ \end{pmatrix} \\
  \end{matrix} 
\right| \\
&= \frac{1}{\partial_{m+n} f_{n} \begin{pmatrix} \mathbf{a} \\ \mathbf{b} \\ \end{pmatrix} }
\left| 
  \begin{matrix}
    \partial_{m + 1}f_{1} \begin{pmatrix} \mathbf{a} \\ \mathbf{b} \\ \end{pmatrix} & \partial_{m + 2}f_{1} \begin{pmatrix} \mathbf{a} \\ \mathbf{b} \\ \end{pmatrix} & \cdots & \partial_{m + n}f_{1} \begin{pmatrix} \mathbf{a} \\ \mathbf{b} \\ \end{pmatrix} \\
    \partial_{m + 1}f_{2} \begin{pmatrix} \mathbf{a} \\ \mathbf{b} \\ \end{pmatrix} & \partial_{m + 2}f_{2} \begin{pmatrix} \mathbf{a} \\ \mathbf{b} \\ \end{pmatrix} & \cdots & \partial_{m + n}f_{2} \begin{pmatrix} \mathbf{a} \\ \mathbf{b} \\ \end{pmatrix} \\
    \vdots & \vdots & \ddots & \vdots \\
    \partial_{m + 1}f_{n} \begin{pmatrix} \mathbf{a} \\ \mathbf{b} \\ \end{pmatrix} & \partial_{m + 2}f_{n} \begin{pmatrix} \mathbf{a} \\ \mathbf{b} \\ \end{pmatrix} & \cdots & \partial_{m + n}f_{n} \begin{pmatrix} \mathbf{a} \\ \mathbf{b} \\ \end{pmatrix} \\
  \end{matrix} 
\right| 
\end{align*}
\begin{comment}
\begin{align*}
\det{{}^{t}\left( \nabla_{*}{}^{t}F \right)}\begin{pmatrix}
\mathbf{a} \\
\mathbf{b}^{*} \\
\end{pmatrix} &= 
\left|
  \begin{matrix}
    \partial_{m + 1}f_{1} \circ h_{\downarrow} & \partial_{m + 2}f_{1} \circ h_{\downarrow} & \cdots & \partial_{m + n - 1}f_{1} \circ h_{\downarrow} \\
    \partial_{m + 1}f_{2} \circ h_{\downarrow} & \partial_{m + 2}f_{2} \circ h_{\downarrow} & \cdots & \partial_{m + n - 1}f_{2} \circ h_{\downarrow} \\
    \vdots & \vdots & \ddots & \vdots \\
    \partial_{m + 1}f_{n - 1} \circ h_{\downarrow} & \partial_{m + 2}f_{n - 1} \circ h_{\downarrow} & \cdots & \partial_{m + n - 1}f_{n - 1} \circ h_{\downarrow} \\
  \end{matrix} 
\right| \begin{pmatrix} \mathbf{a} \\ \mathbf{b}^{*} \\ \end{pmatrix} \\
&\quad - \sum_{j \in \varLambda_{n - 1}} \frac{\partial_{m + j}f_{n}}{\partial_{m + n}f_{n}} \circ h_{\downarrow} \begin{pmatrix} \mathbf{a} \\ \mathbf{b}^{*} \\ \end{pmatrix}
\left|
  \begin{matrix}
    \partial_{m + 1}f_{1} \circ h_{\downarrow} \\
    \partial_{m + 1}f_{2} \circ h_{\downarrow} \\
    \vdots \\
    \partial_{m + 1}f_{n - 1} \circ h_{\downarrow} \\
  \end{matrix} 
\right. \\
&\quad \left.
  \begin{matrix}
    \cdots & \partial_{m + n}f_{1} \circ h_{\downarrow} & \cdots & \partial_{m + n - 1}f_{1} \circ h_{\downarrow} \\
    \cdots & \partial_{m + n}f_{2} \circ h_{\downarrow} & \cdots & \partial_{m + n - 1}f_{2} \circ h_{\downarrow} \\
    \ddots & \vdots & \ddots & \vdots \\
    \cdots & \partial_{m + n}f_{n - 1} \circ h_{\downarrow} & \cdots & \partial_{m + n - 1}f_{n - 1} \circ h_{\downarrow} \\
  \end{matrix} 
\right| \begin{pmatrix} \mathbf{a} \\ \mathbf{b}^{*} \\ \end{pmatrix}\\
&= 
\left| 
  \begin{matrix}
    \partial_{m + 1}f_{1} & \partial_{m + 2}f_{1} & \cdots & \partial_{m + n - 1}f_{1} \\
    \partial_{m + 1}f_{2} & \partial_{m + 2}f_{2} & \cdots & \partial_{m + n - 1}f_{2} \\
    \vdots & \vdots & \ddots & \vdots \\
    \partial_{m + 1}f_{n - 1} & \partial_{m + 2}f_{n - 1} & \cdots & \partial_{m + n - 1}f_{n - 1} \\
  \end{matrix}
\right|\begin{pmatrix} \mathbf{a} \\ \mathbf{b} \\ \end{pmatrix} \\
&\quad - \sum_{j \in \varLambda_{n - 1}} {\frac{\partial_{m + j}f_{n}}{\partial_{m + n}f_{n}}\begin{pmatrix} \mathbf{a} \\ \mathbf{b} \\ \end{pmatrix}
\left|
  \begin{matrix}
    \partial_{m + 1}f_{1} & \cdots & \partial_{m + n}f_{1} & \cdots & \partial_{m + n - 1}f_{1} \\
    \partial_{m + 1}f_{2} & \cdots & \partial_{m + n}f_{2} & \cdots & \partial_{m + n - 1}f_{2} \\
    \vdots & \ddots & \vdots & \ddots & \vdots \\
    \partial_{m + 1}f_{n - 1} & \cdots & \partial_{m + n}f_{n - 1} & \cdots & \partial_{m + n - 1}f_{n - 1} \\
  \end{matrix}
  \right|\begin{pmatrix} \mathbf{a} \\ \mathbf{b} \\ \end{pmatrix}} \\
&= \frac{1}{\partial_{m + n}f_{n}}
\left( \partial_{m + n}f_{n}
  \left|
    \begin{matrix}
      \partial_{m + 1}f_{1} & \partial_{m + 2}f_{1} & \cdots & \partial_{m + n - 1}f_{1} \\
        \partial_{m + 1}f_{2} & \partial_{m + 2}f_{2} & \cdots & \partial_{m + n - 1}f_{2} \\
      \vdots & \vdots & \ddots & \vdots \\
      \partial_{m + 1}f_{n - 1} & \partial_{m + 2}f_{n - 1} & \cdots & \partial_{m + n - 1}f_{n - 1} \\
   \end{matrix}
    \right|
\right. \\
&\quad - \left.
  \sum_{j \in \varLambda_{n - 1}} {\partial_{m + j}f_{n}
  \left|
    \begin{matrix}
      \partial_{m + 1}f_{1} & \cdots & \partial_{m + n}f_{1} & \cdots & \partial_{m + n - 1}f_{1} \\
      \partial_{m + 1}f_{2} & \cdots & \partial_{m + n}f_{2} & \cdots & \partial_{m + n - 1}f_{2} \\
      \vdots & \ddots & \vdots & \ddots & \vdots \\
      \partial_{m + 1}f_{n - 1} & \cdots & \partial_{m + n}f_{n - 1} & \cdots & \partial_{m + n - 1}f_{n - 1} \\
    \end{matrix}
  \right|}
\right)\begin{pmatrix} \mathbf{a} \\ \mathbf{b} \\ \end{pmatrix} \\ 
&= \frac{1}{\partial_{m + n}f_{n}}
\left(
  \partial_{m + n}f_{n}
  \left|
    \begin{matrix}
      \partial_{m + 1}f_{1} & \partial_{m + 2}f_{1} & \cdots & \partial_{m + n - 1}f_{1} \\
      \partial_{m + 1}f_{2} & \partial_{m + 2}f_{2} & \cdots & \partial_{m + n - 1}f_{2} \\
      \vdots & \vdots & \ddots & \vdots \\
      \partial_{m + 1}f_{n - 1} & \partial_{m + 2}f_{n - 1} & \cdots & \partial_{m + n - 1}f_{n - 1} \\
    \end{matrix}
  \right| 
\right. \\
&\quad + \sum_{j \in \varLambda_{n - 1}} ( - 1)^{n + j}\partial_{m + j}f_{n}
\left| 
  \begin{matrix}
    \partial_{m + 1}f_{1} & \cdots & \partial_{m + j - 1}f_{1} \\
    \partial_{m + 1}f_{2} & \cdots & \partial_{m + j - 1}f_{2} \\
    \vdots & \ddots & \vdots \\
    \partial_{m + 1}f_{n - 1} & \cdots & \partial_{m + j - 1}f_{n - 1} \\
  \end{matrix}
\right. \\
&\quad 
\left.
  \left. 
    \begin{matrix}
      \partial_{m + j + 1}f_{1} & \cdots & \partial_{m + n}f_{1} \\
      \partial_{m + j + 1}f_{2} & \cdots & \partial_{m + n}f_{2} \\
      \vdots & \ddots & \vdots \\
      \partial_{m + j + 1}f_{n - 1} & \cdots & \partial_{m + n}f_{n - 1} \\
    \end{matrix}
  \right|
\right) \begin{pmatrix} \mathbf{a} \\ \mathbf{b} \\ \end{pmatrix} \\
&= \frac{1}{\partial_{m + n}f_{n}}\sum_{j \in \varLambda_{n}} ( - 1)^{n + j}\partial_{m + j}f_{n}
\left|
  \begin{matrix}
    \partial_{m + 1}f_{1} & \cdots & \partial_{m + j - 1}f_{1} \\
    \partial_{m + 1}f_{2} & \cdots & \partial_{m + j - 1}f_{2} \\
    \vdots & \ddots & \vdots \\
    \partial_{m + 1}f_{n - 1} & \cdots & \partial_{m + j - 1}f_{n - 1} \\
  \end{matrix} 
\right.\\
&\quad \left. 
  \begin{matrix}
    \partial_{m + j + 1}f_{1} & \cdots & \partial_{m + n}f_{1} \\
    \partial_{m + j + 1}f_{2} & \cdots & \partial_{m + n}f_{2} \\
    \vdots & \ddots & \vdots \\
    \partial_{m + j + 1}f_{n - 1} & \cdots & \partial_{m + n}f_{n - 1} \\
  \end{matrix} 
\right| \begin{pmatrix} \mathbf{a} \\ \mathbf{b} \\ \end{pmatrix}\\
&= \frac{1}{\partial_{m + n}f_{n}}
\left| 
  \begin{matrix}
    \partial_{m + 1}f_{1} & \partial_{m + 2}f_{1} & \cdots & \partial_{m + n}f_{1} \\
    \partial_{m + 1}f_{2} & \partial_{m + 2}f_{2} & \cdots & \partial_{m + n}f_{2} \\
    \vdots & \vdots & \ddots & \vdots \\
    \partial_{m + 1}f_{n} & \partial_{m + 2}f_{n} & \cdots & \partial_{m + n}f_{n} \\
  \end{matrix} 
\right|\begin{pmatrix} \mathbf{a} \\ \mathbf{b} \\ \end{pmatrix}
\end{align*}
\end{comment}
ここで、仮定より$\det{{}^{t}\left( \nabla_{*}{}^{t}F \right)}\begin{pmatrix}
\mathbf{a} \\
\mathbf{b}^{*} \\
\end{pmatrix} \neq 0$が得られる。\par
これにより、数学的帰納法の仮定より、$\exists\begin{pmatrix}
\mathbf{a} \\
\mathbf{b}^{*} \\
\end{pmatrix} \in \mathbb{R}^{m} \times \mathbb{R}^{n - 1}$に対し、$F\begin{pmatrix}
\mathbf{a} \\
\mathbf{b}^{*} \\
\end{pmatrix} = 0$かつ$\det{{}^{t}\left( \nabla_{*}{}^{t}F \right)}\begin{pmatrix}
\mathbf{a} \\
\mathbf{b}^{*} \\
\end{pmatrix} \neq 0$が成り立つので、$V \times W_{0} \subseteq V_{1}$かつ$V \subseteq \mathbb{R}^{m}$かつ$W_{0} \subseteq \mathbb{R}^{n - 1}$なるそれらの点々$\mathbf{a}$、$\mathbf{b}^{*}$の開近傍たちそれぞれ$V$、$W_{0}$と関数$k:V \rightarrow W_{0}$が存在して、次のことが成り立つ。
\begin{itemize}
\item
  $k\left( \mathbf{a} \right) = \mathbf{b}^{*}$が成り立つ。
\item
  その関数$k$はその開近傍$V$で連続である。
\item
  $\forall\begin{pmatrix}
  \mathbf{x} \\
  \mathbf{y}^{*} \\
  \end{pmatrix} \in V \times W_{0}$に対し、$F\begin{pmatrix}
  \mathbf{x} \\
  \mathbf{y}^{*} \\
  \end{pmatrix} = 0$が成り立つならそのときに限り、$\mathbf{y}^{*} = k\left( \mathbf{x} \right)$が成り立つ。
\end{itemize}\par
このとき、$W = W_{0} \times W_{1}$とおけば、その集合$W$はその元$\mathbf{b}$の開近傍である。さらに、次式のように関数$g$が定義されれば、
\begin{align*}
g = \left( g_{i} \right)_{i \in \varLambda_{n}}:V \rightarrow W;\mathbf{x} \mapsto \begin{pmatrix}
k\left( \mathbf{x} \right) \\
h\begin{pmatrix}
\mathbf{x} \\
k\left( \mathbf{x} \right) \\
\end{pmatrix} \\
\end{pmatrix}
\end{align*}
これが求める陰関数となる。実際、点$g\left( \mathbf{a} \right)$について、次のようになる。
\begin{align*}
g\left( \mathbf{a} \right) = \begin{pmatrix}
k\left( \mathbf{a} \right) \\
h\begin{pmatrix}
\mathbf{a} \\
k\left( \mathbf{a} \right) \\
\end{pmatrix} \\
\end{pmatrix} = \begin{pmatrix}
\mathbf{b}^{*} \\
h\begin{pmatrix}
\mathbf{a} \\
\mathbf{b}^{*} \\
\end{pmatrix} \\
\end{pmatrix} = \begin{pmatrix}
\mathbf{b}^{*} \\
b_{n} \\
\end{pmatrix} = \mathbf{b}
\end{align*}\par
さらに、その関数$h$はその開近傍$V_{1}$で連続であるかつ、その関数$k$はその開近傍$V$で連続であるので、その関数$g$もその開近傍$V$で連続である。\par
$\forall\begin{pmatrix}
\mathbf{x} \\
\mathbf{y} \\
\end{pmatrix} \in V \times W$に対し、次のようになる。
\begin{align*}
f\begin{pmatrix}
\mathbf{x} \\
\mathbf{y} \\
\end{pmatrix} = 0 &\Leftrightarrow \left\{ \begin{matrix}
f\begin{pmatrix}
\mathbf{x} \\
\mathbf{y}^{*} \\
y_{n} \\
\end{pmatrix} = 0 \\
f_{n}\begin{pmatrix}
\mathbf{x} \\
\mathbf{y} \\
\end{pmatrix} = 0 \\
\end{matrix} \right. \Leftrightarrow \left\{ \begin{matrix}
f\begin{pmatrix}
\mathbf{x} \\
\mathbf{y}^{*} \\
h\begin{pmatrix}
\mathbf{x} \\
\mathbf{y}^{*} \\
\end{pmatrix} \\
\end{pmatrix} = 0 \\
y_{n} = h\begin{pmatrix}
\mathbf{x} \\
\mathbf{y}^{*} \\
\end{pmatrix} \\
\end{matrix} \right.\ \\
&\Leftrightarrow \left\{ \begin{matrix}
F\begin{pmatrix}
\mathbf{x} \\
\mathbf{y}^{*} \\
\end{pmatrix} = 0 \\
y_{n} = h\begin{pmatrix}
\mathbf{x} \\
\mathbf{y}^{*} \\
\end{pmatrix} \\
\end{matrix} \right.\  \Leftrightarrow \left\{ \begin{matrix}
\mathbf{y}^{*} = k\left( \mathbf{x} \right) \\
y_{n} = h\begin{pmatrix}
\mathbf{x} \\
\mathbf{y}^{*} \\
\end{pmatrix} \\
\end{matrix} \right.\ \\
&\Leftrightarrow \mathbf{y} = \begin{pmatrix}
k\left( \mathbf{x} \right) \\
h\begin{pmatrix}
\mathbf{x} \\
\mathbf{y}^{*} \\
\end{pmatrix} \\
\end{pmatrix} = g\left( \mathbf{x} \right)
\end{align*}\par
それらの関数たち$h$、$k$は陰関数定理より$C^{1}$級であるから、その関数$g$も$C^{1}$級であり、$\forall\mathbf{x} \in V$に対し、$f\begin{pmatrix}
\mathbf{x} \\
g\left( \mathbf{x} \right) \\
\end{pmatrix} = 0$が成り立つので、関数$g_{\downarrow}$が次のように定義されると、
\begin{align*}
g_{\downarrow} = \left( g_{i}^{\downarrow} \right)_{i \in \varLambda_{m + n}}:V \rightarrow \mathbb{R}^{m + n};\mathbf{x} \mapsto \begin{pmatrix}
\mathbf{x} \\
g\left( \mathbf{x} \right) \\
\end{pmatrix}
\end{align*}
$f \circ g_{\downarrow} = 0$で次のようになる\footnote{ここでもEinstein縮約記法を用いれば、$\forall i\in \varLambda_{n} \forall j\in \varLambda_{m} $に対し、$k^{*} \in \varLambda_{m} $、$k_{*} \in \varLambda_{m+n} \setminus \varLambda_{m} $、$k\in \varLambda_{n} $として次のようになる。
\begin{align*}
0 &= \partial_{j} \left( f_{i} \circ g_{\downarrow} \right) \\
&= \left( \partial_{k^{*} } f_{i} \circ g_{\downarrow} \right) \partial_{j} g_{k^{*} }^{\downarrow} + \left( \partial_{k_{*} } f_{i} \circ g_{\downarrow} \right) \partial_{j} g_{k_{*} }^{\downarrow} \\
&= \left( \partial_{k^{*} } f_{i} \circ g_{\downarrow} \right) \delta_{j k^{*}} + \left( \partial_{k_{*} } f_{i} \circ g_{\downarrow} \right) \partial_{j} g_{k_{*} - m} \\
&= \partial_{j} f_{i} \circ g_{\downarrow} + \left( \partial_{m + k} f_{i} \circ g_{\downarrow} \right) \partial_{j} g_{k} \\
&= \left( \partial_{m + k} f_{i} \circ g_{\downarrow} \right) \partial_{j} g_{k} + \partial_{j} f_{i} \circ g_{\downarrow} 
\end{align*} }。
\begin{align*}
O &= J_{f \circ g_{\downarrow}} = \left( J_{f} \circ g_{\downarrow} \right)J_{g_{\downarrow}}\\
&= \left({}^{t}\left( \begin{pmatrix}
\nabla^{*} \\
\nabla_{*} \\
\end{pmatrix}{}^{t}f \right) \circ g_{\downarrow} \right){}^{t}\left( \nabla^{*}{}^{t}g_{\downarrow} \right)\\
&= \left({}^{t}\begin{pmatrix}
\nabla^{*}{}^{t}f \\
\nabla_{*}{}^{t}f \\
\end{pmatrix} \circ g_{\downarrow} \right){}^{t}\left( \nabla^{*}\begin{pmatrix}
I_{V} & {}^{t}g \\
\end{pmatrix} \right)\\
&= \left({}^{t}\begin{pmatrix}
\nabla^{*}{}^{t}f \\
\nabla_{*}{}^{t}f \\
\end{pmatrix} \circ g_{\downarrow} \right){}^{t}\begin{pmatrix}
\nabla^{*}I_{V} & \nabla^{*}{}^{t}g \\
\end{pmatrix}\\
&= \begin{pmatrix}
{}^{t}\left( \nabla^{*}{}^{t}f \right) \circ g_{\downarrow} &{}^{t}\left( \nabla_{*}{}^{t}f \right) \circ g_{\downarrow} \\
\end{pmatrix}\begin{pmatrix}
{}^{t}\begin{pmatrix}
\nabla^{*}I_{V} \\
\end{pmatrix} \\
{}^{t}\begin{pmatrix}
\nabla^{*}{}^{t}g \\
\end{pmatrix} \\
\end{pmatrix}\\
&= \begin{pmatrix}
{}^{t}\left( \nabla^{*}{}^{t}f \right) \circ g_{\downarrow} &{}^{t}\left( \nabla_{*}{}^{t}f \right) \circ g_{\downarrow} \\
\end{pmatrix}\begin{pmatrix}
I_{m} \\
J_{g} \\
\end{pmatrix}\\
&={}^{t}\left( \nabla^{*}{}^{t}f \right) \circ g_{\downarrow} + \left({}^{t}\left( \nabla_{*}{}^{t}f \right) \circ g_{\downarrow} \right)J_{g}
\end{align*}
ここで、仮定より$\det{{}^{t}\left( \nabla_{*}{}^{t}f \right)}\begin{pmatrix}
\mathbf{a} \\
\mathbf{b} \\
\end{pmatrix} \neq 0$が成り立つので、その関数$f$が$C^{1}$級であることとその関数$g$が連続であることから、その点$\mathbf{a}$のある開近傍$V$が存在して、$\forall\mathbf{x} \in V$に対し、次式が成り立つ。
\begin{align*}
\det\left({}^{t}\left( \nabla_{*}{}^{t}f \right) \circ g_{\downarrow} \right)\left( \mathbf{x} \right) = \det{{}^{t}\left( \nabla_{*}{}^{t}f \right)} \circ g_{\downarrow}\left( \mathbf{x} \right) = \det{{}^{t}\left( \nabla_{*}{}^{t}f \right)}\begin{pmatrix}
\mathbf{x} \\
g\left( \mathbf{x} \right) \\
\end{pmatrix} \neq 0
\end{align*}
ゆえに、その行列$\partial_{*}f \circ g_{\downarrow}$は正則行列でこれの逆行列$\left({}^{t}\left( \nabla_{*}{}^{t}f \right) \circ g_{\downarrow} \right)^{- 1}$が存在する。よって、次のようになる。
\begin{align*}
J_{g} &= \left({}^{t}\left( \nabla_{*}{}^{t}f \right) \circ g_{\downarrow} \right)^{- 1}\left({}^{t}\left( \nabla_{*}{}^{t}f \right) \circ g_{\downarrow} \right)J_{g}\\
&= \left({}^{t}\left( \nabla_{*}{}^{t}f \right) \circ g_{\downarrow} \right)^{- 1}\left({}^{t}\left( \nabla^{*}{}^{t}f \right) \circ g_{\downarrow} \right. \\
&\quad \left. + \left({}^{t}\left( \nabla_{*}{}^{t}f \right) \circ g_{\downarrow} \right)J_{g} -{}^{t}\left( \nabla^{*}{}^{t}f \right) \circ g_{\downarrow} \right)\\
&= \left({}^{t}\left( \nabla_{*}{}^{t}f \right) \circ g_{\downarrow} \right)^{- 1}\left({}^{t}\left( \nabla^{*}{}^{t}f \right) \circ g_{\downarrow} \right. \\
&\quad \left. + \left({}^{t}\left( \nabla_{*}{}^{t}f \right) \circ g_{\downarrow} \right)J_{g} \right) - \left({}^{t}\left( \nabla_{*}{}^{t}f \right) \circ g_{\downarrow} \right)^{- 1}\left({}^{t}\left( \nabla^{*}{}^{t}f \right) \circ g_{\downarrow} \right)\\
&= \left({}^{t}\left( \nabla_{*}{}^{t}f \right) \circ g_{\downarrow} \right)^{- 1}O - \left({}^{t}\left( \nabla_{*}{}^{t}f \right) \circ g_{\downarrow} \right)^{- 1}\left({}^{t}\left( \nabla^{*}{}^{t}f \right) \circ g_{\downarrow} \right)\\
&= - \left({}^{t}\left( \nabla_{*}{}^{t}f \right) \circ g_{\downarrow} \right)^{- 1}\left({}^{t}\left( \nabla^{*}{}^{t}f \right) \circ g_{\downarrow} \right)\\
&= - \left({}^{t}\left( \nabla_{*}{}^{t}f \right)^{- 1}{}^{t}\left( \nabla^{*}{}^{t}f \right) \right) \circ g_{\downarrow}
\end{align*}
\begin{comment}
よって、次式が成り立つ。
\begin{align*}
J_{g}\left( \mathbf{x} \right) &= - \left({}^{t}\left( \nabla_{*}{}^{t}f \right)^{- 1}{}^{t}\left( \nabla^{*}{}^{t}f \right) \right) \circ g_{\downarrow}\left( \mathbf{x} \right)\\
&= -{}^{t}\left( \nabla_{*}{}^{t}f \right)^{- 1}{}^{t}\left( \nabla^{*}{}^{t}f \right)\begin{pmatrix}
\mathbf{x} \\
g\left( \mathbf{x} \right) \\
\end{pmatrix}\\
&= - \begin{pmatrix}
\partial_{m + 1}f_{1} & \partial_{m + 2}f_{1} & \cdots & \partial_{m + n}f_{1} \\
\partial_{m + 1}f_{2} & \partial_{m + 2}f_{2} & \cdots & \partial_{m + n}f_{2} \\
 \vdots & \vdots & \ddots & \vdots \\
\partial_{m + 1}f_{n} & \partial_{m + 2}f_{n} & \cdots & \partial_{m + n}f_{n} \\
\end{pmatrix}^{- 1}\begin{pmatrix}
\partial_{1}f_{1} & \partial_{2}f_{1} & \cdots & \partial_{m}f_{1} \\
\partial_{1}f_{2} & \partial_{2}f_{2} & \cdots & \partial_{m}f_{2} \\
 \vdots & \vdots & \ddots & \vdots \\
\partial_{1}f_{n} & \partial_{2}f_{n} & \cdots & \partial_{m}f_{n} \\
\end{pmatrix}\begin{pmatrix}
\mathbf{x} \\
g\left( \mathbf{x} \right) \\
\end{pmatrix}\\
&= - \begin{pmatrix}
\partial_{m + 1}f_{1} \circ g_{\downarrow} & \partial_{m + 2}f_{1} \circ g_{\downarrow} & \cdots & \partial_{m + n}f_{1} \circ g_{\downarrow} \\
\partial_{m + 1}f_{2} \circ g_{\downarrow} & \partial_{m + 2}f_{2} \circ g_{\downarrow} & \cdots & \partial_{m + n}f_{2} \circ g_{\downarrow} \\
 \vdots & \vdots & \ddots & \vdots \\
\partial_{m + 1}f_{n} \circ g_{\downarrow} & \partial_{m + 2}f_{n} \circ g_{\downarrow} & \cdots & \partial_{m + n}f_{n} \circ g_{\downarrow} \\
\end{pmatrix}^{- 1} \\
&\quad \begin{pmatrix}
\partial_{1}f_{1} \circ g_{\downarrow} & \partial_{2}f_{1} \circ g_{\downarrow} & \cdots & \partial_{m}f_{1} \circ g_{\downarrow} \\
\partial_{1}f_{2} \circ g_{\downarrow} & \partial_{2}f_{2} \circ g_{\downarrow} & \cdots & \partial_{m}f_{2} \circ g_{\downarrow} \\
 \vdots & \vdots & \ddots & \vdots \\
\partial_{1}f_{n} \circ g_{\downarrow} & \partial_{2}f_{n} \circ g_{\downarrow} & \cdots & \partial_{m}f_{n} \circ g_{\downarrow} \\
\end{pmatrix}\left( \mathbf{x} \right)
\end{align*}
\end{comment}
\end{proof}
\begin{thm}[よりよい陰関数定理]\label{4.4.1.4}
$U \subseteq \mathbb{R}^{m} \times \mathbb{R}^{n}$なる開集合$U$を用いた$C^{r}$級関数$f:U \rightarrow \mathbb{R}^{n};\begin{pmatrix}
\mathbf{x} \\
\mathbf{y} \\
\end{pmatrix} \mapsto f\begin{pmatrix}
\mathbf{x} \\
\mathbf{y} \\
\end{pmatrix}$が、$\exists\begin{pmatrix}
\mathbf{a} \\
\mathbf{b} \\
\end{pmatrix} \in \mathbb{R}^{m} \times \mathbb{R}^{n}$に対し、$f\begin{pmatrix}
\mathbf{a} \\
\mathbf{b} \\
\end{pmatrix} = \mathbf{0}$かつ$f = \left( f_{i} \right)_{i \in \varLambda_{n}}$、$\nabla^{*} = \left( \partial_{i} \right)_{i \in \varLambda_{m}}$、$\nabla_{*} = \left( \partial_{i} \right)_{i \in \varLambda_{m + n} \setminus \varLambda_{m}}$とおいて次式が成り立つとき、
\begin{align*}
\det{{}^{t}\left( \nabla_{*}{}^{t}f \right)}\begin{pmatrix}
\mathbf{a} \\
\mathbf{b} \\
\end{pmatrix} = \left| \begin{matrix}
\partial_{m + 1}f_{1} & \partial_{m + 2}f_{1} & \cdots & \partial_{m + n}f_{1} \\
\partial_{m + 1}f_{2} & \partial_{m + 2}f_{2} & \cdots & \partial_{m + n}f_{2} \\
 \vdots & \vdots & \ddots & \vdots \\
\partial_{m + 1}f_{n} & \partial_{m + 2}f_{n} & \cdots & \partial_{m + n}f_{n} \\
\end{matrix} \right|\begin{pmatrix}
\mathbf{a} \\
\mathbf{b} \\
\end{pmatrix} \neq 0
\end{align*}
定理\ref{4.4.1.3}での陰関数$g:V \rightarrow W$もその開近傍$V$上で$C^{r}$級である。\par
定理\ref{4.4.1.3}、定理\ref{4.4.1.4}をまとめてここではよりよい陰関数定理ということにする。
\end{thm}
\begin{proof} 定理\ref{4.4.1.2}と同様にして、示される。
\end{proof}\par
この定理は関数の微分からみた形式的な操作で考えることにしよう。$\forall i \in \varLambda_{n}$に対し、$f_{i}\left( x_{1},\ \ x_{2},\ \ \cdots,\ \ x_{m},\right. \ $ $\left. y_{1},\ \ y_{2},\ \ \cdots,\ \ y_{n} \right) = 0$から次のようになる。
\begin{align*}
\sum_{k \in \varLambda_{m}} {\frac{\partial f_{i}}{\partial x_{k}}dx_{k}} + \sum_{l \in \varLambda_{n}} {\frac{\partial f_{i}}{\partial y_{l}}dy_{l}} = 0
\end{align*}
ここで、Einstein縮約記法を用いれば、$\forall i\in \varLambda_{n} \forall j\in \varLambda_{m} $に対し、$k\in \varLambda_{m} $、$l\in \varLambda_{n} $として次のようになるので、
\begin{align*}
\frac{\partial f_{i} }{\partial x_k } dx_{k} + \frac{\partial f_{i} }{\partial y_{l} } dy_{l} =0
\end{align*}
次のようになる。
\begin{align*}
0 &= \left( \frac{\partial f_{i} }{\partial x_{k} } dx_{k} + \frac{\partial f_{i} }{\partial y_{l} } dy_{l} \right) \frac{1}{dx_{j} } \\
&= \frac{\partial f_{i} }{\partial x_{k} } \frac{\partial x_{k} }{\partial x_{j} } + \frac{\partial f_{i} }{\partial y_{l} } \frac{\partial y_{l} }{\partial x_{j} } \\
&= \frac{\partial f_{i} }{\partial x_{k} } \delta_{jk} + \frac{\partial f_{i} }{\partial y_{l} } \frac{\partial y_{l} }{\partial x_{j} } \\
&= \frac{\partial f_{i} }{\partial y_{l} } \frac{\partial y_{l} }{\partial x_{j} } + \frac{\partial f_{i} }{\partial x_{j} } 
\end{align*}
これにより、次式が成り立つことになる。
\begin{align*}
\begin{pmatrix}
  \frac{\partial y_{1}}{\partial x_{1}} & \frac{\partial y_{1}}{\partial x_{2}} & \cdots & \frac{\partial y_{1}}{\partial x_{n}} \\
  \frac{\partial y_{2}}{\partial x_{1}} & \frac{\partial y_{2}}{\partial x_{2}} & \cdots & \frac{\partial y_{2}}{\partial x_{n}} \\
     \vdots & \vdots & \ddots & \vdots \\
  \frac{\partial y_{n}}{\partial x_{1}} & \frac{\partial y_{n}}{\partial x_{2}} & \cdots & \frac{\partial y_{n}}{\partial x_{n}} \\
    \end{pmatrix} = - \begin{pmatrix}
  \frac{\partial f_{1}}{\partial y_{1}} & \frac{\partial f_{1}}{\partial y_{2}} & \cdots & \frac{\partial f_{1}}{\partial y_{n}} \\
  \frac{\partial f_{2}}{\partial y_{1}} & \frac{\partial f_{2}}{\partial y_{2}} & \cdots & \frac{\partial f_{2}}{\partial y_{n}} \\
     \vdots & \vdots & \ddots & \vdots \\
  \frac{\partial f_{n}}{\partial y_{1}} & \frac{\partial f_{n}}{\partial y_{2}} & \cdots & \frac{\partial f_{n}}{\partial y_{n}} \\
\end{pmatrix}^{- 1}
\begin{pmatrix}
  \frac{\partial f_{1}}{\partial x_{1}} & \frac{\partial f_{1}}{\partial x_{2}} & \cdots & \frac{\partial f_{1}}{\partial x_{n}} \\
  \frac{\partial f_{2}}{\partial x_{1}} & \frac{\partial f_{2}}{\partial x_{2}} & \cdots & \frac{\partial f_{2}}{\partial x_{n}} \\
     \vdots & \vdots & \ddots & \vdots \\
  \frac{\partial f_{n}}{\partial x_{1}} & \frac{\partial f_{n}}{\partial x_{2}} & \cdots & \frac{\partial f_{n}}{\partial x_{n}} \\
\end{pmatrix}
\end{align*}
\begin{comment}
これにより、次式が成り立つことになる。
\begin{align*}
\begin{pmatrix}
\sum_{k \in \varLambda_{m}} {\frac{\partial f_{1}}{\partial x_{k}}dx_{k}} + \sum_{l \in \varLambda_{n}} {\frac{\partial f_{1}}{\partial y_{l}}dy_{l}} \\
\sum_{k \in \varLambda_{m}} {\frac{\partial f_{2}}{\partial x_{k}}dx_{k}} + \sum_{l \in \varLambda_{n}} {\frac{\partial f_{2}}{\partial y_{l}}dy_{l}} \\
 \vdots \\
\sum_{k \in \varLambda_{m}} {\frac{\partial f_{n}}{\partial x_{k}}dx_{k}} + \sum_{l \in \varLambda_{n}} {\frac{\partial f_{n}}{\partial y_{l}}dy_{l}} \\
\end{pmatrix} = \begin{pmatrix}
0 \\
0 \\
 \vdots \\
0 \\
\end{pmatrix}
\end{align*}
したがって、次のようになる。
\begin{align*}
&\quad \begin{pmatrix}
\sum_{k \in \varLambda_{m}} {\frac{\partial f_{1}}{\partial x_{k}}dx_{k}} + \sum_{l \in \varLambda_{n}} {\frac{\partial f_{1}}{\partial y_{l}}dy_{l}} \\
\sum_{k \in \varLambda_{m}} {\frac{\partial f_{2}}{\partial x_{k}}dx_{k}} + \sum_{l \in \varLambda_{n}} {\frac{\partial f_{2}}{\partial y_{l}}dy_{l}} \\
 \vdots \\
\sum_{k \in \varLambda_{m}} {\frac{\partial f_{n}}{\partial x_{k}}dx_{k}} + \sum_{l \in \varLambda_{n}} {\frac{\partial f_{n}}{\partial y_{l}}dy_{l}} \\
\end{pmatrix} = \begin{pmatrix}
0 \\
0 \\
 \vdots \\
0 \\
\end{pmatrix}\\
&\Leftrightarrow \begin{pmatrix}
\sum_{l \in \varLambda_{n}} {\frac{\partial f_{1}}{\partial y_{l}}dy_{l}} \\
\sum_{l \in \varLambda_{n}} {\frac{\partial f_{2}}{\partial y_{l}}dy_{l}} \\
 \vdots \\
\sum_{l \in \varLambda_{n}} {\frac{\partial f_{n}}{\partial y_{l}}dy_{l}} \\
\end{pmatrix} = - \begin{pmatrix}
\sum_{k \in \varLambda_{m}} {\frac{\partial f_{1}}{\partial x_{k}}dx_{k}} \\
\sum_{k \in \varLambda_{m}} {\frac{\partial f_{2}}{\partial x_{k}}dx_{k}} \\
 \vdots \\
\sum_{k \in \varLambda_{m}} {\frac{\partial f_{n}}{\partial x_{k}}dx_{k}} \\
\end{pmatrix}\\
&\Leftrightarrow \begin{pmatrix}
\frac{\partial f_{1}}{\partial y_{1}}dy_{1} + \frac{\partial f_{1}}{\partial y_{2}}dy_{2} + \cdots + \frac{\partial f_{1}}{\partial y_{n}}dy_{n} \\
\frac{\partial f_{2}}{\partial y_{1}}dy_{1} + \frac{\partial f_{2}}{\partial y_{2}}dy_{2} + \cdots + \frac{\partial f_{2}}{\partial y_{n}}dy_{n} \\
 \vdots \\
\frac{\partial f_{n}}{\partial y_{1}}dy_{1} + \frac{\partial f_{n}}{\partial y_{2}}dy_{2} + \cdots + \frac{\partial f_{n}}{\partial y_{n}}dy_{n} \\
\end{pmatrix} = - \begin{pmatrix}
\sum_{k \in \varLambda_{m}} {\frac{\partial f_{1}}{\partial x_{k}}dx_{k}} \\
\sum_{k \in \varLambda_{m}} {\frac{\partial f_{2}}{\partial x_{k}}dx_{k}} \\
 \vdots \\
\sum_{k \in \varLambda_{m}} {\frac{\partial f_{n}}{\partial x_{k}}dx_{k}} \\
\end{pmatrix}\\
&\Leftrightarrow \begin{pmatrix}
\frac{\partial f_{1}}{\partial y_{1}} & \frac{\partial f_{1}}{\partial y_{2}} & \cdots & \frac{\partial f_{1}}{\partial y_{n}} \\
\frac{\partial f_{2}}{\partial y_{1}} & \frac{\partial f_{2}}{\partial y_{2}} & \cdots & \frac{\partial f_{2}}{\partial y_{n}} \\
 \vdots & \vdots & \ddots & \vdots \\
\frac{\partial f_{n}}{\partial y_{1}} & \frac{\partial f_{n}}{\partial y_{2}} & \cdots & \frac{\partial f_{n}}{\partial y_{n}} \\
\end{pmatrix}\begin{pmatrix}
dy_{1} \\
dy_{2} \\
 \vdots \\
dy_{n} \\
\end{pmatrix} = - \begin{pmatrix}
\sum_{k \in \varLambda_{m}} {\frac{\partial f_{1}}{\partial x_{k}}dx_{k}} \\
\sum_{k \in \varLambda_{m}} {\frac{\partial f_{2}}{\partial x_{k}}dx_{k}} \\
 \vdots \\
\sum_{k \in \varLambda_{m}} {\frac{\partial f_{n}}{\partial x_{k}}dx_{k}} \\
\end{pmatrix}\\
&\Leftrightarrow \begin{pmatrix}
dy_{1} \\
dy_{2} \\
 \vdots \\
dy_{n} \\
\end{pmatrix} = - \begin{pmatrix}
\frac{\partial f_{1}}{\partial y_{1}} & \frac{\partial f_{1}}{\partial y_{2}} & \cdots & \frac{\partial f_{1}}{\partial y_{n}} \\
\frac{\partial f_{2}}{\partial y_{1}} & \frac{\partial f_{2}}{\partial y_{2}} & \cdots & \frac{\partial f_{2}}{\partial y_{n}} \\
 \vdots & \vdots & \ddots & \vdots \\
\frac{\partial f_{n}}{\partial y_{1}} & \frac{\partial f_{n}}{\partial y_{2}} & \cdots & \frac{\partial f_{n}}{\partial y_{n}} \\
\end{pmatrix}^{- 1}\begin{pmatrix}
\sum_{k \in \varLambda_{m}} {\frac{\partial f_{1}}{\partial x_{k}}dx_{k}} \\
\sum_{k \in \varLambda_{m}} {\frac{\partial f_{2}}{\partial x_{k}}dx_{k}} \\
 \vdots \\
\sum_{k \in \varLambda_{m}} {\frac{\partial f_{n}}{\partial x_{k}}dx_{k}} \\
\end{pmatrix}\\
&\Rightarrow \begin{pmatrix}
dy_{1} \\
dy_{2} \\
 \vdots \\
dy_{n} \\
\end{pmatrix}\begin{pmatrix}
\frac{1}{dx_{1}} & \frac{1}{dx_{2}} & \cdots & \frac{1}{dx_{n}} \\
\end{pmatrix} = - \begin{pmatrix}
\frac{\partial f_{1}}{\partial y_{1}} & \frac{\partial f_{1}}{\partial y_{2}} & \cdots & \frac{\partial f_{1}}{\partial y_{n}} \\
\frac{\partial f_{2}}{\partial y_{1}} & \frac{\partial f_{2}}{\partial y_{2}} & \cdots & \frac{\partial f_{2}}{\partial y_{n}} \\
 \vdots & \vdots & \ddots & \vdots \\
\frac{\partial f_{n}}{\partial y_{1}} & \frac{\partial f_{n}}{\partial y_{2}} & \cdots & \frac{\partial f_{n}}{\partial y_{n}} \\
\end{pmatrix}^{- 1}\begin{pmatrix}
\sum_{k \in \varLambda_{m}} {\frac{\partial f_{1}}{\partial x_{k}}dx_{k}} \\
\sum_{k \in \varLambda_{m}} {\frac{\partial f_{2}}{\partial x_{k}}dx_{k}} \\
 \vdots \\
\sum_{k \in \varLambda_{m}} {\frac{\partial f_{n}}{\partial x_{k}}dx_{k}} \\
\end{pmatrix}\begin{pmatrix}
\frac{1}{dx_{1}} & \frac{1}{dx_{2}} & \cdots & \frac{1}{dx_{n}} \\
\end{pmatrix}\\
&\Leftrightarrow \begin{pmatrix}
\frac{\partial y_{1}}{\partial x_{1}} & \frac{\partial y_{1}}{\partial x_{2}} & \cdots & \frac{\partial y_{1}}{\partial x_{n}} \\
\frac{\partial y_{2}}{\partial x_{1}} & \frac{\partial y_{2}}{\partial x_{2}} & \cdots & \frac{\partial y_{2}}{\partial x_{n}} \\
 \vdots & \vdots & \ddots & \vdots \\
\frac{\partial y_{n}}{\partial x_{1}} & \frac{\partial y_{n}}{\partial x_{2}} & \cdots & \frac{\partial y_{n}}{\partial x_{n}} \\
\end{pmatrix} = - \begin{pmatrix}
\frac{\partial f_{1}}{\partial y_{1}} & \frac{\partial f_{1}}{\partial y_{2}} & \cdots & \frac{\partial f_{1}}{\partial y_{n}} \\
\frac{\partial f_{2}}{\partial y_{1}} & \frac{\partial f_{2}}{\partial y_{2}} & \cdots & \frac{\partial f_{2}}{\partial y_{n}} \\
 \vdots & \vdots & \ddots & \vdots \\
\frac{\partial f_{n}}{\partial y_{1}} & \frac{\partial f_{n}}{\partial y_{2}} & \cdots & \frac{\partial f_{n}}{\partial y_{n}} \\
\end{pmatrix}^{- 1}\begin{pmatrix}
\frac{\partial f_{1}}{\partial x_{1}} & \frac{\partial f_{1}}{\partial x_{2}} & \cdots & \frac{\partial f_{1}}{\partial x_{n}} \\
\frac{\partial f_{2}}{\partial x_{1}} & \frac{\partial f_{2}}{\partial x_{2}} & \cdots & \frac{\partial f_{2}}{\partial x_{n}} \\
 \vdots & \vdots & \ddots & \vdots \\
\frac{\partial f_{n}}{\partial x_{1}} & \frac{\partial f_{n}}{\partial x_{2}} & \cdots & \frac{\partial f_{n}}{\partial x_{n}} \\
\end{pmatrix}
\end{align*}
\end{comment}
\begin{thebibliography}{50}
  \bibitem{1}
  杉浦光夫, 解析入門II, 東京大学出版社, 1985. 第22刷 p1-16 ISBN978-4-13-062006-2
\end{thebibliography}
%\hypertarget{ux8b1dux8f9e}{%
\section*{謝辞}%\label{ux8b1dux8f9e}}\par
本文中の証明の行間を埋めるのに手伝っていただいた明治大学院数学科のTAの方々にこの場を借りてお礼を述べる。おかげさまで証明を書ききることができた。
\end{document}