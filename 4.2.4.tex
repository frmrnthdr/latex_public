\documentclass[dvipdfmx]{jsarticle}
\setcounter{section}{2}
\setcounter{subsection}{3}
\usepackage{amsmath,amsfonts,amssymb,array,comment,mathtools,url,docmute}
\usepackage{longtable,booktabs,dcolumn,tabularx,mathtools,multirow,colortbl,xcolor}
\usepackage[dvipdfmx]{graphics}
\usepackage{bmpsize}
\usepackage{amsthm}
\usepackage{enumitem}
\setlistdepth{20}
\renewlist{itemize}{itemize}{20}
\setlist[itemize]{label=•}
\renewlist{enumerate}{enumerate}{20}
\setlist[enumerate]{label=\arabic*.}
\setcounter{MaxMatrixCols}{20}
\setcounter{tocdepth}{3}
\newcommand{\rotin}{\text{\rotatebox[origin=c]{90}{$\in $}}}
\newcommand{\amap}[6]{\text{\raisebox{-0.7cm}{\begin{tikzpicture} 
  \node (a) at (0, 1) {$\textstyle{#2}$};
  \node (b) at (#6, 1) {$\textstyle{#3}$};
  \node (c) at (0, 0) {$\textstyle{#4}$};
  \node (d) at (#6, 0) {$\textstyle{#5}$};
  \node (x) at (0, 0.5) {$\rotin $};
  \node (x) at (#6, 0.5) {$\rotin $};
  \draw[->] (a) to node[xshift=0pt, yshift=7pt] {$\textstyle{\scriptstyle{#1}}$} (b);
  \draw[|->] (c) to node[xshift=0pt, yshift=7pt] {$\textstyle{\scriptstyle{#1}}$} (d);
\end{tikzpicture}}}}
\newcommand{\twomaps}[9]{\text{\raisebox{-0.7cm}{\begin{tikzpicture} 
  \node (a) at (0, 1) {$\textstyle{#3}$};
  \node (b) at (#9, 1) {$\textstyle{#4}$};
  \node (c) at (#9+#9, 1) {$\textstyle{#5}$};
  \node (d) at (0, 0) {$\textstyle{#6}$};
  \node (e) at (#9, 0) {$\textstyle{#7}$};
  \node (f) at (#9+#9, 0) {$\textstyle{#8}$};
  \node (x) at (0, 0.5) {$\rotin $};
  \node (x) at (#9, 0.5) {$\rotin $};
  \node (x) at (#9+#9, 0.5) {$\rotin $};
  \draw[->] (a) to node[xshift=0pt, yshift=7pt] {$\textstyle{\scriptstyle{#1}}$} (b);
  \draw[|->] (d) to node[xshift=0pt, yshift=7pt] {$\textstyle{\scriptstyle{#2}}$} (e);
  \draw[->] (b) to node[xshift=0pt, yshift=7pt] {$\textstyle{\scriptstyle{#1}}$} (c);
  \draw[|->] (e) to node[xshift=0pt, yshift=7pt] {$\textstyle{\scriptstyle{#2}}$} (f);
\end{tikzpicture}}}}
\renewcommand{\thesection}{第\arabic{section}部}
\renewcommand{\thesubsection}{\arabic{section}.\arabic{subsection}}
\renewcommand{\thesubsubsection}{\arabic{section}.\arabic{subsection}.\arabic{subsubsection}}
\everymath{\displaystyle}
\allowdisplaybreaks[4]
\usepackage{vtable}
\theoremstyle{definition}
\newtheorem{thm}{定理}[subsection]
\newtheorem*{thm*}{定理}
\newtheorem{dfn}{定義}[subsection]
\newtheorem*{dfn*}{定義}
\newtheorem{axs}[dfn]{公理}
\newtheorem*{axs*}{公理}
\renewcommand{\headfont}{\bfseries}
\makeatletter
  \renewcommand{\section}{%
    \@startsection{section}{1}{\z@}%
    {\Cvs}{\Cvs}%
    {\normalfont\huge\headfont\raggedright}}
\makeatother
\makeatletter
  \renewcommand{\subsection}{%
    \@startsection{subsection}{2}{\z@}%
    {0.5\Cvs}{0.5\Cvs}%
    {\normalfont\LARGE\headfont\raggedright}}
\makeatother
\makeatletter
  \renewcommand{\subsubsection}{%
    \@startsection{subsubsection}{3}{\z@}%
    {0.4\Cvs}{0.4\Cvs}%
    {\normalfont\Large\headfont\raggedright}}
\makeatother
\makeatletter
\renewenvironment{proof}[1][\proofname]{\par
  \pushQED{\qed}%
  \normalfont \topsep6\p@\@plus6\p@\relax
  \trivlist
  \item\relax
  {
  #1\@addpunct{.}}\hspace\labelsep\ignorespaces
}{%
  \popQED\endtrivlist\@endpefalse
}
\makeatother
\renewcommand{\proofname}{\textbf{証明}}
\usepackage{tikz,graphics}
\usepackage[dvipdfmx]{hyperref}
\usepackage{pxjahyper}
\hypersetup{
 setpagesize=false,
 bookmarks=true,
 bookmarksdepth=tocdepth,
 bookmarksnumbered=true,
 colorlinks=false,
 pdftitle={},
 pdfsubject={},
 pdfauthor={},
 pdfkeywords={}}
\begin{document}
%\hypertarget{ux43bux430ux43dux434ux430ux443ux306eux8a18ux53f7}{%
\subsection{Landauの記号}%\label{ux43bux430ux43dux434ux430ux443ux306eux8a18ux53f7}}
%\hypertarget{ux43bux430ux43dux434ux430ux443ux306eux8a18ux53f7-1}{%
\subsubsection{Landauの記号}%\label{ux43bux430ux43dux434ux430ux443ux306eux8a18ux53f7-1}}
\begin{dfn}
$D(f) \subseteq \mathbb{R}_{\infty}^{n}$なる関数$f:D(f) \rightarrow \mathbb{R}$を考え$\mathbf{a} \in \mathrm{cl}{D(f)}$とする。$\forall\varepsilon \in \mathbb{R}^{+}\exists\delta \in \mathbb{R}^{+}$に対し、$\mathbf{x} \in U_{0}\left( \mathbf{a},\delta \right) \Rightarrow f\left( \mathbf{x} \right) \in U_{0}(0,\varepsilon)$が成り立つとき、その関数$f$はその点$\mathbf{a}$で無限小であるといい$\lim_{\scriptsize \begin{matrix}
\mathbf{x} \rightarrow \mathbf{a} \\
\mathbf{x} \neq \mathbf{a} \\
\end{matrix}}{f\left( \mathbf{x} \right)} = 0$と書く。拡張$n$次元数空間$\mathbb{R}_{\infty}^{n}$のかわりに補完数直線${}^{*}\mathbb{R}$でおきかえても同様にして定義される。
\end{dfn}
\begin{dfn}
$D(f) \subseteq \mathbb{R}_{\infty}^{n}$なる関数$f:D(f) \rightarrow \mathbb{R}$を考え$\mathbf{a} \in \mathrm{cl}{D(f)}$とする。$\forall\varepsilon \in \mathbb{R}^{+}\exists\delta \in \mathbb{R}^{+}$に対し、$\mathbf{x} \in U_{0}\left( \mathbf{a},\delta \right) \Rightarrow f\left( \mathbf{x} \right) \in U_{0}(\infty,\varepsilon)$が成り立つとき、その関数$f$はその点$\mathbf{a}$で正の無限大であるといい$\lim_{\scriptsize \begin{matrix}
\mathbf{x} \rightarrow \mathbf{a} \\
\mathbf{x} \neq \mathbf{a} \\
\end{matrix}}{f\left( \mathbf{x} \right)} = \infty$と書く。$\forall\varepsilon \in \mathbb{R}^{+}\exists\delta \in \mathbb{R}^{+}$に対し、$\mathbf{x} \in U_{0}\left( \mathbf{a},\delta \right) \Rightarrow f\left( \mathbf{x} \right) \in U_{0}( - \infty,\varepsilon)$が成り立つとき、その関数$f$はその点$\mathbf{a}$で負の無限大であるといい$\lim_{\scriptsize \begin{matrix}
\mathbf{x} \rightarrow \mathbf{a} \\
\mathbf{x} \neq \mathbf{a} \\
\end{matrix}}{f\left( \mathbf{x} \right)} = - \infty$と書く。拡張$n$次元数空間$\mathbb{R}_{\infty}^{n}$のかわりに補完数直線${}^{*}\mathbb{R}$でおきかえても同様にして定義される。
\end{dfn}
\begin{dfn}
$A \subseteq \mathbb{R}_{\infty}^{n}$なる関数たち$f:A \rightarrow \mathbb{R}$、$g:A \rightarrow \mathbb{R}$を考え、$\forall\mathbf{x} \in U_{0}\left( \mathbf{a},\varepsilon \right)$に対し、$g\left( \mathbf{x} \right) \neq 0$が成り立つようなその集合$A$における$\mathbf{a} \in \mathrm{cl}A$なる点$\mathbf{a}$の除外$\varepsilon$近傍$U_{0}\left( \mathbf{a},\varepsilon \right)$が存在するとする。その関数$f$が$\lim_{\scriptsize \begin{matrix}
\mathbf{x} \rightarrow \mathbf{a} \\
\mathbf{x} \neq \mathbf{a} \\
\end{matrix}}{\frac{f}{g}\left( \mathbf{x} \right)} = 0$を満たすとき、その関数$f$はその点$a$においてその関数$g$に比べて無視できるといい$f \ll g$、$g \gg f$などと書く。特に、それらの関数たち$f$、$g$がその点$\mathbf{a}$で無限小である、無限大であるなら、それぞれその関数$f$はその点$\mathbf{a}$においてその関数$g$より高次の無限小である、高次の無限大であるなどという。拡張$n$次元数空間$\mathbb{R}_{\infty}^{n}$のかわりに補完数直線${}^{*}\mathbb{R}$でおきかえても同様にして定義される。
\end{dfn}
\begin{dfn}
$A \subseteq \mathbb{R}_{\infty}^{n}$なる関数$g:A \rightarrow \mathbb{R}$を考え次式のように集合$o_{g,\mathbf{a}}$を定義する。
\begin{align*}
o_{g,\mathbf{a}} = \left\{ f \in \mathfrak{F}\left( A,\mathbb{R} \right) \middle| \lim_{\scriptsize \begin{matrix}
\mathbf{x} \rightarrow \mathbf{a} \\
\mathbf{x} \neq \mathbf{a} \\
\end{matrix}}{\frac{f}{g}\left( \mathbf{x} \right)} = 0 \right\}
\end{align*}
さらに、その点$\mathbf{a}$の除外$\varepsilon$近傍$U_{0}\left( \mathbf{a},\varepsilon \right)$を用いて$U \subseteq U_{0}\left( \mathbf{a},\varepsilon \right)$なるある集合$U$を用いてその関数$f$が、$\forall\mathbf{x} \in A \cap U\exists R \in \mathbb{R}$に対し、$\left| \frac{f}{g}\left( \mathbf{x} \right) \right| < R$を満たすとき、即ち、その関数$\left| \frac{f}{g} \right|$がその集合$A \cap U$上で有界であるとき、その関数$f$は$\mathbf{x} \rightarrow a$のときその関数$g$で押さえられるといい$f \preccurlyeq g$、$g \succcurlyeq f$などと書く。拡張$n$次元数空間$\mathbb{R}_{\infty}^{n}$のかわりに補完数直線${}^{*}\mathbb{R}$でおきかえても同様にして定義される。
\end{dfn}
\begin{dfn}
$A \subseteq \mathbb{R}_{\infty}^{n}$なる関数$g:A \rightarrow \mathbb{R}$を考え次式のように集合$O_{g,\mathbf{a}}$を定義する。
\begin{align*}
O_{g,\mathbf{a}} = \left\{ f \in \mathfrak{F}\left( A,\mathbb{R} \right) \middle| \exists U \in \mathfrak{P}\left( U_{0}\left( \mathbf{a},\varepsilon \right) \right)\forall x \in A \cap U\exists R \in \mathbb{R}\left[ \left| \frac{f}{g}\left( \mathbf{x} \right) \right| < R \right] \right\}
\end{align*}
拡張$n$次元数空間$\mathbb{R}_{\infty}^{n}$のかわりに補完数直線${}^{*}\mathbb{R}$でおきかえても同様にして定義される。
\end{dfn}
\begin{dfn} このようにして定義された集合たち$o_{g,\mathbf{a}}$、$O_{g,\mathbf{a}}$をここでは合わせてLandauの記号と呼ぶことにする。
\end{dfn}
\begin{thm}\label{4.2.4.1}
$\mathbf{a} \in \mathrm{cl}A$なる点$\mathbf{a}$と$A \subseteq \mathbb{R}_{\infty}^{n}$なる関数たち$f:A \rightarrow \mathbb{R}$、$g:A \rightarrow \mathbb{R}$、$h:A \rightarrow \mathbb{R}$を考えると、次式たちが成り立つ。
\begin{align*}
f \in o_{g,\mathbf{a}} &\Rightarrow f \in O_{g,\mathbf{a}}\\
f \in O_{g,\mathbf{a}} \land g \in O_{h,\mathbf{a}} &\Rightarrow f \in O_{h,\mathbf{a}}\\
f \in o_{g,\mathbf{a}} \land g \in O_{h,\mathbf{a}} &\Rightarrow f \in O_{h,\mathbf{a}}\\
f \in O_{g,\mathbf{a}} \land g \in o_{h,\mathbf{a}} &\Rightarrow f \in O_{h,\mathbf{a}}\\
f \in o_{g,\mathbf{a}} \land g \in o_{h,\mathbf{a}} &\Rightarrow f \in o_{h,\mathbf{a}}
\end{align*}
拡張$n$次元数空間$\mathbb{R}_{\infty}^{n}$のかわりに補完数直線${}^{*}\mathbb{R}$でおきかえても同様にして示される。
\end{thm}
\begin{proof}
$A \subseteq \mathbb{R}_{\infty}^{n}$なる関数たち$f:A \rightarrow \mathbb{R}$、$g:A \rightarrow \mathbb{R}$を考え、$\forall\mathbf{x} \in U_{0}\left( \mathbf{a},\varepsilon \right)$に対し、$g\left( \mathbf{x} \right) \neq 0$かつ$h\left( \mathbf{x} \right) \neq 0$が成り立つようなその集合$A$における$\mathbf{a} \in \mathrm{cl}A$なる点$\mathfrak{a}$の除外$\varepsilon$近傍$U_{0}\left( \mathbf{a},\varepsilon \right)$が存在するとする。\par
$f \in o_{g,\mathbf{a}}$が成り立つならそのときに限り、$\lim_{\scriptsize \begin{matrix}
\mathbf{x} \rightarrow \mathbf{a} \\
\mathbf{x} \neq \mathbf{a} \\
\end{matrix}}{\frac{f}{g}\left( \mathbf{x} \right)} = 0$が成り立つ。ここで、$\lim_{\scriptsize \begin{matrix}
\mathbf{x} \rightarrow \mathbf{a} \\
\mathbf{x} \neq \mathbf{a} \\
\end{matrix}}{\frac{f}{g}\left( \mathbf{x} \right)} \in \mathbb{R}$が成り立っているので、その関数$\frac{f}{g}:A \rightarrow \mathbb{R}$はその点$\mathbf{a}$のその集合$A \setminus \left\{ \mathbf{a} \right\}$におけるある$\delta$近傍$U\left( \mathbf{a},\delta \right) \cap \left( A \setminus \left\{ \mathbf{a} \right\} \right)$で有界となるのであった。ここで、その点$\mathbf{a}$の除外$\varepsilon$近傍$U_{0}\left( \mathbf{a},\varepsilon \right)$を用いた次式が成り立つので、
\begin{align*}
U\left( \mathbf{a},\delta \right) \cap \left( A \setminus \left\{ \mathbf{a} \right\} \right) = A \cap \left( U\left( \mathbf{a},\delta \right) \setminus \left\{ \mathbf{a} \right\} \right) = A \cap U_{0}\left( \mathbf{a},\varepsilon \right)
\end{align*}
その関数$\frac{f}{g}$はその集合$A \cap U_{0}\left( \mathbf{a},\delta \right)$上で有界でありその関数$\left| \frac{f}{g} \right|$も同様である。したがって、$f \in O_{g,\mathbf{a}}$が成り立つ。\par
$f \in O_{g,\mathbf{a}} \land g \in O_{h,\mathbf{a}}$が成り立つならそのときに限り、その関数$\left| \frac{f}{g} \right|$がその集合$A \cap U$上で有界であるかつ、その関数$\left| \frac{g}{h} \right|$がその集合$A \cap V$上で有界であるような$U \subseteq U_{0}\left( \mathbf{a},\varepsilon \right)$かつ$V \subseteq U_{0}\left( \mathbf{a},\varepsilon \right)$なる集合たち$U$、$V$が存在する。ここで、$U = U \cap V$のように集合$U$が定義されると、やはり、その関数$\left| \frac{f}{g} \right|$がその集合$A \cap U$上で有界であるかつ、その関数$\left| \frac{g}{h} \right|$がその集合$A \cap U$上で有界である。これにより、$\forall\mathbf{x} \in A \cap U$に対し、次式が成り立つような実数たち$R$、$S$が存在する。
\begin{align*}
\left| \frac{f}{g}\left( \mathbf{x} \right) \right| < R,\ \ \left| \frac{g}{h}\left( \mathbf{x} \right) \right| < S
\end{align*}
したがって、次のようになる。
\begin{align*}
\left| \frac{f}{g}\left( \mathbf{x} \right) \right| < R \land \left| \frac{g}{h}\left( \mathbf{x} \right) \right| < S &\Leftrightarrow \left| f\left( \mathbf{x} \right) \right| < R\left| g\left( \mathbf{x} \right) \right| \land \left| g\left( \mathbf{x} \right) \right| < S\left| h\left( \mathbf{x} \right) \right|\\
&\Leftrightarrow \left| f\left( \mathbf{x} \right) \right| < R\left| g\left( \mathbf{x} \right) \right| \land R\left| g\left( \mathbf{x} \right) \right| < RS\left| h\left( \mathbf{x} \right) \right|\\
&\Rightarrow \left| f\left( \mathbf{x} \right) \right| < RS\left| h\left( \mathbf{x} \right) \right| \Leftrightarrow \left| \frac{f}{h}\left( \mathbf{x} \right) \right| < RS
\end{align*}
これにより、その関数$\left| \frac{f}{h} \right|$はその集合$A \cap U$上で有界であるので、$f \in O_{h,\mathbf{a}}$が成り立つ。\par
$f \in o_{g,\mathbf{a}}$が成り立つならそのときに限り、$\lim_{\scriptsize \begin{matrix}
\mathbf{x} \rightarrow \mathbf{a} \\
\mathbf{x} \neq \mathbf{a} \\
\end{matrix}}{\frac{f}{g}\left( \mathbf{x} \right)} = 0$が成り立つ。ここで、$\lim_{\scriptsize \begin{matrix}
\mathbf{x} \rightarrow \mathbf{a} \\
\mathbf{x} \neq \mathbf{a} \\
\end{matrix}}{\frac{f}{g}\left( \mathbf{x} \right)} \in \mathbb{R}$が成り立っているので、その関数$\frac{f}{g}:A \rightarrow \mathbb{R}$はその点$\mathbf{a}$のその集合$A \setminus \left\{ \mathbf{a} \right\}$におけるある$\delta$近傍$U\left( \mathbf{a},\delta \right) \cap \left( A \setminus \left\{ \mathbf{a} \right\} \right)$で有界となるのであった。ここで、その点$\mathbf{a}$の除外$\varepsilon$近傍$U_{0}\left( \mathbf{a},\varepsilon \right)$を用いた次式が成り立つので、
\begin{align*}
U\left( \mathbf{a},\delta \right) \cap \left( A \setminus \left\{ \mathbf{a} \right\} \right) = A \cap \left( U\left( \mathbf{a},\delta \right) \setminus \left\{ \mathbf{a} \right\} \right) = A \cap U_{0}\left( \mathbf{a},\varepsilon \right)
\end{align*}
その関数$\frac{f}{g}$はその集合$A \cap U_{0}\left( \mathbf{a},\varepsilon \right)$上で有界でありその関数$\left| \frac{f}{g} \right|$も同様である。したがって、$\forall\mathbf{x} \in A \cap U_{0}\left( \mathbf{a},\varepsilon \right)$に対し、$\left| \frac{f}{g}\left( \mathbf{x} \right) \right| < R$が成り立つような実数$R$が存在する。$g \in O_{h,\mathbf{a}}$が成り立つならそのときに限り、その関数$\left| \frac{g}{h} \right|$がその集合$A \cap U$上で有界であるような$U \subseteq U_{0}\left( \mathbf{a},\varepsilon \right)$なる集合$U$が存在する。これにより、$\forall\mathbf{x} \in A \cap U$に対し、$\left| \frac{g}{h}\left( \mathbf{x} \right) \right| < S$が成り立つような実数たち$S$が存在する。したがって、次のようになる。
\begin{align*}
\left| \frac{f}{g}\left( \mathbf{x} \right) \right| < R \land \left| \frac{g}{h}\left( \mathbf{x} \right) \right| < S \Rightarrow \left| \frac{f}{g}\left( \mathbf{x} \right) \right|\left| \frac{g}{h}\left( \mathbf{x} \right) \right| = \left| \frac{f}{h}\left( \mathbf{x} \right) \right| < RS
\end{align*}
これにより、その関数$\left| \frac{f}{h} \right|$はその集合$A \cap U$上で有界であるので、$f \in O_{h,\mathfrak{a}}$が成り立つ。\par
$f \in O_{g,\mathbf{a}}$が成り立つならそのときに限り、その関数$\left| \frac{f}{g} \right|$がその集合$A \cap U$上で有界であるような$U \subseteq U_{0}\left( \mathbf{a},\varepsilon \right)$なる集合$U$が存在する。これにより、$\forall\mathbf{x} \in A \cap U$に対し、$\left| \frac{f}{g}\left( \mathbf{x} \right) \right| < R$が成り立つような実数たち$R$が存在する。$g \in o_{h,\mathbf{a}}$が成り立つならそのときに限り、$\lim_{\scriptsize \begin{matrix}
\mathbf{x} \rightarrow \mathbf{a} \\
\mathbf{x} \neq \mathbf{a} \\
\end{matrix}}{\frac{g}{h}\left( \mathbf{x} \right)} = 0$が成り立つ。ここで、$\lim_{\scriptsize \begin{matrix}
\mathbf{x} \rightarrow \mathbf{a} \\
\mathbf{x} \neq \mathbf{a} \\
\end{matrix}}{\frac{g}{h}\left( \mathbf{x} \right)} \in \mathbb{R}$が成り立っているので、その関数$\frac{g}{h}:A \rightarrow \mathbb{R}$はその点$\mathbf{a}$のその集合$A \setminus \left\{ \mathbf{a} \right\}$におけるある$\delta$近傍$U\left( \mathbf{a},\delta \right) \cap \left( A \setminus \left\{ \mathbf{a} \right\} \right)$で有界となるのであった。ここで、その点$\mathbf{a}$の除外$\varepsilon$近傍$U_{0}\left( \mathbf{a},\varepsilon \right)$を用いた次式が成り立つので、
\begin{align*}
U\left( \mathbf{a},\delta \right) \cap \left( A \setminus \left\{ \mathbf{a} \right\} \right) = A \cap \left( U\left( \mathbf{a},\delta \right) \setminus \left\{ \mathbf{a} \right\} \right) = A \cap U_{0}\left( \mathbf{a},\varepsilon \right)
\end{align*}
その関数$\frac{g}{h}$はその集合$A \cap U_{0}\left( \mathbf{a},\varepsilon \right)$上で有界でありその関数$\left| \frac{g}{h} \right|$も同様である。したがって、$\forall\mathbf{x} \in A \cap U_{0}\left( \mathbf{a} \right)$に対し、$\left| \frac{g}{h}\left( \mathbf{x} \right) \right| < S$が成り立つような実数$S$が存在する。したがって、次のようになる。
\begin{align*}
\left| \frac{f}{g}\left( \mathbf{x} \right) \right| < R \land \left| \frac{g}{h}\left( \mathbf{x} \right) \right| < S \Rightarrow \left| \frac{f}{g}\left( \mathbf{x} \right) \right|\left| \frac{g}{h}\left( \mathbf{x} \right) \right| = \left| \frac{f}{h}\left( \mathbf{x} \right) \right| < RS
\end{align*}
これにより、その関数$\left| \frac{f}{h} \right|$はその集合$A \cap U$上で有界であるので、$f \in O_{h,\mathbf{a}}$が成り立つ。\par
最後に、次のようになる。
\begin{align*}
f \in o_{g,\mathbf{a}} &\Rightarrow f \in O_{g,\mathbf{a}} \Leftrightarrow f \notin o_{g,\mathbf{a}} \vee f \in O_{g,\mathbf{a}}\\
&\Rightarrow f \notin o_{g,\mathbf{a}} \vee f \in O_{g,\mathbf{a}} \vee g \in o_{h,\mathbf{a}}\\
&\Leftrightarrow \left( f \notin o_{g,\mathbf{a}} \vee f \in O_{g,\mathbf{a}} \vee g \notin o_{h,\mathbf{a}} \right) \land \left( f \notin o_{g,\mathbf{a}} \vee g \in o_{h,\mathbf{a}} \vee g \notin o_{h,\mathbf{a}} \right)\\
&\Leftrightarrow f \notin o_{g,\mathbf{a}} \vee g \notin o_{h,\mathbf{a}} \vee \left( f \in O_{g,\mathbf{a}} \land g \in o_{h,\mathbf{a}} \right)\\
&\Leftrightarrow f \in o_{g,\mathbf{a}} \land g \in o_{h,\mathbf{a}} \Rightarrow f \in O_{g,\mathbf{a}} \land g \in o_{h,\mathbf{a}}
\end{align*}
ここで、$f \in O_{g,\mathbf{a}} \land g \in o_{h,\mathbf{a}} \Rightarrow f \in O_{h,\mathbf{a}}$が成り立つので、次式が得られる。
\begin{align*}
f \in o_{g,\mathbf{a}} \land g \in o_{h,\mathbf{a}} \Rightarrow f \in o_{h,\mathbf{a}}
\end{align*}
\end{proof}
\begin{thm}\label{4.2.4.2}
$\mathbf{a} \in \mathrm{cl}A$なる点$\mathbf{a}$と$A \subseteq \mathbb{R}_{\infty}^{n}$なる関数たち$f:A \rightarrow \mathbb{R}$、$g:A \rightarrow \mathbb{R}$、$h:A \rightarrow \mathbb{R}$、$F:A \rightarrow \mathbb{R}$、$G:A \rightarrow \mathbb{R}$を考えると、次式たちが成り立つ。
\begin{align*}
f \in o_{h,\mathbf{a}} \land g \in o_{h,\mathbf{a}} &\Rightarrow f \pm g \in o_{h,\mathbf{a}}\\
f \in o_{F,\mathbf{a}} \land g \in O_{G,\mathbf{a}} &\Rightarrow fg \in o_{FG,\mathbf{a}}
\end{align*}
拡張$n$次元数空間$\mathbb{R}_{\infty}^{n}$のかわりに補完数直線${}^{*}\mathbb{R}$でおきかえても同様にして示される。
\end{thm}
\begin{proof}
$A \subseteq \mathbb{R}_{\infty}^{n}$なる関数たち$f:A \rightarrow \mathbb{R}$、$g:A \rightarrow \mathbb{R}$、$h:A \rightarrow \mathbb{R}$、$F:A \rightarrow \mathbb{R}$、$G:A \rightarrow \mathbb{R}$を考え$\forall\mathbf{x} \in U_{0}\left( \mathbf{a},\varepsilon \right)$に対し、$h\left( \mathbf{x} \right) \neq 0$、$F\left( \mathbf{x} \right) \neq 0$、$G\left( \mathbf{x} \right) \neq 0$が成り立つようなその集合$A$における$\mathbf{a} \in \mathrm{cl}A$なる点$\mathbf{a}$の除外$\varepsilon$近傍$U_{0}\left( \mathbf{a},\varepsilon \right)$が存在するとする。$f \in o_{h,\mathbf{a}}$かつ$g \in o_{h,\mathbf{a}}$が成り立つならそのときに限り、$\lim_{\scriptsize \begin{matrix}
\mathbf{x} \rightarrow \mathbf{a} \\
\mathbf{x} \neq \mathbf{a} \\
\end{matrix}}{\frac{f}{h}\left( \mathbf{x} \right)} = 0$かつ$\lim_{\scriptsize \begin{matrix}
\mathbf{x} \rightarrow \mathbf{a} \\
\mathbf{x} \neq \mathbf{a} \\
\end{matrix}}{\frac{g}{h}\left( \mathbf{x} \right)} = 0$が成り立つ。したがって、$\forall\varepsilon \in \mathbb{R}^{+}$に対し、$\left| \frac{f}{h}\left( \mathbf{x} \right) \right| < \varepsilon$かつ$\left| \frac{g}{h}\left( \mathbf{x} \right) \right| < \varepsilon$が成り立つ。したがって、三角不等式より次のようになる。
\begin{align*}
\left| \frac{(f \pm g)}{h}\left( \mathbf{x} \right) \right| = \left| \frac{f}{h}\left( \mathbf{x} \right) \pm \frac{g}{h}\left( \mathbf{x} \right) \right| \leq \left| \frac{f}{h}\left( \mathbf{x} \right) \right| + \left| \pm \frac{g}{h}\left( \mathbf{x} \right) \right| = \left| \frac{f}{h}\left( \mathbf{x} \right) \right| + \left| \frac{g}{h}\left( \mathbf{x} \right) \right| < 2\varepsilon
\end{align*}
これにより、$\lim_{\scriptsize \begin{matrix}
\mathbf{x} \rightarrow \mathbf{a} \\
\mathbf{x} \neq \mathbf{a} \\
\end{matrix}}{\frac{(f \pm g)}{h}\left( \mathbf{x} \right)} = 0$が成り立つので、よって、$f \pm g \in o_{h,\mathbf{a}}$が成り立つ。\par
$f \in o_{F,\mathbf{a}}$が成り立つならそのときに限り、$\lim_{\scriptsize \begin{matrix}
\mathbf{x} \rightarrow \mathbf{a} \\
\mathbf{x} \neq \mathbf{a} \\
\end{matrix}}{\frac{f}{F}\left( \mathbf{x} \right)} = 0$が成り立つ。したがって、$\mathbf{x} \rightarrow \mathbf{a},\mathbf{x} \neq \mathbf{a}$のとき、$\forall\varepsilon \in \mathbb{R}^{+}$に対し、$\left| \frac{f}{F}\left( \mathbf{x} \right) \right| < \varepsilon$が成り立つ。また、$g \in O_{G,\mathbf{a}}$が成り立つならそのときに限り、その関数$\frac{g}{G}$はその集合$A \cap U_{0}\left( \mathbf{a},\varepsilon \right)$上で有界でありその関数$\left| \frac{g}{G} \right|$も同様である。したがって、$\forall\mathbf{x} \in A \cap U_{0}\left( \mathbf{a},\varepsilon \right)$に対し、$\left| \frac{g}{G}\left( \mathbf{x} \right) \right| < R$が成り立つような実数$R$が存在する。なお、その実数$R$は正である。したがって、次のようになる。
\begin{align*}
\left| \frac{f}{F}\left( \mathbf{x} \right) \right|\left| \frac{g}{G}\left( \mathbf{x} \right) \right| = \left| \frac{fg}{FG}\left( \mathbf{x} \right) \right| < \varepsilon R
\end{align*}
この正の実数$\varepsilon R$は任意であり、したがって、$\mathbf{x} \rightarrow \mathbf{a},\mathbf{x} \neq \mathbf{a}$のとき、$\forall\varepsilon' \in \mathbb{R}^{+}$に対し、$\left| \frac{fg}{FG}\left( \mathbf{x} \right) \right| < \varepsilon'$が成り立つ。したがって、$\lim_{\scriptsize \begin{matrix}
\mathbf{x} \rightarrow \mathbf{a} \\
\mathbf{x} \neq \mathbf{a} \\
\end{matrix}}{\frac{fg}{FG}\left( \mathbf{x} \right)} = 0$が成り立つ。よって、$fg \in o_{FG(x),\mathbf{a}}$が成り立つ。
\end{proof}
%\hypertarget{ux305dux306eux95a2ux6570fux306fux305dux306eux70b9mathbfaux306bux304aux3044ux3066ux305dux306eux95a2ux6570gux3068ux540cux6b21ux3067ux3042ux308b}{%
\subsubsection{その関数$f$はその点$\mathbf{a}$においてその関数$g$と同次である}%\label{ux305dux306eux95a2ux6570fux306fux305dux306eux70b9mathbfaux306bux304aux3044ux3066ux305dux306eux95a2ux6570gux3068ux540cux6b21ux3067ux3042ux308b}}
\begin{dfn}
$A \subseteq \mathbb{R}_{\infty}^{n}$なる関数たち$f:A \rightarrow \mathbb{R}$、$g:A \rightarrow \mathbb{R}$を考え、$\forall\mathbf{x} \in U_{0}\left( \mathbf{a},\varepsilon \right)$に対し、$g\left( \mathbf{x} \right) \neq 0$が成り立つようなその集合$A$における$\mathbf{a} \in \mathrm{cl}A$なる点$\mathbf{a}$の除外$\varepsilon$近傍$U_{0}\left( \mathbf{a},\varepsilon \right)$が存在するとする。その関数$f$が次式を満たすとき、その関数$f$はその点$\mathbf{a}$においてその関数$g$と同次であるなどといい$f \sim g\ \left( \mathbf{x} \rightarrow \mathbf{a} \right)$、$f \approx g\ \left( \mathbf{x} \rightarrow \mathbf{a} \right)$などと書く。
\begin{align*}
\lim_{\scriptsize \begin{matrix}
\mathbf{x} \rightarrow \mathbf{a} \\
\mathbf{x} \neq \mathbf{a} \\
\end{matrix}}{\frac{f}{g}\left( \mathbf{x} \right)} = 1
\end{align*}
拡張$n$次元数空間$\mathbb{R}_{\infty}^{n}$のかわりに補完数直線${}^{*}\mathbb{R}$でおきかえても同様にして定義される。
\end{dfn}
\begin{thm}\label{4.2.4.3}
$I = [ x,a] \cup [ a,x] \subseteq D(f) \subseteq \mathbb{R}$なる区間$I$で$n$回微分可能な関数$f:D(f) \rightarrow \mathbb{R}$に対し、その関数$f$は$r \in o_{(x - a)^{n},a}$なるある関数$r:D(f) \rightarrow \mathbb{R}$を用いてその区間$I$上で次式のように書かれることができる。
\begin{align*}
f(x) = \sum_{i \in \varLambda_{n} \cup \left\{ 0 \right\}} {\frac{(x - a)^{i}}{i!}\partial^{i}f(a)} + r(x)
\end{align*}
\end{thm}
\begin{proof}
$I=[x,a]\cup[a,x]\subseteq D(f)\subseteq\mathbb{R}$なる区間$$I$$で$n$回微分可能な関数$f:D(f)\rightarrow \mathbb{R} $に対し次式のように$n+1$次剰余項$R_{n+1} (x)$が定義されるとき、
\begin{align*}
R_{n + 1}(x) = f(x) - \sum_{i \in \varLambda_{n} \cup \left\{ 0 \right\}} {\frac{(x - a)^{i}}{i!}\partial^{i}f(a)}
\end{align*}
$\lim_{x \rightarrow a}\frac{R_{n + 1}(x)}{(x - a)^{n}} = 0$が成り立つのであった。これにより、関数$r$を次式のように定義されると、
\begin{align*}
r:D(f) \rightarrow \mathbb{R}:x \mapsto R_{n + 1}(x)
\end{align*}
$\lim_{x \rightarrow a}\frac{r(x)}{(x - a)^{n}} = 0$が成り立つので、$r \in o_{(x - a)^{n},a}$が成り立つ。したがって、その関数$f$は$r \in o_{(x - a)^{n},a}$なるある関数$r$を用いてその区間$I$上で次式のように書かれることができる。
\begin{align*}
f(x) = \sum_{i \in \varLambda_{n} \cup \left\{ 0 \right\}} {\frac{(x - a)^{i}}{i!}\partial^{i}f(a)} + r(x)
\end{align*}
\end{proof}
\begin{thm}\label{4.2.4.4}
$A \subseteq \mathbb{R}$なる関数たち$f:A \rightarrow \mathbb{R}$、$g:A \rightarrow \mathbb{R}$がその集合$A$において実数$a$の$\varepsilon$近傍$U(a,\varepsilon)$で$n$回微分可能であり$k \leq n$かつ$l \leq n$なる自然数たち$k$、$l$を用いて次式たちを満たすとき、
\begin{align*}
\forall i &\in \varLambda_{k - 1} \cup \left\{ 0 \right\}\left[ \partial^{i}f(a) = 0 \right],\ \ \partial^{k}f(a) \neq 0,\\
\forall i &\in \varLambda_{l - 1} \cup \left\{ 0 \right\}\left[ \partial^{i}g(a) = 0 \right],\ \ \partial^{l}g(a) \neq 0
\end{align*}
次式たちが成り立つ。
\begin{align*}
f(x) &\approx \frac{(x - a)^{k}}{k!}\partial^{k}f(a)\ (x \rightarrow a)\\
\lim_{\scriptsize \begin{matrix}
x \rightarrow a \\
x \neq a \\
\end{matrix}}{\frac{f}{g}(x)} &= \left\{ \begin{matrix}
0 & \mathrm{if} & k > l \\
\frac{\partial^{k}f}{\partial^{l}g}(a) & \mathrm{if} & k = l \\
\end{matrix} \right.\  \land \lim_{\scriptsize \begin{matrix}
x \rightarrow a \\
x \neq a \\
\end{matrix}}\left| \frac{f}{g}(x) \right| = \infty\ \mathrm{if}\ k < l
\end{align*}
\end{thm}
\begin{proof}
$A \subseteq \mathbb{R}$なる関数$f:A \rightarrow \mathbb{R}$、$g:A \rightarrow \mathbb{R}$がその集合$A$において実数$a$の$\varepsilon$近傍$U(a,\varepsilon)$で$n$回微分可能であり$k \leq n$かつ$l \leq n$なる自然数たち$k$、$l$を用いて次式たちを満たすとき、
\begin{align*}
\forall i &\in \varLambda_{k - 1} \cup \left\{ 0 \right\}\left[ \partial^{i}f(a) = 0 \right],\ \ \partial^{k}f(a) \neq 0,\\
\forall i &\in \varLambda_{l - 1} \cup \left\{ 0 \right\}\left[ \partial^{i}g(a) = 0 \right],\ \ \partial^{l}g(a) \neq 0
\end{align*}
その関数$f$は$r \in o_{(x - a)^{k},a}$なるある関数$r:A \rightarrow \mathbb{R}$を用いてその区間$I$上で次式のように書かれることができる。
\begin{align*}
f(x) &= \sum_{i \in \varLambda_{k} \cup \left\{ 0 \right\}} {\frac{(x - a)^{i}}{i!}\partial^{i}f(a)} + r(x)\\
f(x) &= \sum_{i \in \varLambda_{n} \cup \left\{ 0 \right\}} {\frac{(x - a)^{i}}{i!}\partial^{i}f(a)} + r(x)\\
&= \sum_{i \in \varLambda_{k - 1} \cup \left\{ 0 \right\}} {\frac{(x - a)^{i}}{i!}\partial^{i}f(a)} + \frac{(x - a)^{k}}{k!}\partial^{k}f(a) + r(x)\\
&= \sum_{i \in \varLambda_{k - 1} \cup \left\{ 0 \right\}} {\frac{(x - a)^{i}}{i!} \cdot 0} + \frac{(x - a)^{k}}{k!}\partial^{k}f(a) + r(x) \\
&= \frac{(x - a)^{k}}{k!}\partial^{k}f(a) + r(x)
\end{align*}
同様にして、その関数$g$は$s \in o_{(x - a)^{l},a}$なるある関数$s:A \rightarrow \mathbb{R}$を用いてその区間$I$上で次式のように書かれることができる。
\begin{align*}
g(x) &= \frac{(x - a)^{l}}{l!}\partial^{l}g(a) + s(x)
\lim_{\scriptsize \begin{matrix}
x \rightarrow a \\
x \neq a \\
\end{matrix}}\frac{f(x)}{\frac{(x - a)^{k}}{k!}\partial^{k}f(a)} \\
&= \lim_{\scriptsize \begin{matrix}
x \rightarrow a \\
x \neq a \\
\end{matrix}}\frac{\frac{(x - a)^{k}}{k!}\partial^{k}f(a) + r_{1}(x)}{\frac{(x - a)^{k}}{k!}\partial^{k}f(a)}\\
&= \lim_{\scriptsize \begin{matrix}
x \rightarrow a \\
x \neq a \\
\end{matrix}}\left( \frac{\frac{(x - a)^{k}}{k!}\partial^{k}f(a)}{\frac{(x - a)^{k}}{k!}\partial^{k}f(a)} + \frac{r_{1}(x)}{\frac{(x - a)^{k}}{k!}\partial^{k}f(a)} \right)\\
&= \lim_{\scriptsize \begin{matrix}
x \rightarrow a \\
x \neq a \\
\end{matrix}}\left( 1 + \frac{k!}{\partial^{k}f(a)}\frac{r_{1}(x)}{(x - a)^{k}} \right) = 1 + \frac{k!}{\partial^{k}f(a)}\lim_{\scriptsize \begin{matrix}
x \rightarrow a \\
x \neq a \\
\end{matrix}}\frac{r_{1}(x)}{(x - a)^{k}}
\end{align*}
ここで、$r \in o_{(x - a)^{k},a}$が成り立つので、
\begin{align*}
\lim_{\scriptsize \begin{matrix}
x \rightarrow a \\
x \neq a \\
\end{matrix}}\frac{f(x)}{\frac{(x - a)^{k}}{k!}\partial^{k}f(a)} = 1 + \frac{k!}{\partial^{k}f(a)} \cdot 0 = 1
\end{align*}
よって、次式が成り立つ。
\begin{align*}
f(x) \approx \frac{(x - a)^{k}}{k!}\partial^{k}f(a)\ (x \rightarrow a)
\end{align*}\par
また、次のようになることから、
\begin{align*}
\lim_{\scriptsize \begin{matrix}
x \rightarrow a \\
x \neq a \\
\end{matrix}}{\frac{f}{g}(x)} &= \lim_{\scriptsize \begin{matrix}
x \rightarrow a \\
x \neq a \\
\end{matrix}}\frac{\frac{(x - a)^{k}}{k!}\partial^{k}f(a) + r(x)}{\frac{(x - a)^{l}}{l!}\partial^{l}g(a) + s(x)} = \lim_{\scriptsize \begin{matrix}
x \rightarrow a \\
x \neq a \\
\end{matrix}}\frac{\frac{1}{k!}\partial^{k}f(a) + \frac{r(x)}{(x - a)^{k}}}{\frac{1}{(x - a)^{k - l}}\left( \frac{1}{l!}\partial^{l}g(a) + \frac{s(x)}{(x - a)^{l}} \right)}\\
&= \lim_{\scriptsize \begin{matrix}
x \rightarrow a \\
x \neq a \\
\end{matrix}}\left( (x - a)^{k - l}\frac{\frac{1}{k!}\partial^{k}f(a) + \frac{r(x)}{(x - a)^{k}}}{\frac{1}{l!}\partial^{l}g(a) + \frac{s(x)}{(x - a)^{l}}} \right)\\
&= \lim_{\scriptsize \begin{matrix}
x \rightarrow a \\
x \neq a \\
\end{matrix}}(x - a)^{k - l}\frac{\frac{1}{k!}\partial^{k}f(a) + \lim_{\scriptsize \begin{matrix}
x \rightarrow a \\
x \neq a \\
\end{matrix}}\frac{r(x)}{(x - a)^{k}}}{\frac{1}{l!}\partial^{l}g(a) + \lim_{\scriptsize \begin{matrix}
x \rightarrow a \\
x \neq a \\
\end{matrix}}\frac{s(x)}{(x - a)^{l}}}
\end{align*}
ここで、$r \in o_{(x - a)^{k},a}$、$s \in o_{(x - a)^{l},a}$が成り立つので、
\begin{align*}
\lim_{\scriptsize \begin{matrix}
x \rightarrow a \\
x \neq a \\
\end{matrix}}{\frac{f}{g}(x)} = \lim_{\scriptsize \begin{matrix}
x \rightarrow a \\
x \neq a \\
\end{matrix}}(x - a)^{k - l}\frac{\frac{1}{k!}\partial^{k}f(a) + 0}{\frac{1}{l!}\partial^{l}g(a) + 0} = \lim_{\scriptsize \begin{matrix}
x \rightarrow a \\
x \neq a \\
\end{matrix}}(x - a)^{k - l}\frac{\frac{1}{k!}\partial^{k}f}{\frac{1}{l!}\partial^{l}g}(a)
\end{align*}
$k > l$のとき、$k - l > 0$より次のようになり、
\begin{align*}
\lim_{\scriptsize \begin{matrix}
x \rightarrow a \\
x \neq a \\
\end{matrix}}{\frac{f}{g}(x)} = 0 \cdot \frac{\frac{1}{k!}\partial^{k}f}{\frac{1}{l!}\partial^{l}g}(a) = 0
\end{align*}
$k = l$のとき、次のようになる。
\begin{align*}
\lim_{\scriptsize \begin{matrix}
x \rightarrow a \\
x \neq a \\
\end{matrix}}{\frac{f}{g}(x)} = 1 \cdot \frac{\frac{1}{k!}\partial^{k}f}{\frac{1}{l!}\partial^{l}g}(a) = \frac{\frac{1}{k!}\partial^{k}f}{\frac{1}{l!}\partial^{l}g}(a) = \frac{\partial^{k}f}{\partial^{l}g}(a)
\end{align*}
よって、次式が成り立つ。
\begin{align*}
\lim_{\scriptsize \begin{matrix}
x \rightarrow a \\
x \neq a \\
\end{matrix}}{\frac{f}{g}(x)} = \left\{ \begin{matrix}
0 & \mathrm{if} & k > l \\
\frac{\partial^{k}f}{\partial^{l}g}(a) & \mathrm{if} & k = l \\
\end{matrix} \right.\ 
\end{align*}\par
また、$k < l$のとき、上記の議論より明らかに、次式が成り立つ。
\begin{align*}
\lim_{\scriptsize \begin{matrix}
x \rightarrow a \\
x \neq a \\
\end{matrix}}\left| \frac{f}{g}(x) \right| = \lim_{\scriptsize \begin{matrix}
x \rightarrow a \\
x \neq a \\
\end{matrix}}\left| \frac{1}{\frac{g}{f}(x)} \right| = \lim_{\scriptsize \begin{matrix}
\frac{g}{f}(x) \rightarrow 0 \\
\end{matrix}}\left| \frac{1}{\frac{g}{f}(x)} \right| = \infty
\end{align*}
\end{proof}
%\hypertarget{ux6f38ux8fd1ux5c55ux958b}{%
\subsubsection{漸近展開}%\label{ux6f38ux8fd1ux5c55ux958b}}
\begin{dfn}
$D(f) \subseteq \mathbb{R}$なる関数$f:D(f) \rightarrow \mathbb{R}$が$a \in{}^{*}\mathbb{R}なる点a$における無限小であるまたは無限大であるとき、$\forall k \in \varLambda_{n}$に対し、次のことを満たすような関数たち$c_{k}g_{k}$全体からなる関数族を$\mathcal{E}$とおき和$\sum_{k \in \varLambda_{n}} {c_{k}g_{k}(x)}$のことをその関数族$\mathcal{E}$に関するその関数$f$の$n$項の漸近展開などという。
\begin{itemize}
\item
  $\forall k \in \varLambda_{n}$に対し、$c_{k} \in \mathbb{R} \setminus \left\{ 0 \right\}$が成り立つ。
\item
  $\forall k \in \varLambda_{n}$に対し、$g_{k}:U_{0}(a,\varepsilon) \rightarrow \mathbb{R}$が成り立つようなその点$a$の除外$\varepsilon$近傍$U_{0}(a,\varepsilon)$が存在する。
\item
  $\forall k \in \varLambda_{n - 1}$に対し、関数たち$g_{k}$、$g_{k + 1}$は次式を満たす。
\begin{align*}
g_{k + 1} \in o_{g_{k},a}
\end{align*}
\item
  $r \in o_{g_{n},a}$なる関数$r$を用いた次式が成り立つ。
\begin{align*}
f = \sum_{k \in \varLambda_{n}} {c_{k}g_{k}} + r
\end{align*}
\end{itemize}
このときのその関数$c_{1}g_{1}$をこの漸近展開の主要部といいその関数$r$をこの漸近展開の剰余という。
\end{dfn}
\begin{thm}\label{4.2.4.5}
$I = [ x,a] \cup [ a,x] \subseteq D(f) \subseteq \mathbb{R}$なる区間$I$で$n$回微分可能であり$\partial^{n}f(a) \neq 0$なる関数$f:D(f) \rightarrow \mathbb{R}$に対し、その関数$f$は$r \in o_{(x - a)^{n},a}$なるある関数$r:D(f) \rightarrow \mathbb{R}$を用いてその区間$I$上で次式のように書かれることができるのであった。
\begin{align*}
f(x) = \sum_{i \in \varLambda_{n} \cup \left\{ 0 \right\}} {\frac{(x - a)^{i}}{i!}\partial^{i}f(a)} + r(x)
\end{align*}
このとき、これは集合$\left\{ i\in \varLambda_{n} \cup \{0\} \middle| \partial_{i} f(a)\neq 0\right\} $を$D$とおくと、次のことを満たすような写像$p:\varLambda_{\# D}\rightarrow D$を用いて
\begin{itemize}
\item
  $p(1) = \min D$が成り立つ。
\item
  $\forall n \in \varLambda_{\#D} \setminus \left\{ \#D \right\}$に対し、$p(n + 1) = \min\left( D \setminus V\left( p|\varLambda_{n} \right) \right)$が成り立つ。
\end{itemize}
$j \in \varLambda_{\#D}$なる実数$c_{j}$と関数$g_{j}$が次のようにおかれたその関数$fの\#D$項の漸近展開となる。
\begin{align*}
c_{j} = \frac{1}{p(j)!}\partial^{p(j)}f(a),\ \ g_{j}:U_{0}(a) \rightarrow \mathbb{R};x \mapsto (x - a)^{p(j)}
\end{align*}
\end{thm}
\begin{proof}
$I = [ x,a] \cup [ a,x] \subseteq D(f) \subseteq \mathbb{R}$なる区間$I$で$n$回微分可能であり$\partial^{n}f(a) \neq 0$なる関数$f:D(f) \rightarrow \mathbb{R}$に対し、その関数$f$は$r \in o_{(x - a)^{n},a}$なるある関数$r:D(f) \rightarrow \mathbb{R}$を用いてその区間$I$上で次式のように書かれることができるのであった。
\begin{align*}
f(x) = \sum_{i \in \varLambda_{n} \cup \left\{ 0 \right\}} {\frac{(x - a)^{i}}{i!}\partial^{i}f(a)} + r(x)
\end{align*}
\begin{align*}
D = \left\{ i \in \varLambda_{n} \cup \left\{ 0 \right\} \middle| \partial^{i}f(a) \neq 0 \right\}
\end{align*}
次のようになる。
\begin{align*}
f(x) &= \sum_{i \in D} {\frac{(x - a)^{i}}{i!}\partial^{i}f(a)} + \sum_{i \in \left( \varLambda_{n} \cup \left\{ 0 \right\} \right) \setminus D} {\frac{(x - a)^{i}}{i!}\partial^{i}f(a)} + r(x)\\
&= \sum_{i \in D} {\frac{(x - a)^{i}}{i!}\partial^{i}f(a)} + \sum_{i \in \left( \varLambda_{n} \cup \left\{ 0 \right\} \right) \setminus D} {\frac{(x - a)^{i}}{i!} \cdot 0} + r(x)\\
&= \sum_{i \in D} {\frac{(x - a)^{i}}{i!}\partial^{i}f(a)} + r(x)
\end{align*}
ここで、次の条件たちを満たすような写像$p:\varLambda_{\#D} \rightarrow D$が考えられるとして、
\begin{itemize}
\item
  $p(1) = \min D$が成り立つ。
\item
  $\forall n \in \varLambda_{\#D} \setminus \left\{ \#D \right\}$に対し、$p(n + 1) = \min\left( D \setminus V\left( p|\varLambda_{n} \right) \right)$が成り立つ。
\end{itemize}
$i' < j'$かつ$p\left( i' \right) = p\left( j' \right)$が成り立つと仮定しよう。\par
$j' - i' = 1$のとき、定義より$p\left( j' \right) = p\left( i' + 1 \right) = \min\left( D \setminus V\left( p|\varLambda_{i'} \right) \right)$が成り立つので、$p\left( j' \right) \in D \setminus V\left( p|\varLambda_{i'} \right)$が成り立つ。\par
$j' - i' = k$のとき、$p\left( i' + k \right) \in D \setminus V\left( p|\varLambda_{i'} \right)$が成り立つと仮定しよう。$j' - i' = k + 1$のとき、定義より$p\left( i' + k + 1 \right) = \min\left( D \setminus V\left( p|\varLambda_{i' + k} \right) \right)$が成り立つので、$p\left( i' + k + 1 \right) \in D \setminus V\left( p|\varLambda_{i' + k} \right)$が成り立ち、したがって、次式が成り立つ。
\begin{align*}
p\left( i' + k + 1 \right) \in D \land p\left( i' + k + 1 \right) \notin \varLambda_{i' + k}
\end{align*}
ここで、明らかに、$\varLambda_{i'} \subseteq \varLambda_{i' + k}$が成り立つので、次式が成り立つ。
\begin{align*}
p\left( i' + k + 1 \right) \in D \land p\left( i' + k + 1 \right) \notin \varLambda_{i'}
\end{align*}
これにより、$p\left( i' + k + 1 \right) \in D \setminus V\left( p|\varLambda_{i'} \right)$が成り立つ。\par
以上より数学的帰納法によって、$\forall j' - i' \in \mathbb{N}$に対し、$p\left( j' \right) \in D \setminus V\left( p|\varLambda_{i'} \right)$が成り立つ、即ち、$\forall i',j' \in \mathbb{N}$に対し、$i' < j'$のとき、$p\left( j' \right) \in D \setminus V\left( p|\varLambda_{i'} \right)$が成り立つ。仮定より$p\left( i' \right) = p\left( j' \right)$が成り立つので、$p\left( i' \right) \in D \setminus V\left( p|\varLambda_{i'} \right)$が成り立つが、$i' \in \varLambda_{i'}$より$p\left( i' \right) = p|\varLambda_{i'}\left( i' \right) \in V\left( p|\varLambda_{i'} \right)$が成り立つことに矛盾する。したがって、$i' < j' \Rightarrow p\left( i' \right) \neq p\left( j' \right)$が成り立つ。$j' < i'$のときも同様にして示される。これにより、その関数$p$は単射である。\par
一方で、明らかに、$n = 1$のとき、$\#{V\left( p|\varLambda_{1} \right)} = \#\left\{ p(1) \right\} = 1$が成り立つ。\par
ここで、$n = k$のとき、$\#{V\left( p|\varLambda_{k} \right)} = k$が成り立つと仮定しよう。$n = k + 1$のとき、次のようになるので、
\begin{align*}
p(j) \in V\left( p|\varLambda_{k + 1} \right) &\Leftrightarrow j \in \varLambda_{k + 1}\\
&\Leftrightarrow j \in \varLambda_{k} \vee j = k + 1\\
&\Leftrightarrow p(j) \in V\left( p|\varLambda_{k + 1} \right) \vee p(k + 1) \in \left\{ p(k + 1) \right\}
\end{align*}
$V\left( p|\varLambda_{k + 1} \right) = V\left( p|\varLambda_{k} \right) \cup \left\{ p(k + 1) \right\}$が成り立つ。ここで、$p(k + 1) \in V\left( p|\varLambda_{k} \right)$が成り立つと仮定すると、$p(k + 1) = p(j)$なる自然数$j$がその添数集合$\varLambda_{k}$に存在しその自然数$j$はその自然数$k + 1$とは異なることになるが、これはその写像$p$が単射であることに矛盾する。したがって、$V\left( p|\varLambda_{k + 1} \right) = V\left( p|\varLambda_{k} \right) \sqcup \left\{ p(k + 1) \right\}$が成り立つ。\par
これにより、次式が成り立つ。
\begin{align*}
\#{V\left( p|\varLambda_{k + 1} \right)} = \#\left( V\left( p|\varLambda_{k} \right) \sqcup \left\{ p(k + 1) \right\} \right) = \#{V\left( p|\varLambda_{k} \right)} + \#\left\{ p(k + 1) \right\}
\end{align*}
仮定より$\#{V\left( p|\varLambda_{k} \right)} = k$が成り立つので、
\begin{align*}
\#{V\left( p|\varLambda_{k + 1} \right)} = k + 1
\end{align*}
以上より数学的帰納法によって、$\forall n \in \varLambda_{\#D}$に対し、$\#{V\left( p|\varLambda_{n} \right)} = n$が成り立つ。特に、$\#{V\left( p|\varLambda_{\#D} \right)} = \#{V(p)} = \#D$が成り立つ。これにより、2つの集合たち$V(p)$、$D$との間に全単射であるような写像が存在する。\par
ここで、$i' \in D \setminus V(p)$なる自然数$i'$が存在すると仮定しよう。このとき、次式が成り立つ。
\begin{align*}
\#{D \setminus V(p)} \neq 0
\end{align*}
$D = V(p) \sqcup D \setminus V(p)$が成り立つので、次式が成り立つ。
\begin{align*}
\#D = \#{V(p)} + \#{D \setminus V(p)}
\end{align*}
ここで、$\#{D \setminus V(p)} \neq 0$が成り立つので、次式が成り立つが、
\begin{align*}
\#D = \#{V(p)} + \#{D \setminus V(p)} \neq \#{V(p)}
\end{align*}
これは$\#{V(p)} = \#D$が成り立つことに矛盾する。したがって、$i' \in D \setminus V(p)$なる自然数$i'$が存在せずその写像$p$は全射である。\par
以上の議論よりその写像$p$は全単射となりその写像$p$の逆写像$p^{- 1}$が存在する。これにより、$i \in D \Leftrightarrow p(j) \in D \Leftrightarrow j \in \varLambda_{\#D}$が成り立ち次式のようになる。
\begin{align*}
f(x) = \sum_{j \in \varLambda_{\#D}} {\frac{(x - a)^{p(j)}}{p(j)!}\partial^{p(j)}f(a)} + r(x)
\end{align*}
ここで、その集合$D$の定義より$\forall j \in \varLambda_{\#D}$に対し、$\frac{1}{p(j)!}\partial^{p(j)}f(a) \neq 0$が成り立つ。以下、実数$\frac{1}{p(j)!}\partial^{p(j)}f(a)$を$c_{j}$とおく。また、明らかに、$\forall k \in \varLambda_{n}$に対し、次式が成り立つようなその点$\mathfrak{a}$の除外$\varepsilon$近傍$U_{0}\left( \mathfrak{a} \right)$が存在する。
\begin{align*}
g_{j}:U_{0}(a) \rightarrow \mathbb{R};x \mapsto (x - a)^{p(j)}
\end{align*}\par
ここで、$i' < j'$かつ$p\left( i' \right) > p\left( j' \right)$が成り立つような自然数たち$i'$、$j'$が存在すると仮定しよう。$i' = 1$のとき、定義より$p(1) = \min D$が成り立つので、$\forall i \in D$に対し、$p(1) \leq i$が成り立つ。ここで、その写像$p$の定義より$p\left( j' \right) \in D$が成り立つのであった。これにより、$p\left( j' \right) \in D$かつ$p\left( j' \right) < p\left( i' \right)$が成り立つことになるが、これは、$\forall i \in D$に対し、$p\left( i' \right) = p(1) \leq i$が成り立つことに矛盾する。$2 \leq i'$のとき、定義より$p\left( i' \right) = p\left( i' - 1 + 1 \right) = \min\left( D \setminus V\left( p|\varLambda_{i' - 1} \right) \right)$が成り立つので、$\forall i \in D \setminus V\left( p|\varLambda_{i' - 1} \right)$に対し、$p\left( i' \right) \leq i$が成り立つ。ここで、上記の議論より$i' - 1 < i' < j'$のとき、$p\left( j' \right) \in D \setminus V\left( p|\varLambda_{i' - 1} \right)$が成り立つのであった。これにより、$p\left( j' \right) \in D \setminus V\left( p|\varLambda_{i' - 1} \right)$かつ$p\left( j' \right) < p\left( i' \right)$が成り立つことになるが、これは$\forall i \in D \setminus V\left( p|\varLambda_{i' - 1} \right)$に対し、$p\left( i' \right) \leq i$が成り立つことに矛盾する。\par
以上の議論とその写像$p$が単射であることにより、$\forall i',j' \in \mathbb{N}$に対し、$i' < j' \Rightarrow p\left( i' \right) < p\left( j' \right)$が成り立つ。これにより、$\forall j \in \varLambda_{\#D}$に対し、$p(j) < p(j + 1)$が成り立つので、次式のようになる。
\begin{align*}
\lim_{\scriptsize \begin{matrix}
x \rightarrow a \\
x \neq a \\
\end{matrix}}\frac{(x - a)^{p(j + 1)}}{(x - a)^{p(j)}} = \lim_{\scriptsize \begin{matrix}
x \rightarrow a \\
x \neq a \\
\end{matrix}}(x - a)^{p(j + 1) - p(j)} = 0
\end{align*}
したがって、$g_{j + 1} \in o_{g_{j},a}$が成り立つ。\par
また、仮定より$\partial^{n}f(a) \neq 0$が成り立つので、$n \in D$が成り立ち$\#D = p^{- 1}(n)$が成り立つので、次式のようになる。
\begin{align*}
0 = \lim_{\scriptsize \begin{matrix}
x \rightarrow a \\
x \neq a \\
\end{matrix}}\frac{r(x)}{(x - a)^{n}} = \lim_{\scriptsize \begin{matrix}
x \rightarrow a \\
x \neq a \\
\end{matrix}}\frac{r(x)}{(x - a)^{p\left( \#D \right)}}
\end{align*}
以上より、その和$\sum_{i \in \varLambda_{n} \cup \left\{ 0 \right\}} {\frac{(x - a)^{i}}{i!}\partial^{i}f(a)}$は$j \in \varLambda_{\#D}$なる実数$c_{j}$と関数$g_{j}$が次のようにおかれたその関数$f$の$\#D$項の漸近展開となる。
\begin{align*}
c_{j} = \frac{1}{p(j)!}\partial^{p(j)}f(a),\ \ g_{j}:U_{0}(a) \rightarrow \mathbb{R};x \mapsto (x - a)^{p(j)}
\end{align*}
\end{proof}
\begin{thm}\label{4.2.4.6}
上記の議論でのその関数族$\mathcal{E}$に関するその関数$f$の漸近展開が存在するなら、その関数族$\mathcal{E}$に$f \approx g\ (x \rightarrow a)$が成り立つような関数$g$が存在しその漸近展開の主要部$c_{1}g_{1}$はこれを満たす。
\end{thm}
\begin{proof}
$D(f) \subseteq \mathbb{R}$なる関数$f:D(f) \rightarrow \mathbb{R}$が$a \in{}^{*}\mathbb{R}$なる点$a$における無限小であるまたは無限大でありその関数族$\mathcal{E}$に関するその関数$f$の$n$項の漸近展開$\sum_{k \in \varLambda_{n}} {c_{k}g_{k}}$が存在するならそのときに限り、次式が成り立つ。
\begin{align*}
f = \sum_{k \in \varLambda_{n}} {c_{k}g_{k}} + r
\end{align*}
ここで、定義より明らかに、$g_{2} \in o_{g_{1},a}$が成り立つ。\par
ここで、$k = k'$のとき、$g_{k'} \in o_{g_{1},a}$が成り立つと仮定しよう。$k = k' + 1$のとき、定義より$g_{k' + 1} \in o_{g_{k'},a}$が成り立つかつ、仮定より$g_{k'} \in o_{g_{1},a}$が成り立つかつ、定理より$g_{k' + 1} \in o_{g_{k'},a} \land g_{k'} \in o_{g_{1},a} \Rightarrow g_{k' + 1} \in o_{g_{1},a}$が成り立つのであったので、$g_{k' + 1} \in o_{g_{1},a}$が成り立つ。\par
以上より数学的帰納法によって、$\forall k \in \varLambda_{n} \setminus \left\{ 1 \right\}$に対し、$g_{k} \in o_{g_{1},a}$が成り立つ。また、定義よりその集合$o_{g_{1},a}$が存在するので、その点$a$の除外$\varepsilon$近傍$U_{0}(a,\varepsilon)$で$c_{1}g_{1} \neq 0$が成り立つ。したがって、次のようになる。
\begin{align*}
\lim_{x \rightarrow a}\frac{f(x)}{c_{1}g_{1}(x)} = \lim_{x \rightarrow a}\frac{\sum_{k \in \varLambda_{n}} {c_{k}g_{k}(x)} + r(x)}{c_{1}g_{1}(x)} = 1 + \sum_{k \in \varLambda_{n} \setminus \left\{ 1 \right\}} {\frac{c_{k}}{c_{1}}\lim_{x \rightarrow a}\frac{g_{k}(x)}{g_{1}(x)}} + \frac{1}{c_{1}}\lim_{x \rightarrow a}\frac{r(x)}{g_{1}(x)}
\end{align*}
ここで、$r \in o_{g_{n},a}$が成り立つかつ、$\forall k \in \varLambda_{n} \setminus \left\{ 1 \right\}$に対し、$g_{k} \in o_{g_{1},a}$が成り立つかつ、定理より$r \in o_{g_{n},a} \land g_{k}(x) \in o_{g_{1},a} \Rightarrow r \in o_{g_{1},a}$が成り立つのであったので、$r \in o_{g_{1},a}$が成り立つかつ、上記の議論より$\forall k \in \varLambda_{n} \setminus \left\{ 1 \right\}$に対し、$g_{k} \in o_{g_{1},a}$が成り立つので、次式が成り立つ。
\begin{align*}
\lim_{x \rightarrow a}\frac{f(x)}{c_{1}g_{1}(x)} = 1 + \sum_{k \in \varLambda_{n} \setminus \left\{ 1 \right\}} {\frac{c_{k}}{c_{1}} \cdot 0} + \frac{1}{c_{1}} \cdot 0 = 1
\end{align*}
これにより、その関数族$\mathcal{E}$に属する主要部$c_{1}g_{1}$が$f \approx c_{1}g_{1}\ (x \rightarrow a)$を満たす。
\end{proof}
\begin{thm}\label{4.2.4.7}
上記の議論でのその関数族$\mathcal{E}$に関するその関数$f$の漸近展開が存在しその関数族$\mathcal{E}$が次式を満たすとき、その漸近展開の主要部が一意的になる。
\begin{align*}
g\in \mathcal{E,\ \ }f \approx g\ (x \rightarrow a),\ \ \forall x \in U_{0}(a,\varepsilon)\left[ g(x) \neq 0 \right] \Rightarrow f = g
\end{align*}
\end{thm}
\begin{proof}
$D(f) \subseteq \mathbb{R}$なる関数$f:D(f) \rightarrow \mathbb{R}$が$a \in{}^{*}\mathbb{R}$なる点$a$における無限小であるまたは無限大でありその関数族$\mathcal{E}$に関するその関数$f$の$n$項の漸近展開$\sum_{k \in \varLambda_{n}} {c_{k}g_{k}(x)}$が存在しその関数族$\mathcal{E}$が次式を満たすとする。
\begin{align*}
g\in \mathcal{E,\ \ }f \approx g\ (x \rightarrow a),\ \ \forall x \in U_{0}(a,\varepsilon)\left[ g(x) \neq 0 \right] \Rightarrow f = g
\end{align*}
このとき、その関数族$\mathcal{E}$に$f \approx g\ (x \rightarrow a)$が成り立つような関数$g$が存在しその漸近展開の主要部$c_{1}g_{1}$はこれを満たすのであったので、次式が成り立つ。
\begin{align*}
c_{1}g_{1}\in \mathcal{E,\ \ }f \approx c_{1}g_{1}\ (x \rightarrow a),\ \ \forall x \in U_{0}(a,\varepsilon)\left[ c_{1}g_{1}(x) \neq 0 \right] \Rightarrow f = c_{1}g_{1}
\end{align*}
よって、その漸近展開の主要部$c_{1}g_{1}$は一意的になっている。
\end{proof}
%\hypertarget{ux6f38ux8fd1ux7dda}{%
\subsubsection{漸近線}%\label{ux6f38ux8fd1ux7dda}}
\begin{dfn}
$D(f) \subseteq \mathbb{R}$なる関数$f:D(f) \rightarrow \mathbb{R}$がある実数たち$a、bとr \in o_{1, \pm \infty}$なるある関数$r$を用いて次式のようになるとき、
\begin{align*}
f(x) = ax + b + r(x)
\end{align*}
関数$A_{f}:\mathbb{R} \rightarrow \mathbb{R};x \mapsto ax + b$をその関数$f$の漸近線という。
\end{dfn}
\begin{thm}\label{4.2.4.8}
$D(f) \subseteq \mathbb{R}$なる関数$f:D(f) \rightarrow \mathbb{R}$が$\lim_{x \rightarrow \pm \infty}{f(x)} = \alpha \in \mathbb{R}$を満たすとき、関数$A_{f}:\mathbb{R} \rightarrow \mathbb{R};x \mapsto \alpha$がその関数$f$の漸近線となる。
\end{thm}
\begin{proof}
$D(f) \subseteq \mathbb{R}$なる関数$f:D(f) \rightarrow \mathbb{R}$が$\lim_{x \rightarrow \pm \infty}{f(x)} = \alpha \in \mathbb{R}$を満たすとき、明らかに次式が成り立つので、
\begin{align*}
\lim_{x \rightarrow \pm \infty}\frac{f(x) - \alpha}{1} = 0
\end{align*}
$f - \alpha \in o_{1, \pm \infty}$が成り立つ。これにより、次式のようになる。
\begin{align*}
f(x) = \alpha + f(x) - \alpha
\end{align*}
したがって、関数$A_{f}:\mathbb{R} \rightarrow \mathbb{R};x \mapsto \alpha$がその関数$f$の漸近線となる。
\end{proof}
\begin{thm}\label{4.2.4.9}
$D(f) \subseteq \mathbb{R}$なる関数$f:D(f) \rightarrow \mathbb{R}$が次式を満たすとき、
\begin{align*}
\lim_{x \rightarrow \pm \infty}\frac{f(x)}{x} = \alpha \in \mathbb{R} \land \lim_{x \rightarrow \pm \infty}\left( f(x) - \alpha x \right) = \beta \in \mathbb{R}
\end{align*}
関数$A_{f}:\mathbb{R} \rightarrow \mathbb{R};x \mapsto \alpha x + \beta$がその関数$f$の漸近線となる。
\end{thm}
\begin{proof}
$D(f) \subseteq \mathbb{R}$なる関数$f:D(f) \rightarrow \mathbb{R}$が次式を満たすとき、
\begin{align*}
\lim_{x \rightarrow \pm \infty}\frac{f(x)}{x} = \alpha \in \mathbb{R} \land \lim_{x \rightarrow \pm \infty}\left( f(x) - \alpha x \right) = \beta \in \mathbb{R}
\end{align*}
ここで、関数$r$を次式のようにおくと、
\begin{align*}
r:D(f) \rightarrow \mathbb{R};x \mapsto f(x) - (\alpha x + \beta)
\end{align*}
したがって、次のようになり
\begin{align*}
\lim_{x \rightarrow \pm \infty}\frac{r(x)}{1} = \lim_{x \rightarrow \pm \infty}\left( f(x) - (\alpha x + \beta) \right) = \lim_{x \rightarrow \pm \infty}\left( f(x) - \alpha x \right) - \beta = \beta - \beta = 0
\end{align*}
$r \in o_{1, \pm \infty}$が成り立つ。これにより、次式のようになる。
\begin{align*}
f(x) = \alpha x + \beta + f(x) - (\alpha x + \beta) = \alpha x + \beta + r(x)
\end{align*}
したがって、関数$A_{f}:\mathbb{R} \rightarrow \mathbb{R};x \mapsto \alpha x + \beta$がその関数$f$の漸近線となる。
\end{proof}
\begin{thebibliography}{50}
  \bibitem{1}
  杉浦光夫, 解析入門I, 東京大学出版社, 1985. 第34刷 p113-118 ISBN978-4-13-062005-5
\end{thebibliography}
\end{document}
