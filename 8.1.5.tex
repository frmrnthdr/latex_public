\documentclass[dvipdfmx]{jsarticle}
\setcounter{section}{1}
\setcounter{subsection}{4}
\usepackage{xr}
\externaldocument{8.1.1}
\usepackage{amsmath,amsfonts,amssymb,array,comment,mathtools,url,docmute}
\usepackage{longtable,booktabs,dcolumn,tabularx,mathtools,multirow,colortbl,xcolor}
\usepackage[dvipdfmx]{graphics}
\usepackage{bmpsize}
\usepackage{amsthm}
\usepackage{enumitem}
\setlistdepth{20}
\renewlist{itemize}{itemize}{20}
\setlist[itemize]{label=•}
\renewlist{enumerate}{enumerate}{20}
\setlist[enumerate]{label=\arabic*.}
\setcounter{MaxMatrixCols}{20}
\setcounter{tocdepth}{3}
\newcommand{\rotin}{\text{\rotatebox[origin=c]{90}{$\in $}}}
\newcommand{\amap}[6]{\text{\raisebox{-0.7cm}{\begin{tikzpicture} 
  \node (a) at (0, 1) {$\textstyle{#2}$};
  \node (b) at (#6, 1) {$\textstyle{#3}$};
  \node (c) at (0, 0) {$\textstyle{#4}$};
  \node (d) at (#6, 0) {$\textstyle{#5}$};
  \node (x) at (0, 0.5) {$\rotin $};
  \node (x) at (#6, 0.5) {$\rotin $};
  \draw[->] (a) to node[xshift=0pt, yshift=7pt] {$\textstyle{\scriptstyle{#1}}$} (b);
  \draw[|->] (c) to node[xshift=0pt, yshift=7pt] {$\textstyle{\scriptstyle{#1}}$} (d);
\end{tikzpicture}}}}
\newcommand{\twomaps}[9]{\text{\raisebox{-0.7cm}{\begin{tikzpicture} 
  \node (a) at (0, 1) {$\textstyle{#3}$};
  \node (b) at (#9, 1) {$\textstyle{#4}$};
  \node (c) at (#9+#9, 1) {$\textstyle{#5}$};
  \node (d) at (0, 0) {$\textstyle{#6}$};
  \node (e) at (#9, 0) {$\textstyle{#7}$};
  \node (f) at (#9+#9, 0) {$\textstyle{#8}$};
  \node (x) at (0, 0.5) {$\rotin $};
  \node (x) at (#9, 0.5) {$\rotin $};
  \node (x) at (#9+#9, 0.5) {$\rotin $};
  \draw[->] (a) to node[xshift=0pt, yshift=7pt] {$\textstyle{\scriptstyle{#1}}$} (b);
  \draw[|->] (d) to node[xshift=0pt, yshift=7pt] {$\textstyle{\scriptstyle{#2}}$} (e);
  \draw[->] (b) to node[xshift=0pt, yshift=7pt] {$\textstyle{\scriptstyle{#1}}$} (c);
  \draw[|->] (e) to node[xshift=0pt, yshift=7pt] {$\textstyle{\scriptstyle{#2}}$} (f);
\end{tikzpicture}}}}
\renewcommand{\thesection}{第\arabic{section}部}
\renewcommand{\thesubsection}{\arabic{section}.\arabic{subsection}}
\renewcommand{\thesubsubsection}{\arabic{section}.\arabic{subsection}.\arabic{subsubsection}}
\everymath{\displaystyle}
\allowdisplaybreaks[4]
\usepackage{vtable}
\theoremstyle{definition}
\newtheorem{thm}{定理}[subsection]
\newtheorem*{thm*}{定理}
\newtheorem{dfn}{定義}[subsection]
\newtheorem*{dfn*}{定義}
\newtheorem{axs}[dfn]{公理}
\newtheorem*{axs*}{公理}
\renewcommand{\headfont}{\bfseries}
\makeatletter
  \renewcommand{\section}{%
    \@startsection{section}{1}{\z@}%
    {\Cvs}{\Cvs}%
    {\normalfont\huge\headfont\raggedright}}
\makeatother
\makeatletter
  \renewcommand{\subsection}{%
    \@startsection{subsection}{2}{\z@}%
    {0.5\Cvs}{0.5\Cvs}%
    {\normalfont\LARGE\headfont\raggedright}}
\makeatother
\makeatletter
  \renewcommand{\subsubsection}{%
    \@startsection{subsubsection}{3}{\z@}%
    {0.4\Cvs}{0.4\Cvs}%
    {\normalfont\Large\headfont\raggedright}}
\makeatother
\makeatletter
\renewenvironment{proof}[1][\proofname]{\par
  \pushQED{\qed}%
  \normalfont \topsep6\p@\@plus6\p@\relax
  \trivlist
  \item\relax
  {
  #1\@addpunct{.}}\hspace\labelsep\ignorespaces
}{%
  \popQED\endtrivlist\@endpefalse
}
\makeatother
\renewcommand{\proofname}{\textbf{証明}}
\usepackage{tikz,graphics}
\usepackage[dvipdfmx]{hyperref}
\usepackage{pxjahyper}
\hypersetup{
 setpagesize=false,
 bookmarks=true,
 bookmarksdepth=tocdepth,
 bookmarksnumbered=true,
 colorlinks=false,
 pdftitle={},
 pdfsubject={},
 pdfauthor={},
 pdfkeywords={}}
\begin{document}
%\hypertarget{ux9023ux7d50}{%
\subsection{連結}%\label{ux9023ux7d50}}
%\hypertarget{ux9023ux7d50-1}{%
\subsubsection{連結}%\label{ux9023ux7d50-1}}
\begin{dfn}
任意の位相空間$\left( S,\mathfrak{O} \right)$が与えられたとする。これの台集合$S$、空集合$\emptyset$はどちらもその位相空間$\left( S,\mathfrak{O} \right)$の開集合でもあり閉集合でもあるのであった。ここで、その位相空間$\left( S,\mathfrak{O} \right)$の閉集合系$\mathfrak{A}$を用いて次式を満たすとき、その位相空間$\left( S,\mathfrak{O} \right)$は連結であるという\footnote{定義だけみるとなんじゃあこりゃって感じですが、解析学的にいえば、弧状連結という平たくいえば集合内のどの2点でもある曲線を集合内で引けるような感じの集合と似た概念なので、心象したいときはとりあえずこれでやるといいんじゃあないかなと思います。}。
\begin{align*}
\mathfrak{O \cap A} = \left\{ S,\emptyset \right\}
\end{align*}
\end{dfn}\par
たとえば、任意の密着位相は連結である。また、離散位相は台集合$S$がただ1点のみからなるときに限り連結である。
\begin{thm}\label{8.1.5.1}
位相空間$\left( S,\mathfrak{O} \right)$が連結であるならそのときに限り、その閉集合系$\mathfrak{A}$を用いて$S = M \sqcup N$なるその集合$\mathfrak{O \cap A}$の空集合でない元々$M$、$N$が存在しない。
\end{thm}
\begin{proof}
任意の位相空間$\left( S,\mathfrak{O} \right)$が与えられたとする。その位相空間$\left( S,\mathfrak{O} \right)$が連結でないなら、その閉集合系$\mathfrak{A}$を用いて$\mathfrak{O \cap A} =\left\{ S,\emptyset \right\}$が成り立たなく、その台集合$S$、空集合$\emptyset$はどちらもその位相空間$\left( S,\mathfrak{O} \right)$の開集合でもあり閉集合でもあることに注意すれば、$M\in \mathfrak{O \cap A}$なる集合$M$が存在する。ここで、その集合$S \setminus M$について、$M\in \mathfrak{O}$より$S \setminus M\in \mathfrak{A}$が成り立つかつ、$M\in \mathfrak{A}$より$S \setminus M\in \mathfrak{O}$が成り立つので、$S \setminus M\in \mathfrak{O \cap A}$が成り立つ。さらに、$M \cap S \setminus M = \emptyset$が成り立つので、$M \sqcup S \setminus M = S$が成り立つことになり$S = M \sqcup N$なるその集合$\mathfrak{O \cap A}$の空集合でない元々$M$、$N$が存在する。\par
逆に、$S = M \sqcup N$なるその集合$\mathfrak{O \cap A}$の空集合でない元々$M$、$N$が存在するなら、どちらも$M,N \notin \left\{ S,\emptyset \right\}$が成り立つので、$\mathfrak{O \cap A \neq}\left\{ S,\emptyset \right\}$が成り立つことになりその位相空間$\left( S,\mathfrak{O} \right)$は連結でない。
\end{proof}
\begin{thm}\label{8.1.5.2}
また、その位相空間$\left( S,\mathfrak{O} \right)$が与えられたとする。その台集合$S$の空集合でない部分集合$M$を用いた部分位相空間$\left( M,\mathfrak{O}_{M} \right)$が連結であるならそのときに限り、次式たちいづれもが成り立つようなその位相空間$\left( S,\mathfrak{O} \right)$の開集合たち$O$、$P$が存在しない。
\begin{align*}
M \subseteq O \cup P,\ \ O \cap P \cap M = \emptyset,\ \ O \cap M \neq \emptyset,\ \ P \cap M \neq \emptyset
\end{align*}
\end{thm}
\begin{proof}
任意の位相空間$\left( S,\mathfrak{O} \right)$が与えられたとする。その台集合の空集合でない部分集合$M$を用いたその位相空間$\left( S,\mathfrak{O} \right)$の部分位相空間$\left( M,\mathfrak{O}_{M} \right)$が連結でないなら、その閉集合系$\mathfrak{A}_{M}$を用いて$M = M' \sqcup N'$なるその集合$\mathfrak{O}_{M} \cap \mathfrak{A}_{M}$の空集合でない元々$M'$、$N'$が存在する。ここで定理8.1.4.8
より$\exists O,P \in \mathfrak{O}$に対し、$M' = O \cap M$かつ$N' = P \cap M$が成り立つので、次式たちいづれもが成り立つ。
\begin{align*}
M = (O \cap M) \cup (P \cap M),\ \ O \cap M \cap P \cap M = \emptyset
\end{align*}
したがって、$M \subseteq O \cup P$かつ$O \cap P \cap M = \emptyset$が得られる。また、それらの集合たち$M'$、$N'$は空集合でないので、それらの集合たち$O \cap M$、$P \cap M$は空集合でない。\par
逆に、次式たちいづれもが成り立つようなその位相空間$\left( S,\mathfrak{O} \right)$の開集合たち$O$、$P$が存在するなら、
\begin{align*}
M \subseteq O \cup P,\ \ O \cap P \cap M = \emptyset,\ \ O \cap M \neq \emptyset,\ \ P \cap M \neq \emptyset
\end{align*}
それらの集合たち$O \cap M$、$P \cap M$をそれぞれ$M'$、$N'$とおくと、$M',N' \in \mathfrak{O}_{M}$が成り立つかつ、それらの集合たち$M'$、$N'$は空集合でない。さらに、$M' \cap N' = \emptyset$が成り立ち、$M \subseteq O \cup P$が成り立つので、次のようになる。
\begin{align*}
M' \sqcup N' &= (O \cap M) \cup (P \cap M)\\
&= (O \cup P) \cap M\\
&= M
\end{align*}
したがって、次式が成り立つので、
\begin{align*}
M' = M \setminus N',\ \ N' = M \setminus M'
\end{align*}
$M',N' \in \mathfrak{O}_{M} \cap \mathfrak{A}_{M}$なるある集合たち$M'$、$N'$が$M' \sqcup N' = M$を満たす。これにより、その部分位相空間$\left( M,\mathfrak{O}_{M} \right)$は連結でない。
\end{proof}
\begin{thm}\label{8.1.5.3}
位相空間たち$\left( S,\mathfrak{O} \right)$、$\left( T,\mathfrak{P} \right)$が与えられたとする。その位相空間$\left( S,\mathfrak{O} \right)$は連結で写像$f:S \rightarrow T$がその位相空間$\left( S,\mathfrak{O} \right)$からその位相空間$\left( T,\mathfrak{P} \right)$へ連続であるとするとき、その位相空間$\left( T,\mathfrak{P} \right)$の部分位相空間$\left( V(f),\mathfrak{O}_{V(f)} \right)$は連結である。
\end{thm}
\begin{proof}
位相空間たち$\left( S,\mathfrak{O} \right)$、$\left( T,\mathfrak{P} \right)$が与えられたとする。その位相空間$\left( S,\mathfrak{O} \right)$は連結で写像$f:S \rightarrow T$がその位相空間$\left( S,\mathfrak{O} \right)$からその位相空間$\left( T,\mathfrak{P} \right)$へ連続であるとするとき、その位相空間$\left( T,\mathfrak{P} \right)$の部分位相空間$\left( V(f),\mathfrak{O}_{V(f)} \right)$が連結でないと仮定しよう。このとき、次式が成り立つような次式が成り立つようなその位相空間$\left( T,\mathfrak{P} \right)$の開集合たち$O$、$P$が存在する。
\begin{align*}
V(f) \subseteq O \cup P,\ \ O \cap P \cap V(f) = \emptyset,\ \ O \cap V(f) \neq \emptyset,\ \ P \cap V(f) \neq \emptyset
\end{align*}
ここで、次のようになるかつ、
\begin{align*}
V(f) \subseteq O \cup P &\Rightarrow V\left( f^{- 1}|V(f) \right) \subseteq V\left( f^{- 1}|O \cup P \right)\\
&\Rightarrow S \subseteq V\left( f^{- 1}|O \right) \cup V\left( f^{- 1}|P \right)
\end{align*}
値域の定義より$V\left( f^{- 1}|O \right) \subseteq P$かつ$V\left( f^{- 1}|P \right) \subseteq S$が成り立つので、$S = V\left( f^{- 1}|O \right) \cup V\left( f^{- 1}|P \right)$が成り立つ。また、値域の定義より$V\left( f^{- 1}|O \right) \subseteq S$かつ$V\left( f^{- 1}|P \right) \subseteq S$が成り立つことに注意すれば、次のようになるかつ、
\begin{align*}
O \cap P \cap V(f) = \emptyset &\Rightarrow V\left( f^{- 1}|O \cap P \cap V(f) \right) = V\left( f^{- 1}|\emptyset \right) = \emptyset\\
&\Leftrightarrow V\left( f^{- 1}|O \right) \cap V\left( f^{- 1}|P \right) \cap V\left( f^{- 1}|V(f) \right) = \emptyset\\
&\Rightarrow V\left( f^{- 1}|O \right) \cap V\left( f^{- 1}|P \right) \cap S \subseteq \emptyset\\
&\Rightarrow V\left( f^{- 1}|O_{1} \right) \cap V\left( f^{- 1}|O_{2} \right) \subseteq \emptyset
\end{align*}
空集合はいかなる集合の部分集合であるのであったので、$V\left( f^{- 1}|O \right) \cap V\left( f^{- 1}|P \right) = \emptyset$が成り立つ。以上より、$S = V\left( f^{- 1}|O \right) \sqcup V\left( f^{- 1}|P \right)$が成り立つ。\par
さらに、$O \cap V(f) \neq \emptyset$が成り立つので、$\exists b \in S$に対し、$b \in O \cap V(f)$が成り立ち、$b \in V(f)$より$\exists a \in S$に対し、$f(a) = b$が成り立つ。これにより、明らかに$a \in S$が成り立つかつ、$\exists b \in O \cap V(f)$に対し、$f(a) = b$が成り立つので、定義より$a \in V\left( f^{- 1}|O \cap V(f) \right)$が成り立つ。ここで、次式が成り立つので、
\begin{align*}
V\left( f^{- 1}|O \cap V(f) \right) &= V\left( f^{- 1}|O \right) \cap V\left( f^{- 1}|V(f) \right)\\
&= V\left( f^{- 1}|O \right) \cap S\\
&= V\left( f^{- 1}|O \right)
\end{align*}
$V\left( f^{- 1}|O \right) \neq \emptyset$が成り立つ。同様にして、$V\left( f^{- 1}|P \right) \neq \emptyset$が成り立つ。\par
ここで、その写像$f$は連続であったので、$V\left( f^{- 1}|O \right),V\left( f^{- 1}|P \right) \in \mathfrak{O}$が成り立つ。$S = V\left( f^{- 1}|O \right) \sqcup V\left( f^{- 1}|P \right)$が成り立つことに注意すれば、$S \setminus V\left( f^{- 1}|O \right) = V\left( f^{- 1}|P \right)$かつ$S \setminus V\left( f^{- 1}|P \right) = V\left( f^{- 1}|O \right)$が成り立つので、その位相空間$\left( S,\mathfrak{O} \right)$の閉集合系$\mathfrak{A}$を用いて$V\left( f^{- 1}|O \right),V\left( f^{- 1}|P \right) \in \mathfrak{A}$が成り立つことになり、したがって、$V\left( f^{- 1}|O \right),V\left( f^{- 1}|P \right) \in \mathfrak{O} \cap \mathfrak{A}$が成り立つ。これにより、その位相空間$\left( S,\mathfrak{O} \right)$は連結でない。しかしながら、これは仮定に矛盾するので、その位相空間$\left( T,\mathfrak{P} \right)$の部分位相空間$\left( V(f),\mathfrak{O}_{V(f)} \right)$は連結である。
\end{proof}
\begin{thm}\label{8.1.5.4}
位相空間たち$\left( S,\mathfrak{O} \right)$、$\left( T,\mathfrak{P} \right)$が与えられたとする。その台集合$S$の空集合でない部分集合$M$を用いたその位相空間$\left( S,\mathfrak{O} \right)$の部分位相空間$\left( M,\mathfrak{O}_{M} \right)$は連結で写像$f:S \rightarrow T$がその位相空間$\left( S,\mathfrak{O} \right)$からその位相空間$\left( T,\mathfrak{P} \right)$へ連続であるとするとき、その位相空間$\left( T,\mathfrak{P} \right)$の部分位相空間$\left( V\left( f|M \right),\mathfrak{O}_{V\left( f|M \right)} \right)$は連結である。
\end{thm}
\begin{proof}
位相空間たち$\left( S,\mathfrak{O} \right)$、$\left( T,\mathfrak{P} \right)$が与えられたとする。その台集合$S$の空集合でない部分集合$M$を用いたその位相空間$\left( S,\mathfrak{O} \right)$の部分位相空間$\left( M,\mathfrak{O}_{M} \right)$は連結で写像$f:S \rightarrow T$がその位相空間$\left( S,\mathfrak{O} \right)$からその位相空間$\left( T,\mathfrak{P} \right)$へ連続であるとするとき、その部分集合$M$に制限されたその写像$f|M$もまたその位相空間$\left( M,\mathfrak{O}_{M} \right)$からその位相空間$\left( T,\mathfrak{P} \right)$へ連続であるので、上記の定理よりその位相空間$\left( T,\mathfrak{P} \right)$の部分位相空間$\left( V\left( f|M \right),\mathfrak{O}_{V\left( f|M \right)} \right)$は連結である。
\end{proof}
\begin{thm}\label{8.1.5.5}
位相空間$\left( S,\mathfrak{O} \right)$の台集合$S$の空集合でない部分集合$M$を用いた部分位相空間$\left( M,\mathfrak{O}_{M} \right)$が連結であるなら、$\forall N \in \mathfrak{P}(S)$に対し、$M \subseteq N \subseteq {\mathrm{cl}}M$が成り立つなら、その部分位相空間$\left( N,\mathfrak{O}_{N} \right)$も連結である。
\end{thm}
\begin{proof}
位相空間$\left( S,\mathfrak{O} \right)$の台集合$S$の空集合でない部分集合$M$を用いた部分位相空間$\left( M,\mathfrak{O}_{M} \right)$が連結であるかつ、$\exists N \in \mathfrak{P}(S)$に対し、$M \subseteq N \subseteq {\mathrm{cl}}M$が成り立つかつ、その部分位相空間$\left( N,\mathfrak{O}_{N} \right)$が連結でないと仮定しよう。このとき、$\exists O,P \in \mathfrak{O}$に対し、次式が成り立つ。
\begin{align*}
N \subseteq O \cup P,\ \ O \cap P \cap N = \emptyset,\ \ O \cap N \neq \emptyset,\ \ P \cap N \neq \emptyset
\end{align*}
$M \subseteq N$が成り立つので、次式が成り立つ。
\begin{align*}
M \subseteq O \cup P,\ \ O \cap P \cap M = \emptyset
\end{align*}
また、$O \cap M = \emptyset$が成り立つと仮定すると、定理\ref{8.1.1.10}より$O \cap {\mathrm{cl}}M = \emptyset$が成り立ち、したがって、$O \cap N = \emptyset$が成り立つことになるが、これは$O \cap N \neq \emptyset$が成り立つことに矛盾するので、$O \cap M \neq \emptyset$が成り立つ。同様にして、$O \cap M \neq \emptyset$が得られる。以上より、次式が得られ、
\begin{align*}
M \subseteq O \cup P,\ \ O \cap P \cap M = \emptyset,\ \ O \cap M \neq \emptyset,\ \ P \cap M \neq \emptyset
\end{align*}
これにより、その部分位相空間$\left( N,\mathfrak{O}_{N} \right)$は連結でないことになるが、これはその部分位相空間$\left( M,\mathfrak{O}_{M} \right)$が連結であるに矛盾している。よって、その部分位相空間$\left( N,\mathfrak{O}_{N} \right)$も連結である。
\end{proof}
\begin{thm}\label{8.1.5.6}
位相空間$\left( S,\mathfrak{O} \right)$の台集合$S$の添数集合$\varLambda$によって添数づけられた部分集合系$\left\{ M_{\lambda} \right\}_{\lambda \in \varLambda}$を用いた部分位相空間の族$\left\{ \left( M_{\lambda},\mathfrak{O}_{M_{\lambda}} \right) \right\}_{\lambda \in \varLambda}$が与えられたとする。$\forall\lambda \in \varLambda$に対し、それらの部分位相空間たち$\left( M_{\lambda},\mathfrak{O}_{M_{\lambda}} \right)$が連結であるかつ、$\forall\lambda,\mu \in \varLambda$に対し、$M_{\lambda} \cap M_{\mu} \neq \emptyset$が成り立つとき、その和集合$\bigcup_{\lambda \in \varLambda} M_{\lambda}$を用いた部分位相空間$\left( \bigcup_{\lambda \in \varLambda} M_{\lambda},\mathfrak{O}_{\bigcup_{\lambda \in \varLambda} M_{\lambda}} \right)$も連結である。
\end{thm}
\begin{proof}
位相空間$\left( S,\mathfrak{O} \right)$の台集合$S$の添数集合$\varLambda$によって添数づけられた部分集合系$\left\{ M_{\lambda} \right\}_{\lambda \in \varLambda}$を用いた部分位相空間の族$\left\{ \left( M_{\lambda},\mathfrak{O}_{M_{\lambda}} \right) \right\}_{\lambda \in \varLambda}$が与えられたとする。$\forall\lambda \in \varLambda$に対し、それらの部分位相空間たち$\left( M_{\lambda},\mathfrak{O}_{M_{\lambda}} \right)$が連結であるかつ、$\forall\lambda,\mu \in \varLambda$に対し、$M_{\lambda} \cap M_{\mu} \neq \emptyset$が成り立つかつ、その和集合$\bigcup_{\lambda \in \varLambda} M_{\lambda}$を用いた部分位相空間$\left( \bigcup_{\lambda \in \varLambda} M_{\lambda},\mathfrak{O}_{\bigcup_{\lambda \in \varLambda} M_{\lambda}} \right)$は連結でないと仮定しよう。このとき、$\exists O,P \in \mathfrak{O}$に対し、次式が成り立つ。
\begin{align*}
\bigcup_{\lambda \in \varLambda} M_{\lambda} \subseteq O \cup P,\ \ O \cap P \cap \bigcup_{\lambda \in \varLambda} M_{\lambda} = \emptyset,\ \ O \cap \bigcup_{\lambda \in \varLambda} M_{\lambda} \neq \emptyset,\ \ P \cap \bigcup_{\lambda \in \varLambda} M_{\lambda} \neq \emptyset
\end{align*}
$\forall\lambda \in \varLambda$に対し、$M_{\lambda} \subseteq \bigcup_{\lambda \in \varLambda} M_{\lambda}$が成り立つので、$\forall\lambda \in \varLambda$に対し、次式が成り立つ。
\begin{align*}
M_{\lambda} \subseteq O \cup P,\ \ O \cap P \cap M_{\lambda} = \emptyset
\end{align*}
ここで、$O \cap \bigcup_{\lambda \in \varLambda} M_{\lambda} \neq \emptyset$が成り立つので、$\exists a \in S$に対し、次のようになり、
\begin{align*}
a \in O \cap \bigcup_{\lambda \in \varLambda} M_{\lambda} = \bigcup_{\lambda \in \varLambda} \left( O \cap M_{\lambda} \right)
\end{align*}
$\exists\lambda \in \varLambda$に対し、$a \in O \cap M_{\lambda}$が成り立ち、したがって、$O \cap M_{\lambda} \neq \emptyset$が成り立つ。同様にして、$\exists\mu \in \varLambda$に対し、$P \cap M_{\mu} \neq \emptyset$が成り立つ。ここで、それらの部分位相空間たち$\left( M_{\lambda},\mathfrak{O}_{M_{\lambda}} \right)$、$\left( M_{\mu},\mathfrak{O}_{M_{\mu}} \right)$が連結であるので、$O_{1} \cap M_{\mu} = \emptyset$かつ$O_{2} \cap M_{\lambda} = \emptyset$が成り立つ。これにより、次式が成り立つ。
\begin{align*}
\emptyset &= \left( \emptyset \cap M_{\lambda} \right) \cup \left( \emptyset \cap M_{\mu} \right)\\
&= \left( O \cap M_{\lambda} \cap M_{\mu} \right) \cup \left( P \cap M_{\lambda} \cap M_{\mu} \right)\\
&= (O \cup P) \cap \left( M_{\lambda} \cap M_{\mu} \right)
\end{align*}
ここで、$M_{\lambda} \cap M_{\mu} \subseteq \bigcup_{\lambda \in \varLambda} M_{\lambda} \subseteq O \cup P$が成り立つので、次式が成り立つ。
\begin{align*}
\emptyset &= (O \cup P) \cap \left( M_{\lambda} \cap M_{\mu} \right)\\
&= M_{\lambda} \cap M_{\mu}
\end{align*}
しかしながら、これは仮定の、$\forall\lambda,\mu \in \varLambda$に対し、$M_{\lambda} \cap M_{\mu} \neq \emptyset$が成り立つことに矛盾しているので、その和集合$\bigcup_{\lambda \in \varLambda} M_{\lambda}$を用いた部分位相空間$\left( \bigcup_{\lambda \in \varLambda} M_{\lambda},\mathfrak{O}_{\bigcup_{\lambda \in \varLambda} M_{\lambda}} \right)$も連結である。
\end{proof}
\begin{thm}\label{8.1.5.7}
位相空間$\left( S,\mathfrak{O} \right)$の台集合$S$の添数集合$\varLambda$によって添数づけられた部分集合系$\left\{ M_{\lambda} \right\}_{\lambda \in \varLambda}$を用いた部分位相空間の族$\left\{ \left( M_{\lambda},\mathfrak{O}_{M_{\lambda}} \right) \right\}_{\lambda \in \varLambda}$が与えられたとする。$\forall\lambda \in \varLambda$に対し、それらの部分位相空間たち$\left( M_{\lambda},\mathfrak{O}_{M_{\lambda}} \right)$が連結であるかつ、$\exists a \in S$に対し、$a \in M_{\lambda}$が成り立つとき、その和集合$\bigcup_{\lambda \in \varLambda} M_{\lambda}$を用いた部分位相空間$\left( \bigcup_{\lambda \in \varLambda} M_{\lambda},\mathfrak{O}_{\bigcup_{\lambda \in \varLambda} M_{\lambda}} \right)$も連結である。
\end{thm}
\begin{proof}
位相空間$\left( S,\mathfrak{O} \right)$の台集合$S$の添数集合$\varLambda$によって添数づけられた部分集合系$\left\{ M_{\lambda} \right\}_{\lambda \in \varLambda}$を用いた部分位相空間の族$\left\{ \left( M_{\lambda},\mathfrak{O}_{M_{\lambda}} \right) \right\}_{\lambda \in \varLambda}$が与えられたとする。$\forall\lambda \in \varLambda$に対し、それらの部分位相空間たち$\left( M_{\lambda},\mathfrak{O}_{M_{\lambda}} \right)$が連結であるかつ、$\exists a \in S$に対し、$a \in M_{\lambda}$が成り立つとき、$\forall\lambda,\mu \in \varLambda$に対し、$a \in M_{\lambda} \cap M_{\mu}$が成り立つので、$M_{\lambda} \cap M_{\mu} \neq \emptyset$が成り立ち、したがって、その和集合$\bigcup_{\lambda \in \varLambda} M_{\lambda}$を用いた部分位相空間$\left( \bigcup_{\lambda \in \varLambda} M_{\lambda},\mathfrak{O}_{\bigcup_{\lambda \in \varLambda} M_{\lambda}} \right)$も連結である。
\end{proof}
%\hypertarget{ux9023ux7d50ux6210ux5206}{%
\subsubsection{連結成分}%\label{ux9023ux7d50ux6210ux5206}}
\begin{dfn}
任意の位相空間$\left( S,\mathfrak{O} \right)$が与えられたとする。このとき、$a,b \in S$なる元々$a$、$b$が属するようなその位相空間$\left( S,\mathfrak{O} \right)$の連結な部分位相空間$\left( M,\mathfrak{O}_{M} \right)$の台集合$M$が存在するとき、このことを$a \sim_{\left( S,\mathfrak{O} \right)}b$と書くことにする。
\end{dfn}
\begin{thm}\label{8.1.5.8}
その関係$\sim_{\left( S,\mathfrak{O} \right)}$は同値関係となる。\par
このようにして、その集合$S$を同値関係$\sim_{\left( S,\mathfrak{O} \right)}$で類別して得られる商集合$\frac{S}{\sim_{\left( S,\mathfrak{O} \right)}}$の元をその位相空間$\left( S,\mathfrak{O} \right)$の連結成分という。
\end{thm}
\begin{proof}
任意の位相空間$\left( S,\mathfrak{O} \right)$が与えられたとする。\par
$\forall a \in S$に対し、位相空間$\left( \left\{ a \right\},\left\{ \emptyset,\left\{ a \right\} \right\} \right)$が考えれられれば、これは明らかにその位相空間$\left( S,\mathfrak{O} \right)$の部分位相空間であるかつ、上記の定理より明らかにその位相空間$\left( \left\{ a \right\},\left\{ \emptyset,\left\{ a \right\} \right\} \right)$は連結である。したがって、$a \sim_{\left( S,\mathfrak{O} \right)}a$が成り立つ。\par
また、$\forall a,b \in S$に対し、定義より明らかに$a \sim_{\left( S,\mathfrak{O} \right)}b$が成り立つなら、$b \sim_{\left( S,\mathfrak{O} \right)}a$が成り立つ。\par
最後に、$\forall a,b,c \in S$に対し、$a \sim_{\left( S,\mathfrak{O} \right)}b$かつ$b \sim_{\left( S,\mathfrak{O} \right)}c$が成り立つなら、それらの元々$a$、$b$が属するようなその位相空間$\left( S,\mathfrak{O} \right)$の連結な部分位相空間$\left( M,\mathfrak{O}_{M} \right)$の台集合$M$が存在するかつ、それらの元々$b$、$c$が属するようなその位相空間$\left( S,\mathfrak{O} \right)$の連結な部分位相空間$\left( N,\mathfrak{O}_{N} \right)$の台集合$N$が存在することになる。ここで、$b \in M$かつ$b \in N$が成り立つので、部分位相空間$\left( M \cup N,\mathfrak{O}_{M \cup N} \right)$も連結であることになる。このとき、$a,c \in M \cup N$が成り立つので、$a \sim_{\left( S,\mathfrak{O} \right)}c$が成り立つ。
\end{proof}
\begin{thm}\label{8.1.5.9}
位相空間$\left( S,\mathfrak{O} \right)$の台集合$S$の元$a$が属する連結成分を$C_{a}$とおく。その元$a$を含むその位相空間$\left( S,\mathfrak{O} \right)$の連結な任意の部分位相空間の台集合全体の集合を$\mathcal{C}_{\left( S,\mathfrak{O} \right),a}$とすれば、順序集合$\left( \mathcal{C}_{\left( S,\mathfrak{O} \right),a}, \subseteq \right)$においてその連結成分$C_{a}$は最大元となる。さらに、その連結成分$C_{a}$はその位相空間$\left( S,\mathfrak{O} \right)$において閉集合である。
\end{thm}
\begin{proof}
位相空間$\left( S,\mathfrak{O} \right)$の台集合$S$の元$a$が属する連結成分を$C_{a}$とおく。その元$a$を含むその位相空間$\left( S,\mathfrak{O} \right)$の連結な任意の部分位相空間の台集合全体の集合を$\mathcal{C}_{\left( S,\mathfrak{O} \right),a}$とすれば、$\forall M \in \mathcal{C}_{\left( S,\mathfrak{O} \right),a}$に対し、その位相空間$\left( S,\mathfrak{O} \right)$の部分位相空間たち$\left( M,\mathfrak{O}_{M} \right)$は定義より連結であるかつ、定義より$a \in M$が成り立つので、和集合$\bigcup_{} \mathcal{C}_{\left( S,\mathfrak{O} \right),a}$を用いた部分位相空間$\left( \bigcup_{} \mathcal{C}_{\left( S,\mathfrak{O} \right),a},\mathfrak{O}_{\bigcup_{} \mathcal{C}_{\left( S,\mathfrak{O} \right),a}} \right)$も連結であるので、明らかに$\bigcup_{} \mathcal{C}_{\left( S,\mathfrak{O} \right),a} \in \mathcal{C}_{\left( S,\mathfrak{O} \right),a}$が成り立つ。さらに、$\forall M \in \mathcal{C}_{\left( S,\mathfrak{O} \right),a}$に対し、$M \subseteq \bigcup_{} \mathcal{C}_{\left( S,\mathfrak{O} \right),a}$が成り立つので、順序集合$\left( \mathcal{C}_{\left( S,\mathfrak{O} \right),a}, \subseteq \right)$において、その集合$\bigcup_{M \in \mathcal{C}_{\left( S,\mathfrak{O} \right),a}} M$は最大元となる。\par
一方で、$\forall b \in \bigcup_{} \mathcal{C}_{\left( S,\mathfrak{O} \right),a}$に対し、$a,b \in \bigcup_{} \mathcal{C}_{\left( S,\mathfrak{O} \right),a}$が成り立つので、$a \sim_{\left( S,\mathfrak{O} \right)}b$が成り立ち、したがって、$b \in C_{a}$が成り立つ。これにより、$\bigcup_{} \mathcal{C}_{\left( S,\mathfrak{O} \right),a} \subseteq C_{a}$が得られる。さらに、$\forall b \in C_{a}$に対し、$a \sim_{\left( S,\mathfrak{O} \right)}b$が成り立つことになるので、それらの元々$a$、$b$が属するようなその位相空間$\left( S,\mathfrak{O} \right)$の連結な部分位相空間$\left( M',\mathfrak{O}_{M'} \right)$の台集合$M'$が存在し、明らかに、$M' \in \mathcal{C}_{\left( S,\mathfrak{O} \right),a}$が成り立つことになる。以上より、次式が得られる。
\begin{align*}
C_{a} \subseteq M' \subseteq \bigcup_{M \in \mathcal{C}_{\left( S,\mathfrak{O} \right),a}} M = \bigcup_{} \mathcal{C}_{\left( S,\mathfrak{O} \right),a}
\end{align*}
これにより、$\bigcup_{} \mathcal{C}_{\left( S,\mathfrak{O} \right),a} = C_{a}$が成り立つことになり順序集合$\left( \mathcal{C}_{\left( S,\mathfrak{O} \right),a}, \subseteq \right)$においてその集合$C_{a}$は最大元となる。\par
また、$\bigcup_{} \mathcal{C}_{\left( S,\mathfrak{O} \right),a} = C_{a}$が成り立つことに注意すれば、部分位相空間$\left( C_{a},\mathfrak{O}_{C_{a}} \right)$が連結であるなら、$C_{a} \subseteq M \subseteq {\mathrm{cl}}C_{a}$なるその台集合$S$の任意の部分集合$M$を用いた部分位相空間$\left( M,\mathfrak{O}_{M} \right)$も連結であるのであったので、その部分位相空間$\left( {\mathrm{cl}}C_{a},\mathfrak{O}_{{\mathrm{cl}}C_{a}} \right)$も連結であることになり、したがって、$C_{a} \subseteq {\mathrm{cl}}C_{a} \in \mathcal{C}_{\left( S,\mathfrak{O} \right),a}$が成り立つ。ここで、その連結成分$C_{a}$がその順序集合$\left( \mathcal{C}_{\left( S,\mathfrak{O} \right),a}, \subseteq \right)$において最大元となることに注意すれば、${\mathrm{cl}}C_{a} \subseteq C_{a}$が成り立ち、したがって、$C_{a} = {\mathrm{cl}}C_{a}$が成り立つ。これにより、その連結成分$C_{a}$は閉集合である。
\end{proof}
\begin{thm}\label{8.1.5.10}
位相空間$\left( S,\mathfrak{O} \right)$が連結であるならそのときに限り、その位相空間$\left( S,\mathfrak{O} \right)$はただ1つの連結成分をもちこれがその台集合$S$自身となる。
\end{thm}
\begin{proof}
位相空間$\left( S,\mathfrak{O} \right)$が連結であるなら、位相空間$\left( S,\mathfrak{O} \right)$の台集合$S$の元$a$が属する連結成分を$C_{a}$とおき、その元$a$を含むその位相空間$\left( S,\mathfrak{O} \right)$の連結な任意の部分位相空間の台集合全体の集合を$\mathcal{C}_{\left( S,\mathfrak{O} \right),a}$とすれば、順序集合$\left( \mathcal{C}_{\left( S,\mathfrak{O} \right),a}, \subseteq \right)$においてその連結成分$C_{a}$は最大元となるのであった。このとき、順序集合$\left( \mathcal{C}_{\left( S,\mathfrak{O} \right),a}, \subseteq \right)$において、$S \in \mathcal{C}_{\left( S,\mathfrak{O} \right),a}$が成り立つので、明らかに$C_{a} = S$が成り立つ。これ以外にその位相空間$\left( S,\mathfrak{O} \right)$の連結成分$C$が存在すれば、商集合の定義より$C \sqcup C_{a} \subseteq S$が成り立ち$C \sqcup S \subseteq S$が成り立つことになるが、これは矛盾している。したがって、その位相空間$\left( S,\mathfrak{O} \right)$はただ1つの連結成分をもちこれがその台集合$S$自身となる。逆に、その位相空間$\left( S,\mathfrak{O} \right)$はただ1つの連結成分をもちこれがその台集合$S$自身となるなら、明らかに位相空間$\left( S,\mathfrak{O} \right)$が連結である。
\end{proof}
%\hypertarget{ux76f4ux7a4dux4f4dux76f8ux7a7aux9593ux3068ux9023ux7d50}{%
\subsubsection{直積位相空間と連結}%\label{ux76f4ux7a4dux4f4dux76f8ux7a7aux9593ux3068ux9023ux7d50}}
\begin{thm}\label{8.1.5.11}
添数集合$\varLambda$によって添数づけられた位相空間の族$\left\{ \left( S_{\lambda},\mathfrak{O}_{\lambda} \right) \right\}_{\lambda \in \varLambda}$の直積位相空間$\left( \prod_{\lambda \in \varLambda} S_{\lambda},\mathfrak{O} \right)$が連結であるならそのときに限り、$\forall\lambda \in \varLambda$に対し、それらの位相空間たち$\left( S_{\lambda},\mathfrak{O}_{\lambda} \right)$が連結である。
\end{thm}
\begin{proof}
添数集合$\varLambda$によって添数づけられた位相空間の族$\left\{ \left( S_{\lambda},\mathfrak{O}_{\lambda} \right) \right\}_{\lambda \in \varLambda}$の直積位相空間$\left( \prod_{\lambda \in \varLambda} S_{\lambda},\mathfrak{O} \right)$が連結であるなら、$\forall\lambda \in \varLambda$に対し、射影たち${\mathrm{pr}}_{\lambda}:\prod_{\lambda \in \varLambda} S_{\lambda} \rightarrow S_{\lambda}$は定義より明らかに連続写像で、このとき、その位相空間$\left( S_{\lambda},\mathfrak{O}_{\lambda} \right)$の部分位相空間$\left( V\left( {\mathrm{pr}}_{\lambda} \right),\mathfrak{O}_{V\left( {\mathrm{pr}}_{\lambda} \right)} \right)$は連結であるのであった。このとき、それらの射影たち${\mathrm{pr}}_{\lambda}$の定義より明らかに$V\left( {\mathrm{pr}}_{\lambda} \right) = S_{\lambda}$が成り立ち、さらに、その直積位相$\mathfrak{O}$はその直積位相空間$\left( \prod_{\lambda \in \varLambda} S_{\lambda},\mathfrak{O}_{0} \right)$の初等開集合全体の集合が1つの開基となるので、これの和集合に制限されたそれらの射影たち${\mathrm{pr}}_{\lambda}$の値域がそれらの位相空間たち$\left( S_{\lambda},\mathfrak{O}_{\lambda} \right)$の開集合$O_{\lambda}$となることに注意すれば、やはり$\mathfrak{O}_{V\left( {\mathrm{pr}}_{\lambda} \right)} = \mathfrak{O}_{\lambda}$が成り立つ。以上より、その位相空間$\left( S_{\lambda},\mathfrak{O}_{\lambda} \right)$は連結である。\par
逆に、$\forall\lambda \in \varLambda$に対し、それらの位相空間たち$\left( S_{\lambda},\mathfrak{O}_{\lambda} \right)$が連結であるなら、$\varLambda = \left\{ 1,2 \right\}$のとき、$\forall\left( a_{1},a_{2} \right),\left( b_{1},b_{2} \right) \in S_{1} \times S_{2}$に対し、次式が成り立つので、
\begin{align*}
\left( S_{1},\mathfrak{O}_{1} \right) \approx \left( S_{1} \times \left\{ b_{2} \right\},\mathfrak{O}_{1}' \right) \land \left( S_{2},\mathfrak{O}_{2} \right) \approx \left( \left\{ a_{1} \right\} \times S_{2},\mathfrak{O}_{2}' \right)
\end{align*}
直積位相空間たち$\left( S_{1} \times \left\{ b_{2} \right\},\mathfrak{O}_{1}' \right)$、$\left( \left\{ a_{1} \right\} \times S_{2},\mathfrak{O}_{2}' \right)$はいずれも連結で、$\left( a_{1},b_{2} \right) \in \left( S_{1} \times \left\{ b_{2} \right\} \right) \cap \left( \left\{ a_{1} \right\} \times S_{2} \right)$が成り立つので、$\left( S_{1} \times \left\{ b_{2} \right\} \right) \cap \left( \left\{ a_{1} \right\} \times S_{2} \right) \neq \emptyset$が成り立つ。これにより、部分位相空間$\left( \left( S_{1} \times \left\{ b_{2} \right\} \right) \cup \left( \left\{ a_{1} \right\} \times S_{2} \right),\mathfrak{O}' \right)$も連結となり、明らかに$\left( a_{1},a_{2} \right),\left( b_{1},b_{2} \right) \in \left( S_{1} \times \left\{ b_{2} \right\} \right) \cup \left( \left\{ a_{1} \right\} \times S_{2} \right)$が成り立つので、次式が成り立つ。
\begin{align*}
\left( a_{1},a_{2} \right) \sim_{\left( S_{1} \times S_{2},\mathfrak{O} \right)}\left( b_{1},b_{2} \right)
\end{align*}
ここで、その位相空間$\left( S_{1} \times S_{2},\mathfrak{O} \right)$はただ1つの連結成分をもちこれがその台集合$S$自身となるので、その位相空間$\left( S_{1} \times S_{2},\mathfrak{O} \right)$も連結である。あとは数学的帰納法によって、一般に${\#}\varLambda < \aleph_{0}$なる添数集合$\varLambda$に対しても、$\forall\lambda \in \varLambda$に対し、それらの位相空間たち$\left( S_{\lambda},\mathfrak{O}_{\lambda} \right)$が連結であるなら、その直積位相空間$\left( \prod_{\lambda \in \varLambda} S_{\lambda},\mathfrak{O} \right)$も連結である。\par
さらに、任意の添数集合$\varLambda$において、$\forall\left( a_{\lambda}' \right)_{\lambda \in \varLambda} \in \prod_{\lambda \in \varLambda} S_{\lambda}$に対し、その添数集合$\varLambda$の部分集合のうち有限集合であるもの全体の集合を$\varOmega$とおくとする。$\forall\varLambda' \in \varOmega$に対し、次式が成り立つかつ、
\begin{align*}
\prod_{\lambda \in \varLambda'} S_{\lambda} \times \prod_{\lambda \in \varLambda \setminus \varLambda'} \left\{ a_{\lambda}' \right\} \approx \prod_{\lambda \in \varLambda'} S_{\lambda}
\end{align*}
${\#}\varLambda' < \aleph_{0}$が成り立つので、その直積位相空間$\left( \prod_{\lambda \in \varLambda'} S_{\lambda} \times \prod_{\lambda \in \varLambda \setminus \varLambda'} \left\{ a_{\lambda}' \right\},\mathfrak{O}_{\varLambda'}' \right)$も連結である。このような族$\left\{ \left( \prod_{\lambda \in \varLambda'} S_{\lambda} \times \prod_{\lambda \in \varLambda \setminus \varLambda'} \left\{ a_{\lambda}' \right\},\mathfrak{O}_{\varLambda'}' \right) \right\}_{\varLambda' \in \varOmega}$がその直積位相空間$\left( \prod_{\lambda \in \varLambda} S_{\lambda},\mathfrak{O} \right)$の連結な部分位相空間の族となり、$\forall\varLambda' \in \varOmega$に対し、$\left( a_{\lambda}' \right)_{\lambda \in \varLambda} \in \prod_{\lambda \in \varLambda'} S_{\lambda} \times \prod_{\lambda \in \varLambda \setminus \varLambda'} \left\{ a_{\lambda}' \right\}$が成り立つので、上記の定理よりその和集合を用いた位相空間$\left( \bigcup_{\varLambda' \in \varOmega} \left( \prod_{\lambda \in \varLambda'} S_{\lambda} \times \prod_{\lambda \in \varLambda \setminus \varLambda'} \left\{ a_{\lambda}' \right\} \right),\mathfrak{O}'' \right)$も連結である。$\forall\left( a_{\lambda} \right)_{\lambda \in \varLambda} \in \prod_{\lambda \in \varLambda} S_{\lambda}$に対し、$\varLambda' \in \varOmega$なる添数集合$\varLambda'$と各成分$a_{\lambda}$の全近傍系$\mathbf{V}_{S_{\lambda}}\left( a_{\lambda} \right)$の元々$V_{\lambda}$を用いて$\prod_{\lambda \in \varLambda'} V_{\lambda} \times \prod_{\lambda \in \varLambda \setminus \varLambda'} S_{\lambda}$という形で書かれる集合全体の集合$\mathbf{V}^{*}\left( a_{\lambda} \right)_{\lambda \in \varLambda}$はその直積位相空間$\left( \prod_{\lambda \in \varLambda} S_{\lambda},\mathfrak{O} \right)$の基本近傍系となり、$\left( a_{\lambda} \right)_{\lambda \in \varLambda} \in \prod_{\lambda \in \varLambda'} V_{\lambda} \times \prod_{\lambda \in \varLambda \setminus \varLambda'} \left\{ a_{\lambda}' \right\}$が成り立つことにより、次のようになるので、
\begin{align*}
\left( \prod_{\lambda \in \varLambda'} V_{\lambda} \times \prod_{\lambda \in \varLambda \setminus \varLambda'} S_{\lambda} \right) \cap \left( \prod_{\lambda \in \varLambda'} S_{\lambda} \times \prod_{\lambda \in \varLambda \setminus \varLambda'} \left\{ a_{\lambda}' \right\} \right) &= \prod_{\lambda \in \varLambda'} \left( V_{\lambda} \cap S_{\lambda} \right) \times \prod_{\lambda \in \varLambda \setminus \varLambda'} \left( S_{\lambda} \cap \left\{ a_{\lambda}' \right\} \right)\\
&= \prod_{\lambda \in \varLambda'} V_{\lambda} \times \prod_{\lambda \in \varLambda \setminus \varLambda'} \left\{ a_{\lambda}' \right\} \neq \emptyset
\end{align*}
$\forall V \in \mathbf{V}^{*}\left( a_{\lambda} \right)_{\lambda \in \varLambda}$に対し、$V \cap \bigcup_{\varLambda' \in \varOmega} \left( \prod_{\lambda \in \varLambda'} S_{\lambda} \times \prod_{\lambda \in \varLambda \setminus \varLambda'} \left\{ a_{\lambda}' \right\} \right) \neq \emptyset$が成り立つ。これにより、$\left( a_{\lambda} \right)_{\lambda \in \varLambda} \in {\mathrm{cl}}{\bigcup_{\varLambda' \in \varOmega} \left( \prod_{\lambda \in \varLambda'} S_{\lambda} \times \prod_{\lambda \in \varLambda \setminus \varLambda'} \left\{ a_{\lambda}' \right\} \right)}$が成り立ち、したがって、次式が得られる。
\begin{align*}
\prod_{\lambda \in \varLambda} S_{\lambda} = {\mathrm{cl}}{\bigcup_{\varLambda' \in \varOmega} \left( \prod_{\lambda \in \varLambda'} S_{\lambda} \times \prod_{\lambda \in \varLambda \setminus \varLambda'} \left\{ a_{\lambda}' \right\} \right)}
\end{align*}
ここで、その位相空間$\left( \bigcup_{\varLambda' \in \varOmega} \left( \prod_{\lambda \in \varLambda'} S_{\lambda} \times \prod_{\lambda \in \varLambda \setminus \varLambda'} \left\{ a_{\lambda}' \right\} \right),\mathfrak{O}'' \right)$も連結であり、次式が成り立つので、
\begin{align*}
\bigcup_{\varLambda' \in \varOmega} \left( \prod_{\lambda \in \varLambda'} S_{\lambda} \times \prod_{\lambda \in \varLambda \setminus \varLambda'} \left\{ a_{\lambda}' \right\} \right) &\subseteq \prod_{\lambda \in \varLambda} S_{\lambda}\\
&= {\mathrm{cl}}{\bigcup_{\varLambda' \in \varOmega} \left( \prod_{\lambda \in \varLambda'} S_{\lambda} \times \prod_{\lambda \in \varLambda \setminus \varLambda'} \left\{ a_{\lambda}' \right\} \right)}
\end{align*}
その直積位相空間$\left( \prod_{\lambda \in \varLambda} S_{\lambda},\mathfrak{O} \right)$も連結である。
\end{proof}
\begin{thebibliography}{50}
  \bibitem{1}
  松坂和夫, 集合・位相入門, 岩波書店, 1968. 新装版第2刷 p195-208 ISBN978-4-00-029871-1
\end{thebibliography}
\end{document}
