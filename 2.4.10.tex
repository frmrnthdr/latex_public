\documentclass[dvipdfmx]{jsarticle}
\setcounter{section}{4}
\setcounter{subsection}{9}
\usepackage{xr}
\externaldocument{2.4.9}
\usepackage{amsmath,amsfonts,amssymb,array,comment,mathtools,url,docmute}
\usepackage{longtable,booktabs,dcolumn,tabularx,mathtools,multirow,colortbl,xcolor}
\usepackage[dvipdfmx]{graphics}
\usepackage{bmpsize}
\usepackage{amsthm}
\usepackage{enumitem}
\setlistdepth{20}
\renewlist{itemize}{itemize}{20}
\setlist[itemize]{label=•}
\renewlist{enumerate}{enumerate}{20}
\setlist[enumerate]{label=\arabic*.}
\setcounter{MaxMatrixCols}{20}
\setcounter{tocdepth}{3}
\newcommand{\rotin}{\text{\rotatebox[origin=c]{90}{$\in $}}}
\renewcommand{\thesection}{第\arabic{section}部}
\renewcommand{\thesubsection}{\arabic{section}.\arabic{subsection}}
\renewcommand{\thesubsubsection}{\arabic{section}.\arabic{subsection}.\arabic{subsubsection}}
\everymath{\displaystyle}
\allowdisplaybreaks[4]
\usepackage{vtable}
\theoremstyle{definition}
\newtheorem{thm}{定理}[subsection]
\newtheorem*{thm*}{定理}
\newtheorem{dfn}{定義}[subsection]
\newtheorem*{dfn*}{定義}
\newtheorem{axs}[dfn]{公理}
\newtheorem*{axs*}{公理}
\renewcommand{\headfont}{\bfseries}
\makeatletter
  \renewcommand{\section}{%
    \@startsection{section}{1}{\z@}%
    {\Cvs}{\Cvs}%
    {\normalfont\huge\headfont\raggedright}}
\makeatother
\makeatletter
  \renewcommand{\subsection}{%
    \@startsection{subsection}{2}{\z@}%
    {0.5\Cvs}{0.5\Cvs}%
    {\normalfont\LARGE\headfont\raggedright}}
\makeatother
\makeatletter
  \renewcommand{\subsubsection}{%
    \@startsection{subsubsection}{3}{\z@}%
    {0.4\Cvs}{0.4\Cvs}%
    {\normalfont\Large\headfont\raggedright}}
\makeatother
\makeatletter
\renewenvironment{proof}[1][\proofname]{\par
  \pushQED{\qed}%
  \normalfont \topsep6\p@\@plus6\p@\relax
  \trivlist
  \item\relax
  {
  #1\@addpunct{.}}\hspace\labelsep\ignorespaces
}{%
  \popQED\endtrivlist\@endpefalse
}
\makeatother
\renewcommand{\proofname}{\textbf{証明}}
\usepackage{tikz,graphics}
\usepackage[dvipdfmx]{hyperref}
\usepackage{pxjahyper}
\hypersetup{
 setpagesize=false,
 bookmarks=true,
 bookmarksdepth=tocdepth,
 bookmarksnumbered=true,
 colorlinks=false,
 pdftitle={},
 pdfsubject={},
 pdfauthor={},
 pdfkeywords={}}
\begin{document}
%\hypertarget{tensorux4ee3ux6570}{%
\subsection{tensor代数}%\label{tensorux4ee3ux6570}}
%\hypertarget{tensorux4ee3ux6570-1}{%
\subsubsection{tensor代数}%\label{tensorux4ee3ux6570-1}}
\begin{dfn}
体$K$上の$n$次元vector空間$V$が与えられたとき、次式のように定義される集合$T(V)$が考えられよう。
\begin{align*}
T(V) = \prod_{r \in \mathbb{N} \cup \left\{ 0 \right\}} {T^{r}(V)} = K \times V \times (V \otimes V) \times (V \otimes V \otimes V) \times \cdots
\end{align*}
その直積$T(V)$をtensor代数という。このとき、その集合$T(V)$の元で次のように和とscalar倍を定義する。
\begin{itemize}
\item
  $\forall\left( \mathbf{t}_{r} \right)_{r \in \mathbb{N} \cup \left\{ 0 \right\}},\left( \mathbf{u}_{r} \right)_{r \in \mathbb{N} \cup \left\{ 0 \right\}} \in T(V)$に対し、$\left( \mathbf{t}_{r} \right)_{r \in \mathbb{N} \cup \left\{ 0 \right\}} + \left( \mathbf{u}_{r} \right)_{r \in \mathbb{N} \cup \left\{ 0 \right\}} = \left( \mathbf{t}_{r} + \mathbf{u}_{r} \right)_{r \in \mathbb{N} \cup \left\{ 0 \right\}}$が成り立つとする。
\item
  $\forall k \in K\forall\left( \mathbf{t}_{r} \right)_{r \in \mathbb{N} \cup \left\{ 0 \right\}} \in T(V)$に対し、$k\left( \mathbf{t}_{r} \right)_{r \in \mathbb{N} \cup \left\{ 0 \right\}} = \left( k\mathbf{t}_{r} \right)_{r \in \mathbb{N} \cup \left\{ 0 \right\}}$が成り立つとする。
\end{itemize}
\end{dfn}
\begin{thm}\label{2.4.10.1}
体$K$上の$n$次元vector空間$V$が与えられたとき、そのtensor代数$T(V)$は体$K$上のvector空間となる。
\end{thm}
\begin{proof}
$\forall r \in \mathbb{N} \cup \left\{ 0 \right\}$に対し、反変tensor空間$T^{r}(V)$がvector空間をなすことに注意すれば、定理\ref{2.4.9.1}と同様にして示される。
\end{proof}
\begin{dfn}
体$K$上の$n$次元vector空間$V$が与えられたとき、$r \neq s$なら、$\mathbf{t}_{r} = \mathbf{0}$なるそのtensor代数$T(V)$の元$\left( \mathbf{t}_{r} \right)_{r \in \mathbb{N} \cup \left\{ 0 \right\}}$を斉$s$次の元という。
\end{dfn}
\begin{thm}\label{2.4.10.2}
体$K$上の$n$次元vector空間$V$が与えられたとき、$r \in \mathbb{N} \cup \left\{ 0 \right\}$なる集合たち$\left\{ \mathbf{0} \right\}^{r} \times T^{r}(V) \times \left\{ \mathbf{0} \right\}^{\infty - r}$はいづれもそのtensor代数$T(V)$の部分空間をなし、さらに、次式が成り立つ。
\begin{align*}
T(V) &= \bigoplus_{r \in \mathbb{N} \cup \left\{ 0 \right\}} \left( \left\{ \mathbf{0} \right\}^{r} \times T^{r}(V) \times \left\{ \mathbf{0} \right\}^{\infty - r} \right)\\
&= \begin{matrix}
\  & \left( K \times \left\{ \mathbf{0} \right\} \times \left\{ \mathbf{0} \right\} \times \left\{ \mathbf{0} \right\} \times \cdots \right) \\
 \oplus & \left( \left\{ 0 \right\} \times V \times \left\{ \mathbf{0} \right\} \times \left\{ \mathbf{0} \right\} \times \cdots \right) \\
 \oplus & \left( \left\{ 0 \right\} \times \left\{ \mathbf{0} \right\} \times (V \otimes V) \times \left\{ \mathbf{0} \right\} \times \cdots \right) \\
 \oplus & \left( \left\{ 0 \right\} \times \left\{ \mathbf{0} \right\} \times \left\{ \mathbf{0} \right\} \times (V \otimes V \otimes V) \times \cdots \right) \\
 \oplus & \cdots \\
\end{matrix}
\end{align*}
\end{thm}
\begin{proof} 定理\ref{2.4.9.2}と同様にして示される。
\end{proof}
\begin{thm}\label{2.4.10.3}
体$K$上の$n$次元vector空間$V$が与えられたとき、$s \in \mathbb{N} \cup \left\{ 0 \right\}$なる次式のような線形同型写像が考えられることで、
\begin{align*}
\varphi_{s}:T^{s}(V)\overset{\sim}{\rightarrow}\left\{ \mathbf{0} \right\}^{s} \times T^{s}(V) \times \left\{ \mathbf{0} \right\}^{\infty - s};\mathbf{t}_{s} \mapsto \begin{pmatrix}
\mathbf{0} \\
\mathbf{0} \\
\mathbf{0} \\
 \vdots \\
\mathbf{t}_{s} \\
 \vdots \\
\end{pmatrix}
\end{align*}
\end{thm}
\begin{proof} 定理\ref{2.4.9.3}と同様にして示される。
\end{proof}
\begin{dfn}
体$K$上の$n$次元vector空間$V$が与えられたとき、そのtensor代数$T(V)$の元$\left( \mathbf{t}_{r} \right)_{r \in \mathbb{N} \cup \left\{ 0 \right\}}$で、ある非負整数$s$が存在して、任意の非負整数$r$に対し、$s < r$が成り立つなら、$\mathbf{t}_{r} = \mathbf{0}$が成り立つとき、明らかに次式が成り立つ。
\begin{align*}
\left( \mathbf{t}_{r} \right)_{r \in \mathbb{N} \cup \left\{ 0 \right\}} = \begin{pmatrix}
\mathbf{t}_{0} \\
\mathbf{0} \\
\mathbf{0} \\
 \vdots \\
\mathbf{0} \\
 \vdots \\
\end{pmatrix} + \begin{pmatrix}
\mathbf{0} \\
\mathbf{t}_{1} \\
\mathbf{0} \\
 \vdots \\
\mathbf{0} \\
 \vdots \\
\end{pmatrix} + \begin{pmatrix}
\mathbf{0} \\
\mathbf{0} \\
\mathbf{t}_{2} \\
 \vdots \\
\mathbf{0} \\
 \vdots \\
\end{pmatrix} + \cdots + \begin{pmatrix}
\mathbf{0} \\
\mathbf{0} \\
\mathbf{0} \\
 \vdots \\
\mathbf{t}_{s} \\
 \vdots \\
\end{pmatrix}
\end{align*}
これに倣って、そのtensor代数$T(V)$の任意の元$\left( \mathbf{t}_{r} \right)_{r \in \mathbb{N} \cup \left\{ 0 \right\}}$に対し、次式のように定義されよう。
\begin{align*}
\left( \mathbf{t}_{r} \right)_{r \in \mathbb{N} \cup \left\{ 0 \right\}} = \begin{pmatrix}
\mathbf{t}_{0} \\
\mathbf{0} \\
\mathbf{0} \\
 \vdots \\
\end{pmatrix} + \begin{pmatrix}
\mathbf{0} \\
\mathbf{t}_{1} \\
\mathbf{0} \\
 \vdots \\
\end{pmatrix} + \begin{pmatrix}
\mathbf{0} \\
\mathbf{0} \\
\mathbf{t}_{2} \\
 \vdots \\
\end{pmatrix} + \cdots
\end{align*}
\end{dfn}
\begin{dfn}
体$K$上の$n$次元vector空間$V$が与えられたとき、そのtensor代数$T(V)$の任意の元々$\left( \mathbf{t}_{r} \right)_{r \in \mathbb{N} \cup \left\{ 0 \right\}}$、$\left( \mathbf{u}_{r} \right)_{r \in \mathbb{N} \cup \left\{ 0 \right\}}$に対し、次式のように積$\otimes$が定義されよう。
\begin{align*}
\left( \mathbf{t}_{r} \right)_{r \in \mathbb{N} \cup \left\{ 0 \right\}} \otimes \left( \mathbf{u}_{r} \right)_{r \in \mathbb{N} \cup \left\{ 0 \right\}} = \sum_{u \in \mathbb{N} \cup \left\{ 0 \right\}} {\sum_{s + t = u} \begin{pmatrix}
\left( \mathbf{0} \right)_{r \in \varLambda_{s + t}} \\
\mathbf{t}_{s} \otimes \mathbf{u}_{t} \\
\left( \mathbf{0} \right)_{r \in \mathbb{N} \setminus \varLambda_{s + t}} \\
\end{pmatrix}}
\end{align*}
\end{dfn}
\begin{thm}\label{2.4.10.4}
体$K$上の$n$次元vector空間$V$が与えられたとき、そのtensor代数$T(V)$で先ほどの積$\otimes$が定義され、さらに、tensor積について、次のことが成り立つとみなされれば、
\begin{itemize}
\item
  $\left( \mathbf{t} \otimes \mathbf{u} \right) \otimes \mathbf{v} = \mathbf{t} \otimes \left( \mathbf{u} \otimes \mathbf{v} \right)$が成り立つ、即ち、結合法則が成り立つ。
\item
  その体$K$の元$k$とのtensor積について、$k \otimes \mathbf{t} = \mathbf{t} \otimes k = k\mathbf{t}$が成り立つ、即ち、scalar倍になっている。
\end{itemize}
そのtensor代数$T(V)$は線形環をなす\footnote{結構気力なくしましたので、まあまあ雑な証明になってしまいました…。申し訳ありません…。}。
\end{thm}
\begin{proof}
体$K$上の$n$次元vector空間$V$が与えられたとき、そのtensor代数$T(V)$で先ほどの積$\otimes$が定義され、さらに、tensor積について、次のことが成り立つとみなされよう。
\begin{itemize}
\item
  $\left( \mathbf{t} \otimes \mathbf{u} \right) \otimes \mathbf{v} = \mathbf{t} \otimes \left( \mathbf{u} \otimes \mathbf{v} \right)$が成り立つ、即ち、結合法則が成り立つ。
\item
  その体$K$の元$k$とのtensor積について、$k \otimes \mathbf{t} = \mathbf{t} \otimes k = k\mathbf{t}$が成り立つ、即ち、scalar倍になっている。
\end{itemize}
このとき、その組$\left( T(V), + \right)$は明らかに可換群をなす。さらに、仮定より、その積$\otimes$について、結合的で、vector$\begin{pmatrix}
1 \\
\mathbf{0} \\
\mathbf{0} \\
 \vdots \\
\end{pmatrix}$が考えられれば、これがその積$\otimes$の単位元となっている。最後に、そのtensor積が双線形的であることに注意すれば、分配的であることもすぐわかる。
\end{proof}
\section*{工事中}
\begin{thebibliography}{50}
  \bibitem{1}
  佐武一郎, 線型代数学, 裳華房, 1958. 第53版 p219-220 ISBN4-7853-1301-3
\end{thebibliography}
\end{document}
