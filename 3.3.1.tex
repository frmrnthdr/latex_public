\documentclass[dvipdfmx]{jsarticle}
\setcounter{section}{3}
\setcounter{subsection}{0}
\usepackage{amsmath,amsfonts,amssymb,array,comment,mathtools,url,docmute}
\usepackage{longtable,booktabs,dcolumn,tabularx,mathtools,multirow,colortbl,xcolor}
\usepackage[dvipdfmx]{graphics}
\usepackage{bmpsize}
\usepackage{amsthm}
\usepackage{enumitem}
\setlistdepth{20}
\renewlist{itemize}{itemize}{20}
\setlist[itemize]{label=•}
\renewlist{enumerate}{enumerate}{20}
\setlist[enumerate]{label=\arabic*.}
\setcounter{MaxMatrixCols}{20}
\setcounter{tocdepth}{3}
\newcommand{\rotin}{\text{\rotatebox[origin=c]{90}{$\in $}}}
\newcommand{\amap}[6]{\text{\raisebox{-0.7cm}{\begin{tikzpicture} 
  \node (a) at (0, 1) {$\textstyle{#2}$};
  \node (b) at (#6, 1) {$\textstyle{#3}$};
  \node (c) at (0, 0) {$\textstyle{#4}$};
  \node (d) at (#6, 0) {$\textstyle{#5}$};
  \node (x) at (0, 0.5) {$\rotin $};
  \node (x) at (#6, 0.5) {$\rotin $};
  \draw[->] (a) to node[xshift=0pt, yshift=7pt] {$\textstyle{\scriptstyle{#1}}$} (b);
  \draw[|->] (c) to node[xshift=0pt, yshift=7pt] {$\textstyle{\scriptstyle{#1}}$} (d);
\end{tikzpicture}}}}
\newcommand{\twomaps}[9]{\text{\raisebox{-0.7cm}{\begin{tikzpicture} 
  \node (a) at (0, 1) {$\textstyle{#3}$};
  \node (b) at (#9, 1) {$\textstyle{#4}$};
  \node (c) at (#9+#9, 1) {$\textstyle{#5}$};
  \node (d) at (0, 0) {$\textstyle{#6}$};
  \node (e) at (#9, 0) {$\textstyle{#7}$};
  \node (f) at (#9+#9, 0) {$\textstyle{#8}$};
  \node (x) at (0, 0.5) {$\rotin $};
  \node (x) at (#9, 0.5) {$\rotin $};
  \node (x) at (#9+#9, 0.5) {$\rotin $};
  \draw[->] (a) to node[xshift=0pt, yshift=7pt] {$\textstyle{\scriptstyle{#1}}$} (b);
  \draw[|->] (d) to node[xshift=0pt, yshift=7pt] {$\textstyle{\scriptstyle{#2}}$} (e);
  \draw[->] (b) to node[xshift=0pt, yshift=7pt] {$\textstyle{\scriptstyle{#1}}$} (c);
  \draw[|->] (e) to node[xshift=0pt, yshift=7pt] {$\textstyle{\scriptstyle{#2}}$} (f);
\end{tikzpicture}}}}
\renewcommand{\thesection}{第\arabic{section}部}
\renewcommand{\thesubsection}{\arabic{section}.\arabic{subsection}}
\renewcommand{\thesubsubsection}{\arabic{section}.\arabic{subsection}.\arabic{subsubsection}}
\everymath{\displaystyle}
\allowdisplaybreaks[4]
\usepackage{vtable}
\theoremstyle{definition}
\newtheorem{thm}{定理}[subsection]
\newtheorem*{thm*}{定理}
\newtheorem{dfn}{定義}[subsection]
\newtheorem*{dfn*}{定義}
\newtheorem{axs}[dfn]{公理}
\newtheorem*{axs*}{公理}
\renewcommand{\headfont}{\bfseries}
\makeatletter
  \renewcommand{\section}{%
    \@startsection{section}{1}{\z@}%
    {\Cvs}{\Cvs}%
    {\normalfont\huge\headfont\raggedright}}
\makeatother
\makeatletter
  \renewcommand{\subsection}{%
    \@startsection{subsection}{2}{\z@}%
    {0.5\Cvs}{0.5\Cvs}%
    {\normalfont\LARGE\headfont\raggedright}}
\makeatother
\makeatletter
  \renewcommand{\subsubsection}{%
    \@startsection{subsubsection}{3}{\z@}%
    {0.4\Cvs}{0.4\Cvs}%
    {\normalfont\Large\headfont\raggedright}}
\makeatother
\makeatletter
\renewenvironment{proof}[1][\proofname]{\par
  \pushQED{\qed}%
  \normalfont \topsep6\p@\@plus6\p@\relax
  \trivlist
  \item\relax
  {
  #1\@addpunct{.}}\hspace\labelsep\ignorespaces
}{%
  \popQED\endtrivlist\@endpefalse
}
\makeatother
\renewcommand{\proofname}{\textbf{証明}}
\usepackage{tikz,graphics}
\usepackage[dvipdfmx]{hyperref}
\usepackage{pxjahyper}
\hypersetup{
 setpagesize=false,
 bookmarks=true,
 bookmarksdepth=tocdepth,
 bookmarksnumbered=true,
 colorlinks=false,
 pdftitle={},
 pdfsubject={},
 pdfauthor={},
 pdfkeywords={}}
\begin{document}
%\hypertarget{ux74b0}{%
\subsection{環}%\label{ux74b0}}
%\hypertarget{ux74b0-1}{%
\subsubsection{環}%\label{ux74b0-1}}
\begin{axs}[環の公理]
空集合でない集合$R$に対し2つの算法それぞれ加法$+ :R \times R \rightarrow R;(a,b) \mapsto a + b$、乗法$\cdot :R \times R \rightarrow R;(a,b) \mapsto ab$が与えられたとする。このとき、次の条件たちを満たす集合$R$を環という\footnote{ここから先は心象するのがほぼ不可能な分野になりますので、定義をよく読んでおくことをお勧めします。}。
\begin{itemize}
\item
  集合$R$は加法について可換群$(R, + )$をなす。
\item
  $\forall a,b,c \in R$に対し、$(ab)c = a(bc)$が成り立つ、即ち、乗法について結合的である。
\item
  $\exists e \in R\forall a \in R$に対し、$ae = ea = a$が成り立つ、即ち、乗法について集合$R$の単位元$e$が存在する。
\item
  $\forall a,b,c \in R$に対し、$a(b + c) = ab + ac$かつ$(a + b)c = ac + bc$が成り立つ、即ち、乗法は加法に対して両側から分配的である。
\end{itemize}
さらに、次の条件も満たす環$R$を特に可換環という。
\begin{itemize}
\item
  $\forall a,b \in R$に対し、$ab = ba$が成り立つ、即ち、乗法は可換的である。
\end{itemize}
\end{axs}
\begin{dfn}
可換群$(R, + )$において、その単位元$1_{(R, + )}$を零元といい$0$と、逆元$a^{- 1}$を$- a$と、$\forall n \in \mathbb{Z}$に対し、元$a^{n}$を$na$と、乗法についての単位元$e$を$1$と書く。
\end{dfn}
\begin{thm}\label{3.3.1.1}
環$R$が与えられたとき、$\forall a \in R$に対し、$a0 = 0a = 0$が成り立つ。
\end{thm}
\begin{proof} 環$R$について、$\forall a \in R$に対し、次のようになるかつ、
\begin{align*}
0 &= a0 - a0\\
&= a(0 + 0) - a0\\
&= a0 + a0 - a0 = a0
\end{align*}
次のようになるので、
\begin{align*}
0 &= 0a - 0a\\
&= (0 + 0)a - 0a\\
&= 0a + 0a - 0a = 0a
\end{align*}
$a0 = 0a = 0$が成り立つ。
\end{proof}
\begin{dfn}
環$R$について、$0 = 1$が成り立つとき、$\forall a \in R$に対し、$a = 1a = 0a = 0$が成り立ち$R = \left\{ 0 \right\}$が得られる。これを零環という。以下、環の元が2つ以上現れるのであれば、その環は零環でないので、断りがない場合、そうする。
\end{dfn}
\begin{dfn}
環$R$について、$\exists a,b \in R$に対し、$a \neq 0$かつ$b \neq 0$が成り立つかつ、$ab = 0$が成り立つなら、それらの元々$a$、$b$をそれぞれ左零因子、右零因子といい、あわせて零因子という。これをもたない可換環を、即ち、その環$R$が可換環で、$\forall a,b \in R$に対し、$a \neq 0$かつ$b \neq 0$が成り立つなら、$ab \neq 0$が成り立つような環を整域という。
\end{dfn}
\begin{dfn}\label{体の定義}
環$R$について、$\exists a,b \in R$に対し、$ab = 1$が成り立つなら、その元$a$を環$R$の可逆元、単元といい、その元$b$を逆元といい、後に示すように$a^{- 1}$、$\frac{1}{a}$などと書くことができる。以下、その元$a^{-1}$はその群$\left(R,+\right)$における逆元ではなくその可逆元$a$の積における逆元を意味するものとする。これにより、可逆元からなる集合は乗法について群をなし、$0$以外の元全てが可逆元であるような環を斜体といい、乗法について可換的な斜体を体といい、可換的でない斜体を非可換体という。
\end{dfn}\par
斜体を体、体を可換体というときもある。
\begin{thm}\label{3.3.1.2}
環$R$の可逆元について、次のことが成り立つ。
\begin{itemize}
\item
  その環$R$が零環でなくその環$R$の元$a$が可逆元なら、これは$0$でない。
\item 
  その環$R$の元$a$が可逆元なら、一意的に逆元${a}^{-1}$が定まる。
\item
  その環$R$が斜体であるなら、零因子をもたない。
\end{itemize}
\end{thm}
\begin{proof}
環$R$について、$a \in R$が成り立ちその環$R$が零環でなくその元$a$が可逆元であり${a}^{-1} \in R$なる元${a}^{-1}$を$a$の逆元とする。$a = 0$が成り立つなら、$a{a}^{-1} = 0{a}^{-1} = 0 \neq 1$が成り立つので、可逆元の定義に矛盾する。よって$a \neq 0$が成り立つ。\par
また、その環$R$の元$a$が可逆元でその元$a^{- 1}$でないその元$a$の逆元$b$が与えられたとするとき、次式が成り立つので、
\begin{align*}
{a}^{-1} &= {a}^{-1}1\\
&= {a}^{-1}(ab)\\
&= \left( {a}^{-1}a \right)b\\
&= 1b = b
\end{align*}
仮定に矛盾する。よって、一意的に逆元${a}^{-1}$が定まる。\par
環$R$が斜体であるなら、$0$以外の元全てが可逆元であるので、$\forall a,b \in R$に対し、$a \neq 0$かつ$b \neq 0$が成り立つなら、${a}^{-1},{b}^{-1} \in R$が成り立ち、したがって、次のようになる。
\begin{align*}
ab{b}^{-1}{a}^{-1} &= a1{a}^{-1}\\
&= a{a}^{-1} = 1\\
{b}^{-1}{a}^{-1}ab &= {b}^{-1}1b\\
&= {b}^{-1}b = 1
\end{align*}
したがって、その元$ab$は可逆元であることになるので、$ab \neq 0$が成り立つ。ゆえに、その環$R$は零因子をもたない。
\end{proof}
\begin{thm}\label{3.3.1.3}
環$R$の性質として、次のことが成り立つ。
\begin{itemize}
  \item $\forall a\in R$に対し、$-(-a) =a$が成り立つ。
  \item $\forall a,b\in R$に対し、$a(-b)=(-a)b=-ab$が成り立つ。
  \item $\forall a,b\in R$に対し、$(-a)(-b)=ab$が成り立つ。
  \item $\forall a,b\in R$に対し、$a=0$または$b=0$が成り立つなら、$ab=0$が成り立つ。
  \item $\forall a,b\in R$に対し、$a=0$かつ$b=0$が成り立つなら、$a^2 +b^2 =0$が成り立つ。
  \item $\forall a\in R$に対し、その元$a$が可逆元なら、その元$-a$も可逆元で$(-a)^{-1} =-a^{-1}$が成り立つ。
  \item $\forall a\in R$に対し、それらの元々$a$、$b$が可逆元なら、その元$ab$も可逆元で$(ab)^{-1} =b^{-1} a^{-1}$が成り立つ。
\end{itemize}
\end{thm}
\begin{proof} 環$R$が与えられたとき、$\forall a\in R$に対し、次のようになる。
\begin{align*}
- ( - a) &= 0 - ( - a)\\
&= a - a - ( - a)\\
&= a + ( - a) - ( - a)\\
&= a + 0 = a
\end{align*}\par
$\forall a,b\in R$に対し、次のようになる。
\begin{align*}
a( - b) &= 0 + a( - b)\\
&= - ab + ab + a( - b)\\
&= - ab + a\left( b + ( - b) \right)\\
&= - ab + a0\\
&= - ab + 0\\
&= - ab
\end{align*}\par
$\forall a,b\in R$に対し、次のようになる。
\begin{align*}
( - a)b &= ( - a)b + 0\\
&= ( - a)b + ab - ab\\
&= \left( ( - a) + a \right)b - ab\\
&= 0b - ab\\
&= a - ab\\
&= - ab
\end{align*}\par
$\forall a,b\in R$に対し、次のようになる。
\begin{align*}
( - a)( - b) &= ( - a)( - b) + 0\\
&= ( - a)( - b) + ( - a)b - ( - a)b\\
&= ( - a)\left( ( - b) + b \right) - ( - ab)\\
&= ( - a)0 + ab\\
&= 0 + ab\\
&= ab
\end{align*}
$\forall a,b\in R$に対し、$a=0$または$b=0$が成り立つなら、次のようになるので、
\begin{align*}
a = 0 \vee b = 0 &\Rightarrow ab = 0 \vee ab = 0\\
&\Leftrightarrow ab = 0
\end{align*}
$ab=0$が成り立つ。\par
$\forall a,b\in R$に対し、$a=0$かつ$b=0$が成り立つなら、次のようになるので、
\begin{align*}
a = 0 \land b = 0 &\Rightarrow a^{2} = 0 \land b^{2} = 0\\
&\Rightarrow a^{2} + b^{2} = 0
\end{align*}
$a^2 +b^2 =0$が成り立つ。\par
$\forall a\in R$に対し、その元$a$が可逆元なら、次のようになるので、
\begin{align*}
\left( -a^{-1} \right) \left(-a\right) &= a^{-1} a =1\\
\left( -a\right) \left(-a^{-1} \right) &= aa^{-1} =1
\end{align*}
その元$-a$も可逆元で$(-a)^{-1} =-a^{-1}$が成り立つ。\par
$\forall a\in R$に対し、それらの元々$a$、$b$が可逆元なら、次のようになるので、
\begin{align*}
\left( b^{-1} a^{-1} \right) \left(ab\right) &= b^{-1} \left( a^{-1} a\right) b \\
&= b^{-1} 1 b =b^{-1} b =1\\
\left(ab\right)\left( b^{-1} a^{-1} \right) &= a\left( bb^{-1} \right) a^{-1} \\
&=a1a^{-1} =aa^{-1} =1
\end{align*}
その元$ab$も可逆元で$(ab)^{-1} =b^{-1} a^{-1}$が成り立つ。
\end{proof}
\subsubsection{環の例}
\begin{dfn}
環$R$が与えられたとき、空集合でない集合$S$を用いて$\forall f,g \in \mathfrak{F}(S,R)$に対し、次のように写像$f + g$、$fg$が定義される。
\begin{itemize}
\item
  $\forall a \in S$に対し、$(f + g)(a) = f(a) + g(a)$が成り立つ。
\item
  $\forall a \in S$に対し、$(fg)(a) = f(a)g(a)$が成り立つ。
\end{itemize}
\end{dfn}
\begin{thm}\label{3.3.1.4}
環$R$が与えられたとき、空集合でない集合$S$を用いた集合$\mathfrak{F}(S,R)$は環であり次式のように与えられる。
\begin{align*}
0&:S \rightarrow R;a \mapsto 0\\
1&:S \rightarrow R;a \mapsto 1\\
- f&:S \rightarrow R;a \mapsto - f(a)
\end{align*}
\end{thm}
\begin{proof}
環$R$が与えられたとき、空集合でない集合$S$を用いた集合$\mathfrak{F}(S,R)$について、次のようにおかれれば、
\begin{align*}
0&:S \rightarrow R;a \mapsto 0\\
1&:S \rightarrow R;a \mapsto 1\\
- f&:S \rightarrow R;a \mapsto - f(a)
\end{align*}
次のようになる。
\begin{align*}
\left( (f + g) + h \right)(a) &= (f + g)(a) + h(a)\\
&= \left( f(a) + g(a) \right) + h(a)\\
&= f(a) + \left( g(a) + h(a) \right)\\
&= f(a) + (g + h)(a)\\
&= \left( f + (g + h) \right)(a)\\
(f + 0)(a) &= f(a) + 0(a)\\
&= f(a) + 0 = f(a)\\
(0 + f)(a) &= 0(a) + f(a)\\
&= 0 + f(a) = f(a)\\
(f - f)(a) &= f(a) + ( - f)(a)\\
&= f(a) - f(a) = 0\\
( - f + f)(a) &= ( - f)(a) + f(a)\\
&= - f(a) + f(a) = 0\\
(f + g)(a) &= f(a) + g(a)\\
&= g(a) + f(a)\\
&= (g + f)(a)
\end{align*}
したがって、その組$\left( \mathfrak{F}(S,R), + \right)$は可換群をなす。\par
さらに、次のようになるので、
\begin{align*}
\left( (fg)h \right)(a) &= (fg)(a)h(a)\\
&= \left( f(a)g(a) \right)h(a)\\
&= f(a)\left( g(a)h(a) \right)\\
&= f(a)(gh)(a)\\
&= \left( f(gh) \right)(a)\\
(f1)(a) &= f(a)1(a)\\
&= f(a)1 = f(a)\\
(1f)(a) &= 1(a)f(a)\\
&= 1f(a) = f(a)\\
\left( f(g + h) \right)(a) &= f(a)(g + h)(a)\\
&= f(a)\left( g(a) + h(a) \right)\\
&= f(a)g(a) + f(a)h(a)\\
&= (fg)(a) + (fh)(a)\\
&= (fg + fh)(a)\\
\left( (f + g)h \right)(a) &= (f + g)(a)h(a)\\
&= \left( f(a) + g(a) \right)h(a)\\
&= f(a)h(a) + g(a)h(a)\\
&= (fh)(a) + (gh)(a)\\
&= (fh + gh)(a)
\end{align*}
よって、その集合$\mathfrak{F}(S,R)$は環である。
\end{proof}
\begin{thm}\label{3.3.1.5}
可換群$(G,*)$での自己準同型写像$f:G \rightarrow G$、即ち、$\forall a,b \in G$に対し、$f(a*b) = f(a)*f(b)$なる写像$f$全体の集合$\mathrm{end}(G,*)$が与えられたとき\footnote{自己準同型写像を英語でいうとendomorphismというそうです。終わりのendとは全く関係ございません。}、算法$*$、合成$\circ$がそれぞれ加法、乗法とみなされれば、その集合$\mathrm{end}(G,*)$は環であり次式のように与えられる。なお、$1_{(G,*)}$はその群$(G,*)$の単位元である。
\begin{align*}
0&:S \rightarrow R;a \mapsto 1_{(G,*)}\\
1&:S \rightarrow R;a \mapsto a\\
- f&:S \rightarrow R;a \mapsto {f(a)}^{- 1}
\end{align*}
\end{thm}
\begin{dfn}
この集合$\mathrm{end}(G,*)$をその可換群$(G,*)$の自己準同型環という。
\end{dfn}
\begin{proof}
可換群$(G,*)$での自己準同型写像$f:G \rightarrow G$、即ち、$\forall a,b \in G$に対し、$f(a*b) = f(a)*f(b)$なる写像$f$全体の集合$\mathrm{end}(G,*)$が与えられたとき、$\forall a,b \in G\forall f,g,h \in \mathfrak{F}(S,R)$に対し、次のようになるので、
\begin{align*}
(f*g)(a*b) &= f(a*b)*g(a*b)\\
&= f(a)*f(b)*g(a)*g(b)\\
&= f(a)*g(a)*f(b)*g(b)\\
&= (f*g)(a)*(f*g)(b)\\
f \circ g(a*b) &= f\left( g(a*b) \right)\\
&= f\left( g(a)*g(b) \right)\\
&= f\left( g(a) \right)*f\left( g(b) \right)\\
&= f \circ g(a)*f \circ g(b)
\end{align*}
$f*g,f \circ g \in \mathrm{end}(G,*)$が成り立つ。\par
さらに、次式のように与えられると、
\begin{align*}
0&:S \rightarrow R;a \mapsto 1_{(G,*)}\\
1&:S \rightarrow R;a \mapsto a\\
- f&:S \rightarrow R;a \mapsto {f(a)}^{- 1}
\end{align*}
次のようになるので、
\begin{align*}
\left( (f*g)*h \right)(a) &= (f*g)(a)*h(a)\\
&= \left( f(a)*g(a) \right)*h(a)\\
&= f(a)*\left( g(a)*h(a) \right)\\
&= f(a)*(g*h)(a)\\
&= \left( f*(g*h) \right)(a)\\
(f*0)(a) &= f(a)*0(a)\\
&= f(a)*1_{(G,*)} = f(a)\\
(0*f)(a) &= 0(a)*f(a)\\
&= 1_{(G,*)}*f(a) = f(a)\\
\left( f*( - f) \right)(a) &= f(a)*( - f)(a)\\
&= f(a)*{f(a)}^{- 1} = 1_{(G,*)}\\
\left( ( - f)*f \right)(a) &= ( - f)(a)*f(a)\\
&= {f(a)}^{- 1}*f(a) = 1_{(G,*)}\\
(f*g)(a) &= f(a)*g(a)\\
&= g(a)*f(a)\\
&= (g*f)(a)
\end{align*}
その組$\left( \mathrm{end}(G,*),* \right)$は可換群をなす。\par
さらに、合成$\circ$が結合的なのはいうまでもなく、次のようになるので、
\begin{align*}
f \circ 1(a) &= f\left( 1(a) \right) = f(a)\\
1 \circ f(a) &= 1\left( f(a) \right) = f(a)\\
f \circ (g*h)(a) &= f\left( (g*h)(a) \right)\\
&= f\left( g(a)*h(a) \right)\\
&= f\left( g(a) \right)*f\left( h(a) \right)\\
&= f \circ g(a)*f \circ h(a)\\
&= (f \circ g*f \circ h)(a)\\
(f*g) \circ h(a) &= (f*g)\left( h(a) \right)\\
&= f\left( h(a) \right)*g\left( h(a) \right)\\
&= f \circ h(a)*g \circ h(a)\\
&= (f \circ h*g \circ h)(a)
\end{align*}
算法$*$、合成$\circ$がそれぞれ加法、乗法とみなされれば、その集合$\mathrm{end}(G,*)$は環である。
\end{proof}
\begin{dfn}
環$R$が与えられたとき、$\forall a \in R$に対し、$a^{2} = a$が成り立つようなものをBoole環という。
\end{dfn}
\begin{thm}\label{3.3.1.6}
Boole環は可換環である。
\end{thm}
\begin{proof}
Boole環が与えられたとき、$\forall a,b \in R$に対し、次のようになる。
\begin{align*}
a + b &= (a + b)^{2}\\
&= (a + b)(a + b)\\
&= (a + b)a + (a + b)b\\
&= a^{2} + ba + ab + b^{2}\\
&= a + b + ab + ba
\end{align*}
したがって、次のようになる。
\begin{align*}
ab &= a + b + ab + ba - a - b - ba\\
&= a + b - a - b - ba\\
&= - ba
\end{align*}
特に、$b = 1$とすれば、$a = - a$が得られる。これにより、次のようになる。
\begin{align*}
ab &= - ba\\
&= b( - a)\\
&= ba
\end{align*}
\end{proof}
\begin{thm}\label{3.3.1.7}
集合$S$が与えられたとき、次のように和と積が定義されるとする。
\begin{align*}
+&:\mathfrak{P}(S)\mathfrak{\times P}(S)\mathfrak{\rightarrow P}(S);(A,B) \mapsto (A \cup B) \setminus (A \cap B)\\
\cdot &:\mathfrak{P}(S)\mathfrak{\times P}(S)\mathfrak{\rightarrow P}(S);(A,B) \mapsto A \cap B
\end{align*}
このとき、その集合$\mathfrak{P}(S)$はBoole環であり次のようになる。
\begin{align*}
0 = \emptyset,\ \ 1 = S,\ \  - A = A
\end{align*}
\end{thm}
\begin{proof}
集合$S$が与えられたとき、次のように和と積が定義されるとする。
\begin{align*}
+&:\mathfrak{P}(S)\mathfrak{\times P}(S)\mathfrak{\rightarrow P}(S);(A,B) \mapsto (A \cup B) \setminus (A \cap B)\\
\cdot &:\mathfrak{P}(S)\mathfrak{\times P}(S)\mathfrak{\rightarrow P}(S);(A,B) \mapsto A \cap B
\end{align*}\par
このとき、$\forall A,B,C \in \mathfrak{P}(S)$に対し、次のようおかれると、
\begin{align*}
D &= A \cap B \cap C\\
A' &= A \cap B \setminus D\\
B' &= B \cap C \setminus D\\
C' &= C \cap A \setminus D\\
A'' &= A \setminus \left( A' \sqcup D \sqcup C' \right)\\
B'' &= B \setminus \left( B' \sqcup D \sqcup A' \right)\\
C' &= C \setminus \left( C' \sqcup D \sqcup B' \right)
\end{align*}
次のようになるかつ、
\begin{align*}
(A + B) + C &= \left( (A \cup B) \setminus (A \cap B) \cup C \right) \setminus \left( (A \cup B) \setminus (A \cap B) \cap C \right)\\
&= \left( \left( A'' \sqcup C' \sqcup B'' \sqcup B' \sqcup A' \sqcup D \right) \setminus \left( A' \sqcup D \right) \cup \left( C' \sqcup C'' \sqcup B' \sqcup D \right) \right) \\
&\quad \setminus \left( \left( A'' \sqcup C' \sqcup B'' \sqcup B' \sqcup A' \sqcup D \right) \setminus \left( A' \sqcup D \right) \cap \left( C' \sqcup C'' \sqcup B' \sqcup D \right) \right)\\
&= \left( \left( A'' \sqcup C' \sqcup B'' \sqcup B' \right) \cup \left( C' \sqcup C'' \sqcup B' \sqcup D \right) \right) \\
&\quad \setminus \left( \left( A'' \sqcup C' \sqcup B'' \sqcup B' \right) \cap \left( C' \sqcup C'' \sqcup B' \sqcup D \right) \right)\\
&= \left( A'' \sqcup C' \sqcup B'' \sqcup B' \sqcup C'' \sqcup D \right) \setminus \left( C' \sqcup B' \right)\\
&= A'' \sqcup B'' \sqcup C'' \sqcup D
\end{align*}
次のようになるので、
\begin{align*}
A + (B + C) &= \left( A \cup (B \cup C) \setminus (B \cap C) \right) \setminus \left( A \cap (B \cup C) \setminus (B \cap C) \right)\\
&= \left( \left( A' \sqcup A'' \sqcup C' \sqcup D \right) \cup \left( B'' \sqcup A' \sqcup C'' \sqcup C' \sqcup B' \sqcup D \right) \setminus \left( B' \sqcup D \right) \right) \\
&\quad \setminus \left( \left( A' \sqcup A'' \sqcup C' \sqcup D \right) \cap \left( B'' \sqcup A' \sqcup C'' \sqcup C' \sqcup B' \sqcup D \right) \setminus \left( B' \sqcup D \right) \right)\\
&= \left( \left( A' \sqcup A'' \sqcup C' \sqcup D \right) \cup \left( B'' \sqcup A' \sqcup C'' \sqcup C' \right) \right) \\
&\quad \setminus \left( \left( A' \sqcup A'' \sqcup C' \sqcup D \right) \cap \left( B'' \sqcup A' \sqcup C'' \sqcup C' \right) \right)\\
&= \left( A' \sqcup A'' \sqcup C' \sqcup D \sqcup B'' \sqcup C'' \right) \setminus \left( A' \sqcup C' \right)\\
&= A'' \sqcup B'' \sqcup C'' \sqcup D
\end{align*}
次式が成り立つ。
\begin{align*}
(A + B) + C = A + (B + C)
\end{align*}
また、次のようになるので、
\begin{align*}
A + 0 &= (A \cup \emptyset) \setminus (A \cap \emptyset) = A \setminus \emptyset = A\\
0 + A &= (\emptyset \cup A) \setminus (\emptyset \cap A) = A \setminus \emptyset = A\\
A - A &= A\text{+}A = (A \cup A) \setminus (A \cap A) = A \setminus A = 0\\
- A + A &= A\text{+}A = (A \cup A) \setminus (A \cap A) = A \setminus A = 0\\
A + B &= (A \cup B) \setminus (A \cap B) = (B \cup A) \setminus (B \cap A) = B + A
\end{align*}
その組$\left( \mathfrak{P}(S), + \right)$は可換群をなす。\par
さらに、次のようになるので、
\begin{align*}
(AB)C &= (A \cap B) \cap C\\
&= A \cap (B \cap C)\\
&= A(BC)\\
A1 &= A \cap S = A\\
1A &= S \cap A = A\\
A(B + C) &= A \cap (B \cup C) \setminus (B \cap C)\\
&= \left( A \cap (B \cup C) \right) \setminus (B \cap C)\\
&= \left( (A \cap B) \cup (A \cap C) \right) \setminus (A \cap B \cap C)\\
&= \left( (A \cap B) \cup (A \cap C) \right) \setminus \left( (A \cap B) \cap (A \cap C) \right)\\
&= AB + AC\\
(A + B)C &= (A \cup B) \setminus (A \cap B) \cap C\\
&= \left( (A \cup B) \cap C \right) \setminus (A \cap B)\\
&= \left( (A \cap C) \cup (B \cap C) \right) \setminus (A \cap B \cap C)\\
&= \left( (A \cap C) \cup (B \cap C) \right) \setminus \left( (A \cap C) \cap (B \cap C) \right)\\
&= AC + BC
\end{align*}
その集合$\mathfrak{P}(S)$は環をなす。\par
さらに、$A^{2} = A \cap A = A$が成り立つので、その集合$\mathfrak{P}(S)$はBoole環をなし次のようになる。
\begin{align*}
0 = \emptyset,\ \ 1 = S,\ \  - A = A
\end{align*}
\end{proof}
%\hypertarget{ux4e8cux9805ux5b9aux7406}{%
\subsubsection{二項定理}%\label{ux4e8cux9805ux5b9aux7406}}
\begin{thm}\label{3.3.1.8}
環$R$の元の族たち$\left\{ a_{i} \right\}_{i \in \varLambda_{m}}$、$\left\{ b_{j} \right\}_{j \in \varLambda_{n}}$が与えられたとき、次式が成り立つ。
\begin{align*}
\left( \sum_{i \in \varLambda_{m}} a_{i} \right)\left( \sum_{j \in \varLambda_{n}} b_{j} \right) = \sum_{(i,j) \in \varLambda_{m} \times \varLambda_{n}} {a_{i}b_{j}}
\end{align*}
\end{thm}
\begin{proof}
環$R$の元の族たち$\left\{ a_{i} \right\}_{i \in \varLambda_{m}}$、$\left\{ b_{j} \right\}_{j \in \varLambda_{n}}$が与えられたとき、$n = 1$のとき、数学的帰納法により明らかに次のようになる。
\begin{align*}
\left( \sum_{i \in \varLambda_{m}} a_{i} \right)b_{1} &= \sum_{i \in \varLambda_{m}} {a_{i}b_{1}}\\
&= \sum_{(i,j) \in \varLambda_{m} \times \varLambda_{1}} {a_{i}b_{j}}
\end{align*}
$n = k$のとき、次式が成り立つと仮定しよう。
\begin{align*}
\left( \sum_{i \in \varLambda_{m}} a_{i} \right)\left( \sum_{j \in \varLambda_{k}} b_{j} \right) = \sum_{(i,j) \in \varLambda_{m} \times \varLambda_{k}} {a_{i}b_{j}}
\end{align*}
$n = k + 1$のとき、次のようになる。
\begin{align*}
\left( \sum_{i \in \varLambda_{m}} a_{i} \right)\left( \sum_{j \in \varLambda_{k + 1}} b_{j} \right) &= \left( \sum_{i \in \varLambda_{m}} a_{i} \right)\left( \sum_{j \in \varLambda_{k}} b_{j} + b_{k + 1} \right)\\
&= \left( \sum_{i \in \varLambda_{m}} a_{i} \right)\left( \sum_{j \in \varLambda_{k}} b_{j} \right) + \left( \sum_{i \in \varLambda_{m}} a_{i} \right)b_{k + 1}\\
&= \sum_{(i,j) \in \varLambda_{m} \times \varLambda_{k}} {a_{i}b_{j}} + \sum_{i \in \varLambda_{m}} {a_{i}b_{k + 1}}\\
&= \sum_{i \in \varLambda_{m}} {\sum_{j \in \varLambda_{k}} {a_{i}b_{j}}} + \sum_{i \in \varLambda_{m}} {a_{i}b_{k + 1}}\\
&= \sum_{i \in \varLambda_{m}} \left( \sum_{j \in \varLambda_{k}} {a_{i}b_{j}} + a_{i}b_{k + 1} \right)\\
&= \sum_{i \in \varLambda_{m}} {a_{i}\left( \sum_{j \in \varLambda_{k}} b_{j} + b_{k + 1} \right)}\\
&= \sum_{i \in \varLambda_{m}} {a_{i}\sum_{j \in \varLambda_{k + 1}} b_{j}}\\
&= \sum_{i \in \varLambda_{m}} {\sum_{j \in \varLambda_{k + 1}} {a_{i}b_{j}}}\\
&= \sum_{(i,j) \in \varLambda_{m} \times \varLambda_{k}} {a_{i}b_{j}}
\end{align*}\par
以上、数学的帰納法により次式が成り立つことが示された。
\begin{align*}
\left( \sum_{i \in \varLambda_{m}} a_{i} \right)\left( \sum_{j \in \varLambda_{n}} b_{j} \right) = \sum_{(i,j) \in \varLambda_{m} \times \varLambda_{n}} {a_{i}b_{j}}
\end{align*}
\end{proof}
\begin{dfn}
$k \leq n$なる非負整数に対し、次式のような有理数$\begin{pmatrix}
n \\
k \\
\end{pmatrix}$が定義される。この有理数$\begin{pmatrix}
n \\
k \\
\end{pmatrix}$を二項係数といい${}_{n}\mathrm{C}_{k}$とも書かれることがある。
\begin{align*}
\begin{pmatrix}
n \\
k \\
\end{pmatrix} = \frac{n!}{k!(n - k)!}
\end{align*}
\end{dfn}
\begin{thm}\label{3.3.1.9}
このとき、次のことが成り立つ。
\begin{itemize}
\item
  $k \leq n$なる非負整数に対し、$\begin{pmatrix}
  n \\
  k \\
  \end{pmatrix} = \begin{pmatrix}
  n \\
  n - k \\
  \end{pmatrix}$が成り立つ。
\item
  $1 \leq k \leq n - 1$なる非負整数に対し、$\begin{pmatrix}
  n \\
  k \\
  \end{pmatrix} = \begin{pmatrix}
  n - 1 \\
  k - 1 \\
  \end{pmatrix} + \begin{pmatrix}
  n - 1 \\
  k \\
  \end{pmatrix}$が成り立つ。
\item
  $k \leq n$なる非負整数に対し、$\begin{pmatrix}
  n \\
  k \\
  \end{pmatrix} \in \mathbb{N}$が成り立つ。
\end{itemize}
\end{thm}
\begin{proof} 
$k \leq n$なる非負整数に対し、次のようになる。
\begin{align*}
\begin{pmatrix}
n \\
k \\
\end{pmatrix} &= \frac{n!}{k!(n - k)!}\\
&= \frac{n!}{(n - k)!\left( n - (n - k) \right)!}\\
&= \begin{pmatrix}
n \\
n - k \\
\end{pmatrix}
\end{align*}
$1 \leq k \leq n - 1$なる非負整数に対し、次のようになる。
\begin{align*}
\begin{pmatrix}
n \\
k \\
\end{pmatrix} &= \frac{n!}{k!(n - k)!}\\
&= \frac{(n - 1)!}{(k - 1)!}\frac{n}{k(n - k)!}\\
&= \frac{(n - 1)!}{(k - 1)!}\left( \frac{k}{k(n - k)!} + \frac{n - k}{k(n - k)!} \right)\\
&= \frac{(n - 1)!}{(k - 1)!}\left( \frac{1}{(n - k)!} + \frac{1}{k(n - k - 1)!} \right)\\
&= \frac{(n - 1)!}{(k - 1)!(n - k)!} + \frac{(n - 1)!}{(k - 1)!k(n - k - 1)!}\\
&= \frac{(n - 1)!}{(k - 1)!\left( (n - 1) - (k - 1) \right)!} + \frac{(n - 1)!}{k!\left( (n - 1) - k \right)!}\\
&= \begin{pmatrix}
n - 1 \\
k - 1 \\
\end{pmatrix} + \begin{pmatrix}
n - 1 \\
k \\
\end{pmatrix}
\end{align*}\par
$k \leq n$なる非負整数に対し、$n = 0$のとき、次のようになる。
\begin{align*}
\begin{pmatrix}
0 \\
0 \\
\end{pmatrix} &= \frac{0!}{0!0!} = 1
\end{align*}
$n = 1$のとき、次のようになる。
\begin{align*}
\begin{pmatrix}
1 \\
0 \\
\end{pmatrix} = \frac{1!}{0!1!} = 1,\ \ \begin{pmatrix}
1 \\
1 \\
\end{pmatrix} = \frac{1!}{1!0!} = 1
\end{align*}
ここで、$n = l$のとき、$\begin{pmatrix}
l \\
k \\
\end{pmatrix} \in \mathbb{N}$が成り立つと仮定しよう。$n = l + 1$のとき、$1 \leq k \leq l$が成り立つなら、次のようになる。
\begin{align*}
\begin{pmatrix}
l + 1 \\
k \\
\end{pmatrix} = \begin{pmatrix}
l \\
k - 1 \\
\end{pmatrix} + \begin{pmatrix}
l \\
k \\
\end{pmatrix} \in \mathbb{N}
\end{align*}
$k = 0$が成り立つなら、次のようになる。
\begin{align*}
\begin{pmatrix}
l + 1 \\
0 \\
\end{pmatrix} = \frac{(l + 1)!}{0!(l + 1)!} = 1 \in \mathbb{N}
\end{align*}
$k = l + 1$が成り立つなら、次のようになる。
\begin{align*}
\begin{pmatrix}
l + 1 \\
l + 1 \\
\end{pmatrix} &= \frac{(l + 1)!}{(l + 1)!\left( (l + 1) - (l + 1) \right)!}\\
&= \frac{(l + 1)!}{(l + 1)!0!} = 1 \in \mathbb{N}
\end{align*}
以上、数学的帰納法により$\begin{pmatrix}
n \\
k \\
\end{pmatrix} \in \mathbb{N}$が成り立つことが示された。
\end{proof}
\begin{thm}[二項定理]\label{3.3.1.10}
環$R$について、$\forall a,b \in R\forall n \in \mathbb{N}$に対し、次式が成り立つ。
\begin{align*}
(a + b)^{n} &= \sum_{k \in \varLambda_{n} \cup \left\{ 0 \right\}} {\begin{pmatrix}
n \\
k \\
\end{pmatrix}a^{n - k}b^{k}}
\end{align*}\par
この定理を二項定理という。
\end{thm}
\begin{proof}
環$R$について、$\forall a,b \in R\forall n \in \mathbb{N}$に対し、$n = 1$のとき、次のようになる。
\begin{align*}
a + b &= \frac{1!}{0!1!}a^{1}b^{0} + \frac{1!}{1!0!}a^{0}b^{1}\\
&= \sum_{k \in \varLambda_{1} \cup \left\{ 0 \right\}} {\frac{n!}{k!(n - k)!}a^{n - k}b^{k}}
\end{align*}
$n = l$のとき、次式が成り立つと仮定しよう。
\begin{align*}
(a + b)^{l} = \sum_{k \in \varLambda_{l} \cup \left\{ 0 \right\}} {\begin{pmatrix}
l \\
k \\
\end{pmatrix}a^{l - k}b^{k}}
\end{align*}
$n = l + 1$のとき、次のようになる。
\begin{align*}
(a + b)^{l + 1} &= (a + b)(a + b)^{l}\\
&= (a + b)\sum_{k \in \varLambda_{l} \cup \left\{ 0 \right\}} {\begin{pmatrix}
l \\
k \\
\end{pmatrix}a^{l - k}b^{k}}\\
&= \sum_{k \in \varLambda_{l} \cup \left\{ 0 \right\}} {\begin{pmatrix}
l \\
k \\
\end{pmatrix}a^{l - k + 1}b^{k}} + \sum_{k \in \varLambda_{l} \cup \left\{ 0 \right\}} {\begin{pmatrix}
l \\
k \\
\end{pmatrix}a^{l - k}b^{k + 1}}\\
&= \sum_{k \in \varLambda_{l}} {\begin{pmatrix}
l \\
k \\
\end{pmatrix}a^{(l + 1) - k}b^{k}} + \begin{pmatrix}
l \\
0 \\
\end{pmatrix}a^{l + 1} + \sum_{k \in \varLambda_{l} \cup \left\{ 0 \right\}} {\begin{pmatrix}
l \\
k \\
\end{pmatrix}a^{l - k}b^{k + 1}}\\
&= \begin{pmatrix}
l \\
0 \\
\end{pmatrix}a^{l + 1} + \sum_{k \in \varLambda_{l}} {\begin{pmatrix}
l \\
k \\
\end{pmatrix}a^{(l + 1) - k}b^{k}} + \sum_{k \in \varLambda_{l + 1}} {\begin{pmatrix}
l \\
k - 1 \\
\end{pmatrix}a^{(l + 1) - k}b^{k}}\\
&= a^{l + 1} + \sum_{k \in \varLambda_{l}} {\left( \begin{pmatrix}
l \\
k \\
\end{pmatrix} + \begin{pmatrix}
l \\
k - 1 \\
\end{pmatrix} \right)a^{(l + 1) - k}b^{k}} + b^{l + 1}\\
&= \begin{pmatrix}
l + 1 \\
0 \\
\end{pmatrix}a^{l + 1} + \sum_{k \in \varLambda_{l}} {\begin{pmatrix}
l + 1 \\
k \\
\end{pmatrix}a^{(l + 1) - k}b^{k}} + \begin{pmatrix}
l + 1 \\
l + 1 \\
\end{pmatrix}b^{l + 1}\\
&= \sum_{k \in \varLambda_{l + 1} \cup \left\{ 0 \right\}} {\begin{pmatrix}
l \\
k \\
\end{pmatrix}a^{l - k}b^{k}}
\end{align*}
よって、数学的帰納法により$\forall a,b \in R\forall n \in \mathbb{N}$に対し、次式が成り立つことが示された。
\begin{align*}
(a + b)^{n} = \sum_{k \in \varLambda_{n} \cup \left\{ 0 \right\}} {\begin{pmatrix}
n \\
k \\
\end{pmatrix}a^{n - k}b^{k}}
\end{align*}
\end{proof}
\begin{thm}[多項定理]\label{3.3.1.11}
添数集合$\varLambda_{m}$によって添数づけられた環$R$の元の族$\left\{ a_{i} \right\}_{i \in \varLambda_{m}}$が与えられたとき、$\forall n \in \mathbb{N}$に対し、次式が成り立つ。
\begin{align*}
\left( \sum_{i \in \varLambda_{m}} a_{i} \right)^{n} = \sum_{\scriptsize \begin{matrix} \sum_{i \in \varLambda_{m}} k_{i} = n \\k_{i} \in \mathbb{N} \cup \left\{ 0 \right\} \end{matrix}} {\frac{n!}{\prod_{i \in \varLambda_{m}} {k_{i}!}}\prod_{i \in \varLambda_{m}} a_{i}^{k_{i}}}
\end{align*}\par
この定理を多項定理という。
\end{thm}
\begin{proof}
添数集合$\varLambda_{m}$によって添数づけられた環$R$の元の族$\left\{ a_{i} \right\}_{i \in \varLambda_{m}}$が与えられたとき、$\forall n \in \mathbb{N}$に対し、$m = 1$のときは明らかに次のようになる。
\begin{align*}
\left( \sum_{i \in \varLambda_{1}} a_{i} \right)^{n} &= a_{1}^{n} = \frac{n!}{n!}a_{1}^{n}\\
&= \sum_{k_{1} = n } {\frac{n!}{k_{1}!}a_{1}^{k_{1}}}\\
&= \sum_{\scriptsize \begin{matrix} \sum_{i \in \varLambda_{1}} k_{i} = n \\k_{i} \in \mathbb{N} \cup \left\{ 0 \right\} \end{matrix}} {\frac{n!}{\prod_{i \in \varLambda_{1}} {k_{i}!}}\prod_{i \in \varLambda_{1}} a_{i}^{k_{i}}}
\end{align*}
$m = k$のとき、次式が成り立つと仮定しよう。
\begin{align*}
\left( \sum_{i \in \varLambda_{k}} a_{i} \right)^{n} = \sum_{\scriptsize \begin{matrix} \sum_{i \in \varLambda_{k}} k_{i} = n \\k_{i} \in \mathbb{N} \cup \left\{ 0 \right\} \end{matrix}} {\frac{n!}{\prod_{i \in \varLambda_{k}} {k_{i}!}}\prod_{i \in \varLambda_{k}} a_{i}^{k_{i}}}
\end{align*}
$m = k + 1$のとき、$a = \sum_{i \in \varLambda_{k}} a_{i}$とおかれると、二項定理より次のようになる。
\begin{align*}
\left( \sum_{i \in \varLambda_{k + 1}} a_{i} \right)^{n} &= \left( \sum_{i \in \varLambda_{k}} a_{i} + a_{k + 1} \right)^{n}\\
&= \left( a + a_{k + 1} \right)^{n}\\
&= \sum_{k_{k + 1} \in \varLambda_{n} \cup \left\{ 0 \right\}} {\begin{pmatrix}
n \\
k_{k + 1} \\
\end{pmatrix}a^{n - k_{k + 1}}a_{k + 1}^{k_{k + 1}}}\\
&= \sum_{k_{k + 1} \in \varLambda_{n} \cup \left\{ 0 \right\}} {\frac{n!}{k_{k + 1}!\left( n - k_{k + 1} \right)!}\sum_{\scriptsize \begin{matrix} \sum_{i \in \varLambda_{k}} k_{i} = n - k_{k + 1} \\k_{i} \in \mathbb{N} \cup \left\{ 0 \right\} \end{matrix}} {\frac{\left( n - k_{k + 1} \right)!}{\prod_{i \in \varLambda_{k}} {k_{i}!}}\prod_{i \in \varLambda_{k}} a_{i}^{k_{i}}}a_{k + 1}^{k_{k + 1}}}\\
&= \sum_{k_{k + 1} \in \varLambda_{n} \cup \left\{ 0 \right\}} {\sum_{\scriptsize \begin{matrix} \sum_{i \in \varLambda_{k}} k_{i} = n - k_{k + 1} \\k_{i} \in \mathbb{N} \cup \left\{ 0 \right\} \end{matrix}} {\frac{n!}{k_{k + 1}!\left( n - k_{k + 1} \right)!}\frac{\left( n - k_{k + 1} \right)!}{\prod_{i \in \varLambda_{k}} {k_{i}!}}\prod_{i \in \varLambda_{k + 1}} a_{i}^{k_{i}}}}\\
&= \sum_{\scriptsize \begin{matrix} \sum_{i \in \varLambda_{k}} k_{i} + k_{k + 1} = n \\k_{i} \in \mathbb{N} \cup \left\{ 0 \right\} \\k_{k + 1} \in \varLambda_{n} \cup \left\{ 0 \right\} \end{matrix}} {\frac{\left( n - k_{k + 1} \right)!}{\left( n - k_{k + 1} \right)!}\frac{n!}{\prod_{i \in \varLambda_{k + 1}} {k_{i}!}}\prod_{i \in \varLambda_{k + 1}} a_{i}^{k_{i}}}\\
&= \sum_{\scriptsize \begin{matrix} \sum_{i \in \varLambda_{k + 1}} k_{i} = n \\k_{i} \in \mathbb{N} \cup \left\{ 0 \right\} \end{matrix}} {\frac{n!}{\prod_{i \in \varLambda_{k + 1}} {k_{i}!}}\prod_{i \in \varLambda_{k + 1}} a_{i}^{k_{i}}}
\end{align*}
よって、数学的帰納法により$\forall n \in \mathbb{N}$に対し、次式が成り立つことが示された。
\begin{align*}
\left( \sum_{i \in \varLambda_{m}} a_{i} \right)^{n} = \sum_{\scriptsize \begin{matrix} \sum_{i \in \varLambda_{m}} k_{i} = n \\k_{i} \in \mathbb{N} \cup \left\{ 0 \right\} \end{matrix}} {\frac{n!}{\prod_{i \in \varLambda_{m}} {k_{i}!}}\prod_{i \in \varLambda_{m}} a_{i}^{k_{i}}}
\end{align*}
\end{proof}
%\hypertarget{ux6574ux57df}{%
\subsubsection{整域}%\label{ux6574ux57df}}
\begin{dfn}
環$R$について、$\exists a,b \in R \setminus \left\{ 0 \right\}$に対し、$ab = 0$が成り立つなら、それらの元々$a$、$b$をそれぞれ左零因子、右零因子といい、あわせて零因子という。これをもたない可換環を整域という。
\end{dfn}\par
例えば、集合$\mathbb{Z}$が挙げられる。
\begin{dfn}[定義\ref{体の定義}の再掲]
環$R$について、$\exists a,b \in R$に対し、$ab = 1$が成り立つなら、その元$a$を環$R$の可逆元、単元といい、その元$b$を逆元といい、$a^{- 1}$などと書く。これにより、可逆元からなる集合は乗法について群をなし、$0$以外の元全てが可逆元であるような環を斜体といい、乗法について可換的な斜体を体といい、可換的でない斜体を非可換体という。
\end{dfn}\par
例えば、集合$\mathbb{Q}$、$\mathbb{R}$、$\mathbb{C}$などが挙げられる。斜体を体、体を可換体というときもある。
\begin{thm*}[定理\ref{3.3.1.2}の再掲]
環$R$について、元$a$が可逆元なら、これは$0$でなく一意的に逆元${a}^{-1}$が定まる。
\end{thm*}
\begin{thm}\label{3.3.1.12}
任意の体$K$は整域である。
\end{thm}
\begin{proof}
任意の体$K$について、可逆元からなる集合$K'$は乗法について群をなし、体の定義よりその群$\left( K', \cdot \right)$は可換群で$K' = K \setminus \left\{ 0 \right\}$が成り立つので、$\forall a,b \in K' = K \setminus \left\{ 0 \right\}$に対し、$ab \in K' = K \setminus \left\{ 0 \right\}$が成り立つ。したがって、その体$K$は整域である。
\end{proof}
\begin{thm}\label{3.3.1.13}
有限集合である整域は体である。
\end{thm}
\begin{proof}
有限集合である整域$R$について、$\forall c \in R \setminus \left\{ 0 \right\}$に対し、次式のような写像$f$が与えられたとき、
\begin{align*}
f:R \rightarrow R;a \mapsto ca
\end{align*}
$f(a) = f(b)$が成り立つなら、次のようになるので、
\begin{align*}
0 &= f(a) - f(b)\\
&= ca - cb\\
&= c(a - b)
\end{align*}
ここで、その可換環$R$は零因子をもたないので、$a - b = 0$が成り立つ。したがって、その写像$f$は単射である。ここで、その可換環$R$は有限集合であるから、その写像$f$は全単射である。したがって、$f(a) = ca = 1$なる元$a$が存在する。よって、有限集合である整域は体であることが示された。
\end{proof}
%\hypertarget{ux90e8ux5206ux74b0}{%
\subsubsection{部分環}%\label{ux90e8ux5206ux74b0}}
\begin{dfn}
環$R$が与えられたとき、これの部分集合もまた環となっており、しかも、その環$R$と同じ単位元をもつものをその環$R$の部分環という。特に、これが斜体、体となっているものをそれぞれその環$R$の部分斜体、部分体という。
\end{dfn}
\begin{thm}\label{3.3.1.14}
環$R$の部分集合$R'$が与えられたとき、これがその環$R$の部分環であるならそのときに限り、次のことをいづれも満たす。
\begin{itemize}
\item
  その部分集合$R'$がその環$R$の単位元$1$に属される。
\item
  $\forall a,b \in R'$に対し、$- a,a + b,ab \in R'$が成り立つ。
\end{itemize}
\end{thm}
\begin{proof}
環$R$の部分集合$R'$が与えられたとき、これがその環$R$の部分環であるなら、定義より次のことを満たす。
\begin{itemize}
\item
  その部分集合$R'$がその環$R$の単位元$1$に属される。
\item
  $\forall a,b \in R'$に対し、$- a,a + b,ab \in R'$が成り立つ。
\end{itemize}\par
逆に、その部分集合$R'$が上記のことを満たすなら、$\forall a,b,c \in R'$に対し、$- a \in R'$が成り立つので、$- a + a = 0 \in R'$が成り立つ。これにより、$a,b,c \in R$が成り立ち、次のことを満たすので、
\begin{align*}
(a + b) + c = a + (b + c),\ \ a + 0 = 0 + a,\ \ a + ( - a) = - a + a = 0,\ \ a + b = b + a
\end{align*}
その組$\left( R', + \right)$は可換群をなす。\par
さらに、次のことを満たすので、
\begin{align*}
(ab)c = a(bc),\ \ a1 = 1a = a,\ \ a(b + c) = ab + ac,\ \ (a + b)c = ac + bc
\end{align*}
その集合$R'$は環をなす。
\end{proof}
\begin{thebibliography}{50}
  \bibitem{1}
  松坂和夫, 代数系入門, 岩波書店, 1976. 新装版第2刷 p107-115 ISBN978-4-00-029873-5
  \bibitem{2}
  チャート研究所, チャート式 基礎からの数学II+B, 数研出版, 平成11年. 新課程第6刷 p14 ISBN978-4-410-10585-2
  \bibitem{3}
  理系のための備忘録. "二項係数nCmが整数になることの証明". 理系のための備忘録. \url{https://science-log.com/%E6%95%B0%E5%AD%A6/%E4%BA%8C%E9%A0%85%E4%BF%82%E6%95%B0n%EF%BD%83m%E3%81%8C%E6%95%B4%E6%95%B0%E3%81%AB%E3%81%AA%E3%82%8B%E3%81%93%E3%81%A8%E3%81%AE%E8%A8%BC%E6%98%8E/} (2021-8-20 12:00 閲覧)
\end{thebibliography}
\end{document}
