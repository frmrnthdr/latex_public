\documentclass[dvipdfmx]{jsarticle}
\setcounter{section}{1}
\setcounter{subsection}{0}
\usepackage{amsmath,amsfonts,amssymb,array,comment,mathtools,url,docmute}
\usepackage{longtable,booktabs,dcolumn,tabularx,mathtools,multirow,colortbl,xcolor}
\usepackage[dvipdfmx]{graphics}
\usepackage{bmpsize}
\usepackage{amsthm}
\usepackage{enumitem}
\setlistdepth{20}
\renewlist{itemize}{itemize}{20}
\setlist[itemize]{label=•}
\renewlist{enumerate}{enumerate}{20}
\setlist[enumerate]{label=\arabic*.}
\setcounter{MaxMatrixCols}{20}
\setcounter{tocdepth}{3}
\newcommand{\rotin}{\text{\rotatebox[origin=c]{90}{$\in $}}}
\renewcommand{\thesection}{第\arabic{section}部}
\renewcommand{\thesubsection}{\arabic{section}.\arabic{subsection}}
\renewcommand{\thesubsubsection}{\arabic{section}.\arabic{subsection}.\arabic{subsubsection}}
\everymath{\displaystyle}
\allowdisplaybreaks[4]
\usepackage{vtable}
\theoremstyle{definition}
\newtheorem{thm}{定理}[subsection]
\newtheorem*{thm*}{定理}
\newtheorem{dfn}{定義}[subsection]
\newtheorem*{dfn*}{定義}
\newtheorem{axs}[dfn]{公理}
\newtheorem*{axs*}{公理}
\renewcommand{\headfont}{\bfseries}
\makeatletter
  \renewcommand{\section}{%
    \@startsection{section}{1}{\z@}%
    {\Cvs}{\Cvs}%
    {\normalfont\huge\headfont\raggedright}}
\makeatother
\makeatletter
  \renewcommand{\subsection}{%
    \@startsection{subsection}{2}{\z@}%
    {0.5\Cvs}{0.5\Cvs}%
    {\normalfont\LARGE\headfont\raggedright}}
\makeatother
\makeatletter
  \renewcommand{\subsubsection}{%
    \@startsection{subsubsection}{3}{\z@}%
    {0.4\Cvs}{0.4\Cvs}%
    {\normalfont\Large\headfont\raggedright}}
\makeatother
\makeatletter
\renewenvironment{proof}[1][\proofname]{\par
  \pushQED{\qed}%
  \normalfont \topsep6\p@\@plus6\p@\relax
  \trivlist
  \item\relax
  {
  #1\@addpunct{.}}\hspace\labelsep\ignorespaces
}{%
  \popQED\endtrivlist\@endpefalse
}
\makeatother
\renewcommand{\proofname}{\textbf{証明}}
\usepackage{tikz,graphics}
\usepackage[dvipdfmx]{hyperref}
\usepackage{pxjahyper}
\hypersetup{
 setpagesize=false,
 bookmarks=true,
 bookmarksdepth=tocdepth,
 bookmarksnumbered=true,
 colorlinks=false,
 pdftitle={},
 pdfsubject={},
 pdfauthor={},
 pdfkeywords={}}
\begin{document}
%\hypertarget{ux7fa4}{%
\subsection{群}%\label{ux7fa4}}
%\hypertarget{ux7fa4-1}{%
\subsubsection{群}%\label{ux7fa4-1}}
\begin{axs}[群の公理]
空集合でない集合$G$に対し算法$*:G \times G \rightarrow G;(a,b) \mapsto a*b$が与えられたとする。このとき、次の条件たちを満たす集合$G$と算法$*$を合わせて群といい、集合$G$は算法$*$に対し群をなすといい、$(G,*)$と書く。そのような集合$G$の元の個数が有限なら、その群$(G,*)$は有限群といい、その集合の濃度$\#G$をその群$(G,*)$の位数といい、$o(G,*)$と書く。逆に、その集合$G$の元の個数が無限ならば、その群$(G,*)$は無限群という。単位元$1_{(G,*)}$のみからなる群$\left( \left\{ 1_{(G,*)} \right\},* \right)$を単位群という\footnote{余談ですが、集合$G$の1つの部分集合を$S$、$n$つの部分集合たちのうち1つを$S_{i}$、これの元の1つを$s_{i}$とおき、写像$f:\prod_{i} S_{i} \rightarrow S;\left( s_{i} \right)_{i} \mapsto f\left( s_{i} \right)_{i}$を考えるとき、紛らわしいことに集合$\left\{ f\left( s_{i} \right)_{i} \middle| \forall i\left\lbrack s_{i} \in S_{i} \right\rbrack \right\}$を$f\left( S_{i} \right)_{i}$と表記することがあります…。例えば、$S_{1}*S_{2}$、$a*S_{1}$などといった感じに。}。
\begin{itemize}
\item
  算法$*$について結合的である、即ち、$\forall a,b,c \in G$に対し、$(a*b)*c = a*(b*c)$が成り立つ。
\item
  $\exists b \in G\forall a \in G$に対し、$a*b = b*a = a$が成り立つ。この元$b$をその群$(G,*)$の単位元といい$1_{(G,*)}$などと書く。
\item
  $\forall a \in G\exists b \in G$に対し、$a*b = b*a = 1_{(G,*)}$が成り立つ。この元$b$を$a$の逆元といい、$a^{- 1}$と書く。
\end{itemize}
さらに次の条件も満たす群$(G,*)$を特に可換群、Abel群という。
\begin{itemize}
\item
  算法$*$は可換的である、即ち、$\forall a,b \in G$に対し、$a*b = b*a$が成り立つ。
\end{itemize}
なお、$a*b = b*a$が成り立つような元々$a$、$b$は可換であるという。
\end{axs}
\begin{thm}\label{3.1.1.1}
群$(G,*)$について、その単位元$1_{(G,*)}$、その集合$G$の任意の元$a$の逆元$a^{- 1}$は一意的に存在する。
\end{thm}\par
これはいずれも背理法によって示される。
\begin{proof}
群$(G,*)$において、$\forall a \in G$に対し、$a*1_{(G,*)} = 1_{(G,*)}*a = a$なるその集合$G$の元$1_{(G,*)}$とは異なる、$\forall a \in G$に対し、$a*e = e*a = a$なる元$e$がその集合$G$に存在するとする。このとき、$1_{(G,*)}*e = 1_{(G,*)}$かつ$1_{(G,*)}*e = e$が成り立つので、$1_{(G,*)} = e$が成り立つこととなり、仮定に矛盾する。よって、$\forall a \in G$に対し、$a*1_{(G,*)} = 1_{(G,*)}*a = a$が成り立つようなその単位元$1_{(G,*)}$は一意的に存在する。\par
同様に、$\forall a \in G$に対し、$a*a^{- 1} = a^{- 1}*a = 1_{(G,*)}$なるその集合$G$の元$a^{- 1}$とは異なる$a*b = b*a = 1_{(G,*)}$なる元$b$がその集合$G$に存在するとする。このとき、次のようになり、
\begin{align*}
a^{- 1} &= a^{- 1}*1_{(G,*)}\\
&= a^{- 1}*(a*b)\\
&= \left( a^{- 1}*a \right)*b\\
&= 1_{(G,*)}*b = b
\end{align*}
仮定に矛盾する。よって、$\forall a \in G$に対し、$a*a^{- 1} = a^{- 1}*a = 1_{(G,*)}$となる元$a^{- 1}$が一意的に存在する。
\end{proof}
\begin{thm}[簡易律]\label{3.1.1.2}
群$(G,*)$において、$\forall a,u,v \in G$に対し、次のことが成り立つ。
\begin{itemize}
\item
  $a*u = a*v$が成り立つなら、$u = v$が成り立つ。
\item
  $u*a = v*a$が成り立つなら、$u = v$が成り立つ。
\end{itemize}
\par
この性質を簡易律という。
\end{thm}
\begin{proof}
群$(G,*)$が与えられたとする。$\forall a,u,v \in G$に対し、$a*u = a*v$が成り立つなら、次式が成り立つかつ、
\begin{align*}
a^{- 1}*(a*u) &= \left( a^{- 1}*a \right)*u\\
&= 1_{(G,*)}*u = u
\end{align*}
次式が成り立つので、
\begin{align*}
a^{- 1}*(a*u) &= a^{- 1}*(a*v)\\
&= \left( a^{- 1}*a \right)*v\\
&= 1_{(G,*)}*v = v
\end{align*}
$u = v$が得られる。\par
同様にして、$u*a = v*a$が成り立つなら、$u = v$が成り立つことが示される。
\end{proof}
\begin{thm}\label{3.1.1.3}
群$(G,*)$について、$\forall a,b \in G$に対し、$(a*b)^{- 1} = b^{- 1}*a^{- 1}$が成り立つ。
\end{thm}
\begin{proof}
群$(G,*)$が与えられたとする。$\forall a,b \in G$に対し、$(a*b)*(a*b)^{- 1} = 1_{(G,*)}$が成り立つかつ、次式が成り立つので、
\begin{align*}
(a*b)*\left( b^{- 1}*a^{- 1} \right) &= \left( a*\left( b*b^{- 1} \right) \right)*a^{- 1}\\
&= (a*e)*a^{- 1}\\
&= a*a^{- 1} = 1_{(G,*)}
\end{align*}
$(a*b)*(a*b)^{- 1} = (a*b)*\left( b^{- 1}*a^{- 1} \right)$が得られ、したがって、簡易律により$(a*b)^{- 1} = b^{- 1}*a^{- 1}$が成り立つ。
\end{proof}
\begin{dfn}
群$(G,*)$をなす集合$G$の元$a$について$m,n \in \mathbb{Z}$に対し次式のように記法を定める。
\begin{align*}
a^{m}*a^{n} &= a^{m + n}\\
\left( a^{m} \right)^{n} &= a^{mn}\\
(a*b)^{n} &= a^{n}*b^{n}\ \mathrm{if}\ a*b = b*a
\end{align*}
\end{dfn}
\begin{thm}\label{3.1.1.4}
群$(G,*)$について、$\forall a \in G\forall n \in \mathbb{N}$に対し、次式たちが成り立つ。
\begin{align*}
a^{0} = 1_{(G,*)},\ \ a^{1} = a,\ \ a^{n + 1} = a^{n}*a,\ \ 1_{(G,*)}^{- n} = 1_{(G,*)}^{n} = 1_{(G,*)},\ \ a^{- n} = \left( a^{- 1} \right)^{n}
\end{align*}
\end{thm}
\begin{proof}
群$(G,*)$をなす集合$G$について、$\forall a \in G\forall n \in \mathbb{N}$に対し、次のようになる。
\begin{align*}
a^{0} &= 1_{(G,*)}*a^{0}\\
&= \left( a^{0} \right)^{- 1}*a^{0}*a^{0}\\
&= \left( a^{0} \right)^{- 1}*a^{0 + 0}\\
&= \left( a^{0} \right)^{- 1}*a^{0}\\
&= 1_{(G,*)}\\
a^{1} &= 1_{(G,*)}*a^{1}\\
&= a*a^{- 1}*a^{1}\\
&= a*a^{- 1 + 1}\\
&= a*a^{0}\\
&= a*1_{(G,*)} = a\\
a^{n + 1} &= a^{n}*a^{1}\\
&= a^{n}*a
\end{align*}\par
また、上記の議論より$1_{(G,*)}^{- 1} = 1_{(G,*)}^{0} = 1_{(G,*)}^{1} = 1_{(G,*)}$が成り立つ。$n = k$のとき、$1_{(G,*)}^{k} = 1_{(G,*)}$と仮定しよう。$n = k + 1$のとき、次のようになる。
\begin{align*}
1_{(G,*)}^{k + 1} &= 1_{(G,*)}^{k}*1_{(G,*)}^{1}\\
&= 1_{(G,*)}*1_{(G,*)}\\
&= 1_{(G,*)}
\end{align*}
逆に、$n = k$のとき、$1_{(G,*)}^{- k} = 1_{(G,*)}$と仮定しよう。$n = k + 1$のとき、次のようになる。
\begin{align*}
1_{(G,*)}^{- (k + 1)} &= 1_{(G,*)}^{- k - 1}\\
&= 1_{(G,*)}^{- k}*1_{(G,*)}^{- 1}\\
&= 1_{(G,*)}*1_{(G,*)}\\
&= 1_{(G,*)}
\end{align*}
以上より数学的帰納法によって$\forall n \in \mathbb{Z}$に対し、次式が成り立つ。
\begin{align*}
1_{(G,*)}^{n} = 1_{(G,*)}
\end{align*}\par
また、上記の議論により次のようになる。
\begin{align*}
a^{- n} &= a^{- n}*1_{(G,*)}\\
&= a^{- n}*\left( a^{- 1} \right)^{0}\\
&= a^{- n}*\left( a^{- 1} \right)^{- n + n}\\
&= a^{- n}*\left( a^{- 1} \right)^{- n}*\left( a^{- 1} \right)^{n}\\
&= \left( a*a^{- 1} \right)^{- n}*\left( a^{- 1} \right)^{n}\\
&= 1_{(G,*)}^{- n}*\left( a^{- 1} \right)^{n}\\
&= 1_{(G,*)}*\left( a^{- 1} \right)^{n}\\
&= \left( a^{- 1} \right)^{n}
\end{align*}
\end{proof}
%\hypertarget{ux90e8ux5206ux7fa4}{%
\subsubsection{部分群}%\label{ux90e8ux5206ux7fa4}}
\begin{dfn}
群$(G,*)$をなす集合$G$の部分集合$H$が算法$*$に関して群をなすとき、この集合$H$が算法$*$に関してその群$(G,*)$の部分群$(H,*)$をなすという。
\end{dfn}
\begin{thm}\label{3.1.1.5}
群$(G,*)$をなす集合$G$の部分集合$H$が算法$*$に関して群をなすとき、その集合$H$は群$(G,*)$の単位元$1_{(G,*)}$を含みこれ$1_{(G,*)}$がその群$(H,*)$の単位元である、即ち、次式が成り立つ。
\begin{align*}
1_{(H,*)} = 1_{(G,*)} \in H
\end{align*}
\end{thm}
\begin{proof}
群$(G,*)$をなす集合$G$の部分集合$H$が算法$*$に関して群をなすならば、$a,b \in H$が成り立つなら、$a*b \in H$が成り立つかつ、$a \in H$が成り立つなら、$a^{- 1} \in H$が成り立つので、$a*a^{- 1} \in H$より単位元$1_{(G,*)}$についても$1_{(G,*)} \in H$が成り立つ。\par
また、$\forall a \in H$に対し、$H \subseteq G$が成り立つのであったので、$a \in G$が成り立ち、したがって、次式が成り立つ。
\begin{align*}
1_{(G,*)}*a = a*1_{(G,*)} = a
\end{align*}
よって、その元$1_{(G,*)}$がその群$(H,*)$の単位元である。
\end{proof}
\begin{thm}\label{3.1.1.6}
群$(G,*)$をなす集合$G$の部分集合$H$を用いた組$(H,*)$が次の条件を全て満たすならそのときに限り、その組$(H,*)$が算法$*$に関して部分群$(H,*)$をなす。
\begin{itemize}
\item
  $1_{(G,*)} \in H$が成り立つ。
\item
  $\forall a,b \in H$に対し、$a*b \in H$が成り立つ。
\item
  $\forall a \in H$に対し、$a^{- 1} \in H$が成り立つ。
\end{itemize}
\end{thm}
\begin{proof}
群$(G,*)$をなす集合$G$の部分集合$H$を用いた組$(H,*)$が算法$*$に関して部分群$(H,*)$をなすならば、群の定義と上記の定理より明らかに次のことが成り立つ。
\begin{itemize}
\item
  $1_{(G,*)} \in H$が成り立つ。
\item
  $\forall a,b \in H$に対し、$a*b \in H$が成り立つ。
\item
  $\forall a \in H$に対し、$a^{- 1} \in H$が成り立つ。
\end{itemize}\par
また、上の条件を全て満たすとき、群$(G,*)$が与えられているので、結合律も成り立ち、単位元$1_{(G,*)}$はその群$(H,*)$の単位元でもあるから、その部分集合$H$が算法$*$に関して部分群$(H,*)$をなす。
\end{proof}
\begin{thm}\label{3.1.1.7}
群$(G,*)$をなす集合$G$の部分集合$H$を用いた組$(H,*)$が次の条件を全て満たすならそのときに限り、その組$(H,*)$が算法$*$に関して部分群$(H,*)$をなす。
\begin{itemize}
\item
  集合$H$は空集合でない。
\item
  $\forall a,b \in H$に対し、$a*b \in H$が成り立つ。
\item
  $\forall a \in H$に対し、$a^{- 1} \in H$が成り立つ。
\end{itemize}
\end{thm}
\begin{proof}
群$(G,*)$をなす集合$G$の部分集合$H$を用いた組$(H,*)$が次の条件を全て満たすなら、
\begin{itemize}
\item
  集合$H$は空集合でない。
\item
  $\forall a,b \in H$に対し、$a*b \in H$が成り立つ。
\item
  $\forall a \in H$に対し、$a^{- 1} \in H$が成り立つ。
\end{itemize}
あるその集合$H$の元$a$が存在して$a^{- 1} \in H$が成り立ち、したがって、$1_{(G,*)} = a*a^{- 1} \in H$が成り立つので、次の条件が全て満たされることになり、その組$(H,*)$が算法$*$に関して部分群$(H,*)$をなす。
\begin{itemize}
\item
  $1_{(G,*)} \in H$が成り立つ。
\item
  $\forall a,b \in H$に対し、$a*b \in H$が成り立つ。
\item
  $\forall a \in H$に対し、$a^{- 1} \in H$が成り立つ。
\end{itemize}
逆に、その組$(H,*)$が算法$*$に関して部分群$(H,*)$をなすなら、上の条件が全て満たされることになり、明らかに集合$H$は空集合でないので、次の条件が全て満たされる。
\begin{itemize}
\item
  集合$H$は空集合でない。
\item
  $\forall a,b \in H$に対し、$a*b \in H$が成り立つ。
\item
  $\forall a \in H$に対し、$a^{- 1} \in H$が成り立つ。
\end{itemize}
\end{proof}
\begin{thm}\label{3.1.1.8}
群$(G,*)$において、集合$H$がその集合$G$の空集合でない有限な部分集合で、$\forall a,b \in H$に対し、$a*b \in H$が成り立つなら、その組$(H,*)$はその群$(G,*)$の部分群である。
\end{thm}
\begin{proof}
群$(G,*)$において、集合$H$がその集合$G$の空集合でない有限な部分集合で、$\forall a,b \in H$に対し、$a*b \in H$が成り立つとき、$\forall a \in H$に対し、$a = 1_{(G,*)}$が成り立つなら、$a^{- 1} = 1_{(G,*)} \in H$が成り立つ。\par
$a \neq 1_{(G,*)}$が成り立つなら、$\forall n \in \mathbb{N}$に対し、$a^{n} \in H$が成り立つことになる。ここで、$\forall m,n \in \mathbb{N}$に対し、$m \neq n$が成り立つなら、$a^{m} \neq a^{n}$が成り立つと仮定すれば、写像$\left( a^{n} \right)_{n \in \mathbb{N}}:\mathbb{N} \rightarrow H;n \mapsto a^{n}$が考えられれば、これは単射であるから、$\#\mathbb{N} = \aleph_{0} \leq \#H$が成り立ちこれはその集合$H$が有限集合であることに矛盾する。したがって、$m \neq n$が成り立つかつ、$a^{m} = a^{n}$が成り立つような自然数たち$m$、$n$が存在することになる。これにより、$m < n$と仮定しても一般性は失われることはなく$a^{n - m} = 1_{(G,*)}$が成り立つ。したがって、$a*a^{n - m - 1} = a^{n - m - 1}*a = 1_{(G,*)}$が成り立つ。ここで、$n - m - 1 = 0$と仮定すると、$n - m = 1$であるから、$a = a^{n - m} = 1_{(G,*)}$が得られるが、これは$a \neq 1_{(G,*)}$が成り立つことに矛盾する。したがって、$n - m - 1 > 0$が成り立つことになり、したがって、$a^{n - m - 1} \in H$が成り立つ。これにより、$a^{- 1} = a^{n - m - 1}$が得られる。\par
以上より、$\forall a \in H$に対し、$a^{- 1} \in H$が成り立つので、次の条件が全て満たされ、
\begin{itemize}
\item
  集合$H$は空集合でない。
\item
  $\forall a,b \in H$に対し、$a*b \in H$が成り立つ。
\item
  $\forall a \in H$に対し、$a^{- 1} \in H$が成り立つ。
\end{itemize}
よって、その組$(H,*)$が算法$*$に関して部分群$(H,*)$をなす。
\end{proof}
\begin{dfn}
その集合$H$がその集合$G$であるとき、または集合$\left\{ 1_{(G,*)} \right\}$であるとき、あきらかに群$(G,*)$の部分群$(H,*)$をなす。これら2つの集合たち$G$、$\left\{ 1_{(G,*)} \right\}$がなす群$(G,*)$の部分群たち$(G,*)$、$\left( \left\{ 1_{(G,*)} \right\},* \right)$を自明な部分群といい、その群$(G,*)$の部分群のうち、自明な部分群を除いた部分群を真部分群という。これの存在は群$(G,*)$に依存する。
\end{dfn}
\begin{thm}\label{3.1.1.9}
群$(G,*)$の部分群$(H,*)$が与えられたとき、$H*H = H$が成り立つ。
\end{thm}
\begin{proof}
群$(G,*)$の部分群$(H,*)$が与えられたとき、$\forall g*h \in H*H$に対し、$g,h \in H$が成り立つので、$g*h \in H$も成り立つ。一方で、$\forall h \in H$に対し、$h = 1_{(G,*)}*h$が成り立つので、$h \in H*H$も成り立つ。以上より、$H*H = H$が得られる。
\end{proof}
\begin{thm}\label{3.1.1.10}
2つの集合たち$H$、$I$を用いた組々$(H,*)$、$(I,*)$が群$(G,*)$の部分群たちをなすなら、集合$H \cap I$を用いた組$(H \cap I,*)$も群$(G,*)$の部分群をなす。
\end{thm}
\begin{proof}
2つの集合たち$H$、$I$を用いた組々$(H,*)$、$(I,*)$が群$(G,*)$の部分群たちをなすなら、次の条件が全て満たされる。
\begin{itemize}
\item
  $1_{(G,*)} \in H$が成り立つ。
\item
  $\forall a,b \in H$に対し、$a*b \in H$が成り立つ。
\item
  $\forall a \in H$に対し、$a^{- 1} \in H$が成り立つ。
\item
  $1_{(G,*)} \in I$が成り立つ。
\item
  $\forall a,b \in I$に対し、$a*b \in I$が成り立つ。
\item
  $\forall a \in I$に対し、$a^{- 1} \in I$が成り立つ。
\end{itemize}
これにより、次の条件が全て満たされることになるので、
\begin{itemize}
\item
  $1_{(G,*)} \in H \cap I$が成り立つ。
\item
  $\forall a,b \in H \cap I$に対し、$a*b \in H \cap I$が成り立つ。
\item
  $\forall a \in H \cap I$に対し、$a^{- 1} \in H \cap I$が成り立つ。
\end{itemize}
集合$H \cap I$を用いた組$(H \cap I,*)$も群$(G,*)$の部分群をなす。
\end{proof}
\begin{thm}\label{3.1.1.11}
群$(G,*)$において、$\forall S \in \mathfrak{P}(G) \setminus \left\{ \emptyset  \right\}$に対し、集合$S \cup S^{- 1}$の有限個の元たちの算法$*$の像全体の集合を$H$とおく。このとき、この組$(H,*)$は群$(G,*)$の部分群となる。
\end{thm}
\begin{proof}
群$(G,*)$において、$\forall S \in \mathfrak{P}(G) \setminus \left\{ \emptyset  \right\}$を考える。集合$S \cup S^{- 1}$の有限個の元たちの算法$*$の像全体の集合を$H$とおく。$a,b \in H$なる元たち$a$、$b$を考えると、その元$a*b$も集合$S \cup S^{- 1}$の有限個の元たちの算法$*$の像であるから$a*b \in H$が成り立ち、$a^{- 1}$も始集合の元たちの逆元たちの算法$*$の逆元たちの算法$*$の像であり、これらの逆元たちは全て集合$S$、$S^{- 1}$のいずれかに属するので、$a^{- 1} \in H$である。以上より、次の条件が全て満たされ、
\begin{itemize}
\item
  集合$H$は空集合でない。
\item
  $\forall a,b \in H$に対し、$a*b \in H$が成り立つ。
\item
  $\forall a \in H$に対し、$a^{- 1} \in H$が成り立つ。
\end{itemize}
この集合$H$を用いた組$(H,*)$は群$(G,*)$の部分群をなす。
\end{proof}
\begin{dfn}
このような集合$H$を集合$S$によって生成される群$(G,*)$の部分群といい、集合$S$を集合$H$の生成元の集合、生成系という。特に$S = \left\{ a \right\}$が成り立つなら、この集合によって生成される群$(G,*)$の部分群をなす集合は$\left\{ a^{n} \in G \middle| a \in S,n \in \mathbb{Z} \right\}$となる。
\end{dfn}
%\hypertarget{ux5270ux4f59ux985e}{%
\subsubsection{剰余類}%\label{ux5270ux4f59ux985e}}
\begin{dfn}
群$(G,*)$の部分群$(H,*)$が与えられたとき、$a,b \in G$なる元たち$a$、$b$について次のことを定義しよう。このようなことを元たち$a$、$b$がその部分群$(H,*)$を法として左合同であるという。
\begin{align*}
a \equiv_{l}b\ \mathrm{mod}(H,*) \Leftrightarrow a^{- 1}*b \in H
\end{align*}
\end{dfn}
\begin{thm}\label{3.1.1.12}
群$(G,*)$の部分群$(H,*)$が与えられたとき、この関係$\equiv_{l}\ \mathrm{mod}(H,*)$はその集合$G$における同値関係である。
\end{thm}
\begin{proof}
群$(G,*)$の部分群$(H,*)$が与えられたとき、$\forall a \in G$に対し、次のようになるかつ、
\begin{align*}
1_{(G,*)} \in G &\Rightarrow 1_{(G,*)} \in H\\
&\Leftrightarrow a^{- 1}*a \in H\\
&\Leftrightarrow a \equiv_{l}a\ \mathrm{mod}(H,*)
\end{align*}
$\forall a,b \in G$に対し、$a \equiv_{l}b\ \mathrm{mod}(H,*)$が成り立つなら、次のようになるかつ、
\begin{align*}
a \equiv_{l}b\ \mathrm{mod}(H,*) &\Leftrightarrow a^{- 1}*b \in H\\
&\Leftrightarrow \left( a^{- 1}*b \right)^{- 1} = b^{- 1}*\left( a^{- 1} \right)^{- 1} = b^{- 1}*a \in H\\
&\Leftrightarrow b \equiv_{l}a\ \mathrm{mod}(H,*) \\
a \equiv_{l}b\ \mathrm{mod}(H,*) \land b \equiv_{l}c\ \mathrm{mod}(H,*) &\Leftrightarrow a^{- 1}*b \in H \land b^{- 1}*c \in H\\
&\Rightarrow \left( a^{- 1}*b \right)*\left( b^{- 1}*c \right) = \left( a^{- 1}*1_{(G,*)} \right)*c = a^{- 1}*c \in H\\
&\Leftrightarrow a \equiv_{l}c\ \mathrm{mod}(H,*)
\end{align*}
この関係$\equiv_{l}\ \mathrm{mod}(H,*)$はその集合$G$における同値関係である。
\end{proof}
\begin{thm}\label{3.1.1.13}
群$(G,*)$の部分群$(H,*)$が与えられたとき、この同値関係$\equiv_{l}\ \mathrm{mod}(H,*)$によるその集合$G$の元$a$の同値類$C_{\equiv_{l}\ \mathrm{mod}(H,*)}(a)$は集合$a*H$に等しい。
\end{thm}
\begin{proof} 群$(G,*)$の部分群$(H,*)$が与えられたとき、次のようなり、
\begin{align*}
x \in C_{\equiv_{l}\ \mathrm{mod}(H,*)}(a) &\Leftrightarrow a \equiv_{l}x\ \mathrm{mod}(H,*)\\
&\Leftrightarrow a^{- 1}*x \in H
\end{align*}
ここで、$h = a^{- 1}*x$とすると、次式が成り立ち、
\begin{align*}
x &= 1_{(G,*)}*x\\
&= \left( a*a^{- 1} \right)*x\\
&= a*\left( a^{- 1}*x \right)\\
&= a*h
\end{align*}
明らかに$h \in H$かつ$a*h \in G$が成り立つので、次式が成り立ち、
\begin{align*}
a*h \in C_{\equiv_{l}\ \mathrm{mod}(H,*)}(a) &\Leftrightarrow a*h \in G \land a \equiv_{l}a*h\ \mathrm{mod}(H,*)\\
&\Leftrightarrow a*h \in G \land a^{- 1}*(a*h) \in H\\
&\Leftrightarrow a*h \in G \land \left( a^{- 1}*a \right)*h \in H\\
&\Leftrightarrow a*h \in G \land h \in H\\
&\Leftrightarrow a*h \in \left\{ a*h \in G \middle| h \in H \right\} = a*H
\end{align*}
したがって、次のようになる。
\begin{align*}
C_{\equiv_{l}\ \mathrm{mod}(H,*)}(a) = \left\{ a*h \in G \middle| h \in H \right\} = a*H
\end{align*}
\end{proof}
\begin{dfn}
この集合$a*H$を部分群$(H,*)$を法とする元$a$の左剰余類といいこれ全体の集合${G}/{\equiv_{l}\ \mathrm{mod}(H,*)}$を$G/H$と書くが、ここでは、${G}/_l {H}$と書くことにする。
\end{dfn}
\begin{thm}[左剰余類分解]\label{3.1.1.12s}
群$(G,*)$の部分群$(H,*)$が与えられたとき、次式が成り立つ。
\begin{align*}
G=\bigsqcup G/_l H
\end{align*}\par
この定理を左剰余類分解という。
\end{thm}
\begin{proof}
群$(G,*)$の部分群$(H,*)$が与えられたとき、定理\ref{3.1.1.12}よりその関係$\equiv_{l}\ \mathrm{mod}(H,*)$はその集合$G$における同値関係であるので、その関係$\equiv_{l}\ \mathrm{mod}(H,*)$による類別が次のようになる。
\begin{align*}
G=\bigsqcup {G}/{\equiv_{l}\ \mathrm{mod}(H,*)}=\bigsqcup G/_l H
\end{align*}
\end{proof}
\begin{thm}\label{3.1.1.14}
群$(G,*)$の部分群$(H,*)$が与えられたとき、その集合$H$自身はその部分群$(H,*)$を法とする左剰余類でもある。
\end{thm}
\begin{proof}
$\left\{ h \in G \middle| h \in H \right\} = \left\{ 1_{(G,*)}*h \in G \middle| h \in H \right\}$が成り立つので、$H = 1_{(G,*)}*H$が成り立つことによる。
\end{proof}
\begin{thm}\label{3.1.1.15}
群$(G,*)$の部分群$(H,*)$が与えられたとき、その集合$G$の元々$a$、$b$が部分群$(H,*)$を法として左合同であるなら、$a*H = b*H$が成り立ち、そうでないなら$a*H \cap b*H = \emptyset $が成り立つ。
\end{thm}
\begin{proof}
群$(G,*)$の部分群$(H,*)$が与えられたとき、その集合$G$の元々$a$、$b$が集合$H$に関して左合同であるとする、即ち、$a \equiv_{l}b\ \mathrm{mod}(H,*)$が成り立つとする。このとき、$a \equiv_{l}b\ \mathrm{mod}(H,*) \Leftrightarrow b \equiv_{l}a\ \mathrm{mod}(H,*)$が成り立ち、$\forall x \in a*H$に対し、次のようになる。
\begin{align*}
x \in a*H &\Leftrightarrow x = a*h \land h \in H\\
&\Leftrightarrow a^{- 1}*x = h \in H\\
&\Leftrightarrow x \equiv_{l}a\ \mathrm{mod}(H,*)
\end{align*}
ここで、$a \equiv_{l}b\ \mathrm{mod}(H,*)$が成り立つので、次のようになる。
\begin{align*}
x \equiv_{l}b\ \mathrm{mod}(H,*) \Leftrightarrow b^{- 1}*x \in H
\end{align*}
ここで、$b^{- 1}*x = h'$とおくと、次のようになるので、
\begin{align*}
x = b*b^{- 1}*x = b*h'
\end{align*}
$x \in b*H$が成り立つ。これにより、$a*H \subseteq b*H$が得られる。同様にして、$b*H \subseteq a*H$が得られるので、$a*H = b*H$が成り立つ。\par
$a*H \cap b*H \neq \emptyset $が成り立つなら、$x \in a*H$かつ$x \in b*H$が成り立つような元$x$がその集合$G$に存在し、次のようになる。
\begin{align*}
x \in a*H \land x \in b*H &\Leftrightarrow a^{- 1}*x,b^{- 1}*x \in H\\
&\Leftrightarrow a \equiv_{l}x\ \mathrm{mod}(H,*) \land b \equiv_{l}x\ \mathrm{mod}(H,*)\\
&\Rightarrow a \equiv_{l}b\ \mathrm{mod}(H,*)
\end{align*}
これを対偶律に適用すれば、$\neg a \equiv_{l}b\ \mathrm{mod}(H,*) \Rightarrow a*H \cap b*H = \emptyset $が成り立つ。
\end{proof}
\begin{dfn}
群$(G,*)$の部分群$(H,*)$が与えられたとき、$a,b \in G$なる元々$a$、$b$について次のことを定義しよう。このようなことを元々$a$、$b$がその部分群$(H,*)$を法として右合同であるという。
\begin{align*}
a \equiv_{r}b\ \mathrm{mod}(H,*) \Leftrightarrow a*b^{- 1} \in H
\end{align*}
\end{dfn}
\begin{thm}\label{3.1.1.16}
群$(G,*)$の部分群$(H,*)$が与えられたとき、この関係$\equiv_{r}\ \mathrm{mod}(H,*)$はその集合$G$における同値関係である。
\end{thm}
\begin{proof} 定理\ref{3.1.1.12}と同様にして示される。
\end{proof}
\begin{thm}\label{3.1.1.17}
群$(G,*)$の部分群$(H,*)$が与えられたとき、この同値関係$\equiv_{r}\ \mathrm{mod}(H,*)$によるその集合$G$の元$a$の同値類$C_{\equiv_{r}\ \mathrm{mod}(H,*)}(a)$は集合$H*a$に等しい。
\end{thm}
\begin{proof} 定理\ref{3.1.1.13}と同様にして示される。
\end{proof}
\begin{dfn}
この集合$H*a$を部分群$(H,*)$を法とする元$a$の右剰余類といいこれ全体の集合${G}/{\equiv_{r}\ \mathrm{mod}(H,*)}$を$H\setminus G$と書くが、ここでは、${G}/_r {H}$と書くことにする。
\begin{thm}[右剰余類分解]\label{3.1.1.16s}
群$(G,*)$の部分群$(H,*)$が与えられたとき、次式が成り立つ。
\begin{align*}
G=\bigsqcup G/_r H
\end{align*}\par
この定理を右剰余類分解という。
\end{thm}
\begin{proof} 定理\ref{3.1.1.12s}と同様にして示される。
\end{proof}
\begin{thm}\label{3.1.1.18}
群$(G,*)$の部分群$(H,*)$が与えられたとき、その集合$H$自身はその部分群$(H,*)$を法とする右剰余類でもある。
\end{thm}
\begin{proof} 定理\ref{3.1.1.14}と同様にして示される。
\end{proof}
\begin{thm}\label{3.1.1.19}
群$(G,*)$の部分群$(H,*)$が与えられたとき、その集合$G$の元々$a$、$b$が部分群$(H,*)$を法として右合同であるなら、$H*a = H*b$が成り立ち、そうでないなら$H*a \cap H*b = \emptyset $が成り立つ。
\end{thm}
\begin{proof} 定理\ref{3.1.1.15}と同様にして示される。
\end{proof}\par
なお、任意の部分群$(H,*)$において、その部分群$(H,*)$を法として左合同であることと右合同であることとは必ずしも一致するとは限らないことに注意されたい。
\begin{dfn}
群$(G,*)$の部分群$(H,*)$が与えられたとき、その集合$G$の元々$a$、$b$がその部分群$(H,*)$を法として左合同であることと右合同であることとが一致するとき、それらの元々$a$、$b$が部分群$(H,*)$を法として合同である、合同関係であるといい、このことを$a \equiv b\ \mathrm{mod}(H,*)$と書く。さらに同値類$C_{\equiv \ \mathrm{mod}(H,*)}(a)$を部分群$(H,*)$を法とする元$a$の剰余類という。
\end{dfn}
\begin{thm}\label{3.1.1.20}
群$(G,*)$の部分群$(H,*)$が与えられたとき、$\forall a \in G$に対し、$a*H = H*a$が成り立つなら、$\forall b \in G$に対し、$a \equiv_{l}b\ \mathrm{mod}(H,*)$が成り立つならそのときに限り、$a \equiv_{r}b\ \mathrm{mod}(H,*)$が成り立つ。
\end{thm}
\begin{proof}
群$(G,*)$の部分群$(H,*)$が与えられたとき、$\forall a \in G$に対して、$a*H = H*a$が成り立つとすると、$\forall b \in G$に対し、次のことが成り立つので、
\begin{itemize}
\item
  $b \in a*H$が成り立つならそのときに限り、$b \in H*a$が成り立つ。
\item
  $b \in a*H$が成り立つならそのときに限り、$a \equiv_{l}b\ \mathrm{mod}(H,*)$が成り立つ。
\item
  $b \in H*a$が成り立つならそのときに限り、$a \equiv_{r}b\ \mathrm{mod}(H,*)$が成り立つ。
\end{itemize}
$a \equiv_{l}b\ \mathrm{mod}(H,*)$が成り立つならそのときに限り、$a \equiv_{r}b\ \mathrm{mod}(H,*)$が成り立つことがいえる。
\end{proof}
\begin{thm}\label{3.1.1.21}
群$(G,*)$の部分群$(H,*)$が与えられたとき、$\forall a,b \in G$に対し、$a*b = b*a$が成り立つなら、$a \equiv_{l}b\ \mathrm{mod}(H,*)$が成り立つならそのときに限り、$a \equiv_{r}b\ \mathrm{mod}(H,*)$が成り立つ。
\end{thm}
\begin{proof}
群$(G,*)$の部分群$(H,*)$が与えられたとき、$\forall a,b \in G$に対して、$a*b = b*a$が成り立つとする、即ち、群$(G,*)$が可換群であるとすると、$\forall a \in G$に対して左剰余類と右剰余類の定義より次式が成り立つので、
\begin{align*}
a*H &= \left\{ a*h \in G \middle| h \in H \right\}\\
&= \left\{ h*a \in G \middle| h \in H \right\}\\
&= H*a
\end{align*}
定理\ref{3.1.1.20}により$a \equiv_{l}b\ \mathrm{mod}(H,*)$が成り立つならそのときに限り、$a \equiv_{r}b\ \mathrm{mod}(H,*)$が成り立つ。
\end{proof}
\begin{thm}\label{3.1.1.22}
群$(G,*)$の部分群たち$(H,*)$、$(I,*)$が与えられたとき、$H*I = I*H$が成り立つならそのときに限り、その組$(H*I,*)$がその群$(G,*)$の部分群をなす。
\end{thm}
\begin{proof}
群$(G,*)$の部分群たち$(H,*)$、$(I,*)$が与えられたとき、$H*I = I*H$が成り立つなら、$1_{(G,*)} \in H$かつ$1_{(G,*)} \in I$が成り立つので、$1_{(G,*)} = 1_{(G,*)}*1_{(G,*)} \in H*I$が成り立つ。また、$h',h'' \in H$、$i',i'' \in I$とおき$\forall h'*i',h''*i'' \in H*I$に対し、$H*I = I*H$が成り立つので、$\exists h''' \in H\exists i''' \in I$に対し、$h'''*i''' = i'*h''$が成り立つ。したがって、次のようになり、
\begin{align*}
\left( h'*i' \right)*\left( h''*i'' \right) &= h'*\left( i'*h'' \right)*i''\\
&= h'*\left( h'''*i''' \right)*i''\\
&= \left( h'*h''' \right)*\left( i'''*i'' \right)
\end{align*}
$h'*h''' \in H$かつ$i'''*i'' \in I$が成り立つので、$\left( h'*i' \right)*\left( h''*i'' \right) \in H*I$が成り立つ。さらに、$h' \in H$、$i' \in I$とおき$\forall h'*i' \in H*I$に対し、$\left( h'*i' \right)^{- 1} = {i'}^{- 1}*{h'}^{- 1}$が成り立つのであった。ここで、${h'}^{- 1} \in H$かつ${i'}^{- 1} \in I$が成り立つので、$\left( h'*i' \right)^{- 1} \in I*H$が成り立つ。ここで、$H*I = I*H$が成り立つので、$\left( h'*i' \right)^{- 1} \in H*I$も成り立つ。これにより、次のことが満たされその組$(H*I,*)$がその群$(G,*)$の部分群をなす。
\begin{itemize}
\item
  $1_{(G,*)} \in H*I$が成り立つ。
\item
  $\forall h'*i',h''*i'' \in H*I$に対し、$\left( h'*i' \right)*\left( h''*i'' \right) \in H*I$が成り立つ。
\item
  $\forall h'*i' \in H*I$に対し、$\left( h'*i' \right)^{- 1} \in H*I$が成り立つ。
\end{itemize}\par
逆に、その組$(H*I,*)$がその群$(G,*)$の部分群をなすなら、$\forall h'*i' \in H*I$に対し、$\left( h'*i' \right)^{- 1} \in H*I$が成り立つので、$\left( h'*i' \right)^{- 1} = h''*i''$が成り立つようなその集合$H$の元$h''$とその集合$I$の元$i''$が存在することになる。ここで、$\left( h''*i'' \right)^{- 1} \in H*I$かつ${h''}^{- 1} \in H$かつ${i''}^{- 1} \in I$が成り立つので、次のようになる。
\begin{align*}
h'*i' &= \left( \left( h'*i' \right)^{- 1} \right)^{- 1}\\
&= \left( h''*i'' \right)^{- 1}\\
&= {i''}^{- 1}*{h''}^{- 1} \in I*H
\end{align*}
また、$\forall i'*h' \in I*H$に対し、${h'}^{- 1} \in H$かつ${i'}^{- 1} \in I$かつ$\left( {h'}^{- 1}*{i'}^{- 1} \right)^{- 1} \in H*I$が成り立つので、次のようになる。
\begin{align*}
i'*h' &= \left( {i'}^{- 1} \right)^{- 1}*\left( {h'}^{- 1} \right)^{- 1}\\
&= \left( {h'}^{- 1}*{i'}^{- 1} \right)^{- 1} \in H*I
\end{align*}
以上より、$H*I = I*H$が得られた。
\end{proof}
\begin{thm}\label{3.1.1.23}
群$(G,*)$の部分群$(H,*)$が与えられたとき、$\forall a \in G$に対し、その組$\left( a*H*a^{- 1},* \right)$もその群$(G,*)$の部分群をなす。
\end{thm}
\begin{proof}
群$(G,*)$の部分群$(H,*)$が与えられたとき、$\forall a \in G$に対し、集合$a*H*a^{- 1}$について考えよう。このとき、$1_{(G,*)} \in H$が成り立つので、次のようになることから、
\begin{align*}
1_{(G,*)} &= a*a^{- 1}\\
&= a*1_{(G,*)}*a^{- 1} \in a*H*a^{- 1}
\end{align*}
$1_{(G,*)} \in a*H*a^{- 1}$が成り立つ。また、$\forall a*g*a^{- 1},a*h*a^{- 1} \in a*H*a^{- 1}$に対し、次のようになることから、
\begin{align*}
\left( a*g*a^{- 1} \right)*\left( a*h*a^{- 1} \right) &= a*g*\left( a^{- 1}*a \right)*h*a^{- 1}\\
&= a*g*1_{(G,*)}*h*a^{- 1}\\
&= a*(g*h)*a^{- 1}
\end{align*}
$g*h \in H$が成り立つことにより$\left( a*g*a^{- 1} \right)*\left( a*h*a^{- 1} \right) \in a*H*a^{- 1}$が成り立つ。さらに、$\forall a*h*a^{- 1} \in a*H*a^{- 1}$に対し、次のようになることから、
\begin{align*}
\left( a*h*a^{- 1} \right)^{- 1} &= \left( a^{- 1} \right)^{- 1}*h^{- 1}*a^{- 1}\\
&= a*h^{- 1}*a^{- 1}
\end{align*}
$h^{- 1} \in H$が成り立つことにより$\left( a*h*a^{- 1} \right)^{- 1} \in a*H*a^{- 1}$が成り立つ。以上より、次のことが満たされその組$\left( a*H*a^{- 1},* \right)$がその群$(G,*)$の部分群をなす。
\begin{itemize}
\item
  $1_{(G,*)} \in a*H*a^{- 1}$が成り立つ。
\item
  $\forall a*g*a^{- 1},a*h*a^{- 1} \in a*H*a^{- 1}$に対し、$\left( a*g*a^{- 1} \right)*\left( a*h*a^{- 1} \right) \in a*H*a^{- 1}$が成り立つ。
\item
  $\forall a*h*a^{- 1} \in a*H*a^{- 1}$に対し、$\left( a*h*a^{- 1} \right)^{- 1} \in a*H*a^{- 1}$が成り立つ。
\end{itemize}
\end{proof}
%\hypertarget{ux6b63ux898fux90e8ux5206ux7fa4}{%
\subsubsection{正規部分群}%\label{ux6b63ux898fux90e8ux5206ux7fa4}}
\begin{dfn}
群$(G,*)$とこれの部分群$(N,*)$について、部分群$(N,*)$を法とする左合同と右合同が一致する、即ち、$\forall a \in G$に対し、$a*N = N*a$が成り立つとき、部分群$(N,*)$を群$(G,*)$について正規な部分群$(N,*)$、群$(G,*)$の正規部分群などといい$(N,*) \trianglelefteq (G,*)$とかく。
\end{dfn}
\begin{thm}\label{3.1.1.24}
群$(G,*)$自身、単位部分群$\left( \left\{ 1_{(G,*)} \right\},* \right)$、群$(G,*)$が可換群であるときの任意の部分群はいずれもその群$(G,*)$の正規部分群である、即ち、次式が成り立つ。
\begin{align*}
(G,*) \trianglelefteq (G,*),\ \ \left( \left\{ 1_{(G,*)} \right\},* \right) \trianglelefteq (G,*)
\end{align*}
\end{thm}
\begin{proof}
群$(G,*)$自身は、算法$*$に適用される2つの元の順序付けられた組とこれの順序を逆にしたものどちらも存在するので、その群$(G,*)$の正規部分群である。単位部分群$\left( \left\{ 1_{G} \right\},* \right)$は、単位元は任意の元に対し可換律が成り立つので、その群$(G,*)$の正規部分群である。群$(G,*)$が可換群であるときの任意の部分群$(H,*)$は、上記の定理より可換律で左剰余類$a*H$と右剰余類$H*a$の各元が一致するので、その群$(G,*)$の正規部分群である。
\end{proof}
\begin{thm}\label{3.1.1.25}
群$(G,*)$とこれの部分群$(N,*)$について、次のことは同値である。
\begin{itemize}
\item
  $(N,*) \trianglelefteq (G,*)$が成り立つ。
\item
  $\forall a \in G$に対し、$a*N = N*a$が成り立つ。
\item
  $\forall a \in G$に対し、$a*N*a^{- 1} = N$が成り立つ。
\item
  $\forall a \in G\forall n \in N$に対し、$a*n*a^{- 1} \in N$が成り立つ。
\item
  $\forall a \in G$に対し、$a*N*a^{- 1} \subseteq N$が成り立つ。
\end{itemize}
\end{thm}
\begin{proof}
群$(G,*)$とこれの部分群$(N,*)$について、$(N,*) \trianglelefteq (G,*)$が成り立つならそのときに限り、$\forall a \in G$に対し、$a*N = N*a$が成り立つのであった。したがって、$\forall a*n \in a*N$に対し、$a*n \in a*N$が成り立つならそのときに限り、$a*n \in N*a$が成り立つので、あるその集合$H$の元$n'$を用いて$a*n = n'*a$とおくことができる。したがって、これが成り立つならそのときに限り、次のようになる。
\begin{align*}
a*n*a^{- 1} = n'*a*a^{- 1} = n'
\end{align*}
これにより、$\forall a*n*a^{- 1} \in a*N*a^{- 1}$に対し、$a*n*a^{- 1} \in a*N*a^{- 1}$が成り立つならそのときに限り、$a*n*a^{- 1} \in N$が成り立つので、$a*N*a^{- 1} = N$が成り立つ。これで次のことが同値であることが示された。
\begin{itemize}
\item
  $(N,*) \trianglelefteq (G,*)$が成り立つ。
\item
  $\forall a \in G$に対し、$a*N = N*a$が成り立つ。
\item
  $\forall a \in G$に対し、$a*N*a^{- 1} = N$が成り立つ。
\end{itemize}\par
また、$\forall a \in G\forall n \in N$に対し、$a*n*a^{- 1} \in N$が成り立つなら、元$a$の代わりに元$a^{- 1}$を考えれば、$a^{- 1}*n*a \in N$が成り立ち、したがって、$\forall a*n \in a*N$に対し、あるその集合$N$の元$h'$を用いて$a^{- 1}*n'*a = n$とおくことができ次のようになる。
\begin{align*}
a*n &= a*\left( a^{- 1}*n'*a \right)\\
&= \left( a*a^{- 1} \right)*\left( n'*a \right)\\
&= 1_{(G,*)}*\left( n'*a \right)\\
&= n'*a \in N*a
\end{align*}
逆も同様にして考えれば、したがって、$a*n \in a*N$が成り立つならそのときに限り、$a*n \in N*a$が成り立つので、$a*N = N*a$が成り立ち、したがって、部分群$(N,*)$が群$(G,*)$について正規である。また、明らかに次のことは同値であるので、
\begin{itemize}
\item
  $\forall a \in G\forall n \in N$に対し、$a*n*a^{- 1} \in N$が成り立つ。
\item
  $\forall a \in G$に対し、$a*N*a^{- 1} \subseteq N$が成り立つ。
\end{itemize}
以上の議論により、次のことは同値である。
\begin{itemize}
\item
  $\forall a \in G$に対し、$a*N = N*a$が成り立つ。
\item
  $\forall a \in G$に対し、$a*N*a^{- 1} = N$が成り立つ。
\item
  $\forall a \in G\forall n \in N$に対し、$a*n*a^{- 1} \in N$が成り立つ。
\item
  $\forall a \in G$に対し、$a*N*a^{- 1} \subseteq N$が成り立つ。
\end{itemize}
\end{proof}
\begin{thm}\label{3.1.1.26}
群$(G,*)$の部分群$(H,*)$と正規部分群$(N,*)$が与えられたとき、$H*N = N*H$が成り立つ。
\end{thm}
\begin{proof}
群$(G,*)$の部分群$(H,*)$と正規部分群$(N,*)$が与えられたとき、$h \in H$かつ$n \in N$とおいて$\forall h*n \in H*N$に対し、$h*n \in h*N$が成り立ち、ここで、$h*N = N*h$が成り立つので、$h*n \in N*h$が成り立つ。これにより、$h*n = n'*h$が成り立つようなその集合$N$の元$n'$が存在する。したがって、$h*n \in N*H$が成り立つので、$H*N \subseteq N*H$が成り立つ。同様にして、$N*H \subseteq H*N$が成り立つので、$H*N = N*H$が得られる。
\end{proof}
\begin{thm}\label{3.1.1.27}
群$(G,*)$の部分群$(H,*)$と正規部分群$(N,*)$が与えられたとき、その組$(H*N,*)$もその群$(G,*)$の部分群をなす。特に、その部分群$(H,*)$が正規であるなら、その部分群$(H*N,*)$も正規である。
\end{thm}
\begin{proof}
群$(G,*)$の部分群$(H,*)$と正規部分群$(N,*)$が与えられたとき、$H*N = N*H$が成り立つので、定理\ref{3.1.1.22}よりその組$(H*N,*)$もその群$(G,*)$の部分群をなす。特に、その部分群$(H,*)$が正規であるなら、$\forall a \in G$に対し、$a*H = H*a$が成り立つことになるので、$\forall a \in G$に対し、次のようになり、
\begin{align*}
a*(H*N) &= (a*H)*N\\
&= (H*a)*N\\
&= H*(a*N)\\
&= H*(N*a)\\
&= (H*N)*a
\end{align*}
これにより、その部分群$(H*N,*)$も正規である。
\end{proof}
\begin{thm}\label{3.1.1.28}
群$(G,*)$の2つの正規部分群たち$(M,*)$、$(N,*)$が与えられたとき、その組$(M \cap N,*)$もその群$(G,*)$の正規部分群をなす、即ち、$(M \cap N,*) \trianglelefteq (G,*)$が成り立つ。
\end{thm}
\begin{proof}
群$(G,*)$の2つの正規部分群たち$(M,*)$、$(N,*)$が与えられたとき、定理\ref{3.1.1.10}よりその組$(M \cap N,*)$もその群$(G,*)$の部分群をなす。ここで、$\forall a \in G\forall n \in M \cap N$に対し、$n \in M$が成り立つので、$a*M = M*a$が成り立つことにより、$a*n = m'*a$が成り立つような元$m'$がその集合$M$に存在する。同様にして、$a*n = n'*a$が成り立つような元$n'$がその集合$M$に存在する。したがって、$m'*a = n_{2}*a$が得られ、これにより、$m' = n'$が成り立つので、$m' = n' \in M \cap N$が成り立つ。これにより、$a*n = m'*a$が成り立つような元$m'$がその集合$M \cap N$に存在することになるので、$a*(M \cap N) \subseteq (M \cap N)*a$が得られる。逆も同様にして考えれば、よって、その部分群$(M \cap N,*)$は正規である。
\end{proof}
\begin{thm}\label{3.1.1.29}
群$(G,*)$とこれの1つの部分群$(N,*)$において、$(N,*) \trianglelefteq (G,*)$が成り立つならそのときに限り、$\forall a,b \in G$に対し、次式が成り立つ。
\begin{align*}
a*N*b*N = a*b*N
\end{align*}
\end{thm}
\begin{proof}
群$(G,*)$とこれの1つの部分群$(N,*)$において、$(N,*) \trianglelefteq (G,*)$が成り立つなら、$\forall a,b \in G$に対し、その部分群$(N,*)$を法とする2つの剰余類たち$a*N$、$b*N$を用いた集合$a*N*b*N$について考えよう。$\forall n \in N$に対し、$b*n*b^{- 1} \in N$が成り立つかつ、$\forall m,n \in N$に対し、$m*n \in N$が成り立つことより、$\forall a*m*b*n \in a*N*b*N$に対し、次のようになるかつ、
\begin{align*}
a*m*b*n &= a*b*b^{- 1}*m*b*n\\
&= a*b*\left( b^{- 1}*m*b \right)*n \in a*b*N
\end{align*}
$1_{(G,*)} \in N$が成り立つことにより、$\forall a*b*n \in a*b*N$に対し、次のようになるので、
\begin{align*}
a*b*n = a*1_{(G,*)}*b*n \in a*N*b*N
\end{align*}
$a*N*b*N = a*b*N$が成り立つ。\par
逆に、$\forall a,b \in G$に対し、$a*N*b*N = a*b*N$が成り立つなら、$a = 1_{(G,*)}$とすれば、$\forall c \in G$に対し、$c \in 1_{(G,*)}*N*b*N$が成り立つならそのときに限り、$\exists m,n \in N$に対し、$c = 1_{(G,*)}*m*b*n = m*b*n$が成り立つので、$c \in N*b*N$が成り立つ。これにより、$N*b*N = b*N$が成り立つ。そこで、$\forall n*b \in N*b$に対し、$1_{(G,*)} \in N$が成り立つので、$n*b = n*b*1_{(G,*)} \in N*b*N$が成り立つ。これにより、$N*b \subseteq N*b*N$が成り立つので、$\forall n*b \in N*b$に対し、$b^{- 1}*N*b \in N$が成り立つ。定理\ref{3.1.1.25}よりその部分群$(N,*)$は$(N,*) \trianglelefteq (G,*)$を満たす。
\end{proof}
%\hypertarget{ux5546ux7fa4}{%
\subsubsection{商群}%\label{ux5546ux7fa4}}
\begin{dfn}
群$(G,*)$の正規部分群$(N,*)$が与えられたとき、次式のように定義されるように、その部分群$(N,*)$を法とする剰余類たち$a*N$全体の集合を$G/N$と書く。
\end{dfn}
\begin{align*}
{G}/{N} = \left\{ a*N \middle| a \in G \right\}
\end{align*}
\end{dfn}
\begin{thm}\label{3.1.1.30}
群$(G,*)$の正規部分群$(N,*)$が与えられたとき、その組$\left( {G}/{N},* \right)$は群をなす。このとき、その群$\left( {G}/{N},* \right)$上での単位元は$1_{(G,*)}*N = N$、その群$\left( {G}/{N},* \right)$の任意の元$a*N$の逆元は$a^{- 1}*N$となる、即ち、次式が成り立つ。
\begin{align*}
1_{\left( {G}/{N},* \right)} &= N\\
(a*N)^{- 1} &= a^{- 1}*N
\end{align*}
\end{thm}
\begin{dfn}
群$(G,*)$の正規部分群$(N,*)$が与えられたとき、その群$\left( {G}/{N},* \right)$をその正規部分群$(N,*)$によるその群$(G,*)$の剰余群、商群という。
\end{dfn}
\begin{proof}
群$(G,*)$とこれの正規部分群$(N,*)$を法とする剰余類たち$a*N$全体の集合${G}/{N}$において、$\forall a,b,c \in G$に対し、次のようになることにより、
\begin{align*}
(a*N*b*N)*c*N &= (a*b*N)*c*N\\
&= a*(b*N*c*N)\\
&= a*(b*c*N)\\
&= a*(b*c)*N\\
&= a*N*(b*c*N)\\
&= a*N*(b*N*c*N)
\end{align*}
結合律が満たされる。\par
さらに、次のようになることにより、
\begin{align*}
1_{(G,*)}*N*a*N &= 1_{(G,*)}*a*N = a*N\\
a*N*1_{(G,*)}*N &= a*1_{(G,*)}*N = a*N
\end{align*}
その単位元が$1_{(G,*)}*N = N$である。\par
最後に、次のようになることにより、
\begin{align*}
a*N*a^{- 1}*N &= a*a^{- 1}*N = 1_{(G,*)}*N = N\\
a^{- 1}*N*a*N &= a^{- 1}*a*N = 1_{(G,*)}*N = N
\end{align*}
その集合${G}/{N}$の任意の元$a*N$のその逆元は$a^{- 1}*N$である。\par
以上より部分群$(N,*)$を法とする剰余類全体の集合${G}/{N}$は群$\left( {G}/{N},* \right)$をなす。
\end{proof}
%\hypertarget{ux7fa4ux8ad6ux306bux95a2ux3059ux308blagrangeux306eux5b9aux7406}{%
\subsubsection{群論に関するLagrangeの定理}%\label{ux7fa4ux8ad6ux306bux95a2ux3059ux308blagrangeux306eux5b9aux7406}}
\begin{thm}\label{3.1.1.31}
群$(G,*)$の部分群$(H,*)$が与えられたとき、これらについて次のことが成り立つ。
\begin{itemize}
\item
  集合$Q_{l}$の任意の元$a*H$に対し集合$Q_{r}$の元$H*a^{- 1}$は一意的に決まる。
\item
  写像${G}/_l {H} \rightarrow {G}/_r {H};a*H \mapsto H*a^{- 1}$は全単射であり、したがって、$\#{G}/_l {H} = \#{G}/_r {H}$が成り立つ。
\end{itemize}
\end{thm}
\begin{proof}
群$(G,*)$の部分群$(H,*)$が与えられたとき、$\forall a*H,b*H \in {G}/_l {H}$に対し、$a*H = b*H$が成り立つならそのときに限り、$a^{- 1}*b \in H$が成り立つことになる。ここで、逆元はもとの群をなす集合に属するので、$\left( a^{- 1}*b \right)^{- 1} = b^{- 1}*a \in H$が成り立つことから、$H*a = H*b$が成り立つ。これにより、graph$\mathfrak{G}$が次のような対応${G}/_l {H}\multimap {G}/_r {H}$はたしかに写像となっている。
\begin{align*}
\mathfrak{G}=\left\{ (a*H,H*b) \in {G}/_l {H} \times {G}/_r {H} \middle| H*b = H*a^{- 1} \right\}
\end{align*}\par
写像$f:{G}/_l {H} \rightarrow {G}/_r {H};a*H \mapsto H*a^{- 1}$について、上と同様の議論により、この写像の逆対応$f^{- 1}:{G}/_r {H} \rightarrow {G}/_l {H};H*a^{- 1} \mapsto a*H$も写像であるから、よって、この写像$f$は全単射となり、$\#{G}/_l {H} = \#{G}/_r {H}$が得られる。
\end{proof}
\begin{dfn}
群$(G,*)$の部分群$(H,*)$について、部分群$(H,*)$を法とする互いに異なる左剰余類たちの濃度$\#{G}/_l {H}$は、部分群$(H,*)$を法とする互いに異なる右剰余類たちの濃度$\#{G}/_r {H}$に等しいので、その濃度は群$(G,*)$とこれの部分群$(H,*)$によってのみ決定されることができる。したがって、その部分群$(H,*)$を法とする互いに異なる左剰余類たちの濃度$\#{G}/_l {H}$を群$(G,*)$における部分群$(H,*)$の指数といい、$(G:H)$と書く。
\end{dfn}
\begin{thm}\label{3.1.1.32}
群$(G,*)$の集合$G$の濃度$\#(G,*)$が有限であるなら、群$(G,*)$における部分群$(H,*)$の指数$(G:H)$も有限である。
\end{thm}
\begin{proof}
部分群$(H,*)$についても$\#H \leq \#G$が成り立つことにより、$a \in (G,*)$と$h \in (H,*)$との順序付けられた組の個数もせいぜい$\#G\#H$と有限であるから、定義より群$(G,*)$における部分群$(H,*)$の指数$(G:H)$の濃度$\#(G:H)$も有限となる。
\end{proof}
\begin{thm}\label{3.1.1.33}
群$(G,*)$の有限な部分群$(H,*)$について、部分群$(H,*)$を法とする任意の1つの左剰余類$a*H$の濃度$\#{a*H}$あるいはこれに対応する右剰余類$H*a$の濃度$\#{H*a}$はその部分群$(H,*)$の位数$o(H,*)$に等しい、即ち、次式が成り立つ。
\begin{align*}
\#{a*H} = \#{H*a} = o(H,*)
\end{align*}
\end{thm}
\begin{proof}
群$(G,*)$の有限な部分群$(H,*)$と$g,h \in H$なる元たち$g$、$h$について、写像$f:H \rightarrow a*H;h \mapsto a*h$が与えられると、簡約律より$g = h \Leftrightarrow a*g = a*h$が成り立ちこの写像$f$は全単射$f:H\overset{\sim}{\rightarrow}a*H$である。また、写像$f:H \rightarrow H*a;h \mapsto h*a$も同様にして、全単射$f:H\overset{\sim}{\rightarrow}H*a$であることが示される。
\end{proof}
\begin{thm}[群論に関するLagrangeの定理]\label{3.1.1.34}
群$(G,*)$の有限な部分群$(H,*)$について、次式が成り立つ。
\begin{align*}
o(G,*) = (G:H)o(H,*)
\end{align*}\par
この定理を群論に関するLagrangeの定理という。
\end{thm}
\begin{proof}
群$(G,*)$の有限な部分群$(H,*)$において、その部分群$(H,*)$を法とする左合同な関係によって集合$G$は$(G:H)$つの左剰余類$a_{i}*H$に分割され各左剰余類$a_{i}*H$の濃度$\#{a_{i}*H}$が$o(H,*)$に等しいので、次のようになる\footnote{集合$\varLambda_{n}$は1から自然数$n$までの自然数全体の集合という意味です。ですので、$\varLambda_{0} = \emptyset $となります。}。
\begin{align*}
o(G,*) = \sum_{i \in \varLambda_{(G:H)}} {\#\left( a_{i}*H \right)} = \sum_{i \in \varLambda_{(G:H)}} {o(H,*)} = (G:H)o(H,*)
\end{align*}
\end{proof}
\begin{thm}\label{3.1.1.35}
群$(G,*)$において、群論に関するLagrangeの定理より直ちに次のことが分かる。
\begin{itemize}
\item
  有限群$(G,*)$の任意の部分群$(H,*)$の位数$o(H,*)$はその群$(G,*)$の位数$o(G,*)$の約数である。
\item
  $(G:G) = 1$が成り立つ。
\item
  $\left( G:\left\{ 1_{(G,*)} \right\} \right) = o(G,*)$が成り立つ。
\end{itemize}
\end{thm}
\begin{proof}
有限群$(G,*)$の任意の部分群$(H,*)$の位数$o(H,*)$はその群$(G,*)$の位数$o(G,*)$の約数であることは群論に関するLagrangeの定理より明らかである。また、群論に関するLagrangeの定理より$o(G,*) = (G:G)o(G,*)$が成り立ち、両辺に$o(G,*)$で割れば、$(G:G) = 1$が得られる。また、$o\left( \left\{ 1_{(G,*)} \right\},* \right) = 1$が成り立つので、群論に関するLagrangeの定理より$o(G,*) = \left( G,\left\{ 1_{(G,*)} \right\} \right)o\left( \left\{ 1_{(G,*)} \right\},* \right) = \left( G,\left\{ 1_{(G,*)} \right\} \right)$が得られる。
\end{proof}
\begin{thm}\label{3.1.1.36}
群$(G,*)$の2つの部分群たち$(H,*)$、$(I,*)$が与えられ$H \subseteq I$が成り立ちその部分群$(I,*)$のその群$(G,*)$における指数$(G:I)$、その部分群$(H,*)$のその群$(I,*)$における指数$(I:H)$がともに有限であるとする\footnote{疑い深くなっちゃうとついつい見落としがちですが、定理\ref{3.1.1.7}から分かるようにその部分群$(H,*)$はその部分群$(I,*)$の部分群でもあることに注意してください…。}。このとき、その部分群$(H,*)$のその群$(G,*)$における指数$(G:H)$は有限で次式が成り立つ。
\begin{align*}
(G:H) = (G:I)(I:H)
\end{align*}
\end{thm}
\begin{proof}
群$(G,*)$の2つの部分群たち$(H,*)$、$(I,*)$が与えられ$H \subseteq I$が成り立ちその部分群$(I,*)$のその群$(G,*)$における指数$(G:I)$、その部分群$(H,*)$のその群$(I,*)$における指数$(I:H)$がともに有限であるとする。このとき、群論に関するLagrangeの定理より次式が成り立つ。
\begin{align*}
\left\{ \begin{matrix}
o(G,*) = (G:H)o(H,*) \\
o(G,*) = (G:I)o(I,*) \\
o(I,*) = (I:H)o(H,*) \\
\end{matrix} \right.\ 
\end{align*}
したがって、$o(G,*)$と$o(H,*)$、$o(I,*)$を消去していけば、次のようになり、
\begin{align*}
\left\{ \begin{matrix}
o(G,*) = (G:H)o(H,*) \\
o(G,*) = (G:I)o(I,*) \\
o(I,*) = (I:H)o(H,*) \\
\end{matrix} \right. &\Rightarrow \left\{ \begin{matrix}
(G:H)o(H,*) = (G:I)o(I,*) \\
o(I,*) = (I:H)o(H,*) \\
\end{matrix} \right.\ \\
&\Rightarrow (G:H)o(H,*)o(I,*) = (G:I)o(I*)(I:H)o(H,*)\\
&\Leftrightarrow (G:H)o(H,*)o(I,*) = (G:I)(I:H)o(I,*)o(H,*)\\
&\Leftrightarrow (G:H) = (G:I)(I:H)
\end{align*}
よって、その部分群$(H,*)$のその群$(G,*)$における指数$(G:H)$は有限であることが分かる。
\end{proof}
\begin{thm}\label{3.1.1.37}
群$(G,*)$の2つの部分群たち$(H,*)$、$(I,*)$が与えられその部分群$(H,*)$のその群$(G,*)$における指数$(G:H)$が有限であるとする。このとき、その部分群$(H \cap I,*)$の\footnote{このことは定理\ref{3.1.1.10}によって保障されますので、ご安心ください。}その群$(G,*)$における指数$(G:H \cap I)$、その部分群$(I,*)$における指数$(I:H \cap I)$はいずれも有限で次式が成り立つ\footnote{証明で少し自信がないので、間違っていたらご一報ください…。}。
\begin{align*}
(I:H \cap I) \leq (G:H \cap I) \leq (G:H)
\end{align*}
\end{thm}
\begin{proof}
群$(G,*)$の2つの部分群たち$(H,*)$、$(I,*)$が与えられその部分群$(H,*)$のその群$(G,*)$における指数$(G:H)$が有限であるとする。このとき、次式のような写像$f$が考えられれば、
\begin{align*}
f:{G}/_l {(H \cap I)} \rightarrow {G}/_l {H};a*(H \cap I) \mapsto a*H
\end{align*}
この写像$f$は単射であるから、$\#{G}/_l {(H \cap I)} \leq \#{G}/_l {H}$が成り立ち、これにより、$(G:H \cap I) \leq (G:H)$が成り立つことになる。\par
ここで、その部分群$(H \cap I,*)$を法とするその部分群$(I,*)$の左剰余類たち全体の集合を${I}/_l {(H \cap I)}$とおくと、${I}/_l {(H \cap I)} \subseteq {G}/_l {(H \cap I)}$が成り立つので、$\#{I}/_l {(H \cap I)} \leq \#{G}/_l {(H \cap I)}$が成り立ち、これにより、$(I:H \cap I) \leq (G:H \cap I)$が成り立つことになる。\par
以上より、次式が得られ、
\begin{align*}
(I:H \cap I) \leq (G:H \cap I) \leq (G:H)
\end{align*}
よって、その部分群$(H \cap I,*)$のその群$(G,*)$における指数$(G:H \cap I)$、その部分群$(I,*)$における指数$(I:H \cap I)$はいずれも有限であることが分かる。
\end{proof}
\begin{thm}[群論に関するPoincaréの定理]\label{3.1.1.38}
$\forall i \in \varLambda_{n}$に対し、群$(G,*)$の部分群たち$\left( H_{i},* \right)$が与えられどの部分群$\left( H_{i},* \right)$のその群$(G,*)$における指数$\left( G:H_{i} \right)$も有限であるとする。このとき、その部分群$\left( \bigcap_{i \in \varLambda_{n}} H_{i},* \right)$のその群$(G,*)$における指数$\left( G:\bigcap_{i \in \varLambda_{n}} H_{i} \right)$も有限であり次式が成り立つ。
\begin{align*}
\left( G:\bigcap_{i \in \varLambda_{n}} H_{i} \right) \leq \prod_{i \in \varLambda_{n}} \left( G:H_{i} \right)
\end{align*}\par
この定理を群論に関するPoincaréの定理という。
\end{thm}
\begin{proof}
$\forall i \in \varLambda_{n}$に対し、群$(G,*)$の部分群たち$\left( H_{i},* \right)$が与えられどの部分群$\left( H_{i},* \right)$のその群$(G,*)$における指数$\left( G:H_{i} \right)$も有限であるとする。$n = 1$のときは明らかにその部分群$\left( H_{1},* \right)$のその群$(G,*)$における指数$\left( G:H_{1} \right)$も有限であり次式が成り立つ。
\begin{align*}
\left( G:H_{1} \right) \leq \left( G:H_{1} \right)
\end{align*}\par
$n = k$のとき、定理\ref{3.1.1.10}より数学的帰納法でその組$\left( \bigcap_{i \in \varLambda_{k}} H_{i},* \right)$はその群$(G,*)$の部分群をなすことが保証される。ここで、その部分群$\left( \bigcap_{i \in \varLambda_{k}} H_{i},* \right)$のその群$(G,*)$における指数$\left( G:\bigcap_{i \in \varLambda_{k}} H_{i} \right)$も有限であり次式が成り立つと仮定しよう。
\begin{align*}
\left( G:\bigcap_{i \in \varLambda_{k}} H_{i} \right) \leq \prod_{i \in \varLambda_{k}} \left( G:H_{i} \right)
\end{align*}\par
$n = k + 1$のとき、$\bigcap_{i \in \varLambda_{k + 1}} H_{i} \subseteq \bigcap_{i \in \varLambda_{k}} H_{i}$が成り立つので、定理\ref{3.1.1.36}より次のようになり、
\begin{align*}
\left( G:\bigcap_{i \in \varLambda_{k + 1}} H_{i} \right) = \left( G:\bigcap_{i \in \varLambda_{k}} H_{i} \right)\left( \bigcap_{i \in \varLambda_{k}} H_{i}:\bigcap_{i \in \varLambda_{k + 1}} H_{i} \right)
\end{align*}
ここで、仮定より次のようになる。
\begin{align*}
\left( G:\bigcap_{i \in \varLambda_{k + 1}} H_{i} \right) &= \left( G:\bigcap_{i \in \varLambda_{k}} H_{i} \right)\left( \bigcap_{i \in \varLambda_{k}} H_{i}:\bigcap_{i \in \varLambda_{k + 1}} H_{i} \right)\\
&\leq \prod_{i \in \varLambda_{k}} \left( G:H_{i} \right)\left( \bigcap_{i \in \varLambda_{k}} H_{i}:\bigcap_{i \in \varLambda_{k + 1}} H_{i} \right)
\end{align*}
ここで、定理\ref{3.1.1.37}より次式が成り立つので、
\begin{align*}
\left( \bigcap_{i \in \varLambda_{k}} H_{i}:\bigcap_{i \in \varLambda_{k + 1}} H_{i} \right) &= \left( \bigcap_{i \in \varLambda_{k}} H_{i}:\bigcap_{i \in \varLambda_{k}} H_{i} \cap H_{k + 1} \right)\\
&\leq \left( G:H_{k + 1} \right)
\end{align*}
したがって、次式が成り立つ。
\begin{align*}
\left( G:\bigcap_{i \in \varLambda_{k + 1}} H_{i} \right) &\leq \prod_{i \in \varLambda_{k}} \left( G:H_{i} \right)\left( \bigcap_{i \in \varLambda_{k}} H_{i}:\bigcap_{i \in \varLambda_{k + 1}} H_{i} \right)\\
&\leq \prod_{i \in \varLambda_{k}} \left( G:H_{i} \right)\left( G:H_{k + 1} \right)\\
&= \prod_{i \in \varLambda_{k + 1}} \left( G:H_{i} \right)
\end{align*}
このとき、その部分群$\left( \bigcap_{i \in \varLambda_{k + 1}} H_{i},* \right)$のその群$(G,*)$における指数$\left( G:\bigcap_{i \in \varLambda_{k + 1}} H_{i} \right)$も有限であることが分かる。\par
以上より、数学的帰納法によってその部分群$\left( \bigcap_{i \in \varLambda_{n}} H_{i},* \right)$のその群$(G,*)$における指数$\left( G:\bigcap_{i \in \varLambda_{n}} H_{i} \right)$も有限であり次式が成り立つことが示された。
\begin{align*}
\left( G:\bigcap_{i \in \varLambda_{n}} H_{i} \right) \leq \prod_{i \in \varLambda_{n}} \left( G:H_{i} \right)
\end{align*}
\end{proof}
\begin{thm}\label{3.1.1.39}
群$(G,*)$の正規部分群$(N,*)$を法とする商群$\left( {G}/{N},* \right)$が与えられたとき、その正規部分群$(N,*)$のその群$(G,*)$における指数$(G:N)$が有限であるなら、次のことが満たされる。
\begin{itemize}
\item
  その商群$\left( {G}/{N},* \right)$の位数$o\left( {G}/{N},* \right)$も有限で$o\left( {G}/{N},* \right) = (G:N)$が成り立つ。
\item
  その群$(G,*)$の位数$o(G,*)$が有限であるとき、次式が成り立つ。
\begin{align*}
o\left( {G}/{N},* \right) = \frac{o(G,*)}{o(N,*)}
\end{align*}
\end{itemize}
\end{thm}
\begin{proof}
群$(G,*)$の正規部分群$(N,*)$を法とする商群$\left( {G}/{N},* \right)$が与えられたとき、その正規部分群$(N,*)$のその群$(G,*)$における指数$(G:N)$が有限であるなら、その指数$(G:N)$は定義よりその部分群$(N,*)$を法とするその群$(G,*)$の左剰余類たち全体の集合${G}/_l {N}$の濃度であるから、$\#{G}/_l {N} = (G:N)$が成り立つ。一方で、その部分群$(N,*)$は正規であるから、その部分群$(N,*)$を法とするその群$(G,*)$の左剰余類たちは剰余類たちでもあるので、${G}/_l {N} = {G}/{N}$が成り立ち、したがって、次式が得られる。
\begin{align*}
o\left( {G}/{N},* \right) &= \#{G}/{N}\\
&= \#{G}/_l {N}\\
&= (G:N)
\end{align*}
これにより、その商群$\left( {G}/{N},* \right)$の位数$o\left( {G}/{N},* \right)$も有限であることも分かる。\par
その群$(G,*)$の位数$o(G,*)$が有限であるとき、群論に関するLagrangeの定理より次式が成り立つ。
\begin{align*}
o(G,*) = (G:N)o(N,*)
\end{align*}
ここで、上記の議論により$o\left( {G}/{N},* \right) = (G:N)$が成り立つので、次式が得られる。
\begin{align*}
o\left( {G}/{N},* \right) &= (G:N)\\
&= \frac{(G:N) \cdot o(N,*)}{o(N,*)}\\
&= \frac{o(G,*)}{o(N,*)}
\end{align*}
\end{proof}
%\hypertarget{ux7fa4ux306eux4f4dux6570ux306bux95a2ux3059ux308bux7a4dux306eux6cd5ux5247}{%
\subsubsection{群の位数に関する積の法則}%\label{ux7fa4ux306eux4f4dux6570ux306bux95a2ux3059ux308bux7a4dux306eux6cd5ux5247}}
\begin{thm}[群の位数に関する積の法則]\label{3.1.1.40}
群$(G,*)$の有限な部分群たち$(H,*)$、$(I,*)$が与えられたとき、次式が成り立つ。
\begin{align*}
\#{H*I} = \frac{o(H,*)o(I,*)}{o(H \cap I,*)}
\end{align*}\par
この定理を群の位数に関する積の法則という。
\end{thm}
\begin{proof}
群$(G,*)$の有限な部分群たち$(H,*)$、$(I,*)$が与えられたとき、集合$H \times I$上で次の関係$R$を定義すると、
\begin{align*}
(h,i)R\left( h',i' \right) \Leftrightarrow h*i = h'*i'
\end{align*}
この関係$R$は同値関係となる。実際、$h*i = h*i$より$(h,i)R(h,i)$が成り立つし、$h*i = h'*i'$が成り立つならそのときに限り、$h'*i' = h*i$が成り立つことにより$(h,i)R\left( h',i' \right)$が成り立つなら、$\left( h',i' \right)R(h,i)$が成り立つし、$(h,i)R\left( h',i' \right)$かつ$\left( h',i' \right)R\left( h'',i'' \right)$が成り立つなら、$h*i = h'*i' = h''*i''$が成り立つので、したがって、$(h,i)R\left( h'',i'' \right)$が成り立つことになる。これにより、商集合${(H \times I)}/{R}$が定義されて次式が成り立つ。
\begin{align*}
H \times I = \bigsqcup_{} {(H \times I)}/{R}
\end{align*}\par
ここで、次のように写像$f$が考えられれば、
\begin{align*}
f:H*I \rightarrow {(H \times I)}/{R};h*i \mapsto C_{R}(h,i)
\end{align*}
$\forall h*i \in H*I$に対し、$h*i = h'*i'$とおかれれば、$(h,i)R\left( h',i' \right)$が成り立つので、次のようになる。
\begin{align*}
C_{R}(h,i) &= \left\{ \left( h'',i'' \right) \in H \times I \middle| \left( h'',i'' \right)R(h,i) \land (h,i)R\left( h',i' \right) \right\}\\
&= \left\{ \left( h'',i'' \right) \in H \times I \middle| \left( h'',i'' \right)R\left( h',i' \right) \land \left( h',i' \right)R(h,i) \right\}\\
&= C_{R}\left( h',i' \right)
\end{align*}
したがって、その対応$f$は確かに写像となる。ここで、次のような写像$g$が考えられよう。
\begin{align*}
g:{(H \times I)}/{R} \rightarrow H*I;C_{R}(h,i) \mapsto h*i
\end{align*}
上と同様にして考えれば、これ$g$も確かに写像で、$\forall h*i \in H*I$に対し、次のようになるかつ、
\begin{align*}
g \circ f(h*i) = g\left( C_{R}(h,i) \right) = h*i
\end{align*}
$\forall C_{R}(h,i) \in {(H \times I)}/{R}$に対し、次のようになるので、
\begin{align*}
f \circ g\left( C_{R}(h,i) \right) = f(h*i) = C_{R}(h,i)
\end{align*}
$g = f^{- 1}$が成り立つ。これにより、$\#{H*I} = \#{(H \times I)}/{R}$が成り立つ。\par
$\forall(h,i),\left( h',i' \right) \in H \times I$に対し、$(h,i)R\left( h',i' \right)$が成り立つなら、$h*i = h'*i'$が成り立ち、したがって、次式が成り立つので、
\begin{align*}
h^{- 1}*h' &= 1_{(G,*)}*h^{- 1}*h'\\
&= i*i^{- 1}*h^{- 1}*h'\\
&= i*(h*i)^{- 1}*h'\\
&= i*\left( h'*i' \right)^{- 1}*h'\\
&= i*{i'}^{- 1}*{h'}^{- 1}*h'\\
&= i*{i'}^{- 1}*1_{(G,*)}\\
&= i*{i'}^{- 1}
\end{align*}
$h^{- 1}*h' = i*{i'}^{- 1}$が得られる。ここで、$h^{- 1}*h' \in H$かつ$i*{i'}^{- 1} \in I$が成り立つので、$h^{- 1}*h' \in H \cap I$が得られる。逆に、$\forall j \in H \cap I$に対し、$\left( h',i' \right) = \left( h*j,j^{- 1}*i \right)$とおかれれば、$j = h^{- 1}*h' = i*i^{- 1}$が成り立つので、次のようになることから、
\begin{align*}
h*i &= \left( i^{- 1}*h^{- 1} \right)^{- 1}\\
&= \left( i^{- 1}*h^{- 1}*1_{(G,*)} \right)^{- 1}\\
&= \left( i^{- 1}*h^{- 1}*h'*{h'}^{- 1} \right)^{- 1}\\
&= \left( i^{- 1}*i*{i'}^{- 1}*{h'}^{- 1} \right)^{- 1}\\
&= \left( 1_{(G,*)}*{i'}^{- 1}*{h'}^{- 1} \right)^{- 1}\\
&= \left( {i'}^{- 1}*{h'}^{- 1} \right)^{- 1}\\
&= h'*i'
\end{align*}
$(h,i)R\left( h',i' \right)$が成り立ち、よって、$\left( h',i' \right) \in C_{R}(h,i)$が成り立つ。\par
そこで、$\forall(h,i) \in H \times I$に対し、次のように写像たち$f'$、$g'$が定義されれば、
\begin{align*}
f':C_{R}(h,i) \rightarrow H \cap I;\left( h',i' \right) \mapsto h^{- 1}*h'
\end{align*}
\begin{align*}
g':H \cap I \rightarrow C_{R}(h,i);j \mapsto \left( h*j,j^{- 1}*i \right)
\end{align*}
$\forall\left( h',i' \right) \in C_{R}(h,i)$に対し、$(h,i)R\left( h',i' \right)$が成り立つので、$h*i = h'*i'$が成り立ち、上と同様にして考えれば、$f'\left( h',i' \right) = h^{- 1}*h' = i*{i'}^{- 1}$が成り立つ。したがって、次のようになるかつ、
\begin{align*}
g' \circ f'\left( h',i' \right) &= g'\left( h^{- 1}*i' \right)\\
&= \left( h*h^{- 1}*h',\left( h^{- 1}*h' \right)^{- 1}*h_{2} \right)\\
&= \left( h*h^{- 1}*h',\left( i*{i'}^{- 1} \right)^{- 1}*h_{2} \right)\\
&= \left( h*h^{- 1}*h',i'*i^{- 1}*h_{2} \right)\\
&= \left( 1_{(G,*)}*h',i'*1_{(G,*)} \right)\\
&= \left( h',i' \right)
\end{align*}
$\forall h \in H \cap I$に対し、次のようになることから、
\begin{align*}
f' \circ g'(j) &= f'\left( h*j,j^{- 1}*i \right)\\
&= h^{- 1}*h*j\\
&= 1_{(G,*)}*j = j
\end{align*}
$g' = {f'}^{- 1}$が成り立つ。これにより、$\#{C_{R}(h,i)} = \#{H \cap I}$が成り立つ。\par
以上の議論と定理\ref{3.1.1.33}より次式が成り立つ。
\begin{align*}
o(H,*) \cdot o(I,*) &= \#H\#I\\
&= \#{H \times I}\\
&= \#{\bigsqcup_{} {(H \times I)}/{R}}\\
&= \#{\bigsqcup_{C_{R}(h,i) \in {(H \times I)}/{R}} {C_{R}(h,i)}}\\
&= \sum_{C_{R}(h,i) \in {(H \times I)}/{R}} {\#{C_{R}(h,i)}}\\
&= \sum_{C_{R}(h,i) \in {(H \times I)}/{R}} {\#{H \cap I}}\\
&= \sum_{C_{R}(h,i) \in {(H \times I)}/{R}} {o(H \cap I,*)}\\
&= \#{(H \times I)}/{R} \cdot o(H \cap I,*)\\
&= \#{H*I} \cdot o(H \cap I,*)
\end{align*}
よって、次式が成り立つ。
\begin{align*}
\#{H*I} = \frac{o(H,*)o(I,*)}{o(H \cap I,*)}
\end{align*}
\end{proof}
\begin{thm}\label{3.1.1.41}
有限な群$(G,*)$の部分群たち$(H,*)$、$(I,*)$が与えられたとき、$\sqrt{o(G,*)} < o(H,*)$かつ$\sqrt{o(G,*)} < o(I,*)$が成り立つなら、$H \cap I \neq \left\{ 1_{(G,*)} \right\}$が成り立つ。
\end{thm}
\begin{proof}
有限な群$(G,*)$の部分群たち$(H,*)$、$(I,*)$が与えられたとき、$\sqrt{o(G,*)} < o(H,*)$かつ$\sqrt{o(G,*)} < o(I,*)$が成り立つなら、$H*I \subseteq G$が成り立つので、次のようになる。
\begin{align*}
\frac{o(G,*)}{o(H \cap I,*)} &< \frac{o(H,*)o(I,*)}{o(H \cap I,*)}\\
&= \#{H*I}\\
&\leq \#G = o(G,*)
\end{align*}
これにより、$1 < o(H \cap I,*)$が得られるので、$H \cap I \neq \left\{ 1_{(G,*)} \right\}$が成り立つ。
\end{proof}
%\hypertarget{ux3055ux307eux3056ux307eux306aux6b63ux898fux90e8ux5206ux7fa4}{%
\subsubsection{さまざまな正規部分群}%\label{ux3055ux307eux3056ux307eux306aux6b63ux898fux90e8ux5206ux7fa4}}
\begin{dfn}
任意の群$(G,*)$に対し、集合$G$の全ての元と可換であるようなその集合$G$の元全体の集合、即ち、$\forall a \in G$に対し、$a*z = z*a$が成り立つようなその集合$G$の元$z$全体の集合を群$(G,*)$の群論上中心、または、中心といい$Z(G)$と書く。
\end{dfn}
\begin{thm}\label{3.1.1.42}
組$\left( Z(G),* \right)$はその群$(G,*)$の正規部分群である、即ち、$\left( Z(G),* \right) \trianglelefteq (G,*)$が成り立つ。
\end{thm}
\begin{proof}
群$(G,*)$が与えられたとき、これの群論上中心$Z(G)$を用いた組$\left( Z(G),* \right)$について、この群論上中心$Z(G)$はその集合$G$の部分集合であり、$\forall x,y \in Z(G)\forall a \in G$に対し、次式が成り立つことにより、
\begin{align*}
a*(x*y) &= (a*x)*y\\
&= (x*a)*y\\
&= x*(a*y)\\
&= x*(y*a)\\
&= (x*y)*a
\end{align*}
$x*y \in Z(G)$が成り立ち、さらに、次式が成り立つことにより、
\begin{align*}
a*x^{- 1} &= (e*a)*x^{- 1}\\
&= \left( \left( x^{- 1}*x \right)*a \right)*x^{- 1}\\
&= x^{- 1}*\left( (x*a)*x^{- 1} \right)\\
&= x^{- 1}*\left( a*\left( x*x^{- 1} \right) \right)\\
&= x^{- 1}*\left( a*1_{(G,*)} \right)\\
&= x^{- 1}*a
\end{align*}
$x^{- 1} \in Z(G)$が成り立つので、組$\left( Z(G),* \right)$はその群$(G,*)$の部分群となる。また、次式が成り立つので、
\begin{align*}
a*Z(G) &= \left\{ a*z \in G \middle| z \in Z(G) \right\}\\
&= \left\{ z*a \in G \middle| z \in Z(G) \right\}\\
&= Z(G)*a
\end{align*}
その部分群$\left( Z(G),* \right)$は群$(G,*)$の正規部分群である。
\end{proof}
\begin{dfn}
群$(G,*)$において、その集合$G$の元々$a$、$b$を用いて$a*b*a^{- 1}*b^{- 1}$の形で書かれる元を群$(G,*)$の交換子といい、これらからなる集合によって生成されるその群$(G,*)$の部分群$(H,*)$をその群$(G,*)$の交換子群といい、$\left( D(G),* \right)$と書く。
\end{dfn}
\begin{thm}\label{3.1.1.43}
これについて次のことが成り立つ。
\begin{itemize}
\item
  群$(G,*)$の交換子の逆元も群$(G,*)$の交換子である。
\item
  群$(G,*)$が可換的であるなら、その群$(G,*)$の交換子群は単位群である。
\item
  群$(G,*)$の部分群$(N,*)$とその群$(G,*)$の交換子群$\left( D(G),* \right)$が与えられたとき、$D(G) \subseteq N$が成り立つなら、$(N,*) \trianglelefteq (G,*)$が成り立つ。
\item
  $\left( D(G),* \right) \trianglelefteq (G,*)$が成り立つ。
\end{itemize}
\end{thm}
\begin{proof}
群$(G,*)$において、その集合$G$の元々$a$、$b$を用いた交換子$a*b*a^{- 1}*b^{- 1}$の逆元について考えよう。このとき、次のようになる。
\begin{align*}
\left( a*b*a^{- 1}*b^{- 1} \right)^{- 1} &= \left( b^{- 1} \right)^{- 1}*\left( a^{- 1} \right)^{- 1}*b^{- 1}*a^{- 1}\\
&= b*a*a^{- 1}*b^{- 1}
\end{align*}
以上より、その群$(G,*)$の交換子の逆元もその群$(G,*)$の交換子である。\par
群$(G,*)$が可換的であるなら、その集合$G$の元々$a$、$b$を用いた交換子$a*b*a^{- 1}*b^{- 1}$について、次のようになる。
\begin{align*}
a*b*a^{- 1}*b^{- 1} &= (a*b)*\left( b^{- 1}*a^{- 1} \right)\\
&= a*\left( b*b^{- 1} \right)*a^{- 1}\\
&= a*1_{(G,*)}*a^{- 1}\\
&= a*a^{- 1} = 1_{(G,*)}
\end{align*}
したがって、集合$\left\{ 1_{(G,*)} \right\} \cup \left\{ 1_{(G,*)}^{- 1} \right\}$の有限個の元たちの算法$*$の像全体の集合は$\left\{ 1_{(G,*)} \right\}$であり、その組$\left( \left\{ 1_{(G,*)} \right\},* \right)$も部分群であるから、その群$(G,*)$の交換子群は単位群である。\par
群$(G,*)$の部分群$(N,*)$とその群$(G,*)$の交換子群$\left( D(G),* \right)$が与えられたとき、$D(G) \subseteq N$が成り立つなら、集合$N$はその群$(G,*)$の全ての交換子を含み、$\forall a \in G\forall n \in N$に対し、$a*n*a^{- 1}*n^{- 1}$は、その組$(N,*)$が群$(G,*)$の部分群であるので、集合$N$の元であり次のようになる。
\begin{align*}
\left( a*n*a^{- 1}*n^{- 1} \right)*n &= \left( a*n*a^{- 1} \right)*\left( n^{- 1}*n \right)\\
&= \left( a*n*a^{- 1} \right)*1_{(G,*)}\\
&= a*n*a^{- 1} \in N
\end{align*}
よって、定理\ref{3.1.1.25}より$(N,*) \trianglelefteq (G,*)$が成り立つ。\par
特に、$D(G) \subseteq Z(G)$が成り立つかつ、上記の議論によりその部分群$\left( Z(G),* \right)$は群$(G,*)$について正規であったので、$\left( D(G),* \right) \trianglelefteq (G,*)$が成り立つ。
\end{proof}
\begin{thm}\label{3.1.1.44}
群$(G,*)$の部分群$(N,*)$が与えられたとき、$(G:N) = 2$が成り立つなら、$(N,*) \trianglelefteq (G,*)$が成り立つ。
\end{thm}
\begin{proof}
群$(G,*)$の部分群$(N,*)$が与えられたとき、$(G:N) = 2$が成り立つなら、その部分群$(N,*)$を法とする左剰余類が2つしかないことになる。このようなものうち1つはその集合$N$自身であるので、残り1つの左剰余類が$a*N$とおかれることにする。このとき、次のようになることから、
\begin{align*}
G &= \bigsqcup_{} {G}/ {\equiv_{l}\ \mathrm{mod}(N,*)}\\
&= \bigsqcup_{} {G}/_l {N}\\
&= \bigsqcup_{C \in {G}/_l {N}} C\\
&= N \sqcup a*N
\end{align*}
$\forall b \in G$に対し、$b \in N$が成り立つ場合と$b \in a*N$が成り立つ場合と場合分けができる。\par
$b \in N$が成り立つ場合、$\forall n \in N$に対し、$b^{- 1}*n*b \in N$が成り立つので、定理\ref{3.1.1.22}より$(N,*) \trianglelefteq (G,*)$が成り立つ。\par
$b \in a*N$が成り立つ場合、$a^{- 1}*b \in N$が成り立つことになる。$(N,*) \trianglelefteq (G,*)$が成り立たないと仮定すると、定理\ref{3.1.1.25}より$\exists n \in N$に対し、$b*n*b^{- 1} \in N$が成り立たない。ここで、$G = N \sqcup a*N$より$b*n*b^{- 1} \in a*N$が成り立つことになるので、$a^{- 1}*b*n*b^{- 1} \in N$が成り立ち、したがって、次のようになる。
\begin{align*}
\left( a^{- 1}*b*n*b^{- 1} \right)^{- 1}*\left( a^{- 1}*b \right)*n &= b*n^{- 1}*\left( a^{- 1}*b \right)^{- 1}*\left( a^{- 1}*b \right)*n\\
&= b*n^{- 1}*1_{(G,*)}*n\\
&= b*n^{- 1}*n\\
&= b*1_{(G,*)} = b
\end{align*}
これにより、$b = \left( a^{- 1}*b*n*b^{- 1} \right)^{- 1}*\left( a^{- 1}*b \right)*n \in N$が成り立つことになるので、$G = N \sqcup a*N$より$b \in a*N$が成り立たない。しかしながら、これは仮定の$b \in a*N$が成り立つことに矛盾する。ゆえに、$(N,*) \trianglelefteq (G,*)$が成り立つ。
\end{proof}
\begin{dfn}
群$(G,*)$において、空集合でない任意のその集合$G$の部分集合$S$に対し、$a*S = S*a$なる元$a$全体の集合を$N(G,S)$、$\forall s \in S$に対し、$a*s = s*a$が成り立つような元$a$全体の集合を$C(G,S)$とおく。このとき、のちに述べるようにそれらの組々$\left( N(G,S),* \right)$、$\left( C(G,S),* \right)$はその群$(G,*)$の部分群をなすことから、これらの部分群たち$\left( N(G,S),* \right)$、$\left( C(G,S),* \right)$をその群$(G,*)$の正規化群、中心化群という。
\end{dfn}
\begin{thm}\label{3.1.1.45}
群$(G,*)$と空集合でない任意のその集合$G$の部分集合$S$が与えられたとき、それらの組々$\left( N(G,S),* \right)$、$\left( C(G,S),* \right)$はその群$(G,*)$の部分群をなす。
\end{thm}
\begin{proof}
群$(G,*)$と空集合でない任意のその集合$G$の部分集合$S$が与えられたとき、その組$\left( N(G,S),* \right)$について、もちろん、$1_{(G,*)}*S = S*1_{(G,*)} = S$が成り立つので、$1_{(G,*)} \in N(G,S)$が成り立つ。$\forall a,b \in N(G,S)$に対し、$a*S = S*a$かつ$b*S = S*b$が成り立つことから、$\forall c \in G$に対し、$c \in a*b*S$が成り立つならそのときに限り、$\exists s \in S$に対し、$c = a*b*s$が成り立つ。これが成り立つならそのときに限り、$b*s \in b*S = S*b$より$\exists s' \in S$に対し、$b*s = s'*b$が成り立つので、$c = a*s'*b$が成り立つ。同様にして、$\exists s'' \in S$に対し、$c = s''*a*b$が成り立つことが示されるので、これが成り立つならそのときに限り、$c \in S*a*b$が成り立つ。以上より、$a*b*S = S*a*b$が得られたので、$a*b \in N(G,S)$が成り立つ。さらに、$\forall a \in N(G,S)$に対し、$a*S = S*a$が成り立つことから、$\forall c \in G$に対し、$c \in a^{- 1}*S$が成り立つならそのときに限り、$\exists s \in S$に対し、$c = a^{- 1}*s$が成り立つ。ここで、$a*S = S*a$より$\exists s' \in S$に対し、$s*a = a*s'$が成り立つので、次のようになる。
\begin{align*}
c &= a^{- 1}*s\\
&= a^{- 1}*s*1_{(G,*)}\\
&= a^{- 1}*s*a*a^{- 1}\\
&= a^{- 1}*a*s'*a^{- 1}\\
&= 1_{(G,*)}*s'*a^{- 1}\\
&= s'*a^{- 1}
\end{align*}
これにより、$\exists s \in S$に対し、$c = a^{- 1}*s$が成り立つならそのときに限り、$\exists s' \in S$に対し、$c = s'*a^{- 1}$が成り立つので、$c \in S*a^{- 1}$が成り立つ。これにより、$a^{- 1}*S = S*a^{- 1}$が得られたので、$a^{- 1} \in N(G,S)$が成り立つ。定理\ref{3.1.1.6}よりその組$\left( N(G,S),* \right)$はその群$(G,*)$の部分群をなす。\par
その組$\left( C(G,S),* \right)$について、もちろん、$\forall s \in S$に対し、$1_{(G,*)}*s = s*1_{(G,*)}$が成り立つので、$1_{(G,*)} \in C(G,S)$が成り立つ。$\forall a,b \in C(G,S)\forall s \in S$に対し、$a*s = s*a$かつ$b*s = s*b$が成り立つので、次のようになる。
\begin{align*}
a*b*s &= a*s*b\\
&= s*a*b
\end{align*}
以上より$a*b \in C(G,S)$が成り立つ。$\forall a \in C(G,S)\forall s \in S$に対し、$a*s = s*a$が成り立つので、次のようになる。
\begin{align*}
a^{- 1}*s &= a^{- 1}*s*1_{(G,*)}\\
&= a^{- 1}*s*a*a^{- 1}\\
&= a^{- 1}*a*s*a^{- 1}\\
&= 1_{(G,*)}*s*a^{- 1}\\
&= s*a^{- 1}
\end{align*}
以上より$a^{- 1} \in C(G,S)$が成り立つ。定理\ref{3.1.1.6}よりその組$\left( C(G,S),* \right)$はその群$(G,*)$の部分群をなす。
\end{proof}
\begin{thm}\label{3.1.1.46}
群$(G,*)$と空集合でない任意のその集合$G$の部分集合$S$が与えられたとき、正規化群$\left( N(G,S),* \right)$と中心化群$\left( C(G,S),* \right)$について、$\left( C(G,S),* \right) \trianglelefteq \left( N(G,S),* \right)$が成り立つ。
\end{thm}
\begin{proof}
群$(G,*)$と空集合でない任意のその集合$G$の部分集合$S$が与えられたとき、正規化群$\left( N(G,S),* \right)$と中心化群$\left( C(G,S),* \right)$について、$\forall a \in C(G,S)$に対し、$\forall s \in S$に対し、$a*s = s*a$が成り立つので、$\exists s,s' \in S$に対し、$a*s = s'*a$が成り立ち、したがって、$a*S = S*a$が成り立つ。これにより、$a \in N(G,S)$が成り立つので、$C(G,S) \subseteq N(G,S)$が成り立つ。そこで、定理\ref{3.1.1.6}よりその中心化群$\left( C(G,S),* \right)$はその正規化群$\left( N(G,S),* \right)$の部分群となる。\par
$\forall a \in N(G,S)\forall b \in C(G,S)$に対し、$a*S = S*a$が成り立つかつ、$\forall s \in S$に対し、$b*s = s*b$が成り立つことから、$\exists s' \in S$に対し、$a*s = s'*a$が成り立つことにより$a^{- 1}*s' = s*a^{- 1}$が成り立つので、次のようになる。
\begin{align*}
a^{- 1}*b*a*s &= a^{- 1}*b*s'*a\\
&= a^{- 1}*s'*b*a\\
&= s*a^{- 1}*b*a
\end{align*}
これにより、$a^{- 1}*b*a \in C(G,S)$が成り立つので、定理\ref{3.1.1.25}より$\left( C(G,S),* \right) \trianglelefteq \left( N(G,S),* \right)$が成り立つ。
\end{proof}
\begin{thm}\label{3.1.1.47}
群$(G,*)$とこれの部分群$(H,*)$が与えられたとき、正規化群$\left( N(G,H),* \right)$について、$(H,*) \trianglelefteq \left( N(G,H),* \right)$が成り立つ。
\end{thm}
\begin{proof}
群$(G,*)$とこれの部分群$(H,*)$が与えられたとき、正規化群$\left( N(G,H),* \right)$について、$\forall a,h \in G$に対し、$h \in H$が成り立つとき、$a \in h*H$が成り立つならそのときに限り、$\exists h' \in H$に対し、$a = h*h'$が成り立つ。このとき、$h*h'*h^{- 1} \in H$が成り立つので、$h'' = h*h'*h^{- 1}$とすれば、これが成り立つならそのときに限り、$\exists h'' \in H$に対し、$a = h''*h$が成り立ち、これが成り立つならそのときに限り、$a \in H*h$が成り立つ。以上より$h*H = H*h$が得られるので、$h \in N(G,H)$が成り立つ。以上より$H \subseteq N(G,H)$が得られる。そこで、定理\ref{3.1.1.6}よりその部分群$(H,*)$はその正規化群$\left( N(G,H),* \right)$の部分群となる。\par
ここで、$\forall a \in N(G,H)\forall h \in H$に対し、$a*H = H*a$が成り立つことから、$\exists h' \in H$に対し、$a*h' = h*a$が成り立つので、$a^{- 1}*h = h'*a^{- 1}$が得られ、したがって、次のようになる。
\begin{align*}
a^{- 1}*h*a &= a^{- 1}*a*h'\\
&= 1_{(G,*)}*h'\\
&= h' \in H
\end{align*}
これにより、$a^{- 1}*h*a \in H$が成り立つので、定理\ref{3.1.1.25}より$(H,*) \trianglelefteq \left( N(G,H),* \right)$が成り立つ。
\end{proof}
\begin{thm}\label{3.1.1.48}
群$(G,*)$とこれの部分群$(H,*)$、この部分群$(H,*)$の正規部分群$(N,*)$が与えられたとき、$H \subseteq N(G,N)$が成り立つ。
\end{thm}
\begin{proof}
群$(G,*)$とこれの部分群$(H,*)$、この部分群$(H,*)$の正規部分群$(N,*)$が与えられたとする。$\forall h \in G$に対し、$h \in H$が成り立つとき、$\forall a \in G$に対し、$a \in h*N$が成り立つならそのときに限り、$\exists n \in N$に対し、$a = h*n$が成り立ち、定理\ref{3.1.1.25}より$h*n*h^{- 1} \in N$が成り立つので、$n' = h*n*h^{- 1}$とすれば、次のようになることから、
\begin{align*}
a &= h*n\\
&= h*n*1_{(G,*)}\\
&= h*n*h^{- 1}*h\\
&= n'*h
\end{align*}
$\exists n' \in N$に対し、$a = n'*h$が成り立ち、これが成り立つならそのときに限り、$a \in N*h$が成り立つ。以上より$h*N = N*h$が成り立つので、$h \in N(G,N)$が成り立つ。これにより、$H \subseteq N(G,N)$が得られる。
\end{proof}
\begin{thebibliography}{50}
  \bibitem{1}
  松坂和夫, 代数系入門, 岩波書店, 1976. 新装版第2刷 p45-65 ISBN978-4-00-029873-5
  \bibitem{2}
  花木章秀. "群論". 信州大学. \url{http://math.shinshu-u.ac.jp/~hanaki/edu/group/group2011pre.pdf} (2022-10-24 4:42 閲覧)
\end{thebibliography}
\end{document}
