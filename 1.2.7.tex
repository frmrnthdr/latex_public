\documentclass[dvipdfmx]{jsarticle}
\setcounter{section}{2}
\setcounter{subsection}{6}
\usepackage{xr}
\usepackage{amsmath,amsfonts,amssymb,array,comment,mathtools,url,docmute}
\usepackage{longtable,booktabs,dcolumn,tabularx,mathtools,multirow,colortbl,xcolor}
\usepackage[dvipdfmx]{graphics}
\usepackage{bmpsize}
\usepackage{amsthm}
\usepackage{enumitem}
\setlistdepth{20}
\renewlist{itemize}{itemize}{20}
\setlist[itemize]{label=•}
\renewlist{enumerate}{enumerate}{20}
\setlist[enumerate]{label=\arabic*.}
\setcounter{MaxMatrixCols}{20}
\setcounter{tocdepth}{3}
\newcommand{\rotin}{\text{\rotatebox[origin=c]{90}{$\in $}}}
\newcommand{\amap}[6]{\text{\raisebox{-0.7cm}{\begin{tikzpicture} 
  \node (a) at (0, 1) {$\textstyle{#2}$};
  \node (b) at (#6, 1) {$\textstyle{#3}$};
  \node (c) at (0, 0) {$\textstyle{#4}$};
  \node (d) at (#6, 0) {$\textstyle{#5}$};
  \node (x) at (0, 0.5) {$\rotin $};
  \node (x) at (#6, 0.5) {$\rotin $};
  \draw[->] (a) to node[xshift=0pt, yshift=7pt] {$\textstyle{\scriptstyle{#1}}$} (b);
  \draw[|->] (c) to node[xshift=0pt, yshift=7pt] {$\textstyle{\scriptstyle{#1}}$} (d);
\end{tikzpicture}}}}
\newcommand{\twomaps}[9]{\text{\raisebox{-0.7cm}{\begin{tikzpicture} 
  \node (a) at (0, 1) {$\textstyle{#3}$};
  \node (b) at (#9, 1) {$\textstyle{#4}$};
  \node (c) at (#9+#9, 1) {$\textstyle{#5}$};
  \node (d) at (0, 0) {$\textstyle{#6}$};
  \node (e) at (#9, 0) {$\textstyle{#7}$};
  \node (f) at (#9+#9, 0) {$\textstyle{#8}$};
  \node (x) at (0, 0.5) {$\rotin $};
  \node (x) at (#9, 0.5) {$\rotin $};
  \node (x) at (#9+#9, 0.5) {$\rotin $};
  \draw[->] (a) to node[xshift=0pt, yshift=7pt] {$\textstyle{\scriptstyle{#1}}$} (b);
  \draw[|->] (d) to node[xshift=0pt, yshift=7pt] {$\textstyle{\scriptstyle{#2}}$} (e);
  \draw[->] (b) to node[xshift=0pt, yshift=7pt] {$\textstyle{\scriptstyle{#1}}$} (c);
  \draw[|->] (e) to node[xshift=0pt, yshift=7pt] {$\textstyle{\scriptstyle{#2}}$} (f);
\end{tikzpicture}}}}
\renewcommand{\thesection}{第\arabic{section}部}
\renewcommand{\thesubsection}{\arabic{section}.\arabic{subsection}}
\renewcommand{\thesubsubsection}{\arabic{section}.\arabic{subsection}.\arabic{subsubsection}}
\everymath{\displaystyle}
\allowdisplaybreaks[4]
\usepackage{vtable}
\theoremstyle{definition}
\newtheorem{thm}{定理}[subsection]
\newtheorem*{thm*}{定理}
\newtheorem{dfn}{定義}[subsection]
\newtheorem*{dfn*}{定義}
\newtheorem{axs}[dfn]{公理}
\newtheorem*{axs*}{公理}
\renewcommand{\headfont}{\bfseries}
\makeatletter
  \renewcommand{\section}{%
    \@startsection{section}{1}{\z@}%
    {\Cvs}{\Cvs}%
    {\normalfont\huge\headfont\raggedright}}
\makeatother
\makeatletter
  \renewcommand{\subsection}{%
    \@startsection{subsection}{2}{\z@}%
    {0.5\Cvs}{0.5\Cvs}%
    {\normalfont\LARGE\headfont\raggedright}}
\makeatother
\makeatletter
  \renewcommand{\subsubsection}{%
    \@startsection{subsubsection}{3}{\z@}%
    {0.4\Cvs}{0.4\Cvs}%
    {\normalfont\Large\headfont\raggedright}}
\makeatother
\makeatletter
\renewenvironment{proof}[1][\proofname]{\par
  \pushQED{\qed}%
  \normalfont \topsep6\p@\@plus6\p@\relax
  \trivlist
  \item\relax
  {
  #1\@addpunct{.}}\hspace\labelsep\ignorespaces
}{%
  \popQED\endtrivlist\@endpefalse
}
\makeatother
\renewcommand{\proofname}{\textbf{証明}}
\usepackage{tikz,graphics}
\usepackage[dvipdfmx]{hyperref}
\usepackage{pxjahyper}
\hypersetup{
 setpagesize=false,
 bookmarks=true,
 bookmarksdepth=tocdepth,
 bookmarksnumbered=true,
 colorlinks=false,
 pdftitle={},
 pdfsubject={},
 pdfauthor={},
 pdfkeywords={}}
\begin{document}
\subsection{指示関数}
\subsubsection{指示関数}
\begin{dfn}
集合$A$が与えられたとき、$A'\in \mathfrak{P}(A)$なる集合$A'$を用いて次式のようになる関数$\chi_{A'}$が定義される。このような関数$\chi_{A'}$をその集合$A$におけるその集合$A'$の指示関数、定義関数などという。
\begin{align*}
\chi_{A'}:A \rightarrow \left\{ 0,1 \right\};a \mapsto \left\{ \begin{matrix}
1 & {\mathrm{if}} & a \in A' \\
0 & {\mathrm{if}} & a \in A \setminus A' \\
\end{matrix} \right.\ 
\end{align*}
\end{dfn}
\begin{thm}\label{1.2.5.1a}
$\forall a \in A$に対し、次式たちが成り立つ。
  \begin{align*}
  \chi_{A}(a) = 1, \ \ \chi_{\emptyset} = 0
  \end{align*}
\end{thm}
\begin{proof}
$\forall a \in A$に対し、定義より次のようになる。
  \begin{align*}
  \chi_{A}(a) &= \left\{ \begin{matrix}
  1 & {\mathrm{if}} & a \in A \\
  0 & {\mathrm{if}} & a \in \emptyset \\
  \end{matrix} \right.\  = 1, \\
  \chi_{\emptyset}(a) &= \left\{ \begin{matrix}
  1 & {\mathrm{if}} & a \in \emptyset \\
  0 & {\mathrm{if}} & a \in A \\
  \end{matrix} \right.\  = 0
  \end{align*}
\end{proof}
\begin{thm}\label{1.2.5.1b}
$\forall A',B'\in \mathfrak{P}(A)$に対し、$A' \neq B' $が成り立つなら、$\chi_{A'} \neq \chi_{B'}$が成り立つ。
\end{thm}
\begin{proof}
$\forall A',B'\in \mathfrak{P}(A)$に対し、$A' \neq B'$が成り立つとする。$a \in B' \setminus A'$なる元$a$が存在するとき、これを$a'$とおくと、$\chi_{A'}\left( a' \right) = 0 $かつ$\chi_{B'}\left( a' \right) = 1 $が成り立ち、したがって、$\chi_{A'}\left( a' \right) \neq \chi_{B'}\left( a' \right) $が成り立つので、$\chi_{A'} \neq \chi_{B'} $が成り立つ。\par 
$a \in A' \setminus B'$なる元$a$が存在するときも同様にして示される。
\end{proof}
\begin{thm}\label{1.2.5.1c}
$\forall\chi,\psi \in \mathfrak{F}\left( A,\left\{ 0,1 \right\} \right)$に対し、$\chi \neq \psi $が成り立つなら、$V\left( \chi^{- 1}|\left\{ 1 \right\} \right) \neq V\left( \psi^{- 1}|\left\{ 1 \right\} \right) $が成り立つ。
\end{thm}
\begin{proof}
  $\forall\chi,\psi \in \mathfrak{F}\left( A,\left\{ 0,1 \right\} \right)$に対し、$V\left( \chi^{- 1}|\left\{ 1 \right\} \right)=V\left( \psi^{- 1}|\left\{ 1 \right\} \right)$が成り立つとする。このとき、$\forall a\in A$に対し、$a \in V\left( \chi^{- 1}|\left\{ 1 \right\} \right) $が成り立つならそのときに限り、$1 \in V\left( \chi|\left\{ a \right\} \right) $が成り立つ。これが成り立つならそのときに限り、その対応$\chi$は写像であるので、$\chi(a) = 1 $が成り立つ。同様にして、$a \in V\left( \psi^{- 1}|\left\{ 1 \right\} \right) $が成り立つならそのときに限り、$\psi(a) = 1$次式が成り立つことが示される。\par
  ここで、外延性の公理より$\forall a \in A$に対し、次式が成り立つ。
  \begin{align*}
  \chi(a) = 1 \Leftrightarrow a \in V\left( \chi^{- 1}|\left\{ 1 \right\} \right) \Leftrightarrow a \in V\left( \psi^{- 1}|\left\{ 1 \right\} \right) \Leftrightarrow \psi(a) = 1
  \end{align*}
  これにより、次式が成り立つ。
  \begin{align*}
  \left( \chi(a) = 1 \land \psi(a) = 1 \right) \vee \left( \chi(a) \neq 1 \land \psi(a) \neq 1 \right)
  \end{align*}
  ここで、$\chi,\psi \in \mathfrak{F}\left( A,\left\{ 0,1 \right\} \right)$が成り立つことに注意すれば、次式が成り立つ。
  \begin{align*}
  \left( \chi(a) = \psi(a) \land \chi(a) = 0 \right) \vee \left( \chi(a) = \psi(a) \land \chi(a) = 1 \right)
  \end{align*}
  したがって、次式のようになり、
  \begin{align*}
  \chi(a) = \psi(a) \land \left( \chi(a) = 0 \vee \chi(a) = 1 \right)
  \end{align*}
  $\chi \in \mathfrak{F}\left( A,\left\{ 0,1 \right\} \right)$が成り立つことに注意すれば、論理式$\chi(a) = 0 \vee \chi(a) = 1$は常に真となるので、$\chi(a) = \psi(a)$が成り立つ。したがって、$\chi = \psi$が得られ対偶律より、$\chi \neq \psi $が成り立つなら、$V\left( \chi^{- 1}|\left\{ 1 \right\} \right) \neq V\left( \psi^{- 1}|\left\{ 1 \right\} \right)\in \mathfrak{P}(A) $が成り立つ。
\end{proof}
\begin{thm}\label{1.2.5.1d}
次式のような写像$\varPhi$は全単射となる。
\begin{align*}
  \varPhi:\mathfrak{P}(A)\mathfrak{\rightarrow F}\left( A,\left\{ 0,1 \right\} \right)
\end{align*}
\end{thm}
\begin{proof}
次式のような写像$\varPhi$が与えられたとする。
\begin{align*}
\varPhi:\mathfrak{P}(A) \rightarrow \mathfrak{F}\left( A,\left\{ 0,1 \right\} \right)
\end{align*}
このとき、$\forall A',B'\in \mathfrak{P}(A)$に対し、$A' \neq B' $が成り立つなら、定理\ref{1.2.5.1c}より$\chi_{A'} \neq \chi_{B'} $が成り立つので、その写像$\varPhi$は単射である。さらに、$\chi \in \mathfrak{F}\left( A,\left\{ 0,1 \right\} \right) \setminus V(\varPhi)$なる写像$\chi$が存在すると仮定すると、集合$V\left( \chi^{- 1}|\left\{ 1 \right\} \right)$が与えられたとき、$\forall a \in A$に対し、$a \in V\left( \chi^{- 1}|\left\{ 1 \right\} \right)$が成り立つならそのときに限り、$1 \in V\left( \chi|\left\{ a \right\} \right) $が成り立つのであった。ここで、その対応$\chi$は写像なので、これが成り立つならそのときに限り、$\chi(a) = 1 $が成り立つ。さらに、$a \notin V\left( \chi^{- 1}|\left\{ 1 \right\} \right)$のとき、$\chi(a) \neq 1$が成り立ち、$\chi \in \mathfrak{F}\left( A,\left\{ 0,1 \right\} \right)$が成り立つことに注意すれば、$\chi(a) = 0$が成り立つ。したがって、次のようなその集合$A$におけるその集合$V\left( \chi^{- 1}|\left\{ 1 \right\} \right)$の指示関数が与えられると、
\begin{align*}
\chi_{V\left( \chi^{- 1}|\left\{ 1 \right\} \right)}:A \rightarrow \left\{ 0,1 \right\};a \mapsto \left\{ \begin{matrix}
1 & {\mathrm{if}} & a \in V\left( \chi^{- 1}|\left\{ 1 \right\} \right) \\
0 & {\mathrm{if}} & a \in A \setminus V\left( \chi^{- 1}|\left\{ 1 \right\} \right) \\
\end{matrix} \right.\ 
\end{align*}
したがって、次式が成り立つ。
\begin{align*}
\chi_{V\left( \chi^{- 1}|\left\{ 1 \right\} \right)}:A \rightarrow \left\{ 0,1 \right\};a \mapsto \left\{ \begin{matrix}
1 & {\mathrm{if}} & \chi(a) = 1 \\
0 & {\mathrm{if}} & \chi(a) = 0 \\
\end{matrix} \right.\  = \left\{ \begin{matrix}
\chi(a) & {\mathrm{if}} & \chi(a) = 1 \\
\chi(a) & {\mathrm{if}} & \chi(a) = 0 \\
\end{matrix} \right.\  = \chi(a)
\end{align*}
これにより、$\varPhi\left( V\left( \chi^{- 1}|\left\{ 1 \right\} \right) \right) = \chi_{V\left( \chi^{- 1}|\left\{ 1 \right\} \right)} = \chi$が成り立つが、これは$\chi \notin V(\varPhi)$に矛盾する。\par
したがって、$\mathfrak{F}\left( A,\left\{ 0,1 \right\} \right) \setminus V(\varPhi) = \emptyset$が成り立ち$V(\varPhi) = \mathfrak{F}\left( A,\left\{ 0,1 \right\} \right)$が成り立つので、その写像$\varPhi$は全射である。以上より、その写像$\varPhi$は全単射となる。
\end{proof}
\begin{thm}\label{1.2.5.2}
集合$A$が与えられたとき、$A',B'\in \mathfrak{P}(A)$なる集合たち$A'$、$B'$について、$\forall a \in A$に対し、次式たちが成り立つ。
\begin{align*}
\chi_{A' \cap B'}(a) &= \chi_{A'}(a)\chi_{B'}(a)\\
\chi_{A' \cup B'}(a) &= \chi_{A'}(a) + \chi_{B'}(a) - \chi_{A'}(a)\chi_{B'}(a)\\
\chi_{A' \setminus B'}(a) &= \chi_{A'}(a)\left( 1 - \chi_{B'}(a) \right)
\end{align*}
\end{thm}
\begin{proof}
愚直に場合分けをして計算すればよい。\footnote{
集合$A$が与えられたとき、$A',B'\in \mathfrak{P}(A)$なる集合たち$A'$、$B'$について、$\forall a \in A$に対し、次のようになる。
\begin{align*}
\chi_{A' \cap B'}(a) &= \left\{ \begin{matrix}
1 & {\mathrm{if}} & a \in A' \cap B' \\
0 & {\mathrm{if}} & a \in A \setminus A' \cup A \setminus B' \\
\end{matrix} \right.\ \\
&= \left\{ \begin{matrix}
1 & {\mathrm{if}} & a \in A' \cap B' \\
0 & {\mathrm{if}} & a \in A \setminus A' \vee a \in A \setminus B' \\
\end{matrix} \right.\ \\
&= \left\{ \begin{matrix}
1 \cdot 1 & {\mathrm{if}} & a \in A' \land a \in B' \\
1 \cdot 0 & {\mathrm{if}} & a \in A' \land a \in A \setminus B' \\
0 \cdot 1 & {\mathrm{if}} & a \in A \setminus A' \land a \in B' \\
0 \cdot 0 & {\mathrm{if}} & a \in A \setminus A' \land a \in A \setminus B' \\
\end{matrix} \right.\ \\
&= \left\{ \begin{matrix}
\chi_{A'}(a)\chi_{B'}(a) & {\mathrm{if}} & a \in A' \land a \in B' \\
\chi_{A'}(a)\chi_{B'}(a) & {\mathrm{if}} & a \in A' \land a \in A \setminus B' \\
\chi_{A'}(a)\chi_{B'}(a) & {\mathrm{if}} & a \in A \setminus A' \land a \in B' \\
\chi_{A'}(a)\chi_{B'}(a) & {\mathrm{if}} & a \in A \setminus A' \land a \in A \setminus B' \\
\end{matrix} \right.\ \\
&= \chi_{A'}(a)\chi_{B'}(a) \\
\chi_{A' \cup B'}(a) &= \left\{ \begin{matrix}
1 & {\mathrm{if}} & a \in A' \cup B' \\
0 & {\mathrm{if}} & a \in A \setminus A' \cap A \setminus B' \\
\end{matrix} \right.\ \\
&= \left\{ \begin{matrix}
1 & {\mathrm{if}} & a \in A' \vee a \in B' \\
0 & {\mathrm{if}} & a \in A \setminus A' \land a \in A \setminus B' \\
\end{matrix} \right.\ \\
&= \left\{ \begin{matrix}
1 + 1 - 1 & {\mathrm{if}} & a \in A' \land a \in B' \\
1 + 0 - 0 & {\mathrm{if}} & a \in A' \land a \in A \setminus B' \\
0 + 1 - 0 & {\mathrm{if}} & a \in A \setminus A' \land a \in B' \\
0 + 0 - 0 & {\mathrm{if}} & a \in A \setminus A' \land a \in A \setminus B' \\
\end{matrix} \right.\ \\
&= \left\{ \begin{matrix}
\chi_{A'}(a) + \chi_{B'}(a) - \chi_{A' \cap B'}(a) & {\mathrm{if}} & a \in A' \land a \in B' \\
\chi_{A'}(a) + \chi_{B'}(a) - \chi_{A' \cap B'}(a) & {\mathrm{if}} & a \in A' \land a \in A \setminus B' \\
\chi_{A'}(a) + \chi_{B'}(a) - \chi_{A' \cap B'}(a) & {\mathrm{if}} & a \in A \setminus A' \land a \in B' \\
\chi_{A'}(a) + \chi_{B'}(a) - \chi_{A' \cap B'}(a) & {\mathrm{if}} & a \in A \setminus A' \land a \in A \setminus B' \\
\end{matrix} \right.\ \\
&= \chi_{A'}(a) + \chi_{B'}(a) - \chi_{A' \cap B'}(a)\\
&= \chi_{A'}(a) + \chi_{B'}(a) - \chi_{A'}(a)\chi_{B'}(a)\\
\chi_{A' \setminus B'}(a) &= \left\{ \begin{matrix}
1 & {\mathrm{if}} & a \in A' \setminus B' \\
0 & {\mathrm{if}} & a \in A \setminus \left( A' \setminus B' \right) \\
\end{matrix} \right.\ \\
&= \left\{ \begin{matrix}
1 & {\mathrm{if}} & a \in A' \cap A \setminus B' \\
0 & {\mathrm{if}} & a \in A \setminus A' \cup \left( A \cap B' \right) \\
\end{matrix} \right.\ \\
&= \left\{ \begin{matrix}
1 & {\mathrm{if}} & a \in A' \land a \in A \setminus B' \\
0 & {\mathrm{if}} & a \in A \setminus A' \vee a \in B' \\
\end{matrix} \right.\ \\
&= \left\{ \begin{matrix}
1 \cdot (1 - 0) & {\mathrm{if}} & a \in A' \land a \in A \setminus B' \\
0 \cdot (1 - 0) & {\mathrm{if}} & a \in A \setminus A' \land a \in A \setminus B' \\
1 \cdot (1 - 1) & {\mathrm{if}} & a \in A' \land a \in B' \\
0 \cdot (1 - 1) & {\mathrm{if}} & a \in A \setminus A' \land a \in B' \\
\end{matrix} \right.\ \\
&= \left\{ \begin{matrix}
\chi_{A'}(a)\left( 1 - \chi_{B'}(a) \right) & {\mathrm{if}} & a \in A' \land a \in A \setminus B' \\
\chi_{A'}(a)\left( 1 - \chi_{B'}(a) \right) & {\mathrm{if}} & a \in A \setminus A' \land a \in A \setminus B' \\
\chi_{A'}(a)\left( 1 - \chi_{B'}(a) \right) & {\mathrm{if}} & a \in A' \land a \in B' \\
\chi_{A'}(a)\left( 1 - \chi_{B'}(a) \right) & {\mathrm{if}} & a \in A \setminus A' \land a \in B' \\
\end{matrix} \right.\ \\
&= \chi_{A'}(a)\left( 1 - \chi_{B'}(a) \right)
\end{align*}
よって、次式たちが成り立つ。
\begin{align*}
\chi_{A' \cap B'} &= \chi_{A'}\chi_{B'}\\
\chi_{A' \cup B'} &= \chi_{A'} + \chi_{B'} - \chi_{A'}\chi_{B'}\\
\chi_{A' \setminus B'} &= \chi_{A'}\left( 1 - \chi_{B'} \right)
\end{align*}
}
\end{proof}
\begin{thebibliography}{50}
  \bibitem{1}
    松坂和夫, 集合・位相入門, 岩波書店, 1968. 新装版第2刷 p39-60 ISBM978-4-00-029871-1
  \end{thebibliography}
\end{document}
  
