\documentclass[dvipdfmx]{jsarticle}
\setcounter{section}{1}
\setcounter{subsection}{0}
\usepackage{amsmath,amsfonts,amssymb,array,comment,mathtools,url,docmute}
\usepackage{longtable,booktabs,dcolumn,tabularx,mathtools,multirow,colortbl,xcolor}
\usepackage[dvipdfmx]{graphics}
\usepackage{bmpsize}
\usepackage{amsthm}
\usepackage{enumitem}
\setlistdepth{20}
\renewlist{itemize}{itemize}{20}
\setlist[itemize]{label=•}
\renewlist{enumerate}{enumerate}{20}
\setlist[enumerate]{label=\arabic*.}
\setcounter{MaxMatrixCols}{20}
\setcounter{tocdepth}{3}
\newcommand{\rotin}{\text{\rotatebox[origin=c]{90}{$\in $}}}
\renewcommand{\thesection}{第\arabic{section}部}
\renewcommand{\thesubsection}{\arabic{section}.\arabic{subsection}}
\renewcommand{\thesubsubsection}{\arabic{section}.\arabic{subsection}.\arabic{subsubsection}}
\everymath{\displaystyle}
\allowdisplaybreaks[4]
\usepackage{vtable}
\theoremstyle{definition}
\newtheorem{thm}{定理}[subsection]
\newtheorem*{thm*}{定理}
\newtheorem{dfn}{定義}[subsection]
\newtheorem*{dfn*}{定義}
\newtheorem{axs}[dfn]{公理}
\newtheorem*{axs*}{公理}
\renewcommand{\headfont}{\bfseries}
\makeatletter
  \renewcommand{\section}{%
    \@startsection{section}{1}{\z@}%
    {\Cvs}{\Cvs}%
    {\normalfont\huge\headfont\raggedright}}
\makeatother
\makeatletter
  \renewcommand{\subsection}{%
    \@startsection{subsection}{2}{\z@}%
    {0.5\Cvs}{0.5\Cvs}%
    {\normalfont\LARGE\headfont\raggedright}}
\makeatother
\makeatletter
  \renewcommand{\subsubsection}{%
    \@startsection{subsubsection}{3}{\z@}%
    {0.4\Cvs}{0.4\Cvs}%
    {\normalfont\Large\headfont\raggedright}}
\makeatother
\makeatletter
\renewenvironment{proof}[1][\proofname]{\par
  \pushQED{\qed}%
  \normalfont \topsep6\p@\@plus6\p@\relax
  \trivlist
  \item\relax
  {
  #1\@addpunct{.}}\hspace\labelsep\ignorespaces
}{%
  \popQED\endtrivlist\@endpefalse
}
\makeatother
\renewcommand{\proofname}{\textbf{証明}}
\usepackage{tikz,graphics}
\usepackage[dvipdfmx]{hyperref}
\usepackage{pxjahyper}
\hypersetup{
 setpagesize=false,
 bookmarks=true,
 bookmarksdepth=tocdepth,
 bookmarksnumbered=true,
 colorlinks=false,
 pdftitle={},
 pdfsubject={},
 pdfauthor={},
 pdfkeywords={}}
\begin{document}
%\hypertarget{ux5b9fux6570}{%
\subsection{実数}%\label{ux5b9fux6570}}
\subsubsection{群}%\label{ux7fa4-1}}
\begin{axs}[群の公理]
空集合でない集合$G$に対し算法$*:G \times G \rightarrow G;(a,b) \mapsto a*b$が与えられたとする。このとき、次の条件たちを満たす集合$G$と算法$*$を合わせて群といい、集合$G$は算法$*$に対し群をなすといい、$(G,*)$と書く。そのような集合$G$の元の個数が有限なら、その群$(G,*)$は有限群といい、その集合の濃度$\#G$をその群$(G,*)$の位数といい、$o(G,*)$と書く。逆に、その集合$G$の元の個数が無限ならば、その群$(G,*)$は無限群という。単位元$e$のみからなる群$\left( \left\{ e \right\},* \right)$を単位群という\footnote{余談ですが、集合$G$の1つの部分集合を$S$、$n$つの部分集合たちのうち1つを$S_{i}$、これの元の1つを$s_{i}$とおき、写像$f:\prod_{i} S_{i} \rightarrow S;\left( s_{i} \right)_{i} \mapsto f\left( s_{i} \right)_{i}$を考えるとき、紛らわしいことに集合$\left\{ f\left( s_{i} \right)_{i} \middle| \forall i\left\lbrack s_{i} \in S_{i} \right\rbrack \right\}$を$f\left( S_{i} \right)_{i}$と表記することがあります…。例えば、$S_{1}*S_{2}$、$a*S_{1}$などといった感じに。}。
\begin{itemize}
\item
  算法$*$について結合的である、即ち、$\forall a,b,c \in G$に対し、$(a*b)*c = a*(b*c)$が成り立つ。
\item
  $\exists b \in G\forall a \in G$に対し、$a*b = b*a = a$が成り立つ。この元$b$をその群$(G,*)$の単位元という。
\item
  $\forall a \in G\exists b \in G$に対し、$a*b = b*a = e$が成り立つ。この元$b$を$a$の逆元といい、$a^{- 1}$と書く。
\end{itemize}
さらに次の条件も満たす群$(G,*)$を特に可換群、Abel群という。
\begin{itemize}
\item
  算法$*$は可換的である、即ち、$\forall a,b \in G$に対し、$a*b = b*a$が成り立つ。
\end{itemize}
なお、$a*b = b*a$が成り立つような元々$a$、$b$は可換であるという。
\end{axs}
\begin{thm}\label{4.1.1.1}
群$(G,*)$について、その単位元$e$、その集合$G$の任意の元$a$の逆元$a^{- 1}$は一意的に存在する。
\end{thm}\par
これはいずれも背理法によって示される。
\begin{proof}
群$(G,*)$において、$\forall a \in G$に対し、$a*e = e*a = a$なるその集合$G$の元$e$とは異なる、$\forall a \in G$に対し、$a*e' = e'*a = a$なる元$e'$がその集合$G$に存在するとする。このとき、$e*e' = e$かつ$e*e' = e'$が成り立つので、$e = e'$が成り立つこととなり、仮定に矛盾する。よって、$\forall a \in G$に対し、$a*e = e*a = a$が成り立つようなその単位元$e$は一意的に存在する。\par
同様に、$\forall a \in G$に対し、$a*a^{- 1} = a^{- 1}*a = e$なるその集合$G$の元$a^{- 1}$とは異なる$a*b = b*a = e$なる元$b$がその集合$G$に存在するとする。このとき、次のようになり、
\begin{align*}
a^{- 1} &= a^{- 1}*e\\
&= a^{- 1}*(a*b)\\
&= \left( a^{- 1}*a \right)*b\\
&= e*b = b
\end{align*}
仮定に矛盾する。よって、$\forall a \in G$に対し、$a*a^{- 1} = a^{- 1}*a = e$となる元$a^{- 1}$が一意的に存在する。
\end{proof}
\begin{thm}[簡易律]\label{4.1.1.2}
群$(G,*)$において、$\forall a,u,v \in G$に対し、次のことが成り立つ。
\begin{itemize}
\item
  $a*u = a*v$が成り立つなら、$u = v$が成り立つ。
\item
  $u*a = v*a$が成り立つなら、$u = v$が成り立つ。
\end{itemize}
\par
この性質を簡易律という。
\end{thm}
\begin{proof}
群$(G,*)$が与えられたとする。$\forall a,u,v \in G$に対し、$a*u = a*v$が成り立つなら、次式が成り立つかつ、
\begin{align*}
a^{- 1}*(a*u) &= \left( a^{- 1}*a \right)*u\\
&= e*u = u
\end{align*}
次式が成り立つので、
\begin{align*}
a^{- 1}*(a*u) &= a^{- 1}*(a*v)\\
&= \left( a^{- 1}*a \right)*v\\
&= e*v = v
\end{align*}
$u = v$が得られる。\par
同様にして、$u*a = v*a$が成り立つなら、$u = v$が成り立つことが示される。
\end{proof}
\begin{thm}\label{4.1.1.3}
群$(G,*)$について、$\forall a,b \in G$に対し、$(a*b)^{- 1} = b^{- 1}*a^{- 1}$が成り立つ。
\end{thm}
\begin{proof}
群$(G,*)$が与えられたとする。$\forall a,b \in G$に対し、$(a*b)*(a*b)^{- 1} = e$が成り立つかつ、次式が成り立つので、
\begin{align*}
(a*b)*\left( b^{- 1}*a^{- 1} \right) &= \left( a*\left( b*b^{- 1} \right) \right)*a^{- 1}\\
&= (a*e)*a^{- 1}\\
&= a*a^{- 1} = e
\end{align*}
$(a*b)*(a*b)^{- 1} = (a*b)*\left( b^{- 1}*a^{- 1} \right)$が得られ、したがって、簡易律により$(a*b)^{- 1} = b^{- 1}*a^{- 1}$が成り立つ。
\end{proof}
\begin{dfn}
群$(G,*)$をなす集合$G$の元$a$について$m,n \in \mathbb{Z}$に対し次式のように記法を定める。
\begin{align*}
a^{m}*a^{n} &= a^{m + n}\\
\left( a^{m} \right)^{n} &= a^{mn}\\
(a*b)^{n} &= a^{n}*b^{n}\ \mathrm{if}\ a*b = b*a
\end{align*}
\end{dfn}
\begin{thm}\label{4.1.1.4}
群$(G,*)$について、$\forall a \in G\forall n \in \mathbb{N}$に対し、次式たちが成り立つ。
\begin{align*}
a^{0} = e,\ \ a^{1} = a,\ \ a^{n + 1} = a^{n}*a,\ \ e^{- n} = e^{n} = e,\ \ a^{- n} = \left( a^{- 1} \right)^{n}
\end{align*}
\end{thm}
\begin{proof}
群$(G,*)$をなす集合$G$について、$\forall a \in G\forall n \in \mathbb{N}$に対し、次のようになる。
\begin{align*}
a^{0} &= e*a^{0}\\
&= \left( a^{0} \right)^{- 1}*a^{0}*a^{0}\\
&= \left( a^{0} \right)^{- 1}*a^{0 + 0}\\
&= \left( a^{0} \right)^{- 1}*a^{0}\\
&= e\\
a^{1} &= e*a^{1}\\
&= a*a^{- 1}*a^{1}\\
&= a*a^{- 1 + 1}\\
&= a*a^{0}\\
&= a*e = a\\
a^{n + 1} &= a^{n}*a^{1}\\
&= a^{n}*a
\end{align*}\par
また、上記の議論より$e^{- 1} = e^{0} = e^{1} = e$が成り立つ。$n = k$のとき、$e^{k} = e$と仮定しよう。$n = k + 1$のとき、次のようになる。
\begin{align*}
e^{k + 1} &= e^{k}*e^{1}\\
&= e*e\\
&= e
\end{align*}
逆に、$n = k$のとき、$e^{- k} = e$と仮定しよう。$n = k + 1$のとき、次のようになる。
\begin{align*}
e^{- (k + 1)} &= e^{- k - 1}\\
&= e^{- k}*e^{- 1}\\
&= e*e\\
&= e
\end{align*}
以上より数学的帰納法によって$\forall n \in \mathbb{Z}$に対し、次式が成り立つ。
\begin{align*}
e^{n} = e
\end{align*}\par
また、上記の議論により次のようになる。
\begin{align*}
a^{- n} &= a^{- n}*e\\
&= a^{- n}*\left( a^{- 1} \right)^{0}\\
&= a^{- n}*\left( a^{- 1} \right)^{- n + n}\\
&= a^{- n}*\left( a^{- 1} \right)^{- n}*\left( a^{- 1} \right)^{n}\\
&= \left( a*a^{- 1} \right)^{- n}*\left( a^{- 1} \right)^{n}\\
&= e^{- n}*\left( a^{- 1} \right)^{n}\\
&= e*\left( a^{- 1} \right)^{n}\\
&= \left( a^{- 1} \right)^{n}
\end{align*}
\end{proof}
\subsubsection{環}%\label{ux74b0-1}}
\begin{axs}[環の公理]
空集合でない集合$R$に対し2つの算法それぞれ加法$+ :R \times R \rightarrow R;(a,b) \mapsto a + b$、乗法$\cdot :R \times R \rightarrow R;(a,b) \mapsto ab$が与えられたとする。このとき、次の条件たちを満たす集合$R$を環という\footnote{ここから先は心象するのがほぼ不可能な分野になりますので、定義をよく読んでおくことをお勧めします。}。
\begin{itemize}
\item
  集合$R$は加法について可換群$(R, + )$をなす。
\item
  $\forall a,b,c \in R$に対し、$(ab)c = a(bc)$が成り立つ、即ち、乗法について結合的である。
\item
  $\exists e \in R\forall a \in R$に対し、$ae = ea = a$が成り立つ、即ち、乗法について集合$R$の単位元$e$が存在する。
\item
  $\forall a,b,c \in R$に対し、$a(b + c) = ab + ac$かつ$(a + b)c = ac + bc$が成り立つ、即ち、乗法は加法に対して両側から分配的である。
\end{itemize}
さらに、次の条件も満たす環$R$を特に可換環という。
\begin{itemize}
\item
  $\forall a,b \in R$に対し、$ab = ba$が成り立つ、即ち、乗法は可換的である。
\end{itemize}
\end{axs}
\begin{dfn}
可換群$(R, + )$において、その単位元を零元といい$0$と、逆元$a^{- 1}$を$- a$と、$\forall n \in \mathbb{Z}$に対し、元$a^{n}$を$na$と、乗法についての単位元$e$を$1$と書く。
\end{dfn}
\begin{thm}\label{4.1.1.5}
環$R$が与えられたとき、$\forall a \in R$に対し、$a0 = 0a = 0$が成り立つ。
\end{thm}
\begin{proof} 環$R$について、$\forall a \in R$に対し、次のようになるかつ、
\begin{align*}
0 &= a0 - a0\\
&= a(0 + 0) - a0\\
&= a0 + a0 - a0 = a0
\end{align*}
次のようになるので、
\begin{align*}
0 &= 0a - 0a\\
&= (0 + 0)a - 0a\\
&= 0a + 0a - 0a = 0a
\end{align*}
$a0 = 0a = 0$が成り立つ。
\end{proof}
\begin{dfn}
  環$R$について、$0 = 1$が成り立つとき、$\forall a \in R$に対し、$a = 1a = 0a = 0$が成り立ち$R = \left\{ 0 \right\}$が得られる。これを零環という。以下、環の元が2つ以上現れるのであれば、その環は零環でないので、断りがない場合、そうする。
\end{dfn}
\begin{dfn}
  環$R$について、$\exists a,b \in R$に対し、$a \neq 0$かつ$b \neq 0$が成り立つかつ、$ab = 0$が成り立つなら、それらの元々$a$、$b$をそれぞれ左零因子、右零因子といい、あわせて零因子という。これをもたない可換環を、即ち、その環$R$が可換環で、$\forall a,b \in R$に対し、$a \neq 0$かつ$b \neq 0$が成り立つなら、$ab \neq 0$が成り立つような環を整域という。
\end{dfn}
\begin{dfn}\label{体の定義}
  環$R$について、$\exists a,b \in R$に対し、$ab = 1$が成り立つなら、その元$a$を環$R$の可逆元、単元といい、その元$b$を逆元といい、後に示すように$a^{- 1}$、$\frac{1}{a}$などと書くことができる。以下、その元$a^{-1}$はその群$\left(R,+\right)$における逆元ではなくその可逆元$a$の積における逆元を意味するものとする。これにより、可逆元からなる集合は乗法について群をなし、$0$以外の元全てが可逆元であるような環を斜体といい、乗法について可換的な斜体を体といい、可換的でない斜体を非可換体という。
\end{dfn}\par
斜体を体、体を可換体というときもある。
\begin{thm}\label{4.1.1.6}
  環$R$の可逆元について、次のことが成り立つ。
  \begin{itemize}
  \item
    その環$R$が零環でなくその環$R$の元$a$が可逆元なら、これは$0$でない。
  \item 
    その環$R$の元$a$が可逆元なら、一意的に逆元${a}^{-1}$が定まる。
  \item
    その環$R$が斜体であるなら、零因子をもたない。
  \end{itemize}
\end{thm}
\begin{proof}
  環$R$について、$a \in R$が成り立ちその環$R$が零環でなくその元$a$が可逆元であり${a}^{-1} \in R$なる元${a}^{-1}$を$a$の逆元とする。$a = 0$が成り立つなら、$a{a}^{-1} = 0{a}^{-1} = 0 \neq 1$が成り立つので、可逆元の定義に矛盾する。よって$a \neq 0$が成り立つ。\par
  また、その環$R$の元$a$が可逆元でその元$a^{- 1}$でないその元$a$の逆元$b$が与えられたとするとき、次式が成り立つので、
  \begin{align*}
  {a}^{-1} &= {a}^{-1}1\\
  &= {a}^{-1}(ab)\\
  &= \left( {a}^{-1}a \right)b\\
  &= 1b = b
  \end{align*}
  仮定に矛盾する。よって、一意的に逆元${a}^{-1}$が定まる。\par
  環$R$が斜体であるなら、$0$以外の元全てが可逆元であるので、$\forall a,b \in R$に対し、$a \neq 0$かつ$b \neq 0$が成り立つなら、${a}^{-1},{b}^{-1} \in R$が成り立ち、したがって、次のようになる。
  \begin{align*}
  ab{b}^{-1}{a}^{-1} &= a1{a}^{-1}\\
  &= a{a}^{-1} = 1\\
  {b}^{-1}{a}^{-1}ab &= {b}^{-1}1b\\
  &= {b}^{-1}b = 1
  \end{align*}
  したがって、その元$ab$は可逆元であることになるので、$ab \neq 0$が成り立つ。ゆえに、その環$R$は零因子をもたない。
\end{proof}
\begin{thm}\label{4.1.1.7}
  環$R$の性質として、次のことが成り立つ。
  \begin{itemize}
    \item $\forall a\in R$に対し、$-(-a) =a$が成り立つ。
    \item $\forall a,b\in R$に対し、$a(-b)=(-a)b=-ab$が成り立つ。
    \item $\forall a,b\in R$に対し、$(-a)(-b)=ab$が成り立つ。
    \item $\forall a,b\in R$に対し、$a=0$または$b=0$が成り立つなら、$ab=0$が成り立つ。
    \item $\forall a,b\in R$に対し、$a=0$かつ$b=0$が成り立つなら、$a^2 +b^2 =0$が成り立つ。
    \item $\forall a\in R$に対し、その元$a$が可逆元なら、その元$-a$も可逆元で$(-a)^{-1} =-a^{-1}$が成り立つ。
    \item $\forall a\in R$に対し、それらの元々$a$、$b$が可逆元なら、その元$ab$も可逆元で$(ab)^{-1} =b^{-1} a^{-1}$が成り立つ。
  \end{itemize}
\end{thm}
\begin{proof} 環$R$が与えられたとき、$\forall a\in R$に対し、次のようになる。
  \begin{align*}
  - ( - a) &= 0 - ( - a)\\
  &= a - a - ( - a)\\
  &= a + ( - a) - ( - a)\\
  &= a + 0 = a
  \end{align*}\par
  $\forall a,b\in R$に対し、次のようになる。
  \begin{align*}
  a( - b) &= 0 + a( - b)\\
  &= - ab + ab + a( - b)\\
  &= - ab + a\left( b + ( - b) \right)\\
  &= - ab + a0\\
  &= - ab + 0\\
  &= - ab
  \end{align*}\par
  $\forall a,b\in R$に対し、次のようになる。
  \begin{align*}
  ( - a)b &= ( - a)b + 0\\
  &= ( - a)b + ab - ab\\
  &= \left( ( - a) + a \right)b - ab\\
  &= 0b - ab\\
  &= a - ab\\
  &= - ab
  \end{align*}\par
  $\forall a,b\in R$に対し、次のようになる。
  \begin{align*}
  ( - a)( - b) &= ( - a)( - b) + 0\\
  &= ( - a)( - b) + ( - a)b - ( - a)b\\
  &= ( - a)\left( ( - b) + b \right) - ( - ab)\\
  &= ( - a)0 + ab\\
  &= 0 + ab\\
  &= ab
  \end{align*}
  $\forall a,b\in R$に対し、$a=0$または$b=0$が成り立つなら、次のようになるので、
  \begin{align*}
  a = 0 \vee b = 0 &\Rightarrow ab = 0 \vee ab = 0\\
  &\Leftrightarrow ab = 0
  \end{align*}
  $ab=0$が成り立つ。\par
  $\forall a,b\in R$に対し、$a=0$かつ$b=0$が成り立つなら、次のようになるので、
  \begin{align*}
  a = 0 \land b = 0 &\Rightarrow a^{2} = 0 \land b^{2} = 0\\
  &\Rightarrow a^{2} + b^{2} = 0
  \end{align*}
  $a^2 +b^2 =0$が成り立つ。\par
  $\forall a\in R$に対し、その元$a$が可逆元なら、次のようになるので、
  \begin{align*}
  \left( -a^{-1} \right) \left(-a\right) &= a^{-1} a =1\\
  \left( -a\right) \left(-a^{-1} \right) &= aa^{-1} =1
  \end{align*}
  その元$-a$も可逆元で$(-a)^{-1} =-a^{-1}$が成り立つ。\par
  $\forall a\in R$に対し、それらの元々$a$、$b$が可逆元なら、次のようになるので、
  \begin{align*}
  \left( b^{-1} a^{-1} \right) \left(ab\right) &= b^{-1} \left( a^{-1} a\right) b \\
  &= b^{-1} 1 b =b^{-1} b =1\\
  \left(ab\right)\left( b^{-1} a^{-1} \right) &= a\left( bb^{-1} \right) a^{-1} \\
  &=a1a^{-1} =aa^{-1} =1
  \end{align*}
  その元$ab$も可逆元で$(ab)^{-1} =b^{-1} a^{-1}$が成り立つ。
\end{proof}
%\hypertarget{ux9806ux5e8fux4f53}{%
\subsubsection{順序体}%\label{ux9806ux5e8fux4f53}}
\begin{axs}[順序体の公理]
$\forall a,b,c,d \in O$に対し関係$a \leq b$を次の条件たちを満たすように定める。なお、$a \leq b$を$b \geq a$とも書く。上から1つ目から3つ目までの条件たちを満たす集合$O$を順序集合、順序集合であることに加えて上から4つ目の条件を満たす集合を全順序集合、集合$O$が体で次の条件を全て満たす集合を順序体という。
\begin{itemize}
\item
  $\forall a \in O$に対し、$a \leq a$が成り立つ。これを反射律という。
\item
  $\forall a,b \in O$に対し、$a \leq b$かつ$b \geq a$が成り立つなら、$a = b$が成り立つ。これを反対称律という。
\item
  $\forall a,b,c \in O$に対し、$a \leq b$かつ$b \leq c$が成り立つなら、$a \leq c$が成り立つ。これを推移律という。
\item
  $\forall a,b \in O$に対し、$a \leq b$または$b \leq a$が成り立つ。これを全順序性という。
\item
  $\forall a,b,c \in O$に対し、$a \leq b$が成り立つなら、$a + c \leq b + c$が成り立つ。
\item
  $\forall a,b \in O$に対し、$0 \leq a$かつ$0 \leq b$が成り立つなら、$0 \leq ab$が成り立つ。
\end{itemize}
\end{axs}
\begin{dfn}
$a \leq b$かつ$a \neq b$が成り立つことを$a < b$と書く。なお、$a < b$を$b > a$とも書く。
\end{dfn}
\begin{dfn}
$0 < a$、$a < 0$が成り立つとき、それぞれ$a$は正である、負であるという。
\end{dfn}
\begin{thm}\label{4.1.1.8} 順序体$O$が与えられたとき、次のことが成り立つ。
\begin{itemize}
\item
  $\forall a,b \in O$に対し、$a \leq b$が成り立つならそのときに限り、$a < b$または$a = b$が成り立つ。
\item
  $\forall a,b \in O$に対し、$a < b$または$a = b$または$b < a$が成り立つ。
\item
  $\forall a,b \in O$に対し、$a < b$かつ$a = b$が成り立つことはない。
\item
  $\forall a,b \in O$に対し、$a < b$かつ$b < a$が成り立つことはない。
\item
  $\forall a \in O$に対し、$0 \leq a$が成り立つならそのときに限り、$- a \leq 0$が成り立つ。
\item
  $\forall a \in O$に対し、$0 \leq a^{2} = ( - a)^{2}$が成り立つ。
\item
  $0 \neq 1$が成り立つなら、$0 < 1$が成り立つ。
\item
  $\forall a,b \in O$に対し、$a \leq b$が成り立つならそのときに限り、$0 \leq b - a$が成り立つ。
\item
  $\forall a,b \in O$に対し、$a^{2} + b^{2} = 0$が成り立つならそのときに限り、$a = 0$かつ$b = 0$が成り立つ。
\item
  $\forall a,b \in O$に対し、$ab = 0$が成り立つならそのときに限り、$a = 0$または$b = 0$が成り立つ。
\item
  $\forall a,b \in O$に対し、$a \leq b$が成り立つならそのときに限り、$- b \leq - a$が成り立つ。
\item
  $\forall a,b,c \in O$に対し、$a \leq b$かつ$c \leq 0$が成り立つなら、$bc \leq ac$が成り立つ。
\item
  $\forall a \in O$に対し、$0 < a$が成り立つなら、$0 < a^{- 1}$が成り立つ。
\item
  $\forall a,b,c,d \in O$に対し、$a \leq b$かつ$c \leq d$が成り立つなら、$a + c \leq b + d$が成り立つ。
\item
  $\forall a,b,c,d \in O$に対し、$a \leq b$かつ$c < d$が成り立つなら、$a + c < b + d$が成り立つ。
\item
  $\forall a,b \in O$に対し、$a < b$が成り立つなら、$a < c < b$なる元$c$がその順序体$O$に存在する。
\item
  $\forall a,b \in O$に対し、$\forall c \in O$に対し、$0 < c$が成り立つかつ、$a \leq b + c$が成り立つなら、$a \leq b$が成り立つ。
\item
  $\forall a \in O$に対し、$0 \leq a$が成り立つかつ、$\forall b \in O$に対し、$a < b$が成り立つなら、$a = 0$が成り立つ。
\end{itemize}
最後の性質は順序体の任意の元の近くにいくらでも別の元が存在することを示しており、この性質を稠密性といい集合$O$は稠密順序集合であるという。
\end{thm}
\begin{proof}
順序体$O$が与えられたとき、$\forall a,b \in O$に対し、次のようになるので、
\begin{align*}
a \leq b &\Leftrightarrow a \leq b \land \top\\ 
&\Leftrightarrow (a \leq b \vee a \leq b) \land (a \leq b \vee b \leq a)\\ 
&\Leftrightarrow a \leq b \vee (a \leq b \land b \leq a)\\ 
&\Leftrightarrow a \leq b \vee a = b\\ 
&\Leftrightarrow (a \leq b \vee a = b) \land \top\\ 
&\Leftrightarrow (a \leq b \vee a = b) \land (a \neq b \vee a = b)\\ 
&\Leftrightarrow (a \leq b \land a \neq b) \vee a = b\\ 
&\Leftrightarrow a < b \vee a = b
\end{align*}
$a \leq b$が成り立つならそのときに限り、$a < b$または$a = b$が成り立つ。\par
$\forall a,b \in O$に対し、次のようになるので、
\begin{align*}
a < b \vee a = b \vee b < a &\Leftrightarrow (a \leq b \land a \neq b) \vee a = b \vee (b \leq a \land a \neq b)\\ 
&\Leftrightarrow (a \leq b \vee a = b \vee b \leq a) \land (a \leq b \vee a = b \vee a \neq b) \\
&\quad \land (b \leq a \vee a = b \vee a \neq b) \land (a = b \vee a \neq b \vee a \neq b)\\ 
&\Leftrightarrow (a = b \vee \top) \land (a \leq b \vee \top) \land (b \leq a \vee \top) \land (a \neq b \vee \top)\\ 
&\Leftrightarrow \top \land \top \land \top \land \top \Leftrightarrow \top
\end{align*}
$a < b$または$a = b$または$b < a$が成り立つ。\par
$\forall a,b \in O$に対し、次のようになるので、
\begin{align*}
a < b \land a = b &\Leftrightarrow a \leq b \land a \neq b \land a = b\\ 
&\Leftrightarrow a \leq b \land \bot \Leftrightarrow \bot
\end{align*}
$a < b$かつ$a = b$が成り立つことはない。\par
$\forall a,b \in O$に対し、次のようになるので、
\begin{align*}
a < b \land b < a &\Leftrightarrow a \leq b \land a \neq b \land b \leq a \land a \neq b\\ 
&\Leftrightarrow a \neq b \land (a \leq b \land b \leq a)\\ 
&\Leftrightarrow a \neq b \land a = b \Leftrightarrow \bot
\end{align*}
$a < b$かつ$b < a$が成り立つことはない。\par
$\forall a \in O$に対し、次のようになるので、
\begin{align*}
0 \leq a &\Leftrightarrow 0 + ( - a) \leq a + ( - a)\\ 
&\Leftrightarrow - a \leq 0
\end{align*}
$0 \leq a$が成り立つならそのときに限り、$- a \leq 0$が成り立つ。\par
$\forall a \in O$に対し、次のようになるので、
\begin{align*}
\top &\Leftrightarrow a < 0 \vee a = 0 \vee 0 < a\\ 
&\Leftrightarrow (a < 0 \vee a = 0) \vee (0 < a \vee a = 0)\\ 
&\Leftrightarrow a \leq 0 \vee 0 \leq a\\ 
&\Leftrightarrow - ( - a) \leq 0 \vee 0 \leq a\\ 
&\Leftrightarrow 0 \leq - a \vee 0 \leq a\\ 
&\Leftrightarrow 0 \leq ( - a)^{2} \vee 0 \leq a^{2}\\ 
&\Leftrightarrow 0 \leq ( - a)^{2} = a^{2} \vee 0 \leq a^{2} = ( - a)^{2}\\ 
&\Leftrightarrow 0 \leq a^{2} = ( - a)^{2}
\end{align*}
$0 \leq a^{2} = ( - a)^{2}$が成り立つ。\par
$0 \neq 1$が成り立つなら、次のようになるので、
\begin{align*}
\top &\Leftrightarrow \top \land \top\\ 
&\Leftrightarrow 0 \leq 1^{2} = 1 \land 1 \neq 0\\ 
&\Leftrightarrow 0 \leq 1 \land 0 \neq 1\\ 
&\Leftrightarrow 0 < 1
\end{align*}
$0 < 1$が成り立つ。\par
$\forall a,b \in O$に対し、次のようになるので、
\begin{align*}
a \leq b &\Rightarrow a + ( - a) \leq b + ( - a)\\ 
&\Leftrightarrow 0 \leq b - a\\ 
0 \leq b - a &\Rightarrow a \leq b - a + a = b
\end{align*}
$a \leq b$が成り立つならそのときに限り、$0 \leq b - a$が成り立つ。\par
$\forall a,b \in O$に対し、次のようになるので、
\begin{align*}
a^{2} + b^{2} = 0 &\Rightarrow a^{2} = - b^{2}\\ 
&\Leftrightarrow 0 \leq a^{2} = - b^{2}\\ 
&\Leftrightarrow a^{2} = - b^{2} \land 0 \leq b^{2} \leq 0\\ 
&\Leftrightarrow a^{2} = - b^{2} \land \left( b^{2} = 0 \vee 0 < b^{2} \right) \land \left( b^{2} = 0 \vee b^{2} < 0 \right)\\ 
&\Leftrightarrow a^{2} = - b^{2} \land \left( b^{2} = 0 \vee \left( 0 < b^{2} \land 0 < b^{2} \right) \right)\\ 
&\Leftrightarrow a^{2} = - b^{2} \land b^{2} = 0\\ 
&\Leftrightarrow a^{2} = 0 \land b^{2} = 0\\ 
&\Leftrightarrow a = 0 \land b = 0\\ 
a = 0 \land b = 0 &\Rightarrow a^{2} = 0 \land b^{2} = 0\\ 
&\Rightarrow a^{2} + b^{2} = 0 + 0 = 0
\end{align*}
$a^{2} + b^{2} = 0$が成り立つならそのときに限り、$a = 0$かつ$b = 0$が成り立つ。\par
$\forall a,b \in O$に対し、次のようになるので、
\begin{align*}
\neg(ab \neq 0 \vee a = 0 \vee b = 0) &\Leftrightarrow ab = 0 \land a \neq 0 \land b \neq 0\\ 
&\Rightarrow ab = 0 \land ab \neq 0 \Leftrightarrow \bot
\end{align*}
背理法により$ab = 0$が成り立つなら、$a = 0$または$b = 0$が成り立つ。また、$a = 0$または$b = 0$が成り立つなら、$ab = 0$が成り立つので、$ab = 0$が成り立つならそのときに限り、$a = 0$または$b = 0$が成り立つ。\par
$\forall a,b \in O$に対し、次のようになるので、
\begin{align*}
a \leq b &\Leftrightarrow 0 \leq b - a = - a + b = - a - ( - b)\\ 
&\Leftrightarrow - b \leq - a
\end{align*}
$a \leq b$が成り立つならそのときに限り、$- b \leq - a$が成り立つ。\par
$\forall a,b,c \in O$に対し、次のようになるので、
\begin{align*}
a \leq b \land c = - ( - c) \leq 0 &\Leftrightarrow a \leq b \land 0 \leq - c\\ 
&\Rightarrow a( - c) = - ac \leq b( - c) = - bc\\ 
&\Leftrightarrow bc \leq ac
\end{align*}
$a \leq b$かつ$c \leq 0$が成り立つなら、$bc \leq ac$が成り立つ。
$\forall a \in O$に対し、次のようになるので、
\begin{align*}
\neg\left( 0 < a \Rightarrow 0 < a^{- 1} \right) &\Leftrightarrow \neg\left( \neg 0 < a \vee 0 < a^{- 1} \right)\\ 
&\Leftrightarrow \neg\neg 0 < a \land \neg 0 < a^{- 1}\\ 
&\Leftrightarrow 0 < a \land a^{- 1} \leq 0\\ 
&\Leftrightarrow 0 < a \land \left( a^{- 1} = 0 \vee a^{- 1} < 0 \right)\\ 
&\Leftrightarrow \left( 0 < a \land a^{- 1} = 0 \right) \vee \left( 0 < a \land a^{- 1} < 0 \right)\\ 
&\Leftrightarrow \left( 0 < a \land aaa^{- 1} = a1 = a = aa0 = 0 \right) \vee 0 < \left( - a^{- 1} \right)a = - a^{- 1}a = - 1\\ 
&\Leftrightarrow (0 < a \land a = 0) \vee 1 < 0\\ 
&\Leftrightarrow \bot \vee \bot \Leftrightarrow \bot
\end{align*}
$0 < a$が成り立つなら、$0 < a^{- 1}$が成り立つ。\par
$\forall a,b,c,d \in O$に対し、次のようになるので、
\begin{align*}
a \leq b \land c \leq d &\Leftrightarrow a + c \leq b + c \land b + c \leq b + d\\ 
&\Leftrightarrow a + c \leq b + c \leq b + d\\ 
&\Rightarrow a + c \leq b + d
\end{align*}
$a \leq b$かつ$c \leq d$が成り立つなら、$a + c \leq b + d$が成り立つ。\par
$\forall a,b,c,d \in O$に対し、次のようになるので、
\begin{align*}
a \leq b \land c < d &\Leftrightarrow a + c \leq b + c \land b + c < b + d\\ 
&\Leftrightarrow a + c \leq b + c < b + d\\ 
&\Rightarrow a + c < b + d
\end{align*}
$a \leq b$かつ$c < d$が成り立つなら、$a + c < b + d$が成り立つ。\par
$\forall a,b \in O$に対し、$a < b$が成り立つなら、$\frac{1}{2}a < \frac{1}{2}b$が成り立つので、次式が成り立つ。
\begin{align*}
a = \frac{1}{2}a + \frac{1}{2}a < \frac{1}{2}a + \frac{1}{2}b < \frac{1}{2}b + \frac{1}{2}b = b
\end{align*}
ゆえに、$a < c < b$なる元$c$がその順序体$O$に存在する。\par
$\forall a,b \in O$に対し、$\forall c \in O$に対し、$0 < c$が成り立つかつ、$a \leq b + c$が成り立つなら、$a \leq b$が成り立つことの否定は、$\exists a,b \in O$に対し、$\forall c \in O$に対し、$0 < c$が成り立つかつ、$a \leq b + c$が成り立つかつ、$b < a$が成り立つことである。このとき、$b < a$が成り立つならそのときに限り、$0 < a - b$が成り立ち、あるその順序体$O$の元$c$が存在して、$0 < c < a - b$が成り立つ。したがって、次のようになる。
\begin{align*}
&\quad \forall c \in O[ 0 < c \land a \leq b + c] \land \exists c \in O[ 0 < c < a - b]\\ 
&\Rightarrow \forall c \in O[ 0 < c] \land \forall c \in O[ a \leq b + c] \land \exists c \in O[ c < a - b]\\ 
&\Leftrightarrow \forall c \in O[ 0 < c] \land \forall c \in O[ a \leq b + c] \land \exists c \in O[ b + c < a]\\ 
&\Leftrightarrow \forall c \in O[ 0 < c] \land \forall c \in O[ a \leq b + c] \land \exists c \in O\left[ \neg(a \leq b + c) \right]\\ 
&\Leftrightarrow \forall c \in O[ 0 < c] \land \forall c \in O[ a \leq b + c] \land \neg\forall c \in O[ a \leq b + c]\\ 
&\Leftrightarrow \forall c \in O[ 0 < c] \land \bot \Leftrightarrow \bot
\end{align*}
背理法によりしたがって、$\forall a,b \in O$に対し、$\forall c \in O$に対し、$0 < c$が成り立つかつ、$a \leq b + c$が成り立つなら、$a \leq b$が成り立つ。\par
$\exists a \in O$に対し、$0 \leq a \land \forall b \in O[ a < b] \land a \neq 0$が成り立つと仮定すると、$0 < a$が成り立つが、$\exists c \in O[ 0 < a \Rightarrow 0 < c < a]$より$c < a$が成り立つようなその順序体$O$の元$c$が存在することになり、したがって、$\forall b \in O[ a < b]$に矛盾する。よって、$0 \leq a$が成り立つかつ、$\forall b \in O$に対し、$a < b$が成り立つなら、$a = 0$が成り立つ。
\end{proof}
%\hypertarget{ux6700ux5927ux5143ux3068ux6700ux5c0fux5143}{%
\subsubsection{最大元と最小元}%\label{ux6700ux5927ux5143ux3068ux6700ux5c0fux5143}}
\begin{dfn}
順序体$O$の部分集合$A$について、これを順序集合$(A, \leq )$とみたときのその集合$A$の最大元、即ち、$\forall a \in A$に対し、$a \leq m$が成り立つその集合$A$の元$m$をその集合の最大元という。同様にして、これを順序集合$(A, \leq )$とみたときのその集合$A$の最小元、即ち、$\forall a \in A$に対し、$m \leq a$が成り立つその集合$A$の元$m$をその集合の最小元という。\par
順序集合に関する一定理より、その集合$A$の最大元が存在するなら、これは一意的になり、その集合$A$の最小元が存在するなら、これは一意的になるのであった。これにより、その集合$A$の最大元と最小元をそれぞれ$\max A$、$\min A$と書く。しかしながら、これらは必ずしも存在するとは限らない。
\end{dfn}
\begin{thm}\label{4.1.1.9}
順序体$O$の最大元$\max O$、最小元$\min O$はどちらも存在しない。
\end{thm}\par
このことは背理法によって示される。
\begin{proof}
順序体$O$の最大元$\max O$が存在すると仮定する。このとき、$\forall a \in O$に対し、$a \leq \max O$が成り立つが、不等式$0 < 1$の両辺に$\max O$を加えると、$\max O < \max O + 1$が成り立ち明らかに$\max O + 1$は存在しその順序体$O$に属するので、仮定に矛盾する。順序体の最小元$\min O$についても同様に示される。
\end{proof}
%\hypertarget{ux7d76ux5bfeux5024}{%
\subsubsection{絶対値}%\label{ux7d76ux5bfeux5024}}
\begin{dfn}
順序体$O$の任意の元$a$に対して次式のように定められる$|a|$を$a$の絶対値という。
\begin{align*}
|a| = \max\left\{ - a,a \right\}
\end{align*}
\end{dfn}
\begin{thm}\label{4.1.1.10} 順序体$O$での絶対値について次のようになる。
\begin{itemize}
\item
  $\forall a \in O$に対し、$|a| = \left\{ \begin{matrix}
  a & \mathrm{if} & 0 \leq a \\
   - a & \mathrm{if} & a < 0 \\
  \end{matrix} \right.\ $が成り立つ。
\item
  $\forall a \in O$に対し、$| - a| = |a|$が成り立つ。
\item
  $\forall a \in O$に対し、$- |a| \leq a \leq |a|$が成り立つ。
\item
  $\forall a \in O$に対し、$0 \leq |a|$が成り立つ。
\item
  $\forall a \in O$に対し、$|a| = 0$が成り立つならそのときに限り、$a = 0$が成り立つ。
\item
  $\forall a,b \in O$に対し、$|ab| = |a||b|$が成り立つ。
\item
  $\forall a,b \in O$に対し、$|a + b| \leq |a| + |b|$が成り立つ。特に、この式を三角不等式という。
\item
  $\forall a,b \in O$に対し、$|a| - |b| \leq |a - b|$が成り立つ。
\end{itemize}
\end{thm}
\begin{proof} 順序体$O$について、$\forall a \in O$に対し、次のようになる。
\begin{align*}
|a| &= \max\left\{ - a,a \right\}\\ 
&= \left\{ \begin{matrix}
a & \mathrm{if} & - a \leq a \\
 - a & \mathrm{if} & a < - a \\
\end{matrix} \right.\ \\ 
&= \left\{ \begin{matrix}
a & \mathrm{if} & 0 \leq 2a \\
 - a & \mathrm{if} & 2a < 0 \\
\end{matrix} \right.\ \\ 
&= \left\{ \begin{matrix}
a & \mathrm{if} & 0 \leq a \\
 - a & \mathrm{if} & a < 0 \\
\end{matrix} \right.\ 
\end{align*}\par
$\forall a \in O$に対し、次のようになる。
\begin{align*}
|a| &= \left\{ \begin{matrix}
a & \mathrm{if} & 0 \leq a \\
 - a & \mathrm{if} & a < 0 \\
\end{matrix} \right.\ \\ 
&= \left\{ \begin{matrix}
a & \mathrm{if} & - a \leq 0 \\
 - a & \mathrm{if} & 0 < - a \\
\end{matrix} \right.\ \\ 
&= \left\{ \begin{matrix}
a & \mathrm{if} & - a < 0 \\
 - a & \mathrm{if} & 0 \leq - a \\
\end{matrix} \right.\ \\ 
&= \left\{ \begin{matrix}
 - a & \mathrm{if} & 0 \leq - a \\
a & \mathrm{if} & - a < 0 \\
\end{matrix} \right.\  = | - a|
\end{align*}\par
$\forall a \in O$に対し、次のようになるので、
\begin{align*}
|a| = \left\{ \begin{matrix}
a & \mathrm{if} & 0 \leq a \\
 - a & \mathrm{if} & a < 0 \\
\end{matrix} \right.\  &\Leftrightarrow \left\{ \begin{matrix}
|a| = a & \mathrm{if} & 0 \leq a \\
|a| = - a & \mathrm{if} & a < 0 \\
\end{matrix} \right.\ \\ 
&\Leftrightarrow \left\{ \begin{matrix}
|a| = a & \mathrm{if} & 0 \leq a \\
|a| = - a & \mathrm{if} & a \leq 0 \\
\end{matrix} \right.\  \land \left\{ \begin{matrix}
 - |a| = - a \leq 0 & \mathrm{if} & 0 \leq a \\
 - |a| = a \leq 0 & \mathrm{if} & a \leq 0 \\
\end{matrix} \right.\ \\ 
&\Leftrightarrow \left\{ \begin{matrix}
 - |a| = - a \leq 0 \leq |a| = a & \mathrm{if} & 0 \leq a \\
 - |a| = a \leq 0 \leq |a| = - a & \mathrm{if} & a \leq 0 \\
\end{matrix} \right.\ \\ 
&\Leftrightarrow \left\{ \begin{matrix}
 - |a| \leq |a| = a & \mathrm{if} & 0 \leq a \\
 - |a| = a \leq |a| & \mathrm{if} & a < 0 \\
\end{matrix} \right.\ \\ 
&\Leftrightarrow \left\{ \begin{matrix}
 - |a| \leq a \leq |a| & \mathrm{if} & 0 \leq a \\
 - |a| \leq a \leq |a| & \mathrm{if} & a < 0 \\
\end{matrix} \right.\ \\ 
&\Leftrightarrow - |a| \leq a \leq |a|
\end{align*}
$- |a| \leq a \leq |a|$が成り立つ。\par
$\forall a \in O$に対し、次のようになるので、
\begin{align*}
|a| = \left\{ \begin{matrix}
a & \mathrm{if} & 0 \leq a \\
 - a & \mathrm{if} & a < 0 \\
\end{matrix} \right.\  &\Leftrightarrow |a| = \left\{ \begin{matrix}
a & \mathrm{if} & 0 \leq a \\
 - a & \mathrm{if} & 0 \leq - a \\
\end{matrix} \right.\ \\ 
&\Leftrightarrow \left\{ \begin{matrix}
|a| = a & \mathrm{if} & 0 \leq |a| \\
|a| = - a & \mathrm{if} & 0 \leq |a| \\
\end{matrix} \right.\ \\ 
&\Leftrightarrow \left( |a| = a \land 0 \leq |a| \right) \vee \left( |a| = - a \land 0 \leq |a| \right)\\ 
&\Leftrightarrow 0 \leq |a| \land \left( |a| = a \vee |a| = - a \right)\\ 
&\Leftrightarrow 0 \leq |a|
\end{align*}
$0 \leq |a|$が成り立つ。\par
$\forall a \in O$に対し、次のようになるので、
\begin{align*}
|a| = \left\{ \begin{matrix}
a & \mathrm{if} & 0 \leq a \\
 - a & \mathrm{if} & a < 0 \\
\end{matrix} \right.\  = 0 &\Leftrightarrow \left\{ \begin{matrix}
a = 0 & \mathrm{if} & 0 \leq a \\
 - a = 0 & \mathrm{if} & a < 0 \\
\end{matrix} \right.\ \\ 
&\Leftrightarrow \left\{ \begin{matrix}
a = 0 & \mathrm{if} & 0 \leq a \\
a = 0 & \mathrm{if} & a < 0 \\
\end{matrix} \right.\ \\ 
&\Leftrightarrow a = 0
\end{align*}
$|a| = 0$が成り立つならそのときに限り、$a = 0$が成り立つ。\par
$\forall a,b \in O$に対し、次のようになる。
\begin{align*}
|a||b| &= \left\{ \begin{matrix}
ab & \mathrm{if} & 0 \leq a \land 0 \leq b \\
a( - b) & \mathrm{if} & 0 \leq a \land b < 0 \\
( - a)b & \mathrm{if} & a < 0 \land 0 \leq b \\
( - a)( - b) & \mathrm{if} & a < 0 \land b < 0 \\
\end{matrix} \right.\ \\ 
&= \left\{ \begin{matrix}
ab & \mathrm{if} & 0 \leq a \land 0 \leq b \\
 - ab & \mathrm{if} & 0 \leq a \land 0 < - b \\
 - ab & \mathrm{if} & 0 < - a \land 0 \leq b \\
ab & \mathrm{if} & 0 < - a \land 0 < - b \\
\end{matrix} \right.\ \\ 
&= \left\{ \begin{matrix}
ab & \mathrm{if} & 0 \leq ab \\
 - ab & \mathrm{if} & 0 \leq - ab \\
 - ab & \mathrm{if} & 0 < - ab \\
ab & \mathrm{if} & 0 < ab \\
\end{matrix} \right.\ \\ 
&= \left\{ \begin{matrix}
ab & \mathrm{if} & 0 \leq ab \\
 - ab & \mathrm{if} & 0 < - ab \\
\end{matrix} \right.\ \\ 
&= \left\{ \begin{matrix}
ab & \mathrm{if} & 0 \leq ab \\
 - ab & \mathrm{if} & ab < 0 \\
\end{matrix} \right.\  = |ab|
\end{align*}\par
$\forall a,b \in O$に対し、次のようになるので、
\begin{align*}
- |a| \leq a \leq |a| \land - |b| \leq b \leq |b| &\Leftrightarrow - |a| - |b| = - \left( |a| + |b| \right) \leq a + b \leq |a| + |b|\\ 
&\Leftrightarrow - \left( |a| + |b| \right) \leq a + b \leq |a| + |b| \vee - \left( |a| + |b| \right) \leq - (a + b) \leq |a| + |b|\\ 
&\Leftrightarrow a + b \leq |a| + |b| \vee - (a + b) \leq |a| + |b|\\ 
&\Leftrightarrow |a + b| \leq |a| + |b|
\end{align*}
$|a + b| \leq |a| + |b|$が成り立つ。\par
特に、式$|a + b| \leq |a| + |b|$の$a$に$a - b$と書き換えると、次のようになる。
\begin{align*}
\left| (a - b) + b \right| \leq |a - b| + |b| &\Leftrightarrow |a| \leq |a - b| + |b|\\ 
&\Leftrightarrow |a| - |b| \leq |a - b|
\end{align*}
\end{proof}
%\hypertarget{ux4e0aux754cux3068ux4e0bux754c}{%
\subsubsection{上界と下界}%\label{ux4e0aux754cux3068ux4e0bux754c}}
\begin{dfn}
順序体$O$の部分集合$A$について、これを順序集合$(A, \leq )$とみたときのその集合$A$の上界、即ち、$\forall a \in A$に対し、$a \leq u$が成り立つようなその順序体$O$の元$u$をその集合$A$のその順序体$O$における上界といい、これ全体の集合を次式のように$U(A)$とおく。
\begin{align*}
U(A) = \left\{ u \in O \middle| \forall a \in A[ a \leq u] \right\}
\end{align*}
これが空集合でないとき、即ち、$\exists u \in O\forall a \in A$に対し、$a \leq u$が成り立つとき、その集合$A$はその順序体$O$において上に有界であるという。\par
同様に、これを順序集合$(A, \leq )$とみたときのその集合$A$の下界、即ち、$\forall a \in A$に対し、$l \leq a$が成り立つようなその順序体$O$の元$l$をその集合$A$のその順序体$O$における下界といい、これ全体の集合を次式のように$L(A)$とおく。
\begin{align*}
L(A) = \left\{ l \in O \middle| \forall a \in A[ l \leq a] \right\}
\end{align*}
これが空集合でないとき、即ち、$\exists l \in O\forall a \in A$に対し、$l \leq a$が成り立つとき、その集合$A$はその順序体$O$において下に有界であるという。
\end{dfn}
\begin{dfn}
順序体$O$の部分集合$A$について、これを順序集合$(A, \leq )$とみたときのその集合$A$の上限、即ち、その集合$A$のその順序体$O$における上界全体の集合$U(A)$の最小元$\min{U(A)}$が存在するとき、この元をその集合$A$のその順序体$O$における最小上界、上限といい$\sup M$と書く。同様に、これを順序集合$(A, \leq )$とみたときのその集合$A$の下限、即ち、その集合$A$のその順序体$O$における下界全体の集合$L(A)$の最大元$\max{L(A)}$が存在するとき、この元をその集合$A$のその順序体$O$における最大下界、下限といい$\inf M$と書く。
\end{dfn}
\begin{thm}\label{4.1.1.11}
順序体$O$の部分集合$A$について、その順序体$O$の元$m$がその集合$A$のその順序体$O$における上限となるならそのときに限り、$m \in O$なる元$m$について、次のことが成り立つ。
\begin{itemize}
\item
  $\forall a \in A$に対し、$a \leq m$が成り立つ。
\item
  $\forall b \in O$に対し、$b \in U(A)$が成り立つなら、$m \leq b$が成り立つ。
\end{itemize}\par
同様に、その集合$A$の元$m$がその集合$A$のその順序体$O$における下限となるならそのときに限り、$m \in O$なる元$m$について、次のことが成り立つ。
\begin{itemize}
\item
  $\forall a \in A$に対し、$m \leq a$が成り立つ。
\item
  $\forall b \in O$に対し、$b \in L(A)$が成り立つなら、$b \leq m$が成り立つ。
\end{itemize}
\end{thm}
\begin{proof}
順序体$O$の部分集合$A$について、その順序体$O$の元$m$がその集合$A$のその順序体$O$における上限となるなら、$m = \min{U(A)}$が成り立つ。ここで、$m \in U(A)$が成り立つので、$\forall a \in A$に対し、$a \leq m$が成り立つ。また、$\forall b \in O$に対し、$b \in U(A)$が成り立つなら、$\forall a \in A$に対し、$a \leq b$が成り立ち、したがって、その元$a'$はその集合$A$のその順序体$O$における上界となっておりその集合$U(A)$に属することになる。上限の定義より$m = \min{U(A)}$が成り立つので、$m \leq b$が成り立つ。したがって、次のことが成り立つ。
\begin{itemize}
\item
  $\forall a \in A$に対し、$a \leq m$が成り立つ。
\item
  $\forall b \in O$に対し、$b \in U(A)$が成り立つなら、$m \leq b$が成り立つ。
\end{itemize}\par
逆に、$m \in O$なる元$m$について、次のことが成り立つなら、
\begin{itemize}
\item
  $\forall a \in A$に対し、$a \leq m$が成り立つ。
\item
  $\forall b \in O$に対し、$b \in U(A)$が成り立つなら、$m \leq b$が成り立つ。
\end{itemize}
その元$m$はその集合$A$のその順序体$O$における上界となっており$m \in U(A)$が成り立つ。さらに、$\forall b \in O$に対し、$b \in U(A)$が成り立つなら、$m \leq b$が成り立つので、$m \in U(A)$が成り立つかつ、$\forall b \in U(A)$に対し、$m \leq b$が成り立つことになり、ゆえに、$m = \min{U(A)}$が成り立ちその集合$U(A)$の元$m$がその集合$A$のその順序体$O$における上限となる。\par
下限についても同様にして示される。
\end{proof}
\begin{thm}\label{4.1.1.12}
順序体$O$の部分集合$A$について、その順序体$O$の元$m$がその集合$A$のその順序体$O$における上限となるならそのときに限り、$m \in O$なる元$m$について、次のことが成り立つ。
\begin{itemize}
\item
  $\forall a \in A$に対し、$a \leq m$が成り立つ。
\item
  $\forall b \in O\exists a \in A$に対し、$b < m$が成り立つなら、$b < a$が成り立つ。
\end{itemize}\par
同様に、その集合$A$の元$m$がその集合$A$のその順序体$O$における下限となるならそのときに限り、$m \in O$なる元$m$について、次のことが成り立つ。
\begin{itemize}
\item
  $\forall a \in A$に対し、$m \leq a$が成り立つ。
\item
  $\forall b \in O\exists a \in A$に対し、$a < b$が成り立つなら、$a < b$が成り立つ。
\end{itemize}
\end{thm}
\begin{proof}
順序体$O$の部分集合$A$について、その順序体$O$の元$m$がその集合$A$のその順序体$O$における上限となるなら、$m \in U(A)$が成り立つので、その集合$U(A)$の定義より$\forall a \in A$に対し、$a \leq m$が成り立つかつ、$\forall b \in O$に対し、$b \in U(A) \Rightarrow m \leq b$が成り立つ。これの対偶がとられれば、$b < m \Rightarrow b \notin U(A)$が成り立ち、$b \notin U(A)$が成り立つことと、$\exists a \in A$に対し、$b < a$が成り立つこととは同値であるので、$\forall b \in O\exists a \in A$に対し、$b < m$が成り立つなら、$b < a$が成り立つ。以上の議論により、次のことが成り立つ。
\begin{itemize}
\item
  $\forall a \in A$に対し、$a \leq m$が成り立つ。
\item
  $\forall b \in O\exists a \in A$に対し、$b < m$が成り立つなら、$b < a$が成り立つ。
\end{itemize}\par
逆に、次のことが成り立つなら、
\begin{itemize}
\item
  $\forall a \in A$に対し、$a \leq m$が成り立つ。
\item
  $\forall b \in O\exists a \in A$に対し、$b < m$が成り立つなら、$b < a$が成り立つ。
\end{itemize}
定義より$m \in U(A)$が成り立つかつ、$\exists a \in A$に対し、$a < b$が成り立つことと$b \notin U(A)$が成り立つこととは同値であるので、$b < m \Rightarrow b \notin U(A)$の対偶がとられれば、$b \in U(A) \Rightarrow m \leq b$が成り立ち、したがって、$m \in U(A)$が成り立つかつ、$b \in U(A) \Rightarrow m \leq b$が成り立つので、$m \in U(A)$が成り立つかつ、$\forall b \in U(A)$に対し、$m \leq b$が成り立ち、定義よりよって、$m = \sup A$が成り立つ。\par
下限についても同様にして示される。
\end{proof}
\begin{thm}\label{4.1.1.13} 順序体$O$の部分集合$A$について、次のことが成り立つ。
\begin{itemize}
\item
  その集合$A$の最大元$\max A$はその集合$A$のその順序体$O$における上限でもある。
\item
  その集合$A$の最小元$\min A$はその集合$A$のその順序体$O$における下限でもある。
\end{itemize}
\end{thm}
\begin{proof}
順序体$O$の部分集合$A$について、その集合$A$の最大元$\max A$は定義より、$\forall a \in A$に対し、$a \leq \max A$を満たす。これにより、その元$\max A$はその集合$A$のその順序体$O$における上界で、その集合$A$の上界全体の集合$U(A)$を用いて、$\max A \in U(A)$が成り立つ。ここで、$u < \max A$なる元$u$がその集合$U(A)$に存在すると仮定すると、$\max A \in A$も成り立っているので、$u < a$が成り立つような元$a$がその集合$A$に存在することになるが、これはその元$u$がその集合$A$の上界であることに矛盾する。したがって、$\forall u \in U(A)$に対し、$\max A \leq u$が成り立つので、その元$\max A$はその集合$A$のその順序体$O$における上限でもある。\par
下限についても同様にして示される。
\end{proof}
%\hypertarget{ux4e0aux9650ux6027ux8cea}{%
\subsubsection{上限性質}%\label{ux4e0aux9650ux6027ux8cea}}
\begin{axs}[上限性質]
順序体$O$が与えられたとき、$\forall A \in \mathfrak{P}(O)$に対し、その集合$A$が上に有界で空集合$\emptyset$でないなら、$\exists u \in O$に対し、$u = \sup A$が成り立つ、即ち、その順序体$O$の上に有界な任意の空集合$\emptyset$でない部分集合$A$に対して、その集合$A$の上限$\sup A$がその順序体$O$に存在し属する。この公理を上限性質という。
\end{axs}
\begin{axs}[下限性質]
順序体$O$が与えられたとき、$\forall A \in \mathfrak{P}(O)$に対し、その集合$A$が下に有界で空集合$\emptyset$でないなら、$\exists u \in O$に対し、$u = \inf A$が成り立つ、即ち、その順序体$O$の下に有界な任意の空集合$\emptyset$でない部分集合$A$に対して、その集合$A$の下限$\inf A$がその順序体$O$に存在し属する。この公理を下限性質という。
\end{axs}
\begin{axs}
上記で述べられた順序体であるかつ、上限性質を満たす集合を$\mathbb{R}$などと書きこの集合$\mathbb{R}$の元を実数という。即ち、次の公理たちを満たす集合$\mathbb{R}$の元は実数である。また、しばしば使われるので、集合$\left\{ a \in \mathbb{R} \middle| 0 < a \right\}$を$\mathbb{R}^{+}$とおく。
\begin{itemize}
\item
  $\forall a,b,c \in \mathbb{R}\left[ (a + b) + c = a + (b + c) \right]$
\item
  $\exists 0 \in \mathbb{R}\forall a \in \mathbb{R}[ a + 0 = 0 + a = a]$
\item
  $\forall a \in \mathbb{R}\exists - a \in \mathbb{R}\left[ a + ( - a) = - a + a = 0 \right]$
\item
  $\forall a,b \in \mathbb{R}[ a + b = b + a]$
\item
  $\forall a,b,c \in \mathbb{R}\left[ (ab)c = a(bc) \right]$
\item
  $\exists 1 \in \mathbb{R}\forall a \in \mathbb{R}[ a1 = 1a = a]$
\item
  $\forall a,b,c \in \mathbb{R}\left[ a(b + c) = ab + ac \land (a + b)c = ac + bc \right]$
\item
  $\forall a \in \mathbb{R} \setminus \left\{ 0 \right\}\exists a^{- 1} \in \mathbb{R}\left[ aa^{- 1} = a^{- 1}a = 1 \right]$
\item
  $\forall a,b \in \mathbb{R}[ a + b = b + a]$
\item
  $\forall 0,1 \in \mathbb{R}[ 0 \neq 1]$
\item
  $\forall a \in \mathbb{R}[ a \leq a]$
\item
  $\forall a,b \in \mathbb{R}[ a \leq b \land b \geq a \Rightarrow a = b]$
\item
  $\forall a,b,c \in \mathbb{R}[ a \leq b \land b \leq c \Rightarrow a \leq c]$
\item
  $\forall a,b \in \mathbb{R}[ a \leq b \vee b \leq a \Leftrightarrow \top]$
\item
  $\forall a,b,c \in \mathbb{R}[ a \leq b \Rightarrow a + c \leq b + c]$
\item
  $\forall a,b \in \mathbb{R}[ 0 \leq a \land 0 \leq b \Rightarrow 0 \leq ab]$
\item
  $\forall A \in \mathfrak{P}\left( \mathbb{R} \right) \setminus \left\{ \emptyset \right\}\left[ U(A) \neq \emptyset \Rightarrow \sup A \in \mathbb{R} \right]$
\end{itemize}
\end{axs}
\begin{thm}\label{4.1.1.14}
上限性質が認められれば、順序体$O$が与えられたとき、次のことが成り立つ\footnote{なお、以下、集合の記法として任意の順序体$O$の元$c$と算法$*$、2つの順序体$O$の部分集合たち$A$、$B$について$c*A = \left\{ c*a \middle| a \in A \right\}$、$A*B = \left\{ a*b \middle| a \in A \land b \in B \right\}$と書いてある。}。
\begin{itemize}
\item
  $\forall A \in \mathfrak{P}(O)$に対し、その集合$A$が下に有界で空集合$\emptyset$でないなら、その下限$\inf A$が存在し、$\inf A = - \sup( - A)$が成り立つ。
\item
  $\forall A \in \mathfrak{P}(O)$に対し、その集合$A$が上に有界で空集合$\emptyset$でないなら、その上限$\sup A$が存在し、$\sup A = - \inf( - A)$が成り立つ。
\end{itemize}
\end{thm}
\begin{proof}
順序体$O$が与えられたとき、$\forall A \in \mathfrak{P}(O)$に対し、その集合$A$が下に有界で空集合$\emptyset$でないなら、$\exists l \in O$に対し、$l \in L(A)$が成り立つ、即ち、$\exists l \in O\forall a \in A$に対し、$l \leq a$が成り立ち、したがって、$\forall - a \in - A$に対し、$- a \leq - l$が成り立ち、$- l \in U( - A)$が成り立つので、$U( - A) \neq \emptyset$が成り立つ。上限性質より、集合$- A$の上限$\sup( - A)$が存在し、$\forall - a \in - A$に対し、$- a \leq \sup( - A)$が成り立つかつ、定理\ref{4.1.1.12}より$\forall - b \in O\exists - a \in - A$に対し、$- b < \sup( - A)$が成り立つなら、$- b < - a$が成り立つので、$\forall a \in A$に対し、$- \sup( - A) \leq a$が成り立つかつ、$\forall b \in O\exists a \in A$に対し、$- \sup( - A) < b$が成り立つなら、$a < b$が成り立つ。これにより、定理\ref{4.1.1.12}より$\inf A = - \sup( - A)$が成り立つ。\par
下限についても同様にして示される。
\end{proof}
\begin{thm}\label{4.1.1.15}
順序体$O$が与えられたとき、上限性質が成り立つならそのときに限り、下限性質も成り立つ。
\end{thm}
\begin{proof}
順序体$O$が与えられたとき、上限性質が成り立つなら、下限性質が成り立つことは定理\ref{4.1.1.14}そのものである。逆も同様にして示される。
\end{proof}
\begin{thm}\label{4.1.1.16}
順序体$O$の空集合でない$A \subseteq B$なる2つの集合たち$A$、$B$について、上限性質が認められれば、次のことが成り立つ。
\begin{itemize}
\item
  その集合$B$が上に有界であるなら、その集合$A$もそうであり$\sup A \leq \sup B$が成り立つ。
\item
  その集合$B$が下に有界であるなら、その集合$A$もそうであり$\inf B \leq \inf A$が成り立つ。
\end{itemize}
\end{thm}
\begin{proof}
順序体$O$の空集合でない$A \subseteq B$なる2つの集合たち$A$、$B$について、上限性質が認められれば、その集合$B$が上に有界であるなら、$\exists m \in O$に対し、$m \in U(B)$が成り立ち、したがって、$\forall b \in B$に対し、$b \leq m$が成り立つ。ここで、$A \subseteq B$が成り立つことより$\forall a \in A$に対し、$a \in B$が成り立ち、したがって、$\forall a \in A$に対し、$a \leq m$が成り立つので、$U(A) \neq \emptyset$が成り立ちその集合$A$は上に有界である。また、$\forall u \in O$に対し、$u \in U(B)$が成り立つなら、$\forall b \in B$に対し、$b \leq u$が成り立ち、$A \subseteq B$が成り立つことより$\forall a \in A$に対し、$a \leq u$が成り立ち、したがって、$u \in U(A)$が成り立つので、$U(B) \subseteq U(A)$が成り立つ。ゆえに、その上限$\sup B$に対し、$\sup B \in U(A)$が成り立ち、上限性質よりその上限$\sup A$が存在し、$\forall u \in U(A)$に対し、$\sup A \leq u$が成り立つので、$\sup A \leq \sup B$が成り立つ。\par
下限についても同様にして示される。
\end{proof}
\begin{thm}\label{4.1.1.17}
順序体$O$の空集合でない2つの集合たち$A$、$B$について、上限性質が認められれば、次のことが成り立つ。
\begin{itemize}
\item
  それらの集合たち$A$、$B$がどちらも上に有界であるとき、次式が成り立つ。
\begin{align*}
\sup(A + B) = \sup A + \sup B
\end{align*}
\item
  $A,B \in \mathfrak{P}\left( \left\{ c \in O \middle| 0 \leq c \right\} \right)$でそれらの集合たち$A$、$B$がどちらも上に有界であるとき、次式が成り立つ。
\begin{align*}
\sup(AB) = \sup A\sup B
\end{align*}
\item
  それらの集合たち$A$、$B$がどちらも下に有界であるとき、次式が成り立つ。
\begin{align*}
\inf(A + B) = \inf A + \inf B
\end{align*}
\item
  $A,B \in \mathfrak{P}\left( \left\{ c \in O \middle| 0 \leq c \right\} \right)$のとき、それらの集合たち$A$、$B$がどちらも下に有界であるとき、次式が成り立つ。
\begin{align*}
\inf(AB) = \inf A\inf B
\end{align*}
\end{itemize}
\end{thm}
\begin{proof}
順序体$O$の空集合でない2つの集合たち$A$、$B$について、それらの集合たち$A$、$B$がどちらも上に有界であるとき、上限性質よりそれらの集合たち$A$、$B$の上限たち$\sup A$、$\sup B$が存在し$\sup A \in U(A)$かつ$\sup B \in U(B)$が成り立ち、$\forall a \in A\forall b \in B$に対し、$a \leq \sup A$かつ$b \leq \sup B$が成り立つので、$a + b \leq \sup A + \sup B$が成り立つ。したがって、その集合$A + B$は上に有界で上限性質よりその上限$\sup(A + B)$が存在し$a + b \leq \sup(A + B)$が成り立つ。ここで、$\sup A + \sup B < \sup(A + B)$と仮定すれば、定理\ref{4.1.1.12}より$\sup A + \sup B < a + b \leq \sup(A + B)$なる集合$A + B$の元$a + b$が存在できるが、これは$a + b \leq \sup A + \sup B$が成り立つことに矛盾する。したがって、$\sup(A + B) \leq \sup A + \sup B$が成り立つ、即ち、$0 \leq \sup A + \sup B - \sup(A + B)$が成り立つ。また、定理\ref{4.1.1.12}より$\forall\varepsilon_{A} \in O\exists a \in A$に対し、$0 < \varepsilon_{A}$が成り立つなら、$\sup A - \varepsilon_{A} < a$が成り立つので、したがって、$0 \leq \sup A - a < \varepsilon_{A}$が成り立つ。同様にして、$\forall\varepsilon_{B} \in O\exists b \in B$に対し、$0 < \varepsilon_{B}$が成り立つなら、$0 \leq \sup B - b < \varepsilon_{B}$が成り立つことが示される。ここで、$\varepsilon = \varepsilon_{A} + \varepsilon_{B}$とおけば、次のようなり、
\begin{align*}
&\quad a + b \leq \sup(A + B) \land 0 \leq \sup A + \sup B - \sup(A + B) \land 0 \leq \sup A - a < \varepsilon_{A} \land 0 \leq \sup B - b < \varepsilon_{B}\\ 
&\Rightarrow 0 \leq \sup A + \sup B - \sup(A + B) \leq \sup A + \sup B - (a + b) \land 0 \leq \left( \sup A - a \right) + \left( \sup B - b \right) < \varepsilon_{A} + \varepsilon_{B}\\ 
&\Leftrightarrow 0 \leq \sup A + \sup B - \sup(A + B) \leq \sup A + \sup B - (a + b) \land 0 \leq \sup A + \sup B - (a + b) < \varepsilon\\ 
&\Leftrightarrow 0 \leq \sup A + \sup B - \sup(A + B) \leq \sup A + \sup B - (a + b) < \varepsilon
\end{align*}
したがって、$0 \leq \sup A + \sup B - \sup(A + B) < \varepsilon$が成り立ち、定理\ref{4.1.1.8}よりよって、$0 = \sup A + \sup B - \sup(A + B)$が成り立つ、即ち、$\sup(A + B) = \sup A + \sup B$が成り立つ。\par
$A,B \in \mathfrak{P}\left( \left\{ c \in O \middle| 0 \leq c \right\} \right)$のとき、それらの集合たち$A$、$B$がどちらも上に有界であるとき、$\sup(AB) = \sup A\sup B$が成り立つことも同様にして示される。下限についても同様にして示される。
\end{proof}
%\hypertarget{dedekindux306eux5207ux65ad}{%
\subsubsection{Dedekindの切断}%\label{dedekindux306eux5207ux65ad}}
\begin{dfn}
順序体$O$の空でない部分集合の順序対$\left( O_{-},O_{+} \right)$が次のことを満たすとき、この対$\left( O_{-},O_{+} \right)$をDedekindの切断という。
\begin{itemize}
\item
  $O = O_{-} \sqcup O_{+}$が成り立つ。
\item
  $\forall a \in O_{-}\forall b \in O_{+}$に対し、$a < b$が成り立つ。
\end{itemize}
\end{dfn}
\begin{axs}[Dedekindの公理]
順序体$O$のDedekindの切断$\left( O_{-},O_{+} \right)$が次の2通りのみに限るとする。この公理をDedekindの公理という。
\begin{itemize}
\item
  その集合$O_{-}$の最大元が存在せず、その集合$O_{+}$の最小元が存在する。
\item
  その集合$O_{-}$の最大元が存在し、その集合$O_{+}$の最小元が存在しない。
\end{itemize}
\end{axs}
\begin{thm}\label{4.1.1.18} Dedekindの公理は上限性質と同値である。
\end{thm}
\begin{proof}
順序体$O$のDedekindの切断$\left( O_{-},O_{+} \right)$が次の2通りのみに限るとき、
\begin{itemize}
\item
  その集合$O_{-}$の最大元が存在せず、その集合$O_{+}$の最小元が存在する。
\item
  その集合$O_{-}$の最大元が存在し、その集合$O_{+}$の最小元が存在しない。
\end{itemize}
$\forall A \in \mathfrak{P}(O)$に対し、その集合$A$が上に有界で空集合$\emptyset$でないなら、$\exists u \in O$に対し、$u \in U(A)$が成り立つ、即ち、$\forall a \in A$に対し、$a \leq u$が成り立つ。そこで、$\forall a \in O \setminus U(A)\forall b \in U(A)$に対し、$a \notin U(A)$なので、$\exists c \in A$に対し、$a \leq c$が成り立ち、$c \leq b$より$a \leq b$が成り立つ。そこで、$O = O \setminus U(A) \sqcup U(A)$が成り立つことから、$a < b$が成り立つ。これにより、その組$\left( O \setminus U(A),U(A) \right)$がその順序体$O$のDedekindの切断となっているので、次の2通りのみに限る。
\begin{itemize}
\item
  その集合$O \setminus U(A)$の最大元が存在せず、その集合$U(A)$の最小元が存在する。
\item
  その集合$O \setminus U(A)$の最大元が存在し、その集合$U(A)$の最小元が存在しない。
\end{itemize}
その集合$O \setminus U(A)$の最大元が存在せず、その集合$U(A)$の最小元が存在するとき、その最小元が$u$とおかれれば、$u = \sup A$が成り立つ。その集合$O \setminus U(A)$の最大元が存在し、その集合$U(A)$の最小元が存在しないとき、その最大元が$l$とおかれよう。$\exists a \in O$に対し、$a \in A$かつ$a \in U(A)$が成り立つとすると、$\forall b \in U(A)$に対し、$a \leq b$が成り立つので、$a = \min{U(A)}$が得られるが、これは仮定に矛盾している。したがって、$\forall a \in O$に対し、$a \in A$が成り立つなら、$a \notin U(A)$が成り立つので、$A \subseteq O \setminus U(A)$が得られる。これにより、$\forall a \in A$に対し、$a \leq l$が成り立つので、$l \in U(A)$が得られるが、$l \in O \setminus U(A)$が成り立つことに矛盾する。よって、上限性質が成り立つ。\par
逆に、上限性質が成り立つなら、Dedekindの切断$\left( O_{-},O_{+} \right)$について、$\forall a \in O_{-}\exists b \in O_{+}$に対し、$a < b$が成り立つので、その集合$O_{-}$は上に有界で空集合でない。したがって、$\exists u \in O$に対し、$u = \sup O_{-}$が成り立つ。その集合$O_{-}$の最大元が存在しないなら、その集合$U(A)$の最小元が存在しないと仮定しよう。このとき、$u \in O_{-}$が成り立つと仮定すると、$\forall a \in O_{-}$に対し、$a \leq u$が成り立つので、$u = \max O_{-}$となり仮定に矛盾する。ゆえに、$u \notin O_{-}$が成り立つ。また、$O = O_{-} \sqcup O_{+}$より$u \in O_{+}$が成り立ち、$\forall a \in O$に対し、$a \in O_{+}$が成り立つなら、$\forall b \in O_{-}$に対し、$b < a$が成り立つので、$a \in U\left( O_{+} \right)$が得られる。したがって、$\forall a \in O_{+}$に対し、$u \leq a$が成り立つ。これにより、$u = \min O_{+}$が得られることになるが、これは仮定に矛盾している。また、その集合$O_{-}$の最大元が存在するなら、その集合$O_{+}$の最小元も存在すると仮定しよう。このとき、定理\ref{4.1.1.13}より$u = \sup O_{-} = \max O_{-}$が成り立つ。そこで、その集合$O_{+}$の最小元$\min O_{+}$について、$\forall a \in O_{-}$に対し、$a < \min O_{+}$が成り立つので、$\min O_{+} \in U\left( O_{-} \right)$が得られ、したがって、$\sup O_{-} \leq \min O_{+}$が成り立つ。$O = O_{-} \sqcup O_{+}$より$\sup O_{-} = \max O_{-} < \min O_{+}$が得られるので、次のように元$m$がおかれれば、
\begin{align*}
m = \frac{\max O_{-} + \min O_{+}}{2}
\end{align*}
$\max O_{-} < m < \min O_{+}$が成り立つ。$m \in O_{-}$とすれば、$\max O_{-} < m$が成り立つことに矛盾するので、$O = O_{-} \sqcup O_{+}$より$m \in O_{+}$が成り立つが、これも$m < \min O_{+}$が成り立つことに矛盾する。以上の議論により、次のことが成り立つ。
\begin{itemize}
\item
  その集合$O \setminus U(A)$の最大元が存在せず、その集合$U(A)$の最小元が存在する。
\item
  その集合$O \setminus U(A)$の最大元が存在し、その集合$U(A)$の最小元が存在しない。
\end{itemize}
上の議論により、上の2通りしかないことも示される。
\end{proof}
%\hypertarget{ux81eaux7136ux6570}{%
\subsubsection{自然数}%\label{ux81eaux7136ux6570}}
\begin{dfn}
集合$\mathbb{R}$の部分集合$A$が次の条件たちを満たすとき、その集合$A$は継承的で
あるという。
\begin{itemize}
\item
  $0 \in A$が成り立つ。
\item
  $n \in A \Rightarrow n + 1 \in A$が成り立つ。
\end{itemize}
\end{dfn}
\begin{thm}\label{4.1.1.19}
継承的な集合全体の集合を$\varGamma$とおくと、次のことが成り立つ。
\begin{itemize}
\item
  継承的集合$\varGamma$は存在する。
\item
  いくつかの継承的集合の共通部分も継承的である、即ち、次式が成り立つ。
\begin{align*}
\bigcap_{A \in \varGamma' \subseteq \varGamma} A \in \varGamma
\end{align*}
\end{itemize}
\end{thm}
\begin{proof}
継承的集合$\varGamma$は存在することについては、例えば、集合$\mathbb{R}$を考えればよい。いくつかの継承的集合の共通部分も継承的であることは定義にあてはめられれば、直ちに示される。
\end{proof}
\begin{thm}\label{4.1.1.20}
集合$\mathbb{R}$の全ての継承的な部分集合の共通部分$\bigcap_{} \varGamma$と写像$+_{1}:\bigcap_{} \varGamma \rightarrow \bigcap_{} \varGamma;n \mapsto n + 1$を用いた組$\left( \bigcap_{} \varGamma,0, +_{1} \right)$は1つのPeano系となる。
\end{thm}
\begin{proof}
集合$\mathbb{R}$の全ての継承的な部分集合の共通部分$\bigcap_{} \varGamma$と写像$+_{1}:\bigcap_{} \varGamma \rightarrow \bigcap_{} \varGamma;n \mapsto n + 1$を用いた組$\left( \bigcap_{} \varGamma,0, +_{1} \right)0$が与えられたとき、今のところ次のことが示されているのであった。
\begin{itemize}
\item
  写像$+_{1}:\bigcap_{} \varGamma \rightarrow \bigcap_{} \varGamma$が存在する。
\item
  その写像$+_{1}$は単射である。
\end{itemize}
ここで、集合$\mathbb{R}$の継承的な部分集合$A$のうち$- 1$に属されないものが存在するので、次のことが成り立つ。
\begin{itemize}
\item
  ある1つの元0が存在し$0 \in \bigcap_{} \varGamma \setminus V\left( +_{1} \right)$を満たす。
\end{itemize}
最後に、$\forall A'\in \mathfrak{P}\left( \bigcap_{} \varGamma \right)$に対し、$0 \in A'$が成り立つかつ、$\forall n \in \bigcap_{} \varGamma$に対し、$n \in A' \Rightarrow +_{1}(n) \in A'$が成り立つなら、その集合$A'$も集合$\mathbb{R}$の継承的な部分集合であるから、$A' \in \varGamma$が成り立ち、$\bigcap_{} \varGamma \subseteq A'$が成り立つので、$\bigcap_{} \varGamma = A'$が得られる。\par
以上より、次のことが成り立つので、
\begin{itemize}
\item
  写像$+_{1}:\bigcap_{} \varGamma \rightarrow \bigcap_{} \varGamma$が存在する。
\item
  ある1つの元0が存在し$0 \in \bigcap_{} \varGamma \setminus V\left( +_{1} \right)$を満たす。
\item
  その写像$+_{1}$は単射である。
\item
  $\forall A'\in \mathfrak{P}\left( \bigcap_{} \varGamma \right)$に対し、$0 \in A'$かつ$\forall n \in \bigcap_{} \varGamma\left[ n \in A' \Rightarrow +_{1}(n) \in A' \right]$が成り立つなら、$A' = \bigcap_{} \varGamma$が成り立つ。
\end{itemize}
その組$\left( \bigcap_{} \varGamma,0, +_{1} \right)$は1つのPeano系となる。
\end{proof}
\begin{dfn}
Peano系$\left( \mathcal{N,}\nu,\sigma \right)$が与えられたとき、その集合$\mathcal{N}$から$\nu$を除いた集合を$\mathbb{N}$と書きこの集合の元を自然数という。
\end{dfn}\par
ただし、文献によって$\nu$がその集合$\mathbb{N}$に属するように定義する場合があることに注意されたい。以下、$\varLambda_{n} = \left\{ i \in \mathbb{N}|1 \leq i \leq n \right\}$とおく。
\begin{dfn}
$\exists f \in \mathfrak{F}\left( \varLambda_{n},A \right)$に対し、$f:\varLambda_{n}\overset{\sim}{\rightarrow}A$なる集合$A$は$n$つの元を持つといいこの$n$を$cardA$、$\# A$、$n(A)$、$|A|$などと書きこのような集合を有限集合という。
\end{dfn}
\begin{thm}\label{4.1.1.21}
$\forall A \in \mathfrak{P}\left( \mathbb{N} \right)\exists m \in A$に対し、$A \neq \emptyset$が成り立つなら、$m = \min A$が成り立つ。
\end{thm}
\begin{proof}
$\forall A \in \mathfrak{P}\left( \mathbb{N} \right)$に対し、$A \neq \emptyset$が成り立つとする。空集合でないその集合$\mathbb{N}$の任意の部分集合$A$に対し、$\#A = 1$が成り立つなら、明らかにその唯一の元がその集合$A$の最小元$\min A$である。$\#A = k \in \mathbb{N}$なる集合$A$に最小元$\min A$が存在すると仮定する。\par
$\#A' = k + 1 \in \mathbb{N}$なるその集合$\mathbb{N}$の空集合でない集合$A'$についてその集合$A'$の1つの元$a$を用いて集合$A' \setminus \left\{ a \right\}$を考えると、$\#{A' \setminus \left\{ a \right\}} = k$が成り立ち仮定よりその最小元$\min{A' \setminus \left\{ a \right\}}$が存在する。ここで、その集合$A' \setminus \left\{ a \right\}$も集合$\mathbb{N}$の部分集合でありその集合$A'$も全順序集合であるから、元$\min\left\{ \min{A' \setminus \left\{ a \right\}},a \right\}$がその最小限$\min A'$となり数学的帰納法により示すべきことが示された。
\end{proof}
\begin{dfn}
次式のように定義される2つの集合たち$\mathbb{Z}$、$\mathbb{Q}$の元をそれぞれ整数、有理数という。
\begin{align*}
\mathbb{Z} &= \left\{ n \in \mathbb{R} \middle| n \in - \mathbb{N} \cup \left\{ 0 \right\} \cup \mathbb{N} \right\}\\ 
\mathbb{Q} &= \left\{ \frac{m}{n} \in \mathbb{R} \middle| m \in \mathbb{Z} \land n \in \mathbb{N} \right\}
\end{align*}
ちなみに、集合$\mathbb{R} \setminus \mathbb{Q}$の元を無理数という。
\end{dfn}
%\hypertarget{ux3acux3c1ux3c7ux3b9ux3bcux3aeux3b4ux3b7ux3c2ux306eux6027ux8cea}{%
\subsubsection{Archimedesの性質}%\label{ux3acux3c1ux3c7ux3b9ux3bcux3aeux3b4ux3b7ux3c2ux306eux6027ux8cea}}
\begin{thm}[Archimedesの性質]\label{4.1.1.22}
$\forall a,b \in \mathbb{R}^{+}\exists n \in \mathbb{N}$に対し、$a < nb$が成り立つ。これをArchimedesの性質という。
\end{thm}
\begin{proof}
$\exists a,b \in \mathbb{R}^{+}\forall n \in \mathbb{N}$に対し、$nb \leq a$が成り立つなら、$n \leq \frac{a}{b}$が得られ、したがって、集合$\mathbb{N}$は上に有界であるので、上限$\sup\mathbb{N}$が存在する。これにより、$\forall n \in \mathbb{N}$に対し、次式が成り立つ。
\begin{align*}
n \leq \sup\mathbb{N} \leq \frac{a}{b}
\end{align*}
そこで、上限の定義よりその集合$\mathbb{N}$の任意の上界$u$に対し、$\sup\mathbb{N} \leq u$が成り立つので、その実数$\sup\mathbb{N} - 1$はその集合$\mathbb{N}$の上界でありえない、即ち、$\exists N \in \mathbb{N}$に対し、$\sup\mathbb{N} - 1 < N$が成り立つ。これにより、$\sup\mathbb{N} < N + 1$が得られる。しかしながら、これは上限がその集合$\mathbb{N}$の上界であることに矛盾している。ゆえに、$\forall a,b \in \mathbb{R}^{+}\exists n \in \mathbb{N}$に対し、$a < nb$が成り立つ。
\end{proof}
%\hypertarget{nux4e57ux6839}{%
\subsubsection{$n$乗根}%\label{nux4e57ux6839}}
\begin{thm}\label{4.1.1.23}
$\forall a \in \mathbb{R}^{+}\forall n \in \mathbb{N}\exists b \in \mathbb{R}^{+}$に対し、$b^{n} = a$が成り立つ。さらに、そのような実数$b$はただ1つのみ存在する。
\end{thm}
\begin{dfn}
正の実数$a$と自然数$n$に対し、$b^{n} = a$なる実数$b$をその実数$a$の$n$乗根などといい$a^{\frac{1}{n}}$、$\sqrt[n]{a}$などと書く。特に、$n = 2$のとき、平方根、$n = 3$のとき、立方根といい、平方根は$\sqrt{a}$などと書く。
\end{dfn}
\begin{proof}
$\forall a \in \mathbb{R}^{+}\forall n \in \mathbb{N}$に対し、$A = \left\{ c \in \mathbb{R} \middle| 0 \leq c \land c^{n} \leq a \right\}$とおかれれば、$a < 1$のとき、$a^{n} \leq a$が成り立つので、$a \in A$が成り立つ。$1 \leq a$のとき、$1^{n} = 1 \leq a$が成り立つので、$1 \in A$が成り立つ。さらに、$\forall c \in A$に対し、$c^{n} \leq a$が成り立つので、$a \in U(A)$が成り立つ。上限性質より$\exists u \in \mathbb{R}$に対し、$u = \sup A$が成り立ち、$u \neq 0$より$0 < u$が得られる。\par
ここで、$u^{n} < a$が成り立つと仮定すると、次のように正の実数$\varepsilon$がおかれれば、
\begin{align*}
\varepsilon = \min\left\{ u,\frac{a - u^{n}}{n!nu^{n - 1}} \right\}
\end{align*}
$0 < \varepsilon$で次のようになることから、
\begin{align*}
(u + \varepsilon)^{n} &= \sum_{k \in \varLambda_{n} \cup \left\{ 0 \right\}} {\frac{n!}{k!(n - k)!}u^{n - k}\varepsilon^{k}}\\ 
&= u^{n} + \sum_{k \in \varLambda_{n}} {\frac{n!}{k!(n - k)!}u^{n - k}\varepsilon^{k}}\\ 
&\leq u^{n} + \sum_{k \in \varLambda_{n}} {n!u^{n - k}\varepsilon^{k}}\\ 
&\leq u^{n} + \sum_{k \in \varLambda_{n}} {n!u^{n - 1}\varepsilon}\\ 
&= u^{n} + n!nu^{n - 1}\varepsilon\\ 
&\leq u^{n} + n!nu^{n - 1}\frac{a - u^{n}}{n!nu^{n - 1}}\\ 
&= u^{n} + a - u^{n} = a
\end{align*}
$u < u + \varepsilon$かつ$u + \varepsilon \in A$が得られるが、これはその実数$u$がその集合$A$の上限であることに矛盾する。\par
一方で、$a < u^{n}$が成り立つと仮定すると、次のように正の実数$\varepsilon$がおかれれば、
\begin{align*}
\varepsilon = \min\left\{ u,\frac{u^{n} - a}{n!nu^{n - 1}} \right\}
\end{align*}
$0 < \varepsilon$で、$\forall k \in \varLambda_{n}$に対し、$- n!u^{k - 1} \leq 0 \leq \varepsilon^{k - 1}$が成り立つことに注意すれば、次のようになることから、
\begin{align*}
(u - \varepsilon)^{n} &= \sum_{k \in \varLambda_{n} \cup \left\{ 0 \right\}} {\frac{n!}{k!(n - k)!}u^{n - k}( - \varepsilon)^{k}}\\ 
&= u^{n} + \sum_{\scriptsize \begin{matrix} k \in \varLambda_{n} \\ k \in 2\mathbb{Z} \end{matrix}} {\frac{n!}{k!(n - k)!}u^{n - k}( - \varepsilon)^{k}} + \sum_{\scriptsize \begin{matrix} k \in \varLambda_{n} \\ k \notin 2\mathbb{Z} \end{matrix}} {\frac{n!}{k!(n - k)!}u^{n - k}( - \varepsilon)^{k}}\\ 
&= u^{n} + \sum_{\scriptsize \begin{matrix} k \in \varLambda_{n} \\ k \in 2\mathbb{Z} \end{matrix}} {\frac{n!}{k!(n - k)!}u^{n - k}\varepsilon^{k}} - \sum_{\scriptsize \begin{matrix} k \in \varLambda_{n} \\ k \notin 2\mathbb{Z} \end{matrix}} {\frac{n!}{k!(n - k)!}u^{n - k}\varepsilon^{k}}\\ 
&\geq u^{n} + \sum_{\scriptsize \begin{matrix} k \in \varLambda_{n} \\ k \in 2\mathbb{Z} \end{matrix}} {u^{n - k}\varepsilon^{k}} - \sum_{\scriptsize \begin{matrix} k \in \varLambda_{n} \\ k \notin 2\mathbb{Z} \end{matrix}} {n!u^{n - k}\varepsilon^{k}}\\ 
&\geq u^{n} - \sum_{\scriptsize \begin{matrix} k \in \varLambda_{n} \\ k \in 2\mathbb{Z} \end{matrix}} {n!u^{n - 1}\varepsilon} - \sum_{\scriptsize \begin{matrix} k \in \varLambda_{n} \\ k \notin 2\mathbb{Z} \end{matrix}} {n!u^{n - 1}\varepsilon}\\ 
&= u^{n} - \sum_{k \in \varLambda_{n} } {n!u^{n - 1}\varepsilon}\\ 
&= u^{n} - n!nu^{n - 1}\varepsilon\\ 
&\geq u^{n} - n!nu^{n - 1}\frac{u^{n} - a}{n!nu^{n - 1}}\\ 
&= u^{n} - u^{n} + a = a
\end{align*}
$u - \varepsilon < u$かつ$a \leq (u - \varepsilon)^{n}$が得られる。そこで、$u - \varepsilon \in U(A)$が成り立つと仮定すると、$u = \min{U(A)} \leq u - \varepsilon$より$0 \leq \varepsilon$となって矛盾しているので、$u - \varepsilon \notin U(A)$が成り立つ。したがって、$\exists b \in A$に対し、$u - \varepsilon < b$が得られ、したがって、$(u - \varepsilon)^{n} < b^{n} \leq a$が得られるが、これは$a \leq (u - \varepsilon)^{n}$が成り立つことに矛盾する。\par
以上の議論により、$\exists u \in \mathbb{R}^{+}$に対し、$u^{n} = a$が成り立つ。そこで、$u^{n} = v^{n} = a$なる正の実数たち$u$、$v$が与えられたとき、$\forall k \in \varLambda_{n}$に対し、$0 < u^{n - k}v^{k - 1}$が成り立つかつ、次のようになるので、
\begin{align*}
(u - v)\sum_{k \in \varLambda_{n}} {u^{n - k}v^{k - 1}} &= \sum_{k \in \varLambda_{n}} {u^{n - k + 1}v^{k - 1}} - \sum_{k \in \varLambda_{n}} {u^{n - k}v^{k}}\\ 
&= u^{n} + \sum_{k \in \varLambda_{n} \setminus \left\{ 1 \right\}} {u^{n - k + 1}v^{k - 1}} - \sum_{k \in \varLambda_{n - 1}} {u^{n - k}v^{k}} - v^{n}\\ 
&= u^{n} + \sum_{k \in \varLambda_{n - 1}} {u^{n - k}v^{k}} - \sum_{k \in \varLambda_{n - 1}} {u^{n - k}v^{k}} - v^{n}\\ 
&= u^{n} - v^{n} = a - a = 0
\end{align*}
$u = v$が得られる。
\end{proof}
\begin{thebibliography}{50}
  \bibitem{1}
  杉浦光夫, 解析入門I, 東京大学出版社, 1985. 第34刷 p1-11,19 ISBN978-4-13-062005-5
  \bibitem{2}
  松坂和夫, 代数系入門, 岩波書店, 1976. 新装版第2刷 p45-50,107-112 ISBN978-4-00-029873-5
  \bibitem{3}
  土屋卓也. "実数の定義(その1) -- Dedekind切断". 愛媛大学. \url{http://daisy.math.sci.ehime-u.ac.jp/users/tsuchiya/math/calculus/dedekind.pdf} (2021-1-10 取得)
\end{thebibliography}
\end{document}
