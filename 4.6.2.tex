\documentclass[dvipdfmx]{jsarticle}
\setcounter{section}{6}
\setcounter{subsection}{1}
\usepackage{xr}
\externaldocument{4.5.5}
\externaldocument{4.6.1}
\usepackage{amsmath,amsfonts,amssymb,array,comment,mathtools,url,docmute}
\usepackage{longtable,booktabs,dcolumn,tabularx,mathtools,multirow,colortbl,xcolor}
\usepackage[dvipdfmx]{graphics}
\usepackage{bmpsize}
\usepackage{amsthm}
\usepackage{enumitem}
\setlistdepth{20}
\renewlist{itemize}{itemize}{20}
\setlist[itemize]{label=•}
\renewlist{enumerate}{enumerate}{20}
\setlist[enumerate]{label=\arabic*.}
\setcounter{MaxMatrixCols}{20}
\setcounter{tocdepth}{3}
\newcommand{\rotin}{\text{\rotatebox[origin=c]{90}{$\in $}}}
\newcommand{\amap}[6]{\text{\raisebox{-0.7cm}{\begin{tikzpicture} 
  \node (a) at (0, 1) {$\textstyle{#2}$};
  \node (b) at (#6, 1) {$\textstyle{#3}$};
  \node (c) at (0, 0) {$\textstyle{#4}$};
  \node (d) at (#6, 0) {$\textstyle{#5}$};
  \node (x) at (0, 0.5) {$\rotin $};
  \node (x) at (#6, 0.5) {$\rotin $};
  \draw[->] (a) to node[xshift=0pt, yshift=7pt] {$\textstyle{\scriptstyle{#1}}$} (b);
  \draw[|->] (c) to node[xshift=0pt, yshift=7pt] {$\textstyle{\scriptstyle{#1}}$} (d);
\end{tikzpicture}}}}
\newcommand{\twomaps}[9]{\text{\raisebox{-0.7cm}{\begin{tikzpicture} 
  \node (a) at (0, 1) {$\textstyle{#3}$};
  \node (b) at (#9, 1) {$\textstyle{#4}$};
  \node (c) at (#9+#9, 1) {$\textstyle{#5}$};
  \node (d) at (0, 0) {$\textstyle{#6}$};
  \node (e) at (#9, 0) {$\textstyle{#7}$};
  \node (f) at (#9+#9, 0) {$\textstyle{#8}$};
  \node (x) at (0, 0.5) {$\rotin $};
  \node (x) at (#9, 0.5) {$\rotin $};
  \node (x) at (#9+#9, 0.5) {$\rotin $};
  \draw[->] (a) to node[xshift=0pt, yshift=7pt] {$\textstyle{\scriptstyle{#1}}$} (b);
  \draw[|->] (d) to node[xshift=0pt, yshift=7pt] {$\textstyle{\scriptstyle{#2}}$} (e);
  \draw[->] (b) to node[xshift=0pt, yshift=7pt] {$\textstyle{\scriptstyle{#1}}$} (c);
  \draw[|->] (e) to node[xshift=0pt, yshift=7pt] {$\textstyle{\scriptstyle{#2}}$} (f);
\end{tikzpicture}}}}
\renewcommand{\thesection}{第\arabic{section}部}
\renewcommand{\thesubsection}{\arabic{section}.\arabic{subsection}}
\renewcommand{\thesubsubsection}{\arabic{section}.\arabic{subsection}.\arabic{subsubsection}}
\everymath{\displaystyle}
\allowdisplaybreaks[4]
\usepackage{vtable}
\theoremstyle{definition}
\newtheorem{thm}{定理}[subsection]
\newtheorem*{thm*}{定理}
\newtheorem{dfn}{定義}[subsection]
\newtheorem*{dfn*}{定義}
\newtheorem{axs}[dfn]{公理}
\newtheorem*{axs*}{公理}
\renewcommand{\headfont}{\bfseries}
\makeatletter
  \renewcommand{\section}{%
    \@startsection{section}{1}{\z@}%
    {\Cvs}{\Cvs}%
    {\normalfont\huge\headfont\raggedright}}
\makeatother
\makeatletter
  \renewcommand{\subsection}{%
    \@startsection{subsection}{2}{\z@}%
    {0.5\Cvs}{0.5\Cvs}%
    {\normalfont\LARGE\headfont\raggedright}}
\makeatother
\makeatletter
  \renewcommand{\subsubsection}{%
    \@startsection{subsubsection}{3}{\z@}%
    {0.4\Cvs}{0.4\Cvs}%
    {\normalfont\Large\headfont\raggedright}}
\makeatother
\makeatletter
\renewenvironment{proof}[1][\proofname]{\par
  \pushQED{\qed}%
  \normalfont \topsep6\p@\@plus6\p@\relax
  \trivlist
  \item\relax
  {
  #1\@addpunct{.}}\hspace\labelsep\ignorespaces
}{%
  \popQED\endtrivlist\@endpefalse
}
\makeatother
\renewcommand{\proofname}{\textbf{証明}}
\usepackage{tikz,graphics}
\usepackage[dvipdfmx]{hyperref}
\usepackage{pxjahyper}
\hypersetup{
 setpagesize=false,
 bookmarks=true,
 bookmarksdepth=tocdepth,
 bookmarksnumbered=true,
 colorlinks=false,
 pdftitle={},
 pdfsubject={},
 pdfauthor={},
 pdfkeywords={}}
\begin{document}
%\hypertarget{ux6975ux9650ux3068ux7a4dux5206}{%
\subsection{極限と積分}%\label{ux6975ux9650ux3068ux7a4dux5206}}
%\hypertarget{ux5358ux8abfux53ceux675fux5b9aux7406}{%
\subsubsection{単調収束定理}%\label{ux5358ux8abfux53ceux675fux5b9aux7406}}\par
単調収束定理に関する定理を再掲しておこう。これらはすでにみたので、証明を省くことにする。
\begin{thm*}[定理\ref{4.6.1.23}の再掲]
測度空間$(X,\varSigma,\mu)$と集合$\mathcal{M}_{(X,\varSigma,\mu)}$の元の列$\left( f_{n} \right)_{n \in \mathbb{N}}$、$E \in \varSigma$なる集合$E$が与えられたとき、その集合$E$で$0 \leq f_{n}$が成り立つかつ、その集合$E$で正の実数$y$が$y \leq \liminf_{n \rightarrow \infty}f_{n}$を満たすなら、次式が成り立つ。
\begin{align*}
y\mu(E) \leq \liminf_{n \rightarrow \infty}{\int_{E} {f_{n}\mu}}
\end{align*}
\end{thm*}
\begin{thm*}[定理\ref{4.6.1.24}の再掲]
測度空間$(X,\varSigma,\mu)$と$0 \leq s \in \mathcal{S}(X,\varSigma)$なる単関数$s$、集合$\mathcal{M}_{(X,\varSigma,\mu)}^{+}$の元の列$\left( f_{n} \right)_{n \in \mathbb{N}}$が与えられたとき、$s \leq \liminf_{n \rightarrow \infty}f_{n}$を満たすなら、次式が成り立つ。
\begin{align*}
\int_{X} {s\mu} \leq \liminf_{n \rightarrow \infty}{\int_{X} {f_{n}\mu}}
\end{align*}
\end{thm*}
\begin{thm*}[Fatouの補題\ref{4.6.1.25}の再掲]
測度空間$(X,\varSigma,\mu)$と集合$\mathcal{M}_{(X,\varSigma,\mu)}$の元の列$\left\{ f_{n} \right\}_{n \in \mathbb{N}}$、$E \in \varSigma$なる集合$E$が与えられたとき、その集合$E$で$0 \leq f_{n}$が成り立つなら、次式が成り立つ。
\begin{align*}
\int_{E} {\liminf_{n \rightarrow \infty}f_{n}\mu} \leq \liminf_{n \rightarrow \infty}{\int_{E} {f_{n}\mu}}
\end{align*}
この定理をFatouの補題という。
\end{thm*}
\begin{thm*}[単調収束定理\ref{4.6.1.26}の再掲]
測度空間$(X,\varSigma,\mu)$と集合$\mathcal{M}_{(X,\varSigma,\mu)}$の元の列$\left( f_{n} \right)_{n \in \mathbb{N}}$、$E \in \varSigma$なる集合$E$が与えられたとき、その集合$E$で$0 \leq f_{n}$が成り立つかつ、その元の列$\left( f_{n} \right)_{n \in \mathbb{N}}$がその集合$E$で単調増加するなら、次式が成り立つ。
\begin{align*}
\lim_{n \rightarrow \infty}{\int_{E} {f_{n}\mu}} = \int_{E} {\sup\left\{ f_{n} \right\}_{n \in \mathbb{N}}\mu}
\end{align*}
この定理を単調収束定理という。
\end{thm*}
\begin{thm*}[定理\ref{4.6.1.27}の再掲]
測度空間$(X,\varSigma,\mu)$と$f \in \mathcal{M}_{(X,\varSigma,\mu)}$なる写像$f$、その$\sigma$-加法族$\varSigma$の元の列$\left( A_{n} \right)_{n \in \mathbb{N}}$が与えられたとき、次のことが成り立つ。
\begin{itemize}
\item
  その写像$f$がその集合$\bigcup_{n \in \mathbb{N}} A_{n}$で定積分可能な、または、$f \in \mathcal{M}_{(X,\varSigma,\mu)}^{+}$が成り立つとき、その元の列$\left( A_{n} \right)_{n \in \mathbb{N}}$が単調増加するなら、次式が成り立つ。
\begin{align*}
\lim_{n \rightarrow \infty}{\int_{A_{n}} {f\mu}} = \int_{\bigcup_{n \in \mathbb{N}} A_{n}} {f\mu}
\end{align*}
\item
  その写像$f$がその集合$A_{1}$で定積分可能であるとき、その元の列$\left( A_{n} \right)_{n \in \mathbb{N}}$が単調減少するなら、次式が成り立つ。
\begin{align*}
\lim_{n \rightarrow \infty}{\int_{A_{n}} {f\mu}} = \int_{\bigcap_{n \in \mathbb{N}} A_{n}} {f\mu}
\end{align*}
\end{itemize}
\end{thm*}
\begin{thm*}[定理\ref{4.6.1.28}の再掲]
測度空間$(X,\varSigma,\mu)$と$f \in \mathcal{M}_{(X,\varSigma,\mu)}^{+}$なる写像$f$が与えられたとき、非負可測関数の非負単関数の列による近似におけるその集合$\mathcal{M}_{(X,\varSigma,\mu)}^{+}$の元の列$\left( (f)_{n} \right)_{n \in \mathbb{N}}$は次式を満たす。
\begin{align*}
\int_{E} {f\mu} = \int_{E} {\sup\left\{ (f)_{n} \right\}_{n \in \mathbb{N}}\mu} = \lim_{n \rightarrow \infty}{\int_{E} {(f)_{n}\mu}}
\end{align*}
\end{thm*}
%\hypertarget{lebesgueux306eux53ceux675fux5b9aux7406}{%
\subsubsection{Lebesgueの収束定理}%\label{lebesgueux306eux53ceux675fux5b9aux7406}}
\begin{thm}[項別積分]\label{4.6.2.1}
測度空間$(X,\varSigma,\mu)$と集合$\mathcal{M}_{(X,\varSigma,\mu)}^{+}$の元の列々$\left( f_{n} \right)_{n \in \mathbb{N}}$が与えられたとき、写像$\sum_{n \in \mathbb{N}} f_{n}$は可測で、$\forall E \in \varSigma$に対し、次式が成り立つ。
\begin{align*}
\sum_{n \in \mathbb{N}} {\int_{E} {f_{n}\mu}} = \int_{E} {\sum_{n \in \mathbb{N}} f_{n}\mu}
\end{align*}
この定理を項別積分という。
\end{thm}
\begin{proof}
測度空間$(X,\varSigma,\mu)$と集合$\mathcal{M}_{(X,\varSigma,\mu)}^{+}$の元の列々$\left( f_{n} \right)_{n \in \mathbb{N}}$が与えられたとき、写像$\sum_{k \in \varLambda_{n}} f_{k}$は可測であり、さらに、写像$\lim_{n \rightarrow \infty}{\sum_{k \in \varLambda_{n}} f_{k}}$も可測である。したがって、写像$\sum_{n \in \mathbb{N}} f_{n}$は可測である。このとき、$\forall E \in \varSigma$に対し、元の列$\left( \sum_{k \in \varLambda_{n}} f_{k} \right)_{n \in \mathbb{N}}$がその集合$E$で単調増加するので、単調収束定理より次のようになる。
\begin{align*}
\sum_{n \in \mathbb{N}} {\int_{E} {f_{n}\mu}} &= \lim_{n \rightarrow \infty}{\sum_{k \in \varLambda_{n}} {\int_{E} {f_{k}\mu}}}\\
&= \lim_{n \rightarrow \infty}{\int_{E} {\sum_{k \in \varLambda_{n}} f_{k}\mu}}\\
&= \int_{E} {\sup\left\{ \sum_{k \in \varLambda_{n}} f_{k} \right\}_{n \in \mathbb{N}}\mu}\\
&= \int_{E} {\lim_{n \rightarrow \infty}{\sum_{k \in \varLambda_{n}} f_{k}}\mu}\\
&= \int_{E} {\sum_{n \in \mathbb{N}} f_{n}\mu}
\end{align*}
\end{proof}
\begin{thm}\label{4.6.2.2}
測度空間$(X,\varSigma,\mu)$と$f \in \mathcal{M}_{(X,\varSigma,\mu)}^{+}$なる写像$f$、$E \in \varSigma$なる集合$E$が与えられたとき、$\left\{ \chi_{E}f = \infty \right\} = \emptyset$が成り立つなら、次式たちが成り立つ\footnote{これはLebesgue積分といったら水平方向のsliceによる近似という有名な話(?)を定式化したものです。(しかし、Lebesgue積分は定義式からみれば、sliceしていくってよりかは隙間を埋めていくという心象に近いような気がしますが…\ \_\ ;)}。
\begin{align*}
\lim_{n \rightarrow \infty}{\sum_{k \in \mathbb{N}} {\frac{k}{2^{n}}\mu\left( \left\{ \frac{k}{2^{n}} \leq \chi_{E}f < \frac{k + 1}{2^{n}} \right\} \right)}} = \int_{E} {f\mu}
\end{align*}
\end{thm}
\begin{proof}
測度空間$(X,\varSigma,\mu)$と$f \in \mathcal{M}_{(X,\varSigma,\mu)}^{+}$なる写像$f$、$E \in \varSigma$なる集合$E$が与えられたとき、$\left\{ \chi_{E}f = \infty \right\} = \emptyset$が成り立つなら、集合$\left\{ \frac{k}{2^{n}} \leq \chi_{E}f < \frac{k + 1}{2^{n}} \right\}$が$A_{n,k}$とおかれると、項別積分より次のようになる。
\begin{align*}
\lim_{n \rightarrow \infty}{\sum_{k \in \mathbb{N}} {\frac{k}{2^{n}}\mu\left( \left\{ \frac{k}{2^{n}} \leq \chi_{E}f < \frac{k + 1}{2^{n}} \right\} \right)}} &= \lim_{n \rightarrow \infty}{\sum_{k \in \mathbb{N}} {\frac{k}{2^{n}}\mu\left( A_{n,k} \cap E \right)}}\\
&= \lim_{n \rightarrow \infty}{\sum_{k \in \mathbb{N}} {\frac{k}{2^{n}}\int_{E} {\chi_{A_{n,k} \cap E}\mu}}}\\
&= \lim_{n \rightarrow \infty}{\sum_{k \in \mathbb{N}} {\int_{X} {\chi_{E}\frac{k}{2^{n}}\chi_{A_{n,k}}\mu}}}\\
&= \lim_{n \rightarrow \infty}{\int_{X} {\chi_{E}\sum_{k \in \mathbb{N}} {\frac{k}{2^{n}}\chi_{A_{n,k}}}\mu}}
\end{align*}
ここで、元の列$\left( \sum_{k \in \mathbb{N}} {\frac{k}{2^{n}}\chi_{A_{n,k}}} \right)_{n \in \mathbb{N}}$はその集合$\mathcal{M}_{(X,\varSigma,\mu)}^{+}$の単調増加する元の列で、その集合$E$では$\forall x \in E\forall\varepsilon \in \mathbb{R}^{+}$に対し、自然数$n$が十分大きくなれば、$x \in A_{n,k'}$なる自然数$k'$が存在して次のようになるので、
\begin{align*}
f(x) &\leq \left( \sum_{k \in \mathbb{N}} {\frac{k}{2^{n}}\chi_{A_{n,k}}} \right)(x) + \varepsilon\\
&= \sum_{k \in \mathbb{N}} {\frac{k}{2^{n}}\chi_{A_{n,k}}(x)} + \varepsilon\\
&= \sum_{k \in \mathbb{N} \setminus \left\{ k' \right\}} {\frac{k}{2^{n}}\chi_{A_{n,k}}(x)} + \frac{k'}{2^{n}}\chi_{A_{n,k'}}(x) + \varepsilon\\
&= \sum_{k \in \mathbb{N} \setminus \left\{ k' \right\}} {\frac{k}{2^{n}} \cdot 0} + \frac{k'}{2^{n}} \cdot 1 + \varepsilon\\
&= \frac{k'}{2^{n}} + \varepsilon
\end{align*}
次式が成り立ち
\begin{align*}
0 \leq \left( \chi_{E}f \right)(x) - \left( \sum_{k \in \mathbb{N}} {\frac{k}{2^{n}}\chi_{A_{n,k}}} \right)(x) \leq \varepsilon
\end{align*}
これにより、その集合$E$で$\lim_{n \rightarrow \infty}{\sum_{k \in \mathbb{N}} {\frac{k}{2^{n}}\chi_{A_{n,k}}}} = f$が成り立つことになる\footnote{非負可測関数の非負単関数の列による近似の証明も参考にすれば分かりやすいかも…。}。したがって、単調収束定理より次のようになる。
\begin{align*}
\lim_{n \rightarrow \infty}{\sum_{k \in \mathbb{N}} {\frac{k}{2^{n}}\mu\left( \left\{ \frac{k}{2^{n}} \leq \chi_{E}f < \frac{k + 1}{2^{n}} \right\} \right)}} &= \int_{X} {\chi_{E}\sup\left\{ \sum_{k \in \mathbb{N}} {\frac{k}{2^{n}}\chi_{A_{n,k}}} \right\}_{n \in \mathbb{N}}\mu}\\
&= \int_{X} {\chi_{E}\lim_{n \rightarrow \infty}{\sum_{k \in \mathbb{N}} {\frac{k}{2^{n}}\chi_{A_{n,k}}}}\mu}\\
&= \int_{X} {\chi_{E}f\mu} = \int_{E} {f\mu}
\end{align*}
\end{proof}
\begin{thm}\label{4.6.2.3}
測度空間$(X,\varSigma,\mu)$と$f \in \mathcal{M}_{(X,\varSigma,\mu)}^{+}$なる写像$f$が与えられたとき、次のことが成り立つ。
\begin{itemize}
\item
  次式のように定義される写像$\nu_{f}$は測度で測度空間$\left( X,\varSigma,\nu_{f} \right)$を与える。
\begin{align*}
\nu_{f}:\varSigma \rightarrow{}^{*}\mathbb{R};E \mapsto \int_{E} {f\mu}
\end{align*}
\item
  $g \in \mathcal{M}_{(X,\varSigma,\mu)}^{+}$なる写像$g$が与えられたとき、次式が成り立つ。
\begin{align*}
\int_{X} {g\nu_{f}} = \int_{X} {fg\mu}
\end{align*}
\end{itemize}
\end{thm}
\begin{proof}
測度空間$(X,\varSigma,\mu)$と$f \in \mathcal{M}_{(X,\varSigma,\mu)}^{+}$なる写像$f$が与えられたとき、次式のように定義される写像$\nu_{f}$について、$\forall E \in \varSigma$に対し、積分の単調性より$0 \leq \nu_{f}(E)$が成り立つことになる。また、次のようになる。
\begin{align*}
\nu_{f}(\emptyset) = \int_{\emptyset} {f\mu} = \int_{X} {\chi_{\emptyset}f\mu} = \int_{X} {0f\mu} = 0
\end{align*}
さらに、その$\sigma$-加法族$\varSigma$の互いに素な元の列$\left( E_{n} \right)_{n \in \mathbb{N}}$が与えられたとき、項別積分より次のようになる。
\begin{align*}
\nu_{f}\left( \bigsqcup_{n \in \mathbb{N}} E_{n} \right) &= \int_{\bigsqcup_{n \in \mathbb{N}} E_{n}} {f\mu}\\
&= \int_{X} {\chi_{\bigsqcup_{n \in \mathbb{N}} E_{n}}f\mu}\\
&= \int_{X} {\sum_{n \in \mathbb{N}} \chi_{E_{n}}f\mu}\\
&= \sum_{n \in \mathbb{N}} {\int_{X} {\chi_{E_{n}}f\mu}}\\
&= \sum_{n \in \mathbb{N}} {\int_{E_{n}} {f\mu}}\\
&= \sum_{n \in \mathbb{N}} {\nu_{f}\left( E_{n} \right)}
\end{align*}
以上より、その写像$\nu_{f}$は測度であり測度空間$\left( X,\varSigma,\nu_{f} \right)$を与える。\par
$g \in \mathcal{M}_{(X,\varSigma,\mu)}^{+}$なる写像$g$が与えられたとき、非負可測関数の非負単関数の列による近似よりその集合$\mathcal{M}_{(X,\varSigma,\mu)}^{+}$の元の列$\left( (g)_{m} \right)_{m \in \mathbb{N}}$が存在してこれが単調増加し$g = \sup\left\{ (g)_{m} \right\}_{m \in \mathbb{N}}$を満たす。ここで、$(g)_{m} = \sum_{i \in \varLambda_{n}} {a_{m,i}\chi_{E_{i}}}$とおかれると、次のようになる。
\begin{align*}
\int_{X} {(g)_{m}\nu_{f}} &= \sum_{y \in V\left( (g)_{m} \right)} {\int_{\left\{ (g)_{m} = y \right\}} {(g)_{m}\nu_{f}}}\\
&= \sum_{y \in V\left( (g)_{m} \right)} {\int_{X} {\chi_{\left\{ (g)_{m} = y \right\}}\sum_{i \in \varLambda_{n}} {a_{m,i}\chi_{E_{i}}}\nu_{f}}}\\
&= \sum_{y \in V\left( (g)_{m} \right)} {\sum_{i \in \varLambda_{n}} {a_{m,i}\nu_{f}\left( E_{i} \cap \left\{ (g)_{m} = y \right\} \right)}}\\
&= \sum_{y \in V\left( (g)_{m} \right)} {y\nu_{f}\left( \left\{ (g)_{m} = y \right\} \right)}\\
&= \sum_{y \in V\left( (g)_{m} \right)} {y\int_{\left\{ (g)_{m} = y \right\}} {f\mu}}\\
&= \sum_{y \in V\left( (g)_{m} \right)} {y\int_{X} {\chi_{\left\{ (g)_{m} = y \right\}}f\mu}}\\
&= \int_{X} {\sum_{y \in V\left( (g)_{m} \right)} {y\chi_{\left\{ (g)_{m} = y \right\}}f}\mu}\\
&= \int_{X} {\sum_{a_{in} \in V\left( (g)_{m} \right)} {a_{m,i}\chi_{\left\{ (g)_{m} = a_{m,i} \right\}}f}\mu}\\
&= \int_{X} {f\sum_{i \in \varLambda_{n}} {a_{m,i}\chi_{E_{i}}}\mu}\\
&= \int_{X} {f(g)_{m}\mu}
\end{align*}
単調収束定理より次のようになる。
\begin{align*}
\int_{X} {g\nu_{f}} &= \int_{X} {\sup\left\{ (g)_{m} \right\}_{m \in \mathbb{N}}\nu_{f}}\\
&= \lim_{m \rightarrow \infty}{\int_{X} {(g)_{m}\nu_{f}}}\\
&= \lim_{m \rightarrow \infty}{\int_{X} {f(g)_{m}\mu}}\\
&= \int_{X} {\sup\left\{ f(g)_{m} \right\}_{m \in \mathbb{N}}\mu}\\
&= \int_{X} {f\sup\left\{ (g)_{m} \right\}_{m \in \mathbb{N}}\mu}\\
&= \int_{X} {fg\mu}
\end{align*}
\end{proof}
\begin{thm}[Lebesgue-Fatouの補題]\label{4.6.2.4}
測度空間$(X,\varSigma,\mu)$と集合$\mathcal{M}_{(X,\varSigma,\mu)}$の元の列々$\left( f_{n} \right)_{n \in \mathbb{N}}$、$g \in \mathcal{M}_{(X,\varSigma,\mu)}$なる写像$g$、$E \in \varSigma$なる集合$E$が与えられたとき、次式が成り立つかつ、
\begin{align*}
\liminf_{n \rightarrow \infty}{\int_{E} {f_{n}\mu}} < \infty
\end{align*}
その集合$E$で$g \leq f_{n}$が成り立つとする。このとき、写像$\liminf_{n \rightarrow \infty}f_{n}$はその集合$E$で定積分可能で次式が成り立つ。
\begin{align*}
\int_{E} {\liminf_{n \rightarrow \infty}f_{n}\mu} \leq \liminf_{n \rightarrow \infty}{\int_{E} {f_{n}\mu}}
\end{align*}
この定理をLebesgue-Fatouの補題という。
\end{thm}
\begin{proof}
測度空間$(X,\varSigma,\mu)$と集合$\mathcal{M}_{(X,\varSigma,\mu)}$の元の列々$\left( f_{n} \right)_{n \in \mathbb{N}}$、$g \in \mathcal{M}_{(X,\varSigma,\mu)}$なる写像$g$、$E \in \varSigma$なる集合$E$が与えられたとき、次式が成り立つかつ、
\begin{align*}
\liminf_{n \rightarrow \infty}{\int_{E} {f_{n}\mu}} < \infty
\end{align*}
その集合$E$で$g \leq f_{n}$が成り立つとする。定理\ref{4.6.1.19}よりその写像$g$がその集合$E$上で定積分可能であるので、次式が成り立つ。
\begin{align*}
\mu\left( \left\{ \left| g|E \right| = \infty \right\} \right) = 0
\end{align*}
ここで、集合$E \setminus \left\{ \left| g|E \right| = \infty \right\}$で和$f_{n} - g$が定義されることができ、さらに、$f_{n} - g \in \mathcal{M}_{(X,\varSigma,\mu)}^{+}$が成り立つ。したがって、定理\ref{4.6.1.21}、Fatouの補題より次のようになる。
\begin{align*}
\int_{E} {\liminf_{n \rightarrow \infty}f_{n}\mu} &= \int_{E \setminus \left\{ \left| g|E \right| = \infty \right\}} {\liminf_{n \rightarrow \infty}f_{n}\mu}\\
&= \int_{E \setminus \left\{ \left| g|E \right| = \infty \right\}} {\liminf_{n \rightarrow \infty}f_{n}\mu} - \int_{E \setminus \left\{ \left| g|E \right| = \infty \right\}} {g\mu} + \int_{E \setminus \left\{ \left| g|E \right| = \infty \right\}} {g\mu}\\
&= \int_{E \setminus \left\{ \left| g|E \right| = \infty \right\}} {\left( \liminf_{n \rightarrow \infty}f_{n} - g \right)\mu} + \int_{E \setminus \left\{ \left| g|E \right| = \infty \right\}} {g\mu}\\
&= \int_{E \setminus \left\{ \left| g|E \right| = \infty \right\}} {\liminf_{n \rightarrow \infty}\left( f_{n} - g \right)\mu} + \int_{E \setminus \left\{ \left| g|E \right| = \infty \right\}} {g\mu}\\
&\leq \liminf_{n \rightarrow \infty}{\int_{E \setminus \left\{ \left| g|E \right| = \infty \right\}} {\left( f_{n} - g \right)\mu}} + \int_{E \setminus \left\{ \left| g|E \right| = \infty \right\}} {g\mu}\\
&= \liminf_{n \rightarrow \infty}{\int_{E \setminus \left\{ \left| g|E \right| = \infty \right\}} {f_{n}\mu}} - \liminf_{n \rightarrow \infty}{\int_{E \setminus \left\{ \left| g|E \right| = \infty \right\}} {g\mu}} + \int_{E \setminus \left\{ \left| g|E \right| = \infty \right\}} {g\mu}\\
&= \liminf_{n \rightarrow \infty}{\int_{E \setminus \left\{ \left| g|E \right| = \infty \right\}} {f_{n}\mu}} - \int_{E \setminus \left\{ \left| g|E \right| = \infty \right\}} {g\mu} + \int_{E \setminus \left\{ \left| g|E \right| = \infty \right\}} {g\mu}\\
&= \liminf_{n \rightarrow \infty}{\int_{E \setminus \left\{ \left| g|E \right| = \infty \right\}} {f_{n}\mu}}\\
&= \liminf_{n \rightarrow \infty}{\int_{E} {f_{n}\mu}}
\end{align*}
よって、定理\ref{4.6.1.21}よりその写像$\liminf_{n \rightarrow \infty}f_{n}$はその集合$E$で定積分可能で次式が成り立つ。
\begin{align*}
\int_{E} {\liminf_{n \rightarrow \infty}f_{n}\mu} \leq \liminf_{n \rightarrow \infty}{\int_{E} {f_{n}\mu}}
\end{align*}
\end{proof}
\begin{thm}[Lebesgueの優収束定理]\label{4.6.2.5}
測度空間$(X,\varSigma,\mu)$と集合$\mathcal{M}_{(X,\varSigma,\mu)}$の元の列$\left( f_{n} \right)_{n \in \mathbb{N}}$、$g \in \mathcal{M}_{(X,\varSigma,\mu)}$なる写像$g$、$E \in \varSigma$なる集合$E$が与えられたとき、その集合$E$で$\lim_{n \rightarrow \infty}f_{n} \in \mathrm{cl}\mathbb{R}$と収束しており、その写像$g$はその集合$E$で定積分可能であるかつ、その集合$E$で$\left| f_{n} \right| \leq g$が成り立つとする。このとき、写像たち$f_{n}$、$\lim_{n \rightarrow \infty}f_{n}$はその集合$E$で定積分可能で次式が成り立つ。
\begin{align*}
\int_{E} {\lim_{n \rightarrow \infty}f_{n}\mu} = \lim_{n \rightarrow \infty}{\int_{E} {f_{n}\mu}}
\end{align*}
この定理をLebesgueの優収束定理といいこの写像$g$をその元の列$\left( f_{n} \right)_{n \in \mathbb{N}}$の可積分優関数という。
\end{thm}
\begin{proof}
測度空間$(X,\varSigma,\mu)$と集合$\mathcal{M}_{(X,\varSigma,\mu)}$の元の列$\left( f_{n} \right)_{n \in \mathbb{N}}$、$g \in \mathcal{M}_{(X,\varSigma,\mu)}$なる写像$g$、$E \in \varSigma$なる集合$E$が与えられたとき、その集合$E$で$\lim_{n \rightarrow \infty}f_{n} \in \mathrm{cl}\mathbb{R}$と収束しており、その写像$g$はその集合$E$で定積分可能であるかつ、その集合$E$で$\left| f_{n} \right| \leq g$が成り立つとする。定理\ref{4.6.1.22}より集合$E$で$\left| f_{n} \right| \leq g$が成り立つかつ、その写像$g$が定積分可能であるので、その写像$f_{n}$はその集合$E$で定積分可能で、さらに、その写像$\lim_{n \rightarrow \infty}f_{n}$もその集合$E$で定積分可能である。このとき、$- g \leq f_{n} \leq g$が、即ち、$- g \leq f_{n}$かつ$- g \leq - f_{n}$がその集合$E$で成り立っているので、積分の単調性より次のようになる。
\begin{align*}
\int_{E} {( - g)\mu} \leq \int_{E} {\left( - f_{n} \right)\mu},\ \ \int_{E} {( - g)\mu} \leq \int_{E} {g\mu}
\end{align*}
このとき、Lebesgue-Fatouの補題より写像たち$\liminf_{n \rightarrow \infty}\left( - f_{n} \right)$、$\liminf_{n \rightarrow \infty}f_{n}$はその集合$E$で定積分可能で次式が成り立つ。
\begin{align*}
\int_{E} {\liminf_{n \rightarrow \infty}\left( - f_{n} \right)\mu} \leq \liminf_{n \rightarrow \infty}{\int_{E} {\left( - f_{n} \right)\mu}},\ \ \int_{E} {\liminf_{n \rightarrow \infty}f_{n}\mu} \leq \liminf_{n \rightarrow \infty}{\int_{E} {f_{n}\mu}}
\end{align*}
したがって、次のようになる。
\begin{align*}
\int_{E} {\lim_{n \rightarrow \infty}f_{n}\mu} &= \int_{E} {\liminf_{n \rightarrow \infty}f_{n}\mu}\\
&\leq \liminf_{n \rightarrow \infty}{\int_{E} {f_{n}\mu}}\\
&\leq \limsup_{n \rightarrow \infty}{\int_{E} {f_{n}\mu}}\\
&= - \liminf_{n \rightarrow \infty}\left( - \int_{E} {f_{n}\mu} \right)\\
&= - \liminf_{n \rightarrow \infty}{\int_{E} {\left( - f_{n} \right)\mu}}\\
&\leq - \int_{E} {\liminf_{n \rightarrow \infty}\left( - f_{n} \right)\mu}\\
&= - \int_{E} {\left( - \limsup_{n \rightarrow \infty}f_{n} \right)\mu}\\
&= \int_{E} {\limsup_{n \rightarrow \infty}f_{n}\mu}\\
&= \int_{E} {\lim_{n \rightarrow \infty}f_{n}\mu}
\end{align*}
よって、次式が成り立つ。
\begin{align*}
\int_{E} {\lim_{n \rightarrow \infty}f_{n}\mu} = \lim_{n \rightarrow \infty}{\int_{E} {f_{n}\mu}}
\end{align*}
\end{proof}
\begin{thm}\label{4.6.2.6}
測度空間$(X,\varSigma,\mu)$と集合$\mathcal{M}_{(X,\varSigma,\mu)}^{+}$の単調減少する元の列$\left( f_{n} \right)_{n \in \mathbb{N}}$、$E \in \varSigma$なる集合$E$が与えられたとき、その写像$f_{1}$がその集合$E$で定積分可能であるなら、写像たち$f_{n}$、$\lim_{n \rightarrow \infty}f_{n}$はその集合$E$で定積分可能で次式が成り立つ。
\begin{align*}
\int_{E} {\lim_{n \rightarrow \infty}f_{n}\mu} = \lim_{n \rightarrow \infty}{\int_{E} {f_{n}\mu}}
\end{align*}
\end{thm}
\begin{proof}
測度空間$(X,\varSigma,\mu)$と集合$\mathcal{M}_{(X,\varSigma,\mu)}^{+}$の単調減少する元の列$\left( f_{n} \right)_{n \in \mathbb{N}}$、$E \in \varSigma$なる集合$E$が与えられたとき、その写像$f_{1}$がその集合$E$で定積分可能であるなら、その元の列$\left( f_{n} \right)_{n \in \mathbb{N}}$は定理\ref{4.1.4.1}よりその写像$\inf\left\{ f_{n} \right\}_{n \in \mathbb{N}}$に収束する。このとき、その集合$E$で$\forall n \in \mathbb{N}$に対し、$\left| f_{n} \right| \leq f_{1}$が成り立つので、Lebesgueの優収束定理よりそれらの写像たち$f_{n}$、$\lim_{n \rightarrow \infty}f_{n}$はその集合$E$で定積分可能で次式が成り立つ。
\begin{align*}
\int_{E} {\lim_{n \rightarrow \infty}f_{n}\mu} = \lim_{n \rightarrow \infty}{\int_{E} {f_{n}\mu}}
\end{align*}
\end{proof}
\begin{thm}[Lebesgueの有界収束定理]\label{4.6.2.7}
測度空間$(X,\varSigma,\mu)$と集合$\mathcal{M}_{(X,\varSigma,\mu)}$の元の列$\left( f_{n} \right)_{n \in \mathbb{N}}$、$E \in \varSigma$かつ$\mu(E) < \infty$なる集合$E$が与えられたとき、その集合$E$で$\lim_{n \rightarrow \infty}f_{n} \in{}^{*}\mathbb{R}$と収束しているかつ、その集合$E$で$\exists M \in \mathbb{R}^{+}$に対し、$\left| f_{n} \right| \leq M$が成り立つとする。このとき、写像たち$f_{n}$、$\lim_{n \rightarrow \infty}f_{n}$はその集合$E$で定積分可能で次式が成り立つ。
\begin{align*}
\int_{E} {\lim_{n \rightarrow \infty}f_{n}\mu} = \lim_{n \rightarrow \infty}{\int_{E} {f_{n}\mu}}
\end{align*}
この定理をLebesgueの有界収束定理という。
\end{thm}
\begin{proof}
測度空間$(X,\varSigma,\mu)$と集合$\mathcal{M}_{(X,\varSigma,\mu)}$の元の列$\left( f_{n} \right)_{n \in \mathbb{N}}$、$E \in \varSigma$かつ$\mu(E) < \infty$なる集合$E$が与えられたとき、その集合$E$で$\lim_{n \rightarrow \infty}f_{n} \in{}^{*}\mathbb{R}$と収束しているかつ、その集合$E$で$\exists M \in \mathbb{R}^{+}$に対し、$\left| f_{n} \right| \leq M$が成り立つとする。このとき、次式が成り立つことから、
\begin{align*}
\int_{E} {M\mu} = M\int_{E} \mu = M\mu(E) < \infty
\end{align*}
その定数写像$M$はその集合$E$で定積分可能であるので、Lebesgueの優収束定理よりそれらの写像たち$f_{n}$、$\lim_{n \rightarrow \infty}f_{n}$はその集合$E$で定積分可能で次式が成り立つ。
\begin{align*}
\int_{E} {\lim_{n \rightarrow \infty}f_{n}\mu} = \lim_{n \rightarrow \infty}{\int_{E} {f_{n}\mu}}
\end{align*}
\end{proof}
\begin{thm}\label{4.6.2.8}
測度空間$(X,\varSigma,\mu)$と集合$\mathcal{M}_{(X,\varSigma,\mu)}$の定積分可能な元の列$\left( f_{n} \right)_{n \in \mathbb{N}}$、$E \in \varSigma$かつ$\mu(E) < \infty$なる集合$E$が与えられたとき、その集合$E$で$\lim_{n \rightarrow \infty}f_{n} \in{}^{*}\mathbb{R}$と一様収束している、即ち、$\forall\varepsilon \in \mathbb{R}^{+}\exists N \in \mathbb{N}\forall n \in \mathbb{N}$に対し、$N \leq n$が成り立つなら、$\left| f_{n} - \lim_{n \rightarrow \infty}f_{n} \right| < \varepsilon$が成り立っているとき、写像たち$f_{n}$、$\lim_{n \rightarrow \infty}f_{n}$はその集合$E$で定積分可能で次式が成り立つ。
\begin{align*}
\int_{E} {\lim_{n \rightarrow \infty}f_{n}\mu} = \lim_{n \rightarrow \infty}{\int_{E} {f_{n}\mu}}
\end{align*}
\end{thm}
\begin{proof}
測度空間$(X,\varSigma,\mu)$と集合$\mathcal{M}_{(X,\varSigma,\mu)}$の元の列$\left( f_{n} \right)_{n \in \mathbb{N}}$、$E \in \varSigma$かつ$\mu(E) < \infty$なる集合$E$が与えられたとき、その集合$E$で$\lim_{n \rightarrow \infty}f_{n} \in{}^{*}\mathbb{R}$と一様収束している、即ち、$\forall\varepsilon \in \mathbb{R}^{+}\exists N \in \mathbb{N}\forall n \in \mathbb{N}$に対し、$N \leq n$が成り立つなら、$\left| f_{n} - \lim_{n \rightarrow \infty}f_{n} \right| < \varepsilon$が成り立っているとき、次のようになる。
\begin{align*}
\left\{ \begin{matrix}
\left| f_{N} - \lim_{n \rightarrow \infty}f_{n} \right| < \varepsilon \\
\left| f_{n} - \lim_{n \rightarrow \infty}f_{n} \right| < \varepsilon \\
\end{matrix} \right. &\Rightarrow \left| f_{n} - \lim_{n \rightarrow \infty}f_{n} \right| + \left| f_{N} - \lim_{n \rightarrow \infty}f_{n} \right| < 2\varepsilon\\
&\Leftrightarrow \left| f_{n} - \lim_{n \rightarrow \infty}f_{n} \right| + \left| \lim_{n \rightarrow \infty}f_{n} - f_{N} \right| < 2\varepsilon\\
&\Leftrightarrow \left| f_{n} \right| - \left| f_{N} \right| \leq \left| f_{n} - f_{N} \right| \leq \left| f_{n} - \lim_{n \rightarrow \infty}f_{n} \right| + \left| \lim_{n \rightarrow \infty}f_{n} - f_{N} \right| < 2\varepsilon\\
&\Leftrightarrow \left| f_{n} \right| \leq \left| f_{n} - f_{N} \right| + \left| f_{N} \right| \leq \left| f_{n} - \lim_{n \rightarrow \infty}f_{n} \right| \\
&\quad + \left| \lim_{n \rightarrow \infty}f_{n} - f_{N} \right| + \left| f_{N} \right| < \left| f_{N} \right| + 2\varepsilon
\end{align*}
ここで、その元の列$\left( f_{n} \right)_{n \in \mathbb{N} \setminus \varLambda_{N}}$はその集合$\mathcal{M}_{(X,\varSigma,\mu)}$の元の列であり、もちろん、これもその集合$E$で$\lim_{n \rightarrow \infty}f_{n} \in{}^{*}\mathbb{R}$と収束する。ここで、仮定より$\int_{E} {f_{N}\mu} < \infty$が成り立つので、Lebesgueの優収束定理よりそれらの写像たち$f_{n}$、$\lim_{n \rightarrow \infty}f_{n}$はその集合$E$で定積分可能で次式が成り立つ。
\begin{align*}
\int_{E} {\lim_{n \rightarrow \infty}f_{n}\mu} = \lim_{n \rightarrow \infty}{\int_{E} {f_{n}\mu}}
\end{align*}
\end{proof}
%\hypertarget{ux9805ux5225ux7a4dux5206}{%
\subsubsection{項別積分}%\label{ux9805ux5225ux7a4dux5206}}\par
項別積分を保証する定理は次のように与えられるのであった。
\begin{thm*}[定理\ref{4.6.2.1}の再掲]
測度空間$(X,\varSigma,\mu)$と集合$\mathcal{M}_{(X,\varSigma,\mu)}^{+}$の元の列々$\left( f_{n} \right)_{n \in \mathbb{N}}$が与えられたとき、写像$\sum_{n \in \mathbb{N}} f_{n}$は可測で、$\forall E \in \varSigma$に対し、次式が成り立つ。
\begin{align*}
\sum_{n \in \mathbb{N}} {\int_{E} {f_{n}\mu}} = \int_{E} {\sum_{n \in \mathbb{N}} f_{n}\mu}
\end{align*}
\end{thm*}
\begin{thm}\label{4.6.2.9}
測度空間$(X,\varSigma,\mu)$が与えられたとき、$\forall E \in \varSigma\forall f \in \mathcal{M}_{(X,\varSigma,\mu)}$に対し、その集合$E$上で$f \geq 0$が成り立つなら、次式が成り立つようなその集合$\mathcal{M}_{(X,\varSigma,\mu)}^{+}$の単関数の列$\left( s_{n} \right)_{n \in \mathbb{N}}$が存在する。
\begin{align*}
f = \sum_{n \in \mathbb{N}} s_{n},\ \ \sum_{n \in \mathbb{N}} {\int_{E} {s_{n}\mu}} = \int_{E} {\sum_{n \in \mathbb{N}} s_{n}\mu}
\end{align*}
\end{thm}
\begin{proof}
測度空間$(X,\varSigma,\mu)$が与えられたとき、$\forall E \in \varSigma\forall f \in \mathcal{M}_{(X,\varSigma,\mu)}$に対し、その集合$E$上で$f \geq 0$が成り立つなら、非負可測関数の非負単関数の列による近似よりある元の列$\left( (f)_{n} \right)_{n \in \mathbb{N}}$が存在して、定理\ref{4.6.1.28}より次式が成り立つ。
\begin{align*}
\int_{E} {f\mu} = \int_{E} {\sup\left\{ (f)_{n} \right\}_{n \in \mathbb{N}}\mu} = \lim_{n \rightarrow \infty}{\int_{E} {(f)_{n}\mu}}
\end{align*}
ここで、$s_{1} = (f)_{1}$かつ$s_{n + 1} = (f)_{n + 1} - (f)_{n}$が成り立つような元の列$\left( s_{n} \right)_{n \in \mathbb{N}}$が考えられれば、非負可測関数の非負単関数の列による近似より次のようになる。
\begin{align*}
\sum_{n \in \mathbb{N}} s_{n} &= \lim_{n \rightarrow \infty}{\sum_{i \in \varLambda_{n}} s_{i}}\\
&= \lim_{n \rightarrow \infty}\left( s_{1} + \sum_{i \in \varLambda_{n} \setminus \left\{ 1 \right\}} s_{i} \right)\\
&= \lim_{n \rightarrow \infty}\left( s_{1} + \sum_{i \in \varLambda_{n - 1}} s_{i + 1} \right)\\
&= \lim_{n \rightarrow \infty}\left( (f)_{1} + \sum_{i \in \varLambda_{n - 1}} \left( (f)_{i + 1} - (f)_{i} \right) \right)\\
&= \lim_{n \rightarrow \infty}\left( (f)_{1} + \sum_{i \in \varLambda_{n - 1}} (f)_{i + 1} - \sum_{i \in \varLambda_{n - 1}} (f)_{i} \right)\\
&= \lim_{n \rightarrow \infty}\left( (f)_{1} + (f)_{n} + \sum_{i \in \varLambda_{n - 1}} (f)_{i + 1} - \sum_{i \in \varLambda_{n} \setminus \left\{ 1 \right\}} (f)_{i} - (f)_{1} \right)\\
&= \lim_{n \rightarrow \infty}\left( (f)_{n} + \sum_{i \in \varLambda_{n - 1}} (f)_{i + 1} - \sum_{i \in \varLambda_{n - 1}} (f)_{i + 1} \right)\\
&= \lim_{n \rightarrow \infty}(f)_{n}\\
&= \sup\left\{ (f)_{n} \right\}_{n \in \mathbb{N}} = f
\end{align*}
また、明らかにその元の列$\left( (f)_{n} \right)_{n \in \mathbb{N}}$はその集合$\mathcal{M}_{(X,\varSigma,\mu)}^{+}$の元の列であるから、定理\ref{4.6.2.1}より次式が成り立つ。
\begin{align*}
\sum_{n \in \mathbb{N}} {\int_{E} {s_{n}\mu}} = \int_{E} {\sum_{n \in \mathbb{N}} s_{n}\mu}
\end{align*}
\end{proof}
\begin{thm}\label{4.6.2.10}
測度空間$(X,\varSigma,\mu)$と、$\forall E \in \varSigma$に対し、集合$\mathcal{M}_{(X,\varSigma,\mu)}^{+}$のその集合$E$で単調増加する元の列$\left( f_{n} \right)_{n \in \mathbb{N}}$が与えられたとき、写像$\lim_{n \rightarrow \infty}f_{n}$は次式を満たす。
\begin{align*}
\int_{E} {\lim_{n \rightarrow \infty}f_{n}\mu} = \lim_{n \rightarrow \infty}{\int_{E} {f_{n}\mu}}
\end{align*}
\end{thm}
\begin{proof}
測度空間$(X,\varSigma,\mu)$と、$\forall E \in \varSigma$に対し、集合$\mathcal{M}_{(X,\varSigma,\mu)}^{+}$のその集合$E$で単調増加する元の列$\left( f_{n} \right)_{n \in \mathbb{N}}$が与えられたとき、$\exists n \in \mathbb{N}$に対し、$\mu\left( \left\{ f_{n} = \infty \right\} \right) > 0$が成り立つなら、定理\ref{4.6.1.14}と定理\ref{4.6.1.19}より次式が成り立つ。
\begin{align*}
\int_{E} {\lim_{n \rightarrow \infty}f_{n}\mu} = \lim_{n \rightarrow \infty}{\int_{E} {f_{n}\mu}} = \infty
\end{align*}
したがって、$\forall n \in \mathbb{N}$に対し、$\mu\left( \left\{ f_{n} = \infty \right\} \right) = 0$が成り立つなら、$\bigcup_{n \in \mathbb{N}} \left\{ f_{n} = \infty \right\} \in \varSigma$が成り立つので、$\mu\left( \bigcup_{n \in \mathbb{N}} \left\{ f_{n} = \infty \right\} \right) = 0$が成り立つ。したがって、定理\ref{4.6.1.20}より$\forall f \in \mathfrak{L}$に対し、次式が成り立つので、
\begin{align*}
\int_{\bigcup_{n \in \mathbb{N}} \left\{ f_{n} = \infty \right\}} {f\mu} = 0
\end{align*}
$\forall n \in \mathbb{N}$に対し、$f_{n} < \infty$が成り立つと仮定してもよい。\par
ここで、$g_{1} = f_{1}$かつ$g_{n + 1} = f_{n + 1} - f_{n}$が成り立つような元の列$\left( g_{n} \right)_{n \in \mathbb{N}}$が考えられれば、$0 \leq g_{n}$で次式が成り立つ。
\begin{align*}
\sum_{i \in \varLambda_{n}} g_{i} &= g_{1} + \sum_{i \in \varLambda_{n} \setminus \left\{ 1 \right\}} g_{i} = g_{1} + \sum_{i \in \varLambda_{n - 1}} g_{i + 1}\\
&= f_{1} + \sum_{i \in \varLambda_{n - 1}} \left( f_{i + 1} - f_{i} \right)\\
&= f_{1} + \sum_{i \in \varLambda_{n - 1}} f_{i + 1} - \sum_{i \in \varLambda_{n - 1}} f_{i}\\
&= f_{1} + f_{n} + \sum_{i \in \varLambda_{n - 2}} f_{i + 1} - \sum_{i \in \varLambda_{n - 1} \setminus \left\{ 1 \right\}} f_{i} - f_{1}\\
&= f_{n} + \sum_{i \in \varLambda_{n - 2}} f_{i + 1} - \sum_{i \in \varLambda_{n - 2}} f_{i + 1} = f_{n}
\end{align*}
したがって、項別積分より次のようになる。
\begin{align*}
\int_{E} {\lim_{n \rightarrow \infty}f_{n}\mu} &= \int_{E} {\lim_{n \rightarrow \infty}{\sum_{i \in \varLambda_{n}} g_{i}}\mu} = \int_{E} {\sum_{n \in \mathbb{N}} g_{n}\mu} = \sum_{n \in \mathbb{N}} {\int_{E} {g_{n}\mu}}\\
&= \lim_{n \rightarrow \infty}{\sum_{i \in \varLambda_{n}} {\int_{E} {g_{i}\mu}}} = \lim_{n \rightarrow \infty}{\int_{E} {\sum_{i \in \varLambda_{n}} g_{i}\mu}} = \lim_{n \rightarrow \infty}{\int_{E} {f_{n}\mu}}
\end{align*}
\end{proof}
\begin{thm}\label{4.6.2.11}
測度空間$(X,\varSigma,\mu)$と、$\forall E \in \varSigma$に対し、集合$\mathcal{M}_{(X,\varSigma,\mu)}$のその集合$E$で単調増加する元の列$\left( f_{n} \right)_{n \in \mathbb{N}}$が与えられたとき、次式が成り立つなら、
\begin{align*}
- \infty < \int_{E} {f_{1}\mu}
\end{align*}
写像$\lim_{n \rightarrow \infty}f_{n}$は次式を満たす。
\begin{align*}
\int_{E} {\lim_{n \rightarrow \infty}f_{n}\mu} = \lim_{n \rightarrow \infty}{\int_{E} {f_{n}\mu}}
\end{align*}
\end{thm}
\begin{proof}
測度空間$(X,\varSigma,\mu)$と、$\forall E \in \varSigma$に対し、集合$\mathcal{M}_{(X,\varSigma,\mu)}$のその集合$E$で単調増加する元の列$\left( f_{n} \right)_{n \in \mathbb{N}}$が与えられたとき、次式が成り立つなら、
\begin{align*}
- \infty < \int_{E} {f_{1}\mu}
\end{align*}
積分の単調性と対偶律よりほとんどいたるところで$- \infty < f_{1}$が成り立つ。あとは、定理\ref{4.6.1.20}に気をつければ、定理\ref{4.6.2.10}と同様にして示される。
\end{proof}
%\hypertarget{ux7a4dux5206ux533aux9593ux306eux6975ux9650}{%
\subsubsection{積分区間の極限}%\label{ux7a4dux5206ux533aux9593ux306eux6975ux9650}}
\begin{thm}\label{4.6.2.12}
測度空間$(X,\varSigma,\mu)$と集合$\mathcal{M}_{(X,\varSigma,\mu)}$の写像$f$、$E \in \varSigma$なる集合$E$が与えられたとき、その写像$f$がその集合$E$で定積分をもつなら、$E = \bigsqcup_{n \in \mathbb{N}} E_{n}$とおかれたとき、次式が成り立つ。
\begin{align*}
\int_{E} {f\mu} = \int_{\bigsqcup_{n \in \mathbb{N}} E_{n}} {f\mu} = \sum_{n \in \mathbb{N}} {\int_{E_{n}} {f\mu}}
\end{align*}
\end{thm}
\begin{proof}
測度空間$(X,\varSigma,\mu)$と集合$\mathcal{M}_{(X,\varSigma,\mu)}$の写像$f$、$E \in \varSigma$なる集合$E$が与えられたとき、その写像$f$がその集合$E$で定積分をもつなら、$E = \bigsqcup_{n \in \mathbb{N}} E_{n}$とおかれたとき、$0\infty = 0$と約束しておくと、$\forall n \in \mathbb{N}$に対し、$\chi_{E_{n}}(f)_{-},\chi_{E_{n}}(f)_{+} \in \mathcal{M}_{(X,\varSigma,\mu)}$が成り立つので、項別積分、定理\ref{4.1.6.7}、定理\ref{4.6.1.9}より次のようになる。
\begin{align*}
\int_{E} {f\mu} &= \int_{E} {\left( (f)_{+} - (f)_{-} \right)\mu} = \int_{E} {(f)_{+}\mu} - \int_{E} {(f)_{-}\mu}\\
&= \int_{E} {\sum_{n \in \mathbb{N}} {\chi_{E_{n}}(f)_{+}}\mu} - \int_{E} {\sum_{n \in \mathbb{N}} {\chi_{E_{n}}(f)_{-}}\mu}\\
&= \sum_{n \in \mathbb{N}} {\int_{E} {\chi_{E_{n}}(f)_{+}\mu}} - \sum_{n \in \mathbb{N}} {\int_{E} {\chi_{E_{n}}(f)_{-}\mu}}\\
&= \sum_{n \in \mathbb{N}} {\int_{E_{n}} {(f)_{+}\mu}} - \sum_{n \in \mathbb{N}} {\int_{E_{n}} {(f)_{-}\mu}}\\
&= \sum_{n \in \mathbb{N}} \left( \int_{E_{n}} {(f)_{+}\mu} - \int_{E_{n}} {(f)_{-}\mu} \right)\\
&= \sum_{n \in \mathbb{N}} {\int_{E_{n}} {f\mu}}
\end{align*}
\end{proof}
%\hypertarget{ux5faeux5206ux3068ux7a4dux5206ux306eux9806ux5e8fux4ea4ux63db}{%
\subsubsection{微分と積分の順序交換}%\label{ux5faeux5206ux3068ux7a4dux5206ux306eux9806ux5e8fux4ea4ux63db}}
\begin{thm}\label{4.6.2.13}
測度空間$(X,\varSigma,\mu)$、$E \in \varSigma$なる集合$E$が与えられたとき、$D(f) = D^{*} \times (a,b) \supseteq E \times (a,b)$なる写像$f:D(f) \rightarrow{}^{*}\mathbb{R};(x,t) \mapsto f(x,t)$について、$f^{*}:D^{*} \rightarrow{}^{*}\mathbb{R};x \mapsto f(x,t)$、$f_{*}:(a,b) \rightarrow{}^{*}\mathbb{R};t \mapsto f(x,t)$とすれば、その写像$f^{*}$がその集合$E$上で積分可能でその写像$f_{*}$がその開区間$(a,b)$で微分可能で、最後の成分を$t$成分ということにすれば、その集合$E$上で積分可能な写像$\varphi$が存在して、その定義域$D(f)$上で$\left| \partial_{t}f \right| \leq \varphi$が成り立つなら、その積分$\int_{E} {f^{*}\mu}$は、これがその変数$t$の関数とみなされたとき、$\left( \partial_{t}f \right)^{*}:D^{*} \rightarrow{}^{*}\mathbb{R};x \mapsto \partial f(x,t)$とすれば、微分可能で次式が成り立つ。
\begin{align*}
\partial\int_{E} {f^{*}\mu} = \int_{E} {\left( \partial_{t}f \right)^{*}\mu}
\end{align*}
\end{thm}
\begin{proof}
測度空間$(X,\varSigma,\mu)$、$E \in \varSigma$なる集合$E$が与えられたとき、$D(f) = D^{*} \times (a,b) \supseteq E \times (a,b)$なる写像$f:D(f) \rightarrow{}^{*}\mathbb{R};(x,t) \mapsto f(x,t)$について、$f^{*}:D^{*} \rightarrow{}^{*}\mathbb{R};x \mapsto f(x,t)$、$f_{*}:(a,b) \rightarrow{}^{*}\mathbb{R};t \mapsto f(x,t)$とすれば、その写像$f^{*}$がその集合$E$上で積分可能でその写像$f_{*}$がその開区間$(a,b)$で微分可能で、最後の成分を$t$成分ということにすれば、その集合$E$上で積分可能な写像$\varphi$が存在して、その定義域$D(f)$上で$\left| \partial_{t}f \right| \leq \varphi$が成り立つとする。$\forall t \in (a,b)$に対し、仮定より$\lim_{n \rightarrow \infty}h_{n} = 0$なる実数列$\left( h_{n} \right)_{n \in \mathbb{N}}$を用いれば、次式が成り立つので、
\begin{align*}
\lim_{h \rightarrow 0}\frac{f(x,t + h) - f(x,t)}{h} = \lim_{n \rightarrow \infty}\frac{f\left( x,t + h_{n} \right) - f(x,t)}{h_{n}} = \partial_{t}f(x,t)
\end{align*}
その写像$D^{*} \rightarrow{}^{*}\mathbb{R};x \mapsto \frac{f\left( x,t + h_{n} \right) - f(x,t)}{h_{n}}$は明らかに可測で、定理\ref{4.5.5.11}より$\left( \partial_{t}f \right)^{*}:D^{*} \rightarrow{}^{*}\mathbb{R};x \mapsto \partial f(x,t)$とすれば、その写像$\left( \partial_{t}f \right)^{*}$も可測である。そこで、平均値の定理よりその実数$t$のある近傍$I$と$0 < \theta < 1$なる正の実数$\theta$が存在して、$t + h_{n} \in I$かつ$h_{n} \neq 0$なる実数$h_{n}$がとられれば、次式が成り立つ。
\begin{align*}
\frac{f\left( x,t + h_{n} \right) - f(x,t)}{h_{n}} = \partial_{t}f\left( x,t + \theta h_{n} \right)
\end{align*}
仮定より次式が成り立つので、
\begin{align*}
\left| \frac{f\left( x,t + h_{n} \right) - f(x,t)}{h_{n}} \right| = \left| \partial_{t}f\left( x,t + \theta h_{n} \right) \right| \leq \varphi(x)
\end{align*}
定理\ref{4.6.2.5}、即ち、Lebesgueの優収束定理よりそれらの写像たち$D^{*} \rightarrow{}^{*}\mathbb{R};x \mapsto \frac{f\left( x,t + h_{n} \right) - f(x,t)}{h_{n}}$、$\left( \partial_{t}f \right)^{*}$はその集合$E$で定積分可能で、次のようにすれば、
\begin{align*}
\int_{E} {f^{*}\mu}:(a,b) \rightarrow{}^{*}\mathbb{R};t \mapsto \int_{E} {D^{*} \rightarrow{}^{*}\mathbb{R};x \mapsto f(x,t)\mu}
\end{align*}
次式が成り立つ。
\begin{align*}
\int_{E} {\left( \partial_{t}f \right)^{*}\mu} &= \lim_{n \rightarrow \infty}{\int_{E} {D^{*} \rightarrow{}^{*}\mathbb{R};x \mapsto \frac{f\left( x,t + h_{n} \right) - f(x,t)}{h_{n}}\mu}}\\
&= \lim_{n \rightarrow \infty}{\frac{1}{h_{n}}\left( \int_{E} {D^{*} \rightarrow{}^{*}\mathbb{R};x \mapsto f\left( x,t + h_{n} \right)\mu} + \int_{E} {D^{*} \rightarrow{}^{*}\mathbb{R};x \mapsto f(x,t)\mu} \right)}\\
&= \lim_{h \rightarrow 0}{\frac{1}{h}\left( \int_{E} {D^{*} \rightarrow{}^{*}\mathbb{R};x \mapsto f(x,t + h)\mu} + \int_{E} {D^{*} \rightarrow{}^{*}\mathbb{R};x \mapsto f(x,t)\mu} \right)}\\
&= \lim_{h \rightarrow 0}{\frac{1}{h}\left( \int_{E} {f^{*}\mu}(t + h) + \int_{E} {f^{*}\mu}(t) \right)}\\
&= \lim_{h \rightarrow 0}\frac{\int_{E} {f^{*}\mu}(t + h) + \int_{E} {f^{*}\mu}(t)}{h}\\
&= \partial\int_{E} {f^{*}\mu}
\end{align*}
\end{proof}
\begin{thebibliography}{50}
\bibitem{1}
  伊藤清三, ルベーグ積分入門, 裳華房, 1963. 新装第1版2刷 p90-97 ISBN978-4-7853-1318-0
\bibitem{2}
  伊藤清三, ルベーグ積分入門, 裳華房, 1963. 第24版 p95-96 ISBN4-7853-1304-8
\bibitem{3}
  岩田耕一郎, ルベーグ積分, 森北出版, 2015. 第1版第2刷 p40-43 ISBN978-4-627-05431-8
\end{thebibliography}
\end{document}
