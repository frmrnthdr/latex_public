\documentclass[dvipdfmx]{jsarticle}
\setcounter{section}{6}
\setcounter{subsection}{2}
\usepackage{xr}
\externaldocument{4.5.3}
\externaldocument{4.5.4}
\externaldocument{4.5.5}
\usepackage{amsmath,amsfonts,amssymb,array,comment,mathtools,url,docmute}
\usepackage{longtable,booktabs,dcolumn,tabularx,mathtools,multirow,colortbl,xcolor}
\usepackage[dvipdfmx]{graphics}
\usepackage{bmpsize}
\usepackage{amsthm}
\usepackage{enumitem}
\setlistdepth{20}
\renewlist{itemize}{itemize}{20}
\setlist[itemize]{label=•}
\renewlist{enumerate}{enumerate}{20}
\setlist[enumerate]{label=\arabic*.}
\setcounter{MaxMatrixCols}{20}
\setcounter{tocdepth}{3}
\newcommand{\rotin}{\text{\rotatebox[origin=c]{90}{$\in $}}}
\renewcommand{\thesection}{第\arabic{section}部}
\renewcommand{\thesubsection}{\arabic{section}.\arabic{subsection}}
\renewcommand{\thesubsubsection}{\arabic{section}.\arabic{subsection}.\arabic{subsubsection}}
\everymath{\displaystyle}
\allowdisplaybreaks[4]
\usepackage{vtable}
\theoremstyle{definition}
\newtheorem{thm}{定理}[subsection]
\newtheorem*{thm*}{定理}
\newtheorem{dfn}{定義}[subsection]
\newtheorem*{dfn*}{定義}
\newtheorem{axs}[dfn]{公理}
\newtheorem*{axs*}{公理}
\renewcommand{\headfont}{\bfseries}
\makeatletter
  \renewcommand{\section}{%
    \@startsection{section}{1}{\z@}%
    {\Cvs}{\Cvs}%
    {\normalfont\huge\headfont\raggedright}}
\makeatother
\makeatletter
  \renewcommand{\subsection}{%
    \@startsection{subsection}{2}{\z@}%
    {0.5\Cvs}{0.5\Cvs}%
    {\normalfont\LARGE\headfont\raggedright}}
\makeatother
\makeatletter
  \renewcommand{\subsubsection}{%
    \@startsection{subsubsection}{3}{\z@}%
    {0.4\Cvs}{0.4\Cvs}%
    {\normalfont\Large\headfont\raggedright}}
\makeatother
\makeatletter
\renewenvironment{proof}[1][\proofname]{\par
  \pushQED{\qed}%
  \normalfont \topsep6\p@\@plus6\p@\relax
  \trivlist
  \item\relax
  {
  #1\@addpunct{.}}\hspace\labelsep\ignorespaces
}{%
  \popQED\endtrivlist\@endpefalse
}
\makeatother
\renewcommand{\proofname}{\textbf{証明}}
\usepackage{tikz,graphics}
\usepackage[dvipdfmx]{hyperref}
\usepackage{pxjahyper}
\hypersetup{
 setpagesize=false,
 bookmarks=true,
 bookmarksdepth=tocdepth,
 bookmarksnumbered=true,
 colorlinks=false,
 pdftitle={},
 pdfsubject={},
 pdfauthor={},
 pdfkeywords={}}
\begin{document}
%\hypertarget{ux96f6ux96c6ux5408ux3068ux7a4dux5206}{%
\subsection{零集合と積分}%\label{ux96f6ux96c6ux5408ux3068ux7a4dux5206}}
%\hypertarget{xsigmamu---a.e.ux96c6ux5408}{%
\subsubsection{$(X,\varSigma,\mu) \ \text{-} \ \mathrm{a.e.}$集合}%\label{xsigmamu---a.e.ux96c6ux5408}}
\begin{dfn*}[定義\ref{ほとんどすべて}の再掲]
測度空間$(X,\varSigma,\mu)$が与えられたとき、$A \subseteq X$なる集合$A$の元$a$に関係した命題$p$があって、その集合$A$のある部分集合$A_{0}$を用いて、$a \in A_{0}$のとき、その命題$p$は偽で、$a \in A \setminus A_{0}$のとき、その命題$p$は真で、$A_{0} \subseteq A_{1} \in \varSigma$なる集合が存在して、$\mu\left( A_{1} \right) = 0$が成り立つとき、即ち、その集合$A_{1}$がその測度空間$(X,\varSigma,\mu)$で零集合であるとき、その集合$A$上でその測度$\mu$に関してほとんどいたるところの元$a$、または、ほとんどすべての元$a$に対し、その命題$p$は成り立つといい、$p\ (X,\varSigma,\mu) \ \text{-} \ \mathrm{a.e.}\ a \in A$とか$p\ (\varSigma,\mu) \ \text{-} \ \mathrm{a.e.}\ a \in A\ \mathrm{on}\ X$、$p\ (X,\varSigma,\mu) \ \text{-} \ \mathrm{a.a.}\ a \in A$、$p\ (\varSigma,\mu) \ \text{-} \ \mathrm{a.a.}\ a \in A\ \mathrm{on}\ X$などと書く。特に、$a \in A_{0} \subseteq A\ (X,\varSigma,\mu) \ \text{-} \ \mathrm{a.e.}a \in A$が成り立つような集合$A_{0}$をその集合$A$上の$(X,\varSigma,\mu) \ \text{-} \ \mathrm{a.e.}$集合という。
\end{dfn*}
\begin{thm}\label{4.6.3.1}
測度空間$(X,\varSigma,\mu)$が与えられたとき、$A \subseteq B \subseteq C \subseteq X$なる集合たち$A$、$B$、$C$について、その集合$A$がその集合$C$上の$(X,\varSigma,\mu) \ \text{-} \ \mathrm{a.e.}$集合であるなら、その集合$A$はその集合$B$上の$(X,\varSigma,\mu) \ \text{-} \ \mathrm{a.e.}$集合であるかつ、その集合$B$はその集合$C$上の$(X,\varSigma,\mu) \ \text{-} \ \mathrm{a.e.}$集合でもある。
\end{thm}
\begin{proof}
測度空間$(X,\varSigma,\mu)$が与えられたとき、$A \subseteq B \subseteq C \subseteq X$なる集合たち$A$、$B$、$C$について、その集合$A$がその集合$C$上の$(X,\varSigma,\mu) \ \text{-} \ \mathrm{a.e.}$集合であるなら、$C \setminus A \subseteq A' \in \varSigma$なる集合$A'$が存在して$\mu\left( A' \right) = 0$が成り立つ。このとき、$B \setminus A \subseteq C \setminus A \subseteq A'$が成り立つかつ、$C \setminus B \subseteq C \setminus A \subseteq A'$が成り立つので、その集合$A$はその集合$B$上の$(X,\varSigma,\mu) \ \text{-} \ \mathrm{a.e.}$集合であるかつ、その集合$B$はその集合$C$上の$(X,\varSigma,\mu) \ \text{-} \ \mathrm{a.e.}$集合でもある。
\end{proof}
\begin{thm}\label{4.6.3.2}
測度空間$(X,\varSigma,\mu)$が与えられたとき、$B \subseteq X$なる集合$B$について、次のことが成り立つ。
\begin{itemize}
\item
  その集合$B$上の$(X,\varSigma,\mu) \ \text{-} \ \mathrm{a.e.}$集合の列$\left\{ A_{n} \right\}_{n \in \mathbb{N}}$が与えられたとき、その積集合$\bigcap_{n \in \mathbb{N}} A_{n}$もその集合$B$上の$(X,\varSigma,\mu) \ \text{-} \ \mathrm{a.e.}$集合である。
\item
  $\forall n \in \mathbb{N}$に対し、その集合$A_{n + 1}$がその集合$A_{n}$上の$(X,\varSigma,\mu) \ \text{-} \ \mathrm{a.e.}$集合であるような集合の列$\left\{ A_{n} \right\}_{n \in \mathbb{N}}$が与えられたとき、その積集合$\bigcap_{n \in \mathbb{N}} A_{n}$もその集合$A_{1}$上の$(X,\varSigma,\mu) \ \text{-} \ \mathrm{a.e.}$集合である。
\item
  $\forall n \in \mathbb{N}$に対し、その集合$B$がその集合$A_{n}$の$(X,\varSigma,\mu) \ \text{-} \ \mathrm{a.e.}$集合であるなら、その集合$B$はその和集合$\bigcup_{n \in \mathbb{N}} A_{n}$上の$(X,\varSigma,\mu) \ \text{-} \ \mathrm{a.e.}$集合である。
\end{itemize}
\end{thm}
\begin{proof}
測度空間$(X,\varSigma,\mu)$が与えられたとき、$B \subseteq X$なる集合$B$について、その集合$B$上の$(X,\varSigma,\mu) \ \text{-} \ \mathrm{a.e.}$集合の列$\left\{ A_{n} \right\}_{n \in \mathbb{N}}$が与えられたとき、$B \setminus A_{n} \subseteq A_{n}' \in \varSigma$なる集合$A_{n}'$が存在して$\mu\left( A_{n}' \right) = 0$が成り立つ。このとき、$\bigcup_{n \in \mathbb{N}} A_{n}' \in \varSigma$が成り立つかつ、$\mu\left( \bigcup_{n \in \mathbb{N}} A_{n}' \right) \leq \sum_{n \in \mathbb{N}} {\mu\left( A_{n}' \right)} = 0$が成り立つので、その集合$\bigcup_{n \in \mathbb{N}} A_{n}'$も零集合で、このとき、$B \setminus \bigcap_{n \in \mathbb{N}} A_{n} = \bigcup_{n \in \mathbb{N}} \left( B \setminus A_{n} \right) \subseteq \bigcup_{n \in \mathbb{N}} A_{n}'$が成り立つので、その積集合$\bigcap_{n \in \mathbb{N}} A_{n}$もその集合$B$上の$(X,\varSigma,\mu) \ \text{-} \ \mathrm{a.e.}$集合である。\par
$\forall n \in \mathbb{N}$に対し、その集合$A_{n + 1}$がその集合$A_{n}$上の$(X,\varSigma,\mu) \ \text{-} \ \mathrm{a.e.}$集合であるような集合の列$\left\{ A_{n} \right\}_{n \in \mathbb{N}}$が与えられたとき、$A_{n} \setminus A_{n + 1} \subseteq A_{n}' \in \varSigma$なる集合$A_{n}'$が存在して$\mu\left( A_{n}' \right) = 0$が成り立つ。このとき、その列$\left\{ A_{n} \right\}_{n \in \mathbb{N}}$は単調減少しており、$\bigcup_{n \in \mathbb{N}} A_{n}' \in \varSigma$が成り立つかつ、$\mu\left( \bigcup_{n \in \mathbb{N}} A_{n}' \right) \leq \sum_{n \in \mathbb{N}} {\mu\left( A_{n}' \right)} = 0$が成り立つので、その集合$\bigcup_{n \in \mathbb{N}} A_{n}'$も零集合で、したがって、次のことが成り立つ。
\begin{align*}
A_{1} \setminus \bigcap_{n \in \mathbb{N}} A_{n} = \bigcup_{n \in \mathbb{N}} \left( A_{1} \setminus A_{n} \right) = \bigcup_{n \in \mathbb{N}} \left( A_{1} \setminus A_{n + 1} \right) = \bigcup_{n \in \mathbb{N}} \left( A_{n} \setminus A_{n + 1} \right) \subseteq \bigcup_{n \in \mathbb{N}} A_{n}'
\end{align*}
これにより、その積集合$\bigcap_{n \in \mathbb{N}} A_{n}$もその集合$A_{1}$上の$(X,\varSigma,\mu) \ \text{-} \ \mathrm{a.e.}$集合である。\par
$\forall n \in \mathbb{N}$に対し、その集合$B$がその集合$A_{n}$の$(X,\varSigma,\mu) \ \text{-} \ \mathrm{a.e.}$集合であるなら、$A_{n} \setminus B \subseteq A_{n}' \in \varSigma$なる集合$A_{n}'$が存在して$\mu\left( A_{n}' \right) = 0$が成り立つ。このとき、$\bigcup_{n \in \mathbb{N}} A_{n}' \in \varSigma$が成り立つかつ、$\mu\left( \bigcup_{n \in \mathbb{N}} A_{n}' \right) \leq \sum_{n \in \mathbb{N}} {\mu\left( A_{n}' \right)} = 0$が成り立つので、その集合$\bigcup_{n \in \mathbb{N}} A_{n}'$も零集合で、このとき、$\bigcup_{n \in \mathbb{N}} A_{n} \setminus B = \bigcup_{n \in \mathbb{N}} \left( A_{n} \setminus B \right) \subseteq \bigcup_{n \in \mathbb{N}} A_{n}'$が成り立つので、その集合$B$はその和集合$\bigcup_{n \in \mathbb{N}} A_{n}$上の$(X,\varSigma,\mu) \ \text{-} \ \mathrm{a.e.}$集合である。
\end{proof}
\begin{thm}\label{4.6.3.3}
測度空間$(X,\varSigma,\mu)$が与えられたとき、$C \subseteq X$かつ$D \subseteq X$なる集合たち$C$、$D$について、集合$A$がその集合$C$上の$(X,\varSigma,\mu) \ \text{-} \ \mathrm{a.e.}$集合であるかつ、その集合$D$上の$(X,\varSigma,\mu) \ \text{-} \ \mathrm{a.e.}$集合であるかつ、集合$B$がその集合$D$上の$(X,\varSigma,\mu) \ \text{-} \ \mathrm{a.e.}$集合であるなら、その積集合$A \cap B$はその和集合$C \cup D$上の$(X,\varSigma,\mu) \ \text{-} \ \mathrm{a.e.}$集合である。
\end{thm}
\begin{proof}
測度空間$(X,\varSigma,\mu)$が与えられたとき、$C \subseteq X$かつ$D \subseteq X$なる集合たち$C$、$D$について、集合$A$がその集合$C$上の$(X,\varSigma,\mu) \ \text{-} \ \mathrm{a.e.}$集合であるかつ、その集合$D$上の$(X,\varSigma,\mu) \ \text{-} \ \mathrm{a.e.}$集合であるかつ、集合$B$がその集合$D$上の$(X,\varSigma,\mu) \ \text{-} \ \mathrm{a.e.}$集合であるなら、定理\ref{4.6.3.2}よりその写像$A$はその和集合$C \cup D$上の$(X,\varSigma,\mu) \ \text{-} \ \mathrm{a.e.}$集合である。また、その集合$C$は定理\ref{4.6.3.1}よりその集合$C \cup D$上の$(X,\varSigma,\mu) \ \text{-} \ \mathrm{a.e.}$集合であるから、定理\ref{4.6.3.2}よりその集合$B$もその集合$C \cup D$上の$(X,\varSigma,\mu) \ \text{-} \ \mathrm{a.e.}$集合である。以上、定理\ref{4.6.3.2}よりその積集合$A \cap B$はその和集合$C \cup D$上の$(X,\varSigma,\mu) \ \text{-} \ \mathrm{a.e.}$集合である。
\end{proof}
\begin{thm}\label{4.6.3.4}
測度空間$(X,\varSigma,\mu)$が与えられたとき、$\forall f,g,h \in \mathcal{M}_{(X,\varSigma,\mu)}$に対し、次のことが成り立つ。
\begin{itemize}
\item
  $f(x) = g(x)\ (X,\varSigma,\mu) \ \text{-} \ \mathrm{a.e.}x \in X$かつ$g(x) = h(x)\ (X,\varSigma,\mu) \ \text{-} \ \mathrm{a.e.}x \in X$が成り立つなら、$f(x) = h(x)\ (X,\varSigma,\mu) \ \text{-} \ \mathrm{a.e.}x \in X$が成り立つ。
\item
  $f(x) \leq g(x)\ (X,\varSigma,\mu) \ \text{-} \ \mathrm{a.e.}x \in X$かつ$g(x) \leq h(x)\ (X,\varSigma,\mu) \ \text{-} \ \mathrm{a.e.}x \in X$が成り立つなら、$f(x) \leq h(x)\ (X,\varSigma,\mu) \ \text{-} \ \mathrm{a.e.}x \in X$が成り立つ。
\item
  $f(x) \leq g(x)\ (X,\varSigma,\mu) \ \text{-} \ \mathrm{a.e.}x \in X$かつ$g(x) \leq f(x)\ (X,\varSigma,\mu) \ \text{-} \ \mathrm{a.e.}x \in X$が成り立つなら、$f(x) = g(x)\ (X,\varSigma,\mu) \ \text{-} \ \mathrm{a.e.}x \in X$が成り立つ。
\end{itemize}
\end{thm}
\begin{proof}
$\forall f,g,h \in \mathcal{M}_{(X,\varSigma,\mu)}$に対し、$f(x) = g(x)\ (X,\varSigma,\mu) \ \text{-} \ \mathrm{a.e.}x \in X$かつ$g(x) = h(x)\ (X,\varSigma,\mu) \ \text{-} \ \mathrm{a.e.}x \in X$が成り立つなら、集合たち$\left\{ f = g \right\}$、$\left\{ g = h \right\}$がその集合$X$上の$(X,\varSigma,\mu) \ \text{-} \ \mathrm{a.e.}$集合である。このとき、定理\ref{4.6.3.3}よりこれらの積集合$\left\{ f = g \right\} \cap \left\{ g = h \right\}$もその集合$X$上の$(X,\varSigma,\mu) \ \text{-} \ \mathrm{a.e.}$集合であり、したがって、集合$\left\{ f = h \right\}$はその集合$X$上の$(X,\varSigma,\mu) \ \text{-} \ \mathrm{a.e.}$集合である。よって、$f(x) = h(x)\ (X,\varSigma,\mu) \ \text{-} \ \mathrm{a.e.}x \in X$が成り立つ。\par
同様にして、$f(x) \leq g(x)\ (X,\varSigma,\mu) \ \text{-} \ \mathrm{a.e.}x \in X$かつ$g(x) \leq h(x)\ (X,\varSigma,\mu) \ \text{-} \ \mathrm{a.e.}x \in X$が成り立つなら、$f(x) \leq h(x)\ (X,\varSigma,\mu) \ \text{-} \ \mathrm{a.e.}x \in X$が成り立つことと、$f(x) \leq g(x)\ (X,\varSigma,\mu) \ \text{-} \ \mathrm{a.e.}x \in X$かつ$g(x) \leq f(x)\ (X,\varSigma,\mu) \ \text{-} \ \mathrm{a.e.}x \in X$が成り立つなら、$f(x) = g(x)\ (X,\varSigma,\mu) \ \text{-} \ \mathrm{a.e.}x \in X$が成り立つことが示される。
\end{proof}
\begin{thm}\label{4.6.3.4s}
測度空間$(X,\varSigma,\mu)$が与えられたとき、関係$=\ (X,\varSigma,\mu) \ \text{-} \ \mathrm{a.e.}$が次のように定義されるとき、
\begin{align*}
&\left(=\ (X,\varSigma,\mu) \ \text{-} \ \mathrm{a.e.} \right) =\left( \mathcal{M}_{(X,\varSigma,\mu)}, \mathcal{M}_{(X,\varSigma,\mu)}, G \right) ,\\
&G = \left\{ \left(f,g\right) \in \mathcal{M}_{(X,\varSigma,\mu)} \middle| f(x) =g(x) \ (X,\varSigma,\mu) \ \text{-} \ \mathrm{a.e.}\ x \in X \right\}
\end{align*}
その関係$=\ (X,\varSigma,\mu) \ \text{-} \ \mathrm{a.e.}$は同値関係となる、即ち、次のことが成り立つ。
\begin{itemize}
\item
  $\forall f \in \mathcal{M}_{(X,\varSigma,\mu)}$に対し、$f = f\ (X,\varSigma,\mu) \ \text{-} \ \mathrm{a.e.}$が成り立つ。
\item
  $\forall f,g \in \mathcal{M}_{(X,\varSigma,\mu)}$に対し、$f = g\ (X,\varSigma,\mu) \ \text{-} \ \mathrm{a.e.}$が成り立つなら、$g = f\ (X,\varSigma,\mu) \ \text{-} \ \mathrm{a.e.}$が成り立つ。
\item
$\forall f,g,h \in \mathcal{M}_{(X,\varSigma,\mu)}$に対し、$f = g\ (X,\varSigma,\mu) \ \text{-} \ \mathrm{a.e.}$かつ$g = h\ (X,\varSigma,\mu) \ \text{-} \ \mathrm{a.e.}$が成り立つなら、$f = h\ (X,\varSigma,\mu) \ \text{-} \ \mathrm{a.e.}$が成り立つ。
\end{itemize}
\end{thm}
\begin{proof}
上から1つ目、2つ目の主張は定義より明らかである。上から3つ目の主張は定理\ref{4.6.3.4}から従う。
\end{proof}
%\hypertarget{ux96f6ux96c6ux5408ux3068ux53ceux675fux5b9aux7406}{%
\subsubsection{零集合と収束定理}%\label{ux96f6ux96c6ux5408ux3068ux53ceux675fux5b9aux7406}}
\begin{thm}\label{4.6.3.5}
測度空間$(X,\varSigma,\mu)$が与えられたとき、$\forall f,g \in \mathcal{M}_{(X,\varSigma,\mu)}$に対し、次のことが成り立つ。
\begin{itemize}
\item
  その写像$f$が定積分をもつかつ、$f(x) = g(x)\ (X,\varSigma,\mu) \ \text{-} \ \mathrm{a.e.}x \in X$が成り立つなら、その写像$g$も定積分をもち、さらに、次式が成り立つ。
\begin{align*}
\int_{X} {f\mu} = \int_{X} {g\mu}
\end{align*}
\item
  その写像$f$が定積分可能であるかつ、$f(x) = g(x)\ (X,\varSigma,\mu) \ \text{-} \ \mathrm{a.e.}x \in X$が成り立つなら、その写像$g$も定積分可能で、さらに、次式が成り立つ。
\begin{align*}
\int_{X} {f\mu} = \int_{X} {g\mu}
\end{align*}
\item
  その写像たち$f$、$g$が定積分をもつかつ、$f(x) \leq g(x)\ (X,\varSigma,\mu) \ \text{-} \ \mathrm{a.e.}x \in X$が成り立つなら、次式が成り立つ。
\begin{align*}
\int_{X} {f\mu} \leq \int_{X} {g\mu}
\end{align*}
\end{itemize}
\end{thm}
\begin{proof}
測度空間$(X,\varSigma,\mu)$が与えられたとき、$\forall f,g \in \mathcal{M}_{(X,\varSigma,\mu)}$に対し、$f(x) = g(x)\ (X,\varSigma,\mu) \ \text{-} \ \mathrm{a.e.}x \in X$が成り立つなら、集合$\left\{ f = g \right\}$はその集合$X$上で$(X,\varSigma,\mu) \ \text{-} \ \mathrm{a.e.}$集合であるから、定理\ref{4.5.4.9}より$\left\{ f = g \right\} \in \varSigma$が成り立つことにより、$\mu\left( \left\{ f \neq g \right\} \right) = \mu\left( X \setminus \left\{ f = g \right\} \right) = 0$が成り立つ。したがって、定理\ref{4.5.5.20}より次のようになる。
\begin{align*}
\int_{X} {f\mu} &= \int_{\left\{ f = g \right\} \sqcup \left\{ f \neq g \right\}} {f\mu}\\
&= \int_{\left\{ f = g \right\}} {f\mu} + \int_{\left\{ f \neq g \right\}} {f\mu}\\
&= \int_{\left\{ f = g \right\}} {f\mu}\\
&= \int_{\left\{ f = g \right\}} {g\mu}\\
&= \int_{\left\{ f = g \right\}} {g\mu} + \int_{\left\{ f \neq g \right\}} {g\mu}\\
&= \int_{\left\{ f = g \right\} \sqcup \left\{ f \neq g \right\}} {g\mu}\\
&= \int_{X} {g\mu}
\end{align*}
したがって、次のことが成り立つ。
\begin{itemize}
\item
  その写像$f$が定積分をもつかつ、$f(x) = g(x)\ (X,\varSigma,\mu) \ \text{-} \ \mathrm{a.e.}x \in X$が成り立つなら、その写像$g$も定積分をもち、さらに、次式が成り立つ。
\begin{align*}
\int_{X} {f\mu} = \int_{X} {g\mu}
\end{align*}
\item
  その写像$f$が定積分可能であるかつ、$f(x) = g(x)\ (X,\varSigma,\mu) \ \text{-} \ \mathrm{a.e.}x \in X$が成り立つなら、その写像$g$も定積分可能で、さらに、次式が成り立つ。
\begin{align*}
\int_{X} {f\mu} = \int_{X} {g\mu}
\end{align*}
\end{itemize}
その写像たち$f$、$g$が定積分をもつかつ、$f(x) \leq g(x)\ (X,\varSigma,\mu) \ \text{-} \ \mathrm{a.e.}x \in X$が成り立つなら、$\int_{X} {(g)_{+}\mu} = \infty$または$\int_{X} {(f)_{-}\mu} = \infty$のときでは明らかであるから、$\int_{X} {(g)_{+}\mu} < \infty$または$\int_{X} {(f)_{-}\mu} < \infty$のとき、集合$\left\{ f \leq g \right\}$は定理\ref{4.5.4.9}より$\left\{ f \leq g \right\} \in \varSigma$を満たすので、$\mu\left( \left\{ f > g \right\} \right) = \mu\left( X \setminus \left\{ f \leq g \right\} \right) = 0$が成り立つ。したがって、定理\ref{4.5.5.20}より次のようになる。
\begin{align*}
\int_{X} {f\mu} &= \int_{X} {(f)_{+}\mu} - \int_{X} {(f)_{-}\mu}\\
&= \int_{\left\{ f \leq g \right\} \sqcup \left\{ f > g \right\}} {(f)_{+}\mu} - \int_{\left\{ f \leq g \right\} \sqcup \left\{ f > g \right\}} {(f)_{-}\mu}\\
&= \int_{\left\{ f \leq g \right\}} {(f)_{+}\mu} + \int_{\left\{ f > g \right\}} {(f)_{+}\mu} - \int_{\left\{ f \leq g \right\}} {(f)_{-}\mu} - \int_{\left\{ f > g \right\}} {(f)_{-}\mu}\\
&= \int_{\left\{ f \leq g \right\}} {(f)_{+}\mu} - \int_{\left\{ f \leq g \right\}} {(f)_{-}\mu}
\end{align*}
ここで、$\int_{\left\{ f \leq g \right\}} {(f)_{+}\mu} \leq \int_{\left\{ f \leq g \right\}} {(g)_{+}\mu}$かつ$\int_{\left\{ f \leq g \right\}} {(g)_{-}\mu} \leq \int_{\left\{ f \leq g \right\}} {(f)_{-}\mu}$が成り立つので、次のようになる。
\begin{align*}
\int_{X} {f\mu} &\leq \int_{\left\{ f \leq g \right\}} {(g)_{+}\mu} - \int_{\left\{ f \leq g \right\}} {(g)_{-}\mu}\\
&= \int_{\left\{ f \leq g \right\}} {(g)_{+}\mu} + \int_{\left\{ f > g \right\}} {(g)_{+}\mu} - \int_{\left\{ f \leq g \right\}} {(g)_{-}\mu} - \int_{\left\{ f > g \right\}} {(g)_{-}\mu}\\
&= \int_{\left\{ f \leq g \right\} \sqcup \left\{ f > g \right\}} {(g)_{+}\mu} - \int_{\left\{ f \leq g \right\} \sqcup \left\{ f > g \right\}} {(g)_{-}\mu}\\
&= \int_{X} {(g)_{+}\mu} - \int_{X} {(g)_{-}\mu} = \int_{X} {g\mu}
\end{align*}
\end{proof}
\begin{thm}\label{4.6.3.6}
測度空間$(X,\varSigma,\mu)$が与えられたとき、$\forall f,g \in \mathcal{M}_{(X,\varSigma,\mu)}$に対し、いづれも定積分可能であるとき、$\forall A \in \varSigma$に対し、次式が成り立つなら、
\begin{align*}
\int_{A} {f\mu} \leq \int_{A} {g\mu}
\end{align*}
$f(x) \leq g(x)\ (X,\varSigma,\mu) \ \text{-} \ \mathrm{a.e.}x \in X$が成り立つ。
\end{thm}
\begin{proof}
測度空間$(X,\varSigma,\mu)$が与えられたとき、$\forall f,g \in \mathcal{M}_{(X,\varSigma,\mu)}$に対し、いづれも定積分可能であるとき、$\forall A \in \varSigma$に対し、次式が成り立つなら、
\begin{align*}
\int_{A} {f\mu} \leq \int_{A} {g\mu}
\end{align*}
定理\ref{4.5.5.19}より$\mu\left( \left\{ |g| = \infty \right\} \right) = 0$が成り立つので、集合$\left\{ |g| < \infty \right\}$はその集合$X$上の$(X,\varSigma,\mu) \ \text{-} \ \mathrm{a.e.}$集合である。ここで、仮定より$\left\{ f > g \right\} \in \varSigma$が成り立つことにより、次式が成り立つので、
\begin{align*}
0 \leq \int_{\left\{ f > g \right\}} {(f - g)\mu} = \int_{\left\{ f > g \right\}} {f\mu} - \int_{\left\{ f > g \right\}} {g\mu} \leq 0
\end{align*}
$\int_{\left\{ f > g \right\}} {(f - g)\mu} = 0$が成り立ち、定理\ref{4.5.5.20}より$0 \leq f - g$が成り立つことにより、$\mu\left( \left\{ f > g \right\} \right) = 0$が成り立つ。したがって、その集合$\left\{ f \leq g \right\}$はその集合$X$上で$(X,\varSigma,\mu) \ \text{-} \ \mathrm{a.e.}$集合であるから、$f(x) \leq g(x)\ (X,\varSigma,\mu) \ \text{-} \ \mathrm{a.e.}x \in X$が成り立つ。
\end{proof}
\begin{thm}\label{4.6.3.7}
測度空間$(X,\varSigma,\mu)$が与えられたとき、可測な写像$f_{n}:X \rightarrow \mathbb{R}$の列が次式を満たすとき、
\begin{align*}
\sum_{n \in \mathbb{N}} {\int_{X} {\left| f_{n} \right|\mu}} < \infty
\end{align*}
その級数$\sum_{n \in \varLambda_{n}} {\int_{X} {f_{n}\mu}}$は絶対収束し、次式が成り立つ。
\begin{align*}
\sum_{n \in \mathbb{N}} \left| f_{n}(x) \right| < \infty\ (X,\varSigma,\mu) \ \text{-} \ \mathrm{a.e.}x \in X
\end{align*}
さらに、可測な写像$g:X \rightarrow \mathbb{R}$が次式を満たすなら、
\begin{align*}
g(x) = \liminf_{n \rightarrow \infty}{\sum_{i \in \varLambda_{n}} {f_{i}(x)}}\ (X,\varSigma,\mu) \ \text{-} \ \mathrm{a.e.}x \in X
\end{align*}
その写像$g$も定積分可能であり次式が成り立つ。
\begin{align*}
\int_{X} {g\mu} = \sum_{n \in \mathbb{N}} {\int_{X} {f_{n}\mu}}
\end{align*}
\end{thm}
\begin{proof}
測度空間$(X,\varSigma,\mu)$が与えられたとき、可測な写像$f_{n}:X \rightarrow \mathbb{R}$の列が次式を満たすとき、
\begin{align*}
\sum_{n \in \mathbb{N}} {\int_{X} {\left| f_{n} \right|\mu}} < \infty
\end{align*}
項別積分と定理\ref{4.5.5.18}により次のようになり、
\begin{align*}
\sum_{n \in \mathbb{N}} \left| \int_{X} {f_{n}\mu} \right| \leq \sum_{n \in \mathbb{N}} {\int_{X} {\left| f_{n} \right|\mu}} = \int_{X} {\sum_{n \in \mathbb{N}} \left| f_{n} \right|\mu} < \infty
\end{align*}
したがって、その級数$\sum_{n \in \varLambda_{n}} {\int_{X} {f_{n}\mu}}$は絶対収束する。\par
このとき、定理\ref{4.5.5.19}より$\mu\left( \left\{ \sum_{n \in \mathbb{N}} \left| f_{n} \right| = \infty \right\} \right) = 0$が成り立つ。したがって、集合$\left\{ \sum_{n \in \mathbb{N}} \left| f_{n} \right| < \infty \right\}$はその集合$X$上の$(X,\varSigma,\mu) \ \text{-} \ \mathrm{a.e.}$集合であり、したがって、次式が成り立つ。
\begin{align*}
\sum_{n \in \mathbb{N}} \left| f_{n}(x) \right| < \infty\ (X,\varSigma,\mu) \ \text{-} \ \mathrm{a.e.}x \in X
\end{align*}\par
さらに、可測な写像$g:X \rightarrow \mathbb{R}$が次式を満たすなら、
\begin{align*}
g(x) = \liminf_{n \rightarrow \infty}{\sum_{i \in \varLambda_{n}} {f_{i}(x)}}\ (X,\varSigma,\mu) \ \text{-} \ \mathrm{a.e.}x \in X
\end{align*}
定理\ref{4.5.5.32}より写像$\sum_{n \in \mathbb{N}} \left| f_{n} \right|$は定積分可能で三角不等式より$\left| \sum_{n \in \mathbb{N}} f_{n} \right| \leq \sum_{n \in \mathbb{N}} \left| f_{n} \right|$が成り立つ。したがって、Lebesgueの優収束定理より次式が成り立つ。
\begin{align*}
\lim_{n \rightarrow \infty}{\int_{X} {\sum_{i \in \varLambda_{n}} f_{i}\mu}} &= \int_{X} {\lim_{n \rightarrow \infty}{\sum_{i \in \varLambda_{n}} f_{i}}\mu}\\
&= \int_{\left\{ g = \liminf_{n \rightarrow \infty}{\sum_{i \in \varLambda_{n}} f_{i}} \right\} \sqcup \left\{ g \neq \liminf_{n \rightarrow \infty}{\sum_{i \in \varLambda_{n}} f_{i}} \right\}} {\lim_{n \rightarrow \infty}{\sum_{i \in \varLambda_{n}} f_{i}}\mu}\\
&= \int_{\left\{ g = \liminf_{n \rightarrow \infty}{\sum_{i \in \varLambda_{n}} f_{i}} \right\}} {\lim_{n \rightarrow \infty}{\sum_{i \in \varLambda_{n}} f_{i}}\mu} + \int_{\left\{ g \neq \liminf_{n \rightarrow \infty}{\sum_{i \in \varLambda_{n}} f_{i}} \right\}} {\lim_{n \rightarrow \infty}{\sum_{i \in \varLambda_{n}} f_{i}}\mu}\\
&= \int_{\left\{ g = \liminf_{n \rightarrow \infty}{\sum_{i \in \varLambda_{n}} f_{i}} \right\}} {\lim_{n \rightarrow \infty}{\sum_{i \in \varLambda_{n}} f_{i}}\mu}\\
&= \int_{\left\{ g = \liminf_{n \rightarrow \infty}{\sum_{i \in \varLambda_{n}} f_{i}} \right\}} {\liminf_{n \rightarrow \infty}{\sum_{i \in \varLambda_{n}} f_{i}}\mu}\\
&= \int_{\left\{ g = \liminf_{n \rightarrow \infty}{\sum_{i \in \varLambda_{n}} f_{i}} \right\}} {g\mu}\\
&= \int_{\left\{ g = \liminf_{n \rightarrow \infty}{\sum_{i \in \varLambda_{n}} f_{i}} \right\}} {g\mu} + \int_{\left\{ g \neq \liminf_{n \rightarrow \infty}{\sum_{i \in \varLambda_{n}} f_{i}} \right\}} {g\mu}\\
&= \int_{\left\{ g = \liminf_{n \rightarrow \infty}{\sum_{i \in \varLambda_{n}} f_{i}} \right\} \sqcup \left\{ g \neq \liminf_{n \rightarrow \infty}{\sum_{i \in \varLambda_{n}} f_{i}} \right\}} {g\mu}\\
&= \int_{X} {g\mu}
\end{align*}
したがって、その写像$g$も定積分可能であり次のようになる。
\begin{align*}
\int_{X} {g\mu} = \lim_{n \rightarrow \infty}{\int_{X} {\sum_{i \in \varLambda_{n}} f_{i}\mu}} = \lim_{n \rightarrow \infty}{\sum_{i \in \varLambda_{n}} {\int_{X} {f_{i}\mu}}} = \sum_{n \in \mathbb{N}} {\int_{X} {f_{n}\mu}}
\end{align*}
\end{proof}
\begin{thebibliography}{50}
  \bibitem{1}
  岩田耕一郎, ルベーグ積分, 森北出版, 2015. 第1版第2刷 p35-42 ISBN978-4-627-05431-8
\end{thebibliography}
\end{document}
