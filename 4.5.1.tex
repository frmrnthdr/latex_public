\documentclass[dvipdfmx]{jsarticle}
\setcounter{section}{5}
\setcounter{subsection}{0}
\usepackage{amsmath,amsfonts,amssymb,array,comment,mathtools,url,docmute}
\usepackage{longtable,booktabs,dcolumn,tabularx,mathtools,multirow,colortbl,xcolor}
\usepackage[dvipdfmx]{graphics}
\usepackage{bmpsize}
\usepackage{amsthm}
\usepackage{enumitem}
\setlistdepth{20}
\renewlist{itemize}{itemize}{20}
\setlist[itemize]{label=•}
\renewlist{enumerate}{enumerate}{20}
\setlist[enumerate]{label=\arabic*.}
\setcounter{MaxMatrixCols}{20}
\setcounter{tocdepth}{3}
\newcommand{\rotin}{\text{\rotatebox[origin=c]{90}{$\in $}}}
\newcommand{\amap}[6]{\text{\raisebox{-0.7cm}{\begin{tikzpicture} 
  \node (a) at (0, 1) {$\textstyle{#2}$};
  \node (b) at (#6, 1) {$\textstyle{#3}$};
  \node (c) at (0, 0) {$\textstyle{#4}$};
  \node (d) at (#6, 0) {$\textstyle{#5}$};
  \node (x) at (0, 0.5) {$\rotin $};
  \node (x) at (#6, 0.5) {$\rotin $};
  \draw[->] (a) to node[xshift=0pt, yshift=7pt] {$\textstyle{\scriptstyle{#1}}$} (b);
  \draw[|->] (c) to node[xshift=0pt, yshift=7pt] {$\textstyle{\scriptstyle{#1}}$} (d);
\end{tikzpicture}}}}
\newcommand{\twomaps}[9]{\text{\raisebox{-0.7cm}{\begin{tikzpicture} 
  \node (a) at (0, 1) {$\textstyle{#3}$};
  \node (b) at (#9, 1) {$\textstyle{#4}$};
  \node (c) at (#9+#9, 1) {$\textstyle{#5}$};
  \node (d) at (0, 0) {$\textstyle{#6}$};
  \node (e) at (#9, 0) {$\textstyle{#7}$};
  \node (f) at (#9+#9, 0) {$\textstyle{#8}$};
  \node (x) at (0, 0.5) {$\rotin $};
  \node (x) at (#9, 0.5) {$\rotin $};
  \node (x) at (#9+#9, 0.5) {$\rotin $};
  \draw[->] (a) to node[xshift=0pt, yshift=7pt] {$\textstyle{\scriptstyle{#1}}$} (b);
  \draw[|->] (d) to node[xshift=0pt, yshift=7pt] {$\textstyle{\scriptstyle{#2}}$} (e);
  \draw[->] (b) to node[xshift=0pt, yshift=7pt] {$\textstyle{\scriptstyle{#1}}$} (c);
  \draw[|->] (e) to node[xshift=0pt, yshift=7pt] {$\textstyle{\scriptstyle{#2}}$} (f);
\end{tikzpicture}}}}
\renewcommand{\thesection}{第\arabic{section}部}
\renewcommand{\thesubsection}{\arabic{section}.\arabic{subsection}}
\renewcommand{\thesubsubsection}{\arabic{section}.\arabic{subsection}.\arabic{subsubsection}}
\everymath{\displaystyle}
\allowdisplaybreaks[4]
\usepackage{vtable}
\theoremstyle{definition}
\newtheorem{thm}{定理}[subsection]
\newtheorem*{thm*}{定理}
\newtheorem{dfn}{定義}[subsection]
\newtheorem*{dfn*}{定義}
\newtheorem{axs}[dfn]{公理}
\newtheorem*{axs*}{公理}
\renewcommand{\headfont}{\bfseries}
\makeatletter
  \renewcommand{\section}{%
    \@startsection{section}{1}{\z@}%
    {\Cvs}{\Cvs}%
    {\normalfont\huge\headfont\raggedright}}
\makeatother
\makeatletter
  \renewcommand{\subsection}{%
    \@startsection{subsection}{2}{\z@}%
    {0.5\Cvs}{0.5\Cvs}%
    {\normalfont\LARGE\headfont\raggedright}}
\makeatother
\makeatletter
  \renewcommand{\subsubsection}{%
    \@startsection{subsubsection}{3}{\z@}%
    {0.4\Cvs}{0.4\Cvs}%
    {\normalfont\Large\headfont\raggedright}}
\makeatother
\makeatletter
\renewenvironment{proof}[1][\proofname]{\par
  \pushQED{\qed}%
  \normalfont \topsep6\p@\@plus6\p@\relax
  \trivlist
  \item\relax
  {
  #1\@addpunct{.}}\hspace\labelsep\ignorespaces
}{%
  \popQED\endtrivlist\@endpefalse
}
\makeatother
\renewcommand{\proofname}{\textbf{証明}}
\usepackage{tikz,graphics}
\usepackage[dvipdfmx]{hyperref}
\usepackage{pxjahyper}
\hypersetup{
 setpagesize=false,
 bookmarks=true,
 bookmarksdepth=tocdepth,
 bookmarksnumbered=true,
 colorlinks=false,
 pdftitle={},
 pdfsubject={},
 pdfauthor={},
 pdfkeywords={}}
\begin{document}
%\hypertarget{ux96c6ux5408ux306eux6975ux9650}{%
\subsection{集合の極限}%\label{ux96c6ux5408ux306eux6975ux9650}}\par
%\hypertarget{ux96c6ux5408ux306eux6975ux9650-1}{%
\subsubsection{集合の極限}%\label{ux96c6ux5408ux306eux6975ux9650-1}}
\begin{dfn}
添数集合$\mathbb{N}$によって添数づけられた集合の列$\left( A_{n} \right)_{n \in \mathbb{N}}$が与えられたとき、次式のように極限たち$\limsup_{n \rightarrow \infty}A_{n}$、$\liminf_{n \rightarrow \infty}A_{n}$が定義される。
\begin{align*}
\limsup_{n \rightarrow \infty}A_{n} &= \bigcap_{n \in \mathbb{N}} {\bigcup_{k \in \mathbb{N} \setminus \varLambda_{n - 1}} A_{k}}\\
\liminf_{n \rightarrow \infty}A_{n} &= \bigcup_{n \in \mathbb{N}} {\bigcap_{k \in \mathbb{N} \setminus \varLambda_{n - 1}} A_{k}}
\end{align*}
このような極限たち$\limsup_{n \rightarrow \infty}A_{n}$、$\liminf_{n \rightarrow \infty}A_{n}$をそれぞれその族$\left( A_{n} \right)_{n \in \mathbb{N}}$の上極限集合、下極限集合という。
\end{dfn}\par
この意味合いとして、和集合、積集合をそれぞれ論理和、論理積とみなすことで、上極限集合$\limsup_{n \rightarrow \infty}A_{n}$の元$a$は任意の自然数$n$に対し、これ以上の自然数$k$が存在して$a \in A_{k}$を満たす、即ち、$a \in A_{k}$なる添数$k$がいくらでも大きくとられることができこのような集合$A_{k}$が無限に存在する、下極限集合$\liminf_{n \rightarrow \infty}A_{n}$の元$a$はある自然数$n$が存在してこれ以上の任意の自然数$k$に対し、$a \in A_{k}$を満たす、即ち、添数$k$が十分大きくなれば、$a \in A_{k}$が成り立ち$a \notin A_{k}$なる集合$A_{k}$がたかだか有限にしか存在しないという意味をもつことになる。
\begin{thm}\label{4.5.1.1}
添数集合$\mathbb{N}$によって添数づけられた集合の列$\left( A_{n} \right)_{n \in \mathbb{N}}$が与えられたとき、次式が成り立つ。
\begin{align*}
\liminf_{n \rightarrow \infty}A_{n} \subseteq \limsup_{n \rightarrow \infty}A_{n}
\end{align*}
\end{thm}
\begin{proof}
添数集合$\mathbb{N}$によって添数づけられた集合の列$\left( A_{n} \right)_{n \in \mathbb{N}}$が与えられたとき、$\forall a \in \liminf_{n \rightarrow \infty}A_{n}$に対し、次式が成り立つ。
\begin{align*}
\exists n \in \mathbb{N}\left[ a \in \bigcap_{k \in \mathbb{N} \setminus \varLambda_{n - 1}} A_{k} \right]
\end{align*}
ここで、$\forall m \in \mathbb{N}$に対し、次式が成り立つので、
\begin{align*}
\bigcap_{k \in \mathbb{N} \setminus \varLambda_{n - 1}} A_{k} \subseteq \bigcup_{k \in \mathbb{N} \setminus \varLambda_{m - 1}} A_{k}
\end{align*}
$\forall m \in \mathbb{N}$に対し、$a \in \bigcup_{k \in \mathbb{N} \setminus \varLambda_{m - 1}} A_{k}$が成り立つ。これにより、$a \in \limsup_{n \rightarrow \infty}A_{n}$が成り立つことになるので、次式が成り立つ。
\begin{align*}
\liminf_{n \rightarrow \infty}A_{n} \subseteq \limsup_{n \rightarrow \infty}A_{n}
\end{align*}
\end{proof}
\begin{dfn}
添数集合$\mathbb{N}$によって添数づけられた集合の列$\left( A_{n} \right)_{n \in \mathbb{N}}$が与えられたとする。$\liminf_{n \rightarrow \infty}A_{n} \supseteq \limsup_{n \rightarrow \infty}A_{n}$が成り立つとき、このことは定理\ref{4.5.1.1}より次式のように書かれ
\begin{align*}
\lim_{n \rightarrow \infty}A_{n} = \liminf_{n \rightarrow \infty}A_{n} = \limsup_{n \rightarrow \infty}A_{n}
\end{align*}
この集合$\lim_{n \rightarrow \infty}A_{n}$をその集合の列$\left( A_{n} \right)_{n \in \mathbb{N}}$の極限集合という。
\end{dfn}
\begin{thm}\label{4.5.1.2}
添数集合$\mathbb{N}$によって添数づけられた集合の列$\left( A_{n} \right)_{n \in \mathbb{N}}$が与えられたとする。これが単調増加する、または、単調減少するとき、その集合の列$\left( A_{n} \right)_{n \in \mathbb{N}}$の極限集合$\lim_{n \rightarrow \infty}A_{n}$が存在する。
\end{thm}
\begin{proof}
添数集合$\mathbb{N}$によって添数づけられた集合の列$\left( A_{n} \right)_{n \in \mathbb{N}}$が与えられたとする。これが単調増加するとき、次のようになる。
\begin{align*}
\limsup_{n \rightarrow \infty}A_{n} &= \bigcap_{n \in \mathbb{N}} {\bigcup_{k \in \mathbb{N} \setminus \varLambda_{n - 1}} A_{k}} = \bigcap_{n \in \mathbb{N}} {\bigcup_{k \in \mathbb{N}} A_{k}}\\
&= \bigcup_{k \in \mathbb{N}} A_{k} = \bigcup_{n \in \mathbb{N}} A_{n}\\
&= \bigcup_{n \in \mathbb{N}} {\bigcap_{k \in \mathbb{N} \setminus \varLambda_{n - 1}} A_{k}} = \liminf_{n \rightarrow \infty}A_{n}
\end{align*}
これが単調減少するとき、次のようになる。
\begin{align*}
\limsup_{n \rightarrow \infty}A_{n} &= \bigcap_{n \in \mathbb{N}} {\bigcup_{k \in \mathbb{N} \setminus \varLambda_{n - 1}} A_{k}} = \bigcap_{n \in \mathbb{N}} A_{n}\\
&= \bigcap_{k \in \mathbb{N}} A_{k} = \bigcup_{n \in \mathbb{N}} {\bigcap_{k \in \mathbb{N}} A_{k}}\\
&= \bigcup_{n \in \mathbb{N}} {\bigcap_{k \in \mathbb{N} \setminus \varLambda_{n - 1}} A_{k}} = \liminf_{n \rightarrow \infty}A_{n}
\end{align*}
よって、その集合の列$\left( A_{n} \right)_{n \in \mathbb{N}}$の極限集合$\lim_{n \rightarrow \infty}A_{n}$が存在する。
\end{proof}
\begin{thebibliography}{50}
\bibitem{1}
  杉浦光夫, 解析入門I, 東京大学出版社, 1980, 第34刷. p362-366 ISBN978-4-13-062005-5
\bibitem{2}
  数学の景色. "上極限集合・下極限集合の定義とその包含関係の証明". 数学の景色. \url{https://mathlandscape.com/limit-set/} (2021-8-18 8:15 閲覧)
\end{thebibliography}
\end{document}
