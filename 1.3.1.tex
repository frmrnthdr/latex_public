\documentclass[dvipdfmx]{jsarticle}
\setcounter{section}{3}
\setcounter{subsection}{0}
\usepackage{amsmath,amsfonts,amssymb,array,comment,mathtools,url,docmute}
\usepackage{longtable,booktabs,dcolumn,tabularx,mathtools,multirow,colortbl,xcolor}
\usepackage[dvipdfmx]{graphics}
\usepackage{bmpsize}
\usepackage{amsthm}
\usepackage{enumitem}
\setlistdepth{20}
\renewlist{itemize}{itemize}{20}
\setlist[itemize]{label=•}
\renewlist{enumerate}{enumerate}{20}
\setlist[enumerate]{label=\arabic*.}
\setcounter{MaxMatrixCols}{20}
\setcounter{tocdepth}{3}
\newcommand{\rotin}{\text{\rotatebox[origin=c]{90}{$\in $}}}
\renewcommand{\thesection}{第\arabic{section}部}
\renewcommand{\thesubsection}{\arabic{section}.\arabic{subsection}}
\renewcommand{\thesubsubsection}{\arabic{section}.\arabic{subsection}.\arabic{subsubsection}}
\everymath{\displaystyle}
\allowdisplaybreaks[4]
\usepackage{vtable}
\theoremstyle{definition}
\newtheorem{thm}{定理}[subsection]
\newtheorem*{thm*}{定理}
\newtheorem{dfn}{定義}[subsection]
\newtheorem*{dfn*}{定義}
\newtheorem{axs}[dfn]{公理}
\newtheorem*{axs*}{公理}
\renewcommand{\headfont}{\bfseries}
\makeatletter
  \renewcommand{\section}{%
    \@startsection{section}{1}{\z@}%
    {\Cvs}{\Cvs}%
    {\normalfont\huge\headfont\raggedright}}
\makeatother
\makeatletter
  \renewcommand{\subsection}{%
    \@startsection{subsection}{2}{\z@}%
    {0.5\Cvs}{0.5\Cvs}%
    {\normalfont\LARGE\headfont\raggedright}}
\makeatother
\makeatletter
  \renewcommand{\subsubsection}{%
    \@startsection{subsubsection}{3}{\z@}%
    {0.4\Cvs}{0.4\Cvs}%
    {\normalfont\Large\headfont\raggedright}}
\makeatother
\makeatletter
\renewenvironment{proof}[1][\proofname]{\par
  \pushQED{\qed}%
  \normalfont \topsep6\p@\@plus6\p@\relax
  \trivlist
  \item\relax
  {
  #1\@addpunct{.}}\hspace\labelsep\ignorespaces
}{%
  \popQED\endtrivlist\@endpefalse
}
\makeatother
\renewcommand{\proofname}{\textbf{証明}}
\usepackage{tikz,graphics}
\usepackage[dvipdfmx]{hyperref}
\usepackage{pxjahyper}
\hypersetup{
 setpagesize=false,
 bookmarks=true,
 bookmarksdepth=tocdepth,
 bookmarksnumbered=true,
 colorlinks=false,
 pdftitle={},
 pdfsubject={},
 pdfauthor={},
 pdfkeywords={}}
\begin{document}
%\hypertarget{ux9806ux5e8fux96c6ux5408}{%
\subsection{順序集合}%\label{ux9806ux5e8fux96c6ux5408}}
%\hypertarget{ux9806ux5e8fux96c6ux5408-1}{%
\subsubsection{順序集合}%\label{ux9806ux5e8fux96c6ux5408-1}}
\begin{axs}[順序の公理]
集合$A$における関係$O$が次のことを満たすとき、その関係$O$をその集合$A$における順序関係、順序などという。また、次のことを順序の公理という。
\begin{itemize}
\item
  $\forall a \in A$に対し、$aOa$が成り立つ。
\item
  $\forall a,b \in A$に対し、$aOb$かつ$bOa$が成り立つなら、$a = b$が成り立つ。
\item
  $\forall a,b,c \in A$に対し、$aOb$かつ$bOc$が成り立つなら、$aOc$が成り立つ。
\end{itemize}
\end{axs}
\begin{dfn}
その集合$A$の元々$a$、$b$に対し、$aOb$が成り立つことをその順序関係$O$についてその元$a$はその元$b$以下である、その元$b$はその元$a$以上である、その元$a$はその元$b$を超えないなどという。さらに、$aOb \land a \neq b$が成り立つことをその順序関係$O$についてその元$a$はその元$b$未満である、その元$b$はその元$a$超過である、その元$a$はその元$b$より小さい、その元$b$はその元$a$より大きい、その元$a$はその元$b$より前にある、その元$b$はその元$a$より後にあるなどという。
\end{dfn}
\begin{dfn}
その集合$A$の元々$a$、$b$に対し、$aOb$または$bOa$が成り立つとき、それらの元々$a$、$b$はその順序関係$O$について比較可能であるといい、さらに、$\forall a,b \in A$に対し、それらの元々$a$、$b$はその順序関係$O$について比較可能であるとき、その順序関係$O$はその集合$A$における全順序関係である、線形順序関係である、その集合$A$において全順序である、線形順序であるなどという。
\end{dfn}
\begin{dfn}
集合$A$とこれにおける順序関係$O$が与えられたとき、その順序づけられた組$(A,O)$を順序集合といい、これに対しその集合$A$をその順序集合$(A,O)$の台集合という。さらに、その順序関係$O$がその集合$A$において全順序であるとき、その順序集合$(A,O)$を全順序集合、線形順序集合という。
\end{dfn}
\begin{thm}\label{1.3.1.1}
順序集合$(A,O)$に対し、$A'\in \mathfrak{P}(A)$なる集合$A'$を用いて、$a,b \in A'$に対し、次式のように関係$O_{A'}$が定義されたとき、
\begin{align*}
aOb \Leftrightarrow aO_{A'}b
\end{align*}
その関係$O_{A'}$はその集合$A'$における順序関係となり順序集合$\left( A',O_{A'} \right)$が与えられる。
\end{thm}
\begin{dfn}
この順序集合$\left( A',O_{A'} \right)$をその順序集合$(A,O)$の部分順序集合という。
\end{dfn}
\begin{proof}
順序集合$(A,O)$が与えられているので、次のことが成り立つ。
\begin{itemize}
\item
  $\forall a \in A$に対し、$aOa$が成り立つ。
\item
  $\forall a,b \in A$に対し、$aOb \land bOa \Rightarrow a = b$が成り立つ。
\item
  $\forall a,b,c \in A$に対し、$aOb \land bOc \Rightarrow aOc$が成り立つ。
\end{itemize}
これにより、$A'\in \mathfrak{P}(A)$なる集合$A'$を用いて、$a,b \in A'$に対し、次式のように関係$O_{A'}$が定義されたとき、
\begin{align*}
aOb \Leftrightarrow aO_{A'}b
\end{align*}
次のことが成り立つので、
\begin{itemize}
\item
  $\forall a \in A'$に対し、$aO_{A'}a$が成り立つ。
\item
  $\forall a,b \in A'$に対し、$aO_{A'}b \land bO_{A'}a \Rightarrow a = b$が成り立つ。
\item
  $\forall a,b,c \in A'$に対し、$aO_{A'}b \land bO_{A'}c \Rightarrow aO_{A'}c$が成り立つ。
\end{itemize}
その関係$O_{A'}$はその集合$A'$における順序関係となり順序集合$\left( A',O_{A'} \right)$が与えられる。
\end{proof}
\begin{thm}\label{1.3.1.2}
順序集合$(A,O)$が与えられたとき、次のことが成り立つ。
\begin{itemize}
\item
  $\forall a,b \in A$に対し、$a = b$が成り立つならそのときに限り、$aOb$かつ$bOa$が成り立つ。
\item
  $\forall a,b \in A$に対し、それらの元々$a$、$b$が比較可能であるとき、$aOb$かつ$a \neq b$が成り立つならそのときに限り、$bOa$が成り立たない。
\item
  $\forall a,b \in A$に対し、それらの元々$a$、$b$が比較可能であるとき、$aOb$が成り立つならそのときに限り、$\neg(bOa \land a \neq b)$が成り立つ。
\item
  $\forall a,b \in A$に対し、$aOb$が成り立つならそのときに限り、$aOb$または$a = b$が成り立つ。
\end{itemize}
\end{thm}
\begin{proof}
順序集合$(A,O)$が与えられたとき、$\forall a,b \in A$に対し、順序の公理より$aOb$かつ$bOa$が成り立つなら、$a = b$が成り立つ。逆に、$a = b$が成り立つなら、順序の公理より$aOa$が成り立つので、$aOb$かつ$bOa$が成り立つ。以上より、$a = b$が成り立つならそのときに限り、$aOb$かつ$bOa$が成り立つ。\par
$\forall a,b \in A$に対し、それらの元々$a$、$b$が比較可能であるとき、$aOb \land a \neq b \land bOa$が成り立つと仮定すると、順序の公理より$aOb$かつ$bOa$が成り立つなら、$a = b$が成り立つので、$a = b \land a \neq b$が得られこれは矛盾している。したがって、$aOb \land a \neq b \Rightarrow \neg bOa$が成り立つ。一方で、$\neg bOa \land (\neg aOb \vee a = b)$が成り立つと仮定すると、$(\neg aOb \land \neg bOa) \vee (\neg bOa \land a = b)$が得られ、ここで、$\neg aOb \land \neg bOa$が成り立つならそのときに限り、$\neg(aOb \vee bOa)$が得られこれは元々$a$、$b$は比較可能でないことになり矛盾している。また、$\neg bOa \land a = b$が成り立つならそのときに限り、$\neg aOa \land a = b$が成り立つことになるが、これは順序の公理に矛盾している。したがって、$aOb \land a \neq b \Leftarrow \neg bOa$が成り立つ。以上より、$aOb$かつ$a \neq b$が成り立つならそのときに限り、$bOa$が成り立たない。\par
$\forall a,b \in A$に対し、それらの元々$a$、$b$が比較可能であるとき、上記の議論により$\neg aOb \Leftrightarrow aOb \land a \neq b$が成り立つので、対偶律に注意すれば明らかに$aOb$が成り立つならそのときに限り、$\neg(bOa \land a \neq b)$が成り立つ。\par
また、$\forall a,b \in A$に対し、$\neg aOb \vee aOb \vee a \neq b$は必ず成り立つので、$aOb \Rightarrow aOb \vee a = b$が成り立つ。逆に、$(aOb \vee a = b) \land \neg aOb$が成り立つと仮定すると、これが成り立つならそのときに限り、$\neg aOb \land a = b$が成り立ち$\neg aOa \land a = b$が成り立つが、これは順序の公理に矛盾している。したがって、$aOb \vee a = b$が成り立つなら、$aOb$が成り立つ。以上より、$aOb$が成り立つならそのときに限り、$aOb$または$a = b$が成り立つ。
\end{proof}
\begin{dfn}
順序集合$(A,O)$が与えられたとき、$\forall a \in A$に対し、$aOm$が成り立つその集合$A$の元$m$をその集合$A$の最大元という。同様に、$\forall a \in A$に対し、$mOa$が成り立つとなるその集合$A$の元$m$をその集合$A$の最小元という。後述するように存在するならそれは一意的なので、その集合$A$の最大元、最小元をそれぞれ$\max A$、$\min A$とかく。
\end{dfn}
\begin{thm}[最大元、最小元の存在の一意性]\label{1.3.1.3}
順序集合$(A,O)$が与えられたとき、その集合$A$の最大元が存在するなら、これは一意的になる。同様に、その集合$A$の最小元が存在するなら、これは一意的になる。
\end{thm}\par
これにより、その集合$A$の最大元と最小元をそれぞれ$\max A$、$\min A$と書くことができる。しかしながら、これらは必ずしも存在するとは限らない。
\begin{proof}
順序集合$(A,O)$が与えられその集合$A$の最大元が存在するとき、互いに異なる元々$m$、$n$がその集合$A$の最大元となると仮定しよう。このとき、$\forall a \in A$に対し、$aOm$かつ$aOn$が成り立つ。これにより、$nOm$かつ$mOn$が成り立つことになるが、これは順序の公理より$m = n$が得られてしまい$m \neq n$が成り立つことに矛盾する。したがって、その集合$A$の最大元が存在するなら、これは一意的になる。\par
その集合$A$の最小元についても同様にして示される。
\end{proof}
\begin{dfn}
順序集合$(A,O)$が与えられたとき、$\forall a \in A$に対し、$mOa$が成り立たないか、$a = m$が成り立つようなその集合$A$の元$m$をその集合$A$の極大元という。同様に、$aOm$が成り立たないか、$a = m$が成り立つようなその集合$A$の元$m$をその集合$A$の極小元という。これらは必ずしも一意的になるとは限らない。
\end{dfn}
\begin{thm}\label{1.3.1.4}
順序集合$(A,O)$が与えられたとき、その集合$A$の元$m$が極大元であるならそのときに限り、$\forall a \in A$に対し、$mOa$が成り立つなら、$a = m$が成り立つ。同様に、その集合$A$の元$m$が極小元であるならそのときに限り、$\forall a \in A$に対し、$aOm$が成り立つなら、$a = m$が成り立つ。
\end{thm}
\begin{proof}
順序集合$(A,O)$が与えられたとき、その集合$A$の元$m$が極大元であるならそのときに限り、定義より$mOa$が成り立たないか、$a = m$が成り立つ。したがって、これが成り立つならそのときに限り、明らかに$\forall a \in A$に対し、$mOa$が成り立つなら、$a = m$が成り立つことになる。\par
その集合$A$の極小元についても同様にして示される。
\end{proof}
\begin{thm}\label{1.3.1.5}
順序集合$(A,O)$が与えられたとき、その集合$A$の最大元が存在するなら、これは極大元でもある。同様にその集合$A$の最小元が存在するなら、これは極小元でもある。
\end{thm}
\begin{proof}
順序集合$(A,O)$が与えられたとき、その集合$A$の最大元$\max A$が存在するなら、$\forall a \in A$に対し、$aO\max A$が成り立つ。ここで、$\max AOa$が成り立つかつ、$a \neq m$が成り立つと仮定すると、これは、$\max AOa$が成り立つかつ、$a \neq \max A$が成り立つならそのときに限り、$aO\max A$が成り立たないのであったので、このような元$a$がその集合$A$に存在することになりこれは最大元の定義に矛盾している。したがって、$\max AOa$が成り立つなら、$a = m$が成り立つことになり、上記の定理よりこれが成り立つならそのときに限り、その元$\max A$はその集合$A$の極大元となるのであった。\par
その集合$A$の極小元についても同様にして示される。
\end{proof}
\begin{thm}\label{1.3.1.6}
全順序集合$(A,O)$が与えられたとき、その集合$A$の極大元は最大元となる。
\end{thm}
\begin{proof}
全順序集合$(A,O)$が与えられたとき、その集合$A$の極大元の1つを$m$とおくと、$\forall a \in A$に対し、$mOa$が成り立つなら、$a = m$が成り立つのであった。したがって、次のようになる。
\begin{align*}
mOa \Rightarrow a = m &\Leftrightarrow \neg mOa \vee m = a\\
&\Leftrightarrow (\neg mOa \vee m = a) \land \top
\end{align*}
ここで、その順序集合$(A,O)$は全順序で$mOa$または$aOm$が成り立つので、次のようになる。
\begin{align*}
mOa \Rightarrow a = m &\Leftrightarrow (\neg mOa \vee m = a) \land \top\\
&\Leftrightarrow (\neg mOa \vee m = a) \land (mOa \vee aOm)\\
&\Leftrightarrow (\neg mOa \land mOa) \vee (\neg mOa \land aOm) \vee (m = a \land mOa) \vee (m = a \land aOm)\\
&\Leftrightarrow (\neg mOa \land aOm) \vee m = a\\
&\Leftrightarrow (aOm \land m \neq a) \vee m = a\\
&\Leftrightarrow (aOm \vee m = a) \land (m \neq a \vee m = a)\\
&\Leftrightarrow aOm \vee m = a \Leftrightarrow aOm
\end{align*}
定義よりその元$m$は最大元となっている。\par
その集合$A$の極小元についても同様にして示される。
\end{proof}
\begin{dfn}
順序集合$(A,O)$が与えられたとき、$M \in \mathfrak{P}(A)$なる集合$M$を用いて、$\forall a \in M$に対し、$aOu$が成り立つようなその集合$A$の元$u$をその集合$M$のその集合$A$における上界といい、これ全体の集合を次式のように$U(M)$とおく。
\begin{align*}
U(M) = \left\{ u \in A \middle| \forall a \in M[ aOu] \right\}
\end{align*}
これが存在するとき、その集合$M$はその集合$A$において上に有界であるという。同様に、$\forall a \in M$に対し、$lOa$が成り立つようなその集合$A$の元$l$をその集合$M$のその集合$A$における下界といい、これ全体の集合を次式のように$L(M)$とおく。
\begin{align*}
L(M) = \left\{ l \in A \middle| \forall a \in M[ lOa] \right\}
\end{align*}
これが存在するとき、その集合$M$はその集合$A$において下に有界であるという。また、集合$M$がその集合$A$において上に有界であるかつ、下にも有界であることを、単に、その集合$M$はその集合$A$において有界であるという。
\end{dfn}
\begin{thm}\label{1.3.1.7}
順序集合$(A,O)$が与えられたとき、$\forall M \in \mathfrak{P}(A)$に対し、次のことが成り立つ。
\begin{itemize}
\item
  その集合$M$が集合$A$において上に有界であるならそのときに限り、$U(M) \neq \emptyset $が成り立つ。
\item
  その集合$M$が集合$A$において下に有界であるならそのときに限り、$L(M) \neq \emptyset $が成り立つ。
\end{itemize}
\end{thm}
\begin{proof}
定義より明らかである。
\end{proof}
\begin{dfn}
順序集合$(A,O)$が与えられたとき、$\forall M \in \mathfrak{P}(A)$に対し、その集合$M$のその集合$A$における上界全体の集合$U(M)$の最小元$\min{U(M)}$が存在するとき、この元をその集合$M$のその集合$A$における最小上界、上限といい$\sup M$と書く。同様に、その集合$M$のその集合$A$における下界全体の集合$L(M)$の最大元$\max{L(M)}$が存在するとき、この元をその集合$M$のその集合$A$における最大下界、下限といい$\inf M$と書く。
\end{dfn}
\begin{thm}\label{1.3.1.8}
順序集合$(A,O)$が与えられたとき、その集合$A$の元$m$が$M \in \mathfrak{P}(A)$なる集合$M$のその集合$A$における上限となるならそのときに限り、$m \in A$なる元$m$について、次のことが成り立つ。
\begin{itemize}
\item
  $\forall a \in M$に対し、$aOm$が成り立つ。
\item
  $\forall a' \in A\forall a \in M$に対し、$aOa'$が成り立つなら、$mOa'$が成り立つ。
\end{itemize}
同様に、その集合$A$の元$m$が$M \in \mathfrak{P}(A)$なる集合$M$のその集合$A$における下限となるならそのときに限り、$m \in A$なる元$m$について、次のことが成り立つ。
\begin{itemize}
\item
  $\forall a \in M$に対し、$mOa$が成り立つ。
\item
  $\forall a' \in A\forall a \in M$に対し、$a'Oa$が成り立つなら、$a'Om$が成り立つ。
\end{itemize}
\end{thm}
\begin{proof}
順序集合$(A,O)$が与えられたとき、その集合$A$の元$m$が$M \in \mathfrak{P}(A)$なる集合$M$のその集合$A$における上限となるなら、$m = \min{U(M)}$が成り立つ。ここで、$m \in U(M)$が成り立つので、$\forall a \in M$に対し、$aOm$が成り立つ。また、$\forall a' \in A\forall a \in M$に対し、$aOa'$が成り立つなら、その元$a'$はその集合$M$のその集合$A$における上界となっておりその集合$U(M)$に属することになる。上限の定義より$m = \min{U(M)}$が成り立つので、$mOa'$が成り立つ。したがって、次のことが成り立つ。
\begin{itemize}
\item
  $\forall a \in M$に対し、$aOm$が成り立つ。
\item
  $\forall a' \in A\forall a \in M$に対し、$aOa'$が成り立つなら、$mOa'$が成り立つ。
\end{itemize}\par
逆に、$m \in A$なる元$m$について、次のことが成り立つなら、
\begin{itemize}
\item
  $\forall a \in M$に対し、$aOm$が成り立つ。
\item
  $\forall a' \in A\forall a \in M$に対し、$aOa'$が成り立つなら、$mOa'$が成り立つ。
\end{itemize}
その元$m$はその集合$M$のその集合$A$における上界となっており$m \in U(M)$が成り立つ。さらに、$\forall a' \in A\forall a \in M$に対し、$aOa'$が成り立つならそのときに限り、、その元$a'$もまたその集合$M$のその集合$A$における上界となっており、即ち、$a' \in U(M)$が成り立っており、さらに、$mOa'$が成り立つ。以上より、$m \in U(M)$が成り立つかつ、$\forall a' \in U(M)$に対し、$mOa'$が成り立つので、$m = \min{U(M)}$が成り立ちその集合$U(M)$の元$m$がその集合$M$のその集合$A$における上限となる。\par
下限についても同様にして示される。
\end{proof}
\begin{thm}\label{1.3.1.9}
順序集合$(A,O)$が与えられたとき、次のことが成り立つ。
\begin{itemize}
\item
  その集合$A$の元$m$が$M \in \mathfrak{P}(A)$なる集合$M$の上限$\sup M$であるならそのときに限り、$\forall a \in M$に対し、$aOm$が成り立つかつ、$\forall b \in A\exists a \in M$に対し、$mOb$が成り立たないなら、$aOb$が成り立たない。
\item
  その集合$A$の元$m$が$M \in \mathfrak{P}(A)$なる集合$M$の下限$\inf M$であるならそのときに限り、$\forall a \in M$に対し、$mOa$が成り立つかつ、$\forall b \in A\exists a \in M$に対し、$bOm$が成り立たないなら、$bOa$が成り立たない。
\end{itemize}
\end{thm}
\begin{proof}
順序集合$(A,O)$が与えられたとき、その集合$A$の元$m$が$M \in \mathfrak{P}(A)$なる集合$M$の上限$\sup M$であるなら、$m \in U(M)$が成り立つので、その集合$U(M)$の定義より$\forall a \in M$に対し、$aOm$が成り立つかつ、$\forall b \in A$に対し、$b \in U(M) \Rightarrow mOb$の対偶がとられれば、$\neg mOb \Rightarrow b \notin U(M)$が成り立ち、$b \notin U(M)$が成り立つことと、$\exists a \in M$に対し、$aOb$が成り立たないこととは同値であるので、$\forall b \in A\exists a \in M$に対し、$mOb$が成り立たないなら、$aOb$が成り立つ。逆に、$\forall a \in M$に対し、$aOm$が成り立つかつ、$\forall b \in A\exists a \in M$に対し、$mOb$が成り立たないなら、$aOb$が成り立たないなら、定義より明らかに$m \in U(M)$が成り立つかつ、$\exists a \in M$に対し、$aOb$が成り立たないことと$b \notin U(M)$が成り立つこととは同値であるので、$\neg mOb \Rightarrow b \notin U(M)$の対偶がとられれば、$b \in U(M) \Rightarrow mOb$が成り立ち、したがって、$m \in U(M)$が成り立つかつ、$b \in U(M) \Rightarrow mOb$が成り立つので、$m \in U(M)$が成り立つかつ、$\forall b \in U(M)$に対し、$mOb$が成り立ち、定義よりよって、$m = \sup A$が成り立つ。\par
同様にしてその集合$A$の元$m$が$M \in \mathfrak{P}(A)$なる集合$M$の下限$\inf M$であるならそのときに限り、$\forall a \in M$に対し、$mOa$が成り立つかつ、$\forall b \in A\exists a \in M$に対し、$bOm$が成り立たないなら、$bOa$が成り立たないことも示される。
\end{proof}
\begin{thm}\label{1.3.1.10}
順序集合$(A,O)$が与えられたとき、次のことが成り立つ。
\begin{itemize}
\item
  $M \in \mathfrak{P}(A)$なる集合$M$の最大元$\max M$はその集合$M$のその集合$A$における上限でもある。
\item
  $M \in \mathfrak{P}(A)$なる集合$M$の最小元$\min M$はその集合$M$のその集合$A$における下限でもある。
\end{itemize}
\end{thm}
\begin{proof}
順序集合$(A,O)$が与えられたとき、$M \in \mathfrak{P}(A)$なる集合$M$の最大元$\max M$は定義より、$\forall a \in M$に対し、$aO\max M$を満たす。これにより、その元$\max M$はその集合$M$のその集合$A$における上界で、その集合$M$のその集合$A$における上界全体の集合$U(M)$を用いて、$\max M \in U(M)$が成り立つ。ここで、$uO\max M$なるその元$\max M$とは異なる元$u$がその集合$U(M)$に存在すると仮定すると、$\max M \in M$も成り立っているので、$\neg aOu$が成り立つような元$a$がその集合$M$に存在することになるが、これはその元$u$がその集合$M$のその集合$A$における上界であることに矛盾する。したがって、$\forall u \in U(M)$に対し、$\max MOu$が成り立つので、その元$\max M$はその集合$M$のその集合$A$における上限でもある。\par
下限についても同様にして示される。
\end{proof}
%\hypertarget{ux9806ux5e8fux540cux578b}{%
\subsubsection{順序同型}%\label{ux9806ux5e8fux540cux578b}}
\begin{dfn}
2つの順序集合たち$(A,O)$、$(B,P)$が与えられたとき、写像$f:A \rightarrow B$が、$\forall a,b \in A$に対し、$aOb$が成り立つなら、$f(a)Pf(b)$が成り立つとき、その写像$f$をその順序集合$(A,O)$からその順序集合$(B,P)$への順序写像、順序を保つ写像、単調写像などという。
\end{dfn}
\begin{thm}\label{1.3.1.11}
写像$f:A \rightarrow B$が、$\forall a,b \in A$に対し、$aOb$が成り立たないなら、$f(a)Pf(b)$も成り立たないとき、その写像$f$は単射となる。
\end{thm}
\begin{proof}
2つの順序集合たち$(A,O)$、$(B,P)$が与えられたとき、写像$f:A \rightarrow B$が、$\forall a,b \in A$に対し、$aOb$が成り立たないなら、$f(a)Pf(b)$も成り立たないとき、対偶律より$f(a)Pf(b)$が成り立つなら、$aOb$が成り立つ。ここで、$f(a) = f(b)$が成り立つなら、順序の公理より$f(a)Pf(b)$かつ$f(b)Pf(a)$が成り立つことになり、したがって、$aOb$かつ$bOa$が成り立つ。順序の公理より$a = b$が成り立つので、対偶律より、$\forall a,b \in A$に対し、$a \neq b$が成り立つなら、$f(a) \neq f(b)$が成り立つ。以上より、その写像$f$は単射となる。
\end{proof}
\begin{dfn}
特に、その順序集合$(A,O)$からその順序集合$(B,P)$への順序写像$f$が、$\forall a,b \in A$に対し、$aOb$が成り立たないなら、$f(a)Pf(b)$も成り立たないとき、その写像$f$を順序単射という。ここで、2つの順序集合たち$(A,O)$、$(B,P)$が与えられたとき、写像$f:A \rightarrow B$が順序単射であることは、その写像$f:A \rightarrow B$が順序写像であるかつ、単射であることへの必要十分条件ではないことに注意されたい。
\end{dfn}
\begin{dfn}
2つの順序集合たち$(A,O)$、$(B,P)$が与えられたとき、写像$f:A \rightarrow B$が順序単射であるかつ、全射であるとき、その写像$f$をその順序集合$(A,O)$からその順序集合$(B,P)$への順序同型写像という。
\end{dfn}
\begin{thm}\label{1.3.1.12}
その順序集合$(A,O)$からその順序集合$(B,P)$への順序同型写像$f$の逆対応$f^{- 1}$もその順序集合$(B,P)$からその順序集合$(A,O)$への順序同型写像となる。
\end{thm}
\begin{proof}
2つの順序集合たち$(A,O)$、$(B,P)$が与えられたとき、その順序集合$(A,O)$からその順序集合$(B,P)$への順序同型写像$f$は、$\forall a,b \in A$に対し、$aOb$が成り立たないなら、$f(a)Pf(b)$も成り立たないので、単射であるかつ、定義より明らかに全射であるので、その写像$f$は全単射となる。したがって、逆対応$f^{- 1}$も全単射の写像となる。また、その写像$f$は順序写像でもあったので、$\forall a,b \in A$に対し、$aOb$が成り立つなら、$f(a)Pf(b)$が成り立つ。これにより、$\forall a,b \in B = V(f)$に対し、$f^{- 1}(a)Of^{- 1}(b)$が成り立つなら、$aPb$が成り立つので、その写像$f^{- 1}$は順序単射である。以上より、その写像$f^{- 1}$は順序単射であるかつ、全射であるので、その写像$f$の逆対応$f^{- 1}$もその順序集合$(B,P)$からその順序集合$(A,O)$への順序同型写像となる。
\end{proof}
\begin{thm}\label{1.3.1.13}
3つの順序集合たち$(A,O)$、$(B,P)$、$(C,Q)$が与えられたとき、写像たち$f$、$g$がそれぞれその順序集合$(A,O)$からその順序集合$(B,P)$への順序同型写像、その順序集合$(B,P)$からその順序集合$(C,Q)$への順序同型写像であるなら、その写像$g \circ f$もその順序集合$(A,O)$からその順序集合$(C,Q)$への順序同型写像となる。
\end{thm}
\begin{proof}
3つの順序集合たち$(A,O)$、$(B,P)$、$(C,Q)$が与えられたとき、写像たち$f$、$g$がそれぞれその順序集合$(A,O)$からその順序集合$(B,P)$への順序同型写像、その順序集合$(B,P)$からその順序集合$(C,Q)$への順序同型写像であるなら、それらの写像たち$f$、$g$は全単射であるので、その写像$g \circ f$も全単射となる。ここで、それらの写像たち$f$、$g$は順序単射であるので、$\forall a,b \in A$に対し、次のようになる。
\begin{align*}
\neg aOb \Rightarrow \neg f(a)Pf(b) \Rightarrow \neg g\left( f(a) \right)Qg\left( f(b) \right) \Leftrightarrow \neg g \circ f(a)Qg \circ f(b)
\end{align*}
これにより、その写像$g \circ f$は順序単射となる。以上より、その写像$g \circ f$は順序単射であるかつ、全射であるので、その写像$g \circ f$もその順序集合$(A,O)$からその順序集合$(C,Q)$への順序同型写像となる。
\end{proof}
\begin{dfn}
2つの順序集合たち$(A,O)$、$(B,P)$が与えられたとき、その順序集合$(A,O)$からその順序集合$(B,P)$への順序同型写像$f$が少なくとも1つ存在するとき、その順序集合$(A,O)$とその順序集合$(B,P)$は順序同型であるといい$(A,O) \simeq (B,P)$などと書く。
\end{dfn}
\begin{thm}\label{1.3.1.14}
その関係$\simeq$は同値関係となる。
\end{thm}
\begin{proof}
3つの順序集合たち$(A,O)$、$(B,P)$、$(C,Q)$が与えられたとする。恒等写像$I_{A}$は明らかに順序同型写像なので、$(A,O) \simeq (A,O)$が成り立つ。\par
また、$(A,O) \simeq (B,P)$が成り立つなら、その順序集合$(A,O)$からその順序集合$(B,P)$への順序同型写像$f$が存在し、これの逆対応$f^{- 1}$もその順序集合$(A,O)$からその順序集合$(B,P)$への順序同型写像となるので、$(B,P) \simeq (A,O)$が成り立つ。\par
また、$(A,O) \simeq (B,P)$かつ$(B,P) \simeq (C,Q)$が成り立つなら、その順序集合$(A,O)$からその順序集合$(B,P)$への順序同型写像$f$とその順序集合$(B,P)$からその順序集合$(C,Q)$への順序同型写像$g$が存在し、その写像$g \circ f$もその順序集合$(A,O)$からその順序集合$(C,Q)$への順序同型写像となるので、$(A,O) \simeq (C,Q)$が成り立つ。
\end{proof}
\begin{thm}\label{1.3.1.15}
$(A,O) \simeq (B,P)$が成り立つなら、$\# A = \# B$が成り立つ。
\end{thm}
\begin{proof}
2つの順序集合たち$(A,O)$、$(B,P)$が与えられたとき、$(A,O) \simeq (B,P)$が成り立つなら、その順序集合$(A,O)$からその順序集合$(B,P)$への順序同型写像$f$が存在し、これは全単射となるので、濃度の定義より$\# A = \# B$が成り立つ。
\end{proof}
\begin{thm}\label{1.3.1.16}
2つの順序集合たち$(A,O)$、$(B,P)$が与えられたとき、その順序集合$(A,O)$からその順序集合$(B,P)$への順序単射$f$が存在するなら、その集合$B$の部分集合$B'$を用いたその順序集合$(A,O)$から順序集合$\left( B',P \right)$への順序同型写像も存在する。
\end{thm}
\begin{proof}
2つの順序集合たち$(A,O)$、$(B,P)$が与えられたとき、その順序集合$(A,O)$からその順序集合$(B,P)$への順序単射$f$が存在するなら、その順序集合$(A,O)$から順序集合$\left( V(f),P \right)$への順序単射$f':A \rightarrow V(f);a \mapsto f(a)$は全射の定義より順序同型写像となる。これにより、その順序集合$(A,O)$からその順序集合$(B,P)$への順序単射$f$が存在するなら、その集合$B$の部分集合$B'$を用いたその順序集合$(A,O)$から順序集合$\left( B',P \right)$への順序同型写像が存在する。
\end{proof}
\begin{thebibliography}{50}
  \bibitem{1}
    松坂和夫, 集合・位相入門, 岩波書店, 1968. 新装版第2刷 p87-97 ISBM978-4-00-029871-1
\end{thebibliography}
\end{document}
