\documentclass[dvipdfmx]{jsarticle}
\setcounter{section}{1}
\setcounter{subsection}{6}
\usepackage{xr}
\externaldocument{4.1.1}
\externaldocument{4.1.3}
\externaldocument{4.1.4}
\externaldocument{4.1.5}
\usepackage{amsmath,amsfonts,amssymb,array,comment,mathtools,url,docmute}
\usepackage{longtable,booktabs,dcolumn,tabularx,mathtools,multirow,colortbl,xcolor}
\usepackage[dvipdfmx]{graphics}
\usepackage{bmpsize}
\usepackage{amsthm}
\usepackage{enumitem}
\setlistdepth{20}
\renewlist{itemize}{itemize}{20}
\setlist[itemize]{label=•}
\renewlist{enumerate}{enumerate}{20}
\setlist[enumerate]{label=\arabic*.}
\setcounter{MaxMatrixCols}{20}
\setcounter{tocdepth}{3}
\newcommand{\rotin}{\text{\rotatebox[origin=c]{90}{$\in $}}}
\renewcommand{\thesection}{第\arabic{section}部}
\renewcommand{\thesubsection}{\arabic{section}.\arabic{subsection}}
\renewcommand{\thesubsubsection}{\arabic{section}.\arabic{subsection}.\arabic{subsubsection}}
\everymath{\displaystyle}
\allowdisplaybreaks[4]
\usepackage{vtable}
\theoremstyle{definition}
\newtheorem{thm}{定理}[subsection]
\newtheorem*{thm*}{定理}
\newtheorem{dfn}{定義}[subsection]
\newtheorem*{dfn*}{定義}
\newtheorem{axs}[dfn]{公理}
\newtheorem*{axs*}{公理}
\renewcommand{\headfont}{\bfseries}
\makeatletter
  \renewcommand{\section}{%
    \@startsection{section}{1}{\z@}%
    {\Cvs}{\Cvs}%
    {\normalfont\huge\headfont\raggedright}}
\makeatother
\makeatletter
  \renewcommand{\subsection}{%
    \@startsection{subsection}{2}{\z@}%
    {0.5\Cvs}{0.5\Cvs}%
    {\normalfont\LARGE\headfont\raggedright}}
\makeatother
\makeatletter
  \renewcommand{\subsubsection}{%
    \@startsection{subsubsection}{3}{\z@}%
    {0.4\Cvs}{0.4\Cvs}%
    {\normalfont\Large\headfont\raggedright}}
\makeatother
\makeatletter
\renewenvironment{proof}[1][\proofname]{\par
  \pushQED{\qed}%
  \normalfont \topsep6\p@\@plus6\p@\relax
  \trivlist
  \item\relax
  {
  #1\@addpunct{.}}\hspace\labelsep\ignorespaces
}{%
  \popQED\endtrivlist\@endpefalse
}
\makeatother
\renewcommand{\proofname}{\textbf{証明}}
\usepackage{tikz,graphics}
\usepackage[dvipdfmx]{hyperref}
\usepackage{pxjahyper}
\hypersetup{
 setpagesize=false,
 bookmarks=true,
 bookmarksdepth=tocdepth,
 bookmarksnumbered=true,
 colorlinks=false,
 pdftitle={},
 pdfsubject={},
 pdfauthor={},
 pdfkeywords={}}
\begin{document}
%\hypertarget{compact}{%
\subsection{compact}%\label{compact}}
この周辺の議論のさらなる一般化については位相空間論に詳しいので、そちらのほうを参照されたい。ここでは主に、いわゆる$n$次元Euclid空間というやや特殊化された条件下であるものの、$n$次元Euclid空間特有の性質について述べよう。なお、一般的な位相空間論の知識は仮定しないでおこう。
%\hypertarget{compact-1}{%
\subsubsection{compact}%\label{compact-1}}
\begin{dfn}
$K \subseteq R \subseteq \mathbb{R}_{\infty}^{n}$なる集合$K$が与えられたとき、その集合$R$の集合族$\left\{ U_{\lambda} \right\}_{\lambda \in \varLambda}$が次式を満たすとき、その集合族$\left\{ U_{\lambda} \right\}_{\lambda \in \varLambda}$はその集合$R$におけるその集合$K$の被覆であるという。
\begin{align*}
K \subseteq \bigcup_{\lambda \in \varLambda} U_{\lambda}
\end{align*}
全てのそれらの集合たち$U_{\lambda}$がその集合$R$における開集合であるとき、その被覆$\left\{ U_{\lambda} \right\}_{\lambda \in \varLambda}$を開被覆という。
\end{dfn}
\begin{dfn}
$K \subseteq R \subseteq \mathbb{R}_{\infty}^{n}$なる集合$R$におけるその集合$K$の任意の開被覆$\left\{ U_{\lambda} \right\}_{\lambda \in \varLambda}$に対し、有限集合$\varLambda'$を用いた$\left\{ U_{\lambda} \right\}_{\lambda \in \varLambda' \subseteq \varLambda } \subseteq \left\{ U_{\lambda} \right\}_{\lambda \in \varLambda}$なる集合族$\left\{ U_{\lambda} \right\}_{\lambda \in \varLambda' \subseteq \varLambda }$が存在して、次式を満たすことができるとき、その集合$K$はその集合$R$でcompactである、完閉であるという。
\begin{align*}
K \subseteq \bigcup_{\lambda \in \varLambda' \subseteq \varLambda } U_{\lambda}
\end{align*}
\end{dfn}
\begin{thm}\label{4.1.7.1}
$K \subseteq R \subseteq \mathbb{R}_{\infty}^{n}$なる集合$R$がその集合$R$自身でcompactでその集合$K$がその集合$R$での閉集合であるなら、その集合$K$もその集合$R$でcompactである。
\end{thm}
\begin{proof}
$K \subseteq R \subseteq \mathbb{R}_{\infty}^{n}$なる集合$R$が拡張$n$次元数空間$\mathbb{R}_{\infty}^{n}$でcompactでその集合$K$がその集合$R$での閉集合であるなら、その集合$R$におけるその集合$K$の任意の開被覆$\left\{ U_{\lambda} \right\}_{\lambda \in \varLambda}$が与えられたとき、次のようになる。
\begin{align*}
R = K \sqcup (R \setminus K) = \bigcup_{\lambda \in \varLambda} U_{\lambda} \cup (R \setminus K)
\end{align*}
ここで、その集合$R \setminus K$は定理\ref{4.1.3.12}よりその集合$R$における開集合であるので、集合$\left\{ U_{\lambda} \right\}_{\lambda \in \varLambda} \cup \left\{ R \setminus K \right\}$はその集合$R$のその集合$R$自身での開被覆である。このとき、仮定より有限集合$\varLambda'$を用いた集合族$\left\{ U_{\lambda} \right\}_{\lambda \in \varLambda' \subseteq \varLambda } \cup \left\{ R \setminus K \right\}$が存在して、次式が成り立つ。
\begin{align*}
R = \bigcup_{} \left( \left\{ U_{\lambda} \right\}_{\lambda \in \varLambda' \subseteq \varLambda } \cup \left\{ R \setminus K \right\} \right) = \bigcup_{\lambda \in \varLambda' \subseteq \varLambda} U_{\lambda} \cup (R \setminus K) = K \sqcup (R \setminus K)
\end{align*}
これにより、$K \subseteq \bigcup_{\lambda \in \varLambda' \subseteq \varLambda} U_{\lambda}$が成り立つので、その集合$K$もその集合$R$でcompactである。
\end{proof}
\begin{thm}\label{4.1.7.2}
$K \subseteq R \subseteq \mathbb{R}_{\infty}^{n}$なる集合$K$がその集合$R$でcompactであるなら、その集合$K$はその集合$R$で閉集合である。
\end{thm}\par
この定理から、その集合$K$がその集合$R$で閉集合でなければ、その集合$K$はその集合$R$でcompactになりえないということも分かる。
\begin{proof}
$K \subseteq R \subseteq \mathbb{R}_{\infty}^{n}$なる集合$K$がその集合$R$でcompactであるとき、$\forall\mathbf{a} \in R \setminus K\forall\mathbf{b} \in K$に対し、$\mathbf{a} \neq \mathbf{b}$が成り立つので、定理\ref{4.1.3.5}より$\exists\delta_{\mathbf{b}},\varepsilon_{\mathbf{b}} \in \mathbb{R}^{+}$に対し、$U\left( \mathbf{a},\delta_{\mathbf{b}} \right) \cap U\left( \mathbf{b},\varepsilon_{\mathbf{b}} \right) \cap R = \emptyset$が成り立つ。このような$\varepsilon_{\mathbf{b}}$近傍$U\left( \mathbf{b},\varepsilon_{\mathbf{b}} \right) \cap R$の開核について、$\mathbf{b} \in \mathrm{int}_{R}\left( U\left( \mathbf{b},\varepsilon_{\mathbf{b}} \right) \cap R \right)$が成り立つので、その族$\left\{ \mathrm{int}_{R}\left( U\left( \mathbf{b},\varepsilon_{\mathbf{b}} \right) \cap R \right) \right\}_{\mathbf{b} \in K}$はその集合$K$のその集合$R$での開被覆である。そこで、その集合$K$がその集合$R$でcompactであるので、その集合$K$の有限集合である部分集合$L$が存在して、その族$\left\{ \mathrm{int}_{R}\left( U\left( \mathbf{b},\varepsilon_{\mathbf{b}} \right) \cap R \right) \right\}_{\mathbf{b} \in L}$がその集合$K$の開被覆であることができる。このとき、$\forall\mathbf{b} \in L$に対し、$\mathbf{a} \in \mathrm{int}_{R}\left( U\left( \mathbf{a},\delta_{\mathbf{b}} \right) \cap R \right)$が成り立つので、$\mathbf{a} \in \bigcap_{\mathbf{b} \in L} {\mathrm{int}_{R}\left( U\left( \mathbf{a},\delta_{\mathbf{b}} \right) \cap R \right)}$が成り立つ。したがって、$\bigcap_{\mathbf{b} \in L} {\mathrm{int}_{R}\left( U\left( \mathbf{a},\delta_{\mathbf{b}} \right) \cap R \right)} \subseteq U\left( \mathbf{a},\delta_{\mathbf{b}} \right) \cap R$かつ$U\left( \mathbf{a},\delta_{\mathbf{b}} \right) \cap U\left( \mathbf{b},\varepsilon_{\mathbf{b}} \right) \cap R = \emptyset$が成り立つので、定理\ref{4.1.3.8}より次のようになる。
\begin{align*}
\emptyset &= \bigcup_{\mathbf{b} \in L} \left( U\left( \mathbf{a},\delta_{\mathbf{b}} \right) \cap U\left( \mathbf{b},\varepsilon_{\mathbf{b}} \right) \cap R \right)\\
&\supseteq \bigcup_{\mathbf{b} \in L} {\mathrm{int}_{R}\left( U\left( \mathbf{a},\delta_{\mathbf{b}} \right) \cap U\left( \mathbf{b},\varepsilon_{\mathbf{b}} \right) \cap R \right)}\\
&= \bigcup_{\mathbf{b} \in L} \left( \mathrm{int}_{R}\left( U\left( \mathbf{a},\delta_{\mathbf{b}} \right) \cap R \right) \cap \mathrm{int}_{R}\left( U\left( \mathbf{b},\varepsilon_{\mathbf{b}} \right) \cap R \right) \right)\\
&\supseteq \bigcup_{\mathbf{b} \in L} \left( \bigcap_{\mathbf{c} \in L} {\mathrm{int}_{R}\left( U\left( \mathbf{a},\delta_{\mathbf{c}} \right) \cap R \right)} \cap \mathrm{int}_{R}\left( U\left( \mathbf{b},\varepsilon_{\mathbf{b}} \right) \cap R \right) \right)\\
&= \bigcup_{\mathbf{b} \in L} \left( \mathrm{int}_{R}{\bigcap_{\mathbf{c} \in L} \left( U\left( \mathbf{a},\delta_{\mathbf{c}} \right) \cap R \right)} \cap \mathrm{int}_{R}\left( U\left( \mathbf{b},\varepsilon_{\mathbf{b}} \right) \cap R \right) \right)\\
&= \mathrm{int}_{R}{\bigcap_{\mathbf{c} \in L} \left( U\left( \mathbf{a},\delta_{\mathbf{c}} \right) \cap R \right)} \cap \bigcup_{\mathbf{b} \in L} {\mathrm{int}_{R}\left( U\left( \mathbf{b},\varepsilon_{\mathbf{b}} \right) \cap R \right)}\\
&= \mathrm{int}_{R}{\bigcap_{\mathbf{b} \in L} \left( U\left( \mathbf{a},\delta_{\mathbf{b}} \right) \cap R \right)} \cap \bigcup_{\mathbf{b} \in L} {\mathrm{int}_{R}\left( U\left( \mathbf{b},\varepsilon_{\mathbf{b}} \right) \cap R \right)}\\
&\supseteq \mathrm{int}_{R}{\bigcap_{\mathbf{b} \in L} \left( U\left( \mathbf{a},\delta_{\mathbf{b}} \right) \cap R \right)} \cap K
\end{align*}
これにより、その集合$\mathrm{int}_{R}{\bigcap_{\mathbf{b} \in L} \left( U\left( \mathbf{a},\delta_{\mathbf{b}} \right) \cap R \right)} \cap K$は空集合であるので、次のようになる。
\begin{align*}
\mathrm{int}_{R}{\bigcap_{\mathbf{b} \in L} \left( U\left( \mathbf{a},\delta_{\mathbf{b}} \right) \cap R \right)} \subseteq R \setminus K
\end{align*}
したがって、定理\ref{4.1.3.8}より次のようになるので、
\begin{align*}
\mathbf{a} \in \mathrm{int}_{R}{\bigcap_{\mathbf{b} \in L} \left( U\left( \mathbf{a},\delta_{\mathbf{b}} \right) \cap R \right)} = \mathrm{int}_{R}{\mathrm{int}_{R}{\bigcap_{\mathbf{b} \in L} \left( U\left( \mathbf{a},\delta_{\mathbf{b}} \right) \cap R \right)}} \subseteq \mathrm{int}_{R}(R \setminus K)
\end{align*}
その点$\mathbf{a}$はその集合$R \setminus K$の内点となっており、$\exists\varepsilon \in \mathbb{R}^{+}$に対し、$U\left( \mathbf{a},\varepsilon \right) \cap R \subseteq R \setminus K$が成り立つ。これにより、$R \setminus K = \mathrm{int}_{R}(R \setminus K)$が得られ、定理\ref{4.1.3.12}よりその集合$R \setminus K$はその集合$R$における開集合となり、よって、その集合$K$はその集合$R$で閉集合となる。
\end{proof}
%\hypertarget{ux70b9ux5217compact}{%
\subsubsection{点列compact}%%\label{ux70b9ux5217compact}}
\begin{dfn}
$K \subseteq R \subseteq \mathbb{R}_{\infty}^{n}$なる集合$K$の任意の点列$\left( \mathbf{a}_{m} \right)_{m \in \mathbb{N}}$に対し、これのその集合$R$で広い意味で収束する部分列$\left( \mathbf{a}_{m_{k}} \right)_{k \in \mathbb{N}}$が存在して、$\lim_{k \rightarrow \infty}\mathbf{a}_{m_{k}} \in K$が成り立つとき、その集合$K$はその集合$R$で点列compactである、点列完閉であるという。拡張$n$次元数空間$\mathbb{R}_{\infty}^{n}$のかわりに補完数直線${}^{*}\mathbb{R}$でおきかえても同様にして定義される。
\end{dfn}
\begin{thm}\label{4.1.7.3}
$K \subseteq R \subseteq \mathbb{R}_{\infty}^{n}$なる集合$K$がその集合$R$でcompactであるならそのときに限り、その集合$K$はその集合$R$で点列compactである。
\end{thm}\par
これは次のようにして示される。
\begin{enumerate}
\item
  まず、集合$K$がcompactであるなら、その集合$K$は点列compactであることを示そう。
\item
  その集合$K$の任意の点列$\left( \mathbf{a}_{m} \right)_{m \in \mathbb{N}}$に対し、次のように集合$M\left( \mathbf{a},\varepsilon \right)$がおかれると、
\begin{align*}
M\left( \mathbf{a},\varepsilon \right) = \left\{ m \in \mathbb{N} \middle| \mathbf{a}_{m} \in U\left( \mathbf{a},\varepsilon \right) \cap R \right\}
\end{align*}
\item
  $\exists\mathbf{a} \in K\forall\varepsilon \in \mathbb{R}^{+}$に対し、その集合$M\left( \mathbf{a},\varepsilon \right)$は無限集合となる。
\item
  $m \in M\left( \mathbf{a},1 \right) \setminus \varLambda_{1}$なる自然数$m$を$m_{1}$として、自然数$m_{k}$が与えられたとき、$m \in M\left( \mathbf{a},\frac{1}{k + 1} \right) \setminus \varLambda_{m_{k}}$なる自然数$m$を$m_{k + 1}$とおくことにする。
\item
  4. より、その点列$\left( \mathbf{a}_{m} \right)_{m \in \mathbb{N}}$の部分列$\left( \mathbf{a}_{m_{k}} \right)_{k \in \mathbb{N}}$が得られ、$\lim_{k \rightarrow \infty}\mathbf{a}_{m_{k}} = \mathbf{a}$が成り立つ。
\item
  1. ~5. より集合$K$がcompactであるなら、その集合$K$は点列compactである。
\item
  次に、集合$K$が点列compactであるなら、その集合$K$はcompactであることを背理法で示そう。
\item
  その集合$K$が点列compactであるかつ、その集合$K$のある開被覆$\left\{ U_{\lambda} \right\}_{\lambda \in \varLambda}$が存在して、その集合$\varLambda$のどの有限集合な部分集合$\varLambda'$に対してもその族$\left\{ U_{\lambda} \right\}_{\lambda \in \varLambda'}$がその集合$K$の開被覆になりえないとする。
\item
  次のように集合$\mathfrak{L}$がおかれると、
\begin{align*}
\mathfrak{L} =\left\{ \varLambda'\in \mathfrak{P}(\varLambda) \middle| K \subseteq \bigcup_{\lambda \in \varLambda'} U_{\lambda} \right\}
\end{align*}
その組$\left( \mathfrak{L, \supseteq} \right)$は帰納的な順序集合となっている。
\item
  Zornの補題よりその順序集合$\left( \mathfrak{L, \supseteq} \right)$に極大元$\varLambda_{m}$が存在する。
\item
  $\forall\lambda \in \varLambda_{m}\exists\mathbf{a}_{\lambda} \in U_{\lambda}\forall\mu \in \varLambda_{m} \setminus \left\{ \lambda \right\}$に対し、$\mathbf{a}_{\lambda} \notin U_{\mu}$が成り立つ。
\item
  $\forall\mu \in \varLambda_{m} \setminus \left\{ \lambda \right\}$に対し、$\mathbf{a}_{\lambda} \notin U_{\mu}$が成り立つようなその集合$U_{\lambda}$の元$\mathbf{a}_{\lambda}$全体$V$のうちその添数$\lambda$に自然数を割り当てた元の列$\left( \mathbf{a}_{\lambda_{m}} \right)_{m \in \mathbb{N}}$を考える。
\item
  仮定の8. より広い意味で収束する部分列$\left( \mathbf{a}_{\lambda_{m_{k}}} \right)_{k \in \mathbb{N}}$が存在して、$\lim_{k \rightarrow \infty}\mathbf{a}_{\lambda_{m_{k}}} = \mathbf{a} \in K$が成り立つ。
\item
  $\exists\lambda \in \varLambda_{m}\exists\varepsilon \in \mathbb{R}^{+}\exists N \in \mathbb{N}\forall k \in \mathbb{N}$に対し、$N \leq k$が成り立つなら、$\mathbf{a}_{\lambda_{m_{k}}} \in U\left( \mathbf{a},\varepsilon \right) \cap R \subseteq U_{\lambda}$が成り立つ。
\item
  14. は12. のその点列$\left( \mathbf{a}_{\lambda_{m}} \right)_{m \in \mathbb{N}}$のおき方に矛盾している。
\item
  8. ~15. よりその集合$K$が点列compactであるなら、その集合$K$がcompactである。
\end{enumerate}
\begin{proof}
$K \subseteq R \subseteq \mathbb{R}_{\infty}^{n}$なる集合$K$がその集合$R$でcompactであるとする。その集合$K$の任意の点列$\left( \mathbf{a}_{m} \right)_{m \in \mathbb{N}}$に対し、次のように集合$M\left( \mathbf{a},\varepsilon \right)$がおかれると、
\begin{align*}
M\left( \mathbf{a},\varepsilon \right) = \left\{ m \in \mathbb{N} \middle| \mathbf{a}_{m} \in U\left( \mathbf{a},\varepsilon \right) \cap R \right\}
\end{align*}
$\exists\mathbf{a} \in K\forall\varepsilon \in \mathbb{R}^{+}$に対し、その集合$M\left( \mathbf{a},\varepsilon \right)$は無限集合となる。実際、$\forall\mathbf{a} \in K\exists\varepsilon \in \mathbb{R}^{+}$に対し、その集合$M\left( \mathbf{a},\varepsilon \right)$が有限集合となるなら、その族$\left\{ U\left( \mathbf{a},\varepsilon \right) \cap R \right\}_{\mathbf{a} \in K}$は明らかにその集合$K$の開被覆であるので、仮定よりその集合$K$の有限集合である部分集合$L$が存在して、その族$\left\{ U\left( \mathbf{a},\varepsilon \right) \cap R \right\}_{\mathbf{a} \in L}$がその集合$K$の開被覆であることができる。このとき、$\forall m \in \mathbb{N}\exists\mathbf{a} \in L$に対し、$\mathbf{a}_{m} \in U\left( \mathbf{a},\varepsilon \right) \cap R$が成り立つので、$m \in M\left( \mathbf{a},\varepsilon \right)$が得られる。したがって、$\mathbb{N} \subseteq M\left( \mathbf{a},\varepsilon \right)$となりその集合$M\left( \mathbf{a},\varepsilon \right)$は無限集合となるが、これは仮定の、$\forall\mathbf{a} \in K\exists\varepsilon \in \mathbb{R}^{+}$に対し、その集合$M\left( \mathbf{a},\varepsilon \right)$が有限集合となることに矛盾している。特に、$\exists\mathbf{a} \in K\forall\varepsilon \in \mathbb{R}^{+}\forall k \in \mathbb{N}$に対し、その集合$M\left( \mathbf{a},\varepsilon \right) \setminus \varLambda_{k}$は無限集合となる。そこで、$m \in M\left( \mathbf{a},1 \right) \setminus \varLambda_{1}$なる自然数$m$を$m_{1}$として、自然数$m_{k}$が与えられたとき、$m \in M\left( \mathbf{a},\frac{1}{k + 1} \right) \setminus \varLambda_{m_{k}}$なる自然数$m$を$m_{k + 1}$とおくことにすると、これによって得られるその集合$\mathbb{N}$の元の列$\left( m_{k} \right)_{k \in \mathbb{N}}$は狭義単調増加する。実際、$\forall k \in \mathbb{N}$に対し、$m_{k + 1} \in M\left( \mathbf{a},\frac{1}{k + 1} \right) \setminus \varLambda_{m_{k}}$より$m_{k} < m_{k + 1}$が成り立つ。これにより、その点列$\left( \mathbf{a}_{m} \right)_{m \in \mathbb{N}}$の部分列$\left( \mathbf{a}_{m_{k}} \right)_{k \in \mathbb{N}}$が得られる。このとき、$\forall\varepsilon \in \mathbb{R}^{+}$に対し、定理\ref{4.1.1.22}、即ち、Archimedesの性質より$\exists K \in \mathbb{N}$に対し、$\frac{1}{K} < \varepsilon$が成り立ち、$\forall k \in \mathbb{N}$に対し、$K \leq k$が成り立つなら、$m_{K} \leq m_{k}$で次のようになる。
\begin{align*}
\mathbf{a}_{m_{k}} \in U\left( \mathbf{a},\frac{1}{k} \right) \cap R \subseteq U\left( \mathbf{a},\frac{1}{K} \right) \cap R \subseteq U\left( \mathbf{a},\varepsilon \right) \cap R
\end{align*}
これにより、$\lim_{k \rightarrow \infty}\mathbf{a}_{m_{k}} = \mathbf{a}$が成り立つ。よって、その集合$K$はその集合$R$で点列compactである。\par
逆に、その集合$K$がその集合$R$で点列compactであるかつ、その集合$K$のその集合$R$でのある開被覆$\left\{ U_{\lambda} \right\}_{\lambda \in \varLambda}$が存在して、その集合$\varLambda$のどの有限集合な部分集合$\varLambda'$に対してもその族$\left\{ U_{\lambda} \right\}_{\lambda \in \varLambda'}$がその集合$K$の開被覆になりえないとする。このとき、次のように集合$\mathfrak{L}$がおかれると、
\begin{align*}
\mathfrak{L} =\left\{ \varLambda'\in \mathfrak{P}(\varLambda) \middle| K \subseteq \bigcup_{\lambda \in \varLambda'} U_{\lambda} \right\}
\end{align*}
$\varLambda\in \mathfrak{L}$よりその集合$\mathfrak{L}$は空集合でない。このとき、その組$\left( \mathfrak{L, \supseteq} \right)$は順序集合となっているのは明らかである。さらに、その順序集合$\left( \mathfrak{L, \supseteq} \right)$の部分順序集合で空集合でない全順序集合となっているもの$\left( \mathfrak{M, \supseteq} \right)$が考えられれば、その集合$\bigcap_{} \mathfrak{M}$がその集合$\mathfrak{M}$の上限となっている。さらに、$\forall M\in \mathfrak{M}$に対し、$K \subseteq \bigcup_{\lambda \in M} U_{\lambda}$が成り立つので、次のようになることから、
\begin{align*}
K \subseteq \bigcap_{M\in \mathfrak{M}} {\bigcup_{\lambda \in M} U_{\lambda}} = \bigcup_{\lambda \in \bigcap_{M\in \mathfrak{M}} M} U_{\lambda} = \bigcup_{\lambda \in \bigcap_{} \mathfrak{M}} U_{\lambda}
\end{align*}
$\bigcap_{} \mathfrak{M}\in \mathfrak{L}$が得られる。これにより、その順序集合$\left( \mathfrak{L, \supseteq} \right)$は帰納的である。そこで、Zornの補題よりその順序集合$\left( \mathfrak{L, \supseteq} \right)$に極大元$\varLambda_{m}$が存在する、即ち、$\exists\varLambda_{m}\in \mathfrak{L\forall}\varLambda'\in \mathfrak{L}$に対し、$\varLambda_{m} \supset \varLambda'$が成り立たない。このとき、$\forall\lambda \in \varLambda_{m}\exists\mathbf{a}_{\lambda} \in U_{\lambda}\forall\mu \in \varLambda_{m} \setminus \left\{ \lambda \right\}$に対し、$\mathbf{a}_{\lambda} \notin U_{\mu}$が成り立つ。実際、$\exists\lambda \in \varLambda_{m}\forall\mathbf{a}_{\lambda} \in U_{\lambda}\exists\mu \in \varLambda_{m} \setminus \left\{ \lambda \right\}$に対し、$\mathbf{a}_{\lambda} \in U_{\mu}$が成り立つと仮定すると、もちろん、$\varLambda_{m} \supset \varLambda_{m} \setminus \left\{ \lambda \right\}$が成り立つかつ、次のようになることから、
\begin{align*}
K \subseteq \bigcup_{\mu \in \varLambda_{m}} U_{\mu} = \bigcup_{\mu \in \varLambda_{m} \setminus \left\{ \lambda \right\}} U_{\mu} \cup U_{\lambda} = \bigcup_{\mu \in \varLambda_{m} \setminus \left\{ \lambda \right\}} U_{\mu}
\end{align*}
$\varLambda_{m} \setminus \left\{ \lambda \right\}\in \mathfrak{L}$が得られるが、これはその集合$\varLambda_{m}$がその順序集合$\left( \mathfrak{L, \supseteq} \right)$に極大元であることに矛盾する。そこで、仮定よりその集合$\varLambda_{m}$は無限集合であるので、$\forall\mu \in \varLambda_{m} \setminus \left\{ \lambda \right\}$に対し、$\mathbf{a}_{\lambda} \notin U_{\mu}$が成り立つようなその集合$U_{\lambda}$の元$\mathbf{a}_{\lambda}$全体$V$のうちその添数$\lambda$に自然数を割り当てた元の列$\left( \mathbf{a}_{\lambda_{m}} \right)_{m \in \mathbb{N}}$、即ち、$\forall\mathbf{k} \in K$に対し、$K \subseteq \bigcup_{N \in \mathbb{N}} {U\left( \mathbf{k},\frac{1}{N} \right) \cap R}$が成り立つことから単射な写像$\left( \lambda_{m} \right)_{m \in \mathbb{N}}:\mathbb{N} \rightarrow \varLambda_{m}$が存在して、これと添数集合$\varLambda_{m}$の任意の添数$\lambda$のうち$\forall\mu \in \varLambda_{m} \setminus \left\{ \lambda \right\}$に対し、$\mathbf{a}_{\lambda} \notin U_{\mu}$が成り立つようなその集合$U_{\lambda}$の元$\mathbf{a}_{\lambda}$をどれか1つ割り当てる写像$\left( \mathbf{a}_{\lambda} \right)_{\lambda \in \varLambda_{m}}$を用いて次のような写像が考えられれば、
\begin{comment}
\begin{align*}
\left( \mathbf{a}_{\lambda_{m}} \right)_{m \in \mathbb{N}} = \left( \mathbf{a}_{\lambda} \right)_{\lambda \in \varLambda_{m}} \circ \left( \lambda_{m} \right)_{m \in \mathbb{N}} = \text{
\begin{tikzpicture}
  \node (x) at (0, 0) {};
  \node (a) at (0, 1) {$\mathbb{N} $};
  \node (b) at (2, 1) {$\varLambda_m $};
  \node (c) at (4, 1) {$V$};
  \node (d) at (0, 0) {$m$};
  \node (e) at (2, 0) {$\lambda_{m}$};
  \node (f) at (4, 0) {$\mathbf{a}_{\lambda_{m}}$};
  \node[rotate=90] (x) at (0, 0.5) {$\in $};
  \node[rotate=90] (x) at (2, 0.5) {$\in $};
  \node[rotate=90] (x) at (4, 0.5) {$\in $};
\end{tikzpicture}
}
\end{align*}
\end{comment}
\begin{align*}
\left( \mathbf{a}_{\lambda_{m}} \right)_{m \in \mathbb{N}} = \left( \mathbf{a}_{\lambda} \right)_{\lambda \in \varLambda_{m}} \circ \left( \lambda_{m} \right)_{m \in \mathbb{N}} =\begin{matrix}
\mathbb{N} & \rightarrow & \varLambda_{m} & \rightarrow & V \\
\text{\rotatebox[origin=c]{90}{$\in$}} & & \text{\rotatebox[origin=c]{90}{$\in$}} & & \text{\rotatebox[origin=c]{90}{$\in$}} \\
m & \mapsto & \lambda_{m} & \mapsto & \mathbf{a}_{\lambda_{m}} \\
\end{matrix}
\end{align*}
その集合$K$がその集合$R$で点列compactであるので、これのその集合$R$で広い意味で収束する部分列$\left( \mathbf{a}_{\lambda_{m_{k}}} \right)_{k \in \mathbb{N}}$が存在して、$\lim_{k \rightarrow \infty}\mathbf{a}_{\lambda_{m_{k}}} \in K$が成り立つ。この広い意味での極限値$\lim_{k \rightarrow \infty}\mathbf{a}_{\lambda_{m_{k}}}$が$\mathbf{a}$とおかれれば、$\exists\lambda \in \varLambda_{m}$に対し、$\mathbf{a} \in U_{\lambda}$が得られ、その集合$U_{\lambda}$は開集合なので、$\exists\varepsilon \in \mathbb{R}^{+}$に対し、$U\left( \mathbf{a},\varepsilon \right) \cap R \subseteq U_{\lambda}$が成り立つ。このとき、$\varepsilon$-$N$論法に注意すれば、$\exists N \in \mathbb{N}\forall k \in \mathbb{N}$に対し、$N \leq k$が成り立つなら、$\mathbf{a}_{\lambda_{m_{k}}} \in U\left( \mathbf{a},\varepsilon \right) \cap R \subseteq U_{\lambda}$が成り立つ。しかしながら、$\exists\lambda \in \varLambda_{m} \setminus \left\{ \lambda_{m_{k}} \right\}$に対し、$\mathbf{a}_{\lambda_{m_{k}}} \in U_{\lambda}$が成り立つので、$\mathbf{a}_{\lambda_{m_{k}}} \notin V$が得られる。これはその点列$\left( \mathbf{a}_{\lambda_{m}} \right)_{m \in \mathbb{N}}$のおき方に矛盾している。よって、その集合$K$がその集合$R$で点列compactであるなら、その集合$K$のその集合$R$での任意の開被覆$\left\{ U_{\lambda} \right\}_{\lambda \in \varLambda}$に対し、その集合$\varLambda$のある有限集合な部分集合$\varLambda'$に対するその族$\left\{ U_{\lambda} \right\}_{\lambda \in \varLambda'}$が存在して、これがその集合$K$の開被覆になる、即ち、その集合$K$がその集合$R$でcompactである。
\end{proof}
\begin{thm}\label{4.1.7.4}
$\varepsilon \in \mathbb{R}^{+}$、$\mathbf{a} \in \mathbb{R}_{\infty}^{n}$なる集合たち$\mathbb{R}_{\infty}^{n}$、$\overline{U}\left( \mathbf{a},\varepsilon \right)$、${}^{*}\mathbb{R}$、$[ 0,\infty]$はいづれも拡張$n$次元数空間$\mathbb{R}_{\infty}^{n}$でcompactでこれらの集合の任意の点列に対し、広い意味で収束する点列が存在してこれの広い意味での極限値がもとの集合に属する。
\end{thm}
\begin{proof} 定理\ref{4.1.5.9}より任意の拡張$n$次元数空間$\mathbb{R}_{\infty}^{n}$の点列$\left( \mathbf{a}_{m} \right)_{m \in \mathbb{N}}$に対し、広い意味で収束する部分列をもつので、その拡張$n$次元数空間$\mathbb{R}_{\infty}^{n}$は点列compactである。そこで、定理\ref{4.1.7.3}より拡張$n$次元数空間$\mathbb{R}_{\infty}^{n}$は拡張$n$次元数空間$\mathbb{R}_{\infty}^{n}$自身でcompactである。$\varepsilon \in \mathbb{R}^{+}$、$\mathbf{a} \in \mathbb{R}_{\infty}^{n}$なる集合たち$\mathbb{R}_{\infty}^{n}$、$\overline{U}\left( \mathbf{a},\varepsilon \right)$はいづれも拡張$n$次元数空間$\mathbb{R}_{\infty}^{n}$での閉集合であるから、定理\ref{4.1.7.1}より$\varepsilon \in \mathbb{R}^{+}$、$\mathbf{a} \in \mathbb{R}_{\infty}^{n}$なる集合たち$\mathbb{R}_{\infty}^{n}$、$\overline{U}\left( \mathbf{a},\varepsilon \right)$はいづれも拡張$n$次元数空間$\mathbb{R}_{\infty}^{n}$でcompactである。あとは定理\ref{4.1.7.3}から従う。集合たち${}^{*}\mathbb{R}$、$[ 0,\infty]$についても同様にして示される。
\end{proof}
%\hypertarget{ux5168ux6709ux754c}{%
\subsubsection{全有界}%\label{ux5168ux6709ux754c}}
\begin{dfn}
$K \subseteq R \subseteq \mathbb{R}_{\infty}^{n}$なる集合$K$の任意の点列$\left( \mathbf{a}_{m} \right)_{m \in \mathbb{N}}$に対し、これのその集合$R$で収束する部分列$\left( \mathbf{a}_{m_{k}} \right)_{k \in \mathbb{N}}$が存在するとき、その集合$K$はその集合$R$で全有界であるという\footnote{次の定理\ref{4.1.5.9}と比較されたい。
\begin{quote}
  任意の拡張$n$次元数空間$\mathbb{R}_{\infty}^{n}$の点列$\left( \mathbf{a}_{m} \right)_{m \in \mathbb{N}}$に対し、広い意味で収束する部分列をもつ。
\end{quote}
}。拡張$n$次元数空間$\mathbb{R}_{\infty}^{n}$のかわりに補完数直線${}^{*}\mathbb{R}$でおきかえても同様にして定義される。
\end{dfn}
\begin{thm}\label{4.1.7.5}
$K \subseteq R \subseteq \mathbb{R}^{n}$なる集合$K$が与えられたとき、その集合$K$がその集合$R$で全有界であるならそのときに限り、その集合$K$は有界である。\par
なお、このことは$\Rightarrow$の向きで有界でない集合$K$を考え、$\forall m \in \mathbb{N}$に対し、$m < \left\| \mathbf{a}_{m} \right\|$が成り立つようなその集合$K$の点列$\left( \mathbf{a}_{m} \right)_{m \in \mathbb{N}}$が定義されれば、$\lim_{m \rightarrow \infty}\mathbf{a}_{m} = \infty$が成り立つことにより、$\Leftarrow$の向きでBolzano-Weierstrassの定理より示される。
\end{thm}
\begin{proof}
$K \subseteq R \subseteq \mathbb{R}^{n}$なる集合$K$が与えられたとき、その集合$K$が有界でないなら、$\exists M \in \mathbb{R}^{+}\forall\mathbf{a} \in K$に対し、$\left\| \mathbf{a} \right\| \leq M$が成り立たない、即ち、$\forall M \in \mathbb{R}^{+}\exists\mathbf{a} \in K$に対し、$M < \left\| \mathbf{a} \right\|$が成り立つので、$\forall m \in \mathbb{N}$に対し、$m < \left\| \mathbf{a}_{m} \right\|$が成り立つようにしてその集合$K$の点列$\left( \mathbf{a}_{m} \right)_{m \in \mathbb{N}}$が定義されれば、$\forall\varepsilon \in \mathbb{R}^{+}$に対し、定理\ref{4.1.1.22}、即ち、Archimedesの性質より$\exists N \in \mathbb{N}$に対し、$\varepsilon \leq N$が成り立つので、$\forall m \in \mathbb{N}$に対し、$N \leq m$が成り立つなら、$\varepsilon \leq N \leq m \leq \left\| \mathbf{a}_{m} \right\|$が成り立つ。したがって、$\lim_{m \rightarrow \infty}\mathbf{a}_{m} = \infty$が成り立つので、定理\ref{4.1.4.11}よりその点列$\left( \mathbf{a}_{m} \right)_{m \in \mathbb{N}}$の任意の部分列についても$\lim_{k \rightarrow \infty}\mathbf{a}_{m_{k}} = \infty$が成り立つ。よって、その集合$K$は全有界でない。これより、その集合$K$が全有界であるなら、その集合$K$は有界である。\par
逆に、その集合$K$が有界であるなら、その集合$K$の任意の点列$\left( \mathbf{a}_{m} \right)_{m \in \mathbb{N}}$の値域はその集合$K$に含まれるので、その点列$\left( \mathbf{a}_{m} \right)_{m \in \mathbb{N}}$も有界である。このとき、$K \subseteq R \subseteq \mathbb{R}^{n}$が成り立つことに注意すれば、定理\ref{4.1.5.8}、即ち、Bolzano-Weierstrassの定理よりその点列$\left( \mathbf{a}_{m} \right)_{m \in \mathbb{N}}$は収束する部分列$\left( \mathbf{a}_{m_{k}} \right)_{k \in \mathbb{N}}$をもつので、その集合$K$は全有界である。
\end{proof}
\begin{thm}\label{4.1.7.6}
$K \subseteq R \subseteq \mathbb{R}^{n}$なる集合$K$が与えられたとき、その集合$K$がその集合$R$で有界な閉集合であるならそのときに限り、その集合$K$はその集合$R$で点列compactである。
\end{thm}\par
なお、このことは次の定理\ref{4.1.4.5}、定理\ref{4.1.4.11}、定理\ref{4.1.7.1}に注意すれば示される。
\begin{itemize}
\item
  $A \subseteq R \subseteq \mathbb{R}_{\infty}^{n}$なる集合$A$が与えられたとき、この集合$A$がその集合$R$における閉集合であるならそのときに限り、その集合$A$の任意の点列$\left( \mathbf{a}_{m} \right)_{m \in \mathbb{N}}$に対し、これが広い意味で収束するなら、その集合$R$での広い意味での極限値はその集合$A$に属する。
\item
  $R \subseteq \mathbb{R}_{\infty}^{n}$なる集合$R$が与えられたとする。その集合$R$の点列$\left( \mathbf{a}_{m} \right)_{m \in \mathbb{N}}$がその集合$R$で広い意味で点$\mathbf{a}$に収束するとき、その点列$\left( \mathbf{a}_{m} \right)_{m \in \mathbb{N}}$の任意の部分列もその集合$R$で広い意味でその点$\mathbf{a}$に収束する。
\item
  $K \subseteq \mathbb{R}^{n}$なる集合$K$が与えられたとき、その集合$K$が全有界であるならそのときに限り、その集合$K$は有界である。
\end{itemize}
\begin{proof}
$K \subseteq R \subseteq \mathbb{R}^{n}$なる集合$K$が与えられたとき、その集合$K$がその集合$R$で有界な閉集合であるなら、$R \subseteq \mathbb{R}^{n}$に注意して定理\ref{4.1.7.1}よりその集合$K$は全有界である。また、その集合$K$が閉集合であるので、定義より$\mathrm{cl}_{R}K = K$が成り立ち、定理\ref{4.1.4.5}より収束するその集合$K$の任意の点列$\left( \mathbf{a}_{m} \right)_{m \in \mathbb{N}}$に対し、$\lim_{m \rightarrow \infty}\mathbf{a}_{m} \in K$が成り立つ。ゆえに、全有界な集合$K$の任意の点列$\left( \mathbf{a}_{m} \right)_{m \in \mathbb{N}}$の収束する部分列$\left( \mathbf{a}_{m_{k}} \right)_{k \in \mathbb{N}}$について、定理\ref{4.1.4.11}より$\lim_{k \rightarrow \infty}\left( \mathbf{a}_{m_{k}} \right)_{k \in \mathbb{N}} = \lim_{m \rightarrow \infty}\mathbf{a}_{m} \in K$が成り立つので、その集合$K$はその集合$R$で点列compactである。\par
逆に、その集合$K$がその集合$R$で点列compactであるなら、定義よりその集合$K$は全有界であり定理\ref{4.1.7.1}よりその集合$K$は有界である。また、その集合$K$のある点列$\left( \mathbf{a}_{m} \right)_{m \in \mathbb{N}}$が存在して、$R \subseteq \mathbb{R}^{n}$よりこれがその集合$R$で収束するかつ、その集合$R$での極限値がその集合$K$に属さないとすれば、定理\ref{4.1.4.11}よりこれの部分列$\left( \mathbf{a}_{m_{k}} \right)_{k \in \mathbb{N}}$はその集合$R$で収束して$\lim_{k \rightarrow \infty}\mathbf{a}_{m_{k}} = \lim_{m \rightarrow \infty}\mathbf{a}_{m} \notin K$が成り立つことになりこれはその集合$K$がその集合$R$で点列compactであることに矛盾する。ゆえに、その集合$K$の任意の点列$\left( \mathbf{a}_{m} \right)_{m \in \mathbb{N}}$に対し、これがその集合$R$で収束するなら、その集合$R$での極限値がその集合$K$に属することになる。定理\ref{4.1.4.5}よりしたがって、その集合$K$は閉集合である。以上より、その集合$K$は有界な閉集合であることが示された。
\end{proof}
%\hypertarget{heine-borelux306eux88abux8986ux5b9aux7406}{%
\subsubsection{Heine-Borelの被覆定理}%\label{heine-borelux306eux88abux8986ux5b9aux7406}}
\begin{thm}[Heine-Borelの被覆定理]\label{4.1.7.7}
$K \subseteq R \subseteq \mathbb{R}^{n}$なる集合$K$がその集合$R$でcompactであるならそのときに限り、その集合$K$はその集合$R$で有界な閉集合である。この定理をHeine-Borelの被覆定理、Borel-Lebesgueの被覆定理という\footnote{これはムズいです。ゆっくりしていってね。}。
\end{thm}\par
この定理では$R \subseteq \mathbb{R}_{\infty}^{n}$でなく$R \subseteq \mathbb{R}^{n}$という仮定になっていることに注意されたい。定理\ref{4.1.7.3}、定理\ref{4.1.7.5}、定理\ref{4.1.7.6}よりこの定理から$R \subseteq \mathbb{R}^{n}$のとき、$\forall K \in \mathfrak{P}(R)$に対し、次のことは同値であることも分かる。
\begin{itemize}
\item
  その集合$K$がその集合$R$でcompactである。
\item
  その集合$K$がその集合$R$で点列compactである。
\item
  その集合$K$がその集合$R$で有界な閉集合である。
\item
  その集合$K$がその集合$R$で閉集合で、その集合$K$の任意の点列$\left( \mathbf{a}_{m} \right)_{m \in \mathbb{N}}$に対し、これのその集合$R$で収束する部分列$\left( \mathbf{a}_{m_{k}} \right)_{k \in \mathbb{N}}$が存在する。
\end{itemize}
この定理は次のようにして示される。
\begin{enumerate}
\item
  まず、集合$K$がcompactであるなら、その集合$K$は有界であることを対偶律で示す。
\item
  $K \subseteq R \subseteq \mathbb{R}^{n}$なる集合$K$がその集合$R$で有界でないと仮定する。
\item
  $U_{m} = U\left( \mathbf{0},m \right) \cap R$とおかれた集合$U_{m}$を用いた集合族$\left\{ U_{m} \right\}_{m \in \mathbb{N}}$はその集合$K$のその集合$R$における開被覆である。
\item
  任意のその集合$\mathbb{N}$の有限な部分集合$A$に対し、$N = \max A$とおかれれば、$\bigcup_{m \in A} U_{m} = U_{N}$が成り立つ。
\item
  2. より$\exists\mathbf{k} \in K$に対し、$N < \left\| \mathbf{k} \right\|$が成り立ち3. より$K \subseteq \bigcup_{m \in A} U_{m}$が成り立たない。
\item
  1. ~5. より集合$K$がcompactであるなら、その集合$K$が有界である。
\item
  次に、その集合$K$がcompactであるなら、その集合$K$は閉集合であることを示す。
\item
  $\forall\mathbf{a} \in R \setminus K\forall\mathbf{b} \in K$に対し、$\mathbf{a} \neq \mathbf{b}$が成り立つので、次式で定義される2つの集合たち$U_{\mathbf{b}}$、$V_{\mathbf{b}}$が考えられる。
\begin{align*}
U_{\mathbf{b}} = \left\{ \mathbf{k} \in R \middle| \left\| \mathbf{k} - \mathbf{a} \right\| < \frac{1}{2}\left\| \mathbf{a} - \mathbf{b} \right\| \right\},\ \ V_{\mathbf{b}} = \left\{ \mathbf{k} \in R \middle| \left\| \mathbf{k} - \mathbf{b} \right\| < \frac{1}{2}\left\| \mathbf{a} - \mathbf{b} \right\| \right\}
\end{align*}
\item
  8. より$U_{\mathbf{b}} \cap V_{\mathbf{b}} = \emptyset$が成り立つ。
\item
  $K \subseteq \bigcup_{\mathbf{b} \in L} V_{\mathbf{b}}$なるその集合$K$の有限な部分集合$L$が存在して、$K \cap \bigcap_{\mathbf{b} \in L} U_{\mathbf{b}} \subseteq \emptyset$が成り立つことを示すことよりその集合$R \setminus K$は開集合である。
\item
  10. よりその集合$K$は閉集合である。
\item
  7. ~11. よりその集合$K$がcompactであるなら、その集合$K$は閉集合であることがいえる。
\item
  6. と12. よりその集合$K$がcompactであるなら、その集合$K$は有界な閉集合であることがいえる。
\item
  その集合$K$が有界な閉集合であるなら、その集合$K$はcompactであることを背理法で示す。
\item
  その集合$K$は有界であるから、$\exists a_{i1},b_{i1} \in \mathbb{R}$に対し次式が成り立つ。
\begin{align*}
K \subseteq \prod_{i \in \varLambda_{n}} \left[ a_{i1},b_{i1} \right] \cap R
\end{align*}
\item
  その集合$\prod_{i \in \varLambda_{n}} \left[ a_{i1},b_{i1} \right]$の区間たち$\left[ a_{i1},b_{i1} \right]$をそれぞれ2つの区間たち$\left[ a_{i1},\frac{1}{2}\left( a_{i1} + b_{i1} \right) \right] 、\left[ \frac{1}{2}\left( a_{i1} + b_{i1} \right),b_{i1} \right]$に分割する。
\item
  $K = \prod_{i \in \varLambda_{n}} \left[ a_{i1},b_{i1} \right] \cap K$が成り立つことに注意してその集合$\prod_{i \in \varLambda_{n}} \left[ a_{i1},b_{i1} \right]$から$2^{n}$通りに分割されたそれらの区間たち$I$のうちある区間が存在して、これとその集合$K$との共通部分$K \cap I$が、その集合$K$のある開被覆$\left\{ U_{\lambda} \right\}_{\lambda \in \varLambda}$が与えられたとき、その集合$\varLambda$のどの有限な部分集合$\varLambda'$に対しても$K \cap I \subseteq \bigcup_{\lambda \in \varLambda'} U_{\lambda}$が成り立たないことが背理法で示される。
\item
  16. ~17. の議論を繰り返して$m \in \mathbb{N}$なる集合$\prod_{i \in \varLambda_{n}} \left[ a_{im},b_{im} \right]$が得られる。以下、$\mathbf{a}_{m} = \left( a_{im} \right)_{i \in \varLambda_{n}}$、$\mathbf{b}_{m} = \left( b_{im} \right)_{i \in \varLambda_{n}}$とおく。
\item
  15. ~18. より次のことが分かる。
\begin{itemize}
\item
  $\forall m \in \mathbb{N}$に対し、$\prod_{i \in \varLambda_{n}} \left[ a_{i,m + 1},b_{i,m + 1} \right] \subseteq \prod_{i \in \varLambda_{n}} \left[ a_{im},b_{im} \right]$が成り立つ。
\item
  その集合$\prod_{i \in \varLambda_{n}} \left[ a_{i,m + 1},b_{i,m + 1} \right]$のある区間$\left[ a_{i,m + 1},b_{i,m + 1} \right]$の長さ$\left| b_{i,m + 1} - a_{i,m + 1} \right|$がその区間$\left[ a_{i,m + 1},b_{i,m + 1} \right]$に対応する区間$\left[ a_{im},b_{im} \right]$の長さ$\left| b_{im} - a_{im} \right|$の$\frac{1}{2}$倍である。
\item
  その集合$K \cap \prod_{i \in \varLambda_{n}} \left[ a_{im},b_{im} \right]$が$K_{m}$とおかれると、その集合$K$のある開被覆$\left\{ U_{\lambda} \right\}_{\lambda \in \varLambda}$が与えられたとき、その集合$\varLambda$のどの有限な部分集合$\varLambda'$に対しても$K \cap I \subseteq \bigcup_{\lambda \in \varLambda'} U_{\lambda}$が成り立たない。
\end{itemize}
\item
  $\forall i \in \varLambda_{n}$に対し、その区間たち$\left[ a_{im},b_{im} \right]$は区間縮小法より$\bigcap_{m \in \mathbb{N}} \left[ a_{im},b_{im} \right] = \left\{ a_{i} \right\}$なる実数$a_{i}$が存在する。
\item
  $\mathbf{c}_{m} \in K \cap \prod_{i \in \varLambda_{n}} \left[ a_{im},b_{im} \right]$なる点列$\left( \mathbf{c}_{m} \right)_{m \in \mathbb{N}}$に対し、$\lim_{m \rightarrow \infty}\mathbf{c}_{m} = \mathbf{a} = \left( a_{i} \right)_{i \in \varLambda_{n}}$が成り立つ。
\item
  その集合$K$のその開被覆$\left\{ U_{\lambda} \right\}_{\lambda \in \varLambda}$を用いて$\mathbf{a} \in \mathrm{cl}_{R}K = K \subseteq \bigcup_{\lambda \in \varLambda} U_{\lambda}$が成り立つ。
\item
  $\exists\lambda \in \varLambda\exists\varepsilon \in \mathbb{R}^{+}$に対し、$U\left( \mathbf{a},\varepsilon \right) \subseteq U_{\lambda}$が成り立つ。
\item
  $\exists i \in \varLambda_{n}$に対し、$\left| b_{i,m + 1} - a_{i,m + 1} \right| = \frac{1}{2}\left| b_{im} - a_{im} \right|$が成り立つことにより$\lim_{m \rightarrow \infty}\left( a_{im} - b_{im} \right) = 0$が成り立つ。
\item
  21. と、$\forall i \in \varLambda_{n}$に対し、$a_{im} \leq c_{i} \leq b_{im} \Rightarrow \left| c_{i} - a_{i} \right| \leq \left| a_{im} - b_{im} \right|$が成り立つことに注意すれば、$\forall\mathbf{c} \in \mathbb{R}_{\infty}^{n}$に対し、$\mathbf{c} \in \prod_{i \in \varLambda_{n}} \left[ a_{im},b_{im} \right]$が成り立つなら、$\forall i \in \varLambda_{n}$に対し、$\left| c_{i} - a_{i} \right| \leq \left| a_{im} - b_{im} \right|$が成り立つ。
\item
  24. より$\varepsilon$-$N$論法に注意すれば、次式が成り立つ。
\begin{align*}
\prod_{i \in \varLambda_{n}} \left[ a_{im},b_{im} \right] \subseteq U\left( \mathbf{a},\varepsilon \right)
\end{align*}
\item
  $K \cap \prod_{i \in \varLambda_{n}} \left[ a_{im},b_{im} \right] \subseteq \prod_{i \in \varLambda_{n}} \left[ a_{im},b_{im} \right]$が成り立つことに注意すれば、23. と26. を用いて$K \cap \prod_{i \in \varLambda_{n}} \left[ a_{im},b_{im} \right] \subseteq U_{\lambda}$が成り立つ。
\item
  27. が19. に矛盾している。
\item
  14. ~28. より背理法によりその集合$K$が有界な閉集合であるなら、その集合$K$はcompactであることがいえる。
\item
  13. と29. よりHeine-Borelの被覆定理が成り立つ。
\end{enumerate}
\begin{proof}
$K \subseteq R \subseteq \mathbb{R}^{n}$なる集合$K$がその集合$R$で有界でないとする。このとき、$U_{m} = U\left( \mathbf{0},m \right) \cap R$とおかれた集合$U_{m}$を用いた集合族$\left\{ U_{m} \right\}_{m \in \mathbb{N}}$が与えられたとき、それらの集合たち$U_{m}$は開集合であり、$\forall\mathbf{k} \in K$に対し、もちろん、$\mathbf{k} \in R$が成り立ち、Archimedesの性質より$\exists m \in \mathbb{N}$に対し、$\left\| \mathbf{k} \right\| < m$が成り立つので、$\mathbf{k} \in U_{m}$が得られる。ゆえに、$K \subseteq \bigcup_{m \in \mathbb{N}} U_{m}$が得られその集合族$\left\{ U_{m} \right\}_{m \in \mathbb{N}}$はその集合$K$の開被覆である。任意のその集合$\mathbb{N}$の有限な部分集合$A$に対し、$N = \max A$とおかれれば、$\forall m,n \in A$に対し、$m \leq n$が成り立つなら、$U_{m} \subseteq U_{n}$が成り立つので、$\bigcup_{m \in A} U_{m} = U_{N}$が成り立つ。そこで仮定より、$\exists M \in \mathbb{R}^{+}\forall\mathbf{k} \in K$に対し、$\left\| \mathbf{k} \right\| \leq M$が成り立たない、即ち、$\forall M \in \mathbb{R}^{+}\exists\mathbf{k} \in K$に対し、$M < \left\| \mathbf{k} \right\|$が成り立つので、$\exists\mathbf{k} \in R$に対し、$\mathbf{k} \in K$かつ$N < \left\| \mathbf{k} \right\|$が成り立つ。ここで、その集合$U_{N}$の定義より$\mathbf{k} \notin U_{N}$が成り立つ。したがって、$\exists\mathbf{k} \in R$に対し、$\mathbf{k} \in K$かつ$\mathbf{k} \notin \bigcup_{m \in A} U_{m} = U_{N}$が成り立つ、即ち、$K \subseteq \bigcup_{m \in A} U_{m}$が成り立たないので、その集合$K$はその集合$R$でcompactでない。対偶律より、その集合$K$がその集合$R$でcompactであるなら、その集合はその集合$R$で有界である。\par
また、その集合$K$がその集合$R$でcompactであるなら、$K = \emptyset$かつ$K \subset R$のとき、$\forall\mathbf{a} \in R \setminus K\forall\mathbf{b} \in K$に対し、$\mathbf{a} \neq \mathbf{b}$が成り立つので、次式で定義される2つの集合たち$U_{\mathbf{b}}$、$V_{\mathbf{b}}$が考えられることができる。
\begin{align*}
U_{\mathbf{b}} = \left\{ \mathbf{k} \in R \middle| \left\| \mathbf{k} - \mathbf{a} \right\| < \frac{1}{2}\left\| \mathbf{a} - \mathbf{b} \right\| \right\},\ \ V_{\mathbf{b}} = \left\{ \mathbf{k} \in R \middle| \left\| \mathbf{k} - \mathbf{b} \right\| < \frac{1}{2}\left\| \mathbf{a} - \mathbf{b} \right\| \right\}
\end{align*}
これらは開球とその集合$R$の共通部分であるから、定義より明らかにその集合$R$での開集合である。ここで、$\exists\mathbf{c} \in R$に対し、$\mathbf{c} \in U_{\mathbf{b}} \cap V_{\mathbf{b}}$が成り立つと仮定すると、三角不等式より次のようになる。
\begin{align*}
\left\| \mathbf{a} - \mathbf{b} \right\| &= \left\| \mathbf{a} - \mathbf{c} - \mathbf{b} + \mathbf{c} \right\|\\
&\leq \left\| \mathbf{a} - \mathbf{c} \right\| + \left\| \mathbf{c} - \mathbf{b} \right\|\\
&< \frac{1}{2}\left\| \mathbf{a} - \mathbf{b} \right\| + \frac{1}{2}\left\| \mathbf{a} - \mathbf{b} \right\| = \left\| \mathbf{a} - \mathbf{b} \right\|
\end{align*}
これは矛盾している。したがって、$U_{\mathbf{b}} \cap V_{\mathbf{b}} = \emptyset$が成り立つ。その集合$K$はその集合$R$でcompactであったので、$\mathbf{b} \in V_{\mathbf{b}}$が成り立つことにより、$K \subseteq \bigcup_{\mathbf{b} \in K} V_{\mathbf{b}}$が成り立つことから、上の議論により、その集合族$\left\{ V_{\mathbf{b}} \right\}_{\mathbf{b} \in K}$はその集合の開被覆であるので、$K \subseteq \bigcup_{\mathbf{b} \in L} V_{\mathbf{b}}$なるその集合$K$の有限な部分集合$L$が存在する。このとき、各集合たち$U_{\mathbf{b}}$がその点$\mathbf{a}$の$\frac{1}{2}\left\| \mathbf{a} - \mathbf{b} \right\|$近傍とその集合$R$との共通部分であることから、その集合$\bigcap_{\mathbf{b} \in L} U_{\mathbf{b}}$はその集合$R$での開集合であるかつ、$\forall\mathbf{b} \in K$に対し、$U_{\mathbf{b}} \cap V_{\mathbf{b}} = \emptyset$が成り立つので、次のようになる。
\begin{align*}
K \cap \bigcap_{\mathbf{b} \in L} U_{\mathbf{b}} &\subseteq \bigcup_{\mathbf{b} \in L} V_{\mathbf{b}} \cap \bigcap_{\mathbf{b} \in L} U_{\mathbf{b}}\\
&\subseteq \bigcup_{\mathbf{b} \in L} V_{\mathbf{b}} \cap \bigcup_{\mathbf{b} \in L} U_{\mathbf{b}}\\
&= \bigcup_{\mathbf{b} \in L} \left( V_{\mathbf{b}} \cap U_{\mathbf{b}} \right)\\
&= \bigcup_{\mathbf{b} \in L} \emptyset = \emptyset
\end{align*}
したがって、$\bigcap_{\mathbf{b} \in L} U_{\mathbf{b}} \subseteq R \setminus K$が得られるので、その集合$R \setminus K$はその集合$R$で開集合であり定理\ref{4.1.3.12}よりその集合$K$はその集合$R$で閉集合である。\par
したがって、その集合$K$がその集合$R$でcompactであるなら、その集合$K$はその集合$R$で有界な閉集合である。\par
逆に、その集合$K$がその集合$R$で有界な閉集合であるかつ、その集合$K$がその集合$R$でcompactでない、即ち、その集合$K$のある開被覆$\left\{ U_{\lambda} \right\}_{\lambda \in \varLambda}$が存在し、これからどのように有限個の集合$U_{\lambda}$を取り出しても、これらはその集合$K$の開被覆であることができないと仮定する。その集合$K$は有界であるから、$\exists M \in \mathbb{R}^{+}$に対し、$K \subseteq U\left( \mathbf{0},M \right) \cap R$が成り立つ。ここで、$\forall\mathbf{a} \in \mathbb{R}_{\infty}^{n}$に対し、$\mathbf{a} \in U\left( \mathbf{0},M \right) \cap R$が成り立つなら、$\left\| \mathbf{a} \right\| < M$が成り立つので、$\mathbf{a} = \left( a_{i} \right)_{i \in \varLambda_{n}}$とおかれれば、$\forall i \in \varLambda_{n}$に対し、次のようになる。
\begin{align*}
\left| a_{i} \right|^{2} \leq \sum_{i \in \varLambda_{n}} \left| a_{i} \right|^{2} = \left\| \mathbf{a} \right\| < M
\end{align*}
したがって、$- M \leq a_{i} \leq M$が得られるので、この実数たち$- M$、$M$がそれぞれ$a_{i1}$、$b_{i1}$とおかれれば、次式が成り立つ。
\begin{align*}
K \subseteq U\left( \mathbf{0},M \right) \cap R \subseteq \prod_{i \in \varLambda_{n}} \left[ a_{i1},b_{i1} \right] \cap R
\end{align*}
ここで、その集合$\prod_{i \in \varLambda_{n}} \left[ a_{i1},b_{i1} \right]$の区間たち$\left[ a_{i1},b_{i1} \right]$がそれぞれ2つの区間たち$\left[ a_{i1},\frac{1}{2}\left( a_{i1} + b_{i1} \right) \right]$、$\left[ \frac{1}{2}\left( a_{i1} + b_{i1} \right),b_{i1} \right]$で分割されると、その集合$\prod_{i \in \varLambda_{n}} \left[ a_{i1},b_{i1} \right]$は$2^{n}$通りの区間たちに分割される。全ての$2^{n}$通りに分割されたこれらの区間たち$I$とその集合$K$との共通部分$K \cap I$が、その集合$K$の任意の開被覆$\left\{ U_{\lambda} \right\}_{\lambda \in \varLambda}$が与えられたとき、その集合$\varLambda$の有限な部分集合$\varLambda_{I}$が存在して$K \cap I \subseteq \bigcup_{\lambda \in \varLambda_{I}} U_{\lambda}$が成り立つことができるとすると、区間たち$I$全体の集合が$\mathfrak{I}$とおかれれば、これは有限集合で$\prod_{i \in \varLambda_{n}} \left[ a_{im},b_{im} \right] = \bigcup_{I\in \mathfrak{I}} I$が成り立つので、次のようになる。
\begin{align*}
K \cap \prod_{i \in \varLambda_{n}} \left[ a_{i1},b_{i1} \right] = K \cap \bigcup_{I\in \mathfrak{I}} I = \bigcup_{I\in \mathfrak{I}} (K \cap I) \subseteq \bigcup_{I\in \mathfrak{I}} {\bigcup_{\lambda \in \varLambda_{I}} U_{\lambda}}
\end{align*}
ここで、$K = \prod_{i \in \varLambda_{n}} \left[ a_{i1},b_{i1} \right] \cap K$が成り立つので、次のようになる。
\begin{align*}
K = K \cap \prod_{i \in \varLambda_{n}} \left[ a_{i1},b_{i1} \right] \subseteq \bigcup_{I\in \mathfrak{I}} {\bigcup_{\lambda \in \varLambda_{I}} U_{\lambda}}
\end{align*}
これにより、その集合$K$の任意の開被覆$\left\{ U_{\lambda} \right\}_{\lambda \in \varLambda}$が与えられたとき、その集合$\varLambda$の有限な部分集合$\varLambda'$が存在して、$K \subseteq \bigcup_{\lambda \in \varLambda'} U_{\lambda}$が成り立つ。しかしながら、これはその集合$K$がcompactでないことに矛盾する。したがって、その集合$\prod_{i \in \varLambda_{n}} \left[ a_{i1},b_{i1} \right]$から$2^{n}$通りに分割されたそれらの区間たち$I$のうちある区間が存在して、これとその集合$K$との共通部分$K \cap I$が、その集合$K$のある開被覆$\left\{ U_{\lambda} \right\}_{\lambda \in \varLambda}$が与えられたとき、その集合$\varLambda$のどの有限な部分集合$\varLambda'$に対しても$K \cap I \subseteq \bigcup_{\lambda \in \varLambda'} U_{\lambda}$が成り立たない。このような区間$I$を1つとり$\prod_{i \in \varLambda_{n}} \left[ a_{i2},b_{i2} \right]$とする。以下同様にして、$m \in \mathbb{N}$なる集合$\prod_{i \in \varLambda_{n}} \left[ a_{im},b_{im} \right]$が得られる。以下、$\mathbf{a}_{m} = \left( a_{im} \right)_{i \in \varLambda_{n}}$、$\mathbf{b}_{m} = \left( b_{im} \right)_{i \in \varLambda_{n}}$とおかれよう。このとき、次のことが成り立つ。
\begin{itemize}
\item
  $\forall m \in \mathbb{N}$に対し、$\prod_{i \in \varLambda_{n}} \left[ a_{i,m + 1},b_{i,m + 1} \right] \subseteq \prod_{i \in \varLambda_{n}} \left[ a_{im},b_{im} \right]$が成り立つ。
\item
  その集合$\prod_{i \in \varLambda_{n}} \left[ a_{i,m + 1},b_{i,m + 1} \right]$のある区間$\left[ a_{i,m + 1},b_{i,m + 1} \right]$の長さ$\left| b_{i,m + 1} - a_{i,m + 1} \right|$がその区間$\left[ a_{i,m + 1},b_{i,m + 1} \right]$に対応する区間$\left[ a_{im},b_{im} \right]$の長さ$\left| b_{im} - a_{im} \right|$の$\frac{1}{2}$倍である。
\item
  その集合$K \cap \prod_{i \in \varLambda_{n}} \left[ a_{im},b_{im} \right]$が$K_{m}$とおかれると、その集合$K$のある開被覆$\left\{ U_{\lambda} \right\}_{\lambda \in \varLambda}$が与えられたとき、その集合$\varLambda$のどの有限な部分集合$\varLambda'$に対しても$K \cap I \subseteq \bigcup_{\lambda \in \varLambda'} U_{\lambda}$が成り立たない。
\end{itemize}
$\forall i \in \varLambda_{n}$に対し、その区間たち$\left[ a_{im},b_{im} \right]$は区間縮小法より$\bigcap_{m \in \mathbb{N}} \left[ a_{im},b_{im} \right] = \left\{ a_{i} \right\}$なる実数$a_{i}$が存在する。以下、$\mathbf{a} = \left( a_{i} \right)_{i \in \varLambda_{n}}$とおかれよう。したがって、$\bigcap_{m \in \mathbb{N}} {\prod_{i \in \varLambda_{n}} \left[ a_{im},b_{im} \right]} = \left\{ \mathbf{a} \right\}$が成り立ち、$\mathbf{c}_{m} \in K \cap \prod_{i \in \varLambda_{n}} \left[ a_{im},b_{im} \right]$なる点列$\left( \mathbf{c}_{m} \right)_{m \in \mathbb{N}}$に対し、$\lim_{m \rightarrow \infty}\mathbf{c}_{m} = \mathbf{a}$が成り立つので、$\mathbf{a} \in \mathrm{cl}_{R}K$が成り立ち、その集合$K$は閉集合であったので、その集合$K$のその開被覆$\left\{ U_{\lambda} \right\}_{\lambda \in \varLambda}$を用いて$\mathbf{a} \in \mathrm{cl}_{R}K = K \subseteq \bigcup_{\lambda \in \varLambda} U_{\lambda}$が成り立つ。したがって、$\exists\lambda \in \varLambda$に対し、$\mathbf{a} \in U_{\lambda}$が成り立ち、$\forall\lambda \in \varLambda$に対し、それらの集合たち$U_{\lambda}$は開集合であったので、$\exists\varepsilon \in \mathbb{R}^{+}$に対し、$U\left( \mathbf{a},\varepsilon \right) \subseteq U_{\lambda}$が成り立つ。その集合$\prod_{i \in \varLambda_{n}} \left[ a_{i,m + 1},b_{i,m + 1} \right]$のある区間$\left[ a_{i,m + 1},b_{i,m + 1} \right]$の長さ$\left| b_{i,m + 1} - a_{i,m + 1} \right|$がその区間$\left[ a_{i,m + 1},b_{i,m + 1} \right]$に対応する区間$\left[ a_{im},b_{im} \right]$の長さ$\left| b_{im} - a_{im} \right|$の$\frac{1}{2}$倍であったので、次式が成り立つ。
\begin{align*}
\left| b_{i,m + 1} - a_{i,m + 1} \right| = \frac{1}{2}\left| b_{im} - a_{im} \right| &\Rightarrow \left| b_{im} - a_{im} \right| = \left( \frac{1}{2} \right)^{m - 1}\left| b_{i1} - a_{i1} \right|\\
&\Rightarrow \lim_{m \rightarrow \infty}\left| b_{im} - a_{im} \right| = \lim_{m \rightarrow \infty}{\left( \frac{1}{2} \right)^{m - 1}\left| b_{i1} - a_{i1} \right|} = 0\\
&\Leftrightarrow \lim_{m \rightarrow \infty}\left( a_{im} - b_{im} \right) = 0
\end{align*}
ゆえに、$\lim_{m \rightarrow \infty}\left( \mathbf{a}_{m} - \mathbf{b}_{m} \right) = \mathbf{0}$が成り立つ。したがって、$\mathbf{c} = \left( c_{i} \right)_{i \in \varLambda_{n}}$とおかれれば、$\forall\mathbf{c} \in \mathbb{R}_{\infty}^{n}$に対し、$\mathbf{c} \in \prod_{i \in \varLambda_{n}} \left[ a_{im},b_{im} \right]$が成り立つなら、$\forall i \in \varLambda_{n}$に対し、次のようになる。
\begin{align*}
c_{i} \in \left[ a_{im},b_{im} \right] &\Leftrightarrow a_{im} \leq c_{i} \leq b_{im}\\
&\Rightarrow a_{im} - b_{im} \leq c_{i} - a_{i} \leq - a_{im} + b_{im}\\
&\Leftrightarrow \left| c_{i} - a_{i} \right| \leq \left| a_{im} - b_{im} \right|
\end{align*}
ここで、$\lim_{m \rightarrow \infty}\left( a_{im} - b_{im} \right) = 0$が成り立つので、$\varepsilon$-$N$論法より$\forall\varepsilon \in \mathbb{R}^{+}\exists N_{i} \in \mathbb{N}$に対し、$N = \max\left\{ N_{i} \right\}_{i \in \varLambda_{n}}$とおかれれば、$\forall m \in \mathbb{N}$に対し、$N \leq m$が成り立つなら、$\left| a_{im} - b_{im} \right| < \frac{\varepsilon}{\sqrt{n}}$が成り立つので、次のようになる。
\begin{align*}
\sum_{i \in \varLambda_{n}} \left| c_{i} - a_{i} \right|^{2} \leq \sum_{i \in \varLambda_{n}} \left| b_{im} - a_{im} \right|^{2} < \sum_{i \in \varLambda_{n}} \left( \frac{\varepsilon}{\sqrt{n}} \right)^{2} &\Leftrightarrow \left\| \mathbf{c} - \mathbf{a} \right\|^{2} \leq \left\| \mathbf{b}_{m} - \mathbf{a}_{m} \right\|^{2} < \varepsilon^{2}\\
&\Leftrightarrow \left\| \mathbf{c} - \mathbf{a} \right\| \leq \left\| \mathbf{b}_{m} - \mathbf{a}_{m} \right\| < \varepsilon\\
&\Rightarrow \left\| \mathbf{c} - \mathbf{a} \right\| < \varepsilon\\
&\Leftrightarrow \mathbf{c} \in U\left( \mathbf{a},\varepsilon \right)
\end{align*}
したがって、次式が成り立つ。
\begin{align*}
\prod_{i \in \varLambda_{n}} \left[ a_{im},b_{im} \right] \subseteq U\left( \mathbf{a},\varepsilon \right)
\end{align*}
ここで、これと明らかに次式が成り立つかつ、
\begin{align*}
K \cap \prod_{i \in \varLambda_{n}} \left[ a_{im},b_{im} \right] \subseteq \prod_{i \in \varLambda_{n}} \left[ a_{im},b_{im} \right]
\end{align*}
$\exists\varepsilon \in \mathbb{R}^{+}$に対し、$U\left( \mathbf{a},\varepsilon \right) \subseteq U_{\lambda}$が成り立つのであったので、次式が成り立つ。
\begin{align*}
K \cap \prod_{i \in \varLambda_{n}} \left[ a_{im},b_{im} \right] \subseteq \prod_{i \in \varLambda_{n}} \left[ a_{im},b_{im} \right] \subseteq U\left( \mathbf{a},\varepsilon \right) \subseteq U_{\lambda}
\end{align*}
ここで、その集合$U_{\lambda}$は開集合で1つと有限個であるから、その集合族$\left\{ U_{\lambda} \right\}$はその集合$K \cap \prod_{i \in \varLambda_{n}} \left[ a_{im},b_{im} \right]$の開被覆である。これは、その集合$K$のある開被覆$\left\{ U_{\lambda} \right\}_{\lambda \in \varLambda}$が与えられたとき、その集合$\varLambda$のどの有限な部分集合$\varLambda'$に対しても$K \cap I \subseteq \bigcup_{\lambda \in \varLambda'} U_{\lambda}$が成り立たないことに矛盾する。背理法によりその集合$K$が有界な閉集合であるなら、その集合$K$がcompactである。\par
よって、その集合$K$がその集合$R$でcompactであるならそのときに限り、その集合$K$はその集合$R$で有界な閉集合である。
\end{proof}
\begin{thebibliography}{50}
  \bibitem{1}
  杉浦光夫, 解析入門I, 東京大学出版社, 1980. 第34刷 p64-73 ISBN978-4-13-062005-5
  \bibitem{2}
  室田一雄. "基礎数理 室田 有界閉とコンパクト". 東京大学. \url{http://www.misojiro.t.u-tokyo.ac.jp/~murota/lect-kisosuri/compactRn041202.pdf} (2020-8-25 取得)
  \bibitem{3}
  松坂和夫, 集合・位相入門, 岩波書店, 1968. 新装版第2刷 p90-94,108-111,186-190,208-223,258-268 ISBN978-4-00-029871-1
  \bibitem{4}
  金子晃. "第6章 コンパクト性". アレクセイカーネンコ応用数理研究室. \url{http://www.kanenko.com/~kanenko/KOUGI/Iso/resume4.pdf} (2022-8-7 2:14 閲覧)
\end{thebibliography}
\end{document}
