\documentclass[dvipdfmx]{jsarticle}
\setcounter{section}{4}
\setcounter{subsection}{5}
\usepackage{xr}
\externaldocument{2.4.5}
\usepackage{amsmath,amsfonts,amssymb,array,comment,mathtools,url,docmute}
\usepackage{longtable,booktabs,dcolumn,tabularx,mathtools,multirow,colortbl,xcolor}
\usepackage[dvipdfmx]{graphics}
\usepackage{bmpsize}
\usepackage{amsthm}
\usepackage{enumitem}
\setlistdepth{20}
\renewlist{itemize}{itemize}{20}
\setlist[itemize]{label=•}
\renewlist{enumerate}{enumerate}{20}
\setlist[enumerate]{label=\arabic*.}
\setcounter{MaxMatrixCols}{20}
\setcounter{tocdepth}{3}
\newcommand{\rotin}{\text{\rotatebox[origin=c]{90}{$\in $}}}
\renewcommand{\thesection}{第\arabic{section}部}
\renewcommand{\thesubsection}{\arabic{section}.\arabic{subsection}}
\renewcommand{\thesubsubsection}{\arabic{section}.\arabic{subsection}.\arabic{subsubsection}}
\everymath{\displaystyle}
\allowdisplaybreaks[4]
\usepackage{vtable}
\theoremstyle{definition}
\newtheorem{thm}{定理}[subsection]
\newtheorem*{thm*}{定理}
\newtheorem{dfn}{定義}[subsection]
\newtheorem*{dfn*}{定義}
\newtheorem{axs}[dfn]{公理}
\newtheorem*{axs*}{公理}
\renewcommand{\headfont}{\bfseries}
\makeatletter
  \renewcommand{\section}{%
    \@startsection{section}{1}{\z@}%
    {\Cvs}{\Cvs}%
    {\normalfont\huge\headfont\raggedright}}
\makeatother
\makeatletter
  \renewcommand{\subsection}{%
    \@startsection{subsection}{2}{\z@}%
    {0.5\Cvs}{0.5\Cvs}%
    {\normalfont\LARGE\headfont\raggedright}}
\makeatother
\makeatletter
  \renewcommand{\subsubsection}{%
    \@startsection{subsubsection}{3}{\z@}%
    {0.4\Cvs}{0.4\Cvs}%
    {\normalfont\Large\headfont\raggedright}}
\makeatother
\makeatletter
\renewenvironment{proof}[1][\proofname]{\par
  \pushQED{\qed}%
  \normalfont \topsep6\p@\@plus6\p@\relax
  \trivlist
  \item\relax
  {
  #1\@addpunct{.}}\hspace\labelsep\ignorespaces
}{%
  \popQED\endtrivlist\@endpefalse
}
\makeatother
\renewcommand{\proofname}{\textbf{証明}}
\usepackage{tikz,graphics}
\usepackage[dvipdfmx]{hyperref}
\usepackage{pxjahyper}
\hypersetup{
 setpagesize=false,
 bookmarks=true,
 bookmarksdepth=tocdepth,
 bookmarksnumbered=true,
 colorlinks=false,
 pdftitle={},
 pdfsubject={},
 pdfauthor={},
 pdfkeywords={}}
\begin{document}
%\hypertarget{tensorux7a7aux9593}{%
\subsection{tensor空間}%\label{tensorux7a7aux9593}}
%\hypertarget{tensorux7a4dux306eux7d50ux5408ux6cd5ux5247}{%
\subsubsection{tensor積の結合法則}%\label{tensorux7a4dux306eux7d50ux5408ux6cd5ux5247}}
\begin{thm}[tensor積の結合法則]\label{2.4.6.1}
体$K$上の$m$次元vector空間$U$、$n$次元vector空間$V$、$o$次元vector空間$W$が与えられたとき、ある線形同型写像$\rho:(V \otimes W) \otimes U\overset{\sim}{\rightarrow}U \otimes (V \otimes W)$が一意的に存在して、その線形同型写像$\rho$は、$\forall\left( \mathbf{u},\mathbf{v},\mathbf{w} \right) \in U \times V \times W$に対し、$\rho\left( \left( \mathbf{u} \otimes \mathbf{v} \right) \otimes \mathbf{w} \right) = \mathbf{u} \otimes \left( \mathbf{v} \otimes \mathbf{w} \right)$を満たす。ゆえに、$(V \otimes W) \otimes U \cong U \otimes (V \otimes W)$が成り立つ。\par
この定理をtensor積の結合法則という。
\end{thm}
\begin{proof}
体$K$上の$m$次元vector空間$U$、$n$次元vector空間$V$、$o$次元vector空間$W$が与えられたとき、$\forall\left( \mathbf{u},\mathbf{v},\mathbf{w} \right) \in U \times V \times W$に対し、$\rho\left( \left( \mathbf{u} \otimes \mathbf{v} \right) \otimes \mathbf{w} \right) = \mathbf{u} \otimes \left( \mathbf{v} \otimes \mathbf{w} \right)$を満たすような線形写像$\rho:(V \otimes W) \otimes U \rightarrow U \otimes (V \otimes W)$が考えられれば、同様に、$\forall\left( \mathbf{u},\mathbf{v},\mathbf{w} \right) \in U \times V \times W$に対し、$\sigma\left( \mathbf{u} \otimes \left( \mathbf{v} \otimes \mathbf{w} \right) \right) = \left( \mathbf{u} \otimes \mathbf{v} \right) \otimes \mathbf{w}$を満たすような線形写像$\sigma:V \otimes (W \otimes U) \rightarrow (U \otimes V) \otimes W$も考えられることで、これらのvector空間たち$U$、$V$、$W$の基底たちそれぞれ$\left\langle \mathbf{u}_{h} \right\rangle_{h \in \varLambda_{m}}$、$\left\langle \mathbf{v}_{i} \right\rangle_{i \in \varLambda_{n}}$、$\left\langle \mathbf{w}_{j} \right\rangle_{j \in \varLambda_{o}}$が与えられれば、定理\ref{2.4.5.1}よりそれらのvector空間たち$(V \otimes W) \otimes U$、$U \otimes (V \otimes W)$の基底たちそれぞれ$\left\langle \left( \mathbf{u}_{h} \otimes \mathbf{v}_{i} \right) \otimes \mathbf{w}_{j} \right\rangle_{(h,i,j) \in \varLambda_{m} \times \varLambda_{n} \times \varLambda_{o}}$、$\left\langle \mathbf{u}_{h} \otimes \left( \mathbf{v}_{i} \otimes \mathbf{w}_{j} \right) \right\rangle_{(h,i,j) \in \varLambda_{m} \times \varLambda_{n} \times \varLambda_{o}}$がとられることができることに注意すれば、$\forall\mathbf{T} \in (V \otimes W) \otimes U$に対し、$\mathbf{T} = \sum_{(h,i,j) \in \varLambda_{m} \times \varLambda_{n} \times \varLambda_{o}} {\xi_{hij}\left( \mathbf{u}_{h} \otimes \mathbf{v}_{i} \right) \otimes \mathbf{w}_{j}}$とおくことで、その合成写像$\sigma \circ \rho$も線形写像となって次のようになる。
\begin{align*}
\sigma \circ \rho\left( \mathbf{T} \right) &= \sigma \circ \rho\left( \sum_{(h,i,j) \in \varLambda_{m} \times \varLambda_{n} \times \varLambda_{o}} {\xi_{hij}\left( \mathbf{u}_{h} \otimes \mathbf{v}_{i} \right) \otimes \mathbf{w}_{j}} \right)\\
&= \sum_{(h,i,j) \in \varLambda_{m} \times \varLambda_{n} \times \varLambda_{o}} {\xi_{hij}\sigma \circ \rho\left( \left( \mathbf{u}_{h} \otimes \mathbf{v}_{i} \right) \otimes \mathbf{w}_{j} \right)}\\
&= \sum_{(h,i,j) \in \varLambda_{m} \times \varLambda_{n} \times \varLambda_{o}} {\xi_{hij}\sigma\left( \rho\left( \left( \mathbf{u}_{h} \otimes \mathbf{v}_{i} \right) \otimes \mathbf{w}_{j} \right) \right)}\\
&= \sum_{(h,i,j) \in \varLambda_{m} \times \varLambda_{n} \times \varLambda_{o}} {\xi_{hij}\sigma\left( \mathbf{u}_{h} \otimes \left( \mathbf{v}_{i} \otimes \mathbf{w}_{j} \right) \right)}\\
&= \sum_{(h,i,j) \in \varLambda_{m} \times \varLambda_{n} \times \varLambda_{o}} {\xi_{hij}\left( \mathbf{u}_{h} \otimes \mathbf{v}_{i} \right) \otimes \mathbf{w}_{j}} = \mathbf{T}
\end{align*}
同様にして、$\forall\mathbf{T} \in U \otimes (V \otimes W)$に対し、$\mathbf{T} = \sum_{(h,i,j) \in \varLambda_{m} \times \varLambda_{n} \times \varLambda_{o}} {\xi_{hij}\mathbf{u}_{h} \otimes \left( \mathbf{v}_{i} \otimes \mathbf{w}_{j} \right)}$とおくことで、その合成写像$\rho \circ \sigma$も線形写像となって次のようになる。
\begin{align*}
\rho \circ \sigma\left( \mathbf{T} \right) &= \rho \circ \sigma\left( \sum_{(h,i,j) \in \varLambda_{m} \times \varLambda_{n} \times \varLambda_{o}} {\xi_{hij}\mathbf{u}_{h} \otimes \left( \mathbf{v}_{i} \otimes \mathbf{w}_{j} \right)} \right)\\
&= \sum_{(h,i,j) \in \varLambda_{m} \times \varLambda_{n} \times \varLambda_{o}} {\xi_{hij}\rho \circ \sigma\left( \mathbf{u}_{h} \otimes \left( \mathbf{v}_{i} \otimes \mathbf{w}_{j} \right) \right)}\\
&= \sum_{(h,i,j) \in \varLambda_{m} \times \varLambda_{n} \times \varLambda_{o}} {\xi_{hij}\sigma\left( \rho\left( \mathbf{u}_{h} \otimes \left( \mathbf{v}_{i} \otimes \mathbf{w}_{j} \right) \right) \right)}\\
&= \sum_{(h,i,j) \in \varLambda_{m} \times \varLambda_{n} \times \varLambda_{o}} {\xi_{hij}\sigma\left( \left( \mathbf{u}_{h} \otimes \mathbf{v}_{i} \right) \otimes \mathbf{w}_{j} \right)}\\
&= \sum_{(h,i,j) \in \varLambda_{m} \times \varLambda_{n} \times \varLambda_{o}} {\xi_{hij}\mathbf{u}_{h} \otimes \left( \mathbf{v}_{i} \otimes \mathbf{w}_{j} \right)} = \mathbf{T}
\end{align*}
以上より、$\sigma = \rho^{- 1}$が得られたので、その線形写像$\rho$は線形同型写像である。\par
さらに、$\forall\left( \mathbf{u},\mathbf{v},\mathbf{w} \right) \in U \times V \times W$に対し、$\rho\left( \left( \mathbf{u} \otimes \mathbf{v} \right) \otimes \mathbf{w} \right) = \mathbf{u} \otimes \left( \mathbf{v} \otimes \mathbf{w} \right)$を満たすような線形写像$\rho:(V \otimes W) \otimes U \rightarrow U \otimes (V \otimes W)$がこれのほかに$\rho':(V \otimes W) \otimes U \rightarrow U \otimes (V \otimes W)$と与えられたらば、$\forall\mathbf{T} \in (V \otimes W) \otimes U$に対し、$\mathbf{T} = \sum_{(h,i,j) \in \varLambda_{m} \times \varLambda_{n} \times \varLambda_{o}} {\xi_{hij}\left( \mathbf{u}_{h} \otimes \mathbf{v}_{i} \right) \otimes \mathbf{w}_{j}}$とおくことで、次のようになる。
\begin{align*}
\rho\left( \mathbf{T} \right) &= \rho\left( \sum_{(h,i,j) \in \varLambda_{m} \times \varLambda_{n} \times \varLambda_{o}} {\xi_{hij}\left( \mathbf{u}_{h} \otimes \mathbf{v}_{i} \right) \otimes \mathbf{w}_{j}} \right)\\
&= \sum_{(h,i,j) \in \varLambda_{m} \times \varLambda_{n} \times \varLambda_{o}} {\xi_{hij}\rho\left( \left( \mathbf{u}_{h} \otimes \mathbf{v}_{i} \right) \otimes \mathbf{w}_{j} \right)}\\
&= \sum_{(h,i,j) \in \varLambda_{m} \times \varLambda_{n} \times \varLambda_{o}} {\xi_{hij}\mathbf{u}_{h} \otimes \left( \mathbf{v}_{i} \otimes \mathbf{w}_{j} \right)}\\
&= \sum_{(h,i,j) \in \varLambda_{m} \times \varLambda_{n} \times \varLambda_{o}} {\xi_{hij}\rho'\left( \left( \mathbf{u}_{h} \otimes \mathbf{v}_{i} \right) \otimes \mathbf{w}_{j} \right)}\\
&= \rho'\left( \sum_{(h,i,j) \in \varLambda_{m} \times \varLambda_{n} \times \varLambda_{o}} {\xi_{hij}\left( \mathbf{u}_{h} \otimes \mathbf{v}_{i} \right) \otimes \mathbf{w}_{j}} \right) = \rho'\left( \mathbf{T} \right)
\end{align*}
以上より、$\rho = \rho'$が得られる。\par
よって、ある線形同型写像$\rho:(V \otimes W) \otimes U\overset{\sim}{\rightarrow}U \otimes (V \otimes W)$が一意的に存在して、その線形同型写像$\rho$は、$\forall\left( \mathbf{u},\mathbf{v},\mathbf{w} \right) \in U \times V \times W$に対し、$\rho\left( \left( \mathbf{u} \otimes \mathbf{v} \right) \otimes \mathbf{w} \right) = \mathbf{u} \otimes \left( \mathbf{v} \otimes \mathbf{w} \right)$を満たす。ゆえに、$(V \otimes W) \otimes U \cong U \otimes (V \otimes W)$が成り立つ。
\end{proof}
%\hypertarget{ux91cdux7ddaux5f62ux5199ux50cf}{%
\subsubsection{重線形写像}%\label{ux91cdux7ddaux5f62ux5199ux50cf}}
\begin{axs}
$n$つの体$K$上$m_{i}$次元vector空間たち$V_{i}$、vector空間$W$が与えられたとき、写像$\varPhi:\prod_{i \in \varLambda_{n}} V_{i} \rightarrow W$が、$\forall i \in \varLambda_{n}\forall k,l \in K\forall\left( \mathbf{v}_{i'} \right)_{i' \in \varLambda_{n}} \in \prod_{i' \in \varLambda_{n}} V_{i'}\forall\mathbf{w}_{i} \in V_{i}$に対し、次式を満たすとき、
\begin{align*}
\varPhi\begin{pmatrix}
\mathbf{v}_{1} & \cdots & k\mathbf{v}_{i} + l\mathbf{w}_{i} & \cdots & \mathbf{v}_{n} \\
\end{pmatrix} = k\varPhi\begin{pmatrix}
\mathbf{v}_{1} & \cdots & \mathbf{v}_{i} & \cdots & \mathbf{v}_{n} \\
\end{pmatrix} + l\varPhi\begin{pmatrix}
\mathbf{v}_{1} & \cdots & \mathbf{w}_{i} & \cdots & \mathbf{v}_{n} \\
\end{pmatrix}
\end{align*}
その写像$\varPhi$を$n$重線形写像、または単に重線形写像という。\par
さらに、このような写像全体の集合を以下、$L\left( \left( V_{i} \right)_{i \in \varLambda_{n}};W \right)$と書くことにする。
\end{axs}
\begin{thm}\label{2.4.6.2}
$n$つの体$K$上$m_{i}$次元vector空間たち$V_{i}$、vector空間$W$が与えられたとき、重線形写像全体の集合$L\left( \left( V_{i} \right)_{i \in \varLambda_{n}};W \right)$は体$K$上のvector空間をなす。
\end{thm}
\begin{proof} 定理\ref{2.4.5.1}と同様にして示される。
\end{proof}
%\hypertarget{tensorux7a7aux9593-1}{%
\subsubsection{tensor空間}%\label{tensorux7a7aux9593-1}}
\begin{dfn}
$n$つの体$K$上$m_{i}$次元vector空間たち$V_{i}$が与えられたとき、次式のようにしてvector空間$\bigotimes_{i \in \varLambda_{n}} V_{i}$が定義される。そのvector空間$\bigotimes_{i \in \varLambda_{n}} V_{i}$をそれらのvector空間たち$V_{i}$からなるtensor空間ということにする。
\begin{align*}
\bigotimes_{i \in \varLambda_{n}} V_{i} = \left\{ \begin{matrix}
V_{1} & \mathrm{if} & n = 1 \\
\bigotimes_{i \in \varLambda_{n - 1}} V_{i} \otimes V_{n} & \mathrm{if} & 2 \leq n \\
\end{matrix} \right.\ 
\end{align*}
さらに、それらのvector空間たち$V_{i}$の基底たちが$\left\langle \mathbf{v}_{ij_{i}} \right\rangle_{j_{i} \in \varLambda_{m_{i}}}$と与えられたとき、これらを$\alpha_{i}$とおくと、このようにして得られたそのvector空間$\bigotimes_{i \in \varLambda_{n}} V_{i}$の基底として$\left\langle \bigotimes_{i} \mathbf{v}_{ij_{i}} \right\rangle_{\mathbf{j}\in \prod_{i} \varLambda_{m_{i}}}$があげられる。なお、$\mathbf{j}=\left( j_{i} \right)_{i} $としその添数$i$のとりうる範囲を$\varLambda_{n}$とした。これを$\bigotimes_{i \in \varLambda_{n}} \alpha_{i}$と書くことにする。
\end{dfn}
\begin{thm}\label{2.4.6.3}
$n$つの体$K$上$m_{i}$次元vector空間たち$V_{i}$、vector空間$W$が与えられたとき、任意の重線形写像$\varPhi:\prod_{i \in \varLambda_{n}} V_{i} \rightarrow W$に対し、ある線形写像$\rho:\bigotimes_{i \in \varLambda_{n}} V_{i} \rightarrow W$が一意的に存在して、$\varPhi = \rho \circ \otimes$が成り立つ。
\end{thm}
\begin{proof}
$n$つの体$K$上$m_{i}$次元vector空間たち$V_{i}$、vector空間$W$が与えられたとき、任意の重線形写像$\varPhi:\prod_{i \in \varLambda_{n}} V_{i} \rightarrow W$に対し、任意の重線形写像$\varPhi:\prod_{i \in \varLambda_{n}} V_{i} \rightarrow W$に対し、ある線形写像$\rho:\bigotimes_{i \in \varLambda_{n}} V_{i} \rightarrow W$が存在して、$\varPhi = \rho \circ \otimes$が成り立つとしよう。このような線形写像$\rho$がこれのほかに$\sigma$と与えられたとする。それらのvector空間たち$V_{i}$の基底たちが$\left\langle \mathbf{v}_{ij_{i}} \right\rangle_{j_{i} \in \varLambda_{m_{i}}}$と与えられたとき、このようにして得られたそのvector空間$\bigotimes_{i \in \varLambda_{n}} V_{i}$の基底として$\left\langle \bigotimes_{i} \mathbf{v}_{ij_{i}} \right\rangle_{\mathbf{j}\in \prod_{i} \varLambda_{m_{i}}}$があげられる。なお、$\mathbf{j}=\left( j_{i} \right)_{i} $としその添数$i$のとりうる範囲を$\varLambda_{n}$とした。このとき、$\forall\mathbf{T} \in \bigotimes_{i \in \varLambda_{n}} V_{i}$に対し、$\mathbf{T} = \sum_{\mathbf{j}\in \prod_{i} \varLambda_{m_{i}}} {\xi_{\mathbf{j}}\bigotimes_{i} \mathbf{v}_{ij_{i}}}$とおかれると、次のようになることから、
\begin{align*}
\rho\left( \mathbf{T} \right) &= \rho\left( \sum_{\mathbf{j}\in \prod_{i} \varLambda_{m_{i}}} {\xi_{\mathbf{j}}\bigotimes_{i} \mathbf{v}_{ij_{i}}} \right)\\
&= \sum_{\mathbf{j}\in \prod_{i} \varLambda_{m_{i}}} {\xi_{\mathbf{j}}\rho\left( \bigotimes_{i} \mathbf{v}_{ij_{i}} \right)}\\
&= \sum_{\mathbf{j}\in \prod_{i} \varLambda_{m_{i}}} {\xi_{\mathbf{j}}\rho \circ \otimes \left( \mathbf{v}_{ij_{i}} \right)_{i}}\\
&= \sum_{\mathbf{j}\in \prod_{i} \varLambda_{m_{i}}} {\xi_{\mathbf{j}}\varPhi\left( \mathbf{v}_{ij_{i}} \right)_{i}}\\
&= \sum_{\mathbf{j}\in \prod_{i} \varLambda_{m_{i}}} {\xi_{\mathbf{j}}\sigma \circ \otimes \left( \mathbf{v}_{ij_{i}} \right)_{i}}\\
&= \sum_{\mathbf{j}\in \prod_{i} \varLambda_{m_{i}}} {\xi_{\mathbf{j}}\sigma\left( \bigotimes_{i} \mathbf{v}_{ij_{i}} \right)}\\
&= \sigma\left( \sum_{\mathbf{j}\in \prod_{i} \varLambda_{m_{i}}} {\xi_{\mathbf{j}}\bigotimes_{i} \mathbf{v}_{ij_{i}}} \right) = \sigma\left( \mathbf{T} \right)
\end{align*}
$\rho = \sigma$が得られる。ゆえに、このような線形写像$\rho$が存在するのであれば、その存在は一意的である。\par
任意の重線形写像$\varPhi:\prod_{i \in \varLambda_{n}} V_{i} \rightarrow W$に対し、ある線形写像$\rho:\bigotimes_{i \in \varLambda_{n}} V_{i} \rightarrow W$が存在して、$\varPhi = \rho \circ \otimes$が成り立つことについて、$n = 1$のときは明らかで、$n = 2$のときは定理\ref{2.4.5.2}そのものである。\par
$n = m$のとき、ある線形写像$\rho:\bigotimes_{i \in \varLambda_{m}} V_{i} \rightarrow W$が存在して、$\varPhi = \rho \circ \otimes$が成り立つと仮定しよう。$n = m + 1$のとき、任意の重線形写像$\varPhi:\prod_{i \in \varLambda_{m + 1}} V_{i} \rightarrow W$に対し、$\forall\mathbf{v}_{m + 1} \in V_{m + 1}$に対し、次式のように写像$\varPhi_{\mathbf{v}_{m + 1}}$が定義されると、
\begin{align*}
\varPhi_{\mathbf{v}_{m + 1}}:\prod_{i \in \varLambda_{m}} V_{i} \rightarrow W;\left( \mathbf{v}_{i} \right)_{i \in \varLambda_{m}} \mapsto \varPhi\left( \mathbf{v}_{i} \right)_{i \in \varLambda_{m + 1}}
\end{align*}
これは明らかに重線形写像であり、ある線形写像$\rho_{\mathbf{v}_{m + 1}}:\bigotimes_{i \in \varLambda_{m}} V_{i} \rightarrow W$が存在して、$\varPhi_{\mathbf{v}_{m + 1}} = \rho_{\mathbf{v}_{m + 1}} \circ \otimes$が成り立つ。\par
そこで、次式のように写像$\varPsi$が定義されると、
\begin{align*}
\varPsi:\bigotimes_{i \in \varLambda_{m}} V_{i} \times V_{m + 1} \rightarrow W;\left( \mathbf{T},\mathbf{v}_{m + 1} \right) \mapsto \rho_{\mathbf{v}_{m + 1}}\left( \mathbf{T} \right)
\end{align*}
その写像$\rho_{\mathbf{v}_{m + 1}}$のおき方から、$\forall k,l \in K\forall\mathbf{T},\mathbf{U} \in \bigotimes_{i \in \varLambda_{m}} V_{i}$に対し、次のようになる。
\begin{align*}
\varPsi\begin{pmatrix}
k\mathbf{T} + l\mathbf{U} & \mathbf{v}_{m + 1} \\
\end{pmatrix} &= \rho_{\mathbf{v}_{m + 1}}\left( k\mathbf{T} + l\mathbf{U} \right)\\
&= k\rho_{\mathbf{v}_{m + 1}}\left( \mathbf{T} \right) + l\rho_{\mathbf{v}_{m + 1}}\left( \mathbf{U} \right)\\
&= k\varPsi\left( \mathbf{T},\mathbf{v}_{m + 1} \right) + l\varPsi\left( \mathbf{U},\mathbf{v}_{m + 1} \right)
\end{align*}
一方で、$\forall\mathbf{T} \in \bigotimes_{i \in \varLambda_{m}} V_{i}$に対し、その添数$i$のとりうる範囲を$\varLambda_{m}$として次式のようにおかれると、
\begin{align*}
\mathbf{T} = \sum_{\mathbf{j}\in \prod_{} \varLambda_{\dim V_{i}}} {\xi_{\mathbf{j}}\bigotimes_{i} \mathbf{v}_{ij_{i}}}
\end{align*}
$\forall k,l \in K\forall\mathbf{v}_{m + 1},\mathbf{w}_{m + 1} \in V_{m + 1}$に対し、次のようになることから、
\begin{align*}
\rho_{k\mathbf{v}_{m + 1} + l\mathbf{w}_{m + 1}}\left( \mathbf{T} \right) &= \rho_{k\mathbf{v}_{m + 1} + l\mathbf{w}_{m + 1}}\left( \sum_{\mathbf{j} \in \prod_{i} \varLambda_{m_{i}}} {\xi_{\mathbf{j}}\bigotimes_{i} \mathbf{v}_{ij_{i}}} \right)\\
&= \sum_{\mathbf{j} \in \prod_{i} \varLambda_{m_{i}}} {\xi_{\mathbf{j}}\rho_{k\mathbf{v}_{m + 1} + l\mathbf{w}_{m + 1}}\left( \bigotimes_{i} \mathbf{v}_{ij_{i}} \right)}\\
&= \sum_{\mathbf{j} \in \prod_{i} \varLambda_{m_{i}}} {\xi_{\mathbf{j}}\varPhi\left( \left( \mathbf{v}_{ij_{i}} \right)_{i},k\mathbf{v}_{m + 1} + l\mathbf{w}_{m + 1} \right)}\\
&= \sum_{\mathbf{j} \in \prod_{i} \varLambda_{m_{i}}} {\xi_{\mathbf{j}}\left( k\varPhi\left( \left( \mathbf{v}_{ij_{i}} \right)_{i},\mathbf{v}_{m + 1} \right) + l\varPhi\left( \left( \mathbf{v}_{ij_{i}} \right)_{i},\mathbf{w}_{m + 1} \right) \right)}\\
&= k\sum_{\mathbf{j} \in \prod_{i} \varLambda_{m_{i}}} {\xi_{\mathbf{j}}\varPhi\left( \left( \mathbf{v}_{ij_{i}} \right)_{i},\mathbf{v}_{m + 1} \right)} + l\sum_{\mathbf{j} \in \prod_{i} \varLambda_{m_{i}}} {\xi_{\mathbf{j}}\varPhi\left( \left( \mathbf{v}_{ij_{i}} \right)_{i},\mathbf{w}_{m + 1} \right)}\\
&= k\sum_{\mathbf{j} \in \prod_{i} \varLambda_{m_{i}}} {\xi_{\mathbf{j}}\rho_{\mathbf{v}_{m + 1}}\left( \bigotimes_{i} \mathbf{v}_{ij_{i}} \right)} + l\sum_{\mathbf{j} \in \prod_{i} \varLambda_{m_{i}}} {\xi_{\mathbf{j}}\rho_{\mathbf{w}_{m + 1}}\left( \bigotimes_{i} \mathbf{v}_{ij_{i}} \right)}\\
&= k\rho_{\mathbf{v}_{m + 1}}\left( \sum_{\mathbf{j} \in \prod_{i} \varLambda_{m_{i}}} {\xi_{\mathbf{j}}\bigotimes_{i} \mathbf{v}_{ij_{i}}} \right) + l\rho_{\mathbf{w}_{m + 1}}\left( \sum_{\mathbf{j} \in \prod_{i} \varLambda_{m_{i}}} {\xi_{\mathbf{j}}\bigotimes_{i} \mathbf{v}_{ij_{i}}} \right)\\
&= k\rho_{\mathbf{v}_{m + 1}}\left( \mathbf{T} \right) + l\rho_{\mathbf{w}_{m + 1}}\left( \mathbf{T} \right)\\
&= \left( k\rho_{\mathbf{v}_{m + 1}} + l\rho_{\mathbf{w}_{m + 1}} \right)\left( \mathbf{T} \right)
\end{align*}
$\rho_{k\mathbf{v}_{m + 1} + l\mathbf{w}_{m + 1}} = k\rho_{\mathbf{v}_{m + 1}} + l\rho_{\mathbf{w}_{m + 1}}$が成り立つ。以上の議論により、その写像$\varPsi$は双線形写像である。\par
したがって、定理\ref{2.4.5.3}よりある線形写像$\rho:\bigotimes_{i \in \varLambda_{m + 1}} V_{i} \rightarrow W$が一意的に存在して、$\varPsi = \rho \circ \otimes$が成り立つ。これにより、$\forall\left( \mathbf{v}_{ij_{i}} \right)_{i \in \varLambda_{m + 1}} \in \prod_{i \in \varLambda_{m + 1}} V_{i}$に対し、次のようになる。
\begin{align*}
\varPhi\left( \mathbf{v}_{ij_{i}} \right)_{i \in \varLambda_{m + 1}} &= \varPhi_{\mathbf{v}_{m + 1}}\left( \mathbf{v}_{ij_{i}} \right)_{i \in \varLambda_{m}}\\
&= \rho_{\mathbf{v}_{m + 1}} \circ \otimes \left( \mathbf{v}_{ij_{i}} \right)_{i \in \varLambda_{m}}\\
&= \rho_{\mathbf{v}_{m + 1}}\left( \bigotimes_{i \in \varLambda_{m}} \mathbf{v}_{ij_{i}} \right)\\
&= \varPsi\begin{pmatrix}
\bigotimes_{i \in \varLambda_{m}} \mathbf{v}_{ij_{i}} & \mathbf{v}_{m + 1} \\
\end{pmatrix}\\
&= \rho \circ \otimes \begin{pmatrix}
\bigotimes_{i \in \varLambda_{m}} \mathbf{v}_{ij_{i}} & \mathbf{v}_{m + 1} \\
\end{pmatrix}\\
&= \rho\left( \bigotimes_{i \in \varLambda_{m + 1}} \mathbf{v}_{ij_{i}} \right)\\
&= \rho \circ \otimes \left( \mathbf{v}_{ij_{i}} \right)_{i \in \varLambda_{m + 1}}
\end{align*}
以上より、$n = m + 1$のときでも、任意の重線形写像$\varPhi:\prod_{i \in \varLambda_{m + 1}} V_{i} \rightarrow W$に対し、ある線形写像$\rho:\bigotimes_{i \in \varLambda_{m + 1}} V_{i} \rightarrow W$が存在して、$\varPhi = \rho \circ \otimes$が成り立つ。その存在は上記の議論により一意的である。\par
よって、数学的帰納法により任意の重線形写像$\varPhi:\prod_{i \in \varLambda_{n}} V_{i} \rightarrow W$に対し、ある線形写像$\rho:\bigotimes_{i \in \varLambda_{n}} V_{i} \rightarrow W$が一意的に存在して、$\varPhi = \rho \circ \otimes$が成り立つことが示された。
\end{proof}
\begin{thm}\label{2.4.6.4}
$n$つの体$K$上$m_{i}$次元vector空間たち$V_{i}$が与えられたとき、次式が成り立つ。
\begin{align*}
\dim{\bigotimes_{i \in \varLambda_{n}} V_{i}} = \prod_{i \in \varLambda_{n}} {\dim V_{i}} = m_{1}\cdots m_{n}
\end{align*}
\end{thm}
\begin{proof} 定理\ref{2.4.5.5}と数学的帰納法により直ちにわかる。
\end{proof}
\begin{thm}\label{2.4.6.5}
$n$つの体$K$上$m_{i}$次元vector空間たち$V_{i}$が与えられたとき、次式が成り立つ。
\begin{align*}
\left( \bigotimes_{i \in \varLambda_{n}} V_{i} \right)^{*} \cong \bigotimes_{i \in \varLambda_{n}} V_{i}^{*}
\end{align*}
このとき、ある線形同型写像$\varSigma:\bigotimes_{i \in \varLambda_{n}} V_{i}^{*} \rightarrow \left( \bigotimes_{i \in \varLambda_{n}} V_{i} \right)^{*}$が存在して、$\forall\left( f_{i} \right)_{i \in \varLambda_{n}} \in \prod_{i \in \varLambda_{n}} V_{i}^{*}\forall\left( \mathbf{v}_{i} \right)_{i \in \varLambda_{n}} \in \prod_{i \in \varLambda_{n}} V_{i}$に対し、次式が成り立つ。
\begin{align*}
\varSigma\left( \bigotimes_{i \in \varLambda_{n}} f_{i} \right)\left( \bigotimes_{i \in \varLambda_{n}} \mathbf{v}_{i} \right) = \prod_{i \in \varLambda_{n}} {f_{i}\left( \mathbf{v}_{i} \right)}
\end{align*}
\end{thm}
\begin{proof} 定理\ref{2.4.5.14}と数学的帰納法により直ちにわかる。
\end{proof}
\begin{thm}\label{2.4.6.6}
体$K$が与えられたとき、次式が成り立つ。
\begin{align*}
\bigotimes_{i \in \varLambda_{n}} K \cong K
\end{align*}
\end{thm}
\begin{proof} 定理\ref{2.4.5.16}と数学的帰納法により直ちにわかる。
\end{proof}
\begin{thm}\label{2.4.6.7}
$n$つの体$K$上$m_{i}$次元vector空間たち$V_{i}$、$o$次元vector空間$W$が与えられたとき、次式が成り立つ。
\begin{align*}
L\left( \left( V_{i} \right)_{i \in \varLambda_{n}};W \right) \cong L\left( \bigotimes_{i \in \varLambda_{n}} V_{i},W \right)
\end{align*}
このとき、ある線形同型写像$\otimes^{*}:L\left( \bigotimes_{i \in \varLambda_{n}} V_{i},W \right)\overset{\sim}{\rightarrow}L\left( \left( V_{i} \right)_{i \in \varLambda_{n}};W \right)$が存在して、$\otimes^{*}(\rho) = \rho \circ \otimes$が成り立ち、$\forall\left( \mathbf{v}_{i} \right)_{i \in \varLambda_{n}} \in \prod_{i \in \varLambda_{n}} V_{i}$に対し、$\otimes^{*}(\rho)\left( \mathbf{v}_{i} \right)_{i \in \varLambda_{n}} = \rho\left( \bigotimes_{i \in \varLambda_{n}} \mathbf{v}_{i} \right)$が成り立つ。
\end{thm}
\begin{proof} 定理\ref{2.4.5.18}と数学的帰納法により直ちにわかる。
\end{proof}
\begin{thm}\label{2.4.6.8}
$n$つの体$K$上$m_{i}$次元vector空間たち$V_{i}$、$o$次元vector空間$W$が与えられたとき、次式が成り立つ。
\begin{align*}
L\left( \left( V_{i} \right)_{i \in \varLambda_{n}};W \right) \cong \bigotimes_{i \in \varLambda_{n}} V_{i}^{*} \otimes W
\end{align*}
このとき、ある線形同型写像$\varSigma:\bigotimes_{i \in \varLambda_{n}} V_{i}^{*} \otimes W\overset{\sim}{\rightarrow}L\left( \left( V_{i} \right)_{i \in \varLambda_{n}};W \right)$が存在して、$\forall\left( f_{i} \right)_{i \in \varLambda_{n}} \in \prod_{i \in \varLambda_{n}} V_{i}^{*}\forall\mathbf{w} \in W\forall\left( \mathbf{v}_{i} \right)_{i \in \varLambda_{n}} \in \prod_{i \in \varLambda_{n}} V_{i}$に対し、次式が成り立つ。
\begin{align*}
\varSigma\left( \bigotimes_{i \in \varLambda_{n}} f_{i} \otimes \mathbf{w}_{j} \right)\left( \mathbf{v}_{i} \right)_{i \in \varLambda_{n}} = \prod_{i \in \varLambda_{n}} {f_{i}\left( \mathbf{v}_{i} \right)}\mathbf{w}_{j}
\end{align*}
\end{thm}
\begin{proof} 定理\ref{2.4.5.19}と数学的帰納法により直ちにわかる。
\end{proof}
\begin{thebibliography}{50}
  \bibitem{1}
  佐武一郎, 線型代数学, 裳華房, 1958. 第53版 p208-211 ISBN4-7853-1301-3
\end{thebibliography}
\end{document}
