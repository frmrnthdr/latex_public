\documentclass[dvipdfmx]{jsarticle}
\setcounter{section}{1}
\setcounter{subsection}{8}
\usepackage{xr}
\externaldocument{2.1.8}
\usepackage{amsmath,amsfonts,amssymb,array,comment,mathtools,url,docmute}
\usepackage{longtable,booktabs,dcolumn,tabularx,mathtools,multirow,colortbl,xcolor}
\usepackage[dvipdfmx]{graphics}
\usepackage{bmpsize}
\usepackage{amsthm}
\usepackage{enumitem}
\setlistdepth{20}
\renewlist{itemize}{itemize}{20}
\setlist[itemize]{label=•}
\renewlist{enumerate}{enumerate}{20}
\setlist[enumerate]{label=\arabic*.}
\setcounter{MaxMatrixCols}{20}
\setcounter{tocdepth}{3}
\newcommand{\rotin}{\text{\rotatebox[origin=c]{90}{$\in $}}}
\renewcommand{\thesection}{第\arabic{section}部}
\renewcommand{\thesubsection}{\arabic{section}.\arabic{subsection}}
\renewcommand{\thesubsubsection}{\arabic{section}.\arabic{subsection}.\arabic{subsubsection}}
\everymath{\displaystyle}
\allowdisplaybreaks[4]
\usepackage{vtable}
\theoremstyle{definition}
\newtheorem{thm}{定理}[subsection]
\newtheorem*{thm*}{定理}
\newtheorem{dfn}{定義}[subsection]
\newtheorem*{dfn*}{定義}
\newtheorem{axs}[dfn]{公理}
\newtheorem*{axs*}{公理}
\renewcommand{\headfont}{\bfseries}
\makeatletter
  \renewcommand{\section}{%
    \@startsection{section}{1}{\z@}%
    {\Cvs}{\Cvs}%
    {\normalfont\huge\headfont\raggedright}}
\makeatother
\makeatletter
  \renewcommand{\subsection}{%
    \@startsection{subsection}{2}{\z@}%
    {0.5\Cvs}{0.5\Cvs}%
    {\normalfont\LARGE\headfont\raggedright}}
\makeatother
\makeatletter
  \renewcommand{\subsubsection}{%
    \@startsection{subsubsection}{3}{\z@}%
    {0.4\Cvs}{0.4\Cvs}%
    {\normalfont\Large\headfont\raggedright}}
\makeatother
\makeatletter
\renewenvironment{proof}[1][\proofname]{\par
  \pushQED{\qed}%
  \normalfont \topsep6\p@\@plus6\p@\relax
  \trivlist
  \item\relax
  {
  #1\@addpunct{.}}\hspace\labelsep\ignorespaces
}{%
  \popQED\endtrivlist\@endpefalse
}
\makeatother
\renewcommand{\proofname}{\textbf{証明}}
\usepackage{tikz,graphics}
\usepackage[dvipdfmx]{hyperref}
\usepackage{pxjahyper}
\hypersetup{
 setpagesize=false,
 bookmarks=true,
 bookmarksdepth=tocdepth,
 bookmarksnumbered=true,
 colorlinks=false,
 pdftitle={},
 pdfsubject={},
 pdfauthor={},
 pdfkeywords={}}
\begin{document}
%\hypertarget{ux548cux7a7aux9593ux3068ux4ea4ux7a7aux9593}{%
\subsection{和空間と交空間}%\label{ux548cux7a7aux9593ux3068ux4ea4ux7a7aux9593}}
%\hypertarget{ux548cux7a7aux9593}{%
\subsubsection{和空間}%\label{ux548cux7a7aux9593}}
\begin{dfn}
体$K$上のvector空間$V$の部分空間たち$U$、$W$を用いた次式のような集合$U + W$を考える。この集合$U + W$をそれらの部分空間たち$U$、$W$の和空間という。
\begin{align*}
U + W = \left\{ \mathbf{u} + \mathbf{w} \in V \middle| \mathbf{u} \in U,\ \ \mathbf{w} \in W \right\}
\end{align*}
\end{dfn}
\begin{thm}\label{2.1.9.1}
体$K$上のvector空間$V$の部分空間たち$U$、$W$の和空間$U + W$もそのvector空間$V$の部分空間である。
\end{thm}\par
また、これより明らかにそれらの部分空間たち$U$、$W$はその和空間$U + W$の部分空間でもある。
\begin{proof}
体$K$上のvector空間$V$の部分空間たち$U$、$W$を用いた次式のような集合$U + W$を考える。
\begin{align*}
U + W = \left\{ \mathbf{u} + \mathbf{w} \in V \middle| \mathbf{u} \in U,\ \ \mathbf{w} \in W \right\}
\end{align*}
このとき、定義より明らかにその集合$U + W$は空集合ではなく、$\forall\mathbf{v},\mathbf{w} \in U + W$に対し定義より$\mathbf{v}_{U},\mathbf{w}_{U} \in U$かつ$\mathbf{v}_{W},\mathbf{w}_{W} \in W$かつ$\mathbf{v} = \mathbf{v}_{U} + \mathbf{v}_{W}$かつ$\mathbf{w} = \mathbf{w}_{U} + \mathbf{w}_{W}$なるvectors$\mathbf{v}_{U}$、$\mathbf{w}_{U}$、$\mathbf{v}_{W}$、$\mathbf{w}_{W}$が存在する。ここで、$\forall k,l \in K$に対しそれらの集合たち$U$、$W$はそのvector空間$V$の部分空間であったので、次式が成り立つ。
\begin{align*}
k\mathbf{v}_{U} + l\mathbf{w}_{U} \in U,\ \ k\mathbf{v}_{W} + l\mathbf{w}_{W} \in W
\end{align*}
したがって、次のようになる。
\begin{align*}
k\mathbf{v} + l\mathbf{w} &= k\left( \mathbf{v}_{U} + \mathbf{v}_{W} \right) + l\left( \mathbf{w}_{U} + \mathbf{w}_{W} \right)\\
&= k\mathbf{v}_{U} + k\mathbf{v}_{W} + l\mathbf{w}_{U} + l\mathbf{w}_{W}\\
&= \left( k\mathbf{v}_{U} + l\mathbf{w}_{U} \right) + \left( k\mathbf{v}_{W} + l\mathbf{w}_{W} \right) \in U + W
\end{align*}
\end{proof}
\begin{thm}\label{2.1.9.2}
体$K$上のvector空間$V$の部分空間たち${\mathrm{span}}\left\{ \mathbf{v}_{i} \right\}_{i \in \varLambda_{m}}$、${\mathrm{span}}\left\{ \mathbf{w}_{j} \right\}_{j \in \varLambda_{n}}$が与えられたとき、次式が成り立つ。
\begin{align*}
{\mathrm{span}}\left\{ \mathbf{v}_{i} \right\}_{i \in \varLambda_{m}} + {\mathrm{span}}\left\{ \mathbf{w}_{j} \right\}_{j \in \varLambda_{n}} = {\mathrm{span}}\left\{ \mathbf{v}_{i},\mathbf{w}_{j} \right\}_{(i,j) \in \varLambda_{m} \times \varLambda_{n}}
\end{align*}
\end{thm}
\begin{proof}
体$K$上のvector空間$V$の部分空間たち${\mathrm{span}}\left\{ \mathbf{v}_{i} \right\}_{i \in \varLambda_{m}}$、${\mathrm{span}}\left\{ \mathbf{w}_{j} \right\}_{j \in \varLambda_{n}}$が与えられたとき、$\forall\mathbf{u} \in {\mathrm{span}}\left\{ \mathbf{v}_{i} \right\}_{i \in \varLambda_{m}} + {\mathrm{span}}\left\{ \mathbf{w}_{j} \right\}_{j \in \varLambda_{n}}$に対し、$\mathbf{u} = \mathbf{v} + \mathbf{w}$かつ$\mathbf{v} \in {\mathrm{span}}\left\{ \mathbf{v}_{i} \right\}_{i \in \varLambda_{m}}$かつ$\mathbf{w} \in {\mathrm{span}}\left\{ \mathbf{w}_{j} \right\}_{j \in \varLambda_{n}}$なるvectors$\mathbf{v}$、$\mathbf{w}$が存在し定義より$\forall i \in \varLambda_{m}\forall j \in \varLambda_{n}$に対し$k_{i},l_{j} \in K$なる元々$k_{i}$、$l_{j}$を用いて次式のように書かれることができる。
\begin{align*}
\mathbf{v} = \sum_{i \in \varLambda_{m}} {k_{i}\mathbf{v}_{i}},\ \ \mathbf{w} = \sum_{j \in \varLambda_{n}} {l_{j}\mathbf{w}_{j}}
\end{align*}
したがって、次式が成り立ち
\begin{align*}
\mathbf{u} = \mathbf{v} + \mathbf{w} = \sum_{i \in \varLambda_{m}} {k_{i}\mathbf{v}_{i}} + \sum_{j \in \varLambda_{n}} {l_{j}\mathbf{w}_{j}}
\end{align*}
$\mathbf{v} \in {\mathrm{span}}\left\{ \mathbf{v}_{i},\mathbf{w}_{j} \right\}_{(i,j) \in \varLambda_{m} \times \varLambda_{n}}$が成り立つので、次式が成り立つ。
\begin{align*}
{\mathrm{span}}\left\{ \mathbf{v}_{i} \right\}_{i \in \varLambda_{m}} + {\mathrm{span}}\left\{ \mathbf{w}_{j} \right\}_{j \in \varLambda_{n}} \subseteq {\mathrm{span}}\left\{ \mathbf{v}_{i},\mathbf{w}_{j} \right\}_{(i,j) \in \varLambda_{m} \times \varLambda_{n}}
\end{align*}\par
一方で、$\forall\mathbf{v} \in {\mathrm{span}}\left\{ \mathbf{v}_{i},\mathbf{w}_{j} \right\}_{(i,j) \in \varLambda_{m} \times \varLambda_{n}}$に対しそのvector$\mathbf{v}$は$\forall i \in \varLambda_{m}\forall j \in \varLambda_{n}$に対し$k_{i},l_{j} \in K$なる元々$k_{i}$、$l_{j}$を用いて次式のように書かれることができる。
\begin{align*}
\mathbf{v} = \sum_{i \in \varLambda_{m}} {k_{i}\mathbf{v}_{i}} + \sum_{j \in \varLambda_{n}} {l_{j}\mathbf{w}_{j}}
\end{align*}
ここで、明らかに$\sum_{i \in \varLambda_{m}} {k_{i}\mathbf{v}_{i}} \in {\mathrm{span}}\left\{ \mathbf{v}_{i} \right\}_{i \in \varLambda_{m}}$かつ$\sum_{j \in \varLambda_{n}} {l_{j}\mathbf{w}_{j}} \in {\mathrm{span}}\left\{ \mathbf{w}_{j} \right\}_{j \in \varLambda_{n}}$が成り立つので、$\mathbf{v} \in {\mathrm{span}}\left\{ \mathbf{v}_{i} \right\}_{i \in \varLambda_{m}} + {\mathrm{span}}\left\{ \mathbf{w}_{j} \right\}_{j \in \varLambda_{n}}$が成り立つ。したがって、次式が成り立つ。
\begin{align*}
{\mathrm{span}}\left\{ \mathbf{v}_{i} \right\}_{i \in \varLambda_{m}} + {\mathrm{span}}\left\{ \mathbf{w}_{j} \right\}_{j \in \varLambda_{n}} \supseteq {\mathrm{span}}\left\{ \mathbf{v}_{i},\mathbf{w}_{j} \right\}_{(i,j) \in \varLambda_{m} \times \varLambda_{n}}
\end{align*}
\end{proof}
%\hypertarget{ux4ea4ux7a7aux9593}{%
\subsubsection{交空間}%\label{ux4ea4ux7a7aux9593}}
\begin{dfn}
体$K$上のvector空間$V$の部分空間たち$U$、$W$を用いた次式のような集合$U \cap W$を考える。この集合$U \cap W$をそれらの部分空間たち$U$、$W$の交空間という。
\begin{align*}
U\text{∩}W = \left\{ \mathbf{v} \in V \middle| \mathbf{v} \in U,\ \ \mathbf{v} \in W \right\}
\end{align*}
\end{dfn}
\begin{thm}\label{2.1.9.3}
体$K$上のvector空間$V$の部分空間たち$U$、$W$の交空間$U \cap W$もそのvector空間$V$の部分空間である。
\end{thm}\par
また、これより明らかに、その交空間$U \cap W$はそれらの部分空間たち$U$、$W$の部分空間でもある。ちなみに、その集合$U \cup W$は必ずしも部分空間にならないことに注意されたい。
\begin{proof}
体$K$上のvector空間$V$の部分空間たち$U$、$W$を用いた次式のような集合$U \cap W$を考える。
\begin{align*}
U\text{∩}W = \left\{ \mathbf{v} \in V \middle| \mathbf{v} \in U,\ \ \mathbf{v} \in W \right\}
\end{align*}
このとき、定義より明らかにその集合$U \cap W$は空集合ではなく、$\forall\mathbf{v},\mathbf{w} \in U \cap W$に対し定義より$\mathbf{v},\mathbf{w} \in U$かつ$\mathbf{v},\mathbf{w} \in W$が成り立ち、$\forall k,l \in K$に対しそれらの集合たち$U$、$W$はそのvector空間$V$の部分空間であったので、$k\mathbf{v} + l\mathbf{w} \in U$かつ$k\mathbf{v} + l\mathbf{w} \in W$が成り立ち、したがって、$k\mathbf{v} + l\mathbf{w} \in U \cap W$が成り立つ。
\end{proof}
\begin{thm}\label{2.1.9.4}
体$K$上の$n$次元vector空間$V$の$A_{ln} \in M_{ln}(K)$、$B_{mn} \in M_{mn}(K)$なる行列たち$A_{ln}$、$B_{mn}$を用いた部分空間たち$\left\{ \mathbf{v} \in V \middle| A_{ln}\mathbf{v} = \mathbf{b} \right\}$、$\left\{ \mathbf{v} \in V \middle| B_{mn}\mathbf{v} = \mathbf{c} \right\}$が与えられたとき、次式が成り立つ。
\begin{align*}
\left\{ \mathbf{v} \in V \middle| A_{ln}\mathbf{v} = \mathbf{b} \right\} \cap \left\{ \mathbf{v} \in V \middle| B_{mn}\mathbf{v} = \mathbf{c} \right\} = \left\{ \mathbf{v} \in V \middle| \begin{pmatrix}
A_{ln} \\
B_{mn} \\
\end{pmatrix}\mathbf{v} = \begin{pmatrix}
\mathbf{b} \\
\mathbf{c} \\
\end{pmatrix} \right\}
\end{align*}
\end{thm}
\begin{proof}
体$K$上のvector空間$V$の$A_{ln} \in M_{ln}(K)$、$B_{mn} \in M_{mn}(K)$なる行列たち$A_{ln}$、$B_{mn}$を用いた部分空間たち$\left\{ \mathbf{v} \in V \middle| A_{ln}\mathbf{v} = \mathbf{b} \right\}$、$\left\{ \mathbf{v} \in V \middle| B_{mn}\mathbf{v} = \mathbf{c} \right\}$が与えられたとき、
\begin{align*}
\left\{ \mathbf{v} \in V \middle| A_{ln}\mathbf{v} = \mathbf{b} \right\} \cap \left\{ \mathbf{v} \in V \middle| B_{mn}\mathbf{v} = \mathbf{c} \right\} &= \left\{ \mathbf{v} \in V \middle| A_{ln}\mathbf{v} = \mathbf{b},\ \ B_{mn}\mathbf{v} = \mathbf{c} \right\}\\
&= \left\{ \mathbf{v} \in V \middle| \begin{pmatrix}
A_{ln}\mathbf{v} \\
B_{mn}\mathbf{v} \\
\end{pmatrix} = \begin{pmatrix}
\mathbf{b} \\
\mathbf{c} \\
\end{pmatrix} \right\}\\
&= \left\{ \mathbf{v} \in V \middle| \begin{pmatrix}
A_{ln} \\
B_{mn} \\
\end{pmatrix}\mathbf{v} = \begin{pmatrix}
\mathbf{b} \\
\mathbf{c} \\
\end{pmatrix} \right\}
\end{align*}
\end{proof}
%\hypertarget{ux3064ux306eux90e8ux5206ux7a7aux9593ux305fux3061ux306eux6b21ux5143ux305fux3061ux306eux548c}{%
\subsubsection{2つの部分空間たちの次元たちの和}%\label{ux3064ux306eux90e8ux5206ux7a7aux9593ux305fux3061ux306eux6b21ux5143ux305fux3061ux306eux548c}}
\begin{thm}\label{2.1.9.5}
体$K$上のvector空間$V$の部分空間たち$U$、$W$が与えられたとし、これらの交空間$U \cap W$を考え、これの基底を$\left\langle \mathbf{v}_{i} \right\rangle_{i \in \varLambda_{r}}$とおき、それらの部分空間たち$U$、$W$の基底をそれぞれ$\left\langle \begin{matrix}
\left( \mathbf{v}_{i} \right)_{i \in \varLambda_{r}} & \left( \mathbf{u}_{i} \right)_{i \in \varLambda_{r_{U} - r}} \\
\end{matrix} \right\rangle$、$\left\langle \begin{matrix}
\left( \mathbf{v}_{i} \right)_{i \in \varLambda_{r}} & \left( \mathbf{w}_{i} \right)_{i \in \varLambda_{r_{W} - r}} \\
\end{matrix} \right\rangle$とおく。このとき、その和空間$U + W$の基底の1つは$\left\langle \begin{matrix}
\left( \mathbf{v}_{i} \right)_{i \in \varLambda_{r}} & \left( \mathbf{u}_{i} \right)_{i \in \varLambda_{r_{U} - r}} & \left( \mathbf{w}_{i} \right)_{i \in \varLambda_{r_{W} - r}} \\
\end{matrix} \right\rangle$となり次式が成り立つ。
\begin{align*}
\dim{U \cap W} + \dim{U + W} = \dim U + \dim W
\end{align*}
\end{thm}
\begin{proof}
体$K$上のvector空間$V$の部分空間たち$U$、$W$が与えられたとし、これらの交空間$U \cap W$を考え、これの基底を$\left\langle \mathbf{v}_{i} \right\rangle_{i \in \varLambda_{r}}$とおく。なお、その次元$\dim{U \cap W}$を$r$とおいた。このとき、その交空間$U \cap W$はそれらの部分空間たち$U$、$W$いずれの部分空間となるので、それらの部分空間たち$U$、$W$の基底は、その組$\left\langle \mathbf{v}_{i} \right\rangle_{i \in \varLambda_{r}}$に適切なvectorsが追加されれば、得られるのであったので、それぞれ$\left\langle \begin{matrix}
\left( \mathbf{v}_{i} \right)_{i \in \varLambda_{r}} & \left( \mathbf{u}_{i} \right)_{i \in \varLambda_{r_{U} - r}} \\
\end{matrix} \right\rangle$、$\left\langle \begin{matrix}
\left( \mathbf{v}_{i} \right)_{i \in \varLambda_{r}} & \left( \mathbf{w}_{i} \right)_{i \in \varLambda_{r_{W} - r}} \\
\end{matrix} \right\rangle$とおく。なお、それらの次元たち$\dim U$、$\dim W$をそれぞれ$r_{U}$、$r_{W}$とおいた。このとき、それらの部分空間たち$U$、$W$の和空間$U + W$を考え、$\forall i \in \varLambda_{r}$に対し$k_{i},l_{i} \in K$、$\forall i \in \varLambda_{r_{U} - r}$に対し$u_{i} \in K$、$\forall i \in \varLambda_{r_{W} - r}$に対し$w_{i} \in K$なる元々$k_{i}$、$l_{i}$、$u_{i}$、$w_{i}$を用いれば、$\forall\mathbf{v} \in U + W$に対し次式が成り立つ。
\begin{align*}
\mathbf{v} = \sum_{i \in \varLambda_{r}} {k_{i}\mathbf{v}_{i}} + \sum_{i \in \varLambda_{r_{U} - r}} {u_{i}\mathbf{u}_{i}} + \sum_{i \in \varLambda_{r}} {l_{i}\mathbf{v}_{i}} + \sum_{i \in \varLambda_{r_{W} - r}} {w_{i}\mathbf{w}_{i}}
\end{align*}
したがって、次式が成り立つので、
\begin{align*}
\mathbf{v} = \sum_{i \in \varLambda_{r}} {\left( k_{i} + l_{i} \right)\mathbf{v}_{i}} + \sum_{i \in \varLambda_{r_{U} - r}} {u_{i}\mathbf{u}_{i}} + \sum_{i \in \varLambda_{r_{W} - r}} {w_{i}\mathbf{w}_{i}}
\end{align*}
次式が成り立つ。
\begin{align*}
U + W = {\mathrm{span}}\left( \left\{ \mathbf{v}_{i} \right\}_{i \in \varLambda_{r}} \cup \left\{ \mathbf{u}_{i} \right\}_{i \in \varLambda_{r_{U} - r}} \cup \left\{ \mathbf{w}_{i} \right\}_{i \in \varLambda_{r_{W} - r}} \right)
\end{align*}
ここで、$\forall i \in \varLambda_{r}$に対し、$c_{i} \in K$、$\forall i \in \varLambda_{r_{U} - r}$に対し、$d_{i} \in K$、$\forall i \in \varLambda_{r_{W} - r}$に対し$e_{i} \in K$なる元々$c_{i}$、$d_{i}$、$e_{i}$を用いて次式を考えよう。
\begin{align*}
\sum_{i \in \varLambda_{r}} {c_{i}\mathbf{v}_{i}} + \sum_{i \in \varLambda_{r_{U} - r}} {d_{i}\mathbf{u}_{i}} + \sum_{i \in \varLambda_{r_{W} - r}} {e_{i}\mathbf{w}_{i}} = \mathbf{0}
\end{align*}
したがって、次のようになる。
\begin{align*}
\sum_{i \in \varLambda_{r}} {c_{i}\mathbf{v}_{i}} + \sum_{i \in \varLambda_{r_{U} - r}} {d_{i}\mathbf{u}_{i}} = - \sum_{i \in \varLambda_{r_{W} - r}} {e_{i}\mathbf{w}_{i}} = \sum_{i \in \varLambda_{r_{W} - r}} {\left( - e_{i} \right)\mathbf{w}_{i}}
\end{align*}
ここで、このvector$\sum_{i \in \varLambda_{r}} {c_{i}\mathbf{v}_{i}} + \sum_{i \in \varLambda_{r_{U} - r}} {d_{i}\mathbf{u}_{i}}$を$\mathbf{w}$とおくと、そのvector$\mathbf{w}$はvectors$\mathbf{v}_{i}$とvectors$\mathbf{u}_{i}$によって生成されているので、その部分空間$U$に属し、そのvector$\mathbf{w}$はvectors$\mathbf{w}_{i}$によって生成されているので、その部分空間$W$に属する。したがって、そのvector$\mathbf{w}$はその交空間$U \cap W$に属することになり、その交空間$U \cap W$の基底がその組$\left\langle \mathbf{v}_{i} \right\rangle_{i \in \varLambda_{r}}$であったので、$\forall i \in \varLambda_{r}$に対し$c_{i}' \in K$なる元々$c_{i}'$を用いて$\mathbf{w} = \sum_{i \in \varLambda_{r}} {c_{i}'\mathbf{v}_{i}}$が成り立つ。したがって、次のようになる。
\begin{align*}
\sum_{i \in \varLambda_{r}} {c_{i}'\mathbf{v}_{i}} + \sum_{i \in \varLambda_{r_{W} - r}} {e_{i}\mathbf{w}_{i}} = \mathbf{0}
\end{align*}
ここで、その組$\left\langle \begin{matrix}
\left( \mathbf{v}_{i} \right)_{i \in \varLambda_{r}} & \left( \mathbf{w}_{i} \right)_{i \in \varLambda_{r_{W} - r}} \\
\end{matrix} \right\rangle$はその部分空間$W$の基底であったので、$i \in \varLambda_{r}$なるvectors$\mathbf{v}_{i}$と$i \in \varLambda_{r_{W} - r}$なるvectors$\mathbf{w}_{i}$は線形独立となる。したがって、$\forall i \in \varLambda_{r}$に対し、$c_{i}' = 0$が成り立つかつ、$\forall i \in \varLambda_{r_{W} - r}$に対し、$e_{i} = 0$が成り立つ。これにより、次のようになる。
\begin{align*}
\mathbf{w} = \sum_{i \in \varLambda_{r}} {c_{i}\mathbf{v}_{i}} + \sum_{i \in \varLambda_{r_{U} - r}} {d_{i}\mathbf{u}_{i}} = - \sum_{i \in \varLambda_{r_{W} - r}} {e_{i}\mathbf{w}_{i}} = - 0\sum_{i \in \varLambda_{r_{W} - r}} \mathbf{w}_{i} = \mathbf{0}
\end{align*}
したがって、次式が得られる。
\begin{align*}
\sum_{i \in \varLambda_{r}} {c_{i}\mathbf{v}_{i}} + \sum_{i \in \varLambda_{r_{U} - r}} {d_{i}\mathbf{u}_{i}} = \mathbf{0}
\end{align*}
ここで、その組$\left\langle \begin{matrix}
\left( \mathbf{v}_{i} \right)_{i \in \varLambda_{r}} & \left( \mathbf{u}_{i} \right)_{i \in \varLambda_{r_{U} - r}} \\
\end{matrix} \right\rangle$はその部分空間$U$の基底であったので、$i \in \varLambda_{r}$なるvectors$\mathbf{v}_{i}$と$i \in \varLambda_{r_{U} - r}$なるvectors$\mathbf{u}_{i}$は線形独立となる。したがって、$\forall i \in \varLambda_{r}$に対し、$c_{i} = 0$が成り立つかつ、$\forall i \in \varLambda_{r_{U} - r}$に対し、$d_{i} = 0$が成り立つ。以上より$i \in \varLambda_{r}$なるvectors$\mathbf{v}_{i}$と$i \in \varLambda_{r_{U} - r}$なるvectors$\mathbf{u}_{i}$と$i \in \varLambda_{r_{W} - r}$なるvectors$\mathbf{w}_{i}$は線形独立となる。\par
したがって、その組$\left\langle \begin{matrix}
\left( \mathbf{v}_{i} \right)_{i \in \varLambda_{r}} & \left( \mathbf{u}_{i} \right)_{i \in \varLambda_{r_{U} - r}} & \left( \mathbf{w}_{i} \right)_{i \in \varLambda_{r_{W} - r}} \\
\end{matrix} \right\rangle$はその和空間$U + W$の基底となり次式が成り立つ。
\begin{align*}
\dim{U + W} = r + \left( r_{U} - r \right) + \left( r_{W} - r \right)
\end{align*}
以上より、次のようになる。
\begin{align*}
\dim{U \cap W} + \dim{U + W} &= r + r + \left( r_{U} - r \right) + \left( r_{W} - r \right)\\
&= r_{U} + r_{W} + r - r + r - r\\
&= r_{U} + r_{W}\\
&= \dim U + \dim W
\end{align*}
\end{proof}
%\hypertarget{ux548cux7a7aux9593ux3068ux4ea4ux7a7aux9593ux3067ux4fbfux5229ux305dux3046ux306aux5b9aux7406ux305fux3061}{%
\subsubsection{和空間と交空間で便利そうな定理たち}%\label{ux548cux7a7aux9593ux3068ux4ea4ux7a7aux9593ux3067ux4fbfux5229ux305dux3046ux306aux5b9aux7406ux305fux3061}}\par
以上のような和空間と交空間について議論するときに便利そうな定理たちを次に3つ述べよう。
\begin{thm*}[定理\ref{2.1.8.4}の再掲]
  $\forall A_{mn} \in M_{mn}(K)\forall\mathbf{b} \in K^{m}$に対し、$A_{mn} = \begin{pmatrix}
    a_{11} & a_{12} & \cdots & a_{1n} \\
    a_{21} & a_{22} & \cdots & a_{2n} \\
     \vdots & \vdots & \ddots & \vdots \\
    a_{m1} & a_{m2} & \cdots & a_{mn} \\
    \end{pmatrix}$、$\mathbf{b} = \begin{pmatrix}
    b_{1} \\
    b_{2} \\
     \vdots \\
    b_{m} \\
    \end{pmatrix}$としてvectors$\mathbf{x} = \begin{pmatrix}
    x_{1} \\
    x_{2} \\
     \vdots \\
    x_{n} \\
    \end{pmatrix}$を用いた連立方程式$A_{mn}\mathbf{x} = \mathbf{b}$に随伴する連立方程式$A_{mn}\mathbf{x} = \mathbf{0}$は、次の方法にしたがって式変形されれば、必ずその解をもつことがわかる。この式$A_{mn}\mathbf{x} = \mathbf{0}$を解こう。この方法は次のようになる。
    \begin{enumerate}
    \item
      行基本変形と列の入れ替えによって${\mathrm{rank}}A_{mn} = r$として次のような行標準形に変形する。
    \begin{align*}
    A_{mn} \rightarrow \begin{pmatrix}
    1 & \cdots & 0 & a_{1,r + 1}' & \cdots & a_{1n}' \\
     \vdots & \ddots & \vdots & \vdots & \ddots & \vdots \\
    0 & \cdots & 1 & a_{r,r + 1}' & \cdots & a_{rn}' \\
    0 & \cdots & 0 & 0 & \cdots & 0 \\
     \vdots & \ddots & \vdots & \vdots & \ddots & \vdots \\
    0 & \cdots & 0 & 0 & \cdots & 0 \\
    \end{pmatrix}
    \end{align*}
    \item
      $P_{\mathrm{R}} \in {\mathrm{GL}}_{m}(K)$、$P_{\mathrm{C}} \in {\mathrm{GL}}_{n}(K)$なる行列たちそれぞれ$P_{\mathrm{R}}$、$P_{\mathrm{C}}$を用いると、行標準形にされたその行列は$P_{\mathrm{R}}A_{mn}P_{\mathrm{C}}$と書かれることができるのであった。そのvector$P_{\mathrm{C}}^{- 1}\mathbf{x}$はそのvector$\mathbf{x}$の成分の順序を入れ替えたものになることに注意すると、このvector$P_{\mathrm{C}}^{- 1}\mathbf{x}$は、ある全単射な写像$p:\varLambda_{n}\overset{\sim}{\rightarrow}\varLambda_{n}$が存在して、次式のように書かれることができる。
    \begin{align*}
    \mathbf{x}' = \begin{pmatrix}
    x_{p(1)} \\
    x_{p(2)} \\
     \vdots \\
    x_{p(n)} \\
    \end{pmatrix}
    \end{align*}
    \item
      したがって、次式が成り立つ。
    \begin{align*}
    \begin{pmatrix}
    x_{p(1)} \\
     \vdots \\
    x_{p(r)} \\
    x_{p(r + 1)} \\
     \vdots \\
    x_{p(n)} \\
    \end{pmatrix} = \begin{pmatrix}
     - a_{1,r + 1}'x_{p(r + 1)} - \cdots - a_{1n}'x_{p(n)} \\
     \vdots \\
     - a_{r,r + 1}'x_{p(r + 1)} - \cdots - a_{rn}'x_{p(n)} \\
    x_{p(r + 1)} \\
     \vdots \\
    x_{p(n)} \\
    \end{pmatrix}
    \end{align*}
    \item
      $\forall j \in \varLambda_{n} \setminus \varLambda_{r}$に対し、$t_{j - r} = x_{p(j)} \in K$とおくと、$\forall\begin{pmatrix}
      t_{1} \\
      t_{2} \\
       \vdots \\
      t_{n - r} \\
      \end{pmatrix} \in K^{n - r}$に対し、次式が成り立つ。
    \begin{align*}
    \begin{pmatrix}
    x_{p(1)} \\
     \vdots \\
    x_{p(r)} \\
    x_{p(r + 1)} \\
     \vdots \\
    x_{p(n)} \\
    \end{pmatrix} = t_{1}\begin{pmatrix}
     - a_{1,r + 1}' \\
     \vdots \\
     - a_{r,r + 1}' \\
    1 \\
     \vdots \\
    0 \\
    \end{pmatrix} + \cdots + t_{n - r}\begin{pmatrix}
     - a_{1n}' \\
     \vdots \\
     - a_{rn}' \\
    0 \\
     \vdots \\
    1 \\
    \end{pmatrix}
    \end{align*}
    \item
      これにより、その連立1次方程式$A_{mn}\mathbf{x} = \mathbf{0}$の解空間$\ker L_{A_{mn}}$は次のようになる。
    \begin{align*}
    \begin{pmatrix}
    x_{p(1)} \\
     \vdots \\
    x_{p(r)} \\
    x_{p(r + 1)} \\
     \vdots \\
    x_{p(n)} \\
    \end{pmatrix} \in \ker L_{A_{mn}} = {\mathrm{span} }\left\{ \begin{pmatrix}
     - a_{1,r + 1}' \\
     \vdots \\
     - a_{r,r + 1}' \\
    1 \\
     \vdots \\
    0 \\
    \end{pmatrix},\cdots,\begin{pmatrix}
     - a_{1n}' \\
     \vdots \\
     - a_{rn}' \\
    0 \\
     \vdots \\
    1 \\
    \end{pmatrix} \right\}
    \end{align*}
    \end{enumerate}
\end{thm*}
\begin{thm*}[定理\ref{2.1.8.9}の再掲]
    $\mathbf{a}_{j} \in K^{n}$なるvectors$\mathbf{a}_{j} = \begin{pmatrix}
    a_{1j} \\
    a_{2j} \\
     \vdots \\
    a_{mj} \\
    \end{pmatrix}$を用いた部分空間${\mathrm{span} }\left\{ \mathbf{a}_{j} \right\}_{j \in \varLambda_{n}}$の基底を求めよう。これは次のようにして求められることができる。
    \begin{enumerate}
    \item
      $A_{mn} = \left( \mathbf{a}_{j} \right)_{j \in \varLambda_{n}} = \begin{pmatrix}
      a_{11} & a_{12} & \cdots & a_{1n} \\
      a_{21} & a_{22} & \cdots & a_{2n} \\
       \vdots & \vdots & \ddots & \vdots \\
      a_{m1} & a_{m2} & \cdots & a_{mn} \\
      \end{pmatrix} \in M_{mn}(K)$なる行列$A_{mn}$と$\mathbf{x} = \begin{pmatrix}
      c_{1} \\
      c_{2} \\
       \vdots \\
      c_{m} \\
      \end{pmatrix}$なるvector$\mathbf{x}$を用いて考えよう。
    \item
      その行列$A_{mn}$は行基本変形と列の入れ替えによって次のように変形できるのであった。なお、$P_{\mathrm{R}}$、$P_{\mathrm{C}}$はそれぞれ$P_{\mathrm{R}} \in {\mathrm{GL}}_{m}(K)$、$P_{\mathrm{C}} \in {\mathrm{GL}}_{n}(K)$なる行列たちで${\mathrm{rank}}A_{mn} = r$とした。
    \begin{align*}
    A_{mn} \rightarrow \begin{pmatrix}
    1 & \cdots & 0 & a_{1,r + 1}' & \cdots & a_{1n}' \\
     \vdots & \ddots & \vdots & \vdots & \ddots & \vdots \\
    0 & \cdots & 1 & a_{r,r + 1}' & \cdots & a_{rn}' \\
    0 & \cdots & 0 & 0 & \cdots & 0 \\
     \vdots & \ddots & \vdots & \vdots & \ddots & \vdots \\
    0 & \cdots & 0 & 0 & \cdots & 0 \\
    \end{pmatrix} = P_{\mathrm{R}}A_{mn}P_{\mathrm{C}}
    \end{align*}
    \item
      そのvector$P_{\mathrm{C}}^{- 1}\mathbf{x}$はそのvector$\mathbf{x}$の成分の順序を入れ替えたものになることに注意すると、このvector$P_{\mathrm{C}}^{- 1}\mathbf{x}$は、ある全単射な写像$p:\varLambda_{n}\overset{\sim}{\rightarrow}\varLambda_{n}$が存在して、次式のように書かれることができる。
    \begin{align*}
    P_{\mathrm{C}}^{- 1}\mathbf{x} = \begin{pmatrix}
    c_{p(1)} \\
    c_{p(2)} \\
     \vdots \\
    c_{p(n)} \\
    \end{pmatrix}
    \end{align*}
    \item
      次の組$\left\langle \mathbf{a}_{p(j)} \right\rangle_{j \in \varLambda_{r}}$がその部分空間${\mathrm{span} }\left\{ \mathbf{a}_{j} \right\}_{j \in \varLambda_{n}}$の基底となる。
    \begin{align*}
    \left\langle \mathbf{a}_{p(j)} \right\rangle_{j \in \varLambda_{r}} = \left\langle \begin{pmatrix}
    a_{1p(1)} \\
    a_{2p(1)} \\
     \vdots \\
    a_{mp(1)} \\
    \end{pmatrix},\begin{pmatrix}
    a_{1p(2)} \\
    a_{2p(2)} \\
     \vdots \\
    a_{mp(2)} \\
    \end{pmatrix},\cdots,\begin{pmatrix}
    a_{1p(r)} \\
    a_{2p(r)} \\
     \vdots \\
    a_{mp(r)} \\
    \end{pmatrix} \right\rangle
    \end{align*}
    \end{enumerate}
\end{thm*}
\begin{thm*}[定理\ref{2.1.8.11}の再掲]
  体$K$上のvector空間$V$の部分空間${\mathrm{span} }\left\{ \mathbf{a}_{j} \right\}_{j \in \varLambda_{n}}$が与えられたとき、$A_{mn} = \left( \mathbf{a}_{j} \right)_{j \in \varLambda_{n}}$、${\mathrm{rank}}A_{mn} = r$として$P_{\mathrm{R}} \in {\mathrm{GL}}_{m}(K)$、$P_{\mathrm{C}} \in {\mathrm{GL}}_{n}(K)$なる行列たちを用いて行基本変形と列の入れ替えによって次のように変形されることができるとき、
  \begin{align*}
  A_{mn} \rightarrow \begin{pmatrix}
  I_{r} & * \\
  O & O \\
  \end{pmatrix} = P_{\mathrm{R}}A_{mn}P_{\mathrm{C}}
  \end{align*}
  $P_{\mathrm{R}}^{*} \in M_{m - r,m}(K)$として$P_{\mathrm{R}} = \begin{pmatrix}
  * \\
  P_{\mathrm{R}}^{*} \\
  \end{pmatrix}$とおくと、その部分空間${\mathrm{span} }\left\{ \mathbf{a}_{j} \right\}_{j \in \varLambda_{n}}$は次式のように書き換えられることができる。
  \begin{align*}
  {\mathrm{span} }\left\{ \mathbf{a}_{j} \right\}_{j \in \varLambda_{n}} = \left\{ \mathbf{v} \in K^{m} \middle| P_{\mathrm{R}}^{*}\mathbf{v} = \mathbf{0} \right\}
  \end{align*}
\end{thm*}\par
これはその部分空間${\mathrm{span}}\left\{ \mathbf{a}_{j} \right\}_{j \in \varLambda_{n}}$は連立1次方程式$P_{\mathrm{R}}^{*}\mathbf{v} = \mathbf{0}$の解空間に一致することを表している。
\begin{thebibliography}{50}
  \bibitem{1}
    対馬龍司, 線形代数学講義, 共立出版, 2007. 改訂版8刷 p129-133 ISBN978-4-320-11097-7
\end{thebibliography}
\end{document}
