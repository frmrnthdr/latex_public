\documentclass[dvipdfmx]{jsarticle}
\setcounter{section}{4}
\setcounter{subsection}{3}
\usepackage{xr}
\externaldocument{4.4.1}
\externaldocument{4.4.3}
\usepackage{amsmath,amsfonts,amssymb,array,comment,mathtools,url,docmute}
\usepackage{longtable,booktabs,dcolumn,tabularx,mathtools,multirow,colortbl,xcolor}
\usepackage[dvipdfmx]{graphics}
\usepackage{bmpsize}
\usepackage{amsthm}
\usepackage{enumitem}
\setlistdepth{20}
\renewlist{itemize}{itemize}{20}
\setlist[itemize]{label=•}
\renewlist{enumerate}{enumerate}{20}
\setlist[enumerate]{label=\arabic*.}
\setcounter{MaxMatrixCols}{20}
\setcounter{tocdepth}{3}
\newcommand{\rotin}{\text{\rotatebox[origin=c]{90}{$\in $}}}
\newcommand{\amap}[6]{\text{\raisebox{-0.7cm}{\begin{tikzpicture} 
  \node (a) at (0, 1) {$\textstyle{#2}$};
  \node (b) at (#6, 1) {$\textstyle{#3}$};
  \node (c) at (0, 0) {$\textstyle{#4}$};
  \node (d) at (#6, 0) {$\textstyle{#5}$};
  \node (x) at (0, 0.5) {$\rotin $};
  \node (x) at (#6, 0.5) {$\rotin $};
  \draw[->] (a) to node[xshift=0pt, yshift=7pt] {$\textstyle{\scriptstyle{#1}}$} (b);
  \draw[|->] (c) to node[xshift=0pt, yshift=7pt] {$\textstyle{\scriptstyle{#1}}$} (d);
\end{tikzpicture}}}}
\newcommand{\twomaps}[9]{\text{\raisebox{-0.7cm}{\begin{tikzpicture} 
  \node (a) at (0, 1) {$\textstyle{#3}$};
  \node (b) at (#9, 1) {$\textstyle{#4}$};
  \node (c) at (#9+#9, 1) {$\textstyle{#5}$};
  \node (d) at (0, 0) {$\textstyle{#6}$};
  \node (e) at (#9, 0) {$\textstyle{#7}$};
  \node (f) at (#9+#9, 0) {$\textstyle{#8}$};
  \node (x) at (0, 0.5) {$\rotin $};
  \node (x) at (#9, 0.5) {$\rotin $};
  \node (x) at (#9+#9, 0.5) {$\rotin $};
  \draw[->] (a) to node[xshift=0pt, yshift=7pt] {$\textstyle{\scriptstyle{#1}}$} (b);
  \draw[|->] (d) to node[xshift=0pt, yshift=7pt] {$\textstyle{\scriptstyle{#2}}$} (e);
  \draw[->] (b) to node[xshift=0pt, yshift=7pt] {$\textstyle{\scriptstyle{#1}}$} (c);
  \draw[|->] (e) to node[xshift=0pt, yshift=7pt] {$\textstyle{\scriptstyle{#2}}$} (f);
\end{tikzpicture}}}}
\renewcommand{\thesection}{第\arabic{section}部}
\renewcommand{\thesubsection}{\arabic{section}.\arabic{subsection}}
\renewcommand{\thesubsubsection}{\arabic{section}.\arabic{subsection}.\arabic{subsubsection}}
\everymath{\displaystyle}
\allowdisplaybreaks[4]
\usepackage{vtable}
\theoremstyle{definition}
\newtheorem{thm}{定理}[subsection]
\newtheorem*{thm*}{定理}
\newtheorem{dfn}{定義}[subsection]
\newtheorem*{dfn*}{定義}
\newtheorem{axs}[dfn]{公理}
\newtheorem*{axs*}{公理}
\renewcommand{\headfont}{\bfseries}
\makeatletter
  \renewcommand{\section}{%
    \@startsection{section}{1}{\z@}%
    {\Cvs}{\Cvs}%
    {\normalfont\huge\headfont\raggedright}}
\makeatother
\makeatletter
  \renewcommand{\subsection}{%
    \@startsection{subsection}{2}{\z@}%
    {0.5\Cvs}{0.5\Cvs}%
    {\normalfont\LARGE\headfont\raggedright}}
\makeatother
\makeatletter
  \renewcommand{\subsubsection}{%
    \@startsection{subsubsection}{3}{\z@}%
    {0.4\Cvs}{0.4\Cvs}%
    {\normalfont\Large\headfont\raggedright}}
\makeatother
\makeatletter
\renewenvironment{proof}[1][\proofname]{\par
  \pushQED{\qed}%
  \normalfont \topsep6\p@\@plus6\p@\relax
  \trivlist
  \item\relax
  {
  #1\@addpunct{.}}\hspace\labelsep\ignorespaces
}{%
  \popQED\endtrivlist\@endpefalse
}
\makeatother
\renewcommand{\proofname}{\textbf{証明}}
\usepackage{tikz,graphics}
\usepackage[dvipdfmx]{hyperref}
\usepackage{pxjahyper}
\hypersetup{
 setpagesize=false,
 bookmarks=true,
 bookmarksdepth=tocdepth,
 bookmarksnumbered=true,
 colorlinks=false,
 pdftitle={},
 pdfsubject={},
 pdfauthor={},
 pdfkeywords={}}
\begin{document}
%\hypertarget{lagrangeux306eux672aux5b9aux4e57ux6570ux6cd5}{%
\subsection{Lagrangeの未定乗数法}%\label{lagrangeux306eux672aux5b9aux4e57ux6570ux6cd5}}
%\hypertarget{lagrangeux306eux672aux5b9aux4e57ux6570ux6cd5-1}{%
\subsubsection{Lagrangeの未定乗数法}%\label{lagrangeux306eux672aux5b9aux4e57ux6570ux6cd5-1}}
\begin{dfn}
$U \subseteq \mathbb{R}^{n}$なる開集合$U$を用いた$C^{1}$級関数$f:U \rightarrow \mathbb{R}$、その開集合$U$の部分集合$M$が与えられたとき、$\mathbf{a} \in M$なる点$\mathbf{a}$のある近傍$V$が存在して、$f\left( \mathbf{a} \right) = \max{f|\mathrm{int}V \cap M}$が成り立つようなその点$\mathbf{a}$をその関数$f$のその部分集合$M$上の極大点といいその値$f\left( \mathbf{a} \right)$をその関数$f$のその部分集合$M$上の極大値という。\par
同様にして、$U \subseteq \mathbb{R}^{n}$なる開集合$U$を用いた$C^{1}$級関数$f:U \rightarrow \mathbb{R}$、その開集合$U$の部分集合$M$が与えられたとき、$\mathbf{a} \in M$なる点$\mathbf{a}$のある近傍$V$が存在して、$f\left( \mathbf{a} \right) = \min{f|\mathrm{int}V \cap M}$が成り立つようなその点$\mathbf{a}$をその関数$f$のその部分集合$M$上の極小点といいその値$f\left( \mathbf{a} \right)$をその関数$f$のその部分集合$M$上の極小値という。
\end{dfn}
\begin{dfn}
$U \subseteq \mathbb{R}^{n}$なる開集合$U$を用いた$C^{1}$級関数$f:U \rightarrow \mathbb{R}$、その開集合$U$の部分集合$M$が与えられたとき、その関数$f$のその部分集合$M$上の極大値または極小値のことをその関数$f$のその部分集合$M$上の極値という。
\end{dfn}
\begin{dfn}
$U \subseteq \mathbb{R}^{n}$なる開集合$U$を用いた$C^{1}$級関数$f:U \rightarrow \mathbb{R}$、その開集合$U$の部分集合$M$が与えられたとき、$\mathbf{a} \in M$なる点$\mathbf{a}$のある近傍$V$が存在して、$f\left( \mathbf{a} \right) = \max{f|\mathrm{int}V \cap M}$が成り立つその関数$f$のその部分集合$M$上の極大点$\mathbf{a}$のうち、$\forall\mathbf{x} \in \mathrm{int}V$に対し、$\mathbf{x} \neq \mathbf{a}$が成り立つなら、$f\left( \mathbf{x} \right) < f\left( \mathbf{a} \right)$が成り立つようなものをその関数$f$のその部分集合$M$上の狭義の極大点といいその値$f\left( \mathbf{a} \right)$をその関数$f$のその部分集合$M$上の狭義の極大値という。\par
同様にして、$U \subseteq \mathbb{R}^{n}$なる開集合$U$を用いた$C^{1}$級関数$f:U \rightarrow \mathbb{R}$、その開集合$U$の部分集合$M$が与えられたとき、$\mathbf{a} \in M$なる点$\mathbf{a}$のある近傍$V$が存在して、$f\left( \mathbf{a} \right) = \min{f|\mathrm{int}V \cap M}$が成り立つその関数$f$のその部分集合$M$上の極大点$\mathbf{a}$のうち、$\forall\mathbf{x} \in \mathrm{int}V$に対し、$\mathbf{x} \neq \mathbf{a}$が成り立つなら、$f\left( \mathbf{x} \right) > f\left( \mathbf{a} \right)$が成り立つようなものをその関数$f$のその部分集合$M$上の狭義の極小点といいその値$f\left( \mathbf{a} \right)$をその関数$f$のその部分集合$M$上の狭義の極小値という。
\end{dfn}
\begin{thm}\label{4.4.4.1}
$U \subseteq \mathbb{R}^{n}$なる開集合$U$を用いた$C^{1}$級関数$f:U \rightarrow \mathbb{R}$が与えられたとき、値$f\left( \mathbf{a} \right)$がその関数$f$の極大値となるならそのときに限り、それぞれその値$f\left( \mathbf{a} \right)$はその関数$f$のその集合$U$上の極大値である。同様に、値$f\left( \mathbf{a} \right)$がその関数$f$の極小値となるならそのときに限り、それぞれその値$f\left( \mathbf{a} \right)$はその関数$f$のその集合$U$上の極小値である。
\end{thm}
\begin{proof}
$U \subseteq \mathbb{R}^{n}$なる開集合$U$を用いた$C^{1}$級関数$f:U \rightarrow \mathbb{R}$が与えられたとき、値$f\left( \mathbf{a} \right)$がその関数$f$の極大値となるなら、$\mathbf{a} \in \mathbb{R}^{n}$なる点$\mathbf{a}$のある$\varepsilon$近傍$U\left( \mathbf{a},\varepsilon \right)$がその定義域$U$の部分集合となるようにその点$\mathbf{a}$をとったとき、値$f\left( \mathbf{a} \right)$が最大値$\max{V\left( f|U\left( \mathbf{a},\varepsilon \right) \right)}$に等しい、即ち、その関数$fのその集合U\left( \mathbf{a},\varepsilon \right)$での最大値となることになる。そこで、そのような点$a$全体の集合がまさしくその開核$\mathrm{int}U$であり、その定義域$U$が開集合なので、$U = \mathrm{int}U$が成り立つかつ、$U\left( \mathbf{a},\varepsilon \right) \subseteq U$より$U\left( \mathbf{a},\varepsilon \right) \cap U = U\left( \mathbf{a},\varepsilon \right)$が成り立つかつ、その$\varepsilon$近傍$U\left( \mathbf{a},\varepsilon \right)$はその点$\mathbf{a}$の近傍でもあることに注意すれば、$\mathbf{a} \in \mathrm{int}U = U$なる点$\mathbf{a}$のある近傍$U\left( \mathbf{a},\varepsilon \right)$が存在して、次式が成り立つ。
\begin{align*}
f\left( \mathbf{a} \right) = \max{V\left( f|U\left( \mathbf{a},\varepsilon \right) \right)} = \max{f|U\left( \mathbf{a},\varepsilon \right)} = \max{f|U\left( \mathbf{a},\varepsilon \right) \cap U}
\end{align*}
ゆえに、その値$f\left( \mathbf{a} \right)$はその関数$f$のその集合$U$上の極大値である。\par
逆に、その値$f\left( \mathbf{a} \right)$がその関数$f$のその集合$U$上の極大値であるなら、$\mathbf{a} \in U$なる点$\mathbf{a}$のある近傍$V$が存在して、$f\left( \mathbf{a} \right) = \max{f|\mathrm{int}V \cap U}$が成り立つ。そこで、集合$\mathrm{int}V \cap U$は開集合なので、ある正の実数$\varepsilon$が存在して、$U\left( \mathbf{a},\varepsilon \right) \subseteq \mathrm{int}V \cap U$が成り立つようにすることができる。このとき、もちろん、次式が成り立つ。
\begin{align*}
f\left( \mathbf{a} \right) = \max{f|\mathrm{int}V \cap U} = \max{f|U\left( \mathbf{a},\varepsilon \right)} = \max{V\left( f|U\left( \mathbf{a},\varepsilon \right) \right)}
\end{align*}
ゆえに、その値$f\left( \mathbf{a} \right)$はその関数$f$の極大値となる。\par
極小値についても同様にして示される。
\end{proof}
\begin{thm}\label{4.4.4.2}
$U \subseteq \mathbb{R}^{m}$なる開集合$U$を用いた$C^{1}$級関数たち$f:U \rightarrow \mathbb{R}$、$g:U \rightarrow \mathbb{R}^{n}$が与えられたとき、$S = \left\{ \mathbf{x} \in U \middle| g\left( \mathbf{x} \right) = \mathbf{0} \right\}$において、その関数$f$がその集合$S$上の極値$f\left( \mathbf{a} \right)$をとるかつ、$\mathrm{rank}{J_{g}\left( \mathbf{a} \right)} = n$が成り立つなら、$\exists\mathbf{l} = \left( l_{i} \right)_{i \in \varLambda_{n}} \in \mathbb{R}^{n}$に対し、$\mathrm{grad}f\left( \mathbf{a} \right) ={}^t J_{g}\left( \mathbf{a} \right)\mathbf{l}$が成り立つ。
\end{thm}
\begin{proof}
$U \subseteq \mathbb{R}^{m}$なる開集合$U$を用いた$C^{1}$級関数たち$f:U \rightarrow \mathbb{R}$、$g = \left( g_{i} \right)_{i \in \varLambda_{n}}:U \rightarrow \mathbb{R}^{n}$が与えられたとき、$S = \left\{ \mathbf{x} \in U \middle| g\left( \mathbf{x} \right) = \mathbf{0} \right\}$において、その関数$f$がその集合$S$上の極値$f\left( \mathbf{a} \right)$をとるかつ、$\mathrm{rank}{J_{g}\left( \mathbf{a} \right)} = n$が成り立つなら、$n \leq m$が成り立つことになり、添数を適切におくことにより、$\nabla^{*} = \left( \partial_{i} \right)_{i \in \varLambda_{m - n}}$、$\nabla_{*} = \left( \partial_{i} \right)_{i \in \varLambda_{m} \setminus \varLambda_{m - n}}$とおくと、$\det{{}^t \left( \nabla_{*}{}^t g \right)}\left( \mathbf{a} \right) \neq 0$が成り立つとしてもよい。さらに、$\mathbf{a}^{*} = \left( a_{i} \right)_{i \in \varLambda_{m - n}}$、$\mathbf{a}_{*} = \left( a_{i} \right)_{i \in \varLambda_{m} \setminus \varLambda_{m - n}}$、$\mathbf{x}^{*} = \left( x_{i} \right)_{i \in \varLambda_{m - n}}$、$\mathbf{x}_{*} = \left( x_{i} \right)_{i \in \varLambda_{m} \setminus \varLambda_{m - n}}$とおくと、定理\ref{4.4.1.3}、即ち、よりよい陰関数定理よりそれらの点々$\mathbf{a}_{1}$、$\mathbf{a}_{2}$のある近傍たち$V$、$W$と$C^{1}$級関数$\varphi:V \rightarrow W$が存在して、$V \times W \subseteq U$で、$\forall\mathbf{x} \in V \times W$に対し、次式が成り立ち、
\begin{align*}
\mathbf{x} = \begin{pmatrix}
\mathbf{x}^{*} \\
\mathbf{x}_{*} \\
\end{pmatrix} \in S \Leftrightarrow g\left( \mathbf{x} \right) = \mathbf{0} \Leftrightarrow \mathbf{x}_{*} = \varphi\left( \mathbf{x}^{*} \right)
\end{align*}
さらに、$\varphi_{\downarrow}:V \rightarrow U;\mathbf{x}^{*} \mapsto \begin{pmatrix}
\mathbf{x}^{*} \\
\varphi\left( \mathbf{x}^{*} \right) \\
\end{pmatrix}$とすれば、その陰関数$\varphi$はその近傍$V$で次式を満たす。
\begin{align*}
J_{\varphi} = - \left({}^t \left( \nabla_{*}{}^t g \right)^{- 1}{}^{t}\left( \nabla^{*}{}^{t}g \right) \right) \circ \varphi_{\downarrow}
\end{align*}\par
$F = f \circ \varphi_{\downarrow}:V \rightarrow \mathbb{R}$のように関数$F$が定義されると、その関数$F$は$V \subseteq \mathbb{R}^{m - n}$なる開集合$V$を定義域とする$C^{1}$級の関数であり、仮定より$\mathbf{x} = \mathbf{a}$のときその集合$S$上の極値をとるので、$\mathbf{x}^{*} = \mathbf{a}^{*}$のとき、その関数$F$は極値をとる。そこで、定理\ref{4.4.3.1}より$\mathrm{grad}F\left( \mathbf{a}^{*} \right) = \mathbf{0}$が成り立つ。したがって、次のようになり、
\begin{align*}
\mathrm{grad}F &={}^t J_{F} ={}^t J_{f \circ \varphi_{\downarrow}}\\
&={}^t \left( \left( J_{f} \circ \varphi_{\downarrow} \right)J_{\varphi_{\downarrow}} \right)\\
&={}^t J_{\varphi_{\downarrow}}{}^t \left( J_{f} \circ \varphi_{\downarrow} \right)\\
&={}^t \begin{pmatrix}
I_{m - n} \\
J_{\varphi} \\
\end{pmatrix}\left( \mathrm{grad}f \circ \varphi_{\downarrow} \right)\\
&= \begin{pmatrix}
{}^t I_{m - n} &{}^t J_{\varphi} \\
\end{pmatrix}\left( \mathrm{grad}f \circ \varphi_{\downarrow} \right)\\
&= \begin{pmatrix}
I_{m - n} &{}^t J_{\varphi} \\
\end{pmatrix}\begin{pmatrix}
\nabla^{*}f \circ \varphi_{\downarrow} \\
\nabla_{*}f \circ \varphi_{\downarrow} \\
\end{pmatrix}\\
&= \nabla^{*}f \circ \varphi_{\downarrow} +{}^t J_{\varphi}\left( \nabla_{*}f \circ \varphi_{\downarrow} \right)\\
&= \nabla^{*}f \circ \varphi_{\downarrow} -{}^t \left( \left({}^t \left( \nabla_{*}{}^t g \right)^{- 1}{}^t \left( \nabla^{*}{}^t g \right) \right) \circ \varphi_{\downarrow} \right)\left( \nabla_{*}f \circ \varphi_{\downarrow} \right)\\
&= \nabla^{*}f \circ \varphi_{\downarrow} - \left( \nabla^{*}{}^t g \circ \varphi_{\downarrow} \right)\left( \left( \nabla_{*}{}^t g \right)^{- 1} \circ \varphi_{\downarrow} \right)\left( \nabla_{*}f \circ \varphi_{\downarrow} \right)\\
&= \nabla^{*}f \circ \varphi_{\downarrow} - \left( \nabla^{*}{}^t g \circ \varphi_{\downarrow} \right){}^t \left( \left({}^t \nabla_{*}f \circ \varphi_{\downarrow} \right)\left({}^t \left( \nabla_{*}{}^{t}g \right)^{- 1} \circ \varphi_{\downarrow} \right) \right)\\
&= \nabla^{*}f \circ \varphi_{\downarrow} - \left( \nabla^{*}{}^t g \circ \varphi_{\downarrow} \right){}^t \left( \left({}^t \nabla_{*}f{}^t \left( \nabla_{*}{}^{t}g \right)^{- 1} \right) \circ \varphi_{\downarrow} \right)
\end{align*}
そこで、次式のようにvector$\mathbf{l}$がおかれれば、
\begin{align*}
\mathbf{l} &={}^t \left( \left({}^t \nabla_{*}f{}^t \left( \nabla_{*}{}^{t}g \right)^{- 1} \right) \circ \varphi_{\downarrow}\left( \mathbf{a}^{*} \right) \right)\\
&={}^t \left({}^t \nabla_{*}f{}^t \left( \nabla_{*}{}^{t}g \right)^{- 1} \right) \circ \varphi_{\downarrow}\left( \mathbf{a}^{*} \right)\\
&= \left( \left( \nabla_{*}{}^t g \right)^{- 1}\nabla_{*}f \right) \circ \varphi_{\downarrow}\left( \mathbf{a}^{*} \right)\\
&= \left( \nabla_{*}{}^t g \right)^{- 1} \circ \varphi_{\downarrow}\left( \mathbf{a}^{*} \right)\nabla_{*}f \circ \varphi_{\downarrow}\left( \mathbf{a}^{*} \right)
\end{align*}
次式が成り立つ。
\begin{align*}
\mathrm{grad}F\left( \mathbf{a}^{*} \right) &= \nabla^{*}f \circ \varphi_{\downarrow}\left( \mathbf{a}^{*} \right) - \nabla^{*}{}^t g \circ \varphi_{\downarrow}\left( \mathbf{a}^{*} \right){}^t \left( \left({}^t \nabla_{*}f{}^t \left( \nabla_{*}{}^t g \right)^{- 1} \right) \circ \varphi_{\downarrow}\left( \mathbf{a}^{*} \right) \right)\\
&= \nabla^{*}f \circ \varphi_{\downarrow}\left( \mathbf{a}^{*} \right) - \nabla^{*}{}^t g \circ \varphi_{\downarrow}\left( \mathbf{a}^{*} \right)\mathbf{l} = \mathbf{0}
\end{align*}
以上より、次式が得られる。
\begin{align*}
\nabla^{*}f \circ \varphi_{\downarrow}\left( \mathbf{a}^{*} \right) = \nabla^{*}{}^t g \circ \varphi_{\downarrow}\left( \mathbf{a}^{*} \right)\mathbf{l},\ \ \nabla_{*}f \circ \varphi_{\downarrow}\left( \mathbf{a}^{*} \right) = \nabla_{*}{}^t g \circ \varphi_{\downarrow}\left( \mathbf{a}^{*} \right)\mathbf{l}
\end{align*}
これにより、次のようになる。
\begin{align*}
\mathrm{grad}f\left( \mathbf{a} \right) &= \begin{pmatrix}
\nabla^{*}f \\
\nabla_{*}f \\
\end{pmatrix}\left( \mathbf{a} \right)\\
&= \begin{pmatrix}
\nabla^{*}f \circ \varphi_{\downarrow}\left( \mathbf{a}^{*} \right) \\
\nabla_{*}f \circ \varphi_{\downarrow}\left( \mathbf{a}^{*} \right) \\
\end{pmatrix}\\
&= \begin{pmatrix}
\nabla^{*}{}^t g \circ \varphi_{\downarrow}\left( \mathbf{a}^{*} \right)\mathbf{l} \\
\nabla_{*}{}^t g \circ \varphi_{\downarrow}\left( \mathbf{a}^{*} \right)\mathbf{l} \\
\end{pmatrix}\\
&= \begin{pmatrix}
\nabla^{*}{}^t g \\
\nabla_{*}{}^t g \\
\end{pmatrix} \circ \varphi_{\downarrow}\left( \mathbf{a}^{*} \right)\mathbf{l}\\
&= \left({}^t \begin{pmatrix}
{}^t \nabla^{*}{}^t g &{}^t \nabla_{*}{}^t g \\
\end{pmatrix} \circ \varphi_{\downarrow}\left( \mathbf{a}^{*} \right) \right)\mathbf{l}\\
&={}^t J_{g}\left( \mathbf{a} \right)\mathbf{l}
\end{align*}
\end{proof}
\begin{thm}[Lagrangeの未定乗数法]\label{4.4.4.3}
$U \subseteq \mathbb{R}^{m}$なる開集合$U$を用いた$C^{1}$級関数たち$f:U \rightarrow \mathbb{R}$、$g:U \rightarrow \mathbb{R}^{n}$が与えられたとき、$S = \left\{ \mathbf{x} \in U \middle| g\left( \mathbf{x} \right) = \mathbf{0} \right\}$において、その関数$f$がその集合$S$上の極値$f\left( \mathbf{a} \right)$をとるなら、次のいづれかが成り立つ。
\begin{itemize}
\item
  次式のような関数$\varPhi$が用いられれば、
\begin{align*}
\varPhi:U \times \mathbb{R}^{n} \rightarrow \mathbb{R};\begin{pmatrix}
\mathbf{x} \\
\mathbf{y} \\
\end{pmatrix} \mapsto f\left( \mathbf{x} \right) -{}^t \mathbf{y}g\left( \mathbf{x} \right)
\end{align*}
$\exists\mathbf{l} \in \mathbb{R}^{n}$に対し、$\mathrm{grad}\varPhi\begin{pmatrix}
\mathbf{a} \\
\mathbf{l} \\
\end{pmatrix} = \mathbf{0}$が成り立つ。
\item
  $\mathrm{rank}{J_{g}\left( \mathbf{a} \right)} < n$が成り立つ。
\end{itemize}
この定理をLagrangeの未定乗数法といい、そのvector$\mathbf{l}$をLagrange乗数という。
\end{thm}
\begin{proof}
$U \subseteq \mathbb{R}^{m}$なる開集合$U$を用いた$C^{1}$級関数たち$f:U \rightarrow \mathbb{R}$、$g:U \rightarrow \mathbb{R}^{n}$が与えられたとき、$S = \left\{ \mathbf{x} \in U \middle| g\left( \mathbf{x} \right) = \mathbf{0} \right\}$において、その関数$f$がその集合$S$上の極値$f\left( \mathbf{a} \right)$をとるなら、$\mathrm{rank}{J_{g}\left( \mathbf{a} \right)} < n$が成り立たないとき、$\mathrm{rank}{J_{g}\left( \mathbf{a} \right)} \leq \min\left\{ m,n \right\}$より$m < n$が成り立つとすれば、$\mathrm{rank}{J_{g}\left( \mathbf{a} \right)} \leq m < n \leq \mathrm{rank}{J_{g}\left( \mathbf{a} \right)}$が成り立つことになり、これは矛盾している。ゆえに、$n \leq m$が成り立つことになり、このとき、$n \leq \mathrm{rank}{J_{g}\left( \mathbf{a} \right)} \leq n$が得られるので、$\mathrm{rank}{J_{g}\left( \mathbf{a} \right)} = n$が成り立つ。したがって、定理\ref{4.4.4.2}より$\exists\mathbf{l} = \left( l_{i} \right)_{i \in \varLambda_{n}} \in \mathbb{R}^{n}$に対し、$\mathrm{grad}f\left( \mathbf{a} \right) ={}^t J_{g}\left( \mathbf{a} \right)\mathbf{l}$が成り立つ。そこで、$\mathbf{a} \in S$より$g\left( \mathbf{a} \right) = \mathbf{0}$が成り立つことに注意すれば、次のようにして
\begin{align*}
\varPhi = f \circ 1^{*} -{}^t 1_{*}\left( g \circ 1^{*} \right):U \times \mathbb{R}^{n} &\rightarrow \mathbb{R}\begin{pmatrix}
  \mathbf{x} \\
  \mathbf{y} \\
\end{pmatrix} \mapsto f\left( \mathbf{x} \right) -{}^t \mathbf{y}g\left( \mathbf{x} \right)\\
1^{*}:\mathbb{R}^{m} \times \mathbb{R}^{n} \rightarrow \mathbb{R}^{n};\begin{pmatrix}
\mathbf{x} \\
\mathbf{y} \\
\end{pmatrix} \mapsto \mathbf{x}&,\ \ 
1_{*}:\mathbb{R}^{m} \times \mathbb{R}^{n} \rightarrow \mathbb{R}^{n};\begin{pmatrix}
\mathbf{x} \\
\mathbf{y} \\
\end{pmatrix} \mapsto \mathbf{y} 
\end{align*}
したがって、次のようになり、
\begin{align*}
\mathrm{grad}\varPhi &= \mathrm{grad}{f \circ 1^{*}} - \mathrm{grad}\left({}^t 1_{*}\left( g \circ 1^{*} \right) \right)\\
&= \mathrm{grad}\left( f \circ 1^{*} \right) -{}^t J_{1_{*}}\left( g \circ 1^{*} \right) -{}^{t}J_{g \circ 1^{*}}1_{*}\\
&={}^t J_{1^{*}}\left( \mathrm{grad}f \circ 1^{*} \right) -{}^t J_{1_{*}}\left( g \circ 1^{*} \right) -{}^{t}J_{1^{*}}{}^t \left( J_{g} \circ 1^{*} \right)1_{*}\\
&= \begin{pmatrix}
I \\
O \\
\end{pmatrix}\left( \mathrm{grad}f \circ 1^{*} \right) - \begin{pmatrix}
O \\
I \\
\end{pmatrix}\left( g \circ 1^{*} \right) - \begin{pmatrix}
I \\
O \\
\end{pmatrix}{}^t \left( J_{g} \circ 1^{*} \right)1_{*}\\
&= \begin{pmatrix}
\mathrm{grad}f \circ 1^{*} \\
O \\
\end{pmatrix} - \begin{pmatrix}
O \\
g \circ 1^{*} \\
\end{pmatrix} - \begin{pmatrix}
{}^t \left( J_{g} \circ 1^{*} \right)1_{*} \\
O \\
\end{pmatrix}\\
&= \begin{pmatrix}
\mathrm{grad}f \circ 1^{*} -{}^t \left( J_{g} \circ 1^{*} \right)1_{*} \\
 - g \circ 1^{*} \\
\end{pmatrix}
\end{align*}
したがって、次のようになる。
\begin{align*}
\mathrm{grad}\varPhi\begin{pmatrix}
\mathbf{a} \\
\mathbf{l} \\
\end{pmatrix} &= \begin{pmatrix}
\mathrm{grad}f \circ 1^{*} -{}^t \left( J_{g} \circ 1^{*} \right)1_{*} \\
 - g \circ 1^{*} \\
\end{pmatrix}\begin{pmatrix}
\mathbf{a} \\
\mathbf{l} \\
\end{pmatrix}\\
&= \begin{pmatrix}
\mathrm{grad}f \circ 1^{*}\begin{pmatrix}
\mathbf{a} \\
\mathbf{l} \\
\end{pmatrix} -{}^t \left( J_{g} \circ 1^{*} \right)\begin{pmatrix}
\mathbf{a} \\
\mathbf{l} \\
\end{pmatrix}1_{*}\begin{pmatrix}
\mathbf{a} \\
\mathbf{l} \\
\end{pmatrix} \\
 - g \circ 1^{*}\begin{pmatrix}
\mathbf{a} \\
\mathbf{l} \\
\end{pmatrix} \\
\end{pmatrix}\\
&= \begin{pmatrix}
\mathrm{grad}f\left( \mathbf{a} \right) - \mathrm{grad}f\left( \mathbf{a} \right) \\
 - g\left( \mathbf{a} \right) \\
\end{pmatrix}\\
&= \begin{pmatrix}
\mathbf{0} \\
\mathbf{- 0} \\
\end{pmatrix} = \mathbf{0}
\end{align*}
よって、$\exists\mathbf{l} \in \mathbb{R}^{n}$に対し、$\mathrm{grad}\varPhi\begin{pmatrix}
\mathbf{a} \\
\mathbf{l} \\
\end{pmatrix} = \mathbf{0}$が成り立つ。\par
あとは明らかであろう。
\end{proof}
%\hypertarget{ux6975ux5024ux306eux8a08ux7b97ux4f8b}{%
\subsubsection{極値の計算例}%\label{ux6975ux5024ux306eux8a08ux7b97ux4f8b}}\par
極値を求める際に便利な定理たちを挙げよう。
\begin{dfn*}[定義\ref{首座小行列}の再掲]
行列$B$が$B = \left( b_{ij} \right)_{(i,j) \in \varLambda_{n}^{2}}$と与えられたとき、$\forall k \in \varLambda_{n}$に対し、次のような写像$P_{k}$によるその行列$B$の像を$k$次首座小行列という。
\begin{align*}
P_{k}:M_{nn}\left( \mathbb{R} \right) \rightarrow M_{kk}\left( \mathbb{R} \right);B \mapsto \left( b_{ij} \right)_{(i,j) \in \varLambda_{k}^{2}}
\end{align*}
\end{dfn*}
\begin{thm*}[定理\ref{4.4.3.1}の再掲]
$D(f) \subseteq \mathbb{R}^{n}$なる関数$f:D(f) \rightarrow \mathbb{R}$が与えられたとき、その集合$D(f)の内点\mathbf{a}$で極値をとりその関数$f$がその点$\mathbf{a}$で微分可能であるなら、$\forall i \in \varLambda_{n}$に対し、次式が成り立つ。
\begin{align*}
\partial_{i}f\left( \mathbf{a} \right) = 0
\end{align*}\par
これにより、その点$\mathbf{a}$はその関数$f$の停留点であり、さらに、$(df)_{\mathbf{a}} = 0$が成り立つ。
\end{thm*}
\begin{thm*}[定理\ref{4.4.3.11}の再掲]
$U \subseteq \mathbb{R}^{n}$なる開集合$U$を用いた$C^{2}$級の関数$f:U \rightarrow \mathbb{R}$の停留点$\mathbf{a}$が与えられたとき、次のことが成り立つ。
\begin{itemize}
\item
  2次形式$\left( d^{2}f \right)_{\mathbf{a}}$が正値であるなら、その点$\mathbf{a}$はその関数$f$の狭義の極小点である。
\item
  2次形式$\left( d^{2}f \right)_{\mathbf{a}}$が負値であるなら、その点$\mathbf{a}$はその関数$f$の狭義の極大点である。
\item
  2次形式$\left( d^{2}f \right)_{\mathbf{a}}$が不定符号であるなら、その点$\mathbf{a}$はその関数$f$の極小点でも極大点でもない。
\end{itemize}
\end{thm*}
\begin{thm*}[定理\ref{4.4.3.12}の再掲]
$U \subseteq \mathbb{R}^{n}$なる開集合$U$を用いた$C^{2}$級の関数$f:U \rightarrow \mathbb{R}$の停留点$\mathbf{a}$が与えられたとき、$k \in \varLambda_{n}$として、次式のように関数$D_{k}$が定義されれば、
\begin{align*}
D_{k}:U \rightarrow \mathbb{R};\mathbf{x} \mapsto \det{\left( P_{k} \circ H_{f} \right)\left( \mathbf{x} \right)}
\end{align*}
次のことが成り立つ。
\begin{itemize}
\item
  $\forall k \in \varLambda_{n}$に対し、$0 < D_{k}\left( \mathbf{a} \right)$が成り立つなら、その点$\mathbf{a}$はその関数$f$の狭義の極小点である。
\item
  $\forall k \in \varLambda_{n}$に対し、$0 < ( - 1)^{k}D_{k}\left( \mathbf{a} \right)$が成り立つなら、その点$\mathbf{a}$はその関数$f$の狭義の極大点である。
\item
  $D_{n}\left( \mathbf{a} \right) \neq 0$が成り立つかつ、$\exists k \in \varLambda_{n}$に対し、$0 \geq D_{k}\left( \mathbf{a} \right)$が成り立つかつ、$\exists k \in \varLambda_{n}$に対し、$0 \geq ( - 1)^{k}D_{k}\left( \mathbf{a} \right)$が成り立つなら、その点$\mathbf{a}$はその関数$f$の極小点でも極大点でもない。
\end{itemize}
\end{thm*}
\begin{thm*}[定理\ref{4.4.3.13}の再掲]
$U \subseteq \mathbb{R}^{n}$なる開集合$U$を用いた$C^{2}$級の関数$f:U \rightarrow \mathbb{R}$の正則停留点$\mathbf{a}$が与えられたとき、$k \in \varLambda_{n}$として、次式のように関数$D_{k}$が定義されれば、
\begin{align*}
D_{k}:U \rightarrow \mathbb{R};\mathbf{x} \mapsto \det{\left( P_{k} \circ H_{f} \right)\left( \mathbf{x} \right)}
\end{align*}
次のことが成り立つ。
\begin{itemize}
\item
  $\forall k \in \varLambda_{n}$に対し、$0 < D_{k}\left( \mathbf{a} \right)$が成り立つなら、その点$\mathbf{a}$はその関数$f$の狭義の極小点である。
\item
  $\forall k \in \varLambda_{n}$に対し、$0 < ( - 1)^{k}D_{k}\left( \mathbf{a} \right)$が成り立つなら、その点$\mathbf{a}$はその関数$f$の狭義の極大点である。
\item
  上のこといずれも満たさないなら、その点$\mathbf{a}$はその関数$f$の極小点でも極大点でもない。
\end{itemize}
\end{thm*}
\begin{thm*}[定理\ref{4.4.4.3}の再掲]
$U \subseteq \mathbb{R}^{m}$なる開集合$U$を用いた$C^{1}$級の関数たち$f:U \rightarrow \mathbb{R}$、$g:U \rightarrow \mathbb{R}^{n}$が与えられたとき、$S = \left\{ \mathbf{x} \in U \middle| g\left( \mathbf{x} \right) = \mathbf{0} \right\}$において、その関数$f$がその集合$S$上の極値$f\left( \mathbf{a} \right)$をとるなら、次のいづれかが成り立つ。
\begin{itemize}
\item
  次式のような関数$\varPhi$が用いられれば、
\begin{align*}
\varPhi:U \times \mathbb{R}^{n} \rightarrow \mathbb{R};\begin{pmatrix}
\mathbf{x} \\
\mathbf{y} \\
\end{pmatrix} \mapsto f\left( \mathbf{x} \right) -{}^t \mathbf{y}g\left( \mathbf{x} \right)
\end{align*}
$\exists\mathbf{l} \in \mathbb{R}^{n}$に対し、$\mathrm{grad}\varPhi\begin{pmatrix}
\mathbf{a} \\
\mathbf{l} \\
\end{pmatrix} = \mathbf{0}$が成り立つ。
\item
  $\mathrm{rank}{J_{g}\left( \mathbf{a} \right)} < n$が成り立つ。
\end{itemize}
\end{thm*}\par
そこで、補題として次のようなものを挙げておこう。
\begin{thm}\label{4.4.4.4}
$U \subseteq \mathbb{R}^{m}$なる開集合$U$を用いた$C^{1}$級の関数たち$f:U \rightarrow \mathbb{R}$、$g:U \rightarrow \mathbb{R}^{n}$が与えられており、さらに、よりよい陰関数定理での仮定が満たされており、ある近傍$V \times W \subseteq \mathbb{R}^{n} \times \mathbb{R}^{m - n}$とある関数$\varphi:V \rightarrow W$が存在して、次式が成り立つようにすることができるとして、
\begin{align*}
\mathbf{x} = \begin{pmatrix}
\mathbf{x}^{*} \\
\mathbf{x}_{*} \\
\end{pmatrix} \in S \Leftrightarrow g\left( \mathbf{x} \right) = \mathbf{0} \Leftrightarrow \mathbf{x}_{*} = \varphi\left( \mathbf{x}^{*} \right)
\end{align*}
次のようにおくと、
\begin{align*}
\nabla^{*} = \left( \partial_{i} \right)_{i \in \varLambda_{n}}&,\ \ \nabla_{*} = \left( \partial_{i} \right)_{i \in \varLambda_{m} \setminus \varLambda_{n}} \\
{}^t \partial_{j}\varphi\left( \nabla_{*}{}^t \nabla_{*} \otimes g \right)\partial_{i}\varphi &= \left({}^t \partial_{j}\varphi\left( \nabla_{*}{}^t \nabla_{*}g_{k} \right)\partial_{i}\varphi \right)_{k \in \varLambda_{n}} \\
\varphi_{\downarrow}:V &\rightarrow U;\mathbf{x}^{*} \mapsto \begin{pmatrix}
\mathbf{x}^{*} \\
\varphi\left( \mathbf{x}^{*} \right) \\
\end{pmatrix}
\end{align*}
次のことが成り立つ。
\begin{itemize}
\item
  次式が成り立つ。
\begin{align*}
J_{\varphi} ={}^t \left( \nabla^{*}{}^t \varphi \right) = - \left({}^t \left( \nabla_{*}{}^{t}g \right)^{- 1}{}^{t}\left( \nabla^{*}{}^t g \right) \right) \circ \varphi_{\downarrow}
\end{align*}
\item
  $\forall(i,j) \in \varLambda_{n}^{2}$に対し、次式が成り立つ。
\begin{align*}
\partial_{ji}\varphi &= - \left( J_{g} \circ \varphi_{\downarrow} \right)^{- 1}\left( \partial_{ji}g \circ \varphi_{\downarrow} + \left( \partial_{j}J_{g} \circ \varphi_{\downarrow} \right)\partial_{i}\varphi \right. \\
&\quad \left. + \left( \partial_{i}J_{g} \circ \varphi_{\downarrow} \right)\partial_{j}\varphi +{}^t \partial_{j}\varphi\left( \nabla_{*}{}^{t}\nabla_{*} \otimes g \right)\partial_{i}\varphi \right)
\end{align*}
\end{itemize}
さらに、$F = f \circ \varphi_{\downarrow}:V \rightarrow \mathbb{R}$と関数$F$が定義されるとき、次のようにおくと、
\begin{align*}
{}^t \left( \nabla_{*}f \circ \varphi_{\downarrow} \right)\left( \nabla^{*}{}^t \nabla^{*} \otimes \varphi \right) = \left({}^t \left( \nabla_{*}f \circ \varphi_{\downarrow} \right)\partial_{ji}\varphi \right)_{(i,j) \in \varLambda_{n}^{2}}
\end{align*}
次のことが成り立つ。
\begin{itemize}
\item
  $\forall i \in \varLambda_{n}$に対し、次式が成り立つ。
\begin{align*}
\partial_{i}F = \partial_{i}f \circ \varphi_{\downarrow} +{}^t \left( \nabla_{*}f \circ \varphi_{\downarrow} \right)\partial_{i}\varphi
\end{align*}
\item
  次式が成り立つ。
\begin{align*}
J_{F} ={}^t \left( \nabla^{*}{}^t F \right) = \nabla^{*}f \circ \varphi_{\downarrow} + \left( \nabla_{*}f \circ \varphi_{\downarrow} \right)\nabla^{*}\varphi
\end{align*}
\item
  $\forall(i,j) \in \varLambda_{n}^{2}$に対し、次式が成り立つ。
\begin{align*}
\partial_{ji}F &= \partial_{ji}f \circ \varphi_{\downarrow} +{}^t \left( \partial_{i}\nabla_{*}f \circ \varphi_{\downarrow} \right)\partial_{j}\varphi +{}^t \left( \nabla_{*}\partial_{j}f \circ \varphi_{\downarrow} \right)\partial_{i}\varphi \\
&\quad +{}^t \partial_{i}\varphi\left( \nabla_{*}{}^t \nabla_{*}f \circ \varphi_{\downarrow} \right)\partial_{j}\varphi +{}^t \left( \nabla_{*}f \circ \varphi_{\downarrow} \right)\partial_{ji}\varphi
\end{align*}
\item
  次式が成り立つ。
\begin{align*}
H_{F} &= \nabla^{*}{}^t \nabla^{*}f \circ \varphi_{\downarrow} + \left( \nabla^{*}{}^t \nabla_{*}f \circ \varphi_{\downarrow} \right)\nabla^{*}\varphi + \nabla^{*}\varphi\left( \nabla_{*}{}^t \nabla^{*}f \circ \varphi_{\downarrow} \right) \\
&\quad +{}^t \left( \nabla^{*}{}^t \varphi \right)\left( \nabla_{*}{}^t \nabla_{*}f \circ \varphi_{\downarrow} \right)\left( \nabla^{*}{}^t \varphi \right) +{}^t \left( \nabla_{*}f \circ \varphi_{\downarrow} \right)\left( \nabla^{*}{}^t \nabla^{*} \otimes \varphi \right)
\end{align*}
\end{itemize}
\end{thm}
\begin{proof}
$U \subseteq \mathbb{R}^{m}$なる開集合$U$を用いた$C^{1}$級関数たち$f:U \rightarrow \mathbb{R}$、$g:U \rightarrow \mathbb{R}^{n}$が与えられており、さらに、よりよい陰関数定理での仮定が満たされており、ある近傍$V \times W \subseteq \mathbb{R}^{n} \times \mathbb{R}^{m - n}$とある関数$\varphi = \left( \varphi_{i} \right)_{i \in \varLambda_{m} \setminus \varLambda_{n}}:V \rightarrow W$が存在して、次式が成り立つようにすることができるとする。
\begin{align*}
\mathbf{x} = \begin{pmatrix}
\mathbf{x}^{*} \\
\mathbf{x}_{*} \\
\end{pmatrix} \in S \Leftrightarrow g\left( \mathbf{x} \right) = \mathbf{0} \Leftrightarrow \mathbf{x}_{*} = \varphi\left( \mathbf{x}^{*} \right)
\end{align*}
以下ここでは、記法の煩雑さを避けるため、Einstein縮約記法を用い、即ち、同じ添数が2回現れたとき、その添数が属する添数集合だけ和をとることに約束し、添数集合$\varLambda_{n}$の添数として、$\alpha^{*}$、$\beta^{*}$、$\gamma^{*}$、添数集合$\varLambda_{m} \setminus \varLambda_{n}$の添数として、$\alpha$、$\beta$、$\gamma$、$\delta$を用いることにする。\par
さて、次のようにおくと、
\begin{align*}
\nabla^{*} = \left( \partial_{i} \right)_{i \in \varLambda_{n}}&,\ \ \nabla_{*} = \left( \partial_{i} \right)_{i \in \varLambda_{m} \setminus \varLambda_{n}} \\
\left({}^t \partial_{i}\varphi\nabla_{*}{}^t \nabla_{*}\partial_{j}\varphi \right) \otimes g &= \left({}^t \partial_{i}\varphi\left( \nabla_{*}{}^t \nabla_{*}g_{k} \right)\partial_{j}\varphi \right)_{k \in \varLambda_{n}} \\
\varphi_{\downarrow} = \left( \varphi_{i}^{\downarrow} \right)_{i \in \varLambda_{m}}:V &\rightarrow U;\mathbf{x}^{*} \mapsto \begin{pmatrix}
\mathbf{x}^{*} \\
\varphi\left( \mathbf{x}^{*} \right) \\
\end{pmatrix}
\end{align*}
その関数$\varphi$のJacobi行列はよりよい陰関数定理より次のようになる。
\begin{align*}
J_{\varphi} ={}^t \left( \nabla^{*}{}^t \varphi \right) = - \left({}^t \left( \nabla_{*}{}^{t}g \right)^{- 1}{}^{t}\left( \nabla^{*}{}^t g \right) \right) \circ \varphi_{\downarrow}
\end{align*}
実際、$g \circ \varphi_{\downarrow}:V \rightarrow \mathbb{R}^{n} = 0$より、${}^t \left( \nabla_{*}{}^t g \right)^{- 1} = \left( \left( \partial_{j}g_{i} \right)^{- 1} \right)_{(i,j) \in \left( \varLambda_{m} \setminus \varLambda_{n} \right)^{2}}$とおかれれば、$\forall(i,j) \in \left( \varLambda_{m} \setminus \varLambda_{n} \right) \times \varLambda_{n}$に対し、次のようになることから従う\footnote{次のようにしても示すことができる。
\begin{align*}
0 &= J_{g \circ \varphi_{\downarrow}} = \left( J_{g} \circ \varphi_{\downarrow} \right)J_{\varphi_{\downarrow}} \\
&= \begin{pmatrix}
  {}^t \left( \nabla^{*}{}^t g \right) \circ \varphi_{\downarrow} &{}^{t}\left( \nabla_{*}{}^{t}g \right) \circ \varphi_{\downarrow} \\
\end{pmatrix}\begin{pmatrix}
  I_{n} \\
  J_{\varphi} \\
\end{pmatrix}\\ 
&={}^t \left( \nabla^{*}{}^t g \right) \circ \varphi_{\downarrow} +{}^t \left( \nabla_{*}{}^{t}g \right) \circ \varphi_{\downarrow}J_{\varphi}
\end{align*}}。
\begin{align*}
\partial_{j}\varphi_{i} &= \delta_{i\alpha}\partial_{j}\varphi_{\alpha}\\
&= \left( \left( \partial_{\alpha}g_{i} \right)^{- 1}\partial_{\beta}g_{\alpha} \right) \circ \varphi_{\downarrow}\partial_{j}\varphi_{\beta}\\
&= \left( \partial_{\alpha}g_{i} \right)^{- 1} \circ \varphi_{\downarrow}\left( \partial_{j}g_{\alpha} \circ \varphi_{\downarrow} + \left( \partial_{\beta}g_{\alpha} \circ \varphi_{\downarrow} \right)\partial_{j}\varphi_{\beta} - \partial_{j}g_{\alpha} \circ \varphi_{\downarrow} \right)\\
&= \left( \partial_{\alpha}g_{i} \right)^{- 1} \circ \varphi_{\downarrow}\left( \left( \partial_{\beta^{*}}g_{\alpha} \circ \varphi_{\downarrow} \right)\delta_{\beta^{*}j} + \left( \partial_{\beta}g_{\alpha} \circ \varphi_{\downarrow} \right)\partial_{j}\varphi_{\beta} - \partial_{j}g_{\alpha} \circ \varphi_{\downarrow} \right)\\
&= \left( \partial_{\alpha}g_{i} \right)^{- 1} \circ \varphi_{\downarrow}\left( \left( \partial_{\beta^{*}}g_{\alpha} \circ \varphi_{\downarrow} \right)\partial_{j}\varphi_{\beta^{*}} + \left( \partial_{\beta}g_{\alpha} \circ \varphi_{\downarrow} \right)\partial_{j}\varphi_{\beta} - \partial_{j}g_{\alpha} \circ \varphi_{\downarrow} \right)\\
&= \left( \partial_{\alpha}g_{i} \right)^{- 1} \circ \varphi_{\downarrow}\left( \partial_{j}\left( g_{\alpha} \circ \varphi_{\downarrow} \right) - \partial_{j}g_{\alpha} \circ \varphi_{\downarrow} \right)\\
&= \left( \partial_{\alpha}g_{i} \right)^{- 1} \circ \varphi_{\downarrow}\left( - \partial_{j}g_{\alpha} \circ \varphi_{\downarrow} \right)\\
&= - \left( \left( \partial_{\alpha}g_{i} \right)^{- 1} \circ \varphi_{\downarrow} \right)\left( \partial_{j}g_{\alpha} \circ \varphi_{\downarrow} \right)
\end{align*}\par
一方で、$g \circ \varphi_{\downarrow}:V \rightarrow \mathbb{R}^{n} = 0$が成り立つので、$\forall(i,j) \in \varLambda_{n}^{2}$に対し、第$k$成分でみれば次のようになる。
\begin{align*}
0 &= \partial_{ji}\left( g_{k} \circ \varphi_{\downarrow} \right)\\
&= \left( \partial_{\beta^{*}\alpha^{*}}g_{k} \circ \varphi_{\downarrow} \right)\partial_{i}\varphi_{\alpha^{*}}^{\downarrow}\partial_{j}\varphi_{\beta^{*}}^{\downarrow} + \left( \partial_{\beta\alpha^{*}}g_{k} \circ \varphi_{\downarrow} \right)\partial_{i}\varphi_{\alpha^{*}}^{\downarrow}\partial_{j}\varphi_{\beta}^{\downarrow} \\
&\quad + \left( \partial_{\beta^{*}\alpha}g_{k} \circ \varphi_{\downarrow} \right)\partial_{i}\varphi_{\alpha}^{\downarrow}\partial_{j}\varphi_{\beta^{*}}^{\downarrow} + \left( \partial_{\beta\alpha}g_{k} \circ \varphi_{\downarrow} \right)\partial_{i}\varphi_{\alpha}^{\downarrow}\partial_{j}\varphi_{\beta}^{\downarrow} \\
&\quad + \left( \partial_{\gamma^{*}}g_{k} \circ \varphi_{\downarrow} \right)\partial_{ji}\varphi_{\gamma^{*}}^{\downarrow} + \left( \partial_{\gamma}g_{k} \circ \varphi_{\downarrow} \right)\partial_{ji}\varphi_{\gamma}^{\downarrow}\\
&= \left( \partial_{\beta^{*}\alpha^{*}}g_{k} \circ \varphi_{\downarrow} \right)\delta_{\alpha^{*}i}\delta_{\beta^{*}j} + \left( \partial_{\beta\alpha^{*}}g_{k} \circ \varphi_{\downarrow} \right)\delta_{\alpha^{*}i}\partial_{j}\varphi_{\beta} \\
&\quad + \left( \partial_{\beta^{*}\alpha}g_{k} \circ \varphi_{\downarrow} \right)\partial_{i}\varphi_{\alpha}\delta_{\beta^{*}j} + \left( \partial_{\beta\alpha}g_{k} \circ \varphi_{\downarrow} \right)\partial_{i}\varphi_{\alpha}\partial_{j}\varphi_{\beta} \\
&\quad + \left( \partial_{\gamma}g_{k} \circ \varphi_{\downarrow} \right)\partial_{ji}\varphi_{\gamma}\\
&= \partial_{ji}g_{k} \circ \varphi_{\downarrow} + \left( \partial_{\beta i}g_{k} \circ \varphi_{\downarrow} \right)\partial_{j}\varphi_{\beta} + \left( \partial_{j\alpha}g_{k} \circ \varphi_{\downarrow} \right)\partial_{i}\varphi_{\alpha} \\
&\quad + \left( \partial_{\beta\alpha}g_{k} \circ \varphi_{\downarrow} \right)\partial_{i}\varphi_{\alpha}\partial_{j}\varphi_{\beta} + \left( \partial_{\gamma}g_{k} \circ \varphi_{\downarrow} \right)\partial_{ji}\varphi_{\gamma}
\end{align*}
これにより、次のようになる。
\begin{align*}
\partial_{ji}\varphi_{k} &= \delta_{k\gamma}\partial_{ji}\varphi_{\gamma}\\
&= \left( \left( \partial_{\delta}g_{k} \right)^{- 1}\partial_{\gamma}g_{\delta} \right) \circ \varphi_{\downarrow}\partial_{ji}\varphi_{\gamma}\\
&= \left( \left( \partial_{\delta}g_{k} \right)^{- 1} \circ \varphi_{\downarrow} \right)\left( \partial_{\gamma}g_{\delta} \circ \varphi_{\downarrow} \right)\partial_{ji}\varphi_{\gamma}\\
&= \left( \partial_{\delta}g_{k} \right)^{- 1} \circ \varphi_{\downarrow}\left( \partial_{ji}g_{\delta} \circ \varphi_{\downarrow} + \left( \partial_{\beta i}g_{\delta} \circ \varphi_{\downarrow} \right)\partial_{j}\varphi_{\beta} \right. \\
&\quad + \left( \partial_{j\alpha}g_{\delta} \circ \varphi_{\downarrow} \right)\partial_{i}\varphi_{\alpha} + \left( \partial_{\beta\alpha}g_{\delta} \circ \varphi_{\downarrow} \right)\partial_{i}\varphi_{\alpha}\partial_{j}\varphi_{\beta} + \left( \partial_{\gamma}g_{\delta} \circ \varphi_{\downarrow} \right)\partial_{ji}\varphi_{\gamma} \\
&\quad - \partial_{ji}g_{\delta} \circ \varphi_{\downarrow} - \left( \partial_{\beta i}g_{\delta} \circ \varphi_{\downarrow} \right)\partial_{j}\varphi_{\beta} \\
&\quad \left. - \left( \partial_{j\alpha}g_{\delta} \circ \varphi_{\downarrow} \right)\partial_{i}\varphi_{\alpha} - \left( \partial_{\beta\alpha}g_{\delta} \circ \varphi_{\downarrow} \right)\partial_{i}\varphi_{\alpha}\partial_{j}\varphi_{\beta} \right)\\
&= \left( \partial_{\delta}g_{k} \right)^{- 1} \circ \varphi_{\downarrow}\left( - \partial_{ji}g_{\delta} \circ \varphi_{\downarrow} - \left( \partial_{\beta i}g_{\delta} \circ \varphi_{\downarrow} \right)\partial_{j}\varphi_{\beta} \right. \\
&\quad \left. - \left( \partial_{j\alpha}g_{\delta} \circ \varphi_{\downarrow} \right)\partial_{i}\varphi_{\alpha} - \left( \partial_{\beta\alpha}g_{\delta} \circ \varphi_{\downarrow} \right)\partial_{i}\varphi_{\alpha}\partial_{j}\varphi_{\beta} \right)\\
&= - \left( \partial_{\delta}g_{k} \right)^{- 1} \circ \varphi_{\downarrow}\left( \partial_{ji}g_{\delta} \circ \varphi_{\downarrow} + \left( \partial_{\beta i}g_{\delta} \circ \varphi_{\downarrow} \right)\partial_{j}\varphi_{\beta} \right. \\
&\quad \left. + \left( \partial_{j\alpha}g_{\delta} \circ \varphi_{\downarrow} \right)\partial_{i}\varphi_{\alpha} + \left( \partial_{\beta\alpha}g_{\delta} \circ \varphi_{\downarrow} \right)\partial_{i}\varphi_{\alpha}\partial_{j}\varphi_{\beta} \right)\\
&= - \left( \partial_{\delta}g_{k} \right)^{- 1} \circ \varphi_{\downarrow}\left( \partial_{ji}g_{\delta} \circ \varphi_{\downarrow} + \left( \partial_{\beta i}g_{\delta} \circ \varphi_{\downarrow} \right)\partial_{j}\varphi_{\beta} \right. \\
&\quad \left. + \left( \partial_{j\alpha}g_{\delta} \circ \varphi_{\downarrow} \right)\partial_{i}\varphi_{\alpha} + \left( \partial_{\beta\alpha}g_{\delta} \circ \varphi_{\downarrow} \right)\partial_{i}\varphi_{\alpha}\partial_{j}\varphi_{\beta} \right)
\end{align*}
これにより、次式が成り立つ。
\begin{align*}
\partial_{ji}\varphi &= - \left( J_{g} \circ \varphi_{\downarrow} \right)^{- 1}\left( \partial_{ji}g \circ \varphi_{\downarrow} + \left( \partial_{j}J_{g} \circ \varphi_{\downarrow} \right)\partial_{i}\varphi \right. \\
&\quad \left. + \left( \partial_{i}J_{g} \circ \varphi_{\downarrow} \right)\partial_{j}\varphi + \left({}^t \partial_{j}\varphi\nabla_{*}{}^{t}\nabla_{*}\partial_{i}\varphi \right) \otimes g \right)
\end{align*}\par
さらに、$F = f \circ \varphi_{\downarrow}:V \rightarrow \mathbb{R}$と関数$F$が定義されるとき、$\forall i \in \varLambda_{n}$に対し、次のようになる。
\begin{align*}
\partial_{i}F &= \partial_{i}\left( f \circ \varphi_{\downarrow} \right)\\
&= \left( \partial_{\alpha^{*}}f \circ \varphi_{\downarrow} \right)\partial_{i}\varphi_{\alpha^{*}}^{\downarrow} + \left( \partial_{\alpha}f \circ \varphi_{\downarrow} \right)\partial_{i}\varphi_{\alpha}^{\downarrow}\\
&= \left( \partial_{\alpha^{*}}f \circ \varphi_{\downarrow} \right)\delta_{\alpha^{*}i} + \left( \partial_{\alpha}f \circ \varphi_{\downarrow} \right)\partial_{i}\varphi_{\alpha}\\
&= \partial_{i}f \circ \varphi_{\downarrow} + \left( \partial_{\alpha}f \circ \varphi_{\downarrow} \right)\partial_{i}\varphi_{\alpha}\\
&= \partial_{i}f \circ \varphi_{\downarrow} +{}^t \left( \nabla_{*}f \circ \varphi_{\downarrow} \right)\partial_{i}\varphi
\end{align*}
したがって、次式が成り立つ。
\begin{align*}
J_{F} ={}^t \left( \nabla^{*}{}^t F \right) = \nabla^{*}f \circ \varphi_{\downarrow} + \left( \nabla_{*}f \circ \varphi_{\downarrow} \right)\nabla^{*}\varphi
\end{align*}\par
次のようにおくと、
\begin{align*}
{}^t \left( \nabla_{*}f \circ \varphi_{\downarrow} \right)\left( \nabla^{*}{}^t \nabla^{*} \otimes \varphi \right) = \left({}^t \left( \nabla_{*}f \circ \varphi_{\downarrow} \right)\partial_{ji}\varphi \right)_{(i,j) \in \varLambda_{n}^{2}}
\end{align*}
$\forall(i,j) \in \varLambda_{n}^{2}$に対し、次のようになる。
\begin{align*}
\partial_{ji}F &= \partial_{ji}\left( f \circ \varphi_{\downarrow} \right)\\
&= \left( \partial_{\beta^{*}\alpha^{*}}f \circ \varphi_{\downarrow} \right)\partial_{i}\varphi_{\alpha^{*}}^{\downarrow}\partial_{j}\varphi_{\beta^{*}}^{\downarrow} + \left( \partial_{\beta\alpha^{*}}f \circ \varphi_{\downarrow} \right)\partial_{i}\varphi_{\alpha^{*}}^{\downarrow}\partial_{j}\varphi_{\beta}^{\downarrow} \\
&\quad + \left( \partial_{\beta^{*}\alpha}f \circ \varphi_{\downarrow} \right)\partial_{i}\varphi_{\alpha}^{\downarrow}\partial_{j}\varphi_{\beta^{*}}^{\downarrow} + \left( \partial_{\beta\alpha}f \circ \varphi_{\downarrow} \right)\partial_{i}\varphi_{\alpha}^{\downarrow}\partial_{j}\varphi_{\beta}^{\downarrow} \\
&\quad + \left( \partial_{\gamma^{*}}f \circ \varphi_{\downarrow} \right)\partial_{ji}\varphi_{\gamma^{*}}^{\downarrow} + \left( \partial_{\gamma}f \circ \varphi_{\downarrow} \right)\partial_{ji}\varphi_{\gamma}^{\downarrow}\\
&= \left( \partial_{\beta^{*}\alpha^{*}}f \circ \varphi_{\downarrow} \right)\delta_{\alpha^{*}i}\delta_{\beta^{*}j} + \left( \partial_{\beta\alpha^{*}}f \circ \varphi_{\downarrow} \right)\delta_{\alpha^{*}i}\partial_{j}\varphi_{\beta} \\
&\quad + \left( \partial_{\beta^{*}\alpha}f \circ \varphi_{\downarrow} \right)\partial_{i}\varphi_{\alpha}\delta_{\beta^{*}j} + \left( \partial_{\beta\alpha}f \circ \varphi_{\downarrow} \right)\partial_{i}\varphi_{\alpha}\partial_{j}\varphi_{\beta} \\
&\quad + \left( \partial_{\gamma}f \circ \varphi_{\downarrow} \right)\partial_{ji}\varphi_{\gamma}\\
&= \partial_{ji}f \circ \varphi_{\downarrow} + \left( \partial_{\beta i}f \circ \varphi_{\downarrow} \right)\partial_{j}\varphi_{\beta} + \left( \partial_{j\alpha}f \circ \varphi_{\downarrow} \right)\partial_{i}\varphi_{\alpha} \\
&\quad + \left( \partial_{\beta\alpha}f \circ \varphi_{\downarrow} \right)\partial_{i}\varphi_{\alpha}\partial_{j}\varphi_{\beta} + \left( \partial_{\gamma}f \circ \varphi_{\downarrow} \right)\partial_{ji}\varphi_{\gamma}\\
&= \partial_{ji}f \circ \varphi_{\downarrow} + \left( \partial_{i}\partial_{\beta}f \circ \varphi_{\downarrow} \right)\partial_{j}\varphi_{\beta} + \left( \partial_{\alpha}\partial_{j}f \circ \varphi_{\downarrow} \right)\partial_{i}\varphi_{\alpha} \\
&\quad + \partial_{i}\varphi_{\alpha}\left( \partial_{\beta\alpha}f \circ \varphi_{\downarrow} \right)\partial_{j}\varphi_{\beta} + \left( \partial_{\gamma}f \circ \varphi_{\downarrow} \right)\partial_{ji}\varphi_{\gamma}\\
&= \partial_{ji}f \circ \varphi_{\downarrow} +{}^t \left( \partial_{i}\nabla_{*}f \circ \varphi_{\downarrow} \right)\partial_{j}\varphi +{}^t \left( \nabla_{*}\partial_{j}f \circ \varphi_{\downarrow} \right)\partial_{i}\varphi \\
&\quad +{}^t \partial_{i}\varphi\left( \nabla_{*}{}^t \nabla_{*}f \circ \varphi_{\downarrow} \right)\partial_{j}\varphi +{}^t \left( \nabla_{*}f \circ \varphi_{\downarrow} \right)\partial_{ji}\varphi
\end{align*}
したがって、次式が成り立つ。
\begin{align*}
H_{F} &= \nabla^{*}{}^t \nabla^{*}f \circ \varphi_{\downarrow} + \left( \nabla^{*}{}^t \nabla_{*}f \circ \varphi_{\downarrow} \right)\nabla^{*}\varphi + \nabla^{*}\varphi\left( \nabla_{*}{}^t \nabla^{*}f \circ \varphi_{\downarrow} \right) \\
&\quad +{}^t \left( \nabla^{*}{}^t \varphi \right)\left( \nabla_{*}{}^t \nabla_{*}f \circ \varphi_{\downarrow} \right)\left( \nabla^{*}{}^t \varphi \right) +{}^t \left( \nabla_{*}f \circ \varphi_{\downarrow} \right)\left( \nabla^{*}{}^t \nabla^{*} \otimes \varphi \right)
\end{align*}
\end{proof}\par
この定理から、$U \subseteq \mathbb{R}^{n}$なる開集合$U$を用いた$C^{2}$級関数$f:U \rightarrow \mathbb{R}$が与えられたとき、その関数$f$の極値を求めてみよう。これは次の手順で求められる。
\begin{enumerate}
\item
  $\forall i,j \in \varLambda_{n}$に対し、偏導関数たち$\partial_{i}f$、$\partial_{ji}f$を求めておく。
\item
  $\mathrm{grad}f\left( \mathbf{a} \right) = \mathbf{0}$なる点$\mathbf{a}$を1.
  を用いて求める。これが停留点となる。
\item
  $k \in \varLambda_{n}$として、次式のように関数$D_{k}$を定義してこれを1.
  を用いて求めておく。
\begin{align*}
D_{k}:U \rightarrow \mathbb{R};\mathbf{x} \mapsto \det\left( \partial_{ji}f\left( \mathbf{x} \right) \right)_{(i,j) \in \varLambda_{k}^{2}}
\end{align*}
\item
  上で求めた停留点のうち、$\forall k \in \varLambda_{n}$に対し、$0 < D_{k}\left( \mathbf{a} \right)$が成り立つなら、その点$\mathbf{a}$はその関数$f$の狭義の極小点であることが定理\ref{4.4.3.12}より分かる。
\item
  上で求めた停留点のうち、$\forall k \in \varLambda_{n}$に対し、$0 < ( - 1)^{k}D_{k}\left( \mathbf{a} \right)$が成り立つなら、その点$\mathbf{a}$はその関数$f$の狭義の極大点であることが定理\ref{4.4.3.12}より分かる。
\item
  $D_{n}\left( \mathbf{a} \right) \neq 0$が成り立つとき、4. 、5. いづれも満たさなければ、定理\ref{4.4.3.13}よりその点$\mathbf{a}$はその関数$f$の極小点でも極大点でもないことが分かる。
\item
  $D_{n}\left( \mathbf{a} \right) = 0$が成り立つとき、2次形式$\left( d^{2}f \right)_{\mathbf{a}}$が不定符号であるなら、定理\ref{4.4.3.11}よりその点$\mathbf{a}$はその関数$f$の極小点でも極大点でもないことが分かる。
\item
  $D_{n}\left( \mathbf{a} \right) = 0$が成り立つとき、2次形式$\left( d^{2}f \right)_{\mathbf{a}}$が不定符号でないなら、その点$\mathbf{a}$はその関数$f$の極値をとる点なのかどうかは今までの議論で判断できないことが分かる。
\end{enumerate}\par
この定理から、$U \subseteq \mathbb{R}^{m}$なる開集合$U$を用いた$C^{1}$級関数たち$f:U \rightarrow \mathbb{R}$、$g:U \rightarrow \mathbb{R}^{n}$が与えられたとき、$g\left( \mathbf{x} \right) = \mathbf{0}$の条件の下でその関数$f$の極値を求めてみよう。これは次の手順で求められる。
\begin{enumerate}
\item
  $\forall i,j \in \varLambda_{m}$に対し、偏導関数たち$\partial_{i}f$、$\partial_{ji}f$、$\partial_{i}g$、$\partial_{ji}g$を求めておく。
\item
  $\mathrm{rank}{J_{g}\left( \mathbf{a} \right)} < n$が成り立つような点$\mathbf{a}$を求める。
\item
  上で求めた点$\mathbf{a}$のうち$g\left( \mathbf{a} \right) = \mathbf{0}$が成り立つようなものが極値をとる点でありうる。
\item
  次式のような関数$\varPhi$を用いて
\begin{align*}
\varPhi:U \times \mathbb{R}^{n} \rightarrow \mathbb{R};\begin{pmatrix}
\mathbf{x} \\
\mathbf{y} \\
\end{pmatrix} \mapsto f\left( \mathbf{x} \right) -{}^t \mathbf{y}g\left( \mathbf{x} \right)
\end{align*}
$\exists\mathbf{l} \in \mathbb{R}^{n}$に対し、$\mathrm{grad}\varPhi\begin{pmatrix}
\mathbf{a} \\
\mathbf{l} \\
\end{pmatrix} = \mathbf{0}$が成り立つような点$\mathbf{a}$を求める。
\item
  上で求めた点$\mathbf{a}$のうち$g\left( \mathbf{a} \right) = \mathbf{0}$が成り立つようなものが極値をとる点でありうる。
\item
  3. 、5. よりこれで極値をとりうる点がすべて求まった。
\item
  陰関数定理より次式が成り立つようにすることができるので、
\begin{align*}
\mathbf{x} = \begin{pmatrix}
\mathbf{x}^{*} \\
\mathbf{x}_{*} \\
\end{pmatrix} \in S \Leftrightarrow g\left( \mathbf{x} \right) = \mathbf{0} \Leftrightarrow \mathbf{x}_{*} = \varphi\left( \mathbf{x}^{*} \right)
\end{align*}
次式のように関数$F$が定義されると、
\begin{align*}
F:V \rightarrow \mathbb{R};\mathbf{x}^{*} \mapsto f\begin{pmatrix}
\mathbf{x}^{*} \\
\varphi\left( \mathbf{x}^{*} \right) \\
\end{pmatrix}
\end{align*}
$\mathbf{x} = \mathbf{a} = \begin{pmatrix}
\mathbf{a}^{*} \\
\mathbf{a}_{*} \\
\end{pmatrix}$のときその関数$f$はその集合$S$上の極値をとるので、$\mathbf{x}^{*} = \mathbf{a}^{*}$のとき、その関数$F$は極値をとることに注意する。
\item
  定理\ref{4.4.4.4}より6. における極値の候補となっている点$\mathbf{a}$でのその陰関数$\varphi$のJacobi行列$J_{\varphi}$とHesse行列$H_{\varphi}$の値を求める。
\item
  定理\ref{4.4.4.4}より6. における極値の候補となっている点$\mathbf{a}$でのその関数$F$のJacobi行列$J_{F}$とHesse行列$H_{F}$の値を8. を用いて求める。
\item
  あとは関数$F$の極値を前述した手順で求める。
\end{enumerate}\par
まず、簡単な例から考えよう。関数$f\begin{pmatrix}
x \\
y \\
\end{pmatrix} = x^{3} + y^{3} + 3xy + 2$の極値を求めてみよう。次のようになることから\footnote{1. の「$\forall i,j \in \varLambda_{n}$に対し、偏導関数たち$\partial_{i}f$、$\partial_{ji}f$を求めておく。」にあたります。}、
\begin{align*}
\frac{\partial f}{\partial x}\begin{pmatrix}
x \\
y \\
\end{pmatrix} = 3x^{2} + 3y,\ \ \frac{\partial f}{\partial y}\begin{pmatrix}
x \\
y \\
\end{pmatrix} = 3y^{2} + 3x,
\end{align*}
\begin{align*}
\frac{\partial^{2}f}{{\partial x}^{2}}\begin{pmatrix}
x \\
y \\
\end{pmatrix} = 6x,\ \ \frac{\partial^{2}f}{\partial x\partial y}\begin{pmatrix}
x \\
y \\
\end{pmatrix} = 3,\ \ \frac{\partial^{2}f}{{\partial y}^{2}}\begin{pmatrix}
x \\
y \\
\end{pmatrix} = 6y
\end{align*}
$\mathrm{grad}f\begin{pmatrix}
x \\
y \\
\end{pmatrix} = \begin{pmatrix}
3x^{2} + 3y \\
3y^{2} + 3x \\
\end{pmatrix}$より$x^{2} + y = y^{2} + x = 0$なる点$\begin{pmatrix}
x \\
y \\
\end{pmatrix}$が停留点である。なお、偏微分の順序の入れ替えができることに注意されたい。これを求めると、点々$\begin{pmatrix}
 - 1 \\
 - 1 \\
\end{pmatrix}$、$\begin{pmatrix}
0 \\
0 \\
\end{pmatrix}$が得られる\footnote{2. の「$\mathrm{grad}f\left( \mathbf{a} \right) = \mathbf{0}$なる点$\mathbf{a}$を1. を用いて求める。これが停留点となる。」にあたります。}。このとき、次のようになることから\footnote{3. の「$k \in \varLambda_{n}$として、次式のように関数$D_{k}$を定義してこれを1. を用いて求めておく。」にあたります。
\begin{align*}
D_{k}:U \rightarrow \mathbb{R};\mathbf{x} \mapsto \det\left( \partial_{ji}f\left( \mathbf{x} \right) \right)_{(i,j) \in \varLambda_{k}^{2}}
\end{align*}}、
\begin{align*}
D_{1}\begin{pmatrix}
x \\
y \\
\end{pmatrix} = \left| \frac{\partial^{2}f}{{\partial x}^{2}}\begin{pmatrix}
x \\
y \\
\end{pmatrix} \right| = 6x,
\end{align*}
\begin{align*}
D_{2}\begin{pmatrix}
x \\
y \\
\end{pmatrix} = \left| \begin{matrix}
\frac{\partial^{2}f}{{\partial x}^{2}}\begin{pmatrix}
x \\
y \\
\end{pmatrix} & \frac{\partial^{2}f}{\partial x\partial y}\begin{pmatrix}
x \\
y \\
\end{pmatrix} \\
\frac{\partial^{2}f}{\partial y\partial x}\begin{pmatrix}
x \\
y \\
\end{pmatrix} & \frac{\partial^{2}f}{{\partial y}^{2}}\begin{pmatrix}
x \\
y \\
\end{pmatrix} \\
\end{matrix} \right| = \left| \begin{matrix}
6x & 3 \\
3 & 6y \\
\end{matrix} \right| = 9\left| \begin{matrix}
2x & 1 \\
1 & 2y \\
\end{matrix} \right| = 9(4xy - 1)
\end{align*}
次のようになる。
\begin{align*}
D_{1}\begin{pmatrix}
0 \\
0 \\
\end{pmatrix} = 0,\ \ D_{2}\begin{pmatrix}
0 \\
0 \\
\end{pmatrix} = - 9,\ \ D_{1}\begin{pmatrix}
 - 1 \\
 - 1 \\
\end{pmatrix} = - 6,\ \ D_{2}\begin{pmatrix}
 - 1 \\
 - 1 \\
\end{pmatrix} = 27
\end{align*}
ゆえに、$D_{2}\begin{pmatrix}
 - 1 \\
 - 1 \\
\end{pmatrix} = 27 > 0$、$- D_{1}\begin{pmatrix}
 - 1 \\
 - 1 \\
\end{pmatrix} = 6 > 0$が成り立つので、点$\begin{pmatrix}
 - 1 \\
 - 1 \\
\end{pmatrix}$でその関数$f$は極大値をとる\footnote{5. の「上で求めた停留点のうち、$\forall k \in \varLambda_{n}$に対し、$0 < ( - 1)^{k}D_{k}\left( \mathbf{a} \right)$が成り立つなら、その点$\mathbf{a}$はその関数$f$の狭義の極大点であることが定理\ref{4.4.3.12}より分かる。」にあたります。}。一方で、$D_{2}\begin{pmatrix}
0 \\
0 \\
\end{pmatrix} \neq 0$で$D_{1}\begin{pmatrix}
0 \\
0 \\
\end{pmatrix} = 0$が成り立つので、点$\begin{pmatrix}
0 \\
0 \\
\end{pmatrix}$でその関数$f$は極値をとらない\footnote{6. の「$D_{n}\left( \mathbf{a} \right) \neq 0$が成り立つとき、4. 、5. いづれも満たさなければ、定理\ref{4.4.3.13}よりその点$\mathbf{a}$はその関数$f$の極小点でも極大点でもないことが分かる。」にあたります。}。\par
次に少し難しい例で考えよう。$x^{2} + y^{2} = 1$が成り立つとき、関数$f\begin{pmatrix}
x \\
y \\
\end{pmatrix} = 2xy$の極値を求めてみよう。$g\begin{pmatrix}
x \\
y \\
\end{pmatrix} = x^{2} + y^{2} - 1$とおくと、次のようになる\footnote{1. の「$\forall i,j \in \varLambda_{n}$に対し、偏導関数たち$\partial_{i}f$、$\partial_{ji}f$、$\partial_{i}g$、$\partial_{ji}g$を求めておく。」にあたります。}。なお、偏微分の順序の入れ替えができることに注意されたい。
\begin{align*}
\frac{\partial f}{\partial x}\begin{pmatrix}
x \\
y \\
\end{pmatrix} = 2y&,\ \ \frac{\partial f}{\partial y}\begin{pmatrix}
x \\
y \\
\end{pmatrix} = 2x,\\
\frac{\partial^{2}f}{\partial x^{2}}\begin{pmatrix}
x \\
y \\
\end{pmatrix} = 0,\ \ \frac{\partial^{2}f}{\partial x\partial y}\begin{pmatrix}
x \\
y \\
\end{pmatrix} &= 2,\ \ \frac{\partial^{2}f}{\partial y^{2}}\begin{pmatrix}
x \\
y \\
\end{pmatrix} = 0,\\
\frac{\partial g}{\partial x}\begin{pmatrix}
x \\
y \\
\end{pmatrix} = 2x&,\ \ \frac{\partial g}{\partial y}\begin{pmatrix}
x \\
y \\
\end{pmatrix} = 2y,\\
\frac{\partial^{2}g}{\partial x^{2}}\begin{pmatrix}
x \\
y \\
\end{pmatrix} = 2,\ \ \frac{\partial^{2}g}{\partial x\partial y}\begin{pmatrix}
x \\
y \\
\end{pmatrix} &= 0,\ \ \frac{\partial^{2}g}{\partial y^{2}}\begin{pmatrix}
x \\
y \\
\end{pmatrix} = 2
\end{align*}
$\mathrm{rank}{J_{g}\begin{pmatrix}
x \\
y \\
\end{pmatrix}} < 1$なる点を求めると、次のようになることから、
\begin{align*}
J_{g}\begin{pmatrix}
x \\
y \\
\end{pmatrix} = \begin{pmatrix}
\frac{\partial g}{\partial x}\begin{pmatrix}
x \\
y \\
\end{pmatrix} & \frac{\partial g}{\partial y}\begin{pmatrix}
x \\
y \\
\end{pmatrix} \\
\end{pmatrix} = \begin{pmatrix}
2x & 2y \\
\end{pmatrix}
\end{align*}
点$\begin{pmatrix}
0 \\
0 \\
\end{pmatrix}$が得られる\footnote{2. の「$\mathrm{rank}{J_{g}\left( \mathbf{a} \right)} < n$が成り立つような点$\mathbf{a}$を求める。」にあたります。}。しかしながら、$g\begin{pmatrix}
0 \\
0 \\
\end{pmatrix} \neq 0$となっているので、この点$\begin{pmatrix}
0 \\
0 \\
\end{pmatrix}$は極値をとりうる点ではない\footnote{3. の「上で求めた点$\mathbf{a}$のうち$g\left( \mathbf{a} \right) = \mathbf{0}$が成り立つようなものが極値をとる点でありうる。」にあたります。}。以下、$\begin{pmatrix}
x \\
y \\
\end{pmatrix} \neq \begin{pmatrix}
0 \\
0 \\
\end{pmatrix}$とおきLagrangeの未定乗数法より$\exists\lambda \in \mathbb{R}$に対し、次を満たす点を求めていくと、
\begin{align*}
x^{2} + y^{2} = 1,\ \ \frac{\partial f}{\partial x}\begin{pmatrix}
x \\
y \\
\end{pmatrix} - \lambda\frac{\partial g}{\partial x}\begin{pmatrix}
x \\
y \\
\end{pmatrix} = 0,\ \ \frac{\partial f}{\partial y}\begin{pmatrix}
x \\
y \\
\end{pmatrix} - \lambda\frac{\partial g}{\partial y}\begin{pmatrix}
x \\
y \\
\end{pmatrix} = 0
\end{align*}
実数$\lambda$を消去して点々$\begin{pmatrix}
 - {1}/{\sqrt{2}} \\
 - {1}/{\sqrt{2}} \\
\end{pmatrix}$、$\begin{pmatrix}
 - {1}/{\sqrt{2}} \\
{1}/{\sqrt{2}} \\
\end{pmatrix}$、$\begin{pmatrix}
{1}/{\sqrt{2}} \\
 - {1}/{\sqrt{2}} \\
\end{pmatrix}$、$\begin{pmatrix}
{1}/{\sqrt{2}} \\
{1}/{\sqrt{2}} \\
\end{pmatrix}$が得られる\footnote{4. の「次式のような関数$\varPhi$を用いて
\begin{align*}
\varPhi:U \times \mathbb{R}^{n} \rightarrow \mathbb{R};\begin{pmatrix}
  \mathbf{x} \\
  \mathbf{y} \\
\end{pmatrix} \mapsto f\left( \mathbf{x} \right) -{}^t \mathbf{y}g\left( \mathbf{x} \right)
\end{align*}
  $\exists\mathbf{l} \in \mathbb{R}^{n}$に対し、$\mathrm{grad}\varPhi\begin{pmatrix}
  \mathbf{a} \\
  \mathbf{l} \\
  \end{pmatrix} = \mathbf{0}$が成り立つような点$\mathbf{a}$を求める。」にあたります。}。さらに、$g\begin{pmatrix}
 - {1}/{\sqrt{2}} \\
 - {1}/{\sqrt{2}} \\
\end{pmatrix} = g\begin{pmatrix}
 - {1}/{\sqrt{2}} \\
{1}/{\sqrt{2}} \\
\end{pmatrix} = g\begin{pmatrix}
{1}/{\sqrt{2}} \\
 - {1}/{\sqrt{2}} \\
\end{pmatrix} = g\begin{pmatrix}
{1}/{\sqrt{2}} \\
{1}/{\sqrt{2}} \\
\end{pmatrix} = 0$も成り立つ\footnote{5. の「上で求めた点$\mathbf{a}$のうち$g\left( \mathbf{a} \right) = \mathbf{0}$が成り立つようなものが極値をとる点でありうる。」にあたります。}。これにより、それらの点々$\begin{pmatrix}
 - {1}/{\sqrt{2}} \\
 - {1}/{\sqrt{2}} \\
\end{pmatrix}$、$\begin{pmatrix}
 - {1}/{\sqrt{2}} \\
{1}/{\sqrt{2}} \\
\end{pmatrix}$、$\begin{pmatrix}
{1}/{\sqrt{2}} \\
 - {1}/{\sqrt{2}} \\
\end{pmatrix}$、$\begin{pmatrix}
{1}/{\sqrt{2}} \\
{1}/{\sqrt{2}} \\
\end{pmatrix}$が極値をとりうる点でありうる。さて、$g\begin{pmatrix}
x \\
y \\
\end{pmatrix} = 0 \Leftrightarrow y = \varphi(x)$とおくと、次のようになることから、
\begin{align*}
\frac{\partial g}{\partial x}\begin{pmatrix}
x \\
y \\
\end{pmatrix} &= \frac{\partial g}{\partial\begin{pmatrix}
x \\
y \\
\end{pmatrix}}\begin{pmatrix}
x \\
y \\
\end{pmatrix}\frac{\partial}{\partial x}\begin{pmatrix}
x \\
y \\
\end{pmatrix}\\
&= \begin{pmatrix}
\frac{\partial g}{\partial x}\begin{pmatrix}
x \\
y \\
\end{pmatrix} & \frac{\partial g}{\partial y}\begin{pmatrix}
x \\
y \\
\end{pmatrix} \\
\end{pmatrix}\begin{pmatrix}
\frac{\partial x}{\partial x} \\
\frac{\partial\varphi}{\partial x}(x) \\
\end{pmatrix}\\
&= \begin{pmatrix}
2x & 2y \\
\end{pmatrix}\begin{pmatrix}
1 \\
\frac{\partial\varphi}{\partial x}(x) \\
\end{pmatrix}\\
&= 2x + 2y\frac{\partial\varphi}{\partial x}(x) = 0\\
\frac{\partial^{2}g}{\partial x^{2}}\begin{pmatrix}
x \\
y \\
\end{pmatrix} &= \frac{\partial}{\partial x}\frac{\partial g}{\partial x}\begin{pmatrix}
x \\
y \\
\end{pmatrix} = \frac{\partial}{\partial x}\left( 2x + 2y\frac{\partial\varphi}{\partial x}(x) \right) = 2 + 2\frac{\partial}{\partial x}\left( y\frac{\partial\varphi}{\partial x}(x) \right)\\
&= 2 + 2\left( \frac{\partial\varphi}{\partial x}(x) \right)^{2} + 2y\frac{\partial^{2}\varphi}{\partial x^{2}}(x)\\
&= 2 + 2\left( - \frac{x}{y} \right)^{2} + 2y\frac{\partial^{2}\varphi}{\partial x^{2}}(x)\\
&= 2 + \frac{2x^{2}}{y^{2}} + 2y\frac{\partial^{2}\varphi}{\partial x^{2}}(x) = 0
\end{align*}
次式が得られる。
\begin{align*}
\frac{\partial\varphi}{\partial x}(x) = - \frac{x}{y},\ \ \frac{\partial^{2}\varphi}{\partial x^{2}}(x) = - \frac{x^{2} + y^{2}}{y^{3}}
\end{align*}
したがって、それらの点々$\begin{pmatrix}
 - {1}/{\sqrt{2}} \\
 - {1}/{\sqrt{2}} \\
\end{pmatrix}$、$\begin{pmatrix}
 - {1}/{\sqrt{2}} \\
{1}/{\sqrt{2}} \\
\end{pmatrix}$、$\begin{pmatrix}
{1}/{\sqrt{2}} \\
 - {1}/{\sqrt{2}} \\
\end{pmatrix}$、$\begin{pmatrix}
{1}/{\sqrt{2}} \\
{1}/{\sqrt{2}} \\
\end{pmatrix}$それぞれ次のようになる\footnote{8. の「定理\ref{4.4.4.4}より6. における極値の候補となっている点$\mathbf{a}$でのその陰関数$\varphi$のJacobi行列$J_{\varphi}$とHesse行列$H_{\varphi}$の値を求める。」にあたります。}。
\begin{align*}
\frac{\partial\varphi}{\partial x}\left( - \frac{1}{\sqrt{2}} \right) = - 1&,\ \ \frac{\partial\varphi}{\partial x}\left( - \frac{1}{\sqrt{2}} \right) = 1,\\
\frac{\partial\varphi}{\partial x}\left( \frac{1}{\sqrt{2}} \right) = 1&,\ \ \frac{\partial\varphi}{\partial x}\left( \frac{1}{\sqrt{2}} \right) = - 1,\\
\frac{\partial^{2}\varphi}{\partial x^{2}}\left( - \frac{1}{\sqrt{2}} \right) = 2\sqrt{2}&,\ \ \frac{\partial^{2}\varphi}{\partial x^{2}}\left( - \frac{1}{\sqrt{2}} \right) = - 2\sqrt{2},\\
\frac{\partial^{2}\varphi}{\partial x^{2}}\left( \frac{1}{\sqrt{2}} \right) = 2\sqrt{2}&,\ \ \frac{\partial^{2}\varphi}{\partial x^{2}}\left( \frac{1}{\sqrt{2}} \right) = - 2\sqrt{2}
\end{align*}
次に、次のようになることから、
\begin{align*}
\frac{\partial f}{\partial x}\begin{pmatrix}
x \\
\varphi(x) \\
\end{pmatrix} &= \frac{\partial g}{\partial\begin{pmatrix}
x \\
y \\
\end{pmatrix}}\begin{pmatrix}
x \\
\varphi(x) \\
\end{pmatrix}\frac{\partial}{\partial x}\begin{pmatrix}
x \\
\varphi(x) \\
\end{pmatrix} \\
&= \begin{pmatrix}
\frac{\partial f}{\partial x}\begin{pmatrix}
x \\
\varphi(x) \\
\end{pmatrix} & \frac{\partial f}{\partial y}\begin{pmatrix}
x \\
\varphi(x) \\
\end{pmatrix} \\
\end{pmatrix}\begin{pmatrix}
\frac{\partial x}{\partial x} \\
\frac{\partial\varphi}{\partial x}(x) \\
\end{pmatrix}\\
&= \begin{pmatrix}
2y & 2x \\
\end{pmatrix}\begin{pmatrix}
1 \\
\frac{\partial\varphi}{\partial x}(x) \\
\end{pmatrix} \\
&= 2y + 2x\frac{\partial\varphi}{\partial x}(x)\\
\frac{\partial^{2}f}{\partial x^{2}}\begin{pmatrix}
x \\
\varphi(x) \\
\end{pmatrix} &= \frac{\partial}{\partial x}\frac{\partial f}{\partial x}\begin{pmatrix}
x \\
\varphi(x) \\
\end{pmatrix} \\
&= \frac{\partial}{\partial x}\left( 2y + 2x\frac{\partial\varphi}{\partial x}(x) \right) \\
&= 2\frac{\partial\varphi}{\partial x}(x) + 2\frac{\partial}{\partial x}\left( x\frac{\partial\varphi}{\partial x}(x) \right)\\
&= 4\frac{\partial\varphi}{\partial x}(x) + 2x\frac{\partial^{2}\varphi}{\partial x^{2}}(x)
\end{align*}
したがって、次のようになる\footnote{9. の「定理\ref{4.4.4.4}より6. における極値の候補となっている点$\mathbf{a}$でのその関数$F$のJacobi行列$J_{F}$とHesse行列$H_{F}$の値を8. を用いて求める。」にあたります。}。
\begin{align*}
\frac{\partial f}{\partial x}\begin{pmatrix}
 - {1}/{\sqrt{2}} \\
 - {1}/{\sqrt{2}} \\
\end{pmatrix} = \frac{\partial f}{\partial x}\begin{pmatrix}
 - {1}/{\sqrt{2}} \\
{1}/{\sqrt{2}} \\
\end{pmatrix} &= \frac{\partial f}{\partial x}\begin{pmatrix}
{1}/{\sqrt{2}} \\
 - {1}/{\sqrt{2}} \\
\end{pmatrix} = \frac{\partial f}{\partial x}\begin{pmatrix}
{1}/{\sqrt{2}} \\
{1}/{\sqrt{2}} \\
\end{pmatrix} = 0,\\
\frac{\partial^{2}f}{\partial x^{2}}\begin{pmatrix}
 - {1}/{\sqrt{2}} \\
 - {1}/{\sqrt{2}} \\
\end{pmatrix} = - 8,\ \ \frac{\partial^{2}f}{\partial x^{2}}\begin{pmatrix}
 - {1}/{\sqrt{2}} \\
{1}/{\sqrt{2}} \\
\end{pmatrix} = 8&,\ \ \frac{\partial^{2}f}{\partial x^{2}}\begin{pmatrix}
{1}/{\sqrt{2}} \\
 - {1}/{\sqrt{2}} \\
\end{pmatrix} = 8,\ \ \frac{\partial^{2}f}{\partial x^{2}}\begin{pmatrix}
{1}/{\sqrt{2}} \\
{1}/{\sqrt{2}} \\
\end{pmatrix} = - 8
\end{align*}
もちろん、これらの点々$\begin{pmatrix}
 - {1}/{\sqrt{2}} \\
 - {1}/{\sqrt{2}} \\
\end{pmatrix}$、$\begin{pmatrix}
 - {1}/{\sqrt{2}} \\
{1}/{\sqrt{2}} \\
\end{pmatrix}$、$\begin{pmatrix}
{1}/{\sqrt{2}} \\
 - {1}/{\sqrt{2}} \\
\end{pmatrix}$、$\begin{pmatrix}
{1}/{\sqrt{2}} \\
{1}/{\sqrt{2}} \\
\end{pmatrix}$は停留点であることもわかり次式が成り立つので、
\begin{align*}
\frac{\partial^{2}f}{\partial x^{2}}\begin{pmatrix}
 - {1}/{\sqrt{2}} \\
 - {1}/{\sqrt{2}} \\
\end{pmatrix} < 0,\ \ \frac{\partial^{2}f}{\partial x^{2}}\begin{pmatrix}
 - {1}/{\sqrt{2}} \\
{1}/{\sqrt{2}} \\
\end{pmatrix} > 0,\ \ \frac{\partial^{2}f}{\partial x^{2}}\begin{pmatrix}
{1}/{\sqrt{2}} \\
 - {1}/{\sqrt{2}} \\
\end{pmatrix} > 0,\ \ \frac{\partial^{2}f}{\partial x^{2}}\begin{pmatrix}
{1}/{\sqrt{2}} \\
{1}/{\sqrt{2}} \\
\end{pmatrix} < 0
\end{align*}
これらの点々$\begin{pmatrix}
 - {1}/{\sqrt{2}} \\
 - {1}/{\sqrt{2}} \\
\end{pmatrix}$、$\begin{pmatrix}
 - {1}/{\sqrt{2}} \\
{1}/{\sqrt{2}} \\
\end{pmatrix}$、$\begin{pmatrix}
{1}/{\sqrt{2}} \\
 - {1}/{\sqrt{2}} \\
\end{pmatrix}$、$\begin{pmatrix}
{1}/{\sqrt{2}} \\
{1}/{\sqrt{2}} \\
\end{pmatrix}$はそれぞれその条件$g\begin{pmatrix}
x \\
y \\
\end{pmatrix} = 0$の下でその関数$f$が極大値、極小値、極小値、極大値をとる点々である\footnote{10. の「あとは関数$F$の極値を前述した手順で求める。」にあたります。}。
\begin{thebibliography}{50}
\bibitem{1}
  杉浦光夫, 解析入門I, 東京大学出版社, 1980. 第34刷 p149-161 ISBN978-4-13-062005-5
\bibitem{2}
  杉浦光夫, 解析入門II, 東京大学出版社, 1985. 第22刷 p30-37 ISBN978-4-13-062006-2
\bibitem{3}
  和達三樹, 微分積分, 岩波書店, 1988. 新装版第3刷 p128-135 ISBN978-4-00-029883-4
\end{thebibliography}
\end{document}
