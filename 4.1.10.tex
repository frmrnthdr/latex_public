\documentclass[dvipdfmx]{jsarticle}
\setcounter{section}{1}
\setcounter{subsection}{9}
\usepackage{xr}
\externaldocument{4.1.3}
\externaldocument{4.1.4}
\externaldocument{4.1.5}
\usepackage{amsmath,amsfonts,amssymb,array,comment,mathtools,url,docmute}
\usepackage{longtable,booktabs,dcolumn,tabularx,mathtools,multirow,colortbl,xcolor}
\usepackage[dvipdfmx]{graphics}
\usepackage{bmpsize}
\usepackage{amsthm}
\usepackage{enumitem}
\setlistdepth{20}
\renewlist{itemize}{itemize}{20}
\setlist[itemize]{label=•}
\renewlist{enumerate}{enumerate}{20}
\setlist[enumerate]{label=\arabic*.}
\setcounter{MaxMatrixCols}{20}
\setcounter{tocdepth}{3}
\newcommand{\rotin}{\text{\rotatebox[origin=c]{90}{$\in $}}}
\newcommand{\amap}[6]{\text{\raisebox{-0.7cm}{\begin{tikzpicture} 
  \node (a) at (0, 1) {$\textstyle{#2}$};
  \node (b) at (#6, 1) {$\textstyle{#3}$};
  \node (c) at (0, 0) {$\textstyle{#4}$};
  \node (d) at (#6, 0) {$\textstyle{#5}$};
  \node (x) at (0, 0.5) {$\rotin $};
  \node (x) at (#6, 0.5) {$\rotin $};
  \draw[->] (a) to node[xshift=0pt, yshift=7pt] {$\textstyle{\scriptstyle{#1}}$} (b);
  \draw[|->] (c) to node[xshift=0pt, yshift=7pt] {$\textstyle{\scriptstyle{#1}}$} (d);
\end{tikzpicture}}}}
\newcommand{\twomaps}[9]{\text{\raisebox{-0.7cm}{\begin{tikzpicture} 
  \node (a) at (0, 1) {$\textstyle{#3}$};
  \node (b) at (#9, 1) {$\textstyle{#4}$};
  \node (c) at (#9+#9, 1) {$\textstyle{#5}$};
  \node (d) at (0, 0) {$\textstyle{#6}$};
  \node (e) at (#9, 0) {$\textstyle{#7}$};
  \node (f) at (#9+#9, 0) {$\textstyle{#8}$};
  \node (x) at (0, 0.5) {$\rotin $};
  \node (x) at (#9, 0.5) {$\rotin $};
  \node (x) at (#9+#9, 0.5) {$\rotin $};
  \draw[->] (a) to node[xshift=0pt, yshift=7pt] {$\textstyle{\scriptstyle{#1}}$} (b);
  \draw[|->] (d) to node[xshift=0pt, yshift=7pt] {$\textstyle{\scriptstyle{#2}}$} (e);
  \draw[->] (b) to node[xshift=0pt, yshift=7pt] {$\textstyle{\scriptstyle{#1}}$} (c);
  \draw[|->] (e) to node[xshift=0pt, yshift=7pt] {$\textstyle{\scriptstyle{#2}}$} (f);
\end{tikzpicture}}}}
\renewcommand{\thesection}{第\arabic{section}部}
\renewcommand{\thesubsection}{\arabic{section}.\arabic{subsection}}
\renewcommand{\thesubsubsection}{\arabic{section}.\arabic{subsection}.\arabic{subsubsection}}
\everymath{\displaystyle}
\allowdisplaybreaks[4]
\usepackage{vtable}
\theoremstyle{definition}
\newtheorem{thm}{定理}[subsection]
\newtheorem*{thm*}{定理}
\newtheorem{dfn}{定義}[subsection]
\newtheorem*{dfn*}{定義}
\newtheorem{axs}[dfn]{公理}
\newtheorem*{axs*}{公理}
\renewcommand{\headfont}{\bfseries}
\makeatletter
  \renewcommand{\section}{%
    \@startsection{section}{1}{\z@}%
    {\Cvs}{\Cvs}%
    {\normalfont\huge\headfont\raggedright}}
\makeatother
\makeatletter
  \renewcommand{\subsection}{%
    \@startsection{subsection}{2}{\z@}%
    {0.5\Cvs}{0.5\Cvs}%
    {\normalfont\LARGE\headfont\raggedright}}
\makeatother
\makeatletter
  \renewcommand{\subsubsection}{%
    \@startsection{subsubsection}{3}{\z@}%
    {0.4\Cvs}{0.4\Cvs}%
    {\normalfont\Large\headfont\raggedright}}
\makeatother
\makeatletter
\renewenvironment{proof}[1][\proofname]{\par
  \pushQED{\qed}%
  \normalfont \topsep6\p@\@plus6\p@\relax
  \trivlist
  \item\relax
  {
  #1\@addpunct{.}}\hspace\labelsep\ignorespaces
}{%
  \popQED\endtrivlist\@endpefalse
}
\makeatother
\renewcommand{\proofname}{\textbf{証明}}
\usepackage{tikz,graphics}
\usepackage[dvipdfmx]{hyperref}
\usepackage{pxjahyper}
\hypersetup{
 setpagesize=false,
 bookmarks=true,
 bookmarksdepth=tocdepth,
 bookmarksnumbered=true,
 colorlinks=false,
 pdftitle={},
 pdfsubject={},
 pdfauthor={},
 pdfkeywords={}}
\begin{document}
%\hypertarget{ux95a2ux6570ux306eux6975ux9650}{%
\subsection{関数の極限}%\label{ux95a2ux6570ux306eux6975ux9650}}
%\hypertarget{ux95a2ux6570ux306eux6975ux9650-1}{%
\subsubsection{関数の極限}%\label{ux95a2ux6570ux306eux6975ux9650-1}}
\begin{dfn}
$D(f) \subseteq R \subseteq \mathbb{R}^{m}$、$S \subseteq \mathbb{R}_{\infty}^{n}$なる関数$f:D(f) \rightarrow S$が与えられたとき、$\mathbf{a} \in \mathrm{cl}_{R}{D(f)}$、$\mathbf{b} \in \mathrm{cl}_{S}{V(f)}$なる点々$\mathbf{a}$、$\mathbf{b}$を用いて、$\forall\varepsilon \in \mathbb{R}^{+}\exists\delta \in \mathbb{R}^{+}\forall\mathbf{x} \in D(f)$に対し、$\mathbf{x} \in U\left( \mathbf{a},\delta \right) \cap R$が成り立つなら、$f\left( \mathbf{x} \right) \in U\left( \mathbf{b},\varepsilon \right) \cap S$が成り立つとき、その関数$f$の変数がその集合$R$でその点$\mathbf{a}$に近づくとき、その関数$f$はその集合$S$でその点$\mathbf{b}$に近づくという。この式、またはこの式を用いた議論を$\varepsilon$-$\delta$論法という。このことは例えば次式のように表される。
\begin{align*}
\lim_{\scriptsize \begin{matrix} \mathbf{x} \rightarrow \mathbf{a} \\ R \rightarrow S \\\end{matrix}}{f\left( \mathbf{x} \right)} = \mathbf{b},\ \ f\left( \mathbf{x} \right) \rightarrow \mathbf{b} \in \mathrm{cl}_{S}{V(f)}\ \left( \mathbf{x} \rightarrow \mathbf{a},R \rightarrow S \right)
\end{align*}\par
$\mathbf{x} \rightarrow \mathbf{a}$をその点$\mathbf{x}$がその点$\mathbf{a}$に近づく、$f\left( \mathbf{x} \right) \rightarrow \mathbf{b}$をその関数$f$はその点$\mathbf{b}$に広い意味で収束するといい、その点$\mathbf{b}$をそのときの広い意味での極限値などという。$\mathbf{b} \in \mathbb{R}^{m}$のとき、$f\left( \mathbf{x} \right) \rightarrow \mathbf{b}$をその関数$f$はその点$\mathbf{b}$に収束するといい、その点$\mathbf{b}$をそのときの極限値などという。$\mathbf{b} \notin \mathbb{R}^{m}$が成り立つとき、または、上のその論理式が成り立つようなその点$\mathbf{b}$が存在しないとき、その関数$f$は発散するという。特に、上のその論理式が成り立つようなその点$\mathbf{b}$が存在しないことをその関数$f$は振動するといい、形式的に$\lim_{\scriptsize \begin{matrix} \mathbf{x} \rightarrow \mathbf{a} \\ R \rightarrow S \\\end{matrix}}{f\left( \mathbf{x} \right)} = \mathrm{indefinite}$、$f\left( \mathbf{x} \right) \rightarrow \mathrm{indefinite}\ \left( \mathbf{x} \rightarrow \mathbf{a,\ \ }R \rightarrow S \right)$などと書くこともある。\par
拡張$n$次元数空間$\mathbb{R}_{\infty}^{n}$のかわりに補完数直線${}^{*}\mathbb{R}$でおきかえても同様にして定義される。
\end{dfn}
\begin{dfn}
$A \subseteq D(f) \subseteq R \subseteq \mathbb{R}^{m}$、$S \subseteq \mathbb{R}_{\infty}^{n}$なる関数$f:D(f) \rightarrow S$が与えられたとき、$\mathbf{a} \in \mathrm{cl}_{R}A$、$\mathbf{b} \in \mathrm{cl}_{S}\left( V(f) \right)$なる点々$\mathbf{a}$、$\mathbf{b}$を用いて$\lim_{\scriptsize \begin{matrix} \mathbf{x} \rightarrow \mathbf{a} \\ R \rightarrow S \\\end{matrix}}{f|A\left( \mathbf{x} \right)} = \mathbf{b}$が成り立つことを$\lim_{A \ni \mathbf{x} \rightarrow \mathbf{a} ,\ R \rightarrow S }{f\left( \mathbf{x} \right)} = \mathbf{b}$、$f\left( \mathbf{x} \right) \rightarrow \mathbf{b}\ \left( A \ni \mathbf{x} \rightarrow \mathbf{a},\ \ R \rightarrow S \right)$などと書きその関数$f$の変数がその集合$A$でその点$\mathbf{a}$に近づくとき、その関数$f$はその集合$S$でその点$\mathbf{b}$に近づくという。\par
拡張$n$次元数空間$\mathbb{R}_{\infty}^{n}$のかわりに補完数直線${}^{*}\mathbb{R}$でおきかえても同様にして定義される。
\end{dfn}
\begin{thm}\label{4.1.10.1}
$D(f) \subseteq R \subseteq \mathbb{R}^{m}$、$S \subseteq \mathbb{R}_{\infty}^{n}$なる関数$f:D(f) \rightarrow S$が与えられたとき、$\forall\mathbf{a} \in \mathrm{cl}_{R}{D(f)}$に対し、広い意味での極限値$\lim_{\scriptsize \begin{matrix} \mathbf{x} \rightarrow \mathbf{a} \\ R \rightarrow S \\\end{matrix}}{f\left( \mathbf{x} \right)}$が存在すれば、これはただ1つである。\par
拡張$n$次元数空間$\mathbb{R}_{\infty}^{n}$のかわりに補完数直線${}^{*}\mathbb{R}$でおきかえても同様にして示される。
\end{thm}
\begin{proof}
$D(f) \subseteq R \subseteq \mathbb{R}^{m}$、$S \subseteq \mathbb{R}_{\infty}^{n}$なる関数$f:D(f) \rightarrow S$が与えられたとき、$\forall\mathbf{a} \in \mathrm{cl}_{R}{D(f)}$に対し、広い意味での極限値$\lim_{\scriptsize \begin{matrix} \mathbf{x} \rightarrow \mathbf{a} \\ R \rightarrow S \\\end{matrix}}{f\left( \mathbf{x} \right)}$が存在するとき、これが2つの互いに異なる点々$\mathbf{b}$、$\mathbf{c}$であったとする。$\mathbf{b},\mathbf{c} \in \mathbb{R}^{n}$のとき、$\left\| \mathbf{b} - \mathbf{c} \right\| = 2\varepsilon$とおくと、$\forall\varepsilon \in \mathbb{R}^{+}\exists\delta \in \mathbb{R}^{+}\forall\mathbf{x} \in D(f)$に対し、$\mathbf{x} \in U\left( \mathbf{a},\delta \right) \cap R$が成り立つなら、$\left\| f\left( \mathbf{x} \right) - \mathbf{b} \right\| < \varepsilon$かつ$\left\| f\left( \mathbf{x} \right) - \mathbf{c} \right\| < \varepsilon$が成り立つ。したがって、三角不等式より次のようになる。
\begin{align*}
2\varepsilon = \left\| \mathbf{b} - \mathbf{c} \right\| \leq \left\| \mathbf{b} - f\left( \mathbf{x} \right) \right\| + \left\| f\left( \mathbf{x} \right) - \mathbf{c} \right\| = \left\| f\left( \mathbf{x} \right) - \mathbf{b} \right\| + \left\| f\left( \mathbf{x} \right) - \mathbf{c} \right\| < 2\varepsilon
\end{align*}
これにより$2\varepsilon < 2\varepsilon$が得られるが、これは矛盾している。
$\mathbf{b} \in \mathbb{R}^{n}$、$\mathbf{c} = a_{\infty}$のとき、$\forall\varepsilon \in \mathbb{R}^{+}\exists\delta \in \mathbb{R}^{+}\forall\mathbf{x} \in D(f)$に対し、$\mathbf{x} \in U\left( \mathbf{a},\delta \right) \cap R$が成り立つなら、$\left\| f\left( \mathbf{x} \right) - \mathbf{b} \right\| < \varepsilon$かつ$\varepsilon + \left\| \mathbf{b} \right\| < \left\| f\left( \mathbf{x} \right) \right\|$が成り立つ。したがって、三角不等式より次のようになる。
\begin{align*}
\left\| \mathbf{b} \right\| - \varepsilon < \left\| f\left( \mathbf{x} \right) \right\| < \left\| \mathbf{b} \right\| + \varepsilon < \left\| f\left( \mathbf{x} \right) \right\|
\end{align*}
これは矛盾している。よって、広い意味での極限値$\lim_{\scriptsize \begin{matrix} \mathbf{x} \rightarrow \mathbf{a} \\ R \rightarrow S \\\end{matrix}}{f\left( \mathbf{x} \right)}$が存在すれば、これはただ1つである。
\end{proof}
\begin{thm}\label{4.1.10.2}
$D(f) \subseteq R \subseteq \mathbb{R}^{m}$、$S \subseteq \mathbb{R}_{\infty}^{n}$なる関数$f:D(f) \rightarrow S$が与えられたとき、$\forall\mathbf{a} \in \mathrm{cl}_{R}{D(f)}$に対し、次のことは同値である。
\begin{itemize}
\item
  $f\left( \mathbf{x} \right) \rightarrow \mathbf{b}\ \left( \mathbf{x} \rightarrow \mathbf{a},\ \ R \rightarrow S \right)$が成り立つ。
\item
  任意のその集合$D(f)$の点列$\left( \mathbf{a}_{k} \right)_{k \in \mathbb{N}}$に対し、その集合$R$で$\lim_{k \rightarrow \infty}\mathbf{a}_{k} = \mathbf{a}$が成り立つなら\footnote{少なくともその集合$R$で$\lim_{k \rightarrow \infty}\mathbf{a}_{k} = \mathbf{a}$が成り立つようなその集合$D(f)$の点列$\left( \mathbf{a}_{k} \right)_{k \in \mathbb{N}}$が存在することは次の定理で分かることに注意しよう。
\begin{quote}
  $A \subseteq R \subseteq \mathbb{R}_{\infty}^{n}$なる集合$A$が与えられたとき、これのその集合$R$における閉包$\mathrm{cl}_{R}A$について、$\mathbf{a} \in \mathrm{cl}_{R}A$が成り立つならそのときに限り、その点$\mathbf{a}$にその集合$R$の広い意味で収束するその集合$A$の点列$\left( \mathbf{a}_{m} \right)_{m \in \mathbb{N}}:\mathbb{N} \rightarrow A$が存在する。
\end{quote}
}、$\lim_{k \rightarrow \infty}{f\left( \mathbf{a}_{k} \right)} = \mathbf{b} \in \mathrm{cl}_{S}{V(f)}$が成り立つ。
\end{itemize}\par
拡張$n$次元数空間$\mathbb{R}_{\infty}^{n}$のかわりに補完数直線${}^{*}\mathbb{R}$でおきかえても同様にして示される。
\end{thm}
\begin{proof}
$D(f) \subseteq R \subseteq \mathbb{R}^{m}$、$S \subseteq \mathbb{R}_{\infty}^{n}$なる関数$f:D(f) \rightarrow S$が与えられたとき、$\forall\mathbf{a} \in \mathrm{cl}_{R}{D(f)}$に対し、$f\left( \mathbf{x} \right) \rightarrow \mathbf{b}\ \left( \mathbf{x} \rightarrow \mathbf{a},\ \ R \rightarrow S \right)$が成り立つなら、$\forall\varepsilon \in \mathbb{R}^{+}\exists\delta \in \mathbb{R}^{+}\forall\mathbf{x} \in D(f)$に対し、$\mathbf{x} \in U\left( \mathbf{a},\delta \right) \cap R$が成り立つなら、$f\left( \mathbf{x} \right) \in U\left( \mathbf{b},\varepsilon \right) \cap S$が成り立つ。ここで、任意のその集合$D(f)$の点列$\left( \mathbf{a}_{k} \right)_{k \in \mathbb{N}}$に対し、その集合$R$で$\lim_{k \rightarrow \infty}\mathbf{a}_{k} = \mathbf{a}$が成り立つなら、$\forall\delta \in \mathbb{R}^{+}\exists N \in \mathbb{N}\forall k \in \mathbb{N}$に対し、$N \leq k$が成り立つなら、$\mathbf{a}_{k} \in U\left( \mathbf{a},\delta \right) \cap R$が成り立つ。したがって、$\forall\varepsilon \in \mathbb{R}^{+}\exists\delta \in \mathbb{R}^{+}\exists N \in \mathbb{N}\forall k \in \mathbb{N}$に対し、$N \leq k$が成り立つなら、$\mathbf{a}_{k} \in U\left( \mathbf{a},\delta \right) \cap R$が成り立ち、これが成り立つなら、$f\left( \mathbf{a}_{k} \right) \in U\left( \mathbf{b},\varepsilon \right) \cap S$が成り立つ。以上より、$\lim_{k \rightarrow \infty}{f\left( \mathbf{a}_{k} \right)} = \mathbf{b} \in \mathrm{cl}_{S}{V(f)}$が成り立つ。\par
逆に、$f\left( \mathbf{x} \right) \rightarrow \mathbf{b}\ \left( \mathbf{x} \rightarrow \mathbf{a},\ \ R \rightarrow S \right)$が成り立たなければ、定義より明らかに、$\exists\varepsilon \in \mathbb{R}^{+}\forall\delta \in \mathbb{R}^{+}\exists\mathbf{x} \in A$に対し、$\mathbf{x} \in U\left( \mathbf{a},\delta \right) \cap R$かつ$f\left( \mathbf{x} \right) \in U\left( \mathbf{b},\varepsilon \right) \cap S$が成り立つ。特に、選択の公理より、$\forall k \in \mathbb{N}\exists\mathbf{a}_{k} \in A$に対し、$\mathbf{a}_{k} \in U\left( \mathbf{a},\frac{1}{k} \right) \cap R$かつ$f\left( \mathbf{a}_{k} \right) \notin U\left( \mathbf{b},\varepsilon \right) \cap S$が成り立つような点$\mathbf{a}_{k}$が存在するので、これからその集合$D(f)$の点列$\left( \mathbf{a}_{k} \right)_{k \in \mathbb{N}}$が得られ、上の式より$\lim_{k \rightarrow \infty}\mathbf{a}_{k} = \mathbf{a}$が成り立つが、$f\left( \mathbf{a}_{k} \right) \notin U\left( \mathbf{b},\varepsilon \right) \cap S$が成り立つので、$\lim_{k \rightarrow \infty}{f\left( \mathbf{a}_{k} \right)} = \mathbf{b}$は成り立たない。したがって、対偶律により、任意のその集合$D(f)$の点列$\left( \mathbf{a}_{k} \right)_{k \in \mathbb{N}}$に対し、その集合$R$で$\lim_{k \rightarrow \infty}\mathbf{a}_{k} = \mathbf{a}$が成り立つなら、$\lim_{k \rightarrow \infty}{f\left( \mathbf{a}_{k} \right)} = \mathbf{b} \in \mathrm{cl}_{S}{V(f)}$が成り立つとき、$f\left( \mathbf{x} \right) \rightarrow \mathbf{b}\ \left( \mathbf{x} \rightarrow \mathbf{a},\ \ R \rightarrow S \right)$が成り立つ。
\end{proof}
\begin{thm}\label{4.1.10.3}
$D(f) \subseteq R \subseteq \mathbb{R}^{m}$、$S \subseteq \mathbb{R}_{\infty}^{n}$なる関数$f:D(f) \rightarrow S$が与えられたとき、$\forall\mathbf{a} \in \mathrm{cl}_{R}{D(f)}$に対し、次のことは同値である。
\begin{itemize}
\item
  広い意味での極限値$\lim_{\scriptsize \begin{matrix} \mathbf{x} \rightarrow \mathbf{a} \\ R \rightarrow S \\\end{matrix}}{f\left( \mathbf{x} \right)}$が存在する。
\item
  任意のその集合$D(f)$の点列$\left( \mathbf{a}_{k} \right)_{k \in \mathbb{N}}$に対し、その集合$R$で$\lim_{k \rightarrow \infty}\mathbf{a}_{k} = \mathbf{a}$が成り立つなら、広い意味での極限値$\lim_{k \rightarrow \infty}{f\left( \mathbf{a}_{k} \right)}$が存在する。
\end{itemize}
これが成り立つなら、次式が成り立つ。
\begin{align*}
\lim_{\scriptsize \begin{matrix} \mathbf{x} \rightarrow \mathbf{a} \\ R \rightarrow S \\\end{matrix}}{f\left( \mathbf{x} \right)} = \lim_{k \rightarrow \infty}{f\left( \mathbf{a}_{k} \right)}
\end{align*}\par
拡張$n$次元数空間$\mathbb{R}_{\infty}^{n}$のかわりに補完数直線${}^{*}\mathbb{R}$でおきかえても同様にして示される。
\end{thm}
\begin{proof}
$D(f) \subseteq R \subseteq \mathbb{R}^{m}$、$S \subseteq \mathbb{R}_{\infty}^{n}$なる関数$f:D(f) \rightarrow S$が与えられたとき、$\forall\mathbf{a} \in \mathrm{cl}_{R}{D(f)}$に対し、広い意味での極限値$\lim_{\scriptsize \begin{matrix} \mathbf{x} \rightarrow \mathbf{a} \\ R \rightarrow S \\\end{matrix}}{f\left( \mathbf{x} \right)}$が存在するなら、定理\ref{4.1.10.2}より任意のその集合$D(f)$の点列$\left( \mathbf{a}_{k} \right)_{k \in \mathbb{N}}$に対し、その集合$R$で$\lim_{k \rightarrow \infty}\mathbf{a}_{k} = \mathbf{a}$が成り立つなら、広い意味での極限値$\lim_{k \rightarrow \infty}{f\left( \mathbf{a}_{k} \right)}$も存在する。\par
逆に、任意のその集合$D(f)$の点列$\left( \mathbf{a}_{k} \right)_{k \in \mathbb{N}}$に対し、その集合$R$で$\lim_{k \rightarrow \infty}\mathbf{a}_{k} = \mathbf{a}$が成り立つなら、広い意味での極限値$\lim_{k \rightarrow \infty}{f\left( \mathbf{a}_{k} \right)}$が存在するとき、このような点列たち$\left( \mathbf{a}_{k} \right)_{k \in \mathbb{N}}$によるある自然数$k$の像$\mathbf{a}_{k}$をその点列たちから取り出して得られるどの点列でも、その点列の広い意味での極限値が点$\mathbf{a}$となるので、その広い意味での極限値$\lim_{k \rightarrow \infty}{f\left( \mathbf{a}_{k} \right)}$がその集合$\mathrm{cl}_{S}{V(f)}$に一意的に存在する。したがって、定理\ref{4.1.10.2}よりその広い意味での極限値$\lim_{\scriptsize \begin{matrix} \mathbf{x} \rightarrow \mathbf{a} \\ R \rightarrow S \\\end{matrix}}{f\left( \mathbf{x} \right)}$が存在する。\par
あとは、定理\ref{4.1.10.2}より分かる。
\end{proof}
\begin{thm}\label{4.1.10.4}
$D(f) \subseteq R \subseteq \mathbb{R}^{l}$、$D(g) \subseteq S \subseteq \mathbb{R}^{m}$、$T \subseteq \mathbb{R}_{\infty}^{n}$なる2つの関数たち$f:D(f) \rightarrow S$、$g:D(g) \rightarrow T$が与えられたとき、$V(f) \subseteq D(g)$が成り立つなら、これらの2つの関数たち$f$、$g$は合成可能であり\footnote{合成関数の詳しい議論のところは集合論に参照するといいかもしれません。}合成関数$g \circ f:D(f) \rightarrow T;\mathbf{x} \mapsto g\left( f\left( \mathbf{x} \right) \right)$が定義できるのであった。このとき、$\forall\mathbf{a} \in \mathrm{cl}_{R}{D(f)}$に対し、次のことが成り立つ。
\begin{itemize}
\item
  広い意味での極限値$\lim_{\scriptsize \begin{matrix} \mathbf{x} \rightarrow \mathbf{a} \\ R \rightarrow S \end{matrix}}{f\left( \mathbf{x} \right)}$が存在して$\lim_{\scriptsize \begin{matrix} \mathbf{x} \rightarrow \mathbf{a} \\ R \rightarrow S \end{matrix}}{f\left( \mathbf{x} \right)} = \mathbf{b}$が成り立つなら、$\mathbf{b} \in \mathrm{cl}_{S}{V(f)} \subseteq \mathrm{cl}_{S}{D(g)}$が成り立つ。
\item
  広い意味での極限値たち$\lim_{\scriptsize \begin{matrix} \mathbf{x} \rightarrow \mathbf{a} \\ R \rightarrow S \end{matrix}}{f\left( \mathbf{x} \right)}$、$\lim_{\scriptsize \begin{matrix} \mathbf{x} \rightarrow \mathbf{b} \\S \rightarrow T \end{matrix}}{g\left( \mathbf{x} \right)}$が存在して$\lim_{\scriptsize \begin{matrix} \mathbf{x} \rightarrow \mathbf{a} \\ R \rightarrow S \end{matrix}}{f\left( \mathbf{x} \right)} = \mathbf{b}$、$\lim_{\scriptsize \begin{matrix} \mathbf{x} \rightarrow \mathbf{b} \\S \rightarrow T \end{matrix}}{g\left( \mathbf{x} \right)} = \mathbf{c}$が成り立つなら、広い意味での極限値$\lim_{\scriptsize \begin{matrix} \mathbf{x} \rightarrow \mathbf{a} \\ R \rightarrow T \end{matrix}}{g \circ f\left( \mathbf{x} \right)}$が存在して$\lim_{\scriptsize \begin{matrix} \mathbf{x} \rightarrow \mathbf{a} \\ R \rightarrow T \end{matrix}}{g \circ f\left( \mathbf{x} \right)} = \mathbf{c}$が成り立つ。
\end{itemize}\par
拡張$n$次元数空間$\mathbb{R}_{\infty}^{n}$のかわりに補完数直線${}^{*}\mathbb{R}$でおきかえても同様にして示される。
\end{thm}
\begin{proof}
$D(f) \subseteq R \subseteq \mathbb{R}^{l}$、$D(g) \subseteq S \subseteq \mathbb{R}^{m}$、$T \subseteq \mathbb{R}^{n}$なる2つの関数たち$f:D(f) \rightarrow S$、$g:D(g) \rightarrow T$が与えられたとする。$V(f) \subseteq D(g)$が成り立つとき、$\forall\mathbf{a} \in \mathrm{cl}_{R}{D(f)}$に対し、定理\ref{4.1.4.4}よりその点$\mathbf{a}$に収束するその集合$D(f)$の点列$\left( \mathbf{a}_{k} \right)_{k \in \mathbb{N}}$が存在することに注意すれば、広い意味での極限値$\lim_{\scriptsize \begin{matrix} \mathbf{x} \rightarrow \mathbf{a} \\ R \rightarrow S \\\end{matrix}}{f\left( \mathbf{x} \right)}$が存在して$\lim_{\scriptsize \begin{matrix} \mathbf{x} \rightarrow \mathbf{a} \\ R \rightarrow S \\\end{matrix}}{f\left( \mathbf{x} \right)} = \mathbf{b}$が成り立つなら、定理\ref{4.1.10.2}より$\lim_{k \rightarrow \infty}{f\left( \mathbf{a}_{k} \right)} = \mathbf{b}$が成り立つので、定理\ref{4.1.4.4}より$\mathbf{b} \in \mathrm{cl}_{S}{V(f)}$が成り立つ。ここで、仮定より$V(f) \subseteq D(g)$が成り立つので、$\mathrm{cl}_{S}{V(f)} \subseteq \mathrm{cl}_{S}{D(g)}$が成り立つ。よって、$\mathbf{b} \in \mathrm{cl}_{S}{V(f)} \subseteq \mathrm{cl}_{S}{D(g)}$が成り立つ。\par
広い意味での極限値たち$\lim_{\scriptsize \begin{matrix} \mathbf{x} \rightarrow \mathbf{a} \\ R \rightarrow S \\\end{matrix}}{f\left( \mathbf{x} \right)}$、$\lim_{\scriptsize \begin{matrix} \mathbf{x} \rightarrow \mathbf{b} \\ S \rightarrow T \end{matrix}}{g\left( \mathbf{x} \right)}$が存在して$\lim_{\scriptsize \begin{matrix} \mathbf{x} \rightarrow \mathbf{a} \\ R \rightarrow S \\\end{matrix}}{f\left( \mathbf{x} \right)} = \mathbf{b}$、$\lim_{\scriptsize \begin{matrix} \mathbf{x} \rightarrow \mathbf{b} \\ S \rightarrow T \end{matrix}}{g\left( \mathbf{x} \right)} = \mathbf{c}$が成り立つなら、定義より$\forall\varepsilon \in \mathbb{R}^{+}\exists\delta \in \mathbb{R}^{+}\forall\mathbf{x} \in D(f)$に対し、$\mathbf{x} \in U\left( \mathbf{a},\delta \right) \cap R$が成り立つなら、$f\left( \mathbf{x} \right) \in U\left( \mathbf{b},\varepsilon \right) \cap S$が成り立つかつ、$\forall\varepsilon \in \mathbb{R}^{+}\exists\gamma \in \mathbb{R}^{+}\forall\mathbf{x} \in D(g)$に対し、$\mathbf{x} \in U\left( \mathbf{b},\gamma \right) \cap S$が成り立つなら、$g\left( \mathbf{x} \right) \in U\left( \mathbf{c},\varepsilon \right) \cap T$が成り立つ。したがって、$\forall\varepsilon \in \mathbb{R}^{+}\exists\gamma,\delta \in \mathbb{R}^{+}\forall\mathbf{x} \in D(f)$に対し、$\mathbf{x} \in U\left( \mathbf{a},\delta \right) \cap R$が成り立つなら、$f\left( \mathbf{x} \right) \in U\left( \mathbf{b},\gamma \right) \cap S$が成り立ち、したがって、これが成り立つなら、$g \circ f\left( \mathbf{x} \right) \in U\left( \mathbf{c},\varepsilon \right) \cap T$が成り立つ。よって、広い意味での極限値$\lim_{\scriptsize \begin{matrix} \mathbf{x} \rightarrow \mathbf{a} \\ R \rightarrow T \end{matrix}}{g \circ f\left( \mathbf{x} \right)}$が存在して$\lim_{\scriptsize \begin{matrix} \mathbf{x} \rightarrow \mathbf{a} \\ R \rightarrow T \end{matrix}}{g \circ f\left( \mathbf{x} \right)} = \mathbf{c}$が成り立つ。
\end{proof}
\begin{thm}[極限が開球で表される]\label{4.1.10.5}
$D(f) \subseteq R \subseteq \mathbb{R}^{m}$、$S \subseteq \mathbb{R}_{\infty}^{n}$なる関数$f:D(f) \rightarrow S$が与えられたとき、$\forall\mathbf{a} \in \mathrm{cl}_{R}{D(f)}$に対し、次のことは同値である。
\begin{itemize}
\item
  $f\left( \mathbf{x} \right) \rightarrow \mathbf{b}\ \left( \mathbf{x} \rightarrow \mathbf{a},\ \ R \rightarrow S \right)$が成り立つ。
\item
  $\forall\varepsilon \in \mathbb{R}^{+}\exists\delta \in \mathbb{R}^{+}\forall\mathbf{x} \in D(f)$に対し、$\mathbf{x} \in U\left( \mathbf{a},\delta \right) \cap R$が成り立つなら、$f\left( \mathbf{x} \right) \in U\left( \mathbf{b},\varepsilon \right) \cap S$が成り立つ。
\item
  $\forall\varepsilon \in \mathbb{R}^{+}\exists\delta \in \mathbb{R}^{+}$に対し、$U\left( \mathbf{a},\delta \right) \cap D(f) \subseteq V\left( f^{- 1}|U\left( \mathbf{b},\varepsilon \right) \cap S \right)$が成り立つ。
\item
  $\forall\varepsilon \in \mathbb{R}^{+}\exists\delta \in \mathbb{R}^{+}$に対し、$V\left( f|U\left( \mathbf{a},\delta \right) \cap D(f) \right) \subseteq U\left( \mathbf{b},\varepsilon \right) \cap S$が成り立つ。
\end{itemize}
このことをここでは極限が開球で表されると呼ぶことにする。\par
拡張$n$次元数空間$\mathbb{R}_{\infty}^{n}$のかわりに補完数直線${}^{*}\mathbb{R}$でおきかえても同様にして示される。
\end{thm}
\begin{proof}
$D(f) \subseteq R \subseteq \mathbb{R}^{m}$、$S \subseteq \mathbb{R}_{\infty}^{n}$なる関数$f:D(f) \rightarrow S$が与えられたとき、$\forall\mathbf{a} \in \mathrm{cl}_{R}{D(f)}$に対し、定義より明らかに次のことは同値である。
\begin{itemize}
\item
  $f\left( \mathbf{x} \right) \rightarrow \mathbf{b}\ \left( \mathbf{x} \rightarrow \mathbf{a},\ \ R \rightarrow S \right)$が成り立つ。
\item
  $\forall\varepsilon \in \mathbb{R}^{+}\exists\delta \in \mathbb{R}^{+}\forall\mathbf{x} \in D(f)$に対し、$\mathbf{x} \in U\left( \mathbf{a},\delta \right) \cap R$が成り立つなら、$f\left( \mathbf{x} \right) \in U\left( \mathbf{b},\varepsilon \right) \cap S$が成り立つ。
\end{itemize}
$\forall\varepsilon \in \mathbb{R}^{+}\exists\delta \in \mathbb{R}^{+}\forall\mathbf{x} \in D(f)$に対し、$\mathbf{x} \in U\left( \mathbf{a},\delta \right) \cap R$が成り立つなら、$f\left( \mathbf{x} \right) \in U\left( \mathbf{b},\varepsilon \right) \cap S$が成り立つとき、$\mathbf{x} \in V\left( f^{- 1}|U\left( \mathbf{b},\varepsilon \right) \cap S \right)$が成り立つので、$\forall\varepsilon \in \mathbb{R}^{+}\exists\delta \in \mathbb{R}^{+}$に対し、$U\left( \mathbf{a},\delta \right) \cap D(f) \subseteq V\left( f^{- 1}|U\left( \mathbf{b},\varepsilon \right) \cap S \right)$が成り立つ。逆に、これが成り立つなら、$\forall\mathbf{x} \in D(f)$に対し、$\mathbf{x} \in U\left( \mathbf{a},\delta \right) \cap R$が成り立つなら、$D(f) \subseteq R$より$\mathbf{x} \in U\left( \mathbf{a},\delta \right) \cap D(f)$が成り立つので、仮定より$\mathbf{x} \in V\left( f^{- 1}|U\left( \mathbf{b},\varepsilon \right) \cap S \right)$が成り立つ。ゆえに、$f\left( \mathbf{x} \right) \in U\left( \mathbf{b},\varepsilon \right) \cap S$が得られる。これにより、次のことは同値である。
\begin{itemize}
\item
  $\forall\varepsilon \in \mathbb{R}^{+}\exists\delta \in \mathbb{R}^{+}\forall\mathbf{x} \in D(f)$に対し、$\mathbf{x} \in U\left( \mathbf{a},\delta \right) \cap R$が成り立つなら、$f\left( \mathbf{x} \right) \in U\left( \mathbf{b},\varepsilon \right) \cap S$が成り立つ。
\item
  $\forall\varepsilon \in \mathbb{R}^{+}\exists\delta \in \mathbb{R}^{+}$に対し、$U\left( \mathbf{a},\delta \right) \cap D(f) \subseteq V\left( f^{- 1}|U\left( \mathbf{b},\varepsilon \right) \cap S \right)$が成り立つ。
\end{itemize}
$\forall\varepsilon \in \mathbb{R}^{+}\exists\delta \in \mathbb{R}^{+}$に対し、$U\left( \mathbf{a},\delta \right) \cap D(f) \subseteq V\left( f^{- 1}|U\left( \mathbf{b},\varepsilon \right) \cap S \right)$が成り立つなら、次のようになる。
\begin{align*}
V\left( f|U\left( \mathbf{a},\delta \right) \cap D(f) \right) \subseteq V\left( f|V\left( f^{- 1}|U\left( \mathbf{b},\varepsilon \right) \cap S \right) \right) \subseteq U\left( \mathbf{b},\varepsilon \right) \cap S
\end{align*}
逆に、$\forall\varepsilon \in \mathbb{R}^{+}\exists\delta \in \mathbb{R}^{+}$に対し、$V\left( f|U\left( \mathbf{a},\delta \right) \cap D(f) \right) \subseteq U\left( \mathbf{b},\varepsilon \right) \cap S$が成り立つなら、次のようになる。
\begin{align*}
U\left( \mathbf{a},\delta \right) \cap D(f) \subseteq V\left( f^{- 1}|V\left( f|U\left( \mathbf{a},\delta \right) \cap D(f) \right) \right) \subseteq V\left( f^{- 1}|U\left( \mathbf{b},\varepsilon \right) \cap S \right)
\end{align*}
これにより、次のことは同値である。
\begin{itemize}
\item
  $\forall\varepsilon \in \mathbb{R}^{+}\exists\delta \in \mathbb{R}^{+}$に対し、$U\left( \mathbf{a},\delta \right) \cap D(f) \subseteq V\left( f^{- 1}|U\left( \mathbf{b},\varepsilon \right) \cap S \right)$が成り立つ。
\item
  $\forall\varepsilon \in \mathbb{R}^{+}\exists\delta \in \mathbb{R}^{+}$に対し、$V\left( f|U\left( \mathbf{a},\delta \right) \cap D(f) \right) \subseteq U\left( \mathbf{b},\varepsilon \right) \cap S$が成り立つ。
\end{itemize}
\end{proof}
%\hypertarget{ux95a2ux6570ux306eux6975ux9650ux306eux53ceux675f}{%
\subsubsection{関数の極限の収束}%\label{ux95a2ux6570ux306eux6975ux9650ux306eux53ceux675f}}
\begin{thm}\label{4.1.10.6}
$D(f) \subseteq R \subseteq \mathbb{R}^{m}$、$S \subseteq \mathbb{R}_{\infty}^{n}$なる関数$f = \left( f_{i} \right)_{i \in \varLambda_{n}}:D(f) \rightarrow S$が与えられたとき、$\forall\mathbf{a} \in \mathrm{cl}_{R}{D(f)}$に対し、極限値$\lim_{\scriptsize \begin{matrix} \mathbf{x} \rightarrow \mathbf{a} \\ R \rightarrow S \\\end{matrix}}{f\left( \mathbf{x} \right)}$が$n$次元数空間$\mathbb{R}^{n}$に存在して$\lim_{\scriptsize \begin{matrix} \mathbf{x} \rightarrow \mathbf{a} \\ R \rightarrow S \\\end{matrix}}{f\left( \mathbf{x} \right)} = \mathbf{b}$が成り立つならそのときに限り、$\mathbf{b} = \left( b_{i} \right)_{i \in \varLambda_{n}}$として、$\forall i \in \varLambda_{m}$に対し、極限値$\lim_{\scriptsize \begin{matrix} \mathbf{x} \rightarrow \mathbf{a} \\ R \rightarrow S \\\end{matrix}}{f_{i}\left( \mathbf{x} \right)}$が集合$\mathbb{R}$に存在して$\lim_{\scriptsize \begin{matrix} \mathbf{x} \rightarrow \mathbf{a} \\ R \rightarrow S \\\end{matrix}}{f_{i}\left( \mathbf{x} \right)} = b_{i}$が成り立つ。\par
拡張$n$次元数空間$\mathbb{R}_{\infty}^{n}$のかわりに補完数直線${}^{*}\mathbb{R}$でおきかえても同様にして示される。
\end{thm}
\begin{proof} 定理\ref{4.1.4.6}、定理\ref{4.1.10.2}より明らかである。
\end{proof}
\begin{dfn}
$A \subseteq D(f) \subseteq R \subseteq \mathbb{R}^{m}$、$S \subseteq \mathbb{R}_{\infty}^{n}$なる関数$f:D(f) \rightarrow S$が与えられたとする。その値域$V\left( f|A \right)$が有界であるとき、その関数$f$はその集合$A$で有界であるという。\par
拡張$n$次元数空間$\mathbb{R}_{\infty}^{n}$のかわりに補完数直線${}^{*}\mathbb{R}$でおきかえても同様にして定義される。
\end{dfn}
\begin{thm}\label{4.1.10.7}
$A \subseteq D(f) \subseteq R \subseteq \mathbb{R}^{m}$、$S \subseteq \mathbb{R}_{\infty}^{n}$なる関数$f:D(f) \rightarrow S$が与えられたとき、これが集合$A$で有界であるなら、$\exists M \in \mathbb{R}^{+}\forall\mathbf{x} \in A$に対し、$\left\| f\left( \mathbf{x} \right) \right\| \leq M$が成り立つ。\par
拡張$n$次元数空間$\mathbb{R}_{\infty}^{n}$のかわりに補完数直線${}^{*}\mathbb{R}$でおきかえても同様にして示される。
\end{thm}
\begin{proof} 定理\ref{4.1.3.7}より明らかである。
\end{proof}
\begin{thm}\label{4.1.10.8}
$D(f) \subseteq R \subseteq \mathbb{R}^{m}$、$S \subseteq \mathbb{R}_{\infty}^{n}$なる関数$f:D(f) \rightarrow S$が与えられたとき、$\forall\mathbf{a} \in \mathrm{cl}_{R}{D(f)}$に対し、極限値$\lim_{\scriptsize \begin{matrix} \mathbf{x} \rightarrow \mathbf{a} \\ R \rightarrow S \\\end{matrix}}{f\left( \mathbf{x} \right)}$が$n$次元数空間$\mathbb{R}^{n}$に存在するなら、その関数$f$は、$\exists\delta \in \mathbb{R}^{+}$に対し、集合$U\left( \mathbf{a},\delta \right) \cap D(f)$で有界である。\par
拡張$n$次元数空間$\mathbb{R}_{\infty}^{n}$のかわりに補完数直線${}^{*}\mathbb{R}$でおきかえても同様にして示される。
\end{thm}
\begin{proof}
$D(f) \subseteq R \subseteq \mathbb{R}^{m}$、$S \subseteq \mathbb{R}_{\infty}^{n}$なる関数$f:D(f) \rightarrow S$が与えられたとき、$\forall\mathbf{a} \in \mathrm{cl}{D(f)}$に対し、極限値$\lim_{\scriptsize \begin{matrix} \mathbf{x} \rightarrow \mathbf{a} \\ R \rightarrow S \\\end{matrix}}{f\left( \mathbf{x} \right)}$が$n$次元数空間$\mathbb{R}^{n}$に存在するなら、定理\ref{4.1.10.5}より$\forall\varepsilon \in \mathbb{R}^{+}\exists\delta \in \mathbb{R}^{+}$に対し、$V\left( f|U\left( \mathbf{a},\delta \right) \cap D(f) \right) \subseteq U\left( \mathbf{b},\varepsilon \right) \cap S$が成り立つのであったので、定義よりその関数$f$はその集合$U\left( \mathbf{a},\delta \right) \cap D(f)$で有界である。
\end{proof}
\begin{thm}\label{4.1.10.9}
$D(f),D(g) \subseteq R \subseteq \mathbb{R}^{m}$、$S \subseteq \mathbb{R}_{\infty}^{n}$なる関数たち$f:D(f) \rightarrow S$、$g:D(g) \rightarrow S$が与えられたとき、$\forall\mathbf{a} \in \mathrm{cl}_{R}\left( D(f) \cap D(g) \right)$に対し、極限値たち$\lim_{\scriptsize \begin{matrix} \mathbf{x} \rightarrow \mathbf{a} \\ R \rightarrow S \\\end{matrix}}{f\left( \mathbf{x} \right)}$、$\lim_{\scriptsize \begin{matrix} \mathbf{x} \rightarrow \mathbf{a} \\ R \rightarrow S \\\end{matrix}}{g\left( \mathbf{x} \right)}$が$n$次元数空間$\mathbb{R}^{n}$に存在して$\lim_{\scriptsize \begin{matrix} \mathbf{x} \rightarrow \mathbf{a} \\ R \rightarrow S \\\end{matrix}}{f\left( \mathbf{x} \right)} = \mathbf{b}$、$\lim_{\scriptsize \begin{matrix} \mathbf{x} \rightarrow \mathbf{a} \\ R \rightarrow S \\\end{matrix}}{g\left( \mathbf{x} \right)} = \mathbf{c}$が成り立つなら、$\forall k,l \in \mathbb{R}$に対し、次式が成り立つ\footnote{より正確にいえば、関数$kf + lg:D(f) \cap D(g) \rightarrow S$で考えていることに注意しよう。}。
\begin{align*}
\lim_{\scriptsize \begin{matrix} \mathbf{x} \rightarrow \mathbf{a} \\ R \rightarrow S \\\end{matrix}}{(kf + lg)\left( \mathbf{x} \right)} = k\mathbf{b} + l\mathbf{c}
\end{align*}\par
拡張$n$次元数空間$\mathbb{R}_{\infty}^{n}$のかわりに補完数直線${}^{*}\mathbb{R}$でおきかえても同様にして示される。
\end{thm}
\begin{proof} 定理\ref{4.1.4.8}、定理\ref{4.1.10.2}より明らかである\footnote{つまり、関数の極限を点列の極限だと考えるってことですね! }。
\end{proof}
\begin{thm}\label{4.1.10.10}
$D(f),D(g) \subseteq R \subseteq \mathbb{R}^{m}$、$S \subseteq{}^{*}\mathbb{R}$なる関数たち$f:D(f) \rightarrow S$、$g:D(g) \rightarrow S$が与えられたとき、$\forall\mathbf{a} \in \mathrm{cl}_{R}\left( D(f) \cap D(g) \right)$に対し、極限値たち$\lim_{\scriptsize \begin{matrix} \mathbf{x} \rightarrow \mathbf{a} \\ R \rightarrow S \\\end{matrix}}{f\left( \mathbf{x} \right)}$、$\lim_{\scriptsize \begin{matrix} \mathbf{x} \rightarrow \mathbf{a} \\ R \rightarrow S \\\end{matrix}}{g\left( \mathbf{x} \right)}$が集合$\mathbb{R}$に存在して$\lim_{\scriptsize \begin{matrix} \mathbf{x} \rightarrow \mathbf{a} \\ R \rightarrow S \\\end{matrix}}{f\left( \mathbf{x} \right)} = b$、$\lim_{\scriptsize \begin{matrix} \mathbf{x} \rightarrow \mathbf{a} \\ R \rightarrow S \\\end{matrix}}{g\left( \mathbf{x} \right)} = c$が成り立つなら、次式が成り立つ\footnote{より正確にいえば、関数たち$fg$、$\frac{f}{g}$の定義域は$D(f) \cap D(g)$で考えていることに注意しよう。}。
\begin{align*}
\lim_{\scriptsize \begin{matrix} \mathbf{x} \rightarrow \mathbf{a} \\ R \rightarrow S \\\end{matrix}}{(fg)\left( \mathbf{x} \right)} &= bc\\
\lim_{\scriptsize \begin{matrix} \mathbf{x} \rightarrow \mathbf{a} \\ R \rightarrow S \\\end{matrix}}{\frac{f}{g}\left( \mathbf{x} \right)} &= \frac{b}{c}\ \mathrm{if}\ c \neq 0
\end{align*}
それらの関数たち$f$、$g$の値域が実数のかわりに複素数でおきかえても同様にして示される。
\end{thm}\par
これから直ちに分かることとして、次の系が与えられる。
\begin{thm}\label{4.1.10.10s}
$D(f),D(g) \subseteq R \subseteq \mathbb{R}^{m}$、$S \subseteq \mathbb{R}_{\infty}^{n}$なる関数たち$f:D(f) \rightarrow S$、$g:D(g) \rightarrow S$が与えられたとき、$\forall\mathbf{a} \in \mathrm{cl}_{R}\left( D(f) \cap D(g) \right)$に対し、極限値たち$\lim_{\scriptsize \begin{matrix} \mathbf{x} \rightarrow \mathbf{a} \\ R \rightarrow S \\\end{matrix}}{f\left( \mathbf{x} \right)}$、$\lim_{\scriptsize \begin{matrix} \mathbf{x} \rightarrow \mathbf{a} \\ R \rightarrow S \\\end{matrix}}{g\left( \mathbf{x} \right)}$が$n$次元数空間$\mathbb{R}^{n}$に存在して$\lim_{\scriptsize \begin{matrix} \mathbf{x} \rightarrow \mathbf{a} \\ R \rightarrow S \\\end{matrix}}{f\left( \mathbf{x} \right)} = \mathbf{b}$、$\lim_{\scriptsize \begin{matrix} \mathbf{x} \rightarrow \mathbf{a} \\ R \rightarrow S \\\end{matrix}}{g\left( \mathbf{x} \right)} = \mathbf{c}$が成り立つなら、次式が成り立つ\footnote{証明としては平たくいえば成分表示して先ほどの定理に適用する感じです。}。
\begin{align*}
\lim_{\scriptsize \begin{matrix} \mathbf{x} \rightarrow \mathbf{a} \\ R \rightarrow S \\\end{matrix}}{\left(^{t}fg \right)\left( \mathbf{x} \right)} =^{t}\mathbf{bc}
\end{align*}
\end{thm}
\begin{proof} 定理\ref{4.1.4.10}、定理\ref{4.1.10.2}より明らかである\footnote{つまり、関数の極限を点列の極限だと考えるってことですね! }。
\end{proof}
%\hypertarget{ux95a2ux6570ux306eux6975ux9650ux3068ux4e0dux7b49ux5f0f}{%
\subsubsection{関数の極限と不等式}%\label{ux95a2ux6570ux306eux6975ux9650ux3068ux4e0dux7b49ux5f0f}}
\begin{thm}\label{4.1.10.11}
$D(f),D(g) \subseteq R \subseteq \mathbb{R}^{m}$、$S \subseteq{}^{*}\mathbb{R}$なる関数たち$f:D(f) \rightarrow S$、$g:D(g) \rightarrow S$が与えられたとき、$\forall\mathbf{a} \in \mathrm{cl}_{R}\left( D(f) \cap D(g) \right)$に対し、$f \leq g$が成り立つかつ、$\mathbf{x} \rightarrow \mathbf{a}$のとき、これらの関数たち$f$、$g$がそれぞれ拡大実数$a$、$b$に収束するなら、$a \leq b$が成り立つ。
\end{thm}\par
上記の不等式$f \leq g$は$f < g$または$f = g$であったので、$f < g$であったとしても、$f \leq g$が成り立つとみなされ、$a \leq b$は不等式$f \leq g$の等号、不等号の有無に依存しない。
\begin{proof} 定理\ref{4.1.4.12}、定理\ref{4.1.10.2}より明らかである。
\end{proof}
\begin{thm}[追い出しの原理]\label{4.1.10.12}
$D(f),D(g) \subseteq R \subseteq \mathbb{R}^{m}$、$S \subseteq{}^{*}\mathbb{R}$なる関数たち$f:D(f) \rightarrow S$、$g:D(g) \rightarrow S$が与えられたとき、$\forall\mathbf{a} \in \mathrm{cl}_{R}\left( D(f) \cap D(g) \right)$に対し、次のことが成り立つ。
\begin{itemize}
\item
  $f \leq g$が成り立つかつ、$\mathbf{x} \rightarrow \mathbf{a}$のとき、その関数$f$が正の無限大に発散するなら、その関数$g$も正の無限大に発散する。
\item
  $f \leq g$が成り立つかつ、$\mathbf{x} \rightarrow \mathbf{a}$のとき、その関数$g$が負の無限大に発散するなら、その関数$f$も負の無限大に発散する。
\end{itemize}
この定理も追い出しの原理という。
\end{thm}
\begin{proof} 定理\ref{4.1.4.13}、定理\ref{4.1.10.2}より明らかである。
\end{proof}
\begin{thm}[はさみうちの原理]\label{4.1.10.13}
$D(f),D(g),D(h) \subseteq R \subseteq \mathbb{R}^{m}$、$S \subseteq{}^{*}\mathbb{R}$なる関数たち$f:D(f) \rightarrow S$、$g:D(g) \rightarrow S$、$h:D(h) \rightarrow S$が与えられたとき、$\forall\mathbf{a} \in \mathrm{cl}_{R}\left( D(f) \cap D(g) \cap D(h) \right)$に対し、$f \leq g \leq h$が成り立つかつ、$\mathbf{x} \rightarrow \mathbf{a}$のとき、これらの関数たち$f$、$h$がどちらも実数$a$に収束するなら、その関数$g$もその実数$a$に収束する。この定理もはさみうちの原理という。
\end{thm}
\begin{proof} 定理\ref{4.1.4.14}、定理\ref{4.1.10.2}より明らかである。
\end{proof}
%\hypertarget{ux95a2ux6570ux306eux6975ux9650ux306bux95a2ux3059ux308bcauchyux306eux53ceux675fux6761ux4ef6}{%
\subsubsection{関数の極限に関するCauchyの収束条件}%\label{ux95a2ux6570ux306eux6975ux9650ux306bux95a2ux3059ux308bcauchyux306eux53ceux675fux6761ux4ef6}}
\begin{thm}[関数の極限に関するCauchyの収束条件]\label{4.1.10.14}
$D(f) \subseteq R \subseteq \mathbb{R}^{m}$なる関数$f:D(f) \rightarrow \mathbb{R}_{\infty}^{n}$が与えられたとき、$\forall\mathbf{a} \in \mathrm{cl}_{R}{D(f)}$に対し、極限値$\lim_{\scriptsize \begin{matrix} \mathbf{x} \rightarrow \mathbf{a} \\ R \rightarrow S \\\end{matrix}}{f\left( \mathbf{x} \right)}$が$n$次元数空間$\mathbb{R}^{n}$に存在するならそのときに限り、$\forall\varepsilon \in \mathbb{R}^{+}\exists\delta \in \mathbb{R}^{+}\forall\mathbf{x},\mathbf{y} \in D(f)$に対し、$\mathbf{x},\mathbf{y} \in U\left( \mathbf{a},\delta \right) \cap R$が成り立つなら、$\left\| f\left( \mathbf{x} \right) - f\left( \mathbf{y} \right) \right\| < \varepsilon$が成り立つ。この定理を関数の極限に関するCauchyの収束条件という。
\end{thm}
\begin{proof}
$D(f) \subseteq R \subseteq \mathbb{R}^{m}$なる関数$f:D(f) \rightarrow \mathbb{R}_{\infty}^{n}$が与えられたとき、$\forall\mathbf{a} \in \mathrm{cl}_{R}{D(f)}$に対し、極限値$\lim_{\scriptsize \begin{matrix} \mathbf{x} \rightarrow \mathbf{a} \\ R \rightarrow S \\\end{matrix}}{f\left( \mathbf{x} \right)}$が$n$次元数空間$\mathbb{R}^{n}$に存在するなら、この極限値$\lim_{\scriptsize \begin{matrix} \mathbf{x} \rightarrow \mathbf{a} \\ R \rightarrow S \\\end{matrix}}{f\left( \mathbf{x} \right)}$を$\mathbf{b}$とおけば、$\forall\varepsilon \in \mathbb{R}^{+}\exists\delta \in \mathbb{R}^{+}\forall\mathbf{x},\mathbf{y} \in D(f)$に対し、$\mathbf{x},\mathbf{y} \in U\left( \mathbf{a},\delta \right) \cap R$が成り立つなら、次式が成り立つ。
\begin{align*}
\left\| f\left( \mathbf{x} \right) - \mathbf{b} \right\| + \left\| f\left( \mathbf{y} \right) - \mathbf{b} \right\| &= \left\| f\left( \mathbf{x} \right) - \mathbf{b} \right\| + \left\| - f\left( \mathbf{y} \right) + \mathbf{b} \right\|\\
&\leq \left\| f\left( \mathbf{x} \right) - \mathbf{b} - f\left( \mathbf{y} \right) + \mathbf{b} \right\|\\
&= \left\| f\left( \mathbf{x} \right) - f\left( \mathbf{y} \right) \right\| < \varepsilon
\end{align*}\par
逆に、$\forall\varepsilon \in \mathbb{R}^{+}\exists\delta \in \mathbb{R}^{+}\forall\mathbf{x},\mathbf{y} \in D(f)$に対し、$\mathbf{x},\mathbf{y} \in U\left( \mathbf{a},\delta \right) \cap R$が成り立つなら、$\left\| f\left( \mathbf{x} \right) - f\left( \mathbf{y} \right) \right\| < \varepsilon$が成り立つとき、定理\ref{4.1.4.4}よりその点$\mathbf{a}$に収束するその集合$D(f)$の点列$\left( \mathbf{a}_{k} \right)_{k \in \mathbb{N}}$が存在するので、これを用いて考えれば、$\forall\varepsilon \in \mathbb{R}^{+}\exists N \in \mathbb{N}\forall k,l \in \mathbb{N}$に対し、$N \leq k$かつ$N \leq l$が成り立つなら、$\left\| \mathbf{a}_{k} - \mathbf{a} \right\| < \varepsilon$かつ$\left\| \mathbf{a}_{l} - \mathbf{a} \right\| < \varepsilon$が成り立ち、したがって、$\mathbf{a}_{k},\mathbf{a}_{l} \in U\left( \mathbf{a},\delta \right) \cap R$が成り立つ。これにより、$\left\| f\left( \mathbf{a}_{k} \right) - f\left( \mathbf{a}_{l} \right) \right\| < \varepsilon$が成り立つので、その点列$\left( f\left( \mathbf{a}_{k} \right) \right)_{k \in \mathbb{N}}$はCauchy列である。したがって、定理\ref{4.1.5.10}のCauchyの収束条件と定理\ref{4.1.10.2}よりその極限値$\lim_{\scriptsize \begin{matrix} \mathbf{x} \rightarrow \mathbf{a} \\ R \rightarrow S \\\end{matrix}}{f\left( \mathbf{x} \right)}$が$n$次元数空間$\mathbb{R}^{n}$に存在する。
\end{proof}
%\hypertarget{ux53f3ux6975ux9650ux3068ux5de6ux6975ux9650}{%
\subsubsection{右極限と左極限}%\label{ux53f3ux6975ux9650ux3068ux5de6ux6975ux9650}}
\begin{dfn}
$D(f) \subseteq R \subseteq \mathbb{R}$、$S \subseteq \mathbb{R}_{\infty}^{n}$なる関数$f:D(f) \rightarrow S$が与えられたとき、$a \in \mathrm{cl}_{R}{D(f)}$、$\mathbf{b} \in \mathrm{cl}_{S}{V(f)}$なる点々$a$、$\mathbf{b}$を用いて$\lim_{\scriptsize \begin{matrix} x \rightarrow a \\ R \rightarrow S \end{matrix}}{f|(a,\infty) \cap D(f)} = \mathbf{b}$が成り立つことを$\lim_{\scriptsize \begin{matrix} x \rightarrow a + 0 \\ R \rightarrow S \end{matrix}}{f(x)} = \mathbf{b}$、$f(x) \rightarrow \mathbf{b}\ \left( x \rightarrow a + 0,\ \ R \rightarrow S \right)$などと書きこれを右側極限、右極限という。$\lim_{\scriptsize \begin{matrix} x \rightarrow a \\ R \rightarrow S \end{matrix}}{f|( - \infty,a) \cap D(f)} = \mathbf{b}$が成り立つことを$\lim_{\scriptsize \begin{matrix} x \rightarrow a - 0 \\ R \rightarrow S \end{matrix}}{f(x)} = \mathbf{b}$、$f(x) \rightarrow \mathbf{b}\ (x \rightarrow a - 0,\ \ R \rightarrow S)$などと書きこれを左側極限、左極限という。
\end{dfn}
\begin{thm}\label{4.1.10.15}
$D(f) \subseteq R \subseteq \mathbb{R}$、$S \subseteq \mathbb{R}^{n}$なる関数$f:D(f) \rightarrow S$が与えられたとき、次のことは同値である。
\begin{itemize}
\item
  $\lim_{\scriptsize \begin{matrix} x \rightarrow + 0 \\ R \rightarrow S \end{matrix}}{f(x)} = \lim_{\scriptsize \begin{matrix} x \rightarrow a - 0 \\ R \rightarrow S \end{matrix}}{f(x)} = \mathbf{b}$が成り立つ。
\item
  $f(x) \rightarrow \mathbf{b}\ (x \rightarrow a,\ \ x \neq a,\ \ R \rightarrow S)$が成り立つ。
\end{itemize}
\end{thm}
\begin{proof}
2つの集合たち$(a,\infty)$、$( - \infty,a)$の和集合が$\mathbb{R} \setminus \left\{ a \right\}$であるから、明らかである。
\end{proof}
%\hypertarget{ux9023ux7d9a}{%
\subsubsection{連続}%\label{ux9023ux7d9a}}
\begin{dfn}
$A \subseteq D(f) \subseteq R \subseteq \mathbb{R}^{m}$、$S \subseteq \mathbb{R}_{\infty}^{n}$なる関数$f:D(f) \rightarrow S$が与えられたとき、$\forall\mathbf{a} \in A$に対し、$\lim_{\scriptsize \begin{matrix} A \ni \mathbf{x} \rightarrow \mathbf{a} \\ R \rightarrow S \end{matrix}}{f\left( \mathbf{x} \right)} = f\left( \mathbf{a} \right)$が成り立つとき、その関数$f$はその集合$A$で連続であるなどという。特に、その集合$A$が$A = \left\{ \mathbf{a} \right\}$と与えられているとき、その関数$f$は点$\mathbf{a}$で連続であるなどという。
\end{dfn}
\begin{thm}\label{4.1.10.16}
$D(f) \subseteq R \subseteq \mathbb{R}^{m}$、$S \subseteq \mathbb{R}_{\infty}^{n}$なる関数$f:D(f) \rightarrow S$が与えられたとき、$\forall\mathbf{a} \in D(f)$に対し、次のことは同値である。
\begin{itemize}
\item
  その関数$f$はその点$\mathbf{a}$で連続である。
\item
  $\forall\varepsilon \in \mathbb{R}^{+}\exists\delta \in \mathbb{R}^{+}\forall\mathbf{x} \in D(f)$に対し、$\mathbf{x} \in U\left( \mathbf{a},\delta \right) \cap R$が成り立つなら、$f\left( \mathbf{x} \right) \in U\left( f\left( \mathbf{a} \right),\varepsilon \right)$が成り立つ。
\item
  $\forall\varepsilon \in \mathbb{R}^{+}\exists\delta \in \mathbb{R}^{+}$に対し、$V\left( f|U\left( \mathbf{a},\delta \right) \cap D(f) \right) \subseteq U\left( f\left( \mathbf{a} \right),\varepsilon \right) \cap S$が成り立つ。
\item
  $\forall\varepsilon \in \mathbb{R}^{+}\exists\delta \in \mathbb{R}^{+}$に対し、$U\left( \mathbf{a},\delta \right) \cap D(f) \subseteq V\left( f^{- 1}|U\left( f\left( \mathbf{a} \right),\varepsilon \right) \cap S \right)$が成り立つ。
\item
  $\forall\varepsilon \in \mathbb{R}^{+}$に対し、$\mathbf{a} \in \mathrm{int}_{D(f)}{V\left( f^{- 1}|U\left( f\left( \mathbf{a} \right),\varepsilon \right) \cap S \right)}$が成り立つ。
\item
  $\forall U \in \mathfrak{P}(S)$に対し、$f\left( \mathbf{a} \right) \in \mathrm{int}_{S}U$が成り立つなら、$\mathbf{a} \in \mathrm{int}_{D(f)}{V\left( f^{- 1}|U \right)}$が成り立つ。
\item
  広い意味での極限値$\lim_{\scriptsize \begin{matrix} \mathbf{x} \rightarrow \mathbf{a} \\ \mathbf{x} \neq \mathbf{a} \\ R \rightarrow S \\\end{matrix}}{f\left( \mathbf{x} \right)}$がその集合$S$で存在しこれがその点$f\left( \mathbf{a} \right)$に等しい。
\end{itemize}
\end{thm}
\begin{proof} 定義と定理\ref{4.1.10.5}より明らかに次のことは同値である。
\begin{itemize}
\item
  その関数$f$はその点$\mathbf{a}$で連続である。
\item
  $\forall\varepsilon \in \mathbb{R}^{+}\exists\delta \in \mathbb{R}^{+}\forall\mathbf{x} \in D(f)$に対し、$\mathbf{x} \in U\left( \mathbf{a},\delta \right) \cap R$が成り立つなら、$f\left( \mathbf{x} \right) \in U\left( f\left( \mathbf{a} \right),\varepsilon \right)$が成り立つ。
\item
  $\forall\varepsilon \in \mathbb{R}^{+}\exists\delta \in \mathbb{R}^{+}$に対し、$V\left( f|U\left( \mathbf{a},\delta \right) \cap D(f) \right) \subseteq U\left( f\left( \mathbf{a} \right),\varepsilon \right) \cap S$が成り立つ。
\item
  $\forall\varepsilon \in \mathbb{R}^{+}\exists\delta \in \mathbb{R}^{+}$に対し、$U\left( \mathbf{a},\delta \right) \cap D(f) \subseteq V\left( f^{- 1}|U\left( f\left( \mathbf{a} \right),\varepsilon \right) \cap S \right)$が成り立つ。
\end{itemize}
また、開核の定義より次のことは同値である。
\begin{itemize}
\item
  $\forall\varepsilon \in \mathbb{R}^{+}\exists\delta \in \mathbb{R}^{+}$に対し、$U\left( \mathbf{a},\delta \right) \cap D(f) \subseteq V\left( f^{- 1}|U\left( f\left( \mathbf{a} \right),\varepsilon \right) \cap S \right)$が成り立つ。
\item
  $\forall\varepsilon \in \mathbb{R}^{+}$に対し、$\mathbf{a} \in \mathrm{int}_{D(f)}{V\left( f^{- 1}|U\left( f\left( \mathbf{a} \right),\varepsilon \right) \cap S \right)}$が成り立つ。
\end{itemize}
さらに、$\forall\varepsilon \in \mathbb{R}^{+}\exists\delta \in \mathbb{R}^{+}$に対し、$U\left( \mathbf{a},\delta \right) \cap D(f) \subseteq V\left( f^{- 1}|U\left( f\left( \mathbf{a} \right),\varepsilon \right) \cap S \right)$が成り立つとき、$\forall U \in \mathfrak{P}(S)$に対し、$f\left( \mathbf{a} \right) \in \mathrm{int}_{S}U$が成り立つなら、$\exists\varepsilon \in \mathbb{R}^{+}$に対し、$U\left( f\left( \mathbf{a} \right),\varepsilon \right) \cap S \subseteq U$が成り立つので、$V\left( f^{- 1}|U\left( f\left( \mathbf{a} \right),\varepsilon \right) \cap S \right) \subseteq V\left( f^{- 1}|U \right)$が成り立つかつ、$\exists\delta \in \mathbb{R}^{+}$に対し、$U\left( \mathbf{a},\delta \right) \cap D(f) \subseteq V\left( f^{- 1}|U\left( f\left( \mathbf{a} \right),\varepsilon \right) \cap S \right)$が成り立つことから、$U\left( \mathbf{a},\delta \right) \cap D(f) \subseteq V\left( f^{- 1}|U \right)$が得られる。これにより、$\mathbf{a} \in \mathrm{int}_{D(f)}{V\left( f^{- 1}|U \right)}$が成り立つ。逆に、$\forall U \in \mathfrak{P}(S)$に対し、$f\left( \mathbf{a} \right) \in \mathrm{int}_{S}U$が成り立つなら、$\mathbf{a} \in \mathrm{int}_{D(f)}{V\left( f^{- 1}|U \right)}$が成り立つとき、明らかに、$\forall\varepsilon \in \mathbb{R}^{+}$に対し、$\mathbf{a} \in \mathrm{int}_{D(f)}{V\left( f^{- 1}|U\left( f\left( \mathbf{a} \right),\varepsilon \right) \cap S \right)}$が成り立つ。これにより、次のことは同値である。
\begin{itemize}
\item
  $\forall\varepsilon \in \mathbb{R}^{+}\exists\delta \in \mathbb{R}^{+}$に対し、$U\left( \mathbf{a},\delta \right) \cap D(f) \subseteq V\left( f^{- 1}|U\left( f\left( \mathbf{a} \right),\varepsilon \right) \cap S \right)$が成り立つ。
\item
  $\forall\varepsilon \in \mathbb{R}^{+}$に対し、$\mathbf{a} \in \mathrm{int}_{D(f)}{V\left( f^{- 1}|U\left( f\left( \mathbf{a} \right),\varepsilon \right) \cap S \right)}$が成り立つ。
\item
  $\forall U \in \mathfrak{P}(S)$に対し、$f\left( \mathbf{a} \right) \in \mathrm{int}_{S}U$が成り立つなら、$\mathbf{a} \in \mathrm{int}_{D(f)}{V\left( f^{- 1}|U \right)}$が成り立つ。
\end{itemize}
さらに、$\forall\varepsilon \in \mathbb{R}^{+}\exists\delta \in \mathbb{R}^{+}\forall\mathbf{x} \in D(f)$に対し、$\mathbf{x} \in U\left( \mathbf{a},\delta \right) \cap R$が成り立つなら、$f\left( \mathbf{x} \right) \in U\left( f\left( \mathbf{a} \right),\varepsilon \right)$が成り立つとき、もちろん、$\mathbf{x} \neq \mathbf{a}$としても成り立つので、その極限値$\lim_{\scriptsize \begin{matrix} \mathbf{x} \rightarrow \mathbf{a} \\ \mathbf{x} \neq \mathbf{a}\\R \rightarrow S \end{matrix}}{f\left( \mathbf{x} \right)}$がその集合$S$で存在しこれがその点$f\left( \mathbf{a} \right)$に等しい。逆にこれが成り立つなら、$\forall\varepsilon \in \mathbb{R}^{+}\exists\delta \in \mathbb{R}^{+}$に対し、$\mathbf{a} \in U\left( \mathbf{a},\delta \right) \cap R$が成り立つなら、$f\left( \mathbf{a} \right) \in U\left( f\left( \mathbf{a} \right),\varepsilon \right)$が成り立つので、その関数$f$はその点$\mathbf{a}$で連続である。ゆえに、次のことは同値である。
\begin{itemize}
\item
  その関数$f$はその点$\mathbf{a}$で連続である。
\item
  $\forall\varepsilon \in \mathbb{R}^{+}\exists\delta \in \mathbb{R}^{+}\forall\mathbf{x} \in D(f)$に対し、$\mathbf{x} \in U\left( \mathbf{a},\delta \right) \cap R$が成り立つなら、$f\left( \mathbf{x} \right) \in U\left( f\left( \mathbf{a} \right),\varepsilon \right)$が成り立つ。
\item
  広い意味での極限値$\lim_{\scriptsize \begin{matrix} \mathbf{x} \rightarrow \mathbf{a} \\ \mathbf{x} \neq \mathbf{a} \\ R \rightarrow S \end{matrix}}{f\left( \mathbf{x} \right)}$がその集合$S$で存在しこれがその点$f\left( \mathbf{a} \right)$に等しい。
\end{itemize}
\end{proof}
\begin{thm}\label{4.1.10.17}
$A \subseteq D(f) \subseteq R \subseteq \mathbb{R}^{m}$、$S \subseteq \mathbb{R}_{\infty}^{n}$なる関数$f:D(f) \rightarrow S$が与えられたとき、次のことは同値である。
\begin{itemize}
\item
  その関数$f$がその集合$A$で連続である。
\item
  $\forall U \in \mathfrak{P}(S)$に対し、その集合$U$がその集合$S$で開集合であるなら、その集合$V\left( f^{- 1}|U \right) \cap A$もその集合$A$で開集合である。
\end{itemize}
\end{thm}
\begin{proof}
$A \subseteq D(f) \subseteq R \subseteq \mathbb{R}^{m}$、$S \subseteq \mathbb{R}_{\infty}^{n}$なる関数$f:D(f) \rightarrow S$が与えられたとき、その関数$f$がその集合$A$で連続であるとする。そこで、$\forall U \in \mathfrak{P}(S)$に対し、その集合$U$がその集合$S$で開集合であるなら、$\forall\mathbf{a} \in V\left( f^{- 1}|U \right) \cap A$に対し、$f\left( \mathbf{a} \right) \in U$が成り立つので、$\exists\varepsilon \in \mathbb{R}^{+}$に対し、$U\left( f\left( \mathbf{a} \right),\varepsilon \right) \cap S \subseteq U$が成り立つ。仮定と定理\ref{4.1.10.16}より$\exists\delta \in \mathbb{R}^{+}$に対し、$U\left( \mathbf{a},\delta \right) \cap D(f) \subseteq V\left( f^{- 1}|U\left( f\left( \mathbf{a} \right),\varepsilon \right) \cap S \right) \subseteq V\left( f^{- 1}|U \right)$が成り立つので、$A \subseteq D(f)$より$U\left( \mathbf{a},\delta \right) \cap A \subseteq V\left( f^{- 1}|U \right) \cap A$が成り立つ。\par
逆に、$\forall U \in \mathfrak{P}(S)$に対し、その集合$U$がその集合$S$で開集合であるなら、その集合$V\left( f^{- 1}|U \right) \cap A$もその集合$A$で開集合であるとする。$\forall\mathbf{a} \in A\forall\varepsilon \in \mathbb{R}^{+}$に対し、その集合$U\left( f\left( \mathbf{a} \right),\varepsilon \right) \cap S$がその集合$S$での開集合であるので、仮定よりその集合$V\left( f^{- 1}|U\left( f\left( \mathbf{a} \right),\varepsilon \right) \cap S \right) \cap A$もその集合$A$で開集合である。$f\left( \mathbf{a} \right) \in U\left( f\left( \mathbf{a} \right),\varepsilon \right) \cap S$が成り立つかつ、$\mathbf{a} \in A$も成り立つので、$\mathbf{a} \in V\left( f^{- 1}|U\left( f\left( \mathbf{a} \right),\varepsilon \right) \cap S \right) \cap A$も成り立つことから、$\exists\delta \in \mathbb{R}^{+}$に対し、$U\left( \mathbf{a},\delta \right) \cap A \subseteq V\left( f^{- 1}|U\left( f\left( \mathbf{a} \right),\varepsilon \right) \cap S \right) \cap A$が成り立つ、即ち、$U\left( \mathbf{a},\delta \right) \cap A \subseteq V\left( \left( f|A \right)^{- 1}|U\left( f|A\left( \mathbf{a} \right),\varepsilon \right) \cap S \right)$が成り立つので、定理\ref{4.1.10.16}よりその関数$f|A$はその集合$A$で連続である。よって、その関数$f$はその集合$A$で連続である。
\end{proof}
\begin{thm}\label{4.1.10.18}
$D(f) \subseteq R \subseteq \mathbb{R}^{m}$、$S \subseteq \mathbb{R}^{n}$なる関数$f = \left( f_{i} \right)_{i \in \varLambda_{n}}:D(f) \rightarrow S$が与えられたとき、$\forall\mathbf{a} \in D(f)$に対し、その関数$f$がその点$\mathbf{a}$で連続であるならそのときに限り、$\forall i \in \varLambda_{m}$に対し、その関数$f_{i}$がその点$\mathbf{a}$で連続である。\par
ここで、$S \subseteq \mathbb{R}^{n}$でなく$S \subseteq \mathbb{R}_{\infty}^{n}$と仮定していることに注意しよう。
\end{thm}
\begin{proof} 定理\ref{4.1.10.6}より明らかである。
\end{proof}
\begin{thm}\label{4.1.10.19}
$D(f) \subseteq R \subseteq \mathbb{R}^{l}$、$D(g) \subseteq S \subseteq \mathbb{R}^{m}$、$T \subseteq \mathbb{R}_{\infty}^{n}$なる2つの関数たち$f:D(f) \rightarrow S$、$g:D(g) \rightarrow T$が与えられたとき、$V(f) \subseteq D(g)$が成り立つとき、$\forall\mathbf{a} \in D(f)$に対し、その関数$f$がその点$\mathbf{a}$で連続でその関数$g$がその点$f\left( \mathbf{a} \right)$で連続であるなら、その関数$g \circ f$はその点$\mathbf{a}$で連続である。
\end{thm}
\begin{proof} 定理\ref{4.1.10.4}より明らかである。
\end{proof}
\begin{thebibliography}{50}
  \bibitem{1}
  杉浦光夫, 解析入門I, 東京大学出版社, 1985. 第34刷 p50-63 ISBN978-4-13-062005-5
\end{thebibliography}
\end{document}
