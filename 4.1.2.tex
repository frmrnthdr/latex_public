\documentclass[dvipdfmx]{jsarticle}
\setcounter{section}{1}
\setcounter{subsection}{1}
\usepackage{amsmath,amsfonts,amssymb,array,comment,mathtools,url,docmute}
\usepackage{longtable,booktabs,dcolumn,tabularx,mathtools,multirow,colortbl,xcolor}
\usepackage[dvipdfmx]{graphics}
\usepackage{bmpsize}
\usepackage{amsthm}
\usepackage{enumitem}
\setlistdepth{20}
\renewlist{itemize}{itemize}{20}
\setlist[itemize]{label=•}
\renewlist{enumerate}{enumerate}{20}
\setlist[enumerate]{label=\arabic*.}
\setcounter{MaxMatrixCols}{20}
\setcounter{tocdepth}{3}
\newcommand{\rotin}{\text{\rotatebox[origin=c]{90}{$\in $}}}
\newcommand{\amap}[6]{\text{\raisebox{-0.7cm}{\begin{tikzpicture} 
  \node (a) at (0, 1) {$\textstyle{#2}$};
  \node (b) at (#6, 1) {$\textstyle{#3}$};
  \node (c) at (0, 0) {$\textstyle{#4}$};
  \node (d) at (#6, 0) {$\textstyle{#5}$};
  \node (x) at (0, 0.5) {$\rotin $};
  \node (x) at (#6, 0.5) {$\rotin $};
  \draw[->] (a) to node[xshift=0pt, yshift=7pt] {$\textstyle{\scriptstyle{#1}}$} (b);
  \draw[|->] (c) to node[xshift=0pt, yshift=7pt] {$\textstyle{\scriptstyle{#1}}$} (d);
\end{tikzpicture}}}}
\newcommand{\twomaps}[9]{\text{\raisebox{-0.7cm}{\begin{tikzpicture} 
  \node (a) at (0, 1) {$\textstyle{#3}$};
  \node (b) at (#9, 1) {$\textstyle{#4}$};
  \node (c) at (#9+#9, 1) {$\textstyle{#5}$};
  \node (d) at (0, 0) {$\textstyle{#6}$};
  \node (e) at (#9, 0) {$\textstyle{#7}$};
  \node (f) at (#9+#9, 0) {$\textstyle{#8}$};
  \node (x) at (0, 0.5) {$\rotin $};
  \node (x) at (#9, 0.5) {$\rotin $};
  \node (x) at (#9+#9, 0.5) {$\rotin $};
  \draw[->] (a) to node[xshift=0pt, yshift=7pt] {$\textstyle{\scriptstyle{#1}}$} (b);
  \draw[|->] (d) to node[xshift=0pt, yshift=7pt] {$\textstyle{\scriptstyle{#2}}$} (e);
  \draw[->] (b) to node[xshift=0pt, yshift=7pt] {$\textstyle{\scriptstyle{#1}}$} (c);
  \draw[|->] (e) to node[xshift=0pt, yshift=7pt] {$\textstyle{\scriptstyle{#2}}$} (f);
\end{tikzpicture}}}}
\renewcommand{\thesection}{第\arabic{section}部}
\renewcommand{\thesubsection}{\arabic{section}.\arabic{subsection}}
\renewcommand{\thesubsubsection}{\arabic{section}.\arabic{subsection}.\arabic{subsubsection}}
\everymath{\displaystyle}
\allowdisplaybreaks[4]
\usepackage{vtable}
\theoremstyle{definition}
\newtheorem{thm}{定理}[subsection]
\newtheorem*{thm*}{定理}
\newtheorem{dfn}{定義}[subsection]
\newtheorem*{dfn*}{定義}
\newtheorem{axs}[dfn]{公理}
\newtheorem*{axs*}{公理}
\renewcommand{\headfont}{\bfseries}
\makeatletter
  \renewcommand{\section}{%
    \@startsection{section}{1}{\z@}%
    {\Cvs}{\Cvs}%
    {\normalfont\huge\headfont\raggedright}}
\makeatother
\makeatletter
  \renewcommand{\subsection}{%
    \@startsection{subsection}{2}{\z@}%
    {0.5\Cvs}{0.5\Cvs}%
    {\normalfont\LARGE\headfont\raggedright}}
\makeatother
\makeatletter
  \renewcommand{\subsubsection}{%
    \@startsection{subsubsection}{3}{\z@}%
    {0.4\Cvs}{0.4\Cvs}%
    {\normalfont\Large\headfont\raggedright}}
\makeatother
\makeatletter
\renewenvironment{proof}[1][\proofname]{\par
  \pushQED{\qed}%
  \normalfont \topsep6\p@\@plus6\p@\relax
  \trivlist
  \item\relax
  {
  #1\@addpunct{.}}\hspace\labelsep\ignorespaces
}{%
  \popQED\endtrivlist\@endpefalse
}
\makeatother
\renewcommand{\proofname}{\textbf{証明}}
\usepackage{tikz,graphics}
\usepackage[dvipdfmx]{hyperref}
\usepackage{pxjahyper}
\hypersetup{
 setpagesize=false,
 bookmarks=true,
 bookmarksdepth=tocdepth,
 bookmarksnumbered=true,
 colorlinks=false,
 pdftitle={},
 pdfsubject={},
 pdfauthor={},
 pdfkeywords={}}
\begin{document}
%\hypertarget{vector-ux884cux5217-ux8907ux7d20ux6570}{%
\subsection{行列 複素数}%\label{vector-ux884cux5217-ux8907ux7d20ux6570}}
%\hypertarget{vectorux7a7aux9593}{%
\subsubsection{vector空間}%\label{vectorux7a7aux9593}}
\begin{axs}[vector空間の公理]
集合$K$を体とする。空集合でない集合$V$に対し、加法$+ :V \times V \rightarrow V;\left( \mathbf{v},\mathbf{w} \right) \mapsto \mathbf{v} + \mathbf{w}$が与えられたとする。\par
このとき、次の条件たちを満たす集合$V$を体$K$上のvector空間、有向量空間、線形空間、線型空間という。
\begin{itemize}
\item
  集合$V$は加法について可換群$(V, + )$をなす。
\item
  $\forall k \in K\forall\mathbf{v} \in V$に対し、scalar倍$\cdot :K \times V \rightarrow V;\left( k,\mathbf{v} \right) \mapsto k\mathbf{v}$が定義されている。
\item
  $\forall k \in K\forall\mathbf{v},\mathbf{w} \in V$に対し、$k\left( \mathbf{v} + \mathbf{w} \right) = k\mathbf{v}\mathbf{+}k\mathbf{w}$が成り立つ。
\item
  $\forall k,l \in K\forall\mathbf{v} \in V$に対し、$(k + l)\mathbf{v} = k\mathbf{v}\mathbf{+}l\mathbf{v}$が成り立つ。
\item
  $\forall k,l \in K\forall\mathbf{v} \in V$に対し、$(kl)\mathbf{v} = k\left( l\mathbf{v} \right)$が成り立つ。
\item
  $\exists 1 \in K\forall\mathbf{v} \in V$に対し、$1\mathbf{v} = \mathbf{v}$が成り立つ。
\end{itemize}
体$K$上のvector空間$V$の元をvector、有向量などといいふつう$\mathbf{v}$など小文字太字で表される。特に加法の単位元であるvectorを零vectorといい、\textbf{0}と表す。一方、vector空間を考えるとき体$K$の元をscalar、無向量などという。
\end{axs}
\begin{thm}\label{4.1.2.1} 体$K$上のvector空間$V$において、次のことが成り立つ。
\begin{itemize}
\item
  $\forall k \in K\forall\mathbf{v},\mathbf{w} \in V$に対し、$k\left( \mathbf{v}-\mathbf{w} \right) = k\mathbf{v}-k\mathbf{w}$が成り立つ。
\item
  $\forall k,l \in K\forall\mathbf{v} \in V$に対し、$(k - l)\mathbf{v} = k\mathbf{v} - l\mathbf{v}$が成り立つ。
\item
  $\exists\mathbf{0} \in V\forall k \in K$に対し、$k\mathbf{0} = \mathbf{0}$が成り立つ。
\item
  $\exists 0 \in K\forall\mathbf{v} \in V$に対し、$0\mathbf{v} = \mathbf{0}$が成り立つ。
\item
  $\forall k \in K\forall\mathbf{v} \in V$に対し、$( - k)\mathbf{v} = k\left( - \mathbf{v} \right) = - k\mathbf{v}$が成り立つ。
\end{itemize}
\end{thm}
\begin{proof}
体$K$上のvector空間$V$を考えれば、$\forall k \in K\forall\mathbf{v},\mathbf{w} \in V$に対し、次のようになる。
\begin{align*}
k\left( \mathbf{v}-\mathbf{w} \right) + k\mathbf{w} - k\mathbf{w}&=k\left( \left( \mathbf{v} - \mathbf{w} \right) + \mathbf{w} \right) - k\mathbf{w}\\
&=k\mathbf{v} - k\mathbf{w}\\
&=k\left( \mathbf{v}-\mathbf{w} \right)\\
&= k\mathbf{v} - k\mathbf{w}
\end{align*}
$\forall k,l \in K\forall\mathbf{v} \in V$に対し、次のようになる。
\begin{align*}
(k - l)\mathbf{v} + l\mathbf{v} - l\mathbf{v}&=\left( (k - l) + l \right)\mathbf{v} - l\mathbf{v}\\
&=k\mathbf{v} - l\mathbf{v}\\
&=(k - l)\mathbf{v}\\
&= k\mathbf{v} - l\mathbf{v}
\end{align*}
$\exists\mathbf{0} \in V\forall k \in K$に対し、次のようになる。
\begin{align*}
k\mathbf{0} &= k\left( \mathbf{v}-\mathbf{v} \right)\\
&= k\mathbf{v}-k\mathbf{v} = \mathbf{0}
\end{align*}
$\exists 0 \in K\forall\mathbf{v} \in V$に対し、次のようになる。
\begin{align*}
0\mathbf{v} &= (k - k)\mathbf{v}\\
&= k\mathbf{v} - k\mathbf{v} = \mathbf{0}
\end{align*}
$\forall k \in K\forall\mathbf{v} \in V$に対し、次のようになる。
\begin{align*}
( - k)\mathbf{v} &= (0 - k)\mathbf{v}\\
&= 0\mathbf{v} - k\mathbf{v}\\
&= \mathbf{0} - k\mathbf{v} = - k\mathbf{v}
\end{align*}
$\forall k,l \in K\forall\mathbf{v},\mathbf{w} \in V\exists 0 \in K\exists\mathbf{0} \in V$に対し、次のようになる。
\begin{align*}
k\left( - \mathbf{v} \right) &= k\left( \mathbf{0} - \mathbf{v} \right)\\
&= k\mathbf{0} - k\mathbf{v}\\
&= \mathbf{0} - k\mathbf{v} = - k\mathbf{v}
\end{align*}
\end{proof}
%\hypertarget{ux4f53kux4e0aux306en-vector}{%
\subsubsection{体$K$上の$n$-vector}%\label{ux4f53kux4e0aux306en-vector}}
\begin{dfn}
体$K$の$n$つの積$K^{n}$の元、即ち、体$K$の元$a_{i}$の順序付けられた組$\left( a_{i} \right)_{i \in \varLambda_{n}} = \left( a_{1},a_{2},\cdots,a_{n} \right)$を考え集合$K^{n}$について次のように定義する。
\begin{itemize}
\item
  $\forall\left( a_{i} \right)_{i \in \varLambda_{n}},\left( b_{i} \right)_{i \in \varLambda_{n}} \in K^{n}$に対し、$\left( a_{i} \right)_{i \in \varLambda_{n}} + \left( b_{i} \right)_{i \in \varLambda_{n}} = \left( a_{i} + b_{i} \right)_{i \in \varLambda_{n}}$が成り立つ。
\item
  $\forall k \in K\forall\left( a_{i} \right)_{i \in \varLambda_{n}} \in K^{n}$に対し、${k\left( a_{i} \right)}_{i \in \varLambda_{n}} = \left( {ka}_{i} \right)_{i \in \varLambda_{n}}$が成り立つ。
\end{itemize}
\end{dfn}
\begin{thm}\label{4.1.2.2} 集合$K^{n}$はvector空間である。
\end{thm}
\begin{dfn}
組$\left( a_{i} \right)_{i \in \varLambda_{n}}$を体$K$上の$n$-vectorといい、$a_{i}$をこのvectorの第$i$成分、第$i$座標という。なお、その自然数$n$をそのvector空間の次元ということにする\footnote{ちゃんとした定義でいえば基底の個数を次元ということになります。ただ、これのwell define性の確認がやや大変なので、ここでは省略させていただきました。}。
\end{dfn}
\begin{proof}
体$K$の$n$つの積$K^{n}$の元、即ち体$K$の元$a_{i}$の順序付けられた組$\left( a_{i} \right)_{i \in \varLambda_{n}} = \left( a_{1},a_{2},\cdots,a_{n} \right)$が与えられたとき、集合$K^{n}$について定義より$\forall\left( a_{i} \right)_{i \in \varLambda_{n}},\left( b_{i} \right)_{i \in \varLambda_{n}} \in K^{n}$に対し、$\left( a_{i} \right)_{i \in \varLambda_{n}} + \left( b_{i} \right)_{i \in \varLambda_{n}} = \left( a_{i} + b_{i} \right)_{i \in \varLambda_{n}}$が成り立つ。このとき、体$K$が加法について群$(K, + )$をなし、$a_{i} + b_{i}$のみ着眼して考えると、各成分も加法について群$(K, + )$をなしているので、$K^{n}$は加法について群$\left( K^{n}, + \right)$をなす。\par
また、定義より$\forall k \in K\forall\left( a_{i} \right)_{i \in \varLambda_{n}} \in K^{n}$に対し、$k\left( a_{i} \right)_{i \in \varLambda_{n}} = \left( {ka}_{i} \right)_{i \in \varLambda_{n}}$が成り立ち${ka}_{i} \in K$が成り立つので、$\left( {ka}_{i} \right)_{i \in \varLambda_{n}} \in K^{n}$であり写像$\cdot :K \times K^{n} \rightarrow K^{n};\left( k,\left( a_{i} \right)_{i \in \varLambda_{n}} \right) \mapsto \left( {ka}_{i} \right)_{i \in \varLambda_{n}}$が定義される。\par
さらに、$\forall k,l \in K\forall\left( a_{i} \right)_{i \in \varLambda_{n}},\left( b_{i} \right)_{i \in \varLambda_{n}} \in K^{n}$に対し、次のようになる。
\begin{align*}
k\left( \left( a_{i} \right)_{i \in \varLambda_{n}} + \left( b_{i} \right)_{i \in \varLambda_{n}} \right) &= k\left( a_{i} + b_{i} \right)_{i \in \varLambda_{n}}\\
&= \left( k\left( a_{i} + b_{i} \right) \right)_{i \in \varLambda_{n}}\\
&= \left( ka_{i} + kb_{i} \right)_{i \in \varLambda_{n}}\\
&= \left( ka_{i} \right)_{i \in \varLambda_{n}}\mathbf{+}\left( kb_{i} \right)_{i \in \varLambda_{n}}\\
&= k\left( a_{i} \right)_{i \in \varLambda_{n}}\mathbf{+}k\left( b_{i} \right)_{i \in \varLambda_{n}}\\
(k + l)\left( a_{i} \right)_{i \in \varLambda_{n}} &= \left( (k + l)a_{i} \right)_{i \in \varLambda_{n}}\\
&= \left( ka_{i} + la_{i} \right)_{i \in \varLambda_{n}}\\
&= \left( ka_{i} \right)_{i \in \varLambda_{n}}\mathbf{+}\left( la_{i} \right)_{i \in \varLambda_{n}}\\
&= k\left( a_{i} \right)_{i \in \varLambda_{n}}\mathbf{+}l\left( a_{i} \right)_{i \in \varLambda_{n}}\\
(kl)\left( a_{i} \right)_{i \in \varLambda_{n}} &= \left( (kl)a_{i} \right)_{i \in \varLambda_{n}}\\
&= \left( k\left( la_{i} \right) \right)_{i \in \varLambda_{n}}\\
&= {k\left( la_{i} \right)}_{i \in \varLambda_{n}}\\
&= k\left( l\left( a_{i} \right)_{i \in \varLambda_{n}} \right)\\
&1\left( a_{i} \right)_{i \in \varLambda_{n}} = \left( 1a_{i} \right)_{i \in \varLambda_{n}} = \left( a_{i} \right)_{i \in \varLambda_{n}}
\end{align*}\par
以上より、その集合$K^{n}$はvector空間の定義を満たしているので、vector空間である。
\end{proof}
\begin{thm}\label{4.1.2.3}
体$K$上のvector空間$K^{n}$において、$i \in \varLambda_{n}$なるvectors$\mathbf{e}_{i}$を次式のようにおく。
\begin{align*}
\mathbf{e}_{i} = \left( \delta_{ij} \right)_{j \in \varLambda_{n}},\ \ \delta_{ij} = \left\{ \begin{matrix}
1 & \mathrm{if} & i = j \\
0 & \mathrm{if} & i \neq j \\
\end{matrix} \right.\ 
\end{align*}
このとき、次式が成り立つ\footnote{この組$\left\langle \mathbf{e}_{i} \right\rangle_{i\in \varLambda_{n} }$を自然な正規直交基底とかいったりするらしいです。これについて詳しくは線形代数学のほうを参照していただければ…。}。
\begin{align*}
\sum_{i \in \varLambda_{n}} {a_{i}\mathbf{e}_{i}} = \mathbf{0} \Leftrightarrow \forall i \in \varLambda_{n}\left[ a_{i} = 0 \right]
\end{align*}
\end{thm}
\begin{proof}
体$K$上のvector空間$K^{n}$において、$\forall i \in \varLambda_{n}$に対し、$c_{i} = 0$が成り立つなら、$\forall i \in \varLambda_{n}$に対し$c_{i}\mathbf{e}_{i} = \mathbf{0}$が成り立つので、明らかに$\sum_{i \in \varLambda_{n}} {c_{i}\mathbf{e}_{i}} = \mathbf{0}$が成り立つ。逆に、$a_{i} \in K$なる元々$a_{i}$を用いて式$\sum_{i \in \varLambda_{n}} {a_{i}\mathbf{e}_{i}} = \mathbf{0}$について考えよう。$\exists i \in \varLambda_{n}$に対し、$a_{i} \neq 0$と仮定するとき、$\varLambda' = \left\{ i \in \varLambda_{n} \middle| a_{i} \neq 0 \right\}$とおくと、この集合は空集合でなく、次のようになる。
\begin{align*}
\sum_{i \in \varLambda_{n}} {a_{i}\mathbf{e}_{i}} &= \sum_{i \in \varLambda_{n} \setminus \varLambda' \cup \varLambda'} {a_{i}\mathbf{e}_{i}}\\
&= \sum_{i \in \varLambda_{n} \setminus \varLambda'} {a_{i}\mathbf{e}_{i}} + \sum_{i \in \varLambda'} {a_{i}\mathbf{e}_{i}}\\
&= 0\sum_{i \in \varLambda_{n} \setminus \varLambda'} \mathbf{e}_{i} + \sum_{i \in \varLambda'} {a_{i}\mathbf{e}_{i}}\\
&= \mathbf{0} + \sum_{i \in \varLambda'} {a_{i}\mathbf{e}_{i}}\\
&= \sum_{i \in \varLambda'} {a_{i}\mathbf{e}_{i}}\\
&= \sum_{i \in \varLambda'} {a_{i}\left( \delta_{ij} \right)_{i \in \varLambda_{n}}}\\
&= \sum_{i \in \varLambda'} \left( a_{i}\delta_{ij} \right)_{j \in \varLambda_{n}}
\end{align*}
ここで、総和をとられる$i \in \varLambda'$なる各vectors$\left( a_{i}\delta_{ij} \right)_{j \in \varLambda_{n}}$の添数$i$たちは互いに異なり同じ成分同士の和がとられることはなく$i \in \varLambda'$より$a_{i} \neq 0$が成り立つので、
\begin{align*}
\sum_{i \in \varLambda_{n}} {a_{i}\mathbf{e}_{i}} = \left( \left\{ \begin{matrix}
a_{i} & \mathrm{if} & i \in \varLambda' \\
0 & \mathrm{if} & i \notin \varLambda' \\
\end{matrix} \right.\  \right)_{i \in \varLambda_{n}}
\end{align*}
$i \in \varLambda_{n}$に対する各vectors$\left( \left\{ \begin{matrix}
a_{i} & \mathrm{if} & i \in \varLambda' \\
0 & \mathrm{if} & i \notin \varLambda' \\
\end{matrix} \right.\  \right)_{i \in \varLambda_{n}}$の成分のうち$a_{i} \neq 0$なる成分が存在するので、vector$\sum_{i \in \varLambda_{n}} {a_{i}\mathbf{e}_{i}}$は零vector$\mathbf{0} $ではない。しかし、これは仮定に矛盾するので、$\forall i \in \varLambda_{n}$に対し、$a_{i} = 0$が成り立つ。
\end{proof}
\begin{thm}\label{4.1.2.4}
$\forall\left( a_{i} \right)_{i \in \varLambda_{n}} \in K^{n}$に対し次式が成り立つ。
\begin{align*}
\left( a_{i} \right)_{i \in \varLambda_{n}} = \sum_{i \in \varLambda_{n}} {a_{i}\mathbf{e}_{i}}
\end{align*}
そのvector$\mathbf{v}$が$\sum_{i \in \varLambda_{n}} {a_{i}\mathbf{e}_{i}}$と書かれることができるとき、そのvector$\mathbf{v}$をそのvector$\left( a_{i} \right)_{i \in \varLambda_{n}}$に書きかえることをそのvector$\mathbf{v}$をそのvector$\left( a_{i} \right)_{i \in \varLambda_{n}}$と成分表示するという。
\end{thm}
\begin{proof}
体$K$上のvector空間$K^{n}$において、$\forall i \in \varLambda_{n}$に対しvectors$\mathbf{e}_{i}$を次式のようにおく。
\begin{align*}
\mathbf{e}_{i} = \left( \delta_{ij} \right)_{j \in \varLambda_{n}},\ \ \delta_{ij} = \left\{ \begin{matrix}
1 & \mathrm{if} & i = j \\
0 & \mathrm{if} & i \neq j \\
\end{matrix} \right.\ 
\end{align*}
このとき、$\forall\left( a_{i} \right)_{i \in \varLambda_{n}} \in K^{n}$に対し、次のようになる。
\begin{align*}
\left( a_{i} \right)_{i \in \varLambda_{n}} &= \sum_{i \in \varLambda_{n}} \left( a_{i}\delta_{ij} \right)_{j \in \varLambda_{n}}\\
&= \sum_{i \in \varLambda_{n}} {a_{i}\left( \delta_{ij} \right)_{j \in \varLambda_{n}}}\\
&= \sum_{i \in \varLambda_{n}} {a_{i}\mathbf{e}_{i}}
\end{align*}
\end{proof}
%\hypertarget{ux5ea7ux6a19}{%
\subsubsection{座標}%\label{ux5ea7ux6a19}}
\begin{thm}\label{4.1.2.5}
$\forall\mathbf{v} \in K^{n}$なる元$\mathbf{v}$は$i \in \varLambda_{n}$なるその体$K$の元々$a_{i}$を用いて次式のように表されるのであった。
\begin{align*}
\mathbf{v} = \sum_{i \in \varLambda_{n}} {a_{i}\mathbf{e}_{i}}
\end{align*}
$i \in \varLambda_{n}$なるこのときのこれらの元々$a_{i}$は一意的である。これは背理法によって示される。
\end{thm}
\begin{dfn}
このvector$\left( a_{i} \right)_{i \in \varLambda_{i}}$を標準直交基底によるそのvector$\mathbf{v}$の座標、座標vector、成分、成分vector、位置、位置vectorなどといい$a_{i}$をそのvector$\mathbf{v}$の第$i$成分、第$i$座標という。
\end{dfn}
\begin{proof}
$\forall\mathbf{v} \in K^{n}$に対し、$i \in \varLambda_{n}$なるその体$K$の元々$a_{i}$を用いて次式のように表されるとする。
\begin{align*}
\mathbf{v} = \sum_{i \in \varLambda_{n}} {a_{i}\mathbf{e}_{i}}
\end{align*}
$i \in \varLambda_{n}$なるこのときのこれらの元々$a_{i}$は一意的でないと仮定する。このとき、$\exists i \in \varLambda_{n}$に対し、$a_{i} \neq b_{i}$が成り立ち$b_{i} \in K$なる元々$b_{i}$を用いて次式のように表されることができる。
\begin{align*}
\mathbf{v} = \sum_{i \in \varLambda_{n}} {b_{i}\mathbf{e}_{i}}
\end{align*}
この式を前述した式$\mathbf{v} = \sum_{i \in \varLambda_{n}} {a_{i}\mathbf{e}_{i}}$から引けば、次式が成り立つ。
\begin{align*}
\mathbf{0} &= \mathbf{v} - \mathbf{v}\\
&= \sum_{i \in \varLambda_{n}} {a_{i}\mathbf{e}_{i}} - \sum_{i \in \varLambda_{n}} {b_{i}\mathbf{e}_{i}}\\
&= \sum_{i \in \varLambda_{n}} {\left( a_{i} - b_{i} \right)\mathbf{e}_{i}}
\end{align*}
ここで、上記の定理より次式が成り立つ。
\begin{align*}
\forall i \in \varLambda_{n}\left[ a_{i} - b_{i} = 0 \right] &\Leftrightarrow \forall i \in \varLambda_{n}\left[ a_{i} = b_{i} \right]\\
&\Leftrightarrow \neg\exists i \in \varLambda_{n}\left[ a_{i} \neq b_{i} \right]
\end{align*}
これは仮定の$\exists i \in \varLambda_{n}$に対し、$a_{i} \neq b_{i}$が成り立つことに矛盾する。
\end{proof}
\begin{dfn}
集合$\mathbb{R}^{n}$の元を点といい$\forall\mathbf{r} \in \mathbb{R}^{n}$なる点$\mathbf{r}$が$\forall i \in \varLambda_{n}$に対しそれらのvectors$\mathbf{e}_{i}$を用いて$\sum_{i \in \varLambda_{n}} {x_{i}\mathbf{e}_{i}}$と書かれることができるとき、写像$\varphi:\mathbf{r} \mapsto \left( x_{i} \right)_{i \in \varLambda_{n}}$を$n$次元直交座標系、または、単に直交座標系などという。$\forall\mathbf{r} \in \mathbb{R}^{n}$なる点$\mathbf{r}$のその基底が標準直交基底である座標$\left( x_{i} \right)_{i \in \varLambda_{n}}$をその点$\mathbf{r}$の$n$次元直交座標、または単に、直交座標などといいしばしば$\overrightarrow{r}$などの矢印つきの小文字などを用いるときがある。
\end{dfn}
\begin{dfn}
特に、2次元直交座標系なら$x_{1}$、$x_{2}$をそれぞれ$x$、$y$などと表しこの座標系を$xy$平面ともいい、3次元直交座標系なら$x_{1}$、$x_{2}$、$x_{3}$をそれぞれ$x$、$y$、$z$などと表しこの座標系を$xyz$空間ともいい、任意の実数$x_{j}$を用いて、次式で表される集合$P$を$j = 1$なら$yz$平面、$j = 2$なら$zx$平面、$j = 3$なら$xy$平面という。
\begin{align*}
P = \left\{ \left( x_{i} \right)_{i \in \varLambda_{3}} \in \mathbb{R}^{3} \middle| x_{j} = 0 \right\}
\end{align*}
\end{dfn}
%\hypertarget{ux884cux5217}{%
\subsubsection{行列}%\label{ux884cux5217}}
\begin{dfn}
1つの体を$K$と、2つの空集合でない集合たちを$I$、$J$とおくとき、その2つの集合たち$I$、$J$の直積$I \times J$から体$K$への写像$a:I \times J \rightarrow K;(i,j) \mapsto a_{ij}$によって得られる体$K$の元$a_{ij}$全体の順序付けられた組を体$K$における$(I,J)$型の行列といい、$\left( a_{ij} \right)_{(i,j) \in I \times J}$、または単に、$\left( a_{ij} \right)$と表される。その2つの集合たち$I$、$J$がそれぞれ${\#}I$つ、${\#}J$つの元からなる有限集合であるとき、$(I,J)$型の行列を$\left( {\#}I,{\#}J \right)$型の行列、${\#}I \times {\#}J$型の行列などという。これに対比して体$K$の元をscalar、無向量などという。特に、その2つの集合たち$I$、$J$は、写像$f$を適切に定めれば、体$K$の元$a_{ij}$を定めるのに任意性があるので、自然数$m$、$n$を用いてその2つの集合たち$I$、$J$を次式のようにおいても一般性が失われることはない。
\begin{align*}
\left\{ \begin{matrix}
\varLambda_{m} = \left\{ i \in \mathbb{N} \middle| 1 \leq i \leq m \right\} = I \\
\varLambda_{n} = \left\{ j \in \mathbb{N} \middle| 1 \leq j \leq m \right\} = J \\
\end{matrix} \right.\ 
\end{align*}
このとき、$\left( \varLambda_{m},\varLambda_{n} \right)$型の行列は$(m,n)$型の行列、$m \times n$型の行列にあたり、$\left( a_{ij} \right)_{(i,j) \in \varLambda_{m} \times \varLambda_{n}}$を次式のようにも書き、その元$a_{ij}$が明示的に書かれているものを成分表示された行列などといい、その元$a_{ij}$は集合$K$に属し、これを$(i,j)$成分という。また、これが属する集合を$M_{mn}(K)$、$K^{m \times n}$と書き、特に$m = n$ならば単に$M_{n}(K)$、$K^{n^{2}}$とも書く。
\begin{align*}
\left( a_{ij} \right)_{(i,j) \in \varLambda_{m} \times \varLambda_{n}} = A_{mn} = \left( a_{ij} \right)_{1 \leq i \leq m,1 \leq j \leq n} = \begin{pmatrix}
\begin{matrix}
a_{11} & a_{12} \\
a_{21} & a_{22} \\
\end{matrix} & \begin{matrix}
\cdots & a_{1n} \\
\cdots & a_{2n} \\
\end{matrix} \\
\begin{matrix}
 \vdots & \vdots \\
a_{m1} & a_{m2} \\
\end{matrix} & \begin{matrix}
 \ddots & \vdots \\
\cdots & a_{mn} \\
\end{matrix} \\
\end{pmatrix}
\end{align*}\par
以下、行列たち$A_{mn}$、$B_{mn}$、$C_{mn}$がそれぞれ$\left( a_{ij} \right)_{(i,j) \in \varLambda_{m} \times \varLambda_{n}}$、$\left( b_{ij} \right)_{(i,j) \in \varLambda_{m} \times \varLambda_{n}}$、$\left( c_{ij} \right)_{(i,j) \in \varLambda_{m} \times \varLambda_{n}}$と成分表示されるとする。
\end{dfn}
\begin{dfn}
$\forall\left( a_{ij} \right)_{(i,j) \in \varLambda_{m} \times \varLambda_{n}},\left( b_{ij} \right)_{(i,j) \in \varLambda_{m} \times \varLambda_{n}} \in M_{mn}(K)\forall k,l \in K$に対し、次式のように定義する。
\begin{align*}
k\left( a_{ij} \right)_{(i,j) \in \varLambda_{m} \times \varLambda_{n}} + l\left( b_{ij} \right)_{(i,j) \in \varLambda_{m} \times \varLambda_{n}} = \left( ka_{ij} + lb_{ij} \right)_{(i,j) \in \varLambda_{m} \times \varLambda_{n}}
\end{align*}
このとき、行列はvectorとなり、$\exists\varLambda \in \left\{ \varLambda_{m},\varLambda_{n} \right\}$に対し、${\#}\varLambda = 1$となれば、即ち、その2つの集合たち$\varLambda_{m}$、$\varLambda_{n}$どちらかまたは両方とも元の個数が$1$になっても行列$\begin{pmatrix}
\begin{matrix}
a_{11} & a_{12} \\
a_{21} & a_{22} \\
\end{matrix} & \begin{matrix}
\cdots & a_{1n} \\
\cdots & a_{2n} \\
\end{matrix} \\
\begin{matrix}
 \vdots & \vdots \\
a_{m1} & a_{m2} \\
\end{matrix} & \begin{matrix}
 \ddots & \vdots \\
\cdots & a_{mn} \\
\end{matrix} \\
\end{pmatrix}$はvectorであり、$\left( \varLambda_{1},\varLambda_{n} \right)$型の行列$\begin{pmatrix}
a_{11} & a_{12} & \cdots & a_{1n} \\
\end{pmatrix}$、$\left( \varLambda_{m},\varLambda_{1} \right)$型の行列$\begin{pmatrix}
\begin{matrix}
a_{11} \\
a_{21} \\
\end{matrix} \\
\begin{matrix}
 \vdots \\
a_{m1} \\
\end{matrix} \\
\end{pmatrix}$をそれぞれ$n$-行vector、$m$-列vectorという。また、行列$\begin{pmatrix}
\begin{matrix}
a_{11} & a_{12} \\
a_{21} & a_{22} \\
\end{matrix} & \begin{matrix}
\cdots & a_{1n} \\
\cdots & a_{2n} \\
\end{matrix} \\
\begin{matrix}
 \vdots & \vdots \\
a_{m1} & a_{m2} \\
\end{matrix} & \begin{matrix}
 \ddots & \vdots \\
\cdots & a_{mn} \\
\end{matrix} \\
\end{pmatrix}$から取り出されたvector$\begin{pmatrix}
a_{i1} & a_{i2} & \cdots & a_{in} \\
\end{pmatrix}$、vector$\begin{pmatrix}
\begin{matrix}
a_{1j} \\
a_{2j} \\
\end{matrix} \\
\begin{matrix}
 \vdots \\
a_{mj} \\
\end{matrix} \\
\end{pmatrix}$をそれぞれその行列$A_{mn}$の第$i$行、第$j$列という。
\end{dfn}
\begin{dfn}
次式のように定義される行列$O_{mn}$を$\left( \varLambda_{m},\varLambda_{n} \right)$型の零行列、$(m,n)$型の零行列、$m \times n$型の零行列といい、特に$m = n$ならば$n$次零行列といい、単に$O_{n}$とも表す。
\begin{align*}
O_{mn} = (0)_{(i,j) \in \varLambda_{n} \times \varLambda_{n}}
\end{align*}
\end{dfn}
\begin{thm}\label{4.1.2.7}
$\forall k,l \in K\forall A_{mn},B_{mn},C_{mn} \in M_{mn}(K)$に対し、体$K$上の行列について、次のことが成り立つ。
\begin{itemize}
\item
  $\exists 1 \in K\forall A_{mn} \in M_{mn}(K)$に対し、$1A_{mn} = A_{mn}$が成り立つ。
\item
  $\exists 0 \in K\exists O_{mn} \in M_{mn}(K)\forall A_{mn} \in M_{mn}(K)$に対し、$0A_{mn} = O_{mn}$が成り立つ。
\item
  $\forall k,l \in K\forall A_{mn} \in M_{mn}(K)$に対し、$(k + l)A_{mn} = kA_{mn} + lA_{mn}$が成り立つ。
\item
  $\forall k \in K\forall A_{mn},B_{mn} \in M_{mn}(K)$に対し、$k\left( A_{mn} + B_{mn} \right) = kA_{mn} + kB_{mn}$が成り立つ。
\item
  $\forall k,l \in K\forall A_{mn} \in M_{mn}(K)$に対し、$k\left( lA_{mn} \right) = (kl)A_{mn}$が成り立つ。
\item
  $\forall A_{mn},B_{mn} \in M_{mn}(K)$に対し、$A_{mn} + B_{mn} = B_{mn} + A_{mn}$が成り立つ。
\item
  $\forall A_{mn},B_{mn},C_{mn} \in M_{mn}(K)$に対し、$\left( A_{mn} + B_{mn} \right) + C_{mn} = A_{mn} + \left( B_{mn} + C_{mn} \right)$が成り立つ。
\item
  $\exists O_{mn} \in M_{mn}(K)\forall A_{mn} \in M_{mn}(K)$に対し、$A_{mn} + O_{mn} = A_{mn}$が成り立つ。
\item
  $\forall A_{mn} \in M_{mn}(K)\exists - A_{mn} \in M_{mn}(K)$に対し、$A_{mn} - A_{mn} = O_{mn}$が成り立つ。
\end{itemize}
\end{thm}
\begin{proof}
行列たち$A_{mn}$、$B_{mn}$、$C_{mn}$がそれぞれ$\left( a_{ij} \right)_{(i,j) \in \varLambda_{m} \times \varLambda_{n}}$、$\left( b_{ij} \right)_{(i,j) \in \varLambda_{m} \times \varLambda_{n}}$、$\left( c_{ij} \right)_{(i,j) \in \varLambda_{m} \times \varLambda_{n}}$と成分表示されることができるとすると、$\exists 1 \in K\forall A_{mn} \in M_{mn}(K)$に対し、次のようになる。
\begin{align*}
1A_{mn} &= 1\left( a_{ij} \right)_{(i,j) \in \varLambda_{m} \times \varLambda_{n}}\\
&= \left( 1a_{ij} \right)_{(i,j) \in \varLambda_{m} \times \varLambda_{n}}\\
&= \left( a_{ij} \right)_{(i,j) \in \varLambda_{m} \times \varLambda_{n}} = A_{mn}
\end{align*}
$\exists 0 \in K\exists O_{mn} \in M_{mn}(K)\forall A_{mn} \in M_{mn}(K)$に対し、次のようになる。
\begin{align*}
0A_{mn} &= 0\left( a_{ij} \right)_{(i,j) \in \varLambda_{m} \times \varLambda_{n}}\\
&= \left( 0a_{ij} \right)_{(i,j) \in \varLambda_{m} \times \varLambda_{n}}\\
&= (0)_{(i,j) \in \varLambda_{m} \times \varLambda_{n}} = O_{mn}
\end{align*}
$\forall k,l \in K\forall A_{mn} \in M_{mn}(K)$に対し、次のようになる。
\begin{align*}
(k + l)A_{mn} &= (k + l)\left( a_{ij} \right)_{(i,j) \in \varLambda_{m} \times \varLambda_{n}}\\
&= \left( (k + l)a_{ij} \right)_{(i,j) \in \varLambda_{m} \times \varLambda_{n}}\\
&= \left( ka_{ij} + la_{ij} \right)_{(i,j) \in \varLambda_{m} \times \varLambda_{n}}\\
&= \left( ka_{ij} \right)_{(i,j) \in \varLambda_{m} \times \varLambda_{n}} + \left( la_{ij} \right)_{(i,j) \in \varLambda_{m} \times \varLambda_{n}}\\
&= {k\left( a_{ij} \right)}_{(i,j) \in \varLambda_{m} \times \varLambda_{n}} + l\left( a_{ij} \right)_{(i,j) \in \varLambda_{m} \times \varLambda_{n}}\\
&= kA_{mn} + lA_{mn}
\end{align*}
$\forall k \in K\forall A_{mn},B_{mn} \in M_{mn}(K)$に対し、次のようになる。
\begin{align*}
k\left( A_{mn} + B_{mn} \right) &= k\left( \left( a_{ij} \right)_{(i,j) \in \varLambda_{m} \times \varLambda_{n}} + \left( b_{ij} \right)_{(i,j) \in \varLambda_{m} \times \varLambda_{n}} \right)\\
&= k\left( a_{ij} + b_{ij} \right)_{(i,j) \in \varLambda_{m} \times \varLambda_{n}}\\
&= \left( k\left( a_{ij} + b_{ij} \right) \right)_{(i,j) \in \varLambda_{m} \times \varLambda_{n}}\\
&= \left( ka_{ij} + kb_{ij} \right)_{(i,j) \in \varLambda_{m} \times \varLambda_{n}}\\
&= \left( ka_{ij} \right)_{(i,j) \in \varLambda_{m} \times \varLambda_{n}} + \left( kb_{ij} \right)_{(i,j) \in \varLambda_{m} \times \varLambda_{n}}\\
&= k\left( a_{ij} \right)_{(i,j) \in \varLambda_{m} \times \varLambda_{n}} + k\left( b_{ij} \right)_{(i,j) \in \varLambda_{m} \times \varLambda_{n}}\\
&= kA_{mn} + kB_{mn}
\end{align*}
$\forall k,l \in K\forall A_{mn} \in M_{mn}(K)$に対し、次のようになる。
\begin{align*}
k\left( lA_{mn} \right) &= k\left( l\left( a_{ij} \right)_{(i,j) \in \varLambda_{m} \times \varLambda_{n}} \right)\\
&= k\left( la_{ij} \right)_{(i,j) \in \varLambda_{m} \times \varLambda_{n}}\\
&= \left( k\left( la_{ij} \right) \right)_{(i,j) \in \varLambda_{m} \times \varLambda_{n}}\\
&= \left( (kl)a_{ij} \right)_{(i,j) \in \varLambda_{m} \times \varLambda_{n}}\\
&= (kl)\left( a_{ij} \right)_{(i,j) \in \varLambda_{m} \times \varLambda_{n}} = (kl)A_{mn}
\end{align*}
$\forall A_{mn},B_{mn} \in M_{mn}(K)$に対し、次のようになる。
\begin{align*}
A_{mn} + B_{mn} &= \left( a_{ij} \right)_{(i,j) \in \varLambda_{m} \times \varLambda_{n}} + \left( b_{ij} \right)_{(i,j) \in \varLambda_{m} \times \varLambda_{n}}\\
&= \left( a_{ij} + b_{ij} \right)_{(i,j) \in \varLambda_{m} \times \varLambda_{n}}\\
&= \left( b_{ij} + a_{ij} \right)_{(i,j) \in \varLambda_{m} \times \varLambda_{n}}\\
&= \left( b_{ij} \right)_{(i,j) \in \varLambda_{m} \times \varLambda_{n}} + \left( a_{ij} \right)_{(i,j) \in \varLambda_{m} \times \varLambda_{n}}\\
&= B_{mn} + A_{mn}
\end{align*}
$\forall A_{mn},B_{mn},C_{mn} \in M_{mn}(K)$に対し、次のようになる。
\begin{align*}
\left( A_{mn} + B_{mn} \right) + C_{mn} &= \left( \left( a_{ij} \right)_{(i,j) \in \varLambda_{m} \times \varLambda_{n}} + \left( b_{ij} \right)_{(i,j) \in \varLambda_{m} \times \varLambda_{n}} \right) + \left( c_{ij} \right)_{(i,j) \in \varLambda_{m} \times \varLambda_{n}}\\
&= \left( a_{ij} + b_{ij} \right)_{(i,j) \in \varLambda_{m} \times \varLambda_{n}} + \left( c_{ij} \right)_{(i,j) \in \varLambda_{m} \times \varLambda_{n}}\\
&= \left( \left( a_{ij} + b_{ij} \right) + c_{ij} \right)_{(i,j) \in \varLambda_{m} \times \varLambda_{n}}\\
&= \left( a_{ij} + \left( b_{ij} + c_{ij} \right) \right)_{(i,j) \in \varLambda_{m} \times \varLambda_{n}}\\
&= \left( a_{ij} \right)_{(i,j) \in \varLambda_{m} \times \varLambda_{n}} + \left( b_{ij} + c_{ij} \right)_{(i,j) \in \varLambda_{m} \times \varLambda_{n}}\\
&= \left( a_{ij} \right)_{(i,j) \in \varLambda_{m} \times \varLambda_{n}} + \left( \left( b_{ij} \right)_{(i,j) \in \varLambda_{m} \times \varLambda_{n}} + \left( c_{ij} \right)_{(i,j) \in \varLambda_{m} \times \varLambda_{n}} \right)\\
&= A_{mn} + \left( B_{mn} + C_{mn} \right)
\end{align*}
$\exists O_{mn} \in M_{mn}(K)\forall A_{mn} \in M_{mn}(K)$に対し、次のようになる。
\begin{align*}
A_{mn} + O_{mn} &= \left( a_{ij} \right)_{(i,j) \in \varLambda_{m} \times \varLambda_{n}} + (0)_{(i,j) \in \varLambda_{m} \times \varLambda_{n}}\\
&= \left( a_{ij} + 0 \right)_{(i,j) \in \varLambda_{m} \times \varLambda_{n}}\\
&= \left( a_{ij} \right)_{(i,j) \in \varLambda_{m} \times \varLambda_{n}} = A_{mn}
\end{align*}
$\forall A_{mn} \in M_{mn}(K)\exists - A_{mn} \in M_{mn}(K)$に対し、次のようになる。
\begin{align*}
A_{mn} - A_{mn} &= \left( a_{ij} \right)_{(i,j) \in \varLambda_{m} \times \varLambda_{n}} + \left( - \left( a_{ij} \right)_{(i,j) \in \varLambda_{m} \times \varLambda_{n}} \right)\\
&= \left( a_{ij} \right)_{(i,j) \in \varLambda_{m} \times \varLambda_{n}} + \left( - a_{ij} \right)_{(i,j) \in \varLambda_{m} \times \varLambda_{n}}\\
&= \left( a_{ij} + \left( - a_{ij} \right) \right)_{(i,j) \in \varLambda_{m} \times \varLambda_{n}}\\
&= \left( a_{ij} - a_{ij} \right)_{(i,j) \in \varLambda_{m} \times \varLambda_{n}}\\
&= (0)_{(i,j) \in \varLambda_{m} \times \varLambda_{n}} = O_{mn}
\end{align*}
\end{proof}
\begin{dfn}
2つの行列$A_{lm}$、$B_{mn}$に対して次式のように行列の積を定義する。
\begin{align*}
A_{lm}B_{mn} = \left( \sum_{h \in \varLambda_{m}} {a_{ih}b_{hj}} \right)_{(i,j) \in \varLambda_{l} \times \varLambda_{n}}
\end{align*}
\end{dfn}
\begin{thm}\label{4.1.2.8} 行列の積について次のことが成り立つ。
\begin{itemize}
\item
  $\forall A_{lm} \in M_{lm}(K)\forall B_{mn} \in M_{mn}(K)\forall C_{no} \in M_{no}(K)$に対し、$A_{lm}\left( B_{mn}C_{no} \right) = \left( A_{lm}B_{mn} \right)C_{no}$が成り立つ。
\item
  $\forall A_{lm} \in M_{lm}(K)\forall B_{mn},C_{mn} \in M_{mn}(K)$に対し、$A_{lm}\left( B_{mn} + C_{mn} \right) = A_{lm}B_{mn} + A_{lm}C_{mn}$が成り立つ。
\item
  $\forall A_{lm},B_{lm} \in M_{lm}(K)\forall C_{mn} \in M_{mn}(K)$に対し、$\left( A_{lm} + B_{lm} \right)C_{mn} = A_{lm}C_{mn} + B_{lm}C_{mn}$が成り立つ。
\item
  $\forall k \in K\forall A_{lm} \in M_{lm}(K)\forall B_{mn} \in M_{mn}(K)$に対し、$A_{lm}\left( kB_{mn} \right) = \left( kA_{lm} \right)B_{mn} = k\left( A_{lm}B_{mn} \right)$が成り立つ。
\end{itemize}
\end{thm}\par
以上より、3つの行列同士の積、2つの行列同士の積の$k$倍は結合的であり、行列同士では積が和に対して右からも左からも分配的であるが、2つの行列同士の積は必ずしも可換的であるとは限らない。
\begin{proof}
行列たち$A_{mn}$、$B_{mn}$、$C_{mn}$がそれぞれ$\left( a_{ij} \right)_{(i,j) \in \varLambda_{m} \times \varLambda_{n}}$、$\left( b_{ij} \right)_{(i,j) \in \varLambda_{m} \times \varLambda_{n}}$、$\left( c_{ij} \right)_{(i,j) \in \varLambda_{m} \times \varLambda_{n}}$と成分表示されることができるとすると、$\forall A_{lm} \in M_{lm}(K)\forall B_{mn} \in M_{mn}(K)\forall C_{no} \in M_{no}(K)$に対し、次のようになる。
\begin{align*}
A_{lm}\left( B_{mn}C_{no} \right) &= \left( a_{ij} \right)_{(i,j) \in \varLambda_{l} \times \varLambda_{m}}\left( \left( b_{ij} \right)_{(i,j) \in \varLambda_{m} \times \varLambda_{n}}\left( c_{ij} \right)_{(i,j) \in \varLambda_{n} \times \varLambda_{o}} \right)\\
&= \left( a_{ij} \right)_{(i,j) \in \varLambda_{l} \times \varLambda_{m}}\left( \sum_{h \in \varLambda_{n}} {b_{ih}c_{hj}} \right)_{(i,j) \in \varLambda_{m} \times \varLambda_{o}}\\
&= \left( \sum_{g \in \varLambda_{m}} a_{ig}\sum_{h \in \varLambda_{n}} {b_{gh}c_{hj}} \right)_{(i,j) \in \varLambda_{l} \times \varLambda_{o}}\\
&= \left( \sum_{g \in \varLambda_{m}} {\sum_{h \in \varLambda_{n}} {a_{ig}b_{gh}c_{hj}}} \right)_{(i,j) \in \varLambda_{l} \times \varLambda_{o}}\\
&= \left( \sum_{h \in \varLambda_{n}} {\sum_{g \in \varLambda_{m}} {a_{ig}b_{gh}c_{hj}}} \right)_{(i,j) \in \varLambda_{l} \times \varLambda_{o}}\\
&= \left( \sum_{g \in \varLambda_{m}} {a_{ig}b_{gj}} \right)_{(i,j) \in \varLambda_{l} \times \varLambda_{n}}\left( c_{ij} \right)_{(i,j) \in \varLambda_{n} \times \varLambda_{o}}\\
&= \left( \left( a_{ij} \right)_{(i,j) \in \varLambda_{l} \times \varLambda_{m}}\left( b_{ij} \right)_{(i,j) \in \varLambda_{m} \times \varLambda_{n}} \right)\left( c_{ij} \right)_{(i,j) \in \varLambda_{n} \times \varLambda_{o}}\\
&= \left( A_{lm}B_{mn} \right)C_{no}
\end{align*}
$\forall A_{lm} \in M_{lm}(K)\forall B_{mn},C_{mn} \in M_{mn}(K)$に対し、次のようになる。
\begin{align*}
A_{lm}\left( B_{mn} + C_{mn} \right) &= \left( a_{ij} \right)_{(i,j) \in \varLambda_{l} \times \varLambda_{m}}\left( \left( b_{ij} \right)_{(i,j) \in \varLambda_{m} \times \varLambda_{n}} + \left( c_{ij} \right)_{(i,j) \in \varLambda_{m} \times \varLambda_{n}} \right)\\
&= \left( a_{ij} \right)_{(i,j) \in \varLambda_{l} \times \varLambda_{m}}\left( b_{ij} + c_{ij} \right)_{(i,j) \in \varLambda_{m} \times \varLambda_{n}}\\
&= \left( \sum_{h \in \varLambda_{m}} {a_{ih}\left( b_{hj} + c_{hj} \right)} \right)_{(i,j) \in \varLambda_{l} \times \varLambda_{n}}\\
&= \left( \sum_{h \in \varLambda_{m}} \left( a_{ih}b_{hj} + a_{ih}c_{hj} \right) \right)_{(i,j) \in \varLambda_{l} \times \varLambda_{n}}\\
&= \left( \sum_{h \in \varLambda_{m}} {a_{ih}b_{hj}} + \sum_{h \in \varLambda_{m}} {a_{ih}c_{hj}} \right)_{(i,j) \in \varLambda_{l} \times \varLambda_{n}}\\
&= \left( \sum_{h \in \varLambda_{m}} {a_{ih}b_{hj}} \right)_{(i,j) \in \varLambda_{l} \times \varLambda_{n}} + \left( \sum_{h \in \varLambda_{m}} {a_{ih}c_{hj}} \right)_{(i,j) \in \varLambda_{l} \times \varLambda_{n}}\\
&= \left( a_{ij} \right)_{(i,j) \in \varLambda_{l} \times \varLambda_{m}}\left( b_{ij} \right)_{(i,j) \in \varLambda_{m} \times \varLambda_{n}} + \left( a_{ij} \right)_{(i,j) \in \varLambda_{l} \times \varLambda_{m}}\left( c_{ij} \right)_{(i,j) \in \varLambda_{m} \times \varLambda_{n}}\\
&= A_{lm}B_{mn} + A_{lm}C_{mn}
\end{align*}
$\forall A_{lm},B_{lm} \in M_{lm}(K)\forall C_{mn} \in M_{mn}(K)$に対し、次のようになる。
\begin{align*}
\left( A_{lm} + B_{lm} \right)C_{mn} &= \left( \left( a_{ij} \right)_{(i,j) \in \varLambda_{l} \times \varLambda_{m}} + \left( b_{ij} \right)_{(i,j) \in \varLambda_{l} \times \varLambda_{m}} \right)\left( c_{ij} \right)_{(i,j) \in \varLambda_{m} \times \varLambda_{n}}\\
&= \left( a_{ij} + b_{ij} \right)_{(i,j) \in \varLambda_{l} \times \varLambda_{m}}\left( c_{ij} \right)_{(i,j) \in \varLambda_{m} \times \varLambda_{n}}\\
&= \left( \sum_{h \in \varLambda_{m}} {\left( a_{ih} + b_{ih} \right)c_{hj}} \right)_{(i,j) \in \varLambda_{l} \times \varLambda_{n}}\\
&= \left( \sum_{h \in \varLambda_{m}} \left( a_{ih}c_{hj} + b_{ih}c_{hj} \right) \right)_{(i,j) \in \varLambda_{l} \times \varLambda_{n}}\\
&= \left( \sum_{h \in \varLambda_{m}} {a_{ih}c_{hj}} + \sum_{h \in \varLambda_{m}} {b_{ih}c_{hj}} \right)_{(i,j) \in \varLambda_{l} \times \varLambda_{n}}\\
&= \left( \sum_{h \in \varLambda_{m}} {a_{ih}c_{hj}} \right)_{(i,j) \in \varLambda_{l} \times \varLambda_{n}} + \left( \sum_{h \in \varLambda_{m}} {b_{ih}c_{hj}} \right)_{(i,j) \in \varLambda_{l} \times \varLambda_{n}}\\
&= \left( a_{ij} \right)_{(i,j) \in \varLambda_{l} \times \varLambda_{m}}\left( c_{ij} \right)_{(i,j) \in \varLambda_{m} \times \varLambda_{n}} + \left( a_{ij} \right)_{(i,j) \in \varLambda_{l} \times \varLambda_{m}}\left( c_{ij} \right)_{(i,j) \in \varLambda_{m} \times \varLambda_{n}}\\
&= A_{lm}C_{mn} + A_{lm}C_{mn}
\end{align*}
$\forall k \in K\forall A_{lm} \in M_{lm}(K)\forall B_{mn} \in M_{mn}(K)$に対し、次のようになる。
\begin{align*}
A_{lm}\left( kB_{mn} \right) &= \left( a_{ij} \right)_{(i,j) \in \varLambda_{l} \times \varLambda_{m}}\left( k\left( b_{ij} \right)_{(i,j) \in \varLambda_{m} \times \varLambda_{n}} \right)\\
&= \left( a_{ij} \right)_{(i,j) \in \varLambda_{l} \times \varLambda_{m}}\left( kb_{ij} \right)_{(i,j) \in \varLambda_{m} \times \varLambda_{n}}\\
&= \left( \sum_{h \in \varLambda_{m}} {a_{ih}kb_{hj}} \right)_{(i,j) \in \varLambda_{l} \times \varLambda_{n}}\\
&= \left( \sum_{h \in \varLambda_{m}} {ka_{ih}b_{hj}} \right)_{(i,j) \in \varLambda_{l} \times \varLambda_{n}}\\
&= \left( ka_{ij} \right)_{(i,j) \in \varLambda_{l} \times \varLambda_{m}}\left( b_{ij} \right)_{(i,j) \in \varLambda_{m} \times \varLambda_{n}}\\
&= \left( k\left( a_{ij} \right)_{(i,j) \in \varLambda_{l} \times \varLambda_{m}} \right)\left( b_{ij} \right)_{(i,j) \in \varLambda_{m} \times \varLambda_{n}}\\
&= \left( kA_{lm} \right)B_{mn}\\
A_{lm}\left( kB_{mn} \right) &= \left( a_{ij} \right)_{(i,j) \in \varLambda_{l} \times \varLambda_{m}}\left( k\left( b_{ij} \right)_{(i,j) \in \varLambda_{m} \times \varLambda_{n}} \right)\\
&= \left( a_{ij} \right)_{(i,j) \in \varLambda_{l} \times \varLambda_{m}}\left( kb_{ij} \right)_{(i,j) \in \varLambda_{m} \times \varLambda_{n}}\\
&= \left( \sum_{h \in \varLambda_{m}} {a_{ih}kb_{hj}} \right)_{(i,j) \in \varLambda_{l} \times \varLambda_{n}}\\
&= \left( \sum_{h \in \varLambda_{m}} {ka_{ih}b_{hj}} \right)_{(i,j) \in \varLambda_{l} \times \varLambda_{n}}\\
&= k\left( \sum_{h \in \varLambda_{m}} {a_{ih}b_{hj}} \right)_{(i,j) \in \varLambda_{l} \times \varLambda_{n}}\\
&= k\left( \left( a_{ij} \right)_{(i,j) \in \varLambda_{l} \times \varLambda_{m}}\left( b_{ij} \right)_{(i,j) \in \varLambda_{m} \times \varLambda_{n}} \right)\\
&= k\left( A_{lm}B_{mn} \right)
\end{align*}
\end{proof}
%\hypertarget{ux8907ux7d20ux6570}{%
\subsubsection{複素数}%\label{ux8907ux7d20ux6570}}
\begin{dfn}
集合$\mathbb{R}^{2}$の元々$\left( a_{1},a_{2} \right)$、$\left( b_{1},b_{2} \right)$を考え次式のように乗法$\cdot$を定義する。
\begin{align*}
\cdot :\mathbb{R}^{2} \times \mathbb{R}^{2} \rightarrow \mathbb{R}^{2};\left( \left( a_{1},a_{2} \right),\left( b_{1},b_{2} \right) \right) \mapsto \left( a_{1},a_{2} \right)\left( b_{1},b_{2} \right) = \left( a_{1}b_{1} - a_{2}b_{2},a_{1}b_{2} + a_{2}b_{1} \right)
\end{align*}
この乗法$\cdot$が定義されているかつ、$(1,0) = 1 \in \mathbb{R}、(0,1) = i$としたその集合$\mathbb{R}^{2}$を$\mathbb{C}$と書きこれの元$(a,b)$を複素数といい$a + bi$と書くことにする。また、複素数$z$の射影たち$\mathrm{pr}_{1}z$、$\mathrm{pr}_{2}z$、即ち、$z = a + bi$としたときの$a$、$b$をそれぞれその複素数$z$の実部、虚部といいそれぞれ$\mathrm{Re}z$、$\mathrm{Im}z$などと書く。これにより、$z = \mathrm{Re}z + \mathrm{Im}zi$が成り立つ。
\end{dfn}
\begin{dfn}
次のような写像$c$を考え$z \in \mathbb{C}$なる$c(z)$を$\overline{z}$などと書きその複素数$z$の共役複素数という。
\begin{align*}
c:\mathbb{C} \rightarrow \mathbb{C};z \mapsto \mathrm{Re}z - \mathrm{Im}zi
\end{align*}
\end{dfn}
\begin{dfn}
次のような写像$a$を考え$z \in \mathbb{C}$なる$a(z)$を$|z|$などと書く。
\begin{align*}
a:\mathbb{C} \rightarrow \mathbb{R};z \mapsto \sqrt{{\mathrm{Re}z}^{2} + {\mathrm{Im}z}^{2}}
\end{align*}
\end{dfn}
\begin{thm}\label{4.1.2.9}
この集合$\mathbb{C}$は体であり複素数$z$の加法、乗法の逆元はそれぞれ$- z$、$\frac{\overline{z}}{|z|^{2}}$となる。
\end{thm}
\begin{proof}
この集合$\mathbb{C}$について考えよう。このとき、$\forall z,v,w \in \mathbb{C}$に対し、次のようになる。
\begin{align*}
(z + v) + w &= \left( \mathrm{Re}z + \mathrm{Im}zi + \mathrm{Re}v + \mathrm{Im}vi \right) + \mathrm{Re}w + \mathrm{Im}wi\\
&= \left( \mathrm{Re}z + \mathrm{Re}v \right) + \left( \mathrm{Im}z + \mathrm{Im}v \right)i + \mathrm{Re}w + \mathrm{Im}wi\\
&= \left( \mathrm{Re}z + \mathrm{Re}v + \mathrm{Re}w \right) + \left( \mathrm{Im}z + \mathrm{Im}v + \mathrm{Im}w \right)i\\
&= \mathrm{Re}z + \mathrm{Im}zi + \left( \mathrm{Re}v + \mathrm{Re}w \right) + \left( \mathrm{Im}v + \mathrm{Im}w \right)i\\
&= \mathrm{Re}z + \mathrm{Im}zi + \left( \mathrm{Re}v + \mathrm{Im}vi + \mathrm{Re}w + \mathrm{Im}wi \right)\\
&= z + (v + w)
\end{align*}
$\exists 0 \in \mathbb{C}\forall z \in \mathbb{C}$に対し、次のようになる。
\begin{align*}
z + 0 &= \mathrm{Re}z + \mathrm{Im}zi + 0 + 0i\\
&= \left( \mathrm{Re}z + 0 \right) + \left( \mathrm{Im}z + 0 \right)i\\
&= \mathrm{Re}z + \mathrm{Im}zi = z\\
0 + z &= 0 + 0i + \mathrm{Re}z + \mathrm{Im}zi\\
&= \left( 0 + \mathrm{Re}z \right) + \left( 0 + \mathrm{Im}z \right)i\\
&= \mathrm{Re}z + \mathrm{Im}zi = z
\end{align*}
$\forall z \in \mathbb{C}\exists - z \in \mathbb{C}$に対し、次のようになる。
\begin{align*}
z - z &= \mathrm{Re}z + \mathrm{Im}zi - \mathrm{Re}z - \mathrm{Im}zi\\
&= \left( \mathrm{Re}z - \mathrm{Re}z \right) + \left( \mathrm{Im}z - \mathrm{Im}z \right)i = 0\\
&- z + z = - \mathrm{Re}z - \mathrm{Im}zi + \mathrm{Re}z + \mathrm{Im}zi\\
&= \left( - \mathrm{Re}z + \mathrm{Re}z \right) + \left( - \mathrm{Im}z + \mathrm{Im}z \right)i = 0
\end{align*}
$\forall z,w \in \mathbb{C}$に対し、次のようになる。
\begin{align*}
z + w &= \mathrm{Re}z + \mathrm{Im}zi + \mathrm{Re}w + \mathrm{Im}wi\\
&= \left( \mathrm{Re}z + \mathrm{Re}w \right) + \left( \mathrm{Im}z + \mathrm{Im}w \right)i\\
&= \left( \mathrm{Re}w + \mathrm{Re}z \right) + \left( \mathrm{Im}w + \mathrm{Im}z \right)i\\
&= \mathrm{Re}w + \mathrm{Im}wi + \mathrm{Re}z + \mathrm{Im}zi\\
&= w + z
\end{align*}
$\forall z,v,w \in \mathbb{C}$に対し、次のようになる。
\begin{align*}
(zv)w &= \left( \left( \mathrm{Re}z + \mathrm{Im}zi \right)\left( \mathrm{Re}v + \mathrm{Im}vi \right) \right)\left( \mathrm{Re}w + \mathrm{Im}wi \right)\\
&= \left( \left( \mathrm{Re}z\mathrm{Re}v - \mathrm{Im}z\mathrm{Im}v \right) + \left( \mathrm{Re}z\mathrm{Im}v + \mathrm{Re}v\mathrm{Im}z \right)i \right)\left( \mathrm{Re}w + \mathrm{Im}wi \right)\\
&= \left( \left( \mathrm{Re}z\mathrm{Re}v - \mathrm{Im}z\mathrm{Im}v \right)\mathrm{Re}w - \left( \mathrm{Re}z\mathrm{Im}v + \mathrm{Re}v\mathrm{Im}z \right)\mathrm{Im}w \right) \\
&\quad + \left( \left( \mathrm{Re}z\mathrm{Re}v - \mathrm{Im}z\mathrm{Im}v \right)\mathrm{Im}w + \mathrm{Re}w\left( \mathrm{Re}z\mathrm{Im}v + \mathrm{Re}v\mathrm{Im}z \right) \right)i\\
&= \left( \mathrm{Re}z\mathrm{Re}v\mathrm{Re}w - \mathrm{Re}w\mathrm{Im}z\mathrm{Im}v - \mathrm{Re}z\mathrm{Im}v\mathrm{Im}w - \mathrm{Re}v\mathrm{Im}z\mathrm{Im}w \right) \\
&\quad + \left( \mathrm{Re}z\mathrm{Re}v\mathrm{Im}w - \mathrm{Im}z\mathrm{Im}v\mathrm{Im}w + \mathrm{Re}z\mathrm{Re}w\mathrm{Im}v + \mathrm{Re}v\mathrm{Re}w\mathrm{Im}z \right)i\\
&= \left( \mathrm{Re}z\mathrm{Re}v\mathrm{Re}w - \mathrm{Re}z\mathrm{Im}v\mathrm{Im}w - \mathrm{Re}v\mathrm{Im}z\mathrm{Im}w - \mathrm{Re}w\mathrm{Im}z\mathrm{Im}v \right) \\
&\quad + \left( \mathrm{Re}z\mathrm{Re}v\mathrm{Im}w + \mathrm{Re}z\mathrm{Re}w\mathrm{Im}v + \mathrm{Re}v\mathrm{Re}w\mathrm{Im}z - \mathrm{Im}z\mathrm{Im}v\mathrm{Im}w \right)i\\
&= \left( \mathrm{Re}z\left( \mathrm{Re}v\mathrm{Re}w - \mathrm{Im}v\mathrm{Im}w \right) - \mathrm{Im}z\left( \mathrm{Re}v\mathrm{Im}w + \mathrm{Re}w\mathrm{Im}v \right) \right) \\
&\quad + \left( \mathrm{Re}z\left( \mathrm{Re}v\mathrm{Im}w + \mathrm{Re}w\mathrm{Im}v \right) + \mathrm{Im}z\left( \mathrm{Re}v\mathrm{Re}w - \mathrm{Im}v\mathrm{Im}w \right) \right)i\\
&= \left( \mathrm{Re}z + \mathrm{Im}zi \right)\left( \left( \mathrm{Re}v\mathrm{Re}w - \mathrm{Im}v\mathrm{Im}w \right) + \left( \mathrm{Re}v\mathrm{Im}w + \mathrm{Re}w\mathrm{Im}v \right)i \right)\\
&= \left( \mathrm{Re}z + \mathrm{Im}zi \right)\left( \left( \mathrm{Re}v + \mathrm{Im}vi \right)\left( \mathrm{Re}w + \mathrm{Im}wi \right) \right) = z(vw)
\end{align*}
$\exists 1 \in \mathbb{C}\forall z \in \mathbb{C}$に対し、次のようになる。
\begin{align*}
z1 &= \left( \mathrm{Re}z + \mathrm{Im}zi \right)(1 + 0i)\\
&= \left( 1\mathrm{Re}z - 0\mathrm{Im}z \right) + \left( 0\mathrm{Re}z + 1\mathrm{Im}z \right)i\\
&= \mathrm{Re}z + \mathrm{Im}zi = z\\
1z &= (1 + 0i)\left( \mathrm{Re}z + \mathrm{Im}zi \right)\\
&= \left( 1\mathrm{Re}z - 0\mathrm{Im}z \right) + \left( 1\mathrm{Im}z + 0\mathrm{Re}z \right)i\\
&= \mathrm{Re}z + \mathrm{Im}zi = z
\end{align*}
$\forall z,w \in \mathbb{C}$に対し、次のようになる。
\begin{align*}
zw &= \left( \mathrm{Re}z + \mathrm{Im}zi \right)\left( \mathrm{Re}w + \mathrm{Im}wi \right)\\
&= \left( \mathrm{Re}z\mathrm{Re}w - \mathrm{Im}z\mathrm{Im}w \right) + \left( \mathrm{Re}z\mathrm{Im}w + \mathrm{Re}w\mathrm{Im}z \right)i\\
&= \left( \mathrm{Re}wRez - \mathrm{Im}w\mathrm{Im}z \right) + \left( \mathrm{Re}w\mathrm{Im}z + \mathrm{Re}z\mathrm{Im}w \right)i\\
&= \left( \mathrm{Re}w + \mathrm{Im}wi \right)\left( \mathrm{Re}z + \mathrm{Im}zi \right) = wz
\end{align*}
$\forall z,v,w \in \mathbb{C}$に対し、次のようになる。
\begin{align*}
z(v + w) &= \left( \mathrm{Re}z + \mathrm{Im}zi \right)\left( \mathrm{Re}v + \mathrm{Im}vi + \mathrm{Re}w + \mathrm{Im}wi \right)\\
&= \left( \mathrm{Re}z + \mathrm{Im}zi \right)\left( \left( \mathrm{Re}v + \mathrm{Re}w \right) + \left( \mathrm{Im}v + \mathrm{Im}w \right)i \right)\\
&= \left( \mathrm{Re}z\left( \mathrm{Re}v + \mathrm{Re}w \right) - \mathrm{Im}z\left( \mathrm{Im}v + \mathrm{Im}w \right) \right) \\
&\quad + \left( \mathrm{Re}z\left( \mathrm{Im}v + \mathrm{Im}w \right) + \mathrm{Im}z\left( \mathrm{Re}v + \mathrm{Re}w \right) \right)i\\
&= \left( \mathrm{Re}z\mathrm{Re}v + \mathrm{Re}z\mathrm{Re}w - \mathrm{Im}z\mathrm{Im}v - \mathrm{Im}z\mathrm{Im}w \right) \\
&\quad + \left( \mathrm{Re}z\mathrm{Im}v + \mathrm{Re}z\mathrm{Im}w + \mathrm{Re}v\mathrm{Im}z + \mathrm{Re}w\mathrm{Im}z \right)i\\
&= \left( \mathrm{Re}z\mathrm{Re}v - \mathrm{Im}z\mathrm{Im}v \right) + \left( \mathrm{Re}z\mathrm{Im}v + \mathrm{Re}v\mathrm{Im}z \right)i \\
&\quad + \left( \mathrm{Re}z\mathrm{Re}w - \mathrm{Im}z\mathrm{Im}w \right) + \left( \mathrm{Re}z\mathrm{Im}w + \mathrm{Re}w\mathrm{Im}z \right)i\\
&= \left( \mathrm{Re}z + \mathrm{Im}zi \right)\left( \mathrm{Re}v + \mathrm{Im}vi \right) + \left( \mathrm{Re}z + \mathrm{Im}zi \right)\left( \mathrm{Re}w + \mathrm{Im}wi \right)\\
&= zv + zw\\
(z + v)w &= \left( \mathrm{Re}z + \mathrm{Im}zi + \mathrm{Re}v + \mathrm{Im}vi \right)\left( \mathrm{Re}w + \mathrm{Im}wi \right)\\
&= \left( \left( \mathrm{Re}z + \mathrm{Re}v \right) + \left( \mathrm{Im}z + \mathrm{Im}v \right)i \right)\left( \mathrm{Re}w + \mathrm{Im}wi \right)\\
&= \left( \left( \mathrm{Re}z + \mathrm{Re}v \right)\mathrm{Re}w - \left( \mathrm{Im}z + \mathrm{Im}v \right)\mathrm{Im}w \right) \\
&\quad + \left( \left( \mathrm{Re}z + \mathrm{Re}v \right)\mathrm{Im}w + \left( \mathrm{Im}z + \mathrm{Im}v \right)\mathrm{Re}w \right)i\\
&= \left( \mathrm{Re}z\mathrm{Re}w + \mathrm{Re}v\mathrm{Re}w - \mathrm{Im}z\mathrm{Im}w - \mathrm{Im}v\mathrm{Im}w \right) \\
&\quad + \left( \mathrm{Re}z\mathrm{Im}w + \mathrm{Re}v\mathrm{Im}w + \mathrm{Re}w\mathrm{Im}z + \mathrm{Re}w\mathrm{Im}v \right)i\\
&= \left( \mathrm{Re}z\mathrm{Re}w - \mathrm{Im}z\mathrm{Im}w \right) + \left( \mathrm{Re}z\mathrm{Im}w + \mathrm{Re}w\mathrm{Im}z \right)i \\
&\quad+ \left( \mathrm{Re}v\mathrm{Re}w - \mathrm{Im}v\mathrm{Im}w \right) + \left( \mathrm{Re}v\mathrm{Im}w + \mathrm{Re}w\mathrm{Im}v \right)i\\
&= \left( \mathrm{Re}z + \mathrm{Im}zi \right)\left( \mathrm{Re}w + \mathrm{Im}wi \right) + \left( \mathrm{Re}v + \mathrm{Im}vi \right)\left( \mathrm{Re}w + \mathrm{Im}wi \right)\\
&= zw + vw
\end{align*}
$\forall z \in \mathbb{C} \setminus \left\{ 0 \right\}\exists\frac{\overline{z}}{|z|^{2}} \in \mathbb{C} \setminus \left\{ 0 \right\}$に対し、次のようになる。
\begin{align*}
z\frac{\overline{z}}{|z|^{2}} &= \left( \mathrm{Re}z + \mathrm{Im}zi \right)\left( \frac{\mathrm{Re}z - \mathrm{Im}zi}{\left( \mathrm{Re}z \right)^{2} + \left( \mathrm{Im}z \right)^{2}} \right)\\
&= \frac{1}{\left( \mathrm{Re}z \right)^{2} + \left( \mathrm{Im}z \right)^{2}}\left( \mathrm{Re}z + \mathrm{Im}zi \right)\left( \mathrm{Re}z - \mathrm{Im}zi \right)\\
&= \frac{1}{\left( \mathrm{Re}z \right)^{2} + \left( \mathrm{Im}z \right)^{2}}\left( \left( \left( \mathrm{Re}z \right)^{2} + \left( \mathrm{Im}z \right)^{2} \right) + \left( - \mathrm{Re}z\mathrm{Im}z + \mathrm{Re}z\mathrm{Im}z \right)i \right)\\
&= \frac{\left( \left( \mathrm{Re}z \right)^{2} + \left( \mathrm{Im}z \right)^{2} \right) + 0i}{\left( \mathrm{Re}z \right)^{2} + \left( \mathrm{Im}z \right)^{2}}\\
&= \frac{\left( \mathrm{Re}z \right)^{2} + \left( \mathrm{Im}z \right)^{2}}{\left( \mathrm{Re}z \right)^{2} + \left( \mathrm{Im}z \right)^{2}} = 1\\
\frac{\overline{z}}{|z|^{2}}z &= \left( \frac{\mathrm{Re}z - \mathrm{Im}zi}{\left( \mathrm{Re}z \right)^{2} + \left( \mathrm{Im}z \right)^{2}} \right)\left( \mathrm{Re}z + \mathrm{Im}zi \right)\\
&= \frac{1}{\left( \mathrm{Re}z \right)^{2} + \left( \mathrm{Im}z \right)^{2}}\left( \mathrm{Re}z - \mathrm{Im}zi \right)\left( \mathrm{Re}z + \mathrm{Im}zi \right)\\
&= \frac{1}{\left( \mathrm{Re}z \right)^{2} + \left( \mathrm{Im}z \right)^{2}}\left( \left( \left( \mathrm{Re}z \right)^{2} + \left( \mathrm{Im}z \right)^{2} \right) + \left( \mathrm{Re}z\mathrm{Im}z - \mathrm{Re}z\mathrm{Im}z \right)i \right)\\
&= \frac{\left( \left( \mathrm{Re}z \right)^{2} + \left( \mathrm{Im}z \right)^{2} \right) + 0i}{\left( \mathrm{Re}z \right)^{2} + \left( \mathrm{Im}z \right)^{2}}\\
&= \frac{\left( \mathrm{Re}z \right)^{2} + \left( \mathrm{Im}z \right)^{2}}{\left( \mathrm{Re}z \right)^{2} + \left( \mathrm{Im}z \right)^{2}} = 1
\end{align*}
\end{proof}
\begin{thm}\label{4.1.2.10} これの系として次式が成り立つ。
\begin{align*}
z\overline{z} = \overline{z}z = |z|^{2}
\end{align*}
\end{thm}
\begin{proof}
$\forall z \in \mathbb{C} \setminus \left\{ 0 \right\}$に対し、$\frac{1}{z} = \frac{\overline{z}}{|z|^{2}}$が成り立つのであったので、
\begin{align*}
1 &= z\frac{1}{z} = z\frac{\overline{z}}{|z|^{2}} = \frac{1}{|z|^{2}}z\overline{z} \Leftrightarrow z\overline{z} = |z|^{2}\\
1 &= \frac{1}{z} = \frac{\overline{z}}{|z|^{2}}z = \frac{1}{|z|^{2}}\overline{z}z \Leftrightarrow \overline{z}z = |z|^{2}
\end{align*}
$z = 0$のときも明らかに成り立つ。
\end{proof}
\begin{thm}\label{4.1.2.11}
この写像$c$は$\mathbb{R}$-線形同型写像であるかつ、算法$+ 、 \cdot$どちらについても同型写像である、即ち、$\forall a,b \in \mathbb{R}\forall z,w \in \mathbb{C}$に対し次式が成り立つような全単射となり$c^{- 1} = c$が成り立つ。
\begin{longtable}[c]{cc}
\hspace{-0.5em}\begin{tabular}{c}
  $c(z + w) = c(z) + c(w)$\\
  $c(zw) = c(z)c(w)$\\
  $c(az + bw) = ac(z) + bc(w)$\\
\end{tabular} & \hspace{-0.5em}\begin{tabular}{c}
  $\overline{z + w} = \overline{z} + \overline{w}$\\
  $\overline{zw} = \overline{z}\ \overline{w}$\\
  $\overline{az + bw} = a\overline{z} + b\overline{w}$\\
\end{tabular} \\
\end{longtable}
当然ながら、次式も成り立つ。
\begin{longtable}[c]{cc}
\hspace{-0.5em}\begin{tabular}{c}
  $c(z - w) = c(z) - c(w)$\\
  $c\left( \frac{z}{w} \right) = \frac{c(z)}{c(w)}\ \mathrm{if}\ w \neq 0$\\
\end{tabular} & \hspace{-0.5em}\begin{tabular}{c}
  $\overline{z - w} = \overline{z} - \overline{w}$\\
  $\overline{\frac{z}{w}} = \frac{\overline{z}}{\overline{w}}\ \mathrm{if}\ w \neq 0$\\
\end{tabular} \\
\end{longtable}
\end{thm}
\begin{proof} 次のような写像$c$を考えよう。
\begin{align*}
c:\mathbb{C} \rightarrow \mathbb{C};z \mapsto \mathrm{Re}z - \mathrm{Im}zi
\end{align*}\par
このとき、$\forall z,w \in \mathbb{C}$に対し、次のようになる。
\begin{align*}
c(z + w) &= c\left( \mathrm{Re}z + \mathrm{Im}zi + \mathrm{Re}w + \mathrm{Im}wi \right)\\
&= c\left( \left( \mathrm{Re}z + \mathrm{Re}w \right) + \left( \mathrm{Im}z + \mathrm{Im}w \right)i \right)\\
&= \left( \mathrm{Re}z + \mathrm{Re}w \right) - \left( \mathrm{Im}z + \mathrm{Im}w \right)i\\
&= \mathrm{Re}z + \mathrm{Re}w - \mathrm{Im}zi - \mathrm{Im}wi\\
&= \mathrm{Re}z - \mathrm{Im}zi + \mathrm{Re}w - \mathrm{Im}wi\\
&= c\left( \mathrm{Re}z + \mathrm{Im}zi \right) + c\left( \mathrm{Re}w + \mathrm{Im}wi \right)\\
&= c(z) + c(w)\\
c(zw) &= c\left( \left( \mathrm{Re}z + \mathrm{Im}zi \right)\left( \mathrm{Re}w + \mathrm{Im}wi \right) \right)\\
&= c\left( \left( \mathrm{Re}z\mathrm{Re}w - \mathrm{Im}z\mathrm{Im}w \right) + \left( \mathrm{Re}z\mathrm{Im}w + \mathrm{Re}w\mathrm{Im}z \right)i \right)\\
&= \left( \mathrm{Re}z\mathrm{Re}w - \mathrm{Im}z\mathrm{Im}w \right) - \left( \mathrm{Re}z\mathrm{Im}w + \mathrm{Re}w\mathrm{Im}z \right)i\\
&= \left( \mathrm{Re}z\mathrm{Re}w - \left( - \mathrm{Im}z \right)\left( - \mathrm{Im}w \right) \right) + \left( \mathrm{Re}z\left( - \mathrm{Im}w \right) + \mathrm{Re}w\left( - \mathrm{Im}z \right) \right)i\\
&= \left( \mathrm{Re}z - \mathrm{Im}zi \right)\left( \mathrm{Re}w - \mathrm{Im}wi \right)\\
&= c\left( \mathrm{Re}z + \mathrm{Im}zi \right)c\left( \mathrm{Re}w + \mathrm{Im}wi \right)\\
&= c(z)c(w)
\end{align*}
以上より、$c(z + w) = c(z) + c(w)$と$c(zw) = c(z)c(w)$が成り立つことが示された。\par
一方で、$\forall a,b \in \mathbb{R}$に対し、次のようになる。
\begin{align*}
c(az + bw) &= c\left( a\left( \mathrm{Re}z + \mathrm{Im}zi \right) + b\left( \mathrm{Re}w + \mathrm{Im}wi \right) \right)\\
&= c\left( a\mathrm{Re}z + a\mathrm{Im}zi + b\mathrm{Re}w + b\mathrm{Im}wi \right)\\
&= c\left( \left( a\mathrm{Re}z + b\mathrm{Re}w \right) + \left( a\mathrm{Im}z + b\mathrm{Im}w \right)i \right)\\
&= \left( a\mathrm{Re}z + b\mathrm{Re}w \right) - \left( a\mathrm{Im}z + b\mathrm{Im}w \right)i\\
&= a\mathrm{Re}z - a\mathrm{Im}zi + b\mathrm{Re}w - b\mathrm{Im}wi\\
&= a\left( \mathrm{Re}z - \mathrm{Im}zi \right) + b\left( \mathrm{Re}w - \mathrm{Im}wi \right)\\
&= ac\left( \mathrm{Re}z + \mathrm{Im}zi \right) + bc\left( \mathrm{Re}w + \mathrm{Im}wi \right)\\
&= ac(z) + bc(w)
\end{align*}
これにより、この写像$c$は$\mathbb{R}$-線形写像であり、このとき、次のようになるので、
\begin{align*}
c \circ c(z) &= c\left( c(z) \right)\\
&= c\left( \mathrm{Re}z - \mathrm{Im}zi \right)\\
&= Re\left( \mathrm{Re}z - \mathrm{Im}zi \right) - Im\left( \mathrm{Re}z - \mathrm{Im}zi \right)i\\
&= \mathrm{Re}z - \left( - \mathrm{Im}z \right)i\\
&= \mathrm{Re}z + \mathrm{Im}zi = z
\end{align*}
$c^{- 1} = c$が成り立つ。\par
さらに、次のようになる。
\begin{align*}
c(z - w) = c\left( z + ( - 1)w \right) = c(z) + ( - 1)c(w) = c(z) - c(w)
\end{align*}
$w \neq 0$のとき、次のようになる。
\begin{align*}
c\left( \frac{z}{w} \right) &= c\left( z\frac{\overline{w}}{|w|^{2}} \right)\\
&= c\left( \frac{1}{w\overline{w}}z\overline{w} \right)\\
&= c\left( \frac{1}{wc(w)}zc(w) \right)\\
&= \frac{1}{wc(w)}c(z)c\left( c(w) \right)\\
&= \frac{1}{c(w)c\left( c(w) \right)}c(z)c\left( c(w) \right)\\
&= c(z)\frac{c\left( c(w) \right)}{c(w)c\left( c(w) \right)}\\
&= c(z)\frac{\overline{c(w)}}{c(w)\ \overline{c(w)}}\\
&= c(z)\frac{\overline{c(w)}}{\left| c(w) \right|^{2}}\\
&= c(z)\frac{1}{c(w)} = \frac{c(z)}{c(w)}
\end{align*}
\end{proof}
\begin{thm}\label{4.1.2.12}
この写像$a$は算法$\cdot$について準同型写像である、即ち、$\forall z,w \in \mathbb{C}$に対し次式が成り立つ。
\begin{longtable}[c]{cc}
\hspace{-0.5em}\begin{tabular}{c}
$a(zw) = a(z)a(w)$\\
$a\left( \frac{z}{w} \right) = \frac{a(z)}{a(w)}\ \mathrm{if}\ w \neq 0$\\
\end{tabular} & \begin{tabular}{c}
$|zw| = |z||w|$\\
$\left| \frac{z}{w} \right| = \frac{|z|}{|w|}\ \mathrm{if}\ w \neq 0$\\
\end{tabular} \\
\end{longtable}
\end{thm}
\begin{proof} 次のような写像$a$を考えよう。
\begin{align*}
a:\mathbb{C} \rightarrow \mathbb{R};z \mapsto \sqrt{\left( \mathrm{Re}z \right)^{2} + \left( \mathrm{Im}z \right)^{2}}
\end{align*}\par
このとき、$\forall z,w \in \mathbb{C}$に対し、次のようになる。
\begin{align*}
a(zw) &= a\left( \left( \mathrm{Re}z + \mathrm{Im}zi \right)\left( \mathrm{Re}w + \mathrm{Im}wi \right) \right)\\
&= a\left( \left( \mathrm{Re}z\mathrm{Re}w - \mathrm{Im}z\mathrm{Im}w \right) + \left( \mathrm{Re}z\mathrm{Im}w + \mathrm{Re}w\mathrm{Im}z \right)i \right)\\
&= \left( \left( \mathrm{Re}z\mathrm{Re}w - \mathrm{Im}z\mathrm{Im}w \right)^{2} + \left( \mathrm{Re}z\mathrm{Im}w + \mathrm{Re}w\mathrm{Im}z \right)^{2} \right)^{\frac{1}{2}}\\
&= \left( \left( \mathrm{Re}z \right)^{2}\left( \mathrm{Re}w \right)^{2} - 2\mathrm{Re}z\mathrm{Re}w\mathrm{Im}z\mathrm{Im}w + \left( \mathrm{Im}z \right)^{2}\left( \mathrm{Im}w \right)^{2} \right. \\
&\quad \left.+ \left( \mathrm{Re}z \right)^{2}\left( \mathrm{Im}w \right)^{2} + 2\mathrm{Re}z\mathrm{Re}w\mathrm{Im}w\mathrm{Im}z + \left( \mathrm{Re}w \right)^{2}\left( \mathrm{Im}z \right)^{2} \right)^{\frac{1}{2}}\\
&= \left( \left( \mathrm{Re}z \right)^{2}\left( \mathrm{Re}w \right)^{2} + \left( \mathrm{Im}z \right)^{2}\left( \mathrm{Im}w \right)^{2} + \left( \mathrm{Re}z \right)^{2}\left( \mathrm{Im}w \right)^{2} + \left( \mathrm{Re}w \right)^{2}\left( \mathrm{Im}z \right)^{2} \right)^{\frac{1}{2}}\\
&= \left( \left( \left( \mathrm{Re}z \right)^{2} + \left( \mathrm{Im}z \right)^{2} \right)\left( \left( \mathrm{Re}w \right)^{2} + \left( \mathrm{Im}w \right)^{2} \right) \right)^{\frac{1}{2}}\\
&= \left( \left( \mathrm{Re}z \right)^{2} + \left( \mathrm{Im}z \right)^{2} \right)^{\frac{1}{2}}\left( \left( \mathrm{Re}w \right)^{2} + \left( \mathrm{Im}w \right)^{2} \right)^{\frac{1}{2}}\\
&= a\left( \mathrm{Re}z + \mathrm{Im}zi \right)a\left( \mathrm{Re}w + \mathrm{Im}wi \right) = a(z)a(w)
\end{align*}
$w \neq 0$のとき、次のようになる。
\begin{align*}
a\left( \frac{z}{w} \right) &= a\left( z\frac{\overline{w}}{|w|^{2}} \right)\\
&= a\left( \left( \mathrm{Re}z + \mathrm{Im}zi \right)\left( \frac{\mathrm{Re}w - \mathrm{Im}wi}{\left( \mathrm{Re}w \right)^{2} + \left( \mathrm{Im}w \right)^{2}} \right) \right)\\
&= a\left( \frac{1}{\left( \mathrm{Re}w \right)^{2} + \left( \mathrm{Im}w \right)^{2}}\left( \mathrm{Re}z + \mathrm{Im}zi \right)\left( \mathrm{Re}w - \mathrm{Im}wi \right) \right)\\
&= a\left( \frac{1}{\left( \mathrm{Re}w \right)^{2} + \left( \mathrm{Im}w \right)^{2}}\left( \left( \mathrm{Re}z\mathrm{Re}w - \mathrm{Im}z\mathrm{Im}w \right) + \left( \mathrm{Re}z\mathrm{Im}w + \mathrm{Re}w\mathrm{Im}z \right)i \right) \right)\\
&= a\left( \left( \frac{\mathrm{Re}zRe\left( z_{2} \right) - \mathrm{Im}z\mathrm{Im}w}{\left( \mathrm{Re}w \right)^{2} + \left( \mathrm{Im}w \right)^{2}} + \frac{\mathrm{Re}z\mathrm{Im}w + \mathrm{Re}w\mathrm{Im}z}{\left( \mathrm{Re}w \right)^{2} + \left( \mathrm{Im}w \right)^{2}}i \right) \right)\\
&= \left( \left( \frac{\mathrm{Re}z\mathrm{Re}w - \mathrm{Im}z\mathrm{Im}w}{\left( \mathrm{Re}w \right)^{2} + \left( \mathrm{Im}w \right)^{2}} \right)^{2} + \left( \frac{\mathrm{Re}z\mathrm{Im}w + \mathrm{Re}w\mathrm{Im}z}{\left( \mathrm{Re}w \right)^{2} + \left( \mathrm{Im}w \right)^{2}} \right)^{2} \right)^{\frac{1}{2}}\\
&= \frac{1}{\left( \mathrm{Re}w \right)^{2} + \left( \mathrm{Im}w \right)^{2}}\left( \left( \mathrm{Re}z\mathrm{Re}w - \mathrm{Im}z\mathrm{Im}w \right)^{2} + \left( \mathrm{Re}z\mathrm{Im}w + \mathrm{Re}w\mathrm{Im}z \right)^{2} \right)^{\frac{1}{2}}\\
&= \frac{1}{\left( \mathrm{Re}w \right)^{2} + \left( \mathrm{Im}w \right)^{2}}\left( \left( \mathrm{Re}z \right)^{2}\left( \mathrm{Re}w \right)^{2} - 2\mathrm{Re}z\mathrm{Re}w\mathrm{Im}z\mathrm{Im}w \right. \\
&\quad \left. + \left( \mathrm{Im}z \right)^{2}\left( \mathrm{Im}w \right)^{2} + \left( \mathrm{Re}z \right)^{2}\left( \mathrm{Im}w \right)^{2} + 2\mathrm{Re}z\mathrm{Re}w\mathrm{Im}w\mathrm{Im}z + \left( \mathrm{Re}w \right)^{2}\left( \mathrm{Im}z \right)^{2} \right)^{\frac{1}{2}}\\
&= \frac{1}{\left( \mathrm{Re}w \right)^{2} + {Im\left( z_{2} \right)}^{2}}\left( \left( \mathrm{Re}z \right)^{2}\left( \mathrm{Re}w \right)^{2} + \left( \mathrm{Im}z \right)^{2}\left( \mathrm{Im}w \right)^{2} + \left( \mathrm{Re}z \right)^{2}\left( \mathrm{Im}w \right)^{2} + \left( \mathrm{Re}w \right)^{2}\left( \mathrm{Im}z \right)^{2} \right)^{\frac{1}{2}}\\
&= \frac{1}{\left( \mathrm{Re}w \right)^{2} + \left( \mathrm{Im}w \right)^{2}}\left( \left( \left( \mathrm{Re}z \right)^{2} + \left( \mathrm{Im}z \right)^{2} \right)\left( \left( \mathrm{Re}w \right)^{2} + \left( \mathrm{Im}w \right)^{2} \right) \right)^{\frac{1}{2}}\\
&= \frac{\left( \left( \mathrm{Re}z \right)^{2} + \left( \mathrm{Im}z \right)^{2} \right)^{\frac{1}{2}}}{\left( \left( \mathrm{Re}w \right)^{2} + \left( \mathrm{Im}w \right)^{2} \right)^{\frac{1}{2}}}\\
&= \frac{a\left( \mathrm{Re}z + \mathrm{Im}zi \right)}{a\left( \mathrm{Re}w + \mathrm{Im}wi \right)} = \frac{a(z)}{a(w)}
\end{align*}
\end{proof}
\begin{thm}\label{4.1.2.13}
この写像$a$はnormをなす、即ち、$\forall b \in \mathbb{R}\forall z,w \in \mathbb{C}$に対し次式が成り立つ。
\begin{longtable}[c]{cc}
\begin{tabular}{c}
$0 \leq a(z)$\\
$a(z) = 0 \Leftrightarrow z = 0$\\
$a(bz) = |b|a(z)$\\
$a(z + w) \leq a(z) + a(w)$\\
\end{tabular} & \begin{tabular}{c}
$0 \leq |z|$\\
$|z| = 0 \Leftrightarrow z = 0$\\
$|bz| = |b||z|$\\
$|z + w| \leq |z| + |w|$\\
\end{tabular} \\
\end{longtable}
\end{thm}
\begin{proof}
$b \in \mathbb{R}\forall z,w \in \mathbb{C}$に対し、もちろん、$0 \leq a(z)$が成り立つ。また、$a(z) = 0$が成り立つかつ、$z \neq 0$が成り立つなら、$\mathrm{Re}z \neq 0$または$\mathrm{Im}z \neq 0$が成り立つので、$0 < \left( \mathrm{Re}z \right)^{2} + \left( \mathrm{Im}z \right)^{2}$が成り立ち、したがって、$0 < a(z)$が成り立つが、これは$a(z) = 0$が成り立つことに矛盾する。$z = 0$が成り立つなら、明らかに$a(z) = 0$が成り立つ。\par
また、次のようになる。
\begin{align*}
a(bz) &= a\left( b\mathrm{Re}z + b\mathrm{Im}zi \right)\\
&= \left( \left( b\mathrm{Re}w \right)^{2} + \left( b\mathrm{Im}w \right)^{2} \right)^{\frac{1}{2}}\\
&= \left( b^{2}\left( \left( \mathrm{Re}w \right)^{2} + \left( \mathrm{Im}w \right)^{2} \right) \right)^{\frac{1}{2}}\\
&= |b|\left( \left( \mathrm{Re}w \right)^{2} + \left( \mathrm{Im}w \right)^{2} \right)^{\frac{1}{2}}\\
&= |b|a\left( \mathrm{Re}z + \mathrm{Im}zi \right) = |b|a(z)
\end{align*}
ここで、次式が成り立つことから、
\begin{align*}
0 &\leq \left| \mathrm{Re}z\mathrm{Im}w - \mathrm{Re}w\mathrm{Im}z \right|^{2}\\
&= \left( \mathrm{Re}z \right)^{2}\left( \mathrm{Im}w \right)^{2} + \left( \mathrm{Re}w \right)^{2}\left( \mathrm{Im}z \right)^{2} - 2\mathrm{Re}z\mathrm{Re}w\mathrm{Im}w\mathrm{Im}z
\end{align*}
次のようになるので、
\begin{align*}
\mathrm{Re}z\mathrm{Re}w + \mathrm{Im}z\mathrm{Im}w &= \left( \left( \mathrm{Re}z\mathrm{Re}w + \mathrm{Im}z\mathrm{Im}w \right)^{2} \right)^{\frac{1}{2}}\\
&= \left( \left( \mathrm{Re}z \right)^{2}\left( \mathrm{Re}w \right)^{2} + 2\mathrm{Re}z\mathrm{Re}w\mathrm{Im}w\mathrm{Im}z + \left( \mathrm{Im}z \right)^{2}\left( \mathrm{Im}w \right)^{2} \right)^{\frac{1}{2}}\\
&\leq \left( \left( \mathrm{Re}z \right)^{2}\left( \mathrm{Re}w \right)^{2} + \left( \mathrm{Re}z \right)^{2}\left( \mathrm{Im}w \right)^{2} + \left( \mathrm{Re}w \right)^{2}\left( \mathrm{Im}z \right)^{2} + \left( \mathrm{Im}z \right)^{2}\left( \mathrm{Im}w \right)^{2} \right)^{\frac{1}{2}}\\
&= \left( \left( \left( \mathrm{Re}z \right)^{2} + \left( \mathrm{Im}z \right)^{2} \right)\left( \left( \mathrm{Re}w \right)^{2} + \left( \mathrm{Im}w \right)^{2} \right) \right)^{\frac{1}{2}}\\
&= \left( \left( \mathrm{Re}z \right)^{2} + \left( \mathrm{Im}z \right)^{2} \right)^{\frac{1}{2}}\left( \left( \mathrm{Re}w \right)^{2} + \left( \mathrm{Im}w \right)^{2} \right)^{\frac{1}{2}}
\end{align*}
したがって、次のようになる。
\begin{align*}
a(z + w) &= a\left( \mathrm{Re}z + \mathrm{Im}zi + \mathrm{Re}w + \mathrm{Im}wi \right)\\
&= a\left( \left( \mathrm{Re}z + \mathrm{Re}w \right) + \left( \mathrm{Im}z + \mathrm{Im}w \right)i \right)\\
&= \left( \left( \mathrm{Re}z + \mathrm{Re}w \right)^{2} + \left( \mathrm{Im}z + \mathrm{Im}w \right)^{2} \right)^{\frac{1}{2}}\\
&= \left( \left( \mathrm{Re}z \right)^{2} + 2\mathrm{Re}z\mathrm{Re}w + \left( \mathrm{Re}w \right)^{2} + \left( \mathrm{Im}z \right)^{2} + 2\mathrm{Im}z\mathrm{Im}w + \left( \mathrm{Im}w \right)^{2} \right)^{\frac{1}{2}}\\
&= \left( \left( \mathrm{Re}z \right)^{2} + \left( \mathrm{Re}w \right)^{2} + \left( \mathrm{Im}z \right)^{2} + \left( \mathrm{Im}w \right)^{2} + 2\left( \mathrm{Re}z\mathrm{Re}w + \mathrm{Im}z\mathrm{Im}w \right) \right)^{\frac{1}{2}}\\
&\leq \left( \left( \mathrm{Re}z \right)^{2} + \left( \mathrm{Re}w \right)^{2} + \left( \mathrm{Im}z \right)^{2} + \left( \mathrm{Im}w \right)^{2} \right. \\
&\quad \left. + 2\left( \left( \mathrm{Re}z \right)^{2} + \left( \mathrm{Im}z \right)^{2} \right)^{\frac{1}{2}}\left( \left( \mathrm{Re}w \right)^{2} + \left( \mathrm{Im}w \right)^{2} \right)^{\frac{1}{2}} \right)^{\frac{1}{2}}\\
&= \left( \left( \mathrm{Re}z \right)^{2} + \left( \mathrm{Im}z \right)^{2} + 2\left( \left( \mathrm{Re}z \right)^{2} + \left( \mathrm{Im}z \right)^{2} \right)^{\frac{1}{2}}\left( \left( \mathrm{Re}w \right)^{2} + \left( \mathrm{Im}w \right)^{2} \right)^{\frac{1}{2}} \right. \\
&\quad \left. + \left( \mathrm{Re}w \right)^{2} + \left( \mathrm{Im}w \right)^{2} \right)^{\frac{1}{2}}\\
&= \left( \left( \left( \left( \mathrm{Re}z \right)^{2} + \left( \mathrm{Im}z \right)^{2} \right)^{\frac{1}{2}} + \left( \left( \mathrm{Re}w \right)^{2} + \left( \mathrm{Im}w \right)^{2} \right)^{\frac{1}{2}} \right)^{2} \right)^{\frac{1}{2}}\\
&= \left| \left( \left( \mathrm{Re}z \right)^{2} + \left( \mathrm{Im}z \right)^{2} \right)^{\frac{1}{2}} + \left( \left( \mathrm{Re}w \right)^{2} + \left( \mathrm{Im}w \right)^{2} \right)^{\frac{1}{2}} \right|\\
&= \left( \left( \mathrm{Re}z \right)^{2} + \left( \mathrm{Im}z \right)^{2} \right)^{\frac{1}{2}} + \left( \left( \mathrm{Re}w \right)^{2} + \left( \mathrm{Im}w \right)^{2} \right)^{\frac{1}{2}}\\
&= a\left( \mathrm{Re}z + \mathrm{Im}zi \right) + a\left( \mathrm{Re}w + \mathrm{Im}wi \right)\\
&= a(z) + a(w)
\end{align*}
\end{proof}\par
最後に体に関する性質を述べよう。
\begin{thm}\label{4.1.2.14}
$K \subseteq \mathbb{C}$なる任意の体$K$は$\mathbb{Q} \subseteq K$を満たす。
\end{thm}
\begin{proof}
$K \subseteq \mathbb{C}$なる任意の体$K$が与えられたとき、体の定義より$0,1 \in K$が成り立つ。さらに、体の定義より$1 + 1 = 2 \in K$が成り立つので、自然数$n$に対し、$n = k$のとき、$k \in K$が成り立つと仮定すれば、$n = k + 1$のとき、$1 \in K$より$k + 1 \in K$が成り立つので、数学的帰納法により$\forall n \in \mathbb{N}$に対し、$n \in K$が成り立つ。さらに、$\forall n \in \mathbb{N}$に対し、$- n \in K$が成り立つので、$0 \in K$も併せて$\mathbb{Z} = - \mathbb{N} \sqcup \left\{ 0 \right\} \sqcup \mathbb{N}$が成り立つことにより、$\forall n \in \mathbb{Z}$に対し、$n \in K$が成り立つ。さらに、$\forall m \in \mathbb{Z} \setminus \left\{ 0 \right\}$に対し、$\frac{1}{m} \in K$が成り立つので、$\forall m \in \mathbb{Z}\forall n \in \mathbb{Z} \setminus \left\{ 0 \right\}$に対し、$\frac{m}{n} \in K$が成り立つ、即ち、$\forall q \in \mathbb{Q}$に対し、$q \in K$が成り立つ。以上より、$\mathbb{Q} \subseteq K$が得られる。
\end{proof}
\begin{thebibliography}{50}
  \bibitem{1}
  松坂和夫, 線型代数入門, 岩波書店, 1980. 新装版第2刷 p1-73 ISBN978-4-00-029873-8
  \bibitem{2}
  松坂和夫, 代数系入門, 岩波書店, 1976. 新装版第2刷 p39-51,65-71,107-116,170-182 ISBN978-4-00-029873-5
  \bibitem{3}
  杉浦光夫, 解析入門I, 東京大学出版社, 1985. 第34刷 p33-38 ISBN978-4-13-062005-5
  \bibitem{4}
  対馬龍司. "第9章 標準形の応用 第10章 体と多項式". 明治大学. \url{http://www.isc.meiji.ac.jp/~tsushima/senkei/furoku.pdf} (2021-12-31 0:55 取得)
\end{thebibliography}
\end{document}
