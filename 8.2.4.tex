\documentclass[dvipdfmx]{jsarticle}
\setcounter{section}{2}
\setcounter{subsection}{3}
\usepackage{xr}
\externaldocument{8.1.2}
\externaldocument{8.1.3}
\externaldocument{8.1.4}
\externaldocument{8.1.6}
\externaldocument{8.1.8}
\externaldocument{8.1.9}
\externaldocument{8.2.1}
\externaldocument{8.2.3}
\usepackage{amsmath,amsfonts,amssymb,array,comment,mathtools,url,docmute}
\usepackage{longtable,booktabs,dcolumn,tabularx,mathtools,multirow,colortbl,xcolor}
\usepackage[dvipdfmx]{graphics}
\usepackage{bmpsize}
\usepackage{amsthm}
\usepackage{enumitem}
\setlistdepth{20}
\renewlist{itemize}{itemize}{20}
\setlist[itemize]{label=•}
\renewlist{enumerate}{enumerate}{20}
\setlist[enumerate]{label=\arabic*.}
\setcounter{MaxMatrixCols}{20}
\setcounter{tocdepth}{3}
\newcommand{\rotin}{\text{\rotatebox[origin=c]{90}{$\in $}}}
\newcommand{\amap}[6]{\text{\raisebox{-0.7cm}{\begin{tikzpicture} 
  \node (a) at (0, 1) {$\textstyle{#2}$};
  \node (b) at (#6, 1) {$\textstyle{#3}$};
  \node (c) at (0, 0) {$\textstyle{#4}$};
  \node (d) at (#6, 0) {$\textstyle{#5}$};
  \node (x) at (0, 0.5) {$\rotin $};
  \node (x) at (#6, 0.5) {$\rotin $};
  \draw[->] (a) to node[xshift=0pt, yshift=7pt] {$\textstyle{\scriptstyle{#1}}$} (b);
  \draw[|->] (c) to node[xshift=0pt, yshift=7pt] {$\textstyle{\scriptstyle{#1}}$} (d);
\end{tikzpicture}}}}
\newcommand{\twomaps}[9]{\text{\raisebox{-0.7cm}{\begin{tikzpicture} 
  \node (a) at (0, 1) {$\textstyle{#3}$};
  \node (b) at (#9, 1) {$\textstyle{#4}$};
  \node (c) at (#9+#9, 1) {$\textstyle{#5}$};
  \node (d) at (0, 0) {$\textstyle{#6}$};
  \node (e) at (#9, 0) {$\textstyle{#7}$};
  \node (f) at (#9+#9, 0) {$\textstyle{#8}$};
  \node (x) at (0, 0.5) {$\rotin $};
  \node (x) at (#9, 0.5) {$\rotin $};
  \node (x) at (#9+#9, 0.5) {$\rotin $};
  \draw[->] (a) to node[xshift=0pt, yshift=7pt] {$\textstyle{\scriptstyle{#1}}$} (b);
  \draw[|->] (d) to node[xshift=0pt, yshift=7pt] {$\textstyle{\scriptstyle{#2}}$} (e);
  \draw[->] (b) to node[xshift=0pt, yshift=7pt] {$\textstyle{\scriptstyle{#1}}$} (c);
  \draw[|->] (e) to node[xshift=0pt, yshift=7pt] {$\textstyle{\scriptstyle{#2}}$} (f);
\end{tikzpicture}}}}
\renewcommand{\thesection}{第\arabic{section}部}
\renewcommand{\thesubsection}{\arabic{section}.\arabic{subsection}}
\renewcommand{\thesubsubsection}{\arabic{section}.\arabic{subsection}.\arabic{subsubsection}}
\everymath{\displaystyle}
\allowdisplaybreaks[4]
\usepackage{vtable}
\theoremstyle{definition}
\newtheorem{thm}{定理}[subsection]
\newtheorem*{thm*}{定理}
\newtheorem{dfn}{定義}[subsection]
\newtheorem*{dfn*}{定義}
\newtheorem{axs}[dfn]{公理}
\newtheorem*{axs*}{公理}
\renewcommand{\headfont}{\bfseries}
\makeatletter
  \renewcommand{\section}{%
    \@startsection{section}{1}{\z@}%
    {\Cvs}{\Cvs}%
    {\normalfont\huge\headfont\raggedright}}
\makeatother
\makeatletter
  \renewcommand{\subsection}{%
    \@startsection{subsection}{2}{\z@}%
    {0.5\Cvs}{0.5\Cvs}%
    {\normalfont\LARGE\headfont\raggedright}}
\makeatother
\makeatletter
  \renewcommand{\subsubsection}{%
    \@startsection{subsubsection}{3}{\z@}%
    {0.4\Cvs}{0.4\Cvs}%
    {\normalfont\Large\headfont\raggedright}}
\makeatother
\makeatletter
\renewenvironment{proof}[1][\proofname]{\par
  \pushQED{\qed}%
  \normalfont \topsep6\p@\@plus6\p@\relax
  \trivlist
  \item\relax
  {
  #1\@addpunct{.}}\hspace\labelsep\ignorespaces
}{%
  \popQED\endtrivlist\@endpefalse
}
\makeatother
\renewcommand{\proofname}{\textbf{証明}}
\usepackage{tikz,graphics}
\usepackage[dvipdfmx]{hyperref}
\usepackage{pxjahyper}
\hypersetup{
 setpagesize=false,
 bookmarks=true,
 bookmarksdepth=tocdepth,
 bookmarksnumbered=true,
 colorlinks=false,
 pdftitle={},
 pdfsubject={},
 pdfauthor={},
 pdfkeywords={}}
\begin{document}
%\hypertarget{ux4e00ux69d8ux9023ux7d9a}{%
\subsection{一様連続}%\label{ux4e00ux69d8ux9023ux7d9a}}
%\hypertarget{ux4e00ux69d8ux9023ux7d9a-1}{%
\subsubsection{一様連続}%\label{ux4e00ux69d8ux9023ux7d9a-1}}
\begin{dfn}
2つの距離空間たち$(S,d)$、$(T,e)$とこれらの間の写像$f:S \rightarrow T$が与えられたとする。$\forall\varepsilon \in \mathbb{R}^{+}\exists\delta \in \mathbb{R}^{+}\forall a,b \in S$に対し、$d(a,b) < \delta$が成り立つなら、$e\left( f(a),f(b) \right) < \varepsilon$が成り立つとき、その写像$f$はその距離空間$(S,d)$からその距離空間$(T,e)$へ一様連続であるといいこのような写像を一様連続写像という。
\end{dfn}\par
そこで、定理\ref{8.2.1.14}の主張によれば、その写像$f$がその集合$S$で連続であることと、$\forall\varepsilon \in \mathbb{R}^{+}\forall a \in S\exists\delta \in \mathbb{R}^{+}\forall b \in S$に対し、$d(a,b) < \delta$が成り立つなら、$e\left( f(a),f(b) \right) < \varepsilon$が成り立つことと同値であった。これに対し、その写像$f$が一様連続のときでは、$\exists\delta \in \mathbb{R}^{+}$のあとに$\forall a \in S$がきていることとなっている。このことから、次に述べるようにその写像$f$が一様連続であることはその写像$f$が連続であることより強い主張となっている。
\begin{thm}\label{8.2.4.1}
2つの距離空間たち$(S,d)$、$(T,e)$とこれらの間の写像$f:S \rightarrow T$が与えられたとする。その写像$f$が一様連続であるなら、その写像はその集合$S$上で連続である。
\end{thm}
\begin{proof} 定理\ref{8.2.1.14}と先ほどの議論よりわかる。
\end{proof}\par
距離空間$(S,d)$が与えられたとき、$\forall M \in \mathfrak{P}(S)$に対し、その集合$M$が空集合でないなら、次式のように写像$f_{M}$が定義されると、
\begin{align*}
f_{M}:S \rightarrow \mathbb{R};a \mapsto \mathrm{dist}\left( \left\{ a \right\},M \right)
\end{align*}
その写像$f_{M}$はその集合$S$上で一様連続である。これは定理\ref{8.2.3.11}の証明を追えばすぐ分かるであろう。ここで、逆は成り立たないことに注意されたい。たとえば、1次元Euclid空間$E$の部分距離空間から1次元Euclid空間$E$への関数$f:\mathbb{R}^{+} \rightarrow \mathbb{R};x \mapsto \frac{1}{x}$が与えられたとき、これは明らかに連続であるが、一様連続ではない。実際、$\exists\varepsilon \in \mathbb{R}^{+}\forall\delta \in \mathbb{R}^{+}$に対し、$0 < x < \min\left\{ \delta,\frac{1}{2\varepsilon} \right\}$なる実数$x$がとられれば、確かに$d_{E}(2x,x) = |2x - x| = x < \delta$が成り立つが、$d_{E}\left( f(2x),f(x) \right) = \left| \frac{1}{2x} - \frac{1}{x} \right| = \frac{1}{2x} > \varepsilon$が成り立つことから、そうでないと分かる。
\begin{thm}\label{8.2.4.2}
3つの距離空間たち$(S,c)$、$(T,d)$、$(U,e)$と写像たち$f:S \rightarrow T$、$g:T \rightarrow U$が与えられたとき、それらの写像たち$f:S \rightarrow T$、$g:T \rightarrow U$がどちらも一様連続写像であるなら、その合成写像$g \circ f$も一様連続写像である。
\end{thm}
\begin{proof}
3つの距離空間たち$(S,c)$、$(T,d)$、$(U,e)$と写像たち$f:S \rightarrow T$、$g:T \rightarrow U$が与えられたとする。それらの写像たち$f:S \rightarrow T$、$g:T \rightarrow U$がどちらも一様連続写像であるなら、定義より$\forall\varepsilon \in \mathbb{R}^{+}\exists\gamma,\delta \in \mathbb{R}^{+}\forall a,b \in S$に対し、$c(a,b) < \delta$が成り立つなら、$d\left( f(a),f(b) \right) < \gamma$が成り立ち、さらに、$d\left( f(a),f(b) \right) < \delta$が成り立つなら、$e\left( g\left( f(a) \right),g\left( f(b) \right) \right) < \varepsilon$が成り立つので、その合成写像$g \circ f$も一様連続写像である。
\end{proof}
\begin{thm}\label{8.2.4.3}
2つの距離空間たち$(S,d)$、$(T,e)$とこれらの間の任意の写像$f:S \rightarrow T$が与えられたとする。その距離空間$(S,d)$における位相空間$\left( S,\mathfrak{O}_{d} \right)$がcompact空間であるとき、その写像$f$が連続であるなら、その写像$f$は一様連続である。
\end{thm}
\begin{proof}
2つの距離空間たち$(S,d)$、$(T,e)$とこれらの間の任意の写像$f:S \rightarrow T$が与えられたとする。その距離空間$(S,d)$における位相空間$\left( S,\mathfrak{O}_{d} \right)$がcompact空間であるとき、その写像$f$が連続であるなら、定理\ref{8.1.3.3}より$\forall a \in S$に対し、その元$a$においてその写像$f$は連続である。したがって、$\forall\varepsilon \in \mathbb{R}^{+}\exists\delta_{a} \in \mathbb{R}^{+}\forall b \in S$に対し、$d(b,a) < \delta_{a}$が成り立つなら、$e\left( f(b),f(a) \right) < \varepsilon$が成り立つ。ここで、次式のようにその位相$\mathfrak{O}_{d}$の部分集合$\mathfrak{U}$が与えられたとき、
\begin{align*}
\mathfrak{U} = \left\{ B\left( a,\frac{\delta_{a}}{2} \right) \right\}_{a \in S}
\end{align*}
これはその集合$S$の開被覆となるが、仮定よりその集合$S$の有限な部分集合$\left\{ a_{i} \right\}_{i \in \varLambda_{n}}$が存在して次式のようにその位相$\mathfrak{O}_{d}$の部分集合$\mathfrak{U}'$が与えられたとき、
\begin{align*}
\mathfrak{U}' = \left\{ B\left( a,\frac{\delta_{a}}{2} \right) \right\}_{a \in \left\{ a_{i} \right\}_{i \in \varLambda_{n}}} = \left\{ B\left( a_{i},\frac{\delta_{a_{i}}}{2} \right) \right\}_{i \in \varLambda_{n}}
\end{align*}
これもその集合$S$の開被覆であることになる。ここで、$\delta = \min\left\{ \delta_{a_{i}} \right\}_{i \in \varLambda_{n}}$のように正の実数$\delta$が定義されれば、$\forall a,b \in S$に対し、$d(a,b) < \delta$が成り立つなら、$S = \bigcup_{} \mathfrak{U}'$が成り立つことにより、$\exists i \in \varLambda_{n}$に対し、$a \in B\left( a_{i},\frac{\delta_{a_{i}}}{2} \right)$が成り立つ、即ち、$d\left( a_{i},a \right) < \frac{\delta_{a_{i}}}{2} < \delta_{a_{i}}$が成り立つ。このとき、次のようになることから、
\begin{align*}
d\left( a_{i},b \right) &\leq d\left( a_{i},a \right) + d(a,b)\\
&< \frac{\delta_{a_{i}}}{2} + \delta\\
&\leq \frac{\delta_{a_{i}}}{2} + \frac{\delta_{a_{i}}}{2} = \delta_{a_{i}}
\end{align*}
$d\left( a_{i},a \right) < \delta_{a_{i}}$かつ$d\left( a_{i},b \right) < \delta_{a_{i}}$が得られる。ここで、仮定より$e\left( f\left( a_{i} \right),f(a) \right) < \frac{\varepsilon}{2}$かつ$e\left( f\left( a_{i} \right),f(b) \right) < \frac{\varepsilon}{2}$が成り立つので、次のようになる。
\begin{align*}
e\left( f(b),f(a) \right) &\leq e\left( f\left( a_{i} \right),f(a) \right) + e\left( f\left( a_{i} \right),f(b) \right)\\
&< \frac{\varepsilon}{2} + \frac{\varepsilon}{2} = \varepsilon
\end{align*}
以上より、$\forall\varepsilon \in \mathbb{R}^{+}\exists\delta \in \mathbb{R}^{+}\forall a,b \in S$に対し、$d(a,b) < \delta$なら$e\left( f(a),f(b) \right) < \varepsilon$が成り立つので、その写像$f$は一様連続である。
\end{proof}
%\hypertarget{ux4e00ux69d8ux540cux76f8ux5199ux50cf}{%
\subsubsection{一様同相写像}%\label{ux4e00ux69d8ux540cux76f8ux5199ux50cf}}
\begin{dfn}
2つの距離空間たち$(S,d)$、$(T,e)$とこれらの間の写像$f:S \rightarrow T$が与えられたとする。この写像$f$が全単射であるかつ、それらの写像たち$f$、$f^{- 1}$がどちらも一様連続であるとき、その写像$f$をその距離空間$(S,d)$からその距離空間$(T,e)$への一様同相写像、一様位相写像などという。
\end{dfn}
\begin{thm}\label{8.2.4.4}
2つの距離空間たち$(S,d)$、$(T,e)$とこれらの間の写像$f:S \rightarrow T$が与えられたとき、その写像$f$が一様同相写像であるならそのときに限り、これの逆写像$f^{- 1}$は一様同相写像である。
\end{thm}
\begin{proof}
2つの距離空間たち$(S,d)$、$(T,e)$とこれらの間の写像$f:S \rightarrow T$が与えられたとき、その写像$f$が一様同相写像であるならそのときに限り、定義より写像$f:S \rightarrow T$が全単射であるかつ、その写像$f$が一様連続であるかつ、これの逆写像$f^{- 1}$も一様連続であることになる。ここで、その逆写像$f^{- 1}$も全単射で$\left( f^{- 1} \right)^{- 1} = f$が成り立つので、その写像$f$が一様同相写像であるならそのときに限り、これの逆写像$f^{- 1}$も一様同相写像である。
\end{proof}
\begin{dfn}
2つの距離空間たち$(S,d)$、$(T,e)$が与えられたとき、これらの間に一様同相写像が存在するとき、これらの距離空間たち$(S,d)$、$(T,e)$は一様同相である、一様同位相であるなどといい、ここでは、$(S,d) \approx_{U}(T,e)$と書くことにする。
\end{dfn}
\begin{thm}\label{8.2.4.5}
その関係$\approx_{U}$は同値関係である、即ち、次のことが成り立つ。
\begin{itemize}
\item
  その関係$\approx_{U}$は反射的である、即ち、$(S,d) \approx_{U}(S,d)$が成り立つ。
\item
  その関係$\approx_{U}$は対称的である、即ち、$(S,d) \approx_{U}(T,e)$が成り立つなら、$(T,e) \approx_{U}(S,d)$が成り立つ。
\item
  その関係$\approx_{U}$は推移的である、即ち、$(S,c) \approx_{U}(T,d)$が成り立つかつ、$(T,d) \approx_{U}(U,e)$が成り立つなら、$(S,c) \approx_{U}(U,e)$が成り立つ。
\end{itemize}
\end{thm}
\begin{proof}
1つの距離空間$(S,d)$が与えられたとする。このとき、恒等写像$I_{S}:S \rightarrow S$は明らかに一様同相写像であるので、$(S,d) \approx_{U}(S,d)$が成り立つ。\par
2つの距離空間たち$(S,d)$、$(T,e)$が与えられたとする。$(S,d) \approx_{U}(T,e)$が成り立つなら、これらの間に一様同相写像$f:S \rightarrow T$が存在することになる。ここで、その写像$f$の逆写像$f^{- 1}:T \rightarrow S$もまた一様同相写像となるのであったので、$(T,e) \approx_{U}(S,d)$が成り立つ。\par
3つの距離空間たち$(S,c)$、$(T,d)$、$(U,e)$が与えられたとする。$(S,c) \approx_{U}(T,d)$が成り立つかつ、$(T,d) \approx_{U}(U,e)$が成り立つなら、一様同相写像たち$f:S \rightarrow T$、$g:T \rightarrow U$が存在することになる。ここで、その合成写像$g \circ f:S \rightarrow U$もまた一様同相写像となるのであったので、$(S,d) \approx_{U}(U,e)$が成り立つ。
\end{proof}
\begin{thm}\label{8.2.4.6}
2つの距離空間たち$(S,d)$\emph{、}$(T,e)$が与えられたとする。その距離空間$(S,d)$における位相空間$\left( S,\mathfrak{O}_{d} \right)$がcompact空間であるとき、$\left( S,\mathfrak{O}_{d} \right) \approx \left( T,\mathfrak{O}_{e} \right)$が成り立つならそのときに限り、$(S,d) \approx_{U}(T,e)$が成り立つ。
\end{thm}
\begin{proof}
2つの距離空間たち$(S,d)$、$(T,e)$が与えられたとする。その距離空間$(S,d)$における位相空間$\left( S,\mathfrak{O}_{d} \right)$がcompact空間であるとき、$\left( S,\mathfrak{O}_{d} \right) \approx \left( T,\mathfrak{O}_{e} \right)$が成り立つならそのときに限り、これらの距離空間たち$(S,d)$、$(T,e)$の間の連続な写像$f:S \rightarrow T$が存在する。定理\ref{8.2.4.3}よりその写像$f$は一様連続でもあったので、これが成り立つならそのときに限り、$(S,d) \approx_{U}(T,e)$が成り立つ。
\end{proof}
%\hypertarget{ux5b8cux5099ux8dddux96e2ux7a7aux9593}{%
\subsubsection{完備距離空間}%\label{ux5b8cux5099ux8dddux96e2ux7a7aux9593}}
\begin{dfn}
距離空間$(S,d)$とその集合$S$の元の列$\left( a_{n} \right)_{n \in \mathbb{N}}$が与えられたとき、$\forall\varepsilon \in \mathbb{R}^{+}\exists n_{0} \in \mathbb{N}\forall m,n \in \mathbb{N}$に対し、$n_{0} < m$かつ$n_{0} < n$が成り立つなら、$d\left( a_{m},a_{n} \right) < \varepsilon$が成り立つようなその元の列$\left( a_{n} \right)_{n \in \mathbb{N}}$をその距離空間$(S,d)$におけるCauchy列、基本点列などという。
\end{dfn}
\begin{thm}\label{8.2.4.7}
距離空間$(S,d)$におけるその集合$S$の元の列$\left( a_{n} \right)_{n \in \mathbb{N}}$が収束するとき、その元の列$\left( a_{n} \right)_{n \in \mathbb{N}}$はCauchy列である。
\end{thm}\par
ここで、任意のCauchy列は収束するとは限らないことに注意されたい。例えば、距離空間$\left(\mathbb{Q},\left( x,y\right) \mapsto \left| x - y \right| \right)$を考えたとき、床関数$\lfloor \bullet \rfloor$を用いた有理数列$\left(\frac{\lfloor n\sqrt{2} \rfloor}{n}\right)_{n\in \mathbb{N}}$はCauchy列であるが、これは有理数に収束しない。これ以外にも、距離空間$\left((0,1),\left( x,y\right) \mapsto \left| x - y \right| \right)$を考えたとき、実数列$\left( \frac{1}{n} \right)_{n\in \mathbb{N}}$は確かにCauchy列となっているものの、距離空間$\left(\mathbb{R},\left( x,y\right) \mapsto \left| x - y \right| \right)$でいえば$0$に収束していて、その距離空間$\left((0,1),\left( x,y\right) \mapsto \left| x - y \right| \right)$では収束しないことになってしまう。
\begin{proof}
距離空間$(S,d)$におけるその集合$S$の元の列$\left( a_{n} \right)_{n \in \mathbb{N}}$が収束するとき、$\forall\varepsilon \in \mathbb{R}^{+}\exists n_{0} \in \mathbb{N}\forall m,n \in \mathbb{N}$に対し、$m < n_{0}$かつ$n < n_{0}$が成り立つなら、定義より$d\left( a_{m},\lim_{n \rightarrow \infty}a_{n} \right) < \frac{\varepsilon}{2}$かつ$\left( a_{n},\lim_{n \rightarrow \infty}a_{n} \right) < \frac{\varepsilon}{2}$が成り立つ。このとき、次のようになることから、
\begin{align*}
d\left( a_{m},a_{n} \right) &\leq d\left( a_{m},\lim_{n \rightarrow \infty}a_{n} \right) + d\left( a_{n},\lim_{n \rightarrow \infty}a_{n} \right)\\
&< \frac{\varepsilon}{2} + \frac{\varepsilon}{2} = \varepsilon
\end{align*}
その元の列$\left( a_{n} \right)_{n \in \mathbb{N}}$はCauchy列である。
\end{proof}
\begin{dfn}
距離空間$(S,d)$において任意のCauchy列$\left( a_{n} \right)_{n \in \mathbb{N}}$が収束するとき、その距離空間$(S,d)$は完備であるといい、そのような距離空間$(S,d)$を完備距離空間という。
\end{dfn}
\begin{thm}\label{8.2.4.8}
完備距離空間$\left( S^{*},d^{*} \right)$に一様同相な距離空間$(S,d)$は完備である。
\end{thm}\par
ここで、同相であるのみの場合では、成り立たない場合があることに注意されたい。例えば、距離空間たち$\left([0,1),\left( x,y\right) \mapsto \left| x - y \right| \right)$、$\left([0,\infty ),\left( x,y\right) \mapsto \left| x - y \right| \right)$について、写像$f$を次のように定義されると、
\begin{align*}
f:[0,1) \rightarrow [0,\infty) ;x\mapsto \frac{x}{1 - x} 
\end{align*}
その写像$f$の逆写像$f^{-1}$が次のように存在しており
\begin{align*}
f^{-1}:[0,\infty ) \rightarrow [0,1) ;x\mapsto \frac{x}{1 + x} 
\end{align*}
これらの写像たちはそれらの距離空間たち$\left([0,1),\left( x,y\right) \mapsto \left| x - y \right| \right)$、$\left([0,\infty ),\left( x,y\right) \mapsto \left| x - y \right| \right)$における位相空間の間で同相写像である。しかしながら、その距離空間$\left([0,1),\left( x,y\right) \mapsto \left| x - y \right| \right)$は実数列$\left( 1-\frac{1}{n} \right)_{n\in \mathbb{N}}$はCauchy列であるが距離空間$\left(\mathbb{R},\left( x,y\right) \mapsto \left| x - y \right| \right)$でいえば$1$に収束していて、その距離空間$\left([0,1),\left( x,y\right) \mapsto \left| x - y \right| \right)$では収束しないといった例が挙げられるように完備ではなく、一方で、その距離空間$\left([0,\infty ),\left( x,y\right) \mapsto \left| x - y \right| \right)$は完備である。
\begin{proof}
完備距離空間$\left( S^{*},d^{*} \right)$に一様同相な距離空間$(S,d)$において、その距離空間$(S,d)$におけるその集合$S$の任意のCauchy列$\left( a_{n} \right)_{n \in \mathbb{N}}$が与えられたとき、$\forall\varepsilon' \in \mathbb{R}^{+}\exists n_{0} \in \mathbb{N}\forall m,n \in \mathbb{N}$に対し、$n_{0} < m$かつ$n_{0} < n$が成り立つなら、$d\left( a_{m},a_{n} \right) < \varepsilon'$が成り立つ。ここで、その距離空間$(S,d)$からその距離空間$\left( S^{*},d^{*} \right)$への一様同相写像$f:S \rightarrow S^{*}$が存在するので、$\forall\varepsilon \in \mathbb{R}^{+}\exists\varepsilon' \in \mathbb{R}^{+}$に対し、$d\left( a_{m},a_{n} \right) < \varepsilon'$が成り立つなら、$d^{*}\left( f\left( a_{m} \right),f\left( a_{n} \right) \right) < \varepsilon$が成り立つので、その集合$S^{*}$の元の列$\left( f\left( a_{n} \right) \right)_{n \in \mathbb{N}}$はCauchy列である。ここで、その距離空間$\left( S^{*},d^{*} \right)$は完備であるから、$\lim_{n \rightarrow \infty}{f\left( a_{n} \right)} = a^{*} \in S^{*}$が成り立つ。このとき、$\forall\varepsilon' \in \mathbb{R}^{+}\exists n_{0} \in \mathbb{N}\forall n \in \mathbb{N}$に対し、$n_{0} < n$が成り立つなら、$d^{*}\left( f\left( a_{n} \right),a^{*} \right) < \varepsilon'$が成り立つことになる。ここで、その逆写像$f^{- 1}$も一様連続であるから、$\forall\varepsilon \in \mathbb{R}^{+}\exists\varepsilon' \in \mathbb{R}^{+}$に対し、$d^{*}\left( f\left( a_{n} \right),a^{*} \right) < \varepsilon'$が成り立つなら、$d\left( a_{n},f^{- 1}\left( a^{*} \right) \right) < \varepsilon$が成り立つ。これにより、$\lim_{n \rightarrow \infty}{f\left( a_{n} \right)} = f^{- 1}\left( a^{*} \right)$が成り立つので、その距離空間$(S,d)$は完備である。
\end{proof}
\begin{thm}\label{8.2.4.9}
完備距離空間$(S,d)$が与えられたとき、その部分距離空間$\left( M,d_{M} \right)$が完備であるならそのときに限り、その集合$M$がその距離空間$(S,d)$における位相空間$\left( S,\mathfrak{O}_{d} \right)$での閉集合である。
\end{thm}
\begin{proof}
完備距離空間$(S,d)$が与えられたとき、その集合$S$の部分集合$M$がその距離空間$(S,d)$における位相空間$\left( S,\mathfrak{O}_{d} \right)$での閉集合であるなら、その部分距離空間$\left( M,d_{M} \right)$における任意のCauchy列$\left( a_{n} \right)_{n \in \mathbb{N}}$において、この元の列はその完備距離空間$(S,d)$における元の列でもあるので、$\lim_{n \rightarrow \infty}a_{n} = a \in S$が成り立つ。これにより、$\forall\varepsilon \in \mathbb{R}^{+}\exists n_{0} \in \mathbb{N}\forall n \in \mathbb{N}$に対し、$n_{0} < n$が成り立つなら、$d\left( a_{n},a \right) < \varepsilon$が成り立つことになる。このとき、$a_{n} \in B(a,\varepsilon)$が成り立つことになり、定理\ref{8.1.2.15}、定理\ref{8.2.1.6}より$a \in {\mathrm{cl}}M = M$が成り立つ。ゆえに、そのCauchy列$\left( a_{n} \right)_{n \in \mathbb{N}}$はその部分距離空間$\left( M,d_{M} \right)$で収束するので、その部分距離空間$\left( M,d_{M} \right)$は完備である。\par
逆に、その集合$M$が閉集合でなければ、${\mathrm{cl}}M \setminus M \neq \emptyset$が成り立つので、$\forall a \in {\mathrm{cl}}M \setminus M$に対し、その元$a$はその集合$M$の触点であるから、定理\ref{8.2.1.13}よりその元$a$がその集合$M$のある元の列$\left( a_{n} \right)_{n \in \mathbb{N}}$が存在してこれの極限となる。定理\ref{8.2.4.7}よりその部分距離空間$\left( M,d_{M} \right)$のその元の列$\left( a_{n} \right)_{n \in \mathbb{N}}$がCauchy列であるかつ、その元の列$\left( a_{n} \right)_{n \in \mathbb{N}}$はその部分距離空間$\left( M,d_{M} \right)$で収束しないことになるので、その部分距離空間$\left( M,d_{M} \right)$は完備でない。
\end{proof}
\begin{thm}\label{8.2.4.10}
添数集合$\varLambda_{n}$によって添数づけられた距離空間たちの族$\left\{ \left( S_{i},d_{i} \right) \right\}_{i \in \varLambda_{n}}$が与えられたとき、これの直積距離空間$\left( \prod_{i \in \varLambda_{n}} S_{i},d \right)$について、次のことが成り立つ。
\begin{itemize}
\item
  その直積距離空間$\left( \prod_{i \in \varLambda_{n}} S_{i},d \right)$における元の列$\left( \left( a_{i,m} \right)_{i \in \varLambda_{n}} \right)_{m \in \mathbb{N}}$がCauchy列であるならそのときに限り、$\forall i \in \varLambda_{n}$に対し、その距離空間$\left( S_{i},d_{i} \right)$における元の列$\left( a_{i,m} \right)_{m \in \mathbb{N}}$はCauchy列である。
\item
  その直積距離空間$\left( \prod_{i \in \varLambda_{n}} S_{i},d \right)$が完備であるならそのときに限り、$\forall i \in \varLambda_{n}$に対し、その距離空間$\left( S_{i},d_{i} \right)$は完備である。
\end{itemize}
\end{thm}
\begin{proof}
添数集合$\varLambda_{n}$によって添数づけられた距離空間たちの族$\left\{ \left( S_{i},d_{i} \right) \right\}_{i \in \varLambda_{n}}$が与えられたとき、これの直積距離空間$\left( \prod_{i \in \varLambda_{n}} S_{i},d \right)$における元の列$\left( \left( a_{i,m} \right)_{i \in \varLambda_{n}} \right)_{m \in \mathbb{N}}$がCauchy列であるなら、$\forall\varepsilon \in \mathbb{R}^{+}\exists n_{0} \in \mathbb{N}\forall l,m \in \mathbb{N}$に対し、$n_{0} < m$かつ$n_{0} < n$が成り立つなら、$d\left( \left( a_{i,l} \right)_{i \in \varLambda_{n}},\ \ \left( a_{i,m} \right)_{i \in \varLambda_{n}} \right) < \varepsilon$が成り立つ。ここで、$\forall i \in \varLambda_{n}$に対し、$d_{i}\left( a_{i,l},a_{i,m} \right) \leq d\left( \left( a_{i,l} \right)_{i \in \varLambda_{n}},\left( a_{i,m} \right)_{i \in \varLambda_{n}} \right) < \varepsilon$が成り立つので、その距離空間$\left( S_{i},d_{i} \right)$における元の列$\left( a_{i,m} \right)_{m \in \mathbb{N}}$はCauchy列である。逆に、$\forall i \in \varLambda_{n}$に対し、その距離空間$\left( S_{i},d_{i} \right)$における元の列$\left( a_{i,m} \right)_{m \in \mathbb{N}}$がCauchy列であるなら、$\forall\varepsilon \in \mathbb{R}^{+}\exists n_{0} \in \mathbb{N}\forall l,m \in \mathbb{N}$に対し、$n_{0} < m$かつ$n_{0} < n$が成り立つなら、$d_{i}\left( a_{i,l},a_{i,m} \right) < \frac{\varepsilon}{\sqrt{n}}$が成り立つ。そのとき、次のようになるので、
\begin{align*}
d\left( \left( a_{i,l} \right)_{i \in \varLambda_{n}},\left( a_{i,m} \right)_{i \in \varLambda_{n}} \right) &= \sqrt{\sum_{i \in \varLambda_{n}} {d_{i}\left( a_{i,l},a_{i,m} \right)}^{2}}\\
&< \sqrt{\sum_{i \in \varLambda_{n}} \left( \frac{\varepsilon}{\sqrt{n}} \right)^{2}}\\
&= \sqrt{n\left( \frac{\varepsilon}{\sqrt{n}} \right)^{2}}\\
&= \sqrt{n\frac{\varepsilon^{2}}{n}} = \varepsilon
\end{align*}
その直積距離空間$\left( \prod_{i \in \varLambda_{n}} S_{i},d \right)$における元の列$\left( \left( a_{i,m} \right)_{i \in \varLambda_{n}} \right)_{m \in \mathbb{N}}$はCauchy列である。\par
その直積距離空間$\left( \prod_{i \in \varLambda_{n}} S_{i},d \right)$が完備であるなら、$\forall i \in \varLambda_{n}$に対し、その距離空間$\left( S_{i},d_{i} \right)$の任意のCauchy列$\left( a_{i,m} \right)_{m \in \mathbb{N}}$が与えられたとき、直積$\prod_{i' \in \varLambda_{n} \setminus \left\{ i \right\}} S_{i'}$の元$\left( a_{i'} \right)_{i' \in \varLambda_{n} \setminus \left\{ i \right\}}$を用いて$\left( a_{i'} \right)_{i' \in \varLambda_{n} \setminus \left\{ i \right\}} = \left( a_{i',m} \right)_{i' \in \varLambda_{n} \setminus \left\{ i \right\}}$とおかれると、$\forall i' \in \varLambda_{n} \setminus \left\{ i \right\}$に対し、その距離空間$\left( S_{i'},d_{i'} \right)$における元の列$\left( a_{i',m} \right)_{m \in \mathbb{N}}$は明らかにCauchy列であるから、上記の議論により、その直積距離空間$\left( \prod_{i \in \varLambda_{n}} S_{i},d \right)$の元の列$\left( \left( a_{i,m} \right)_{i \in \varLambda_{n}} \right)_{m \in \mathbb{N}}$はCauchy列である。したがって、仮定よりこれは収束し$\lim_{m \rightarrow \infty}\left( a_{i,m} \right)_{i \in \varLambda_{n}} = \left( a_{i} \right)_{i \in \varLambda_{n}} \in \prod_{i \in \varLambda_{n}} S_{i}$が成り立つ。あとは、定理\ref{8.2.1.23}より$\lim_{m \rightarrow \infty}a_{i,m} = a_{i} \in S_{i}$が成り立つので、その距離空間$\left( S_{i},d_{i} \right)$は完備である。逆に、$\forall i \in \varLambda_{n}$に対し、その距離空間$\left( S_{i},d_{i} \right)$は完備であるなら、その直積距離空間$\left( \prod_{i \in \varLambda_{n}} S_{i},d \right)$の任意のCauchy列$\left( \left( a_{i,m} \right)_{i \in \varLambda_{n}} \right)_{m \in \mathbb{N}}$が与えられたとき、$\forall i \in \varLambda_{n}$に対し、上記に議論によりその距離空間$\left( S_{i},d_{i} \right)$における元の列$\left( a_{i,m} \right)_{m \in \mathbb{N}}$はCauchy列である。したがって、仮定よりこれは収束し、$\lim_{m \rightarrow \infty}a_{i,m} = a_{i}$とおかれると、定理\ref{8.2.1.23}より$\lim_{m \rightarrow \infty}\left( a_{i,m} \right)_{i \in \varLambda_{n}} = \left( a_{i} \right)_{i \in \varLambda_{n}} \in \prod_{i \in \varLambda_{n}} S_{i}$が成り立つので、その直積距離空間$\left( \prod_{i \in \varLambda_{n}} S_{i},d \right)$は完備である。
\end{proof}
%\hypertarget{ux5168ux6709ux754cux8dddux96e2ux7a7aux9593}{%
\subsubsection{全有界距離空間}%\label{ux5168ux6709ux754cux8dddux96e2ux7a7aux9593}}
\begin{dfn}
距離空間$(S,d)$が与えられたとき、ある正の実数$\varepsilon$が存在してその集合$S$の被覆$\mathfrak{U}$の任意の元$U$に対し、これの直径$\delta(U)$がその正の実数$\varepsilon$未満である、即ち、$\delta(U) < \varepsilon$が成り立つとき、その被覆$\mathfrak{U}$をその距離空間$(S,d)$の$\varepsilon$被覆という。
\end{dfn}
\begin{dfn}
距離空間$(S,d)$が与えられたとき、任意の正の実数$\varepsilon$に対し、その集合$S$の有限な$\varepsilon$被覆が存在するとき、その距離空間$(S,d)$は全有界であるといい、そのような距離空間を全有界距離空間、pre-compact空間などという\footnote{そのpre-compact空間という語句は書籍によって別の意味で用いられている場合があることに注意してください…。}。
\end{dfn}
\begin{dfn}
距離空間$(S,d)$のその集合$S$の部分集合$M$が与えられたとき、$\forall a \in S$に対し、$\mathrm{dist}\left( \left\{ a \right\},M \right) < \varepsilon$が成り立つとき、その部分集合$M$はその距離空間$(S,d)$で$\varepsilon$網であるという。
\end{dfn}
\begin{thm}\label{8.2.4.11}
距離空間$(S,d)$が与えられたとき、$\forall M \in \mathfrak{P}(S)$に対し、その部分集合$M$がその距離空間$(S,d)$で$\varepsilon$網であるならそのときに限り、$\bigcup_{a \in M} {B(a,\varepsilon)} = S$が成り立つ。
\end{thm}
\begin{proof}
距離空間$(S,d)$が与えられたとき、$\exists M \in \mathfrak{P}(S)\exists a \in S\forall b \in M$に対し、$\mathrm{dist}\left( \left\{ a \right\},M \right) < \varepsilon$かつ$\varepsilon \leq d(a,b)$が成り立つとすれば、次のようになるが、
\begin{align*}
\mathrm{dist}\left( \left\{ a \right\},M \right) < \varepsilon \leq \inf{V\left( d|\left\{ a \right\} \times M \right)} = \mathrm{dist}\left( \left\{ a \right\},M \right) \leq d(a,b)
\end{align*}
これは矛盾している。ゆえに、$\forall M \in \mathfrak{P}(S)$に対し、その部分集合$M$がその距離空間$(S,d)$で$\varepsilon$網であるなら、$\forall a \in S$に対し、$\mathrm{dist}\left( \left\{ a \right\},M \right) < \varepsilon$が成り立って、$\exists b \in M$に対し、$d(a,b) < \varepsilon$が成り立つ。これにより、$a \in \bigcup_{b \in M} {B(b,\varepsilon)}$が成り立つので、$\bigcup_{a \in M} {B(a,\varepsilon)} \supseteq S$が成り立つ。$\bigcup_{a \in M} {B(a,\varepsilon)} \subseteq S$が成り立つことは定義より明らかであるので、$\bigcup_{a \in M} {B(a,\varepsilon)} = S$が成り立つ。\par
逆に、$\bigcup_{a \in M} {B(a,\varepsilon)} = S$が成り立つなら、$\forall a \in S$に対し、$a \in \bigcup_{b \in M} {B(b,\varepsilon)}$が成り立つので、$\exists b \in M$に対し、$d(a,b) < \varepsilon$が成り立って、$\mathrm{dist}\left( \left\{ a \right\},M \right) = \inf{V\left( d|\left\{ a \right\} \times M \right)} \leq d(a,b)$より$\mathrm{dist}\left( \left\{ a \right\},M \right) < \varepsilon$が成り立ち、したがって、その部分集合$M$はその距離空間$(S,d)$で$\varepsilon$網である。
\end{proof}
\begin{thm}\label{8.2.4.12}
距離空間$(S,d)$が与えられたとき、次のことは同値である。
\begin{itemize}
\item
  その距離空間$(S,d)$は全有界である。
\item
  $\forall\varepsilon \in \mathbb{R}^{+}$に対し、その集合$S$の有限な部分集合$M$が存在して、これがその距離空間$(S,d)$で$\varepsilon$網である。
\item
  $\forall\varepsilon \in \mathbb{R}^{+}$に対し、その集合$S$の有限な部分集合$M$が存在して、$\bigcup_{a \in M} {B(a,\varepsilon)} = S$が成り立つ。
\end{itemize}
\end{thm}
\begin{proof}
距離空間$(S,d)$が与えられたとき、その距離空間$(S,d)$が全有界であるなら、任意の正の実数$\varepsilon$に対し、その集合$S$の有限な$\varepsilon$被覆$\mathfrak{U}'$が存在する、即ち、$\bigcup_{} \mathfrak{U}' = S$が成り立って、$\forall U \in \mathfrak{U}'$に対し、$\delta(U) < \varepsilon$が成り立つ。ここで、その$\varepsilon$被覆$\mathfrak{U}'$の各元$U$に対し、定義よりその集合$U$が空集合でないとしてもよくそうすると、その集合$U$の元$a_{U}$が存在して族$\left\{ a_{U} \right\}_{U \in \mathfrak{U}'}$が考えられれば、$\forall a \in S$に対し、$a \in \bigcup_{} \mathfrak{U}'$が成り立つので、$\exists U \in \mathfrak{U}'$に対し、$a \in U$が成り立つかつ、$\varepsilon$被覆の定義より$\delta(U) < \varepsilon$が成り立つ。したがって、$d\left( a,a_{U} \right) \leq \delta(U) < \varepsilon$が成り立つので、$\mathrm{dist}\left( \left\{ a \right\},\left\{ a_{U} \right\} \right) < \varepsilon$が成り立つので、$\mathrm{dist}\left( \left\{ a \right\},\left\{ a_{U} \right\}_{U \in \mathfrak{U}'} \right) \leq \mathrm{dist}\left( \left\{ a \right\},\left\{ a_{U} \right\} \right) < \varepsilon$が成り立つ。ゆえに、$\forall\varepsilon \in \mathbb{R}^{+}$に対し、その部分集合$\left\{ a_{U} \right\}_{U \in \mathfrak{U}'}$は有限でその距離空間$(S,d)$で$\varepsilon$網である。\par
定理\ref{8.2.4.11}より明らかに、$\forall\varepsilon \in \mathbb{R}^{+}$に対し、その集合$S$の有限な部分集合$M$が存在してこれがその距離空間$(S,d)$で$\varepsilon$網であるなら、$\forall\varepsilon \in \mathbb{R}^{+}$に対し、その集合$S$の有限な部分集合$M$が存在して$\bigcup_{a \in M} {B(a,\varepsilon)} = S$が成り立つ。\par
$\forall\varepsilon \in \mathbb{R}^{+}$に対し、その集合$S$の有限な部分集合$M$が存在して$\bigcup_{a \in M} {B(a,\varepsilon)} = S$が成り立つなら、その正の実数$\varepsilon$の任意性よりその集合$S$の有限な部分集合$M$が存在して$\bigcup_{a \in M} {B\left( a,\frac{\varepsilon}{2} \right)} = S$が成り立つとしてもよい。このとき、$\forall b,c \in B\left( a,\frac{\varepsilon}{2} \right)$に対し、$d(b,c) \leq d(a,b) + d(a,c) < \frac{\varepsilon}{2} + \frac{\varepsilon}{2} = \varepsilon$が成り立つので、$\delta\left( B\left( a,\frac{\varepsilon}{2} \right) \right) < \varepsilon$が成り立つ。これにより、族$\left\{ B\left( a,\frac{\varepsilon}{2} \right) \right\}_{a \in M}$がその集合$S$の有限な$\varepsilon$被覆となるので、その距離空間$(S,d)$は全有界である。
\end{proof}
\begin{thm}\label{8.2.4.13}
全有界距離空間$(S,d)$の部分距離空間$\left( M,d_{M} \right)$も全有界である。
\end{thm}
\begin{proof}
全有界距離空間$(S,d)$が与えられたとき、任意の正の実数$\varepsilon$に対し、その集合$S$の有限な$\varepsilon$被覆$\mathfrak{U}'$が存在する。ここで、その被覆$\mathfrak{U}'$はその集合$M$の$\varepsilon$被覆でもあるので、その部分距離空間$\left( M,d_{M} \right)$も全有界である。
\end{proof}
\begin{thm}\label{8.2.4.14} 全有界距離空間$(S,d)$でのその集合$S$は有界である。
\end{thm}
\begin{proof}
全有界距離空間$(S,d)$が与えられたとき、1つの正の実数$\varepsilon$を用いてその集合$S$の有限な$\varepsilon$被覆$\mathfrak{U}'$が存在する。このとき、$S = \bigcup_{} \mathfrak{U}'$が成り立ち、$\forall U \in \mathfrak{U}'$に対し、$\delta(U) < \varepsilon < \infty$が成り立つので、定理\ref{8.2.3.8}よりその集合$S$は有界である。
\end{proof}\par
ただし、距離空間$(S,d)$のその集合$S$が有界であるとき、その距離空間$(S,d)$は全有界であるとは限らないことに注意されたい。例えば、次のような写像$\rho $をもつ組$\left( \mathbb{N} ,\rho \right)$は距離空間となっている。
\begin{align*}
\rho : \mathbb{N}\times \mathbb{N} \rightarrow \mathbb{R} ; \left(m,n\right) \mapsto \rho \left(m,n\right) = \left\{ \begin{matrix}
0 & \mathrm{if} & m = n \\
1 & \mathrm{if} & m \ne n \\
\end{matrix} \right.
\end{align*}
さらに、$V\left(\rho \right)=\left\{0,1 \right\}$なので、$\delta(\mathbb{N} ) = 1$よりその集合$\mathbb{N}$はその距離空間$\left( \mathbb{N} ,\rho \right)$で有界である。しかしながら、その集合$\mathbb{N}$のどの有限な部分集合$\left\{ k_i \right\}_{i = 1}^{n} $に対し、$\bigcup_{i = 1}^{n} B(k_i,1) = \left\{k_i \right\}_{i = 1}^{n} \ne \mathbb{N}$となるので、その距離空間$\left( \mathbb{N} ,\rho \right)$は全有界でない。
\begin{thm}\label{8.2.4.15}
全有界距離空間$\left( S^{*},d^{*} \right)$に一様同相な距離空間$(S,d)$は全有界である。
\end{thm}\par
ここでも、同相であるのみの場合では、成り立たない場合があることに注意されたい。例えば、距離空間たち$\left([0,1),\left( x,y\right) \mapsto \left| x - y \right| \right)$、$\left([0,\infty ),\left( x,y\right) \mapsto \left| x - y \right| \right)$について、次のような写像$f$は上記の通りそれらの距離空間たちにおける位相空間の間で同相写像である。
\begin{align*}
f:[0,1) \rightarrow [0,\infty) ;x\mapsto \frac{x}{1 - x} 
\end{align*}
ここで、距離空間$\left([0,1],\left( x,y\right) \mapsto \left| x - y \right| \right)$は全有界であるので、定理\ref{8.2.4.13}よりその距離空間$\left([0,1),\left( x,y\right) \mapsto \left| x - y \right| \right)$も全有界である。しかしながら、その距離空間$\left([0,\infty),\left( x,y\right) \mapsto \left| x - y \right| \right)$は定理\ref{8.2.4.14}より全有界でない。
\begin{proof}
全有界距離空間$\left( S^{*},d^{*} \right)$に一様同相な距離空間$(S,d)$において、任意の正の実数$\varepsilon$に対し、その集合$S^{*}$の有限な$\varepsilon$被覆$\mathfrak{U}'$が存在する、即ち、$S = \bigcup_{} \mathfrak{U}'$が成り立ち$\delta(U) < \varepsilon$が成り立つ。ここで、その距離空間$(S,d)$からその距離空間$\left( S^{*},d^{*} \right)$への一様同相写像$f:S \rightarrow S^{*}$が存在するので、$\forall\varepsilon \in \mathbb{R}^{+}\exists\delta \in \mathbb{R}^{+}\forall a,b \in S^{*}$に対し、$d^{*}(a,b) < \delta$なら$d\left( f^{- 1}(a),f^{- 1}(b) \right) < \varepsilon$が成り立つ。ここで、その距離空間$\left( S^{*},d^{*} \right)$は全有界であるから、その集合$S^{*}$の有限な部分集合$\left\{ a_{i} \right\}_{i \in \varLambda_{n}}$が存在して$\bigcup_{i \in \varLambda_{n}} {B\left( a_{i},\delta \right)} = S$が成り立つ。このとき、$\forall a,b \in S^{*}$に対し、$d^{*}(a,b) < \delta \Rightarrow d\left( f^{- 1}(a),f^{- 1}(b) \right) < \varepsilon$が成り立つことから、$\forall i \in \varLambda_{n}$に対し、$B\left( a_{i},\delta \right) \subseteq V\left( f|B\left( f^{- 1}\left( a_{i} \right),\varepsilon \right) \right)$が成り立ち、したがって、次のようになる。
\begin{align*}
V\left( f^{- 1}|B\left( a_{i},\delta \right) \right) &\subseteq V\left( f^{- 1}|V\left( f|B\left( f^{- 1}\left( a_{i} \right),\varepsilon \right) \right) \right)\\
&= B\left( f^{- 1}\left( a_{i} \right),\varepsilon \right)
\end{align*}
ここで、その写像$f$が全単射であるから、$\forall c \in S\exists c' \in S^{*}$に対し、$c = f^{- 1}\left( c' \right)$が成り立つかつ、$\bigcup_{i \in \varLambda_{n}} {B\left( a_{i},\delta \right)} = S$が成り立つので、$\exists i \in \varLambda_{n}$に対し、$c' \in B\left( a_{i},\delta \right)$が成り立つ。したがって、次のようになるので、
\begin{align*}
c &= f^{- 1}\left( c' \right) \in V\left( f^{- 1}|B\left( a_{i},\delta \right) \right) \subseteq B\left( f^{- 1}\left( a_{i} \right),\varepsilon \right)\\
&\subseteq \bigcup_{i \in \varLambda_{n}} {B\left( f^{- 1}\left( a_{i} \right),\varepsilon \right)}
\end{align*}
$S \subseteq \bigcup_{i \in \varLambda_{n}} {B\left( f^{- 1}\left( a_{i} \right),\varepsilon \right)}$が成り立つ。もちろん、これは$S = \bigcup_{i \in \varLambda_{n}} {B\left( f^{- 1}\left( a_{i} \right),\varepsilon \right)}$が成り立つことを意味する。よって、定理\ref{8.2.4.12}よりその距離空間$(S,d)$は全有界である。
\end{proof}
\begin{thm}\label{8.2.4.16}
全有界距離空間$(S,d)$における位相空間$\left( S,\mathfrak{O}_{d} \right)$は第2可算公理を満たす。
\end{thm}
\begin{proof}
全有界距離空間$(S,d)$における位相空間$\left( S,\mathfrak{O}_{d} \right)$が与えられたとき、定理\ref{8.2.4.12}より$\forall n \in \mathbb{N}$に対し、その集合$S$の有限な部分集合$M_{\frac{1}{n}}$が存在してこれがその距離空間$(S,d)$で$\frac{1}{n}$網である、即ち、$\forall a \in S$に対し、$\mathrm{dist}\left( \left\{ a \right\},M_{\frac{1}{n}} \right) < \frac{1}{n}$が成り立つ。\par
ここで、$\forall a \in S$に対し、$M_{\frac{1}{n}} \subseteq \bigcup_{n \in \mathbb{N}} M_{\frac{1}{n}}$が成り立つことにより、次式が成り立つ。
\begin{align*}
0 &\leq \mathrm{dist}\left( \left\{ a \right\},\bigcup_{n \in \mathbb{N}} M_{\frac{1}{n}} \right)\\
&\leq \mathrm{dist}\left( \left\{ a \right\},M_{\frac{1}{n}} \right) < \frac{1}{n}
\end{align*}
したがって、$0 = \mathrm{dist}\left( \left\{ a \right\},\bigcup_{n \in \mathbb{N}} M_{\frac{1}{n}} \right)$が成り立つ。ここで、定理\ref{8.2.3.5}より$a \in {\mathrm{cl}}{\bigcup_{n \in \mathbb{N}} M_{\frac{1}{n}}}$が成り立つことから、$S \subseteq {\mathrm{cl}}{\bigcup_{n \in \mathbb{N}} M_{\frac{1}{n}}}$が成り立つ。また、明らかに${\mathrm{cl}}{\bigcup_{n \in \mathbb{N}} M_{\frac{1}{n}}} \subseteq S$が成り立つので、$S = {\mathrm{cl}}{\bigcup_{n \in \mathbb{N}} M_{\frac{1}{n}}}$が得られる。これにより、その集合$\bigcup_{n \in \mathbb{N}} M_{\frac{1}{n}}$はその集合$S$で稠密である。\par
以上の議論と、$\forall n \in \mathbb{N}$に対し、その集合$M_{\frac{1}{n}}$は有限集合であるので、その和集合$\bigcup_{n \in \mathbb{N}} M_{\frac{1}{n}}$もたかだか可算であることに注意すれば、その位相空間$\left( S,\mathfrak{O}_{d} \right)$は可分である。よって、定理\ref{8.2.1.9}より全有界距離空間$(S,d)$における位相空間$\left( S,\mathfrak{O}_{d} \right)$は第2可算公理を満たす。
\end{proof}
\begin{thm}\label{8.2.4.17}
添数集合$\varLambda_{n}$によって添数づけられた距離空間たちの族$\left\{ \left( S_{i},d_{i} \right) \right\}_{i \in \varLambda_{n}}$が与えられたとき、これの直積距離空間$\left( \prod_{i \in \varLambda_{n}} S_{i},d \right)$が全有界であるならそのときに限り、$\forall i \in \varLambda_{n}$に対し、その距離空間$\left( S_{i},d_{i} \right)$は全有界である。
\end{thm}
\begin{proof}
添数集合$\varLambda_{n}$によって添数づけられた距離空間たちの族$\left\{ \left( S_{i},d_{i} \right) \right\}_{i \in \varLambda_{n}}$が与えられたとき、これの直積距離空間$\left( \prod_{i \in \varLambda_{n}} S_{i},d \right)$が全有界であるなら、定理\ref{8.2.4.12}より$\forall\varepsilon \in \mathbb{R}^{+}$に対し、その集合$\prod_{i \in \varLambda_{n}} S_{i}$の有限な部分集合$M$が存在してこれがその距離空間$\left( \prod_{i \in \varLambda_{n}} S_{i},d \right)$で$\varepsilon$網である、即ち、$\forall\left( a_{i} \right)_{i \in \varLambda_{n}} \in \prod_{i \in \varLambda_{n}} S_{i}$に対し、$\mathrm{dist}\left( \left\{ \left( a_{i} \right)_{i \in \varLambda_{n}} \right\},M \right) < \varepsilon$が成り立つ。ゆえに、その集合が有限集合であることに注意すれば、$\exists\left( b_{i} \right)_{i \in \varLambda_{n}} \in M$に対し、$d\left( \left( a_{i} \right)_{i \in \varLambda_{n}},\left( b_{i} \right)_{i \in \varLambda_{n}} \right) < \varepsilon$が成り立つので、$\forall i \in \varLambda_{n}$に対し、次のようになる。
\begin{align*}
d_{i}\left( a_{i},b_{i} \right) &\leq \sqrt{\sum_{i \in \varLambda_{n}} {d_{i}\left( a_{i},b_{i} \right)}^{2}}\\
&= d\left( \left( a_{i} \right)_{i \in \varLambda_{n}},\left( b_{i} \right)_{i \in \varLambda_{n}} \right) < \varepsilon
\end{align*}
これにより、$\forall a_{i} \in S_{i}\exists b_{i} \in V\left( {\mathrm{pr}}_{i}|M \right)$に対し、$d_{i}\left( a_{i},b_{i} \right) < \varepsilon$が成り立ち、したがって、$\mathrm{dist}\left( \left\{ a_{i} \right\},\ \ V\left( {\mathrm{pr}}_{i}|M \right) \right) < \varepsilon$が成り立つので、その射影$V\left( {\mathrm{pr}}_{i}|M \right)$がその距離空間$\left( S_{i},d_{i} \right)$で$\varepsilon$網である。また、その射影$V\left( {\mathrm{pr}}_{i}|M \right)$も有限集合であることに注意すれば、その距離空間$\left( S_{i},d_{i} \right)$も全有界である。\par
$\forall i \in \varLambda_{n}$に対し、その距離空間$\left( S_{i},d_{i} \right)$は全有界であるなら、定理\ref{8.2.4.12}より$\forall\varepsilon \in \mathbb{R}^{+}$に対し、その集合$S_{i}$の有限な部分集合$M_{i}$が存在してこれがその距離空間$\left( S_{i},d_{i} \right)$で$\frac{\varepsilon}{\sqrt{n}}$網である、即ち、$\forall a_{i} \in S_{i}$に対し、$\mathrm{dist}\left( \left\{ a_{i} \right\},M_{i} \right) < \frac{\varepsilon}{\sqrt{n}}$が成り立つ。これにより、$\exists b_{i} \in M_{i}$に対し、即ち、$\exists\left( b_{i} \right)_{i \in \varLambda_{n}} \in \prod_{i \in \varLambda_{n}} M_{i}$に対し、$d\left( a_{i},b_{i} \right) < \frac{\varepsilon}{\sqrt{n}}$が成り立つので、次のようになる。
\begin{align*}
d\left( \left( a_{i} \right)_{i \in \varLambda_{n}},\left( b_{i} \right)_{i \in \varLambda_{n}} \right) &= \sqrt{\sum_{i \in \varLambda_{n}} {d_{i}\left( a_{i},b_{i} \right)}^{2}}\\
&< \sqrt{\sum_{i \in \varLambda_{n}} \frac{\varepsilon^{2}}{n}}\\
&= \varepsilon\sqrt{\sum_{i \in \varLambda_{n}} \frac{1}{n}}\\
&= \varepsilon\sqrt{\frac{1}{n} \cdot n} = \varepsilon
\end{align*}
したがって、$\mathrm{dist}\left( \left\{ \left( a_{i} \right)_{i \in \varLambda_{n}} \right\},\prod_{i \in \varLambda_{n}} M_{i} \right) < \varepsilon$が成り立つ。以上より、$\forall\varepsilon \in \mathbb{R}^{+}$に対し、その集合$\prod_{i \in \varLambda_{n}} S_{i}$の有限な部分集合$\prod_{i \in \varLambda_{n}} M_{i}$が存在してこれがその距離空間$\left( \prod_{i \in \varLambda_{n}} S_{i},d \right)$で$\varepsilon$網である、即ち、$\forall\left( a_{i} \right)_{i \in \varLambda_{n}} \in \prod_{i \in \varLambda_{n}} S_{i}$に対し、$\mathrm{dist}\left( \left\{ \left( a_{i} \right)_{i \in \varLambda_{n}} \right\},\prod_{i \in \varLambda_{n}} M_{i} \right) < \varepsilon$が成り立つので、定理\ref{8.2.4.12}よりその直積距離空間$\left( \prod_{i \in \varLambda_{n}} S_{i},d \right)$が全有界である。
\end{proof}
%\hypertarget{ux90e8ux5206ux5217}{%
\subsubsection{部分列}%\label{ux90e8ux5206ux5217}}
\begin{dfn*}[定義\ref{部分列}の再掲]
集合$A$の元の列$\left( a_{n} \right)_{n \in \mathbb{N}}$が与えられたとき、集合$\mathbb{N}$の元の列$\left( n_{k} \right)_{k \in \mathbb{N}}$を用いた元の列$\left( a_{n_{k}} \right)_{k \in \mathbb{N}}$、即ち、合成写像$\left( a_{n} \right)_{n \in \mathbb{N}} \circ \left( n_{k} \right)_{k \in \mathbb{N}}$をその元の列$\left( a_{n} \right)_{n \in \mathbb{N}}$の部分列という。
\end{dfn*}\par
もちろん、その元の列$\left( a_{n} \right)_{n \in \mathbb{N}}$自身もその元の列$\left( a_{n} \right)_{n \in \mathbb{N}}$の部分列である。
\begin{thm}\label{8.2.4.18} 距離空間$(S,d)$が与えられたとき、次のことが成り立つ。
\begin{itemize}
\item
  その集合$S$の元の列$\left( a_{n} \right)_{n \in \mathbb{N}}$がある元$\alpha$に収束するなら、その元の列$\left( a_{n} \right)_{n \in \mathbb{N}}$の任意の部分列$\left( a_{n_{k}} \right)_{k \in \mathbb{N}}$もその元$\alpha$に収束する。
\item
  その集合$S$の元の列$\left( a_{n} \right)_{n \in \mathbb{N}}$がCauchy列であるなら、その元の列$\left( a_{n} \right)_{n \in \mathbb{N}}$の任意の部分列$\left( a_{n_{k}} \right)_{k \in \mathbb{N}}$もCauchy列である。
\end{itemize}
\end{thm}\par
しかしながら、その元の列$\left( a_{n} \right)_{n \in \mathbb{N}}$のある部分列$\left( a_{n_{k}} \right)_{k \in \mathbb{N}}$が収束する、あるいは、Cauchy列であっても、もとの元の列$\left( a_{n} \right)_{n \in \mathbb{N}}$も収束する、あるいは、Cauchy列であるとは限らないことに注意されたい。例えば、1次元Euclid空間$E$における点列$\left((-1)^n \right)_{n\in \mathbb{N}}$が挙げられる。
\begin{proof}
距離空間$(S,d)$が与えられたとき、その集合$S$の元の列$\left( a_{n} \right)_{n \in \mathbb{N}}$がある元$\alpha$に収束するなら、$\forall\varepsilon \in \mathbb{R}^{+}\exists n_{0} \in \mathbb{R}^{+}\forall n \in \mathbb{N}$に対し、$n_{0} < n$が成り立つなら、$d\left( a_{n},\alpha \right) < \varepsilon$が成り立つ。ここで、その元の列$\left( a_{n} \right)_{n \in \mathbb{N}}$の任意の部分列$\left( a_{n_{k}} \right)_{k \in \mathbb{N}}$について、$\exists k_{0} \in \mathbb{N}$に対し、$n_{0} < n_{k_{0}}$が成り立ち、$\forall k \in \mathbb{N}$に対し、$k_{0} < k$が成り立つなら、$n_{0} < n_{k}$が成り立ち、$d\left( a_{n_{k}},\alpha \right) < \varepsilon$が成り立つので、その部分列$\left( a_{n_{k}} \right)_{k \in \mathbb{N}}$もその元$\alpha$に収束する。\par
その集合$S$の元の列$\left( a_{n} \right)_{n \in \mathbb{N}}$がCauchy列であるなら、$\forall\varepsilon \in \mathbb{R}^{+}\exists n_{0} \in \mathbb{N}\forall m,n \in \mathbb{N}$に対し、$n_{0} < m$かつ$n_{0} < n$が成り立つなら、$d\left( a_{m},a_{n} \right) < \varepsilon$が成り立つ。ここで、その元の列$\left( a_{n} \right)_{n \in \mathbb{N}}$の任意の部分列$\left( a_{n_{k}} \right)_{k \in \mathbb{N}}$について、$\exists k_{0},l_{0} \in \mathbb{N}$に対し、$n_{0} < n_{k_{0}}$かつ$n_{0} < n_{l_{0}}$が成り立ち、$\forall k,l \in \mathbb{N}$に対し、$k_{0} < k$、$l_{0} < l$が成り立つなら、$n_{0} < n_{k}$かつ$n_{0} < n_{l}$が成り立ち、$d\left( a_{n_{k}},a_{n_{l}} \right) < \varepsilon$が成り立つので、その部分列$\left( a_{n_{k}} \right)_{k \in \mathbb{N}}$もCauchy列である。
\end{proof}
\begin{thm}\label{8.2.4.19}
距離空間$(S,d)$が与えられたとき、Cauchy列$\left( a_{n} \right)_{n \in \mathbb{N}}$の部分列が存在してこれがある元$\alpha$に収束するなら、そのCauchy列$\left( a_{n} \right)_{n \in \mathbb{N}}$もその元$\alpha$に収束する。
\end{thm}
\begin{proof}
距離空間$(S,d)$が与えられたとき、Cauchy列$\left( a_{n} \right)_{n \in \mathbb{N}}$の部分列$\left( a_{n_{k}} \right)_{n \in \mathbb{N}}$が存在してこれがある元$\alpha$に収束するとき、$\forall\varepsilon \in \mathbb{R}^{+}\exists n_{0} \in \mathbb{N}\forall m,n \in \mathbb{N}$に対し、$n_{0} < m$かつ$n_{0} < n$が成り立つなら、$d\left( a_{m},a_{n} \right) < \varepsilon$が成り立つかつ、$\exists k_{0} \in \mathbb{N}\forall k \in \mathbb{N}$に対し、$k_{0} < k$が成り立つなら、$d\left( a_{n_{k}},\alpha \right) < \varepsilon$が成り立つ。ここで、$\exists k_{0}' \in \mathbb{N}\forall k \in \mathbb{N}$に対し、$n_{0} < n_{k_{0}'}$が成り立つかつ、$k_{0} < k_{0}' < k$が成り立つなら、$d\left( a_{n_{k}},\alpha \right) < \varepsilon$が成り立つことに注意すれば、$\forall\varepsilon \in \mathbb{R}^{+}\exists n_{0} \in \mathbb{N}\forall n \in \mathbb{N}\exists k_{0}' \in \mathbb{N}\forall k \in \mathbb{N}$に対し、$n_{0} < n$かつ$k_{0} < k_{0}' < k$が成り立つなら、$n_{0} < n_{k}$が成り立ち、したがって、次のようになるので、
\begin{align*}
d\left( a_{n},\alpha \right) &\leq d\left( a_{n},a_{n_{k}} \right) + d\left( a_{n_{k}},\alpha \right)\\
&< \varepsilon + \varepsilon = 2\varepsilon
\end{align*}
そのCauchy列$\left( a_{n} \right)_{n \in \mathbb{N}}$もその元$\alpha$に収束する。
\end{proof}
\begin{dfn}
距離空間$(S,d)$が与えられたとき、その集合$S$の元の列$\left( a_{n} \right)_{n \in \mathbb{N}}$が、$\forall m,n \in \mathbb{N}$に対し、$d\left( a_{m},a_{n} \right) < \varepsilon$が成り立つとき、その元の列$\left( a_{n} \right)_{n \in \mathbb{N}}$を$\varepsilon$列という。
\end{dfn}
\begin{thm}\label{8.2.4.20}
全有界距離空間$(S,d)$が与えられたとき、任意の元の列$\left( a_{n} \right)_{n \in \mathbb{N}}$と任意の正の実数$\varepsilon$に対し、その元の列の部分列$\left( a_{n_{k}} \right)_{k \in \mathbb{N}}$が存在して、これが$\varepsilon$列であるものが存在する。
\end{thm}
\begin{proof}
全有界距離空間$(S,d)$が与えられたとき、任意の正の実数$\varepsilon$に対し、その集合$S$の有限な$\varepsilon$被覆が存在するので、これを$\mathfrak{U}'$とおく。このとき、任意の元の列$\left( a_{n} \right)_{n \in \mathbb{N}}$のどの項$a_{n}$に対し、$S = \bigcup_{} \mathfrak{U}'$が成り立つので、その$\varepsilon$被覆$\mathfrak{U}'$のある集合$U$が存在して$a_{n} \in U$が成り立つ。ここで、その$\varepsilon$被覆のある元$U$に対し、次のように集合$N_{U}$が定義されると、
\begin{align*}
N_{U} = \left\{ n \in \mathbb{N} \middle| a_{n} \in U \right\}
\end{align*}
先ほどの議論により$\mathbb{N} = \bigcup_{U \in \mathfrak{U}'} N_{U}$が成り立つ。ここで、全ての集合たち$N_{U}$が有限集合であるなら、その集合$\mathbb{N}$も有限集合であることになり矛盾する。ゆえに、$\exists U' \in \mathfrak{U}'$に対し、集合$N_{U'}$は無限集合である、詳しくいえば、${\#}N_{U'} = \aleph_{0}$が成り立つ。これにより、その集合$N_{U'}$と集合$\mathbb{N}$との間に全単射が存在するので、その全単射が順序同型写像となるように定義されれば、順序同型写像$\left( n_{k} \right)_{k \in \mathbb{N}}:\mathbb{N} \rightarrow N_{U'};k \mapsto n_{k}$が存在して、$N_{U'} = \left\{ n_{k} \right\}_{k \in \mathbb{N}}$が成り立つ。このとき、その元の列$\left( a_{n_{k}} \right)_{k \in \mathbb{N}}$はその集合$N_{U'}$の元の列でありその元の列$\left( a_{n} \right)_{n \in \mathbb{N}}$の部分列である。しかも、$\forall k,l \in \mathbb{N}$に対し、$a_{n_{k}},a_{n_{l}} \in U'$が成り立つので、次のようになる。
\begin{align*}
d\left( a_{n_{k}},a_{n_{l}} \right) \leq \delta\left( U' \right) < \varepsilon
\end{align*}
これにより、その元の列$\left( a_{n_{k}} \right)_{k \in \mathbb{N}}$は$\varepsilon$列でもある。
\end{proof}
\begin{thm}\label{8.2.4.21}
距離空間$(S,d)$が与えられたとき、その距離空間$(S,d)$が全有界であるならそのときに限り、その集合$S$の任意の元の列に対し、ある部分列が存在して、これがCauchy列となる。
\end{thm}
\begin{proof}
距離空間$(S,d)$が与えられたとき、その距離空間$(S,d)$が全有界であるなら、定理\ref{8.2.4.20}より任意の元の列$\left( a_{n} \right)_{n \in \mathbb{N}}$と任意の正の実数$\varepsilon$に対し、その元の列の部分列$\left( a_{n_{k}} \right)_{k \in \mathbb{N}}$で$\varepsilon$列であるものが存在する。ここで、元の列$\left( a_{n} \right)_{n \in \mathbb{N}}$と正の実数$\varepsilon$に対し、これの$\varepsilon$列であるような部分列を$\varPhi\left( \left( a_{n} \right)_{n \in \mathbb{N}},\varepsilon \right)$と書くことにして、次式のように元の列たち$\left( a_{np}' \right)_{n \in \mathbb{N}}$が定義されると、
\begin{align*}
\left( a_{n1}' \right)_{n \in \mathbb{N}} = \varPhi\left( \left( a_{n} \right)_{n \in \mathbb{N}},1 \right),\ \ \left( a_{n,p + 1}' \right)_{n \in \mathbb{N}} = \varPhi\left( \left( a_{np}' \right)_{n \in \mathbb{N}},\frac{1}{p + 1} \right)
\end{align*}
元の列$\left( a_{nn}' \right)_{n \in \mathbb{N}}$はその元の列$\left( a_{n} \right)_{n \in \mathbb{N}}$の部分列である。\par
$\forall\varepsilon \in \mathbb{R}^{+}$に対し、$\frac{1}{n_{0}} < \varepsilon$が成り立つように自然数$n_{0}$をとると、$\forall p,q \in \mathbb{N}$に対し、$n_{0} < p$かつ$n_{0} < q$が成り立つなら、元の列たち$\left( a_{np}' \right)_{n \in \mathbb{N}}$、$\left( a_{nq}' \right)_{n \in \mathbb{N}}$はいづれも元の列$\left( a_{nn_{0}}' \right)_{n \in \mathbb{N}}$の部分列であり元々$a_{pp}'$、$a_{qq}'$いづれもその元の列$\left( a_{nn_{0}}' \right)_{n \in \mathbb{N}}$の項である。ここで、その元の列$\left( a_{nn_{0}}' \right)_{n \in \mathbb{N}}$は$\frac{1}{n_{0}}$列であるので、$d\left( a_{np}',a_{nq}' \right) < \frac{1}{n_{0}} < \varepsilon$が得られる。ゆえに、その元の列$\left( a_{nn}' \right)_{n \in \mathbb{N}}$はCauchy列である。\par
逆に、その距離空間$(S,d)$が全有界でないなら、ある正の実数$\varepsilon$が存在して、その集合$S$の有限な$\varepsilon$被覆が存在しないことになる。したがって、正の実数$\varepsilon_{0}$が$2\varepsilon_{0} < \varepsilon$を満たすようにとられれば、その集合$S$の任意の有限な部分集合$M$に対し、集合$\left\{ B\left( b,\varepsilon_{0} \right) \right\}_{b \in M}$は有限集合であり、$\forall b \in M$に対し、$\delta\left( B\left( b,\varepsilon_{0} \right) \right) \leq 2\varepsilon_{0} < \varepsilon$が成り立つので、その集合$\left\{ B\left( b,\varepsilon_{0} \right) \right\}_{b \in M}$はその集合$S$の被覆となることができない。このとき、明らかに$\bigcup_{b \in M} {B\left( b,\varepsilon_{0} \right)} \subseteq S$が成り立つので、差集合$S \setminus \bigcup_{b \in M} {B\left( b,\varepsilon_{0} \right)}$は空集合でない。\par
そこで、その集合$S$の有限な各部分集合$M$に対し、その差集合$S \setminus \bigcup_{b \in M} {B\left( b,\varepsilon_{0} \right)}$の元1つが$\varphi_{M}$とおかれると、$\varphi_{M} \notin \bigcup_{b \in M} {B\left( b,\varepsilon_{0} \right)}$が成り立つので、$\forall b \in M$に対し、$d\left( \varphi_{M},b \right) < \varepsilon_{0}$が成り立たない、即ち、$d\left( \varphi_{M},b \right) \geq \varepsilon_{0}$が成り立つ。\par
そこで、その集合$S$の任意の元$a$を用いて次式のように元の列$\left( a_{n} \right)_{n \in \mathbb{N}}$が定義されるとする。
\begin{align*}
a_{1} = a,\ \ a_{n + 1} = \varphi_{\left\{ a_{i} \right\}_{i \in \varLambda_{n}}}
\end{align*}
このとき、$\forall i,j \in \mathbb{N}$に対し、$i \neq j$が成り立つなら、$i < j$が成り立つとしても一般性は失われずそうするとき、$a_{j} = \varphi_{\left\{ a_{i'} \right\}_{i' \in \varLambda_{j}}}$が成り立つかつ、$a_{i} \in \left\{ a_{i'} \right\}_{i' \in \varLambda_{j}}$が成り立つので、上記の議論により$d\left( a_{i},a_{j} \right) \geq \varepsilon_{0}$が成り立つ。\par
そこで、その元の列$\left( a_{n} \right)_{n \in \mathbb{N}}$に対し、ある部分列$\left( a_{n_{k}} \right)_{k \in \mathbb{N}}$が存在して、これがCauchy列となると仮定すると、$\forall\varepsilon' \in \mathbb{R}^{+}\exists n_{0} \in \mathbb{N}\forall k,l \in \mathbb{N}$に対し、$n_{0} < k$かつ$n_{0} < l$が成り立つなら、$d\left( a_{n_{k}},a_{n_{l}} \right) < \varepsilon'$が成り立つ。ここで、$\varepsilon' = \varepsilon_{0}$とおかれると、$\exists n_{0} \in \mathbb{N}\forall k,l \in \mathbb{N}$に対し、$n_{0} < k$かつ$n_{0} < l$が成り立つなら、$d\left( a_{n_{k}},a_{n_{l}} \right) < \varepsilon_{0}$が成り立つ。これにより、$\exists n_{k},n_{l} \in \mathbb{N}$に対し、$n_{k} \neq n_{l}$が成り立つかつ、$d\left( a_{n_{k}},a_{n_{l}} \right) < \varepsilon_{0}$が成り立つことになるが、これは仮定の、$\forall i,j \in \mathbb{N}$に対し、$i \neq j$が成り立つなら、$d\left( a_{i},a_{j} \right) \geq \varepsilon_{0}$が成り立つことに矛盾する。\par
ゆえに、その元の列$\left( a_{n} \right)_{n \in \mathbb{N}}$の任意の部分列$\left( a_{n_{k}} \right)_{k \in \mathbb{N}}$に対し、これがCauchy列とならない。あとは、対偶律によりその集合$S$の任意の元の列に対し、ある部分列が存在して、これがCauchy列となるなら、その距離空間$(S,d)$は全有界である。
\end{proof}
\begin{thebibliography}{50}
\bibitem{1}
  松坂和夫, 集合・位相入門, 岩波書店, 1968. 新装版第2刷 p247-263 ISBN978-4-00-029871-1
\end{thebibliography}
\end{document}
