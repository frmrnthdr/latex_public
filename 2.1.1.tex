\documentclass[dvipdfmx]{jsarticle}
\setcounter{section}{1}
\setcounter{subsection}{0}
\usepackage{amsmath,amsfonts,amssymb,array,comment,mathtools,url,docmute}
\usepackage{longtable,booktabs,dcolumn,tabularx,mathtools,multirow,colortbl,xcolor}
\usepackage[dvipdfmx]{graphics}
\usepackage{bmpsize}
\usepackage{amsthm}
\usepackage{enumitem}
\setlistdepth{20}
\renewlist{itemize}{itemize}{20}
\setlist[itemize]{label=•}
\renewlist{enumerate}{enumerate}{20}
\setlist[enumerate]{label=\arabic*.}
\setcounter{MaxMatrixCols}{20}
\setcounter{tocdepth}{3}
\newcommand{\rotin}{\text{\rotatebox[origin=c]{90}{$\in $}}}
\renewcommand{\thesection}{第\arabic{section}部}
\renewcommand{\thesubsection}{\arabic{section}.\arabic{subsection}}
\renewcommand{\thesubsubsection}{\arabic{section}.\arabic{subsection}.\arabic{subsubsection}}
\everymath{\displaystyle}
\allowdisplaybreaks[4]
\usepackage{vtable}
\theoremstyle{definition}
\newtheorem{thm}{定理}[subsection]
\newtheorem*{thm*}{定理}
\newtheorem{dfn}{定義}[subsection]
\newtheorem*{dfn*}{定義}
\newtheorem{axs}[dfn]{公理}
\newtheorem*{axs*}{公理}
\renewcommand{\headfont}{\bfseries}
\makeatletter
  \renewcommand{\section}{%
    \@startsection{section}{1}{\z@}%
    {\Cvs}{\Cvs}%
    {\normalfont\huge\headfont\raggedright}}
\makeatother
\makeatletter
  \renewcommand{\subsection}{%
    \@startsection{subsection}{2}{\z@}%
    {0.5\Cvs}{0.5\Cvs}%
    {\normalfont\LARGE\headfont\raggedright}}
\makeatother
\makeatletter
  \renewcommand{\subsubsection}{%
    \@startsection{subsubsection}{3}{\z@}%
    {0.4\Cvs}{0.4\Cvs}%
    {\normalfont\Large\headfont\raggedright}}
\makeatother
\makeatletter
\renewenvironment{proof}[1][\proofname]{\par
  \pushQED{\qed}%
  \normalfont \topsep6\p@\@plus6\p@\relax
  \trivlist
  \item\relax
  {
  #1\@addpunct{.}}\hspace\labelsep\ignorespaces
}{%
  \popQED\endtrivlist\@endpefalse
}
\makeatother
\renewcommand{\proofname}{\textbf{証明}}
\usepackage{tikz,graphics}
\usepackage[dvipdfmx]{hyperref}
\usepackage{pxjahyper}
\hypersetup{
 setpagesize=false,
 bookmarks=true,
 bookmarksdepth=tocdepth,
 bookmarksnumbered=true,
 colorlinks=false,
 pdftitle={},
 pdfsubject={},
 pdfauthor={},
 pdfkeywords={}}
\begin{document}
%\hypertarget{vectorux7a7aux9593}{%
\subsection{vector空間}%\label{vectorux7a7aux9593}}
\subsubsection{群}%\label{ux7fa4-1}}
\begin{axs}[群の公理]
空集合でない集合$G$に対し算法$*:G \times G \rightarrow G;(a,b) \mapsto a*b$が与えられたとする。このとき、次の条件たちを満たす集合$G$と算法$*$を合わせて群といい、集合$G$は算法$*$に対し群をなすといい、$(G,*)$と書く。そのような集合$G$の元の個数が有限なら、その群$(G,*)$は有限群といい、その集合の濃度$\#G$をその群$(G,*)$の位数といい、$o(G,*)$と書く。逆に、その集合$G$の元の個数が無限ならば、その群$(G,*)$は無限群という。単位元$e$のみからなる群$\left( \left\{ e \right\},* \right)$を単位群という\footnote{余談ですが、集合$G$の1つの部分集合を$S$、$n$つの部分集合たちのうち1つを$S_{i}$、これの元の1つを$s_{i}$とおき、写像$f:\prod_{i} S_{i} \rightarrow S;\left( s_{i} \right)_{i} \mapsto f\left( s_{i} \right)_{i}$を考えるとき、紛らわしいことに集合$\left\{ f\left( s_{i} \right)_{i} \middle| \forall i\left\lbrack s_{i} \in S_{i} \right\rbrack \right\}$を$f\left( S_{i} \right)_{i}$と表記することがあります…。例えば、$S_{1}*S_{2}$、$a*S_{1}$などといった感じに。}。
\begin{itemize}
\item
  算法$*$について結合的である、即ち、$\forall a,b,c \in G$に対し、$(a*b)*c = a*(b*c)$が成り立つ。
\item
  $\exists b \in G\forall a \in G$に対し、$a*b = b*a = a$が成り立つ。この元$b$をその群$(G,*)$の単位元という。
\item
  $\forall a \in G\exists b \in G$に対し、$a*b = b*a = e$が成り立つ。この元$b$を$a$の逆元といい、$a^{- 1}$と書く。
\end{itemize}
さらに次の条件も満たす群$(G,*)$を特に可換群、Abel群という。
\begin{itemize}
\item
  算法$*$は可換的である、即ち、$\forall a,b \in G$に対し、$a*b = b*a$が成り立つ。
\end{itemize}
なお、$a*b = b*a$が成り立つような元々$a$、$b$は可換であるという。
\end{axs}
\begin{thm}\label{2.1.1.1}
群$(G,*)$について、その単位元$e$、その集合$G$の任意の元$a$の逆元$a^{- 1}$は一意的に存在する。
\end{thm}\par
これはいずれも背理法によって示される。
\begin{proof}
群$(G,*)$において、$\forall a \in G$に対し、$a*e = e*a = a$なるその集合$G$の元$e$とは異なる、$\forall a \in G$に対し、$a*e' = e'*a = a$なる元$e'$がその集合$G$に存在するとする。このとき、$e*e' = e$かつ$e*e' = e'$が成り立つので、$e = e'$が成り立つこととなり、仮定に矛盾する。よって、$\forall a \in G$に対し、$a*e = e*a = a$が成り立つようなその単位元$e$は一意的に存在する。\par
同様に、$\forall a \in G$に対し、$a*a^{- 1} = a^{- 1}*a = e$なるその集合$G$の元$a^{- 1}$とは異なる$a*b = b*a = e$なる元$b$がその集合$G$に存在するとする。このとき、次のようになり、
\begin{align*}
a^{- 1} &= a^{- 1}*e\\
&= a^{- 1}*(a*b)\\
&= \left( a^{- 1}*a \right)*b\\
&= e*b = b
\end{align*}
仮定に矛盾する。よって、$\forall a \in G$に対し、$a*a^{- 1} = a^{- 1}*a = e$となる元$a^{- 1}$が一意的に存在する。
\end{proof}
\begin{thm}[簡易律]\label{2.1.1.2}
群$(G,*)$において、$\forall a,u,v \in G$に対し、次のことが成り立つ。
\begin{itemize}
\item
  $a*u = a*v$が成り立つなら、$u = v$が成り立つ。
\item
  $u*a = v*a$が成り立つなら、$u = v$が成り立つ。
\end{itemize}
\par
この性質を簡易律という。
\end{thm}
\begin{proof}
群$(G,*)$が与えられたとする。$\forall a,u,v \in G$に対し、$a*u = a*v$が成り立つなら、次式が成り立つかつ、
\begin{align*}
a^{- 1}*(a*u) &= \left( a^{- 1}*a \right)*u\\
&= e*u = u
\end{align*}
次式が成り立つので、
\begin{align*}
a^{- 1}*(a*u) &= a^{- 1}*(a*v)\\
&= \left( a^{- 1}*a \right)*v\\
&= e*v = v
\end{align*}
$u = v$が得られる。\par
同様にして、$u*a = v*a$が成り立つなら、$u = v$が成り立つことが示される。
\end{proof}
\begin{thm}\label{2.1.1.3}
群$(G,*)$について、$\forall a,b \in G$に対し、$(a*b)^{- 1} = b^{- 1}*a^{- 1}$が成り立つ。
\end{thm}
\begin{proof}
群$(G,*)$が与えられたとする。$\forall a,b \in G$に対し、$(a*b)*(a*b)^{- 1} = e$が成り立つかつ、次式が成り立つので、
\begin{align*}
(a*b)*\left( b^{- 1}*a^{- 1} \right) &= \left( a*\left( b*b^{- 1} \right) \right)*a^{- 1}\\
&= (a*e)*a^{- 1}\\
&= a*a^{- 1} = e
\end{align*}
$(a*b)*(a*b)^{- 1} = (a*b)*\left( b^{- 1}*a^{- 1} \right)$が得られ、したがって、簡易律により$(a*b)^{- 1} = b^{- 1}*a^{- 1}$が成り立つ。
\end{proof}
\begin{dfn}
群$(G,*)$をなす集合$G$の元$a$について$m,n \in \mathbb{Z}$に対し次式のように記法を定める。
\begin{align*}
a^{m}*a^{n} &= a^{m + n}\\
\left( a^{m} \right)^{n} &= a^{mn}\\
(a*b)^{n} &= a^{n}*b^{n}\ \mathrm{if}\ a*b = b*a
\end{align*}
\end{dfn}
\begin{thm}\label{2.1.1.4}
群$(G,*)$について、$\forall a \in G\forall n \in \mathbb{N}$に対し、次式たちが成り立つ。
\begin{align*}
a^{0} = e,\ \ a^{1} = a,\ \ a^{n + 1} = a^{n}*a,\ \ e^{- n} = e^{n} = e,\ \ a^{- n} = \left( a^{- 1} \right)^{n}
\end{align*}
\end{thm}
\begin{proof}
群$(G,*)$をなす集合$G$について、$\forall a \in G\forall n \in \mathbb{N}$に対し、次のようになる。
\begin{align*}
a^{0} &= e*a^{0}\\
&= \left( a^{0} \right)^{- 1}*a^{0}*a^{0}\\
&= \left( a^{0} \right)^{- 1}*a^{0 + 0}\\
&= \left( a^{0} \right)^{- 1}*a^{0}\\
&= e\\
a^{1} &= e*a^{1}\\
&= a*a^{- 1}*a^{1}\\
&= a*a^{- 1 + 1}\\
&= a*a^{0}\\
&= a*e = a\\
a^{n + 1} &= a^{n}*a^{1}\\
&= a^{n}*a
\end{align*}\par
また、上記の議論より$e^{- 1} = e^{0} = e^{1} = e$が成り立つ。$n = k$のとき、$e^{k} = e$と仮定しよう。$n = k + 1$のとき、次のようになる。
\begin{align*}
e^{k + 1} &= e^{k}*e^{1}\\
&= e*e\\
&= e
\end{align*}
逆に、$n = k$のとき、$e^{- k} = e$と仮定しよう。$n = k + 1$のとき、次のようになる。
\begin{align*}
e^{- (k + 1)} &= e^{- k - 1}\\
&= e^{- k}*e^{- 1}\\
&= e*e\\
&= e
\end{align*}
以上より数学的帰納法によって$\forall n \in \mathbb{Z}$に対し、次式が成り立つ。
\begin{align*}
e^{n} = e
\end{align*}\par
また、上記の議論により次のようになる。
\begin{align*}
a^{- n} &= a^{- n}*e\\
&= a^{- n}*\left( a^{- 1} \right)^{0}\\
&= a^{- n}*\left( a^{- 1} \right)^{- n + n}\\
&= a^{- n}*\left( a^{- 1} \right)^{- n}*\left( a^{- 1} \right)^{n}\\
&= \left( a*a^{- 1} \right)^{- n}*\left( a^{- 1} \right)^{n}\\
&= e^{- n}*\left( a^{- 1} \right)^{n}\\
&= e*\left( a^{- 1} \right)^{n}\\
&= \left( a^{- 1} \right)^{n}
\end{align*}
\end{proof}
\subsubsection{環}%\label{ux74b0-1}}
\begin{axs}[環の公理]
空集合でない集合$R$に対し2つの算法それぞれ加法$+ :R \times R \rightarrow R;(a,b) \mapsto a + b$、乗法$\cdot :R \times R \rightarrow R;(a,b) \mapsto ab$が与えられたとする。このとき、次の条件たちを満たす集合$R$を環という\footnote{ここから先は心象するのがほぼ不可能な分野になりますので、定義をよく読んでおくことをお勧めします。}。
\begin{itemize}
\item
  集合$R$は加法について可換群$(R, + )$をなす。
\item
  $\forall a,b,c \in R$に対し、$(ab)c = a(bc)$が成り立つ、即ち、乗法について結合的である。
\item
  $\exists e \in R\forall a \in R$に対し、$ae = ea = a$が成り立つ、即ち、乗法について集合$R$の単位元$e$が存在する。
\item
  $\forall a,b,c \in R$に対し、$a(b + c) = ab + ac$かつ$(a + b)c = ac + bc$が成り立つ、即ち、乗法は加法に対して両側から分配的である。
\end{itemize}
さらに、次の条件も満たす環$R$を特に可換環という。
\begin{itemize}
\item
  $\forall a,b \in R$に対し、$ab = ba$が成り立つ、即ち、乗法は可換的である。
\end{itemize}
\end{axs}
\begin{dfn}
可換群$(R, + )$において、その単位元を零元といい$0$と、逆元$a^{- 1}$を$- a$と、$\forall n \in \mathbb{Z}$に対し、元$a^{n}$を$na$と、乗法についての単位元$e$を$1$と書く。
\end{dfn}
\begin{thm}\label{2.1.1.5}
環$R$が与えられたとき、$\forall a \in R$に対し、$a0 = 0a = 0$が成り立つ。
\end{thm}
\begin{proof} 環$R$について、$\forall a \in R$に対し、次のようになるかつ、
\begin{align*}
0 &= a0 - a0\\
&= a(0 + 0) - a0\\
&= a0 + a0 - a0 = a0
\end{align*}
次のようになるので、
\begin{align*}
0 &= 0a - 0a\\
&= (0 + 0)a - 0a\\
&= 0a + 0a - 0a = 0a
\end{align*}
$a0 = 0a = 0$が成り立つ。
\end{proof}
\begin{dfn}
環$R$について、$0 = 1$が成り立つとき、$\forall a \in R$に対し、$a = 1a = 0a = 0$が成り立ち$R = \left\{ 0 \right\}$が得られる。これを零環という。以下、環の元が2つ以上現れるのであれば、その環は零環でないので、断りがない場合、そうする。
\end{dfn}
\begin{dfn}
環$R$について、$\exists a,b \in R$に対し、$a \neq 0$かつ$b \neq 0$が成り立つかつ、$ab = 0$が成り立つなら、それらの元々$a$、$b$をそれぞれ左零因子、右零因子といい、あわせて零因子という。これをもたない可換環を、即ち、その環$R$が可換環で、$\forall a,b \in R$に対し、$a \neq 0$かつ$b \neq 0$が成り立つなら、$ab \neq 0$が成り立つような環を整域という。
\end{dfn}
\begin{dfn}\label{体の定義}
環$R$について、$\exists a,b \in R$に対し、$ab = 1$が成り立つなら、その元$a$を環$R$の可逆元、単元といい、その元$b$を逆元といい、後に示すように$a^{- 1}$、$\frac{1}{a}$などと書くことができる。以下、その元$a^{-1}$はその群$\left(R,+\right)$における逆元ではなくその可逆元$a$の積における逆元を意味するものとする。これにより、可逆元からなる集合は乗法について群をなし、$0$以外の元全てが可逆元であるような環を斜体といい、乗法について可換的な斜体を体といい、可換的でない斜体を非可換体という。
\end{dfn}\par
斜体を体、体を可換体というときもある。
\begin{thm}\label{2.1.1.6}
環$R$の可逆元について、次のことが成り立つ。
\begin{itemize}
\item
  その環$R$が零環でなくその環$R$の元$a$が可逆元なら、これは$0$でない。
\item 
  その環$R$の元$a$が可逆元なら、一意的に逆元${a}^{-1}$が定まる。
\item
  その環$R$が斜体であるなら、零因子をもたない。
\end{itemize}
\end{thm}
\begin{proof}
環$R$について、$a \in R$が成り立ちその環$R$が零環でなくその元$a$が可逆元であり${a}^{-1} \in R$なる元${a}^{-1}$を$a$の逆元とする。$a = 0$が成り立つなら、$a{a}^{-1} = 0{a}^{-1} = 0 \neq 1$が成り立つので、可逆元の定義に矛盾する。よって$a \neq 0$が成り立つ。\par
また、その環$R$の元$a$が可逆元でその元$a^{- 1}$でないその元$a$の逆元$b$が与えられたとするとき、次式が成り立つので、
\begin{align*}
{a}^{-1} &= {a}^{-1}1\\
&= {a}^{-1}(ab)\\
&= \left( {a}^{-1}a \right)b\\
&= 1b = b
\end{align*}
仮定に矛盾する。よって、一意的に逆元${a}^{-1}$が定まる。\par
環$R$が斜体であるなら、$0$以外の元全てが可逆元であるので、$\forall a,b \in R$に対し、$a \neq 0$かつ$b \neq 0$が成り立つなら、${a}^{-1},{b}^{-1} \in R$が成り立ち、したがって、次のようになる。
\begin{align*}
ab{b}^{-1}{a}^{-1} &= a1{a}^{-1}\\
&= a{a}^{-1} = 1\\
{b}^{-1}{a}^{-1}ab &= {b}^{-1}1b\\
&= {b}^{-1}b = 1
\end{align*}
したがって、その元$ab$は可逆元であることになるので、$ab \neq 0$が成り立つ。ゆえに、その環$R$は零因子をもたない。
\end{proof}
\begin{thm}\label{2.1.1.7}
環$R$の性質として、次のことが成り立つ。
\begin{itemize}
  \item $\forall a\in R$に対し、$-(-a) =a$が成り立つ。
  \item $\forall a,b\in R$に対し、$a(-b)=(-a)b=-ab$が成り立つ。
  \item $\forall a,b\in R$に対し、$(-a)(-b)=ab$が成り立つ。
  \item $\forall a,b\in R$に対し、$a=0$または$b=0$が成り立つなら、$ab=0$が成り立つ。
  \item $\forall a,b\in R$に対し、$a=0$かつ$b=0$が成り立つなら、$a^2 +b^2 =0$が成り立つ。
  \item $\forall a\in R$に対し、その元$a$が可逆元なら、その元$-a$も可逆元で$(-a)^{-1} =-a^{-1}$が成り立つ。
  \item $\forall a\in R$に対し、それらの元々$a$、$b$が可逆元なら、その元$ab$も可逆元で$(ab)^{-1} =b^{-1} a^{-1}$が成り立つ。
\end{itemize}
\end{thm}
\begin{proof} 環$R$が与えられたとき、$\forall a\in R$に対し、次のようになる。
\begin{align*}
- ( - a) &= 0 - ( - a)\\
&= a - a - ( - a)\\
&= a + ( - a) - ( - a)\\
&= a + 0 = a
\end{align*}\par
$\forall a,b\in R$に対し、次のようになる。
\begin{align*}
a( - b) &= 0 + a( - b)\\
&= - ab + ab + a( - b)\\
&= - ab + a\left( b + ( - b) \right)\\
&= - ab + a0\\
&= - ab + 0\\
&= - ab
\end{align*}\par
$\forall a,b\in R$に対し、次のようになる。
\begin{align*}
( - a)b &= ( - a)b + 0\\
&= ( - a)b + ab - ab\\
&= \left( ( - a) + a \right)b - ab\\
&= 0b - ab\\
&= a - ab\\
&= - ab
\end{align*}\par
$\forall a,b\in R$に対し、次のようになる。
\begin{align*}
( - a)( - b) &= ( - a)( - b) + 0\\
&= ( - a)( - b) + ( - a)b - ( - a)b\\
&= ( - a)\left( ( - b) + b \right) - ( - ab)\\
&= ( - a)0 + ab\\
&= 0 + ab\\
&= ab
\end{align*}
$\forall a,b\in R$に対し、$a=0$または$b=0$が成り立つなら、次のようになるので、
\begin{align*}
a = 0 \vee b = 0 &\Rightarrow ab = 0 \vee ab = 0\\
&\Leftrightarrow ab = 0
\end{align*}
$ab=0$が成り立つ。\par
$\forall a,b\in R$に対し、$a=0$かつ$b=0$が成り立つなら、次のようになるので、
\begin{align*}
a = 0 \land b = 0 &\Rightarrow a^{2} = 0 \land b^{2} = 0\\
&\Rightarrow a^{2} + b^{2} = 0
\end{align*}
$a^2 +b^2 =0$が成り立つ。\par
$\forall a\in R$に対し、その元$a$が可逆元なら、次のようになるので、
\begin{align*}
\left( -a^{-1} \right) \left(-a\right) &= a^{-1} a =1\\
\left( -a\right) \left(-a^{-1} \right) &= aa^{-1} =1
\end{align*}
その元$-a$も可逆元で$(-a)^{-1} =-a^{-1}$が成り立つ。\par
$\forall a\in R$に対し、それらの元々$a$、$b$が可逆元なら、次のようになるので、
\begin{align*}
\left( b^{-1} a^{-1} \right) \left(ab\right) &= b^{-1} \left( a^{-1} a\right) b \\
&= b^{-1} 1 b =b^{-1} b =1\\
\left(ab\right)\left( b^{-1} a^{-1} \right) &= a\left( bb^{-1} \right) a^{-1} \\
&=a1a^{-1} =aa^{-1} =1
\end{align*}
その元$ab$も可逆元で$(ab)^{-1} =b^{-1} a^{-1}$が成り立つ。
\end{proof}
\subsubsection{vector空間}
\begin{axs}[vector空間の公理]
集合$K$を体とする。空集合でない集合$V$に対し加法$+ :V \times V \rightarrow V;\left( \mathbf{v},\mathbf{w} \right) \mapsto \mathbf{v} + \mathbf{w}$が与えられたとする。このとき、次の条件たちを満たす集合$V$を体$K$上のvector空間、有向量空間、線形空間、線型空間という。
\begin{itemize}
\item
  集合$V$は加法について可換群$(V, + )$をなす。
\item
  $\forall k \in K\forall\mathbf{v} \in V$に対し、scalar倍$\cdot :K \times V \rightarrow V;\left( k,\mathbf{v} \right) \mapsto k\mathbf{v}$が定義されている。
\item
  $\forall k \in K\forall\mathbf{v},\mathbf{w} \in V$に対し、$k\left( \mathbf{v} + \mathbf{w} \right) = k\mathbf{v}\mathbf{+}k\mathbf{w}$が成り立つ。
\item
  $\forall k,l \in K\forall\mathbf{v} \in V$に対し、$(k + l)\mathbf{v} = k\mathbf{v}\mathbf{+}l\mathbf{v}$が成り立つ。
\item
  $\forall k,l \in K\forall\mathbf{v} \in V$に対し、$(kl)\mathbf{v} = k\left( l\mathbf{v} \right)$が成り立つ。
\item
  $\exists 1 \in K\forall\mathbf{v} \in V$に対し、$1\mathbf{v} = \mathbf{v}$が成り立つ。
\end{itemize}
体$K$上のvector空間$V$の元をvector、有向量などといいふつう$\mathbf{v}$など小文字太字で表される。特に加法の単位元であるvectorを零vectorといい、\textbf{0}と表す。一方、vector空間を考えるとき体$K$の元をscalar、無向量などという。
\end{axs}
\begin{thm}\label{2.1.1.8}
体$K$上のvector空間$V$が与えられたとき、次のことが成り立つ。
\begin{itemize}
\item
  $\forall k \in K\forall\mathbf{v},\mathbf{w} \in V$に対し、$k\left( \mathbf{v}\mathbf{-}\mathbf{w} \right) = k\mathbf{v}\mathbf{-}k\mathbf{w}$が成り立つ。
\item
  $\forall k,l \in K\forall\mathbf{v} \in V$に対し、$(k - l)\mathbf{v} = k\mathbf{v} - l\mathbf{v}$が成り立つ。
\item
  $\exists\mathbf{0} \in V\forall k \in K$に対し、$k\mathbf{0} = \mathbf{0}$が成り立つ。
\item
  $\exists 0 \in K\exists\mathbf{0} \in V\forall\mathbf{v} \in V$に対し、$0\mathbf{v} = \mathbf{0}$が成り立つ。
\item
  $\forall k \in K\forall\mathbf{v} \in V$に対し、$( - k)\mathbf{v} = k\left( - \mathbf{v} \right) = - k\mathbf{v}$が成り立つ。
\end{itemize}
\end{thm}
\begin{proof}
体$K$上のvector空間$V$が与えられたとき、$\forall k \in K\forall\mathbf{v},\mathbf{w} \in V$に対し、次のようになる。
\begin{align*}
k\left( \mathbf{v}\mathbf{-}\mathbf{w} \right) &= k\left( \mathbf{v}\mathbf{-}\mathbf{w} \right) + k\mathbf{w} - k\mathbf{w}\\
&=k\left( \left( \mathbf{v} - \mathbf{w} \right) + \mathbf{w} \right) - k\mathbf{w}\\
&=k\mathbf{v} - k\mathbf{w}\\
&=k\left( \mathbf{v}\mathbf{-}\mathbf{w} \right)\\
&= k\mathbf{v} - k\mathbf{w}
\end{align*}\par
$\forall k,l \in K\forall\mathbf{v} \in V$に対し、次のようになる。
\begin{align*}
(k - l)\mathbf{v} &= (k - l)\mathbf{v} + l\mathbf{v} - l\mathbf{v}\\
&=\left( (k - l) + l \right)\mathbf{v} - l\mathbf{v}\\
&=k\mathbf{v} - l\mathbf{v}\\
&=(k - l)\mathbf{v}\\
&= k\mathbf{v} - l\mathbf{v}
\end{align*}\par
$\exists\mathbf{0} \in V\forall k \in K\forall\mathbf{v} \in V$に対し、次のようになる。
\begin{align*}
k\mathbf{0} &= k\left( \mathbf{v}\mathbf{-}\mathbf{v} \right)\\
&= k\mathbf{v}\mathbf{-}k\mathbf{v} = \mathbf{0}
\end{align*}\par
$\exists 0 \in K\exists\mathbf{0} \in V\forall k \in K\forall\mathbf{v} \in V$に対し、次のようになる。
\begin{align*}
0\mathbf{v} &= (k - k)\mathbf{v}\\
&= k\mathbf{v} - k\mathbf{v} = \mathbf{0}
\end{align*}\par
$\exists 0 \in K\forall k \in K\forall\mathbf{v} \in V$に対し、次のようになる。
\begin{align*}
( - k)\mathbf{v} &= (0 - k)\mathbf{v}\\
&= 0\mathbf{v} - k\mathbf{v}\\
&= \mathbf{0} - k\mathbf{v} = - k\mathbf{v}
\end{align*}\par
$\exists\mathbf{0} \in V\forall k \in K\forall\mathbf{v} \in V$に対し、次のようになる。
\begin{align*}
k\left( - \mathbf{v} \right) &= k\left( \mathbf{0} - \mathbf{v} \right)\\
&= k\mathbf{0} - k\mathbf{v}\\
&= \mathbf{0} - k\mathbf{v} = - k\mathbf{v}
\end{align*}
\end{proof}
%\hypertarget{ux90e8ux5206ux7a7aux9593}{%
\subsubsection{部分空間}%\label{ux90e8ux5206ux7a7aux9593}}
\begin{dfn}
体$K$上のvector空間$V$の部分集合$W$もvector空間となっているものをそのvector空間$V$の部分空間という。
\end{dfn}
\begin{thm}\label{2.1.1.9}
体$K$上のvector空間$V$の部分集合$W$について、次のことは同値である。
\begin{itemize}
\item
  その部分集合$W$は部分空間である。
\item
  次のことが成り立つ。
  \begin{itemize}
  \item
    $\mathbf{0} \in W$が成り立つ。
  \item
    $\forall k,l \in K\forall\mathbf{v},\mathbf{w} \in W$に対し、$k\mathbf{v} + l\mathbf{w} \in W$が成り立つ。
  \end{itemize}
\item
  次のことが成り立つ。
  \begin{itemize}
  \item
    その集合$W$は空でない。
  \item
    $\forall\mathbf{v},\mathbf{w} \in W$に対し、$\mathbf{v} + \mathbf{w} \in W$が成り立つ。
  \item
    $\forall k \in K\forall\mathbf{v} \in W$に対し、$k\mathbf{v} \in W$が成り立つ。
  \end{itemize}
\end{itemize}
\end{thm}
\begin{proof}
体$K$上のvector空間$V$の部分集合$W$について、その部分集合$W$が部分空間であるなら、明らかに次のことが成り立つ。
\begin{itemize}
\item
  $\mathbf{0} \in W$が成り立つ。
\item
  $\forall k,l \in K\forall\mathbf{v},\mathbf{w} \in W$に対し、$k\mathbf{v} + l\mathbf{w} \in W$が成り立つ。
\end{itemize}\par
逆に、次のことが成り立つなら、
\begin{itemize}
\item
  $\mathbf{0} \in W$が成り立つ。
\item
  $\forall k,l \in K\forall\mathbf{v},\mathbf{w} \in W$に対し、$k\mathbf{v} + l\mathbf{w} \in W$が成り立つ。
\end{itemize}
$\forall\mathbf{u},\mathbf{v},\mathbf{w} \in W$に対し、$\mathbf{u},\mathbf{v},\mathbf{w} \in V$が成り立つので、$- \mathbf{v} \in W$が成り立つ。これに注意すれば、次のようになる。
\begin{itemize}
\item
  $\forall\mathbf{u},\mathbf{v},\mathbf{w} \in W$に対し、$\left( \mathbf{u} + \mathbf{v} \right) + \mathbf{w} = \mathbf{u} + \left( \mathbf{v} + \mathbf{w} \right)$が成り立つ。
\item
  $\exists\mathbf{0} \in W\forall\mathbf{v} \in W$に対し、$\mathbf{v} + \mathbf{0} = \mathbf{0} + \mathbf{v} = \mathbf{v}$が成り立つ。
\item
  $\forall\mathbf{v} \in W\exists - \mathbf{v} \in W$に対し、$\mathbf{v} - \mathbf{v} = - \mathbf{v} + \mathbf{v} = \mathbf{0}$が成り立つ。
\item
  $\forall\mathbf{v},\mathbf{w} \in W$に対し、$\mathbf{v} + \mathbf{w} = \mathbf{w} + \mathbf{v}$が成り立つ。
\end{itemize}
ゆえに、その組$(W, + )$は可換群をなす。さらに、次のことも成り立つ。
\begin{itemize}
\item
  $\forall k \in K\forall\mathbf{v},\mathbf{w} \in W$に対し、$k\left( \mathbf{v} + \mathbf{w} \right) = k\mathbf{v}\mathbf{+}k\mathbf{w}$が成り立つ。
\item
  $\forall k,l \in K\forall\mathbf{v} \in W$に対し、$(k + l)\mathbf{v} = k\mathbf{v}\mathbf{+}l\mathbf{v}$が成り立つ。
\item
  $\forall k,l \in K\forall\mathbf{v} \in W$に対し、$(kl)\mathbf{v} = k\left( l\mathbf{v} \right)$が成り立つ。
\item
  $\exists 1 \in K\forall\mathbf{v} \in W$に対し、$1\mathbf{v} = \mathbf{v}$が成り立つ。
\end{itemize}
よって、その集合$W$はそのvector空間$V$の部分空間である。\par
次の条件たちを満たす体$K$上のvector空間$V$の部分集合$W$が与えられたとき、
\begin{itemize}
\item
  $\mathbf{0} \in W$が成り立つ。
\item
  $\forall k,l \in K\forall\mathbf{v},\mathbf{w} \in W$に対し、$k\mathbf{v} + l\mathbf{w} \in W$が成り立つ。
\end{itemize}
$\mathbf{0} \in W$が成り立つなら、明らかにその集合$W$は空でない。$\forall k,l \in K\forall\mathbf{v},\mathbf{w} \in W$に対し、$k\mathbf{v} + l\mathbf{w} \in W$が成り立つことより、$k = l = 1$とすれば、$\forall\mathbf{v},\mathbf{w} \in W$に対し、$\mathbf{v} + \mathbf{w} \in W$が成り立つ。さらに、$\mathbf{w} = \mathbf{0}$とすれば、$\forall k \in K\forall\mathbf{v} \in W$に対し、$k\mathbf{v} \in W$が成り立つ。以上より、次のことが成り立つ。
\begin{itemize}
\item
  その集合$W$は空でない。
\item
  $\forall\mathbf{v},\mathbf{w} \in W$に対し、$\mathbf{v} + \mathbf{w} \in W$が成り立つ。
\item
  $\forall k \in K\forall\mathbf{v} \in W$に対し、$k\mathbf{v} \in W$が成り立つ。
\end{itemize}\par
逆に、これらが成り立つなら、その体$K$は加法について群$(K, + )$をなしているのであったので、零元$0$はその体$K$に属しその集合$W$は空集合でなく、$\forall k \in K\forall\mathbf{v} \in W$に対し、$k\mathbf{v} \in W$が成り立つのであったので、上記の定理より$0\mathbf{v} = \mathbf{0} \in W$が成り立ち、したがって、$\mathbf{0} \in W$が成り立つ。また、$\forall\mathbf{v},\mathbf{w} \in W$に対し、$\mathbf{v} + \mathbf{w} \in W$が成り立つかつ、$\mathbf{v} \in W$が成り立つなら、$\forall k \in K\forall\mathbf{v} \in W$に対し、$k\mathbf{v} \in W$が成り立つのであったので、$\forall k,l \in K\forall\mathbf{v},\mathbf{w} \in W$に対し、$k\mathbf{v},l\mathbf{w} \in W$が成り立ち、これが成り立つなら、$k\mathbf{v} + l\mathbf{w} \in W$も成り立つ。以上より、次のことが成り立つ。
\begin{itemize}
\item
  $\mathbf{0} \in W$が成り立つ。
\item
  $\forall k,l \in K\forall\mathbf{v},\mathbf{w} \in W$に対し、$k\mathbf{v} + l\mathbf{w} \in W$が成り立つ。
\end{itemize}
\end{proof}
\begin{thm}\label{2.1.1.10}
体$K$上のvector空間$V$の部分集合の族$\left\{ W_{\lambda} \right\}_{\lambda \in \varLambda}$が与えられたとき、この集合$\bigcap_{\lambda \in \varLambda} W_{\lambda}$もそのvector空間$V$の部分空間である。
\end{thm}
\begin{proof}
体$K$上のvector空間$V$の部分集合の族$\left\{ W_{\lambda} \right\}_{\lambda \in \varLambda}$が与えられたとき、この集合$\bigcap_{\lambda \in \varLambda} W_{\lambda}$において、定理\ref{2.1.1.9}より$\forall\lambda \in \varLambda$に対し、$\mathbf{0} \in W_{\lambda}$が成り立つので、$\mathbf{0} \in \bigcap_{\lambda \in \varLambda} W_{\lambda}$が成り立つ。さらに、$\forall k,l \in K\forall\mathbf{v},\mathbf{w} \in \bigcap_{\lambda \in \varLambda} W_{\lambda}$に対し、積集合の定義より$\forall\lambda \in \varLambda$に対し、$\mathbf{v},\mathbf{w} \in W_{\lambda}$が成り立つので、定理\ref{2.1.1.9}より$k\mathbf{v} + l\mathbf{w} \in W_{\lambda}$が成り立つ。したがって、$k\mathbf{v} + l\mathbf{w} \in \bigcap_{\lambda \in \varLambda} W_{\lambda}$が成り立つ。以上、定理\ref{2.1.1.9}よりこの集合$\bigcap_{\lambda \in \varLambda} W_{\lambda}$もそのvector空間$V$の部分空間である。
\end{proof}
%\hypertarget{ux7ddaux5f62ux7d50ux5408}{%
\subsubsection{線形結合}%\label{ux7ddaux5f62ux7d50ux5408}}
\begin{dfn}
体$K$上のvector空間$V$の任意の部分集合$M$が与えられたとき、そのvector空間$V$の部分空間全体の集合$\mathfrak{S}_{V}$を用いた集合$\bigcap_{M\subseteq W\in \mathfrak{S}_{V} } W$も定理\ref{2.1.1.10}よりそのvector空間$V$の部分空間となるのであった。このとき、その部分空間をその部分集合$M$から生成される部分空間、その部分集合$M$によって張られる部分空間などといい、${\mathrm{span}}M$、$\left\langle \mathbf{v} \right\rangle_{\mathbf{v} \in M}$などと書く。その部分集合$M$の元をこの部分空間$W$の生成元という。
\end{dfn}
\begin{dfn}
体$K$上のvector空間$V$の族$\left\{ \mathbf{v}_i \right\}_{i\in \varLambda_n} $が与えられたとき、そのvector空間$V$の元で$c_i \in K$なる元$c_i $を用いて$\sum_{i \in \varLambda_{n}} {c_{i}\mathbf{v}_{i}}$と表されるものを族$\left\{ \mathbf{v}_i \right\}_{i\in \varLambda_n } $の線形結合、線型結合、1次結合という。また、明らかに$\forall k,c_{i},d_{i} \in K\forall\mathbf{v}_{i} \in V$に対し、$\sum_{i \in \varLambda_{n}} {c_{i}\mathbf{v}_{i}}$、$\sum_{i \in \varLambda_{n}} {d_{i}\mathbf{v}_{i}}$について、$\sum_{i \in \varLambda_{n}} {c_{i}\mathbf{v}_{i}} + \sum_{i \in \varLambda_{n}} {d_{i}\mathbf{v}_{i}}$、$k\sum_{i \in \varLambda_{n}} {c_{i}\mathbf{v}_{i}}$も族$\left\{ \mathbf{v}_i \right\}_{i\in \varLambda_n } $の線形結合である。
\end{dfn}
\begin{thm}\label{2.1.1.11}
体$K$上のvector空間$V$の任意の部分集合$M$が与えられたとき、これが有限集合で$M = \left\{ \mathbf{v}_{i} \right\}_{i \in \varLambda_{n}}$とおくとき、次式が成り立つ。
\begin{align*}
{\mathrm{span}}M = \left\{ \mathbf{v} \in V \middle| \exists k_{i} \in K\left\lbrack \mathbf{v} = \sum_{i \in \varLambda_{n}} {k_{i}\mathbf{v}_{i}} \right\rbrack \right\}
\end{align*}
\end{thm}
\begin{proof}
体$K$上のvector空間$V$の任意の部分集合$M$が与えられたとき、これが有限集合で$M = \left\{ \mathbf{v}_{i} \right\}_{i \in \varLambda_{n}}$とおくとき、次式のようにおくと、
\begin{align*}
M' = \left\{ \mathbf{v} \in V \middle| \exists k_{i} \in K\left\lbrack \mathbf{v} = \sum_{i \in \varLambda_{n}} {k_{i}\mathbf{v}_{i}} \right\rbrack \right\}
\end{align*}
もちろん、$M \subseteq M'$が成り立つ。そこで、明らかに$\mathbf{0} \in M'$が成り立つ。$\forall k,l \in K\forall\mathbf{v},\mathbf{w} \in M'$に対し、その体$K$のある元々$k_{i}$、$l_{i}$が存在して、次式が成り立つ。
\begin{align*}
\mathbf{v} = \sum_{i \in \varLambda_{n}} {k_{i}\mathbf{v}_{i}},\ \ \mathbf{w} = \sum_{i \in \varLambda_{n}} {l_{i}\mathbf{v}_{i}}
\end{align*}
このとき、次のようになることから、
\begin{align*}
k\mathbf{v} + l\mathbf{w} = k\sum_{i \in \varLambda_{n}} {k_{i}\mathbf{v}_{i}} + l\sum_{i \in \varLambda_{n}} {l_{i}\mathbf{v}_{i}} = \sum_{i \in \varLambda_{n}} {kk_{i}\mathbf{v}_{i}} + \sum_{i \in \varLambda_{n}} {ll_{i}\mathbf{v}_{i}} = \sum_{i \in \varLambda_{n}} {\left( kk_{i} + ll_{i} \right)\mathbf{v}_{i}}
\end{align*}
$k\mathbf{v} + l\mathbf{w} \in M'$が成り立つ。ゆえに、定理\ref{2.1.1.9}よりその集合$M'$はそのvector空間$V$の部分空間である。これにより、${\mathrm{span}}M \subseteq M'$が成り立つ。\par
逆に、$\forall\mathbf{v} \in M'$に対し、$\mathbf{v} = \sum_{i \in \varLambda_{n}} {k_{i}\mathbf{v}_{i}}$とおくと、$M \subseteq {\mathrm{span}}M$が成り立つので、$\forall i \in \varLambda_{n}$に対し、$\mathbf{v}_{i} \in {\mathrm{span}}M$が成り立つ。そこで、その集合${\mathrm{span}}M$は部分空間なので、定理\ref{2.1.1.9}と数学的帰納法により$\sum_{i \in \varLambda_{n}} {k_{i}\mathbf{v}_{i}} \in {\mathrm{span}}M$が成り立つので、$M' \subseteq {\mathrm{span}}M$が成り立つ。\par
よって、$M' = {\mathrm{span}}M$が得られる。
\end{proof}\par
もちろん、体$K$上のvector空間$V$において、$\forall i \in \varLambda_{n}$に対し、$c_{i} = 0$が成り立つなら、vector$\sum_{i \in \varLambda_{n}} {c_{i}\mathbf{v}_{i}}$は零vectorとなる。ところが、これの逆はいつでも成り立つとは限らない。例えば、$\mathbf{v} \in V$なるvector$\mathbf{v}$を用いて、$\mathbf{v} - \mathbf{v} = \mathbf{0}$を考えればよい。
\begin{dfn}
体$K$上のvector空間$V$が与えられたとき、$\sum_{i \in \varLambda_{n}} {c_{i}\mathbf{v}_{i}} = \mathbf{0}$が成り立つなら、$\mathbf{\forall}i \in \varLambda_{n}$に対し、$c_{i} = 0$が成り立つとき、族$\left\{ \mathbf{v}_i \right\}_{i\in \varLambda_n } $は体$K$上で線形独立、線型独立、1次独立であるという。\par
一方、この式の否定が成り立つとき、即ち、$\sum_{i \in \varLambda_{n}} {c_{i}\mathbf{v}_{i}} = \mathbf{0}$が成り立つかつ、$\exists i \in \varLambda_{n}$に対し、$c_{i} \neq 0$が成り立つとき、族$\left\{ \mathbf{v}_i \right\}_{i\in \varLambda_n } $は体$K$上で線形従属、線型従属、1次従属であるという。
\end{dfn}
\begin{thm}\label{2.1.1.12}
体$K$上のvector空間$V$が与えられたとき、線形独立な族$\left\{ \mathbf{v} \right\}_{i\in \varLambda_n } $に零vector$\mathbf{0}$は含まれない。
\end{thm}\par
これは背理法で示される。
\begin{proof}
体$K$上のvector空間$V$が与えられたとき、線形独立な族$\left\{ \mathbf{v} \right\}_{i\in \varLambda_n } $に零vector$\mathbf{0}$は含まれるとすると、その添数集合$\varLambda_{n}$の部分集合$\left\{ i \in \varLambda_{n} \middle| \mathbf{v}_{i} = \mathbf{0} \right\}$は空集合でなく、これを$\varLambda'$とおけば、$\mathbf{\forall}i \in \varLambda'$に対し、$c_{i} \neq 0$が成り立っても、$\sum_{i \in \varLambda_{n}} {c_{i}\mathbf{v}_{i}} = \mathbf{0}$が成り立ち、したがって、次のようになる。
\begin{align*}
\sum_{i \in \varLambda_{n}} {c_{i}\mathbf{v}_{i}} = \mathbf{0 \land \exists}i \in \varLambda_{n}\left\lbrack c_{i} \neq 0 \right\rbrack &\Leftrightarrow \neg\neg\left( \sum_{i \in \varLambda_{n}} {c_{i}\mathbf{v}_{i}} = \mathbf{0} \right)\mathbf{\land \neg}\left( \mathbf{\forall}i \in \varLambda_{n}\left\lbrack c_{i} = 0 \right\rbrack \right)\\
&\mathbf{\Leftrightarrow \neg}\left( \neg\left( \sum_{i \in \varLambda_{n}} {c_{i}\mathbf{v}_{i}} = \mathbf{0} \right)\mathbf{\vee \forall}i \in \varLambda_{n}\left\lbrack c_{i} = 0 \right\rbrack \right)\\
&\mathbf{\Leftrightarrow \neg}\left( \sum_{i \in \varLambda_{n}} {c_{i}\mathbf{v}_{i}} = \mathbf{0 \Rightarrow \forall}i \in \varLambda_{n}\left\lbrack c_{i} = 0 \right\rbrack \right)
\end{align*}
ゆえに、族$\left\{ \mathbf{v}_i \right\}_{i\in \varLambda_n } $が線形独立でないことになるが、これは矛盾している。
\end{proof}
\begin{thm}\label{2.1.1.13}
体$K$上のvector空間$V$が与えられたとき、線形独立な族$\left\{ \mathbf{v} \right\}_{i\in \varLambda_n } $が与えられたとき、その添数集合$\varLambda_{n}$の空集合でない任意の部分集合$\varLambda'$に対し、族$\left\{ \mathbf{v}_i \right\}_{i \in \varLambda' } $も再び線形独立である。
\end{thm}\par
これは、線形独立な族$\left\{ \mathbf{v} \right\}_{i\in \varLambda_n } $のうちどのようにとっても、とられたvectorsもまた線形独立であるということを主張している。ここでも、やはり背理法で示した。
\begin{proof}
体$K$上のvector空間$V$が与えられたとき、そのvector空間$V$の族$\left\{ \mathrm{v}_i \right\}_{i\in \varLambda_n } $が線形独立であるとする。その添数集合$\varLambda_{n}$の空集合でないある部分集合$\varLambda'$が存在して、その体$K$の族々$\left\{ c_{i} \right\}_{i \in \varLambda_{n}}$、$\left\{ c_{i}' \right\}_{i \in \varLambda'}$を用いて$\sum_{i \in \varLambda' \subseteq \varLambda_{n}} {c_{i}'\mathbf{v}_{i}} = \mathbf{0}$が成り立つかつ、$c_{i}' \neq 0$となるような添数$i$がその集合$\varLambda'$に存在すると仮定しよう。このとき、$\sum_{i \in \varLambda_{n}} {c_{i}\mathbf{v}_{i}} = \mathbf{0}$が成り立つなら、$\mathbf{\forall}i \in \varLambda_{n}$に対し、$c_{i} = 0$が成り立つので、やはり、$\sum_{i \in \varLambda_{n} \setminus \varLambda'} {c_{i}\mathbf{v}_{i}} = \mathbf{0}$が成り立ち、したがって、次式が成り立つ。
\begin{align*}
\sum_{i \in \varLambda' \subseteq \varLambda_{n}} {c_{i}'\mathbf{v}_{i}} + \sum_{i \in \varLambda_{n} \setminus \varLambda'} {c_{i}\mathbf{v}_{i}} = \mathbf{0}
\end{align*}
ここで、族$\left\{ \mathbf{v}_i \right\}_{i\in \varLambda_n } $が線形独立であるので、$\mathbf{\forall}i \in \varLambda_{n}$に対し、$c_{i} = c_{i}' = 0$が成り立つことになる。しかしながら、これは$c_{i}' \neq 0$となるような添数$i$がその集合$\varLambda'$に存在することに矛盾する。よって、その添数集合$\varLambda_{n}$の空集合でないある部分集合$\varLambda'$が存在して、$\sum_{i \in \varLambda' \subseteq \varLambda_{n}} {c_{i}'\mathbf{v}_{i}} = \mathbf{0}$が成り立つかつ、$c_{i}' \neq 0$となるような添数$i$がその集合$\varLambda'$に存在することが否定され、その添数集合$\varLambda_{n}$の空集合でない任意の部分集合$\varLambda'$に対し、$\sum_{i \in \varLambda' \subseteq \varLambda_{n}} {c_{i}'\mathbf{v}_{i}} = \mathbf{0}$が成り立つなら、$\forall i \in \varLambda'$に対し、$c_{i}' = 0$が成り立つ。
\end{proof}
\begin{thm}\label{2.1.1.14}
体$K$上のvector空間$V$が与えられたとき、線形独立なそのvector空間$V$の族$\left\{ \mathbf{v}_i \right\}_{i\in \varLambda_n } $のうち異なるどの2つ$\mathbf{v}_{i}$、$\mathbf{v}_{j}$もその体のある元$k$を用いて$\mathbf{v}_{i} = k\mathbf{v}_{j}$が成り立つようなことはない、即ち、$\mathbf{\forall}i,j \in \varLambda_{n}\forall k \in K$に対し、$i \neq j$が成り立つなら、$\mathbf{v}_{i} \neq k\mathbf{v}_{j}$が成り立つ。
\end{thm}\par
これもやはり背理法で示される。
\begin{proof}
体$K$上のvector空間$V$が与えられたとき、そのvector空間$V$の族$\left\{ \mathrm{v}_i \right\}_{i\in \varLambda_n } $が線形独立であるとき、$i \neq j$かつ$\mathbf{v}_{i} = k\mathbf{v}_{j}$が成り立つようなvectors$\mathbf{v}_{i}$、$\mathbf{v}_{j}$とその体$K$の元$k$が存在すると仮定しよう。このとき、$k = 0$と仮定すると、vector$\mathbf{v}_{i}$は零vectorになり定理\ref{2.1.1.12}の線形独立なvectorsに零vectorが含まれないことに矛盾する。したがって、$k \neq 0$が成り立つことになり、したがって、次のようなり、
\begin{align*}
\sum_{i' \in \varLambda_{n}} {c_{i'}\mathbf{v}_{i'}} &= \sum_{i' \in \varLambda_{n} \setminus \left\{ i,j \right\}} {c_{i'}\mathbf{v}_{i'}} + c_{i}\mathbf{v}_{i} + c_{j}\mathbf{v}_{j}\\
&= \sum_{i' \in \varLambda_{n} \setminus \left\{ i,j \right\}} {c_{i'}\mathbf{v}_{i'}} + c_{i}\mathbf{v}_{i} + kc_{j}\mathbf{v}_{i}\\
&= \sum_{i' \in \varLambda_{n} \setminus \left\{ i,j \right\}} {c_{i'}\mathbf{v}_{i'}} + \left( c_{i} + kc_{j} \right)\mathbf{v}_{j} = \mathbf{0}
\end{align*}
ここで、上記の定理\ref{2.1.1.13}により族$\left\{ \mathbf{v}_i \right\}_{i \in \varLambda_{n} \setminus \left\{ j \right\} } $もまた線形独立であるので、$\forall i \in \varLambda_{n} \setminus \left\{ i,j \right\}$に対し、$c_{i} = 0$が成り立つかつ、$c_{i} + kc_{j} = 0$が成り立つことになる。ここで、$c_{i} = 1$かつ$c_{j} = - \frac{1}{k}$とおけば、$c_{i} \neq 0$かつ$c_{j} \neq 0$が成り立つことになり、したがって、$\sum_{i \in \varLambda_{n}} {c_{i}\mathbf{v}_{i}} = \mathbf{0}$が成り立つかつ、$c_{i} \neq 0$かつ$c_{j} \neq 0$が成り立つような添数$i$、$j$がその添数集合$\varLambda_{n}$に存在することになる。しかしながら、これは族$\left\{ \mathbf{v}_i \right\}_{i\in \varLambda_n } $が線形独立であることに矛盾する。よって、線形独立な族$\left\{ \mathbf{v} \right\}_{i\in \varLambda_n } $のうちどの2つ$\mathbf{v}_{i}$、$\mathbf{v}_{j}$もその体のある元$k$を用いて$\mathbf{v}_{i} = k\mathbf{v}_{j}$が成り立つようなことはない。
\end{proof}
\begin{thm}\label{2.1.1.15}
体$K$上のvector空間$V$が与えられ、そのvector空間$V$の族$\left\{ \mathrm{v}_i \right\}_{i\in \varLambda_n } $が線形独立であるかつ、族$\left\{ \mathbf{v}_i \right\}_{i \in \varLambda_{n + 1} } $が線形従属であるとき、そのvector$\mathbf{v}_{n + 1}$は$\left\{ \mathbf{v}_i \right\}_{i \in \varLambda_{n} } $の線形結合である、即ち、次式が成り立つようなその体$K$の族$\left\{ k_{i} \right\}_{i \in \varLambda_{n}}$が存在する。
\begin{align*}
\mathbf{v}_{n + 1} = \sum_{i \in \varLambda_{n}} {k_{i}\mathbf{v}_{i}}
\end{align*}
\end{thm}
\begin{proof}
体$K$上のvector空間$V$が与えられ、そのvector空間$V$の族$\left\{ \mathrm{v}_i \right\}_{i\in \varLambda_n } $が線形独立であるかつ、族$\left\{ \mathbf{v}_i \right\}_{i \in \varLambda_{n + 1} } $が線形従属であるとき、$c_{n + 1} = 0$が成り立つなら、$c_{i} \neq 0$が成り立つような添数$i$が添数集合$\varLambda_{n}$に存在するかつ、次のようになり、
\begin{align*}
\sum_{i \in \varLambda_{n + 1}} {c_{i}\mathbf{v}_{i}} = \sum_{i \in \varLambda_{n}} {c_{i}\mathbf{v}_{i}} + c_{n + 1}\mathbf{v}_{n + 1} = \sum_{i \in \varLambda_{n}} {c_{i}\mathbf{v}_{i}} = \mathbf{0}
\end{align*}
族$\left\{ \mathbf{v}_i \right\}_{i\in \varLambda_n } $が線形独立であるので、$\forall i \in \varLambda_{n}$に対し、$c_{i} = 0$が成り立つことに矛盾する。したがって、$c_{n + 1} \neq 0$が成り立つことになる。このとき、次のようになる。
\begin{align*}
\sum_{i \in \varLambda_{n + 1}} {c_{i}\mathbf{v}_{i}} = \mathbf{0} &\Leftrightarrow \sum_{i \in \varLambda_{n}} {c_{i}\mathbf{v}_{i}} + c_{n + 1}\mathbf{v}_{n + 1} = \mathbf{0} \\
&\Leftrightarrow c_{n + 1}\mathbf{v}_{n + 1} = - \sum_{i \in \varLambda_{n}} {c_{i}\mathbf{v}_{i}} \\
&\Leftrightarrow \mathbf{v}_{n + 1} = - \frac{1}{c_{n + 1}}\sum_{i \in \varLambda_{n}} {c_{i}\mathbf{v}_{i}} \\
&\Leftrightarrow \mathbf{v}_{n + 1} = \sum_{i \in \varLambda_{n}} {\left( - \frac{c_{i}}{c_{n + 1}} \right)\mathbf{v}_{i}}
\end{align*}
ここで$- \frac{c_{i}}{c_{n + 1}} \in K$が成り立つので、$k_{i} = - \frac{c_{i}}{c_{n + 1}}$とおくと、その体$K$の族$\left\{ k_{i} \right\}_{i \in \varLambda_{n}}$が存在して、$\mathbf{v}_{n + 1} = \sum_{i \in \varLambda_{n}} {k_{i}\mathbf{v}_{i}}$が成り立つ。
\end{proof}
%\hypertarget{ux57faux5e95}{%
\subsubsection{基底}%\label{ux57faux5e95}}
\begin{dfn}
体$K$上のvector空間$V$が与えられたとする。そのvector空間$V$の部分集合$B$が次のことを満たすとき、その部分集合$B$をそのvector空間$V$の基底という。
\begin{itemize}
\item
  その集合$B$の任意の有限な部分集合$\left\{ \mathbf{v}_{i} \right\}_{i \in \varLambda_{n}}$が線形独立である。
\item
  その集合$B$がそのvector空間$V$を生成する、即ち、${\mathrm{span}}B = V$が成り立つ。
\end{itemize}
特に、${card}B < \aleph_{0}$のとき、このvector空間$V$は有限生成であるといい、${card}B = n$とおいて組$\left( \mathbf{v}_{i} \right)_{i \in \varLambda_{n}}$のほうをそのvector空間$V$の基底というときがあり、さらに、このような意味でその組が$\left\langle \mathbf{v}_{i} \right\rangle_{i \in \varLambda_{n}}$と書かれることがある。なお、$V = \left\{ \mathbf{0} \right\}$のときの基底は空集合である。
\end{dfn}
\begin{thm}[基底の存在性]\label{2.1.1.16}
体$K$上の任意のvector空間$V$は基底をもつ。これは次のようにして示される。
\begin{enumerate}
\item
  $V = \left\{ \mathbf{0} \right\}$のときは明らかである。
\item
  そのvector空間$V$は明らかに任意の有限の個数のvectorsが線形独立な部分集合をもつ。
\item
  そのvector空間$V$の部分集合の任意の有限の個数の元々が線形独立であることは有限的な性質となる\footnote{ある集合$A$の部分集合$A'$に関する命題$P\left( A' \right)$があって、その集合$A$の部分集合$A''$について$P\left( A'' \right)$が成り立つこととその集合$A''$の全ての有限な部分集合たち$A'''$について$P\left( A''' \right)$が成り立つことが同値であるとき、その命題$P$を有限的な性質、有限的な条件などといったりします。}。
\item
  Tukeyの補題より\footnote{これは集合$A$の部分集合に関する有限的な性質$P$を満たすようなその集合$A$の部分集合が少なくとも1つ存在するのであれば、その命題$P$を満たすようなその集合$A$の部分集合全体の集合を$\mathfrak{M}$とおくとき、順序集合$\mathfrak{(M, \subseteq )}$で極大な部分集合が存在するということを主張する定理です。}3. の性質をもつ部分集合のうち順序関係$\subseteq$の意味で極大となるもの$B$が存在する。
\item
  $\forall\mathbf{v} \in V$に対し、$\mathbf{v} \in B$が成り立つなら、これは明らかである。
\item
  $\forall\mathbf{v} \in V$に対し、$\mathbf{v} \notin B$が成り立つなら、その部分集合$B \cup \left\{ \mathbf{v} \right\}$は3. の性質をもたない。
\item
  これにより、集合$B \cup \left\{ \mathbf{v} \right\}$のある有限の個数のそのvector$\mathbf{v}$が含まれる元々が存在して、線形従属であることになる。
\item
  さらに、このようなvectorsが$\left\{ \mathbf{v}_{i} \right\}_{i \in \varLambda_{n} \cup \left\{ 0 \right\}}$、$\mathbf{v}_{0} = \mathbf{v}$とおかれると、$\exists i \in \varLambda_{n} \cup \left\{ 0 \right\}$に対し、$c_{i} \neq 0$なるその体$K$の元々$c_{i}$を用いて次式のように表されることができるかつ、$c_{0} \neq 0$が成り立つ。
\begin{align*}
\sum_{i \in \varLambda_{n} \cup \left\{ 0 \right\}} {c_{i}\mathbf{v}_{i}} = \mathbf{0}
\end{align*}
\item
  これにより、結論が得られる。
\end{enumerate}
\end{thm}
\begin{proof}
体$K$上のvector空間$V$が与えられたとき、$V = \left\{ \mathbf{0} \right\}$のときは明らかであるから、そのvector空間$V$が$\left\{ \mathbf{0} \right\}$でないとしよう。このとき、そのvector空間$V$は任意の有限の個数のvectorsが線形独立な部分集合をもつ。例えば、零vectorでないvector$\mathbf{v}$を用いた集合$\left\{ \mathbf{v} \right\}$が挙げられる。\par
ここで、そのvector空間$V$の部分集合$V'$の任意の有限の個数の元々は線形独立であるなら、その集合$V'$の任意の有限な部分集合の任意の有限の個数の元々も線形独立となる。逆に、その集合$V'$の任意の有限な部分集合の任意の有限の個数の元々も線形独立であるなら、その部分集合$V'$の任意の有限の個数の元々はその部分集合$V'$の有限な部分集合をなすので、これらは線形独立である。以上より、そのvector空間$V$の部分集合の任意の有限の個数の元々が線形独立であるならそのときに限り、その集合$V'$の任意の有限な部分集合の任意の有限の個数の元々も線形独立である、即ち、そのvector空間$V$の部分集合の任意の有限の個数の元々が線形独立であることは有限的な性質である。\par
したがって、Tukeyの補題よりそのvector空間$V$の部分集合の任意の有限の個数の元々が線形独立であるようなその部分集合のうち順序関係$\subseteq$の意味で極大となるものが存在する。これのうち1つを$B$とすると、その部分集合$B$の任意の有限の個数の元々は線形独立である。\par
$\forall\mathbf{v} \in V$に対し、$\mathbf{v} \in B$が成り立つなら、そのvector$\mathbf{v}$はその部分集合$B$をなすvectorのうちどれかであるから、その部分集合$B$のある1つの元の線形結合である。$\mathbf{v} \notin B$が成り立つなら、その部分集合$B$がこれの任意の有限の個数の元々が線形独立であるようなその部分集合のうち順序関係$\subseteq$の意味で極大であったので、集合$B \cup \left\{ \mathbf{v} \right\}$のある有限の個数の元々が線形従属であることになる。ここで、このように線形従属なvectorsのうちそのvector$\mathbf{v}$が含まれていなければ、これらのvectors全体の集合はその集合$B$の部分集合であるので、定理\ref{2.1.1.15}よりこれらのvectorsは線形独立であることになり矛盾する。したがって、集合$B \cup \left\{ \mathbf{v} \right\}$のある有限の個数の元々が線形従属であるなら、これらの元々、即ち、vectorsのうちそのvector$\mathbf{v}$が含まれることになる。このようなvectorsが$\left\{ \mathbf{v}_{i} \right\}_{i \in \varLambda_{n} \cup \left\{ 0 \right\}}$、$\mathbf{v}_{0} = \mathbf{v}$とおかれると、$\exists i \in \varLambda_{n} \cup \left\{ 0 \right\}$に対し、$c_{i} \neq 0$なるその体$K$の元々$c_{i}$を用いて次式のように表されることができる。
\begin{align*}
\sum_{i \in \varLambda_{n} \cup \left\{ 0 \right\}} {c_{i}\mathbf{v}_{i}} = \mathbf{0}
\end{align*}
ここで、$c_{0} = 0$が成り立つとすれば、次式のようになり、
\begin{align*}
\sum_{i \in \varLambda_{n}} {c_{i}\mathbf{v}_{i}} = \mathbf{0}
\end{align*}
$\forall i \in \varLambda_{n}$に対し、$\mathbf{v}_{i} \in B$が成り立つことになり、定理\ref{2.1.1.15}よりこれらのvectors$\mathbf{v}_{i}$は線形独立であるので、$\forall i \in \varLambda_{n}$に対し、$c_{i} = 0$が成り立つことになり、したがって、$\forall i \in \varLambda_{n} \cup \left\{ 0 \right\}$に対し、$c_{i} = 0$が成り立つ。しかしながら、これは仮定に矛盾する。ゆえに、$c_{0} \neq 0$が成り立つ。したがって、次のようになる。
\begin{align*}
\sum_{i \in \varLambda_{n} \cup \left\{ 0 \right\}} {c_{i}\mathbf{v}_{i}} = \mathbf{0} &\Leftrightarrow c_{0}\mathbf{v}_{0} + \sum_{i \in \varLambda_{n}} {c_{i}\mathbf{v}_{i}} = \mathbf{0}\\
&\Leftrightarrow c_{0}\mathbf{v}_{0} = - \sum_{i \in \varLambda_{n}} {c_{i}\mathbf{v}_{i}}\\
&\Leftrightarrow \mathbf{v}_{0} = \sum_{i \in \varLambda_{n}} {\left( - \frac{c_{i}}{c_{0}} \right)\mathbf{v}_{i}}
\end{align*}\par
以上より、このvector空間$V$の任意の元が有限な個数のvectors$\mathbf{v}_{i}$の線形結合によって表されることができた。さらに、これらのvectors$\mathbf{v}_{i}$はいづれもその集合$B$に含まれるので、これらのvectors$\mathbf{v}_{i}$は線形独立である。よって、体$K$上の任意のvector空間$V$は基底をもつ。
\end{proof}
%\hypertarget{ux6b21ux5143}{%
\subsubsection{次元}%\label{ux6b21ux5143}}
\begin{thm}\label{2.1.1.17}
$\left( a_{ij} \right)_{i \in \varLambda_{m}} \in K^{m}$なる組$\left( a_{ij} \right)_{i \in \varLambda_{m}}$が与えられたとする。$m < n$が成り立つなら、$\forall i \in \varLambda_{m}$に対し、$\sum_{j \in \varLambda_{n}} {c_{j}a_{ij}}=0$が成り立つかつ、$\exists j \in \varLambda_{n}$に対し、$c_{j} \neq 0$が成り立つ。\par
このことは以下のようにして示される。
\begin{enumerate}
\item
  $m = 1$のとき、$\forall j \in \varLambda_{n}$に対し、$a_{1j} = 0$が成り立つときとそうでないときで場合分けする。
\item
  $m = k < n - 1$のとき、$\left( a_{ij} \right)_{i \in \varLambda_{k}} \in K^{k}$なる組$\left( a_{ij} \right)_{i \in \varLambda_{k}}$が与えられたら、$\forall i \in \varLambda_{k}$に対し、$\sum_{j \in \varLambda_{n - 1}} {c_{j}a_{ij}}=0$が成り立つかつ、$\exists j \in \varLambda_{n - 1}$に対し、$c_{j} \neq 0$が成り立つと仮定する。
\item
  $m = k + 1 < n$のとき、$\forall i \in \varLambda_{k + 1}\forall j \in \varLambda_{n}$に対し、$a_{ij} = 0$が成り立つなら、明らかである。
\item
  そうでないなら、和$- \frac{a_{i'j}}{a_{ij}}\sum_{j' \in \varLambda_{n}} {c_{j'}a_{ij'}} + \sum_{j' \in \varLambda_{n}} {c_{j'}a_{i'j'}}$を考えることで3. の状況に帰着させる。
\item
  以上より数学的帰納法で示す。
\end{enumerate}
\end{thm}
\begin{proof}
$\left( a_{ij} \right)_{i \in \varLambda_{m}} \in K^{m}$なる組$\left( a_{ij} \right)_{i \in \varLambda_{m}}$が与えられ$m < n$が成り立つとし、$\forall i \in \varLambda_{m}$に対し、$\sum_{j \in \varLambda_{n}} {c_{j}a_{ij}} = 0$が成り立つとする。\par
$m = 1$のとき、$a_{1j} = a_{j}$とおくと、$\forall j \in \varLambda_{n}$に対し、$a_{j} = 0$が成り立つなら、$\forall j \in \varLambda_{n}$に対し、$c_{j} = 1$とすればよいし、$\exists j \in \varLambda_{n}$に対し、$a_{j} \neq 0$が成り立つなら、$\forall j' \in \varLambda_{n} \setminus \left\{ j \right\}$に対し、$c_{j'} = 1$とし、さらに、次のようにすれば、
\begin{align*}
c_{j} = - \frac{1}{a_{j}}\sum_{j' \in \varLambda_{n} \setminus \left\{ j \right\}} a_{j'}
\end{align*}
次のようになる。
\begin{align*}
\sum_{j' \in \varLambda_{n}} {c_{j'}a_{j'}} &=\sum_{j' \in \varLambda_{n} \setminus \left\{ j \right\}} {c_{j'}a_{j'}} + c_{j}a_{j}\\
&=\sum_{j' \in \varLambda_{n} \setminus \left\{ j \right\}} a_{j'} + \left( - \frac{1}{a_{j}}\sum_{j' \in \varLambda_{n} \setminus \left\{ j \right\}} a_{j'} \right)a_{j}\\
&=\sum_{j' \in \varLambda_{n} \setminus \left\{ j \right\}} a_{j'} - \sum_{j' \in \varLambda_{n} \setminus \left\{ j \right\}} a_{j'} = 0
\end{align*}\par
$m = k < n - 1$のとき、$\left( a_{ij} \right)_{i \in \varLambda_{k}} \in K^{k}$なる組$\left( a_{ij} \right)_{i \in \varLambda_{k}}$が与えられたら、$\forall i \in \varLambda_{k}$に対し、$\sum_{j \in \varLambda_{n - 1}} {c_{j}a_{ij}}=0$が成り立つかつ、$\exists j \in \varLambda_{n - 1}$に対し、$c_{j} \neq 0$が成り立つと仮定すると、$m = k + 1 < n$のとき、$\forall i \in \varLambda_{k + 1}\forall j \in \varLambda_{n}$に対し、$a_{ij} = 0$が成り立つなら、$\forall j \in \varLambda_{n}$に対し、$c_{j} = 1$とすればよいので、$\exists i \in \varLambda_{k + 1}\exists j \in \varLambda_{n}$に対し、$a_{ij} \neq 0$が成り立つとする。このとき、$\forall i' \in \varLambda_{k + 1}$に対し、次のようになるので、
\begin{align*}
0 &= - \frac{a_{i'j}}{a_{ij}}\sum_{j' \in \varLambda_{n}} {c_{j'}a_{ij'}} + \sum_{j' \in \varLambda_{n}} {c_{j'}a_{i'j'}}\\
&= \sum_{j' \in \varLambda_{n}} {c_{j'}\left( a_{i'j'} - \frac{a_{i'j}a_{ij'}}{a_{ij}} \right)}\\
&= \sum_{j' \in \varLambda_{n} \setminus \left\{ j \right\}} {c_{j'}\left( a_{i'j'} - \frac{a_{i'j}a_{ij'}}{a_{ij}} \right)} + c_{j'}\left( a_{i'j} - \frac{a_{i'j}a_{ij}}{a_{ij}} \right)\\
&= \sum_{j' \in \varLambda_{n} \setminus \left\{ j \right\}} {c_{j'}\left( a_{i'j'} - \frac{a_{i'j}a_{ij'}}{a_{ij}} \right)} + c_{j'}\left( a_{i'j} - a_{i'j} \right)\\
&= \sum_{j' \in \varLambda_{n} \setminus \left\{ j \right\}} {c_{j'}\left( a_{i'j'} - \frac{a_{i'j}a_{ij'}}{a_{ij}} \right)}
\end{align*}
ここで、$a_{i'j'} - \frac{a_{i'j}a_{ij'}}{a_{ij}} = b_{i'j'}$とすれば、$i' = i$のときは自明なので、$\forall i' \in \varLambda_{k + 1} \setminus \left\{ i \right\}$に対し、次式が成り立つ。
\begin{align*}
0 = \sum_{j' \in \varLambda_{n} \setminus \left\{ j \right\}} {c_{j'}b_{i'j'}}
\end{align*}
ここで、仮定より$\exists j' \in \varLambda_{n} \setminus \left\{ j \right\}$に対し、$c_{j'} \neq 0$が成り立つので、$\exists j' \in \varLambda_{n}$に対し、$c_{j'} \neq 0$が成り立つ。\par
以上より数学的帰納法で示すべきことは示された。
\end{proof}
\begin{thm}\label{2.1.1.18}
体$K$上のvector空間$V$の基底をなすvectorsの個数が有限で、$\forall(i,j) \in \varLambda_{m} \times \varLambda_{n}$に対し、$\mathbf{v}_{i},\mathbf{w}_{j} \in V$なるvectors$\mathbf{v}_{i}$、$\mathbf{w}_{j}$が与えられたとし、それらのvectors$\mathbf{w}_{j}$はいずれもそれらのvectors$\mathbf{v}_{i}$の線形結合であるとする。このとき、$m < n$が成り立つなら、それらのvectors$\mathbf{w}_{j}$は線形従属である。このことは以下のようにして示される。
\begin{enumerate}
\item
  仮定より$\forall j \in \varLambda_{n}$に対し、$\mathbf{w}_{j} = \sum_{i \in \varLambda_{m}} {k_{ij}\mathbf{v}_{i}}$が成り立つ。
\item
  $l_{j} \in K$を用いて$\sum_{j \in \varLambda_{n}} {l_{j}\mathbf{w}_{j}} = \mathbf{0}$を考え、1.
  の式を代入する。
\item
  定理\ref{2.1.1.17}に注意すれば、$\exists j \in \varLambda_{n}$に対し、$l_{j} \neq 0$が成り立つ。
\item
  以上の議論により、それらのvectors$\mathbf{w}_{j}$は線形従属であることが示される。
\end{enumerate}
\end{thm}
\begin{proof}
体$K$上のvector空間$V$の基底をなすvectorsの個数が有限で、$\forall(i,j) \in \varLambda_{m} \times \varLambda_{n}$に対し、$\mathbf{v}_{i},\mathbf{w}_{j} \in V$なるvectors$\mathbf{v}_{i}$、$\mathbf{w}_{j}$が与えられたとし、それらのvectors$\mathbf{w}_{j}$はいずれもそれらのvectors$\mathbf{v}_{i}$の線形結合であるとする。このとき、$k_{ij} \in K$を用いて$\forall j \in \varLambda_{n}$に対し、$\mathbf{w}_{j} = \sum_{i \in \varLambda_{m}} {k_{ij}\mathbf{v}_{i}}$が成り立つことになる。$m < n$が成り立つなら、$\forall j \in \varLambda_{n}$に対し、その体$K$の元$l_{j}$を用いて$\sum_{j \in \varLambda_{n}} {l_{j}\mathbf{w}_{j}} = \mathbf{0}$を考えると、次のようになる。
\begin{align*}
\sum_{j \in \varLambda_{n}} {l_{j}\mathbf{w}_{j}} = \sum_{j \in \varLambda_{n}} {l_{j}\sum_{i \in \varLambda_{m}} {k_{ij}\mathbf{v}_{i}}} = \sum_{i \in \varLambda_{m}} {\sum_{j \in \varLambda_{n}} {k_{ij}l_{j}}\mathbf{v}_{i}} = \mathbf{0}
\end{align*}
ここで、定理\ref{2.1.1.17}より、$\forall i \in \varLambda_{m}$に対し、$\sum_{j \in \varLambda_{n}} {l_{j}k_{ij}}=0$が成り立つかつ、$\exists j \in \varLambda_{n}$に対し、$l_{j} \neq 0$が成り立つ。これにより、$\sum_{j \in \varLambda_{n}} {l_{j}\mathbf{w}_{j}} = \mathbf{0}$が成り立つかつ、$\exists j \in \varLambda_{n}$に対し、$l_{j} \neq 0$が成り立つ。よって、vectors$\mathbf{w}_{j}$は線形従属であることが示された。
\end{proof}
\begin{thm}[有限次元の一意性]\label{2.1.1.19}
$\left\{ \mathbf{0} \right\}$でない体$K$上のvector空間$V$が与えられたとき、vectorsの組々$\left\langle \mathbf{v}_{i} \right\rangle_{i \in \varLambda_{m}}$、$\left\langle \mathbf{w}_{i} \right\rangle_{i \in \varLambda_{n}}$がともにそのvector空間$V$の基底であるなら、$m = n$が成り立つ。\par
このことは以下のようにして示される。
\begin{enumerate}
\item
  基底の定義よりvectors$\mathbf{w}_{j}$はどれもvectors$\mathbf{v}_{i}$の線形結合であるかつ、線形独立であるということに注意する。
\item
  上記の定理\ref{2.1.1.18}の対偶をとることで$m \geq n$を得る。
\item
  vectors$\mathbf{v}_{i}、\mathbf{w}_{j}$を逆にして1. から2. の議論に適用させる。
\item
  2. 、3. より$m = n$が得られる。
\end{enumerate}
\end{thm}
\begin{proof}
$\left\{ \mathbf{0} \right\}$でない体$K$上のvector空間$V$が与えられvectorsの組々$\left\langle \mathbf{v}_{i} \right\rangle_{i \in \varLambda_{m}}$、$\left\langle \mathbf{w}_{i} \right\rangle_{i \in \varLambda_{n}}$がともにそのvector空間$V$の基底であるとき、基底の定義よりそれらのvectors$\mathbf{w}_{j}$はどれも族$\left\{ \mathbf{v}_i \right\}_{i \in \varLambda_{m} } $の線形従属であるかつ、それらのvectors$\mathbf{w}_{j}$は線形独立である。このとき、定理\ref{2.1.1.18}の対偶をとれば、それらのvectors$\mathbf{w}_{j}$は線形独立であるなら、$m \geq n$が成り立つ。vectors$\mathbf{v}_{i}$、$\mathbf{w}_{j}$を逆にしても、同様にして$m \leq n$が得られる。したがって、$m = n$が成り立つ。
\end{proof}
\begin{dfn}
$\left\{ \mathbf{0} \right\}$でない体$K$上のvector空間$V$の基底に含まれるvectorsの個数は上記の定理より一意的であることが分かった。この個数をvector空間$V$の体$K$上の次元といい、$\dim_{K}V$、または単に$\dim V$と書く。体$K$上のvector空間$\left\{ \mathbf{0} \right\}$の次元$\dim\left\{ \mathbf{0} \right\}$は、基底が空集合$\emptyset $であると考え、$0$とする。次元が有限ならば、そのvector空間は有限次元であるといい、次元が無限ならば、そのvector空間は無限次元であるという。より詳しくいえば、$\dim V = n$なるvector空間を$n$次元vector空間という。
\end{dfn}
\begin{thm}\label{2.1.1.20}
体$K$上の$n$次元vector空間$V$が与えられたとき、そのvector空間$V$の族$\left\{ \mathbf{v}_{i} \right\}_{i \in \varLambda_{r}} $が線形独立であるとする。$r = n$なら、これらのvectors$\mathbf{v}_{i}$はこのvector空間$V$の基底であり、$r < n$なら、これらのvectors$\mathbf{v}_{i}$に適切な$n - r$つの元を加えることでこのvector空間$V$の基底をつくることができる。
\end{thm}\par
これは$r = n$の場合、背理法で示され、$r < n$の場合、定義に従って示される。
\begin{proof}
体$K$上の$n$次元vector空間$V$が与えられたとき、そのvector空間$V$の族$\left\{ \mathbf{v}_{i} \right\}_{i \in \varLambda_{r}} $が線形独立であるとし、vectors$\mathbf{v}_{i}$によって生成されるvector空間$V$の部分空間を$W$とおく。\par
まず、$r = n$のときを考えよう。$V \neq W$と仮定すると、$\mathbf{v} \in V \setminus W$なるvector$\mathbf{v}$が存在して、その部分空間$W$はvector$\mathbf{v}$を生成することができないことになる。ここで、$\sum_{i \in \varLambda_{r}} {c_{i}\mathbf{v}_{i}} + c_{r + 1}\mathbf{v} = \mathbf{0} \Rightarrow$が成り立つかつ、$\exists i \in \varLambda_{r + 1}$に対し、$c_{i} \neq 0$が成り立つと仮定すると、$c_{r + 1} = 0$のとき、$\sum_{i \in \varLambda_{r}} {c_{i}\mathbf{v}_{i}} = \mathbf{0}$となるような$c_{i} = 0$でない添数$i$が存在することになりこれは仮定に矛盾するので、$c_{r + 1} \neq 0$が得られる。このとき、$\sum_{i \in \varLambda_{r}} {c_{i}\mathbf{v}_{i}} + c_{r + 1}\mathbf{v} = \mathbf{0}$が成り立つならそのときに限り、$\mathbf{v} = \sum_{i \in \varLambda_{r}} \left( - \frac{c_{i}}{c_{r + 1}}\mathbf{v}_{i} \right)$が成り立ち、これはその部分空間$W$がvector$\mathbf{v}$を生成することができないことに矛盾する。以上より、$\sum_{i \in \varLambda_{r}} {c_{i}\mathbf{v}_{i}} + c_{r + 1}\mathbf{v} = \mathbf{0}$が成り立つなら、$\forall i \in \varLambda_{r + 1}$に対し、$c_{i} = 0$が成り立つ。このとき、族$\left\{ \mathbf{v}_{i} \right\}_{i \in \varLambda_{r}} $が$n + 1$つ存在し、これは仮定に矛盾する。よって、$V = W$が成り立ちその族$\left\{ \mathbf{v}_{i} \right\}_{i \in \varLambda_{r}} $はそのvector空間$V$の基底である。\par
次に、$r < n$のときを考えよう。その族$\left\{ \mathbf{v}_{i} \right\}_{i \in \varLambda_{r}} $では、そのvector空間の任意の元が線形結合されるとは限らなく、その族$\left\{ \mathbf{v}_{i} \right\}_{i \in \varLambda_{r}} $はそのvector空間$V$の基底でないので、$V \neq W$が成り立つ。そこで、$\mathbf{v}_{r + 1} \in V \setminus W$なるvector$\mathbf{v}_{r + 1}$をとれば、$i \in \varLambda_{r + 1}$に対しvectors$\mathbf{v}_{i}$は線形独立である。その部分空間${\mathrm{span}}\left\{ \mathbf{v}_{i} \right\}_{i \in \varLambda_{r + 1}}$に対しても同じ議論を繰り返すことで、その族$\left\{ \mathbf{v}_{i} \right\}_{i \in \varLambda_{r}} $に適切な$n - r$つの元を加えることでこのvector空間$V$の基底をつくることができる。
\end{proof}
\begin{thm}\label{2.1.1.21}
体$K$上の$n$次元vector空間$V$が与えられこれの基底$\left\langle \mathbf{v}_{i} \right\rangle_{i \in \varLambda_{n}}$が与えられているとき、そのvector空間$V$は族$\left\{ \mathbf{v}_{i} \right\}_{i \in \varLambda_{n}} $によって張られているので、$\forall\mathbf{v} \in V$に対し、その体$K$のある元々$k_{i}$を用いて$\mathbf{v} = \sum_{i \in \varLambda_{n}} {k_{i}\mathbf{v}_{i}}$のように表されるのであった。このときの組$\left( k_{i} \right)_{i \in \varLambda_{n}}$は一意的である。
\end{thm}\par
これは背理法によって示される。
\begin{dfn}
体$K$上の$n$次元vector空間$V$が与えられこれの基底$\left\langle \mathbf{v}_{i} \right\rangle_{i \in \varLambda_{n}}$が与えられているとき、$\forall\mathbf{v} \in V$に対し、$\mathbf{v} = \sum_{i \in \varLambda_{n}} {k_{i}\mathbf{v}_{i}}$なる組$\left( k_{i} \right)_{i \in \varLambda_{i}}$をその基底$\left\langle \mathbf{v}_{i} \right\rangle_{i \in \varLambda_{n}}$でのそのvector$\mathbf{v}$の成分、座標といい、その元$k_{i}$をそのvector$\mathbf{v}$のその基底$\left\langle \mathbf{v}_{i} \right\rangle_{i \in \varLambda_{n}}$での第$i$成分、第$i$座標という。
\end{dfn}
\begin{proof}
体$K$上の$n$次元vector空間$V$が与えられこれの基底$\left\langle \mathbf{v}_{i} \right\rangle_{i \in \varLambda_{n}}$が与えられているとき、そのvector空間$V$は族$\left\{ \mathbf{v}_{i} \right\}_{i \in \varLambda_{n}} $によって張られているので、$\forall\mathbf{v} \in V$に対し、その体$K$のある元々$k_{i}$を用いて$\mathbf{v} = \sum_{i \in \varLambda_{n}} {k_{i}\mathbf{v}_{i}}$のように表されるのであった。このときの組$\left( k_{i} \right)_{i \in \varLambda_{n}}$のほかに異なる組$\left( l_{i} \right)_{i \in \varLambda_{n}}$が存在して、$\mathbf{v} = \sum_{i \in \varLambda_{n}} {l_{i}\mathbf{v}_{i}}$が成り立つなら、$\exists i \in \varLambda_{n}$に対し、$k_{i} \neq l_{i}$が成り立つ、即ち、$k_{i} - l_{i} \neq 0$が成り立つかつ、次のようになる。
\begin{align*}
\mathbf{0} = \mathbf{v} - \mathbf{v} = \sum_{i \in \varLambda_{n}} {k_{i}\mathbf{v}_{i}} - \sum_{i \in \varLambda_{n}} {l_{i}\mathbf{v}_{i}} = \sum_{i \in \varLambda_{n}} {\left( k_{i} - l_{i} \right)\mathbf{v}_{i}}
\end{align*}
ゆえに、その族$\left\{ \mathbf{v}_{i} \right\}_{i \in \varLambda_{n}} $は線形従属であるが、これはその族$\left\{ \mathbf{v}_{i} \right\}_{i \in \varLambda_{n}} $がそのvector空間$V$の基底をなすので、その族$\left\{ \mathbf{v}_{i} \right\}_{i \in \varLambda_{n}} $は線形独立であることに矛盾する。
\end{proof}
\begin{thm}\label{2.1.1.22}
体$K$上の$n$次元vector空間$V$の$r$次元部分空間$W$が与えられたとき、次のことが成り立つ。
\begin{itemize}
\item
  $\dim W \leq \dim V$が成り立ちその部分空間$W$の次元が有限である。
\item
  $\dim W = \dim V$が成り立つならそのときに限り、$W = V$が成り立つ。
\item
  その部分空間$W$の任意の基底$\left\langle \mathbf{w}_{i} \right\rangle_{i \in \varLambda_{r}}$に対し、これを拡大したそのvector空間$V$の基底$\left\langle \mathbf{w}_{i} \right\rangle_{i \in \varLambda_{n}}$が存在する。
\end{itemize}
\end{thm}
\begin{proof}
体$K$上の$n$次元vector空間$V$の$r$次元部分空間$W$が与えられたとき、その部分空間$W$の基底$\left\langle \mathbf{w}_{i} \right\rangle_{i \in \varLambda_{r}}$について、$\dim W > \dim V$が成り立つなら、その部分空間$W$の基底$\left\langle \mathbf{w}_{i} \right\rangle_{i \in \varLambda_{r}}$をなすvectors$\mathbf{w}_{i}$のうち$\mathbf{w}_{i} \in W$かつ$\mathbf{w}_{i} \notin V$が成り立つようなものが存在する。これのうち1つを$\mathbf{w}'$とおくと、$\mathbf{w}' \notin V$が成り立つので、$W \subseteq V$が成り立たないことになるが、その集合$W$はそのvector空間の部分集合であることに矛盾する。よって、$\dim W \leq \dim V$が成り立ちその部分空間$W$の次元は有限である。\par
その部分空間$W$の基底$\left\langle \mathbf{w}_{i} \right\rangle_{i \in \varLambda_{r}}$をなすvectors$\mathbf{w}_{i}$はvector空間$V$の線形独立な元でもある。$\dim W = \dim V$が成り立つなら、これらのvectors$\mathbf{w}_{i}$はこのvector空間$V$の基底であり、$\dim W < \dim V$が成り立つなら、これらのvectors$\mathbf{w}_{i}$に適切な$n - r$つの元を加えることでこのvector空間$V$の基底をつくることができるのであった。これにより、$\dim W = \dim V$が成り立つならそのときに限り、$W = V$が成り立ち、$\dim W < \dim V$が成り立つなら、その部分空間$W$の任意の基底$\left\langle \mathbf{w}_{i} \right\rangle_{i \in \varLambda_{r}}$に対し、これを拡大したそのvector空間$V$の基底$\left\langle \mathbf{w}_{i} \right\rangle_{i \in \varLambda_{n}}$が存在する。
\end{proof}
%\hypertarget{ux57faux5e95ux306eux53d6ux308aux304bux3048}{%
\subsubsection{基底の取りかえ}%\label{ux57faux5e95ux306eux53d6ux308aux304bux3048}}
\begin{thm}\label{2.1.1.23}
体$K$上の$n$次元のvector空間$V$が与えられたとき、これの基底を$\left\langle \mathbf{v}_{i} \right\rangle_{i \in \varLambda_{n}}$とおく。そのvector空間はそれらのvectors$\mathbf{v}_{i}$によって生成されるので、$\forall\mathbf{w} \in V$に対し、そのvector$\mathbf{w}$は$\mathbf{w} = \sum_{i \in \varLambda_{n}} {k_{i}\mathbf{v}_{i}}$とおくことができる。ここで、$\exists i' \in \varLambda_{n}$に対し、$k_{i'} \neq 0$が成り立てば、そのvector$\mathbf{v}_{i'}$をそのvector$\mathbf{w}$に置きかえた組$\left\langle \left\{ \begin{matrix}
\mathbf{v}_{i} & \mathrm{if}  & i \neq i' \\
\mathbf{w} & \mathrm{if}  & i = i' \\
\end{matrix} \right.\  \right\rangle_{i \in \varLambda_{n}}$もそのvector空間$V$の基底である。
\end{thm}
\begin{proof}
体$K$上の$n$次元のvector空間$V$が与えられたとき、これの基底を$\left\langle \mathbf{v}_{i} \right\rangle_{i \in \varLambda_{n}}$とおく。そのvector空間はそれらのvectors$\mathbf{v}_{i}$によって生成されるので、$\forall\mathbf{w} \in V$に対し、そのvector$\mathbf{w}$は$\mathbf{w} = \sum_{i \in \varLambda_{n}} {k_{i}\mathbf{v}_{i}}$とおくことができるのであった。ここで、$\exists i' \in \varLambda_{n}$に対し、$k_{i'} \neq 0$が成り立てば、vector$\mathbf{v}_{i'}$をvector$\mathbf{w}$で置きかえた組$\left\langle \left\{ \begin{matrix}
\mathbf{v}_{i} & \mathrm{if}  & i \neq i' \\
\mathbf{w} & \mathrm{if}  & i = i' \\
\end{matrix} \right.\  \right\rangle_{i \in \varLambda_{n}}$について、次のことから、
\begin{align*}
\mathbf{w} = \sum_{i \in \varLambda_{n}} {k_{i}\mathbf{v}_{i}} = \sum_{i \in \varLambda_{n} \setminus \left\{ i' \right\}} {k_{i}\mathbf{v}_{i}} + k_{i'}\mathbf{v}_{i'} &\Leftrightarrow k_{i'}\mathbf{v}_{i'} = \mathbf{w} - \sum_{i \in \varLambda_{n} \setminus \left\{ i' \right\}} {k_{i}\mathbf{v}_{i}}\\
&\Leftrightarrow \mathbf{v}_{i'} = \frac{1}{k_{i'}}\mathbf{w} - \sum_{i \in \varLambda_{n} \setminus \left\{ i' \right\}} {\frac{k_{i}}{k_{i'}}\mathbf{v}_{i}}
\end{align*}
その組$\left\langle \left\{ \begin{matrix}
\mathbf{v}_{i} & \mathrm{if}  & i \neq i' \\
\mathbf{w} & \mathrm{if}  & i = i' \\
\end{matrix} \right.\  \right\rangle_{i \in \varLambda_{n}}$からそのvector空間$V$が生成される。\par
さらに、次のようになり、
\begin{align*}
\sum_{i \in \varLambda_{n}} {c_{i}\mathbf{v}_{i}} = \mathbf{0} &\Leftrightarrow \sum_{i \in \varLambda_{n} \setminus \left\{ i' \right\}} {c_{i}\mathbf{v}_{i}} + c_{i'}\mathbf{v}_{i'} = \mathbf{0}\\
&\Leftrightarrow \sum_{i \in \varLambda_{n} \setminus \left\{ i' \right\}} {c_{i}\mathbf{v}_{i}} + c_{i'}\left( \frac{1}{k_{i'}}\mathbf{w} - \sum_{i \in \varLambda_{n} \setminus \left\{ i' \right\}} {\frac{k_{i}}{k_{i'}}\mathbf{v}_{i}} \right) = \mathbf{0}\\
&\Leftrightarrow \sum_{i \in \varLambda_{n} \setminus \left\{ i' \right\}} {c_{i}\mathbf{v}_{i}} + \frac{c_{i'}}{k_{i'}}\mathbf{w} - \sum_{i \in \varLambda_{n} \setminus \left\{ i' \right\}} {\frac{c_{i'}k_{i}}{k_{i'}}\mathbf{v}_{i}} = \mathbf{0}\\
&\Leftrightarrow \sum_{i \in \varLambda_{n} \setminus \left\{ i' \right\}} {\left( c_{i} - \frac{c_{i'}k_{i}}{k_{i'}} \right)\mathbf{v}_{i}} + \frac{c_{i'}}{k_{i'}}\mathbf{w} = \mathbf{0}
\end{align*}
ここで、その組$\left\langle \mathbf{v}_{i} \right\rangle_{i \in \varLambda_{n}}$は線形独立なので、$\forall i \in \varLambda_{n}$に対し、$c_{i} = 0$が成り立つ。これにより、$\forall i \in \varLambda_{n} \setminus \left\{ i' \right\}$に対し、$c_{i} - \frac{c_{i'}k_{i}}{k_{i'}} = 0$が成り立つかつ、$\frac{c_{i'}}{k_{i'}} = 0$が成り立つので、その組$\left\langle \left\{ \begin{matrix}
\mathbf{v}_{i} & \mathrm{if}  & i \neq i' \\
\mathbf{w} & \mathrm{if}  & i = i' \\
\end{matrix} \right.\  \right\rangle_{i \in \varLambda_{n}}$も線形独立である。\par
よって、そのvector$\mathbf{v}_{i'}$をそのvector$\mathbf{w}$に置きかえた組$\left\langle \left\{ \begin{matrix}
\mathbf{v}_{i} & \mathrm{if}  & i \neq i' \\
\mathbf{w} & \mathrm{if}  & i = i' \\
\end{matrix} \right.\  \right\rangle_{i \in \varLambda_{n}}$もそのvector空間$V$の基底である。
\end{proof}
\begin{thm}\label{2.1.1.24}
体$K$上の$n$次元のvector空間$V$が与えられたとき、これの基底を$\left\langle \mathbf{v}_{i} \right\rangle_{i \in \varLambda_{n}}$とおく。そのvector空間はそれらのvectors$\mathbf{v}_{i}$によって生成されるので、$\forall\mathbf{w} \in V$に対し、そのvector$\mathbf{w}$は$\mathbf{w} = \sum_{i \in \varLambda_{n}} {k_{i}\mathbf{v}_{i}}$とおくことができる。ここで、$\forall\mathbf{v} \in V$に対し、その基底$\left\langle \mathbf{v}_{i} \right\rangle_{i \in \varLambda_{n}}$でのこれの座標が$\left( l_{i} \right)_{i \in \varLambda_{n}}$と与えられたとすると、その基底$\left\langle \left\{ \begin{matrix}
\mathbf{v}_{i} & \mathrm{if}  & i \neq i' \\
\mathbf{w} & \mathrm{if}  & i = i' \\
\end{matrix} \right.\  \right\rangle_{i \in \varLambda_{n}}$でのそのvector$\mathbf{v}$の座標は$\left( \left\{ \begin{matrix}
l_{i} - \frac{k_{i}l_{i'}}{k_{i'}} & \mathrm{if}  & i \neq i' \\
\frac{l_{i'}}{k_{i'}} & \mathrm{if}  & i = i' \\
\end{matrix} \right.\  \right)_{i \in \varLambda_{n}}$となる。
\end{thm}
\begin{proof}
体$K$上の$n$次元のvector空間$V$が与えられたとき、これの基底を$\left\langle \mathbf{v}_{i} \right\rangle_{i \in \varLambda_{n}}$とおく。そのvector空間はそれらのvectors$\mathbf{v}_{i}$によって生成されるので、$\forall\mathbf{w} \in V$に対し、そのvector$\mathbf{w}$は$\mathbf{w} = \sum_{i \in \varLambda_{n}} {k_{i}\mathbf{v}_{i}}$とおくことができる。ここで、$\forall\mathbf{v} \in V$に対し、その基底$\left\langle \mathbf{v}_{i} \right\rangle_{i \in \varLambda_{n}}$でのこれの座標が$\left( l_{i} \right)_{i \in \varLambda_{n}}$と与えられたとすると、定義より次式が成り立つ。
\begin{align*}
\mathbf{v} = \sum_{i \in \varLambda_{n}} {l_{i}\mathbf{v}_{i}} = \sum_{i \in \varLambda_{n} \setminus \left\{ i' \right\}} {l_{i}\mathbf{v}_{i}} + l_{i'}\mathbf{v}_{i'}
\end{align*}
ここで、そのvector$\mathbf{v}_{i'}$は$\frac{1}{k_{i'}}\mathbf{w} - \sum_{i \in \varLambda_{n} \setminus \left\{ i' \right\}} {\frac{k_{i}}{k_{i'}}\mathbf{v}_{i}}$と書きかえられることができたので、次のようになる。
\begin{align*}
\mathbf{v} &= \sum_{i \in \varLambda_{n} \setminus \left\{ i' \right\}} {l_{i}\mathbf{v}_{i}} + l_{i'}\left( \frac{1}{k_{i'}}\mathbf{w} - \sum_{i \in \varLambda_{n} \setminus \left\{ i' \right\}} {\frac{k_{i}}{k_{i'}}\mathbf{v}_{i}} \right)\\
&= \sum_{i \in \varLambda_{n} \setminus \left\{ i' \right\}} {l_{i}\mathbf{v}_{i}} + \frac{l_{i'}}{k_{i'}}\mathbf{w}\mathbf{+}\sum_{i \in \varLambda_{n} \setminus \left\{ i' \right\}} {\left( - \frac{k_{i}l_{i'}}{k_{i'}} \right)\mathbf{v}_{i}}\\
&= \frac{l_{i'}}{k_{i'}}\mathbf{w}\mathbf{+}\sum_{i \in \varLambda_{n} \setminus \left\{ i' \right\}} {\left( l_{i} - \frac{k_{i}l_{i'}}{k_{i'}} \right)\mathbf{v}_{i}}
\end{align*}
以上の議論により、よって、その基底$\left\langle \left\{ \begin{matrix}
\mathbf{v}_{i} & \mathrm{if}  & i \neq i' \\
\mathbf{w} & \mathrm{if}  & i = i' \\
\end{matrix} \right.\  \right\rangle_{i \in \varLambda_{n}}$でのそのvector$\mathbf{v}$の座標は$\left( \left\{ \begin{matrix}
l_{i} - \frac{k_{i}l_{i'}}{k_{i'}} & \mathrm{if}  & i \neq i' \\
\frac{l_{i'}}{k_{i'}} & \mathrm{if}  & i = i' \\
\end{matrix} \right.\  \right)_{i \in \varLambda_{n}}$となる。
\end{proof}
\begin{thebibliography}{50}
\bibitem{1}
  松坂和夫, 集合・位相入門, 岩波書店, 1968. 新装版第2刷 p111,132-136 ISBN978-4-00-029871-1
\bibitem{2}
  松坂和夫, 線型代数入門, 岩波書店, 1980. 新装版第2刷 p41-73 ISBN978-4-00-029872-8
\bibitem{3}
  松坂和夫, 代数系入門, 岩波書店, 1976. 新装版第2刷 p45-51,107-112,170-199 ISBN978-4-00-029873-5
\bibitem{4}
  中西敏浩. "数学基礎 IV (線形代数学) 講義ノート". 島根大学. \url{http://www.math.shimane-u.ac.jp/~tosihiro/skks4main.pdf} (2020-9-7 取得)
\end{thebibliography}
\end{document}
