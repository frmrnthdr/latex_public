\documentclass[dvipdfmx]{jsarticle}
\setcounter{section}{3}
\setcounter{subsection}{2}
\usepackage{xr}
\externaldocument{8.3.1}
\externaldocument{8.3.2}
\usepackage{amsmath,amsfonts,amssymb,array,comment,mathtools,url,docmute}
\usepackage{longtable,booktabs,dcolumn,tabularx,mathtools,multirow,colortbl,xcolor}
\usepackage[dvipdfmx]{graphics}
\usepackage{bmpsize}
\usepackage{amsthm}
\usepackage{enumitem}
\setlistdepth{20}
\renewlist{itemize}{itemize}{20}
\setlist[itemize]{label=•}
\renewlist{enumerate}{enumerate}{20}
\setlist[enumerate]{label=\arabic*.}
\setcounter{MaxMatrixCols}{20}
\setcounter{tocdepth}{3}
\newcommand{\rotin}{\text{\rotatebox[origin=c]{90}{$\in $}}}
\newcommand{\amap}[6]{\text{\raisebox{-0.7cm}{\begin{tikzpicture} 
  \node (a) at (0, 1) {$\textstyle{#2}$};
  \node (b) at (#6, 1) {$\textstyle{#3}$};
  \node (c) at (0, 0) {$\textstyle{#4}$};
  \node (d) at (#6, 0) {$\textstyle{#5}$};
  \node (x) at (0, 0.5) {$\rotin $};
  \node (x) at (#6, 0.5) {$\rotin $};
  \draw[->] (a) to node[xshift=0pt, yshift=7pt] {$\textstyle{\scriptstyle{#1}}$} (b);
  \draw[|->] (c) to node[xshift=0pt, yshift=7pt] {$\textstyle{\scriptstyle{#1}}$} (d);
\end{tikzpicture}}}}
\newcommand{\twomaps}[9]{\text{\raisebox{-0.7cm}{\begin{tikzpicture} 
  \node (a) at (0, 1) {$\textstyle{#3}$};
  \node (b) at (#9, 1) {$\textstyle{#4}$};
  \node (c) at (#9+#9, 1) {$\textstyle{#5}$};
  \node (d) at (0, 0) {$\textstyle{#6}$};
  \node (e) at (#9, 0) {$\textstyle{#7}$};
  \node (f) at (#9+#9, 0) {$\textstyle{#8}$};
  \node (x) at (0, 0.5) {$\rotin $};
  \node (x) at (#9, 0.5) {$\rotin $};
  \node (x) at (#9+#9, 0.5) {$\rotin $};
  \draw[->] (a) to node[xshift=0pt, yshift=7pt] {$\textstyle{\scriptstyle{#1}}$} (b);
  \draw[|->] (d) to node[xshift=0pt, yshift=7pt] {$\textstyle{\scriptstyle{#2}}$} (e);
  \draw[->] (b) to node[xshift=0pt, yshift=7pt] {$\textstyle{\scriptstyle{#1}}$} (c);
  \draw[|->] (e) to node[xshift=0pt, yshift=7pt] {$\textstyle{\scriptstyle{#2}}$} (f);
\end{tikzpicture}}}}
\renewcommand{\thesection}{第\arabic{section}部}
\renewcommand{\thesubsection}{\arabic{section}.\arabic{subsection}}
\renewcommand{\thesubsubsection}{\arabic{section}.\arabic{subsection}.\arabic{subsubsection}}
\everymath{\displaystyle}
\allowdisplaybreaks[4]
\usepackage{vtable}
\theoremstyle{definition}
\newtheorem{thm}{定理}[subsection]
\newtheorem*{thm*}{定理}
\newtheorem{dfn}{定義}[subsection]
\newtheorem*{dfn*}{定義}
\newtheorem{axs}[dfn]{公理}
\newtheorem*{axs*}{公理}
\renewcommand{\headfont}{\bfseries}
\makeatletter
  \renewcommand{\section}{%
    \@startsection{section}{1}{\z@}%
    {\Cvs}{\Cvs}%
    {\normalfont\huge\headfont\raggedright}}
\makeatother
\makeatletter
  \renewcommand{\subsection}{%
    \@startsection{subsection}{2}{\z@}%
    {0.5\Cvs}{0.5\Cvs}%
    {\normalfont\LARGE\headfont\raggedright}}
\makeatother
\makeatletter
  \renewcommand{\subsubsection}{%
    \@startsection{subsubsection}{3}{\z@}%
    {0.4\Cvs}{0.4\Cvs}%
    {\normalfont\Large\headfont\raggedright}}
\makeatother
\makeatletter
\renewenvironment{proof}[1][\proofname]{\par
  \pushQED{\qed}%
  \normalfont \topsep6\p@\@plus6\p@\relax
  \trivlist
  \item\relax
  {
  #1\@addpunct{.}}\hspace\labelsep\ignorespaces
}{%
  \popQED\endtrivlist\@endpefalse
}
\makeatother
\renewcommand{\proofname}{\textbf{証明}}
\usepackage{tikz,graphics}
\usepackage[dvipdfmx]{hyperref}
\usepackage{pxjahyper}
\hypersetup{
 setpagesize=false,
 bookmarks=true,
 bookmarksdepth=tocdepth,
 bookmarksnumbered=true,
 colorlinks=false,
 pdftitle={},
 pdfsubject={},
 pdfauthor={},
 pdfkeywords={}}
\begin{document}
\subsection{接vector空間}
\subsubsection{$p$-同値}
\begin{dfn}
  $n$次元多様体$\left(\mathcal{M},\mathfrak{O}\right)$の台集合$\mathcal{M}$の元$p$が与えられたとき、その元$p$の近傍$V$を用いた$C^{\infty} $級写像$f:V\rightarrow \mathbb{R} $全体の集合$C^\infty_p \left( V\right)$について、その和集合$\bigsqcup_{V\in \mathbf{V}} C^\infty_p \left( V\right)$を$C^\infty_p \mathcal{M}$と書き、さらに、その多様体$\left(\mathcal{M},\mathfrak{O}\right)$上の$C^\infty $級関数全体、即ち、和集合$\bigsqcup_{U\in \mathfrak{O} } C^\infty \left(U\right)$を$C^\infty \mathcal{M}$と書く。
\end{dfn}
\begin{thm}\label{8.3.3.1}
  $n$次元多様体$\left(\mathcal{M},\mathfrak{O}\right)$の台集合$\mathcal{M}$の元$p$が与えられたとき、$\forall a,b\in \mathbb{R} \forall f,g\in C^\infty_p \mathcal{M}$に対し、$af+bg,fg\in C^\infty_p \mathcal{M} $が成り立つ。
\end{thm}
\begin{proof}
  $n$次元多様体$\left(\mathcal{M},\mathfrak{O}\right)$の台集合$\mathcal{M}$の元$p$が与えられたとき、$\forall a,b\in \mathbb{R} \forall f,g\in C^\infty_p \mathcal{M}$に対し、それらの写像たち$f$、$g$の定義域をそれぞれ$V$、$W$とすれば、$V\cap W \in \mathbf{V} \left( p\right) $より適切に座標近傍系$\left(U,\psi\right)$がとられれば、次の関数たちは$C^\infty $級関数であり
  \begin{align*}
    f\circ \psi^{-1} |V\left(\psi | U\cap V\cap W \right) &: V\left(\psi | U\cap V\cap W \right) \rightarrow \mathbb{R}, \\
    g\circ \psi^{-1} |V\left(\psi | U\cap V\cap W \right) &: V\left(\psi | U\cap V\cap W \right) \rightarrow \mathbb{R} 
  \end{align*}
  次の関数たちも$C^\infty $級関数である。
  \begin{align*}
    af\circ \psi^{-1} +bg\circ \psi^{-1} |V\left(\psi | U\cap V\cap W \right) &: V\left(\psi | U\cap V\cap W \right) \rightarrow \mathbb{R};\psi\left(p\right) \mapsto af\left(p\right) +bg\left(p\right), \\
    \left( f\circ \psi^{-1} \right) \left( g\circ \psi^{-1} \right) |V\left(\psi | U\cap V\cap W \right) &: V\left(\psi | U\cap V\cap W \right) \rightarrow \mathbb{R} ;\psi\left(p\right) \mapsto f\left(p\right) g\left(p\right) 
  \end{align*}
  ここで、次のことが成り立つことに注意すれば、
  \begin{align*}
    af\circ \psi^{-1} +bg\circ \psi^{-1} |V\left(\psi | U\cap V\cap W \right) &= \left( af+bg \right) \circ \psi^{-1} |V\left(\psi | U\cap V\cap W \right) ,\\
    \left( f\circ \psi^{-1} \right) \left( g\circ \psi^{-1} \right) |V\left(\psi | U\cap V\cap W \right) &= fg\circ \psi^{-1} |V\left(\psi | U\cap V\cap W \right) 
  \end{align*}
  よって、これらの写像たち$af+bg$、$fg$はその元$p$の近傍で定義される写像で$C^\infty $級であり$af+bg,fg\in C^\infty_p \mathcal{M} $が成り立つ。
\end{proof}
\begin{thm}\label{8.3.3.2}
  $n$次元多様体$\left(\mathcal{M},\mathfrak{O}\right)$の台集合$\mathcal{M}$の元$p$が与えられたとき、その集合$C^\infty_p \mathcal{M} $はvector空間になりえない\footnote{なお、終集合が$\mathbb{R}$かこれの部分集合となっているような2つの写像たち$f$、$g$の実数たち$a$、$b$を用いた写像$af+bg$を次のように定義するものとしています。
  \begin{align*}
    af+bg:D\left(f\right)\cap D\left(g\right)\rightarrow \mathbb{R} ;q\mapsto af\left(q\right) +bg\left(q\right)
  \end{align*}}。
\end{thm}
\begin{proof}
  $n$次元多様体$\left(\mathcal{M},\mathfrak{O}\right)$の台集合$\mathcal{M}$の元$p$が与えられたとき、その集合$C^\infty_p \mathcal{M} $がvector空間であると仮定しよう。vector空間の公理より零vector$\mathbf{0} $が存在する。このとき、$\forall f\in C^\infty_p \mathcal{M} $に対し、$f+\mathbf{0}=f$が成り立つ。そこで、その集合$C^\infty_p \mathcal{M} $の元の定義域の和集合を$\mathcal{U}$としその写像$\mathbf{0}$の定義域がその集合$\mathcal{U}$でないとすれば、その定義域を$D\left(\mathbf{0}\right)$として$D\left(\mathbf{0}\right)\subset \mathcal{U} $が成り立つ。このとき、その集合$\mathcal{U}$はその元$p$の近傍であるので、$q\in \mathcal{U} \setminus D\left(\mathbf{0}\right)$とすれば、その集合$\mathcal{U}$の定義よりその元$q$が属するような定義域$V$をもつその集合$C^\infty_p \mathcal{M} $の元$g$が存在する。このとき、$V\cap D\left(\mathbf{0}\right) \ne V$なので、$g+\mathbf{0}=g$が成り立たない。ゆえに、その写像$\mathbf{0}$の定義域はその集合$\mathcal{U}$である。\par
  次に、その関数$f$のその可換群$\left(C^\infty_p \mathcal{M},+\right) $における逆元を$-f$とおくと、$f+\left(-f\right)=\mathbf{0}$が成り立つので、その関数$f+\left(-f\right)$の定義域はその集合$\mathcal{U}$である。ゆえに、その関数$f$の定義域は$\mathcal{U}$となる。\par
  ここで、ある座標近傍$\left(U ,\psi \right)$を用いれば、その集合$V\left(\psi\right) $は$n$次元Euclid空間における位相空間での開集合であり異なる2点$\psi \left(q\right)$、$\psi\left(r\right)$がとられることができる。このとき、その写像$\psi $は同相写像なので、$q\ne r$が成り立つ。そこで、その多様体$\left(\mathcal{M},\mathfrak{O}\right)$がHaudorff空間であることから、それらの元々$p$、$q$、$r$の互いに素な近傍たち$V_p $、$V_q $、$V_r $がとられることができる。このとき、それらの関数たち$f|V_p \sqcup V_q$、$f|V_p \sqcup V_r$もその集合$C^\infty_p \mathcal{M}$の元々である。しかしながら、それらの関数たちの定義域は$\mathcal{U}$でありえない。このことは上記の議論に矛盾する。よって、その集合$C^\infty_p \mathcal{M} $はvector空間になりえない。
\end{proof}
\begin{dfn}
  $n$次元多様体$\left(\mathcal{M},\mathfrak{O}\right)$の台集合$\mathcal{M}$の元$p$が与えられたとする。その集合$C_p\infty \mathcal{M}$の元々$f$、$g$において、$\exists V\in \mathbf{V}\left(p\right)$に対し、$f|V=g|V$が成り立つとき、それらの写像たち$f$、$g$は$p$-同値であるといい、$f\sim_p g$と書く。
\end{dfn}
\begin{thm}\label{8.3.3.3}
  $n$次元多様体$\left(\mathcal{M},\mathfrak{O}\right)$の台集合$\mathcal{M}$の元$p$が与えられたとき、その関係$\sim_p $は同値関係となる。
\end{thm}
\begin{proof}
  $n$次元多様体$\left(\mathcal{M},\mathfrak{O}\right)$の台集合$\mathcal{M}$の元$p$が与えられたとき、$\forall f\in C^\infty_p \mathcal{M}$に対し、その写像$f$の定義域を$V$とすれば、もちろん$V\in \mathbf{V}\left(p\right)$で$f|V=f|V$が成り立つ。$\forall f,g\in C^\infty_p \mathcal{M}$に対し、$f\sim_p g$が成り立つなら、$\exists V\in \mathbf{V}\left(p\right)$に対し、$f|V=g|V$が成り立つので、$g\sim_p f$が成り立つ。最後に、$\forall f,g,h\in C^\infty_p \mathcal{M}$に対し、$f\sim_p g$かつ$g\sim_p f$が成り立つなら、$\exists V,W\in \mathbf{V}\left(p\right)$に対し、$f|V=g|V$かつ$g|W=h|W$が成り立つ。そこで、$V\cap W\in \mathbf{V}\left(p\right)$で$f|V\cap W=g|V\cap W$かつ$g|V\cap W=h|V\cap W$が成り立っているので、$f|V\cap W=h|V\cap W$が成り立つ。これにより、$f\sim_p h$が得られた。
\end{proof}
\begin{thm}\label{8.3.3.4}
  $n$次元多様体$\left(\mathcal{M},\mathfrak{O}\right)$の台集合$\mathcal{M}$の元$p$が与えられたとき、次のように和とscalar倍が定義されれば、
  \begin{align*}
    +&:C^\infty_p \mathcal{M} /\sim_p \times C^\infty_p \mathcal{M} /\sim_p \rightarrow C^\infty_p \mathcal{M} /\sim_p ; \left( C_{\sim_p} \left(f\right),C_{\sim_p} \left(g\right)\right) \mapsto C_{\sim_p} \left(f+g\right) \\
    \cdot &: \mathbb{R} \times C^\infty_p \mathcal{M} /\sim_p \rightarrow C^\infty_p \mathcal{M} /\sim_p ; \left( a,C_{\sim_p } \left(f\right) \right) \mapsto C_{\sim_p} \left(af\right)
  \end{align*}
  その商集合$C^\infty_p \mathcal{M} /\sim_p $は体$\mathbb{R}$上のvector空間をなす。
\end{thm}
\begin{proof}
  $n$次元多様体$\left(\mathcal{M},\mathfrak{O}\right)$の台集合$\mathcal{M}$の元$p$が与えられたとする。次のように和とscalar倍が定義されたとき、
  \begin{align*}
    +&:C^\infty_p \mathcal{M} /\sim_p \times C^\infty_p \mathcal{M} /\sim_p \rightarrow C^\infty_p \mathcal{M} /\sim_p ; \left( C_{\sim_p} \left(f\right),C_{\sim_p} \left(g\right)\right) \mapsto C_{\sim_p} \left(f+g\right) \\
    \cdot &: \mathbb{R} \times C^\infty_p \mathcal{M} /\sim_p \rightarrow C^\infty_p \mathcal{M} /\sim_p ; \left( a,C_{\sim_p } \left(f\right) \right) \mapsto C_{\sim_p} \left(af\right)
  \end{align*}
  そのscalar倍はうまく定義されているので、その和がうまく定義されていることを示そう。$\forall f,f',g,g' \in C^\infty_p \mathcal{M}$に対し、$f\sim_p f'$かつ$g\sim_p g'$が成り立つとすれば、即ち、$C_{\sim_p} \left(f\right) =C_{\sim_p } \left(f'\right)$かつ$C_{\sim_p} \left(g\right) =C_{\sim_p } \left(g'\right)$が成り立つとすれば、$\exists V,W\in \mathbf{V} \left(p\right)$に対し、$f|V=f'|V$かつ$g|W=g'|W$が成り立っているので、$V\cap W\in \mathbf{V} \left(p\right)$に注意すれば、$\left(f+g\right)|V\cap W=\left(f'+g'\right)|V\cap W$が成り立つので、$f+g\sim_p f'+g'$が成り立つ、即ち、$C_{\sim_p } \left(f+g\right) =C_{\sim_p } \left(f'+g'\right)$が成り立つ。\par
  次に、$0:\mathcal{M} \rightarrow \mathbb{R} ;q\mapsto 0$は定数関数なので、その元$p$で$C^\infty $級である。このとき、組$\left(C^\infty_p \mathcal{M}/\sim_p,+\right)$が可換群をなすことを示そう。$\forall C_{\sim_p } \left(f\right) ,C_{\sim_p } \left(g\right) ,C_{\sim_p } \left(h\right) \in C^\infty_p \mathcal{M}/\sim_p$に対し、次のようになるかつ、
  \begin{align*}
    \left(C_{\sim_p} \left(f\right) +C_{\sim_p } \left(g\right) \right) +C_{\sim_p } \left(h\right) &=C_{\sim_p } \left(f+g\right) +C_{\sim_p } \left(h\right) \\
    &=C_{\sim_p } \left(\left(f+g\right)+h\right) \\
    &=C_{\sim_p } \left(f+\left(g+h\right)\right) \\
    &=C_{\sim_p } \left(f\right) +C_{\sim_p } \left(g+h\right) \\
    &=C_{\sim_p } \left(f\right) +\left(C_{\sim_p } \left(g\right) +C_{\sim_p } \left(h\right) \right) 
  \end{align*}
  $\forall C_{\sim_p } \left(f\right) \in C^\infty_p \mathcal{M}/\sim_p$に対し、その写像$f$の定義域を$V$とすれば、$0|V\sim_p 0$より次のようになるかつ、
  \begin{align*}
    C_{\sim_p }\left(f\right)+C_{\sim_p}\left(0\right)&=C_{\sim_p }\left(f\right)+C_{\sim_p}\left(0|V\right)\\
    &=C_{\sim_p }\left(f+0|V\right)\\
    &=C_{\sim_p }\left(f\right)
  \end{align*}
  さらに、$\forall C_{\sim_p } \left(f\right) \in C^\infty_p \mathcal{M}/\sim_p$に対し、その写像$f$の定義域を$V$とすれば、$0|V\sim_p 0$より次のようになるかつ、
  \begin{align*}
    C_{\sim_p }\left(f\right)+C_{\sim_p}\left(-f\right)&=C_{\sim_p }\left(f-f\right)\\
    &=C_{\sim_p }\left(0|V\right)\\
    &=C_{\sim_p }\left(0\right)
  \end{align*}
  $\forall C_{\sim_p } \left(f\right) ,C_{\sim_p } \left(g\right) \in C^\infty_p \mathcal{M}/\sim_p$に対し、次のようになる。
  \begin{align*}
    C_{\sim_p} \left(f\right) +C_{\sim_p } \left(g\right) &=C_{\sim_p } \left(f+g\right) \\
    &=C_{\sim_p } \left(g+f\right) \\
    &=C_{\sim_p } \left(g\right) +C_{\sim_p } \left(f\right) 
  \end{align*}
  以上の議論よりその組$\left(C^\infty_p \mathcal{M}/\sim_p,+\right)$は可換群をなす。あとは、次のことがただちに示される。
  \begin{itemize}
    \item $\forall a\in \mathbb{R} \forall C_{\sim_p }\left(f\right),C_{\sim_p }\left(g\right) \in C^\infty_p \mathcal{M}/\sim_p$に対し、$a\left(C_{\sim_p} \left(f\right)+C_{\sim_p} \left(g\right)\right)=aC_{\sim_p} \left(f\right)+aC_{\sim_p} \left(g\right)$が成り立つ。
    \item $\forall a,b\in \mathbb{R} \forall C_{\sim_p }\left(f\right)\in C^\infty_p \mathcal{M}/\sim_p$に対し、$\left(a+b\right)C_{\sim_p} \left(f\right)=aC_{\sim_p} \left(f\right)+bC_{\sim_p} \left(f\right)$が成り立つ。
    \item $\forall a,b\in \mathbb{R} \forall C_{\sim_p }\left(f\right)\in C^\infty_p \mathcal{M}/\sim_p$に対し、$\left(ab\right)C_{\sim_p} \left(f\right)=a\left(bC_{\sim_p} \left(f\right)\right)$が成り立つ。
    \item $\forall C_{\sim_p }\left(f\right)\in C^\infty_p \mathcal{M}/\sim_p$に対し、$1C_{\sim_p} \left(f\right)=C_{\sim_p} \left(f\right)$が成り立つ。
  \end{itemize}
  よって、その商集合$C^\infty_p \mathcal{M} /\sim_p $は体$\mathbb{R}$上のvector空間をなす。
\end{proof}
\begin{thm}\label{8.3.3.5}
  $n$次元多様体$\left(\mathcal{M},\mathfrak{O}\right)$の台集合$\mathcal{M}$の元$p$が与えられたとき、次のように積が定義されれば、
  \begin{align*}
    \cdot &:C^\infty_p \mathcal{M} /\sim_p \times C^\infty_p \mathcal{M} /\sim_p \rightarrow C^\infty_p \mathcal{M} /\sim_p ; \left( C_{\sim_p} \left(f\right),C_{\sim_p} \left(g\right)\right) \mapsto C_{\sim_p} \left(fg\right) 
  \end{align*}
  その商集合$C^\infty_p \mathcal{M} /\sim_p $は可換環をなす。
\end{thm}
\begin{proof}
  $n$次元多様体$\left(\mathcal{M},\mathfrak{O}\right)$の台集合$\mathcal{M}$の元$p$が与えられたとする。次のように積が定義されれば、
  \begin{align*}
    \cdot &:C^\infty_p \mathcal{M} /\sim_p \times C^\infty_p \mathcal{M} /\sim_p \rightarrow C^\infty_p \mathcal{M} /\sim_p ; \left( C_{\sim_p} \left(f\right),C_{\sim_p} \left(g\right)\right) \mapsto C_{\sim_p} \left(fg\right) 
  \end{align*}  
  その積がうまく定義されていることを示そう。$\forall f,f',g,g' \in C^\infty_p \mathcal{M}$に対し、$f\sim_p f'$かつ$g\sim_p g'$が成り立つとすれば、即ち、$C_{\sim_p} \left(f\right) =C_{\sim_p } \left(f'\right)$かつ$C_{\sim_p} \left(g\right) =C_{\sim_p } \left(g'\right)$が成り立つとすれば、$\exists V,W\in \mathbf{V} \left(p\right)$に対し、$f|V=f'|V$かつ$g|W=g'|W$が成り立っているので、$V\cap W\in \mathbf{V} \left(p\right)$に注意すれば、$\left(fg\right)|V\cap W=\left(f'g'\right)|V\cap W$が成り立つので、$fg\sim_p f'g'$が成り立つ、即ち、$C_{\sim_p } \left(fg\right) =C_{\sim_p } \left(f'g'\right)$が成り立つ。\par
  定理\ref{8.3.3.4}の証明と同様にすれば、その組$\left(C^\infty_p \mathcal{M}/\sim_p \right)$は可換群をなす。さらに、$1:\mathcal{M} \rightarrow \mathbb{R} ;q\mapsto 1$は定数関数なので、その元$p$で$C^\infty $級である。これにより、$\forall C_{\sim_p}\left(f\right),C_{\sim_p}\left(g\right),C_{\sim_p}\left(h\right)\in C^\infty_p \mathcal{M}/\sim_p$に対し、次のようになるかつ、
  \begin{align*}
    \left(C_{\sim_p}\left(f\right)C_{\sim_p}\left(g\right)\right)C_{\sim_p}\left(h\right)&=C_{\sim_p}\left(fg\right)C_{\sim_p}\left(h\right)\\
    &=C_{\sim_p}\left(\left(fg\right)h\right)\\
    &=C_{\sim_p}\left(f\left(gh\right)\right)\\
    &=C_{\sim_p}\left(f\right)C_{\sim_p}\left(gh\right)\\
    &=C_{\sim_p}\left(f\right)\left(C_{\sim_p}\left(g\right)C_{\sim_p}\left(h\right)\right)
  \end{align*}
  $\forall C_{\sim_p}\left(f\right)\in C^\infty_p \mathcal{M}/\sim_p$に対し、次のようになるかつ、
  \begin{align*}
    C_{\sim_p}\left(1\right)C_{\sim_p}\left(f\right)&=C_{\sim_p}\left(1f\right)\\
    &=C_{\sim_p}\left(f\right)
  \end{align*}
  $\forall C_{\sim_p}\left(f\right),C_{\sim_p}\left(g\right),C_{\sim_p}\left(h\right)\in C^\infty_p \mathcal{M}/\sim_p$に対し、次のようになるかつ、
  \begin{align*}
    C_{\sim_p}\left(f\right)\left(C_{\sim_p}\left(g\right)+C_{\sim_p}\left(h\right)\right)&=C_{\sim_p}\left(f\right)C_{\sim_p}\left(g+h\right)\\
    &=C_{\sim_p}\left(f\left(g+h\right)\right)\\
    &=C_{\sim_p}\left(fg+fh\right)\\
    &=C_{\sim_p}\left(fg\right)+C_{\sim_p}\left(fh\right)\\
    &=C_{\sim_p}\left(f\right)C_{\sim_p}\left(g\right)+C_{\sim_p}\left(f\right)C_{\sim_p}\left(h\right)
  \end{align*}
  $\forall C_{\sim_p}\left(f\right),C_{\sim_p}\left(g\right)\in C^\infty_p \mathcal{M}/\sim_p$に対し、次のようになる\footnote{これがあるので、積の順序を逆にしたものは示さなくても済む。}。
  \begin{align*}
    C_{\sim_p}\left(f\right)C_{\sim_p}\left(g\right)&=C_{\sim_p}\left(fg\right)\\
    &=C_{\sim_p}\left(gf\right)\\
    &=C_{\sim_p}\left(g\right)C_{\sim_p}\left(f\right)
  \end{align*}
  これにより、その商集合$C^\infty_p \mathcal{M}/\sim_p$は可換環をなす。
\end{proof}
\subsubsection{接vector空間}
\begin{axs}[接vector空間の公理]
  $n$次元多様体$\left(\mathcal{M},\mathfrak{O}\right)$の台集合$\mathcal{M}$の元$p$が与えられたとき、写像$\widetilde{v}:C^\infty_p \mathcal{M}/\sim_p \rightarrow \mathbb{R}$のうち次のことを満たすものの合成写像$\widetilde{v} \circ C_{\sim_p}$をその元$p$におけるその多様体$\left( \mathcal{M},\mathfrak{O} \right) $の接vectorといい、さらに、これ全体の集合をその元$p$におけるその多様体$\left( \mathcal{M},\mathfrak{O} \right) $の接vector空間といい$T_p \mathcal{M} $などと書く。
  \begin{itemize}
    \item その写像$\widetilde{v}$はそのvector 空間$C^\infty_p \mathcal{M}/\sim_p $からvector空間$\mathbb{R}$への線形写像である、即ち、$\forall a,b\in \mathbb{R} \forall C_{\sim_p}\left(f\right),C_{\sim_p}\left(g\right)\in C^\infty_p \mathcal{M}/\sim_p $に対し、次式が成り立つ。このことを線形性という。
    \begin{align*}
      \widetilde{v}\left( aC_{\sim_p}\left(f\right)+bC_{\sim_p}\left(g\right)\right) =a\widetilde{v}\left(C_{\sim_p}\left(f\right)\right) +b\widetilde{v}\left(C_{\sim_p}\left(g\right)\right) 
    \end{align*}
    \item $\forall C_{\sim_p} \left(f\right),C_{\sim_p} \left(g\right)\in C^\infty_p \mathcal{M}/\sim_p $に対し、次式が成り立つ。このことをLeibniz則という。
    \begin{align*}
      \widetilde{v}\left( C_{\sim_p}\left(f\right)C_{\sim_p}\left(g\right)\right) =\widetilde{v}\left(C_{\sim_p}\left(f\right)\right) g\left( p\right) +f\left( p\right) \widetilde{v}\left(C_{\sim_p}\left(g\right)\right) 
    \end{align*}
  \end{itemize}
\end{axs}
\begin{thm}\label{8.3.3.6}
  $n$次元多様体$\left(\mathcal{M},\mathfrak{O}\right)$の台集合$\mathcal{M}$の元$p$が与えられたとき、写像$v:C^\infty_p \mathcal{M}\rightarrow \mathbb{R}$がその元$p$における接vectorであるならそのときに限り、次のことを満たす。
  \begin{itemize}
    \item $\forall a,b\in \mathbb{R} \forall f,g\in C^\infty_p \mathcal{M}$に対し、次式が成り立つ。このことも線形性という。
    \begin{align*}
      v\left( af+bg\right) =av\left(f\right) +bv\left(g\right) 
    \end{align*}
    \item $\forall f,g\in C^\infty_p \mathcal{M}$に対し、次式が成り立つ。このこともLeibniz則という。
    \begin{align*}
      v\left(fg\right)=v\left(f\right)g\left( p\right)+f\left(p\right)v\left(g\right) 
    \end{align*}
    \item $\forall f,g \in C^\infty_p \mathcal{M}$に対し、$f\sim_p g$が成り立つなら、$v\left(f\right) =v\left(g\right)$が成り立つ。
  \end{itemize}
\end{thm}
\begin{proof}
  $n$次元多様体$\left(\mathcal{M},\mathfrak{O}\right)$の台集合$\mathcal{M}$の元$p$が与えられたとき、写像$v:C^\infty_p \mathcal{M}\rightarrow \mathbb{R}$がその元$p$における接vectorであるなら、公理より$v=\widetilde{v}\circ C_{\sim_p}$な写像$\widetilde{v}:C^\infty_p \mathcal{M}/\sim_p \rightarrow \mathbb{R}$が存在して、次のようになる。
  \begin{itemize}
    \item $\forall a,b\in \mathbb{R} \forall f,g\in C^\infty_p \mathcal{M}$に対し、次のようになる。
    \begin{align*}
      v\left( af+bg\right) &=\widetilde{v} \left( C_{\sim_p}\left( af+bg\right) \right) \\
      &=\widetilde{v} \left( aC_{\sim_p}\left(f\right)+bC_{\sim_p}\left(g\right) \right) \\
      &=a\widetilde{v} \left( C_{\sim_p}\left(f\right) \right) +b\widetilde{v} \left( C_{\sim_p}\left(g\right) \right) \\
      &=av\left(f\right) +bv\left(g\right) 
    \end{align*}
    \item $\forall f,g\in C^\infty_p \mathcal{M}$に対し、次のようになる。
    \begin{align*}
      v\left(fg\right)&=\widetilde{v}\left(C_{\sim_p} \left(fg\right)\right)\\
      &=\widetilde{v}\left( C_{\sim_p}\left(f\right)C_{\sim_p}\left(g\right)\right) \\
      &=\widetilde{v}\left(C_{\sim_p}\left(f\right)\right) g\left( p\right) +f\left( p\right) \widetilde{v}\left(C_{\sim_p}\left(g\right)\right) \\
      &=v\left(f\right)g\left( p\right)+f\left(p\right)v\left(g\right) 
    \end{align*}
    \item $\forall f,g \in C^\infty_p \mathcal{M}$に対し、$f\sim_p g$が成り立つなら、$C_{\sim_p }\left(f\right)=C_{\sim_p } \left(g\right)$より$v\left(f\right) =\widetilde{v}\left(C_{\sim_p }\left(f\right)\right)=\widetilde{v}\left(C_{\sim_p }\left(g\right)\right)=v\left(g\right)$が成り立つ。
  \end{itemize}\par
  一方で、その商集合$C^\infty_p \mathcal{M}/\sim_p$の元$C_{\sim_p} f$から$g\in C_{\sim_p}\left(f\right)$な写像$g$をとって実数$v\left(g\right)$へうつす写像$\widetilde{v}:C^\infty_p \mathcal{M}/\sim_p \rightarrow \mathbb{R}$が考えられよう。これはうまく定義されている。実際、$\forall C_{\sim_p} \left(f\right)\in C^\infty_p \mathcal{M}/\sim_p$に対し、その同値類$C_{\sim_p} \left(f\right)$の元々$g$、$h$がとられれば、$f\sim_p g\sim_p h$より$v\left(f\right)=v\left(g\right)=v\left(h\right)$が成り立つ。このとき、次のようになる。
  \begin{itemize}
    \item $\forall a,b\in \mathbb{R} \forall C_{\sim_p}\left(f\right),C_{\sim_p}\left(g\right)\in C^\infty_p \mathcal{M}/\sim_p $に対し、次のようになる。
    \begin{align*}
      \widetilde{v}\left( aC_{\sim_p}\left(f\right)+bC_{\sim_p}\left(g\right)\right) &=\widetilde{v}\left( C_{\sim_p}\left(af+bg\right)\right) \\
      &=v\left(af+bg\right) \\
      &=av\left(f\right)+bv\left(g\right)\\
      &=a\widetilde{v}\left(C_{\sim_p}\left(f\right)\right) +b\widetilde{v}\left(C_{\sim_p}\left(g\right)\right) 
    \end{align*}
    \item $\forall C_{\sim_p} \left(f\right),C_{\sim_p} \left(g\right)\in C^\infty_p \mathcal{M}/\sim_p $に対し、次のようになる。
    \begin{align*}
      \widetilde{v}\left( C_{\sim_p}\left(f\right)C_{\sim_p}\left(g\right)\right) &= \widetilde{v}\left( C_{\sim_p}\left(fg\right)\right) \\
      &=v\left(fg\right) \\
      &=v\left(f\right) g\left( p\right) +f\left( p\right) v\left(g\right) \\
      &=\widetilde{v}\left(C_{\sim_p}\left(f\right)\right) g\left( p\right) +f\left( p\right) \widetilde{v}\left(C_{\sim_p}\left(g\right)\right) 
    \end{align*}
  \end{itemize}
\end{proof}
\begin{thm}\label{8.3.3.7}
  $n$次元多様体$\left(\mathcal{M},\mathfrak{O}\right)$の台集合$\mathcal{M}$の元$p$が与えられたとき、その元$p$におけるその多様体$\left( \mathcal{M},\mathfrak{O} \right) $の接vector空間$T_p \mathcal{M}$において、次のように和とscalar倍が定義されれば、
  \begin{align*}
    +&:T_p \mathcal{M} \times T_p \mathcal{M} \rightarrow T_p \mathcal{M} ; \left( v,w\right) \mapsto v+w:C^\infty_p \mathcal{M} \rightarrow \mathbb{R} ; f\mapsto v\left(f\right) +w\left(f\right) \\
    \cdot &: \mathbb{R} \times T_p \mathcal{M} \rightarrow T_p \mathcal{M} ; \left( \xi,v\right) \mapsto \xi v:C^\infty_p \mathcal{M} \rightarrow \mathbb{R} ; f\mapsto \xi v\left(f\right) 
  \end{align*}
  その接vector空間$T_p \mathcal{M}$は体$\mathbb{R}$上のvector空間をなす。
\end{thm}
\begin{proof}
  $n$次元多様体$\left(\mathcal{M},\mathfrak{O}\right)$の台集合$\mathcal{M}$の元$p$が与えられたとき、その元$p$におけるその多様体$\left( \mathcal{M},\mathfrak{O} \right) $の接vector空間$T_p \mathcal{M}$において、次のように和とscalar倍が定義されれば、
  \begin{align*}
    +&:T_p \mathcal{M} \times T_p \mathcal{M} \rightarrow T_p \mathcal{M} ; \left( v,w\right) \mapsto v+w:C^\infty_p \mathcal{M} \rightarrow \mathbb{R} ; f\mapsto v\left(f\right) +w\left(f\right) \\
    \cdot &: \mathbb{R} \times T_p \mathcal{M} \rightarrow T_p \mathcal{M} ; \left( \xi,v\right) \mapsto \xi v:C^\infty_p \mathcal{M} \rightarrow \mathbb{R} ; f\mapsto \xi v\left(f\right) 
  \end{align*}
  これがうまく定義されていることを示そう。このとき、$\forall v,w\in T_p \mathcal{M}$に対し、和$v+w$が定義されれば、次のようになる。
  \begin{itemize}
    \item $\forall a,b\in \mathbb{R} \forall f,g\in C^\infty_p \mathcal{M}$に対し、次のようになる。
    \begin{align*}
      \left(v+w\right) \left(af+bg\right) &= v \left(af+bg\right) +w\left(af+bg\right) \\
      &= av\left(f\right) +bv\left(g\right) +aw\left(f\right) +bw\left(g\right) \\
      &= a\left(v+w\right) \left(f\right) +b\left(v+w\right) \left(g\right)
    \end{align*}
    \item $\forall f,g\in C^\infty_p \mathcal{M}$に対し、次のようになる。
    \begin{align*}
      \left(v+w\right) \left(fg\right) &= v \left(fg\right) +w\left(fg\right) \\
      &= v\left(f\right) g\left(p\right) +f\left(p\right) v\left(g\right) +w\left(f\right) g\left(p\right) +f\left(p\right) w\left(g\right) \\
      &= \left(v+w\right)\left(f\right) g\left(p\right) +f\left(p\right) \left(v+w\right)\left(g\right) 
    \end{align*}
    \item $\forall f,g\in C^\infty_p \mathcal{M}$に対し、$f\sim_p g$が成り立つなら、次のようになる。
    \begin{align*}
      \left(v+w\right) \left(f\right) &= v \left(f\right) +w\left(f\right) \\
      &=v\left(g\right) +w\left(g\right) \\ 
      &=\left(v+w\right)\left(g\right) 
    \end{align*}
\end{itemize}
  ゆえに、定理\ref{8.3.3.6}より$v+w\in T_p \mathcal{M}$が成り立つ。\par
  また、$\forall \xi \in \mathbb{R}\forall v\in T_p \mathcal{M}$に対し、scalar倍$\xi v$が定義されれば、次のようになる。
  \begin{itemize}
    \item $\forall a,b\in \mathbb{R} \forall f,g\in C^\infty_p \mathcal{M}$に対し、次のようになる。
    \begin{align*}
      \xi v\left(af+bg\right) &= \xi \left( av\left(f\right) +bv\left(g\right) \right) \\
      &= a\xi v\left(f\right) +b\xi v\left(g\right)
    \end{align*}
    \item $\forall f,g\in C^\infty_p \mathcal{M}$に対し、次のようになる。
    \begin{align*}
      \xi v\left(fg\right) &= \xi \left( v\left(f\right) g\left(p\right) +f\left(p\right) v\left(g\right) \right)\\
      &= \xi v\left(f\right) g\left(p\right) +f\left(p\right) \xi v\left(g\right) 
    \end{align*}
    \item $\forall f,g\in C^\infty_p \mathcal{M}$に対し、$f\sim_p g$が成り立つなら、$\xi v\left(f\right) = \xi v\left(g\right)$が成り立つ。
    \end{itemize}
  ゆえに、定理\ref{8.3.3.6}より$\xi v\in T_p \mathcal{M}$が成り立つ。\par
  あとは、次のように写像$0$が定義されれば、
  \begin{align*}
    0:C^\infty_p \mathcal{M} \rightarrow \mathbb{R};f\mapsto 0
  \end{align*}
  次のことがただちに示される。
  \begin{itemize}
    \item その集合$T_p \mathcal{M}$を用いた組$\left(T_p \mathcal{M},+\right)$は可換群をなす。
    \item $\forall \xi \in \mathbb{R} \forall v,w\in T_p \mathcal{M}$に対し、$\xi \left(v+w\right)=\xi v+\xi w$が成り立つ。
    \item $\forall \xi ,\eta \in \mathbb{R} \forall v\in T_p \mathcal{M}$に対し、$\left(\xi +\eta\right)v=\xi v+\eta v$が成り立つ。
    \item $\forall \xi ,\eta \in \mathbb{R} \forall v\in T_p \mathcal{M}$に対し、$\left(\xi \eta\right)v=\xi \left( \eta v\right) $が成り立つ。
    \item $\forall v\in T_p \mathcal{M}$に対し、$1v=v$が成り立つ。
  \end{itemize}
\end{proof}
\begin{thm}\label{8.3.3.8}
  $n$次元多様体$\left(\mathcal{M},\mathfrak{O}\right)$の台集合$\mathcal{M}$の元$p$が与えられたとき、$\forall v\in T_p \mathcal{M} \exists V\in \mathbf{V} \left(p\right) \forall f\in C^\infty_p \mathcal{M}$に対し、その写像$f|V:V\rightarrow \mathbb{R}$が定数関数となっているとき、$v\left(f\right)=0$が成り立つ。
\end{thm}
\begin{proof}
  $n$次元多様体$\left(\mathcal{M},\mathfrak{O}\right)$の台集合$\mathcal{M}$の元$p$が与えられたとき、$\forall v\in T_p \mathcal{M} \exists V\in \mathbf{V} \left(p\right) \forall f\in C^\infty_p \mathcal{M}$に対し、その写像$f|V:V\rightarrow \mathbb{R}$が定数関数となっているとき、$\forall q\in V$に対し、$C=f\left(q\right)$とおき、さらに、$\forall c\in \mathbb{R}$に対し、写像$c$を次のように定義すると、
  \begin{align*}
    c:\mathcal{M} \rightarrow \mathbb{R} ;q\mapsto c
  \end{align*}
  $f\sim_p C$となっており、したがって、次のようになる。
  \begin{align*}
    v\left(C\right) &=v\left(C\cdot 1\right) \\
    &=v\left(C\right) \cdot 1+C\cdot v\left(1\right) \\
    &=v\left(C\right) +v\left(C\right)
  \end{align*}
  これにより、$v\left(f\right)=v\left(C\right)=0$が得られる。
\end{proof}
\begin{dfn}
  $n$次元多様体$\left(\mathcal{M},\mathfrak{O}\right)$の台集合$\mathcal{M}$の元$p$が与えられたとき、$p\in U$、$\psi =\left(\psi^i \right)_{i\in \varLambda_n }$なる座標近傍$\left(U,\psi\right)$を用いて次の写像$\left. \frac{\partial}{\partial \psi^i }\right|_p $が定義される。
  \begin{align*}
    \left. \frac{\partial}{\partial \psi^i }\right|_p :C^\infty_p \mathcal{M} \rightarrow \mathbb{R} ;f\mapsto \frac{\partial f}{\partial \psi^i } \left(p\right)
  \end{align*}
  その写像$\left. \frac{\partial}{\partial \psi^i }\right|_p $をその元$p$におけるその座標近傍$\left(U,\psi\right)$での第$i$偏微分作用素という。
\end{dfn}
\begin{thm}\label{8.3.3.9}
  $n$次元多様体$\left(\mathcal{M},\mathfrak{O}\right)$の台集合$\mathcal{M}$の元$p$が与えられたとき、その元$p$における$\psi =\left(\psi^i \right)_{i\in \varLambda_n }$なる座標近傍$\left(U,\psi\right)$での第$i$偏微分作用素$\left. \frac{\partial}{\partial \psi^i }\right|_p $はその元$p$における接vectorである。
\end{thm}
\begin{proof}
  $n$次元多様体$\left(\mathcal{M},\mathfrak{O}\right)$の台集合$\mathcal{M}$の元$p$が与えられたとき、その元$p$における$\psi =\left(\psi^i \right)_{i\in \varLambda_n }$なる座標近傍$\left(U,\psi\right)$での第$i$偏微分作用素$\left. \frac{\partial}{\partial \psi^i }\right|_p $について、次のようになる。
  \begin{itemize}
    \item $\forall a,b\in \mathbb{R} \forall f,g \in C^\infty_p \mathcal{M} $に対し、次のようになる。
    \begin{align*}
      \left. \frac{\partial}{\partial \psi^i }\right|_p \left(af+bg\right) &= \frac{\partial }{\partial \psi^i } \left(af+bg\right) \left(p\right) \\
      &=a\frac{\partial f}{\partial \psi^i } \left(p\right) +b\frac{\partial g}{\partial \psi^i }\left(p\right) \\
      &=a\left. \frac{\partial }{\partial \psi^i } \right|_p \left(f\right) +b\left. \frac{\partial }{\partial \psi^i } \right|_p \left(g\right) 
    \end{align*}
    \item $\forall f,g \in C^\infty_p \mathcal{M} $に対し、次のようになる。
    \begin{align*}
      \left. \frac{\partial}{\partial \psi^i }\right|_p \left(fg\right) &= \frac{\partial }{\partial \psi^i } \left(fg\right) \left(p\right) \\
      &= \partial_i \left(fg\circ \psi^{-1} \right) \circ \psi \left(p\right) \\
      &= \partial_i \left(\left(f\circ \psi^{-1} \right) \left(g\circ \psi^{-1} \right)\right) \circ \psi \left(p\right) \\
      &= \left( \partial_i \left(f\circ \psi^{-1} \right) g\circ \psi^{-1} +f\circ \psi^{-1} \partial_i \left(g\circ \psi^{-1} \right)\right) \circ \psi \left(p\right) \\
      &= \partial_i \left(f\circ \psi^{-1} \right) \circ \psi \left(p\right) g\left(p\right) +f\left(p\right) \partial_i \left(g\circ \psi^{-1} \right) \circ \psi \left(p\right) \\
      &=\frac{\partial f}{\partial \psi^i } \left(p\right) g\left(p\right) +f\left(p\right) \frac{\partial g}{\partial \psi^i } \left(p\right) \\
      &=\left. \frac{\partial}{\partial \psi^i }\right|_p \left(f\right) g\left(p\right) +f\left(p\right) \left. \frac{\partial}{\partial \psi^i }\right|_p \left(g\right)
    \end{align*}
    \item $\forall f,g\in C^\infty_p \mathcal{M}$に対し、$f\sim_p g$が成り立つなら、$\exists V\in \mathbf{V} \left(p\right)$に対し、$f|V=g|V$が成り立つので、次のようになる。
    \begin{align*}
      \left. \frac{\partial}{\partial \psi^i } \right|_p \left(f\right) =\frac{\partial f}{\partial \psi^i } \left(p\right) =\frac{\partial f|V}{\partial \psi^i } \left(p\right) =\frac{\partial g|V}{\partial \psi^i } \left(p\right) =\frac{\partial g}{\partial \psi^i } \left(p\right) =\left. \frac{\partial}{\partial \psi^i } \right|_p \left(g\right)
    \end{align*}
  \end{itemize}
  以上、定理\ref{8.3.3.6}より$\left. \frac{\partial }{\partial \psi^i }\right|_p \in T_p \mathcal{M}$が成り立つ。
\end{proof}
\begin{thm}\label{8.3.3.10}
  $n$次元多様体$\left(\mathcal{M},\mathfrak{O}\right)$の台集合$\mathcal{M}$の元$p$が与えられたとき、その元$p$における$\psi =\left(\psi^i \right)_{i\in \varLambda_n }$なる座標近傍$\left(U,\psi\right)$での第$i$偏微分作用素$\left. \frac{\partial}{\partial \psi^i }\right|_p $からなる組$\left\langle \left. \frac{\partial}{\partial \psi^i }\right|_p \right\rangle_{i\in \varLambda_n} $はその接vector空間$T_p \mathcal{M}$の基底で$k\in \varLambda_n $として、$\forall v\in T_p \mathcal{M}$に対し、次式が成り立つ。
  \begin{align*}
    v=v\left(\psi^k \right) \left. \frac{\partial }{\partial \psi^k } \right|_p 
  \end{align*}
  特に、$\dim T_p \mathcal{M} =n$が成り立つ。
\end{thm}
\begin{dfn}
  $n$次元多様体$\left(\mathcal{M},\mathfrak{O}\right)$の台集合$\mathcal{M}$の元$p$が与えられたとき、その元$p$における$\psi =\left(\psi^i \right)_{i\in \varLambda_n }$なる座標近傍$\left(U,\psi\right)$での第$i$偏微分作用素$\left. \frac{\partial}{\partial \psi^i }\right|_p $からなる基底$\left\langle \left. \frac{\partial}{\partial \psi^i }\right|_p \right\rangle_{i\in \varLambda_n} $をその元$p$における座標近傍$\left(U,\psi\right)$での自然標構、自然基底という。
\end{dfn}
\begin{proof}
  $n$次元多様体$\left(\mathcal{M},\mathfrak{O}\right)$の台集合$\mathcal{M}$の元$p$が与えられたとき、その元$p$における$\psi =\left(\psi^i \right)_{i\in \varLambda_n }$なる座標近傍$\left(U,\psi\right)$での第$i$偏微分作用素$\left. \frac{\partial}{\partial \psi^i }\right|_p $からなる組$\left\langle \left. \frac{\partial}{\partial \psi^i }\right|_p \right\rangle_{i\in \varLambda_n} $が与えられたとき、$k\in \varLambda_n $、$c_k \in \mathbb{R}$として$c_k \left. \frac{\partial }{\partial \psi^k } \right|_p =0$が成り立つなら、$\forall i\in \varLambda_n $に対し、$\psi^i \in C^\infty_p \mathcal{M}$が成り立つので、次のようになる。
  \begin{align*}
    0 &= c_k \left. \frac{\partial }{\partial \psi^k } \right|_p \left( \psi^i \right) \\
    &= c_k \frac{\partial \psi^i }{\partial \psi^k } \left(p\right) \\
    &= c_k \partial_k \left( \psi^i \circ \psi^{-1} \right) \circ \psi \left( p \right)\\
    &= c_k \delta_{ik} = c_i 
  \end{align*}
  これにより、その組$\left\langle \left. \frac{\partial}{\partial \psi^i }\right|_p \right\rangle_{i\in \varLambda_n} $は線形独立である。\par
  一方で、その集合$V\left(\psi\right)$は開集合であるので、$\exists \varepsilon \in \mathbb{R}^+$に対し、$U\left(\psi\left(p\right),\varepsilon\right)\subseteq V\left(\psi\right)$が成り立つ\footnote{$U\left(\psi\left(p\right),\varepsilon\right)$は中心$\psi\left(p\right)$、半径$\varepsilon$の開球です。}。そこで、その関数$f\circ \psi^{-1}$は$C^2$級でもあるので、Taylorの定理より$h=\left(h^i \right)_{i\in \varLambda_n } :V\left(\psi^{-1}|U\left(\psi\left(p\right),\varepsilon\right)\right) \rightarrow \mathbb{R} ;q\mapsto \psi \left(q\right) -\psi \left(p\right)$とすれば、$\forall q\in V\left(\psi^{-1}|U\left(\psi\left(p\right),\varepsilon\right)\right)\exists c\in \left(0,1\right)$に対し、次式が成り立つ、
  \begin{align*}
    f\circ \psi^{-1} \left(\psi \left(q\right) \right) &=f\circ \psi^{-1} \left(\psi\left(p\right)\right) +\partial_k \left(f\circ \psi^{-1}\right) \left(\psi\left(p\right)\right) h^k \left(q \right) \\
    &\quad +\frac{1}{2}\partial_{kl} \left(f\circ \psi^{-1}\right) \left(\psi\left(p\right) +ch\left(q\right)\right) h^k \left(q\right) h^l \left(q\right)
  \end{align*}
  即ち、次式が成り立つ。
  \begin{align*}
    f\left(q\right) &=f\left(p\right) +\partial_k \left(f\circ \psi^{-1}\right) \circ \psi\left(p\right) h^k \left(q\right) +\frac{1}{2}\partial_{kl} \left(f\circ \psi^{-1}\right) \left(\psi\left(p\right) +ch\left(q\right)\right) h^k \left(q\right) h^l \left(q\right) \\
    &=f\left(p\right) +\frac{\partial f}{\partial \psi^k } \left(p\right) h^k \left(q\right) +\frac{1}{2}\partial_{kl} \left(f\circ \psi^{-1}\right) \left(\psi\left(p\right) +ch\left(q\right)\right) h^k \left(q\right) h^l \left(q\right) 
  \end{align*}
  これを写像とみれば、次のようになる。
  \begin{align*}
    f|V\left(\psi^{-1}|U\left(\psi\left(p\right),\varepsilon\right)\right) =f\left(p\right) +\frac{\partial f}{\partial \psi^k } \left(p\right) h^k +\frac{1}{2}\partial_{kl} \left(f\circ \psi^{-1}\right) \left(\psi\left(p\right) +ch \right) h^k h^l 
  \end{align*} 
  したがって、次のようになる。
  \begin{align*}
    v\left(f\right) &=v\left(f\left(p\right) +\frac{\partial f}{\partial \psi^k } \left(p\right) h^k +\frac{1}{2}\partial_{kl} \left(f\circ \psi^{-1}\right) \left(\psi\left(p\right) +ch \right) h^k h^l \right) \\
    &=v\left(f\left(p\right) \right)+\frac{\partial f}{\partial \psi^k } \left(p\right) v\left(h^k \right) +\frac{1}{2}v\left(\partial_{kl} \left(f\circ \psi^{-1}\right) \left(\psi\left(p\right) +ch \right) h^k h^l \right) \\
    &=v\left(f\left(p\right) \right)+\frac{\partial f}{\partial \psi^k } \left(p\right) v\left(h^k \right) \\
    &\quad +\frac{1}{2}v\left(\partial_{kl} \left(f\circ \psi^{-1}\right) \left(\psi\left(p\right) +ch \right) \right) h^k \left(p\right) h^l \left(p\right) \\
    &\quad +\frac{1}{2}\partial_{kl} \left(f\circ \psi^{-1}\right) \left(\psi\left(p\right) +ch\left(p\right) \right) v\left(h^k\right) h^l \left(p\right) \\
    &\quad +\frac{1}{2}\partial_{kl} \left(f\circ \psi^{-1}\right) \left(\psi\left(p\right) +ch\left(p\right) \right) h^k \left(p\right) v\left(h^l\right) \\
    &=v\left(f\left(p\right) \right)+\frac{\partial f}{\partial \psi^k } \left(p\right) v\left(\psi^k -\psi^k  \left(p\right) \right) \\
    &\quad +\frac{1}{2}v\left(\partial_{kl} \left(f\circ \psi^{-1}\right) \left(\psi\left(p\right) +ch \right) \right) \left(\psi^k \left(p\right) -\psi^k  \left(p\right)\right) \left(\psi^l \left(p\right) -\psi^l \left(p\right)\right)\\
    &\quad +\frac{1}{2}\partial_{kl} \left(f\circ \psi^{-1}\right) \left(\psi\left(p\right) +ch\left(p\right) \right) v\left(\psi^k-\psi^k  \left(p\right)\right) \left(\psi^l \left(p\right) -\psi^l \left(p\right)\right)\\
    &\quad +\frac{1}{2}\partial_{kl} \left(f\circ \psi^{-1}\right) \left(\psi\left(p\right) +ch\left(p\right) \right) \left(\psi^k \left(p\right) -\psi^k  \left(p\right)\right) v\left(\psi^l -\psi^l \left(p\right)\right) \\
    &=v\left(f\left(p\right) \right)+\frac{\partial f}{\partial \psi^k } \left(p\right) \left(v\left(\psi^k \right)-v\left(\psi^k \left(p\right) \right)\right) \\
    &\quad +\frac{1}{2}v\left(\partial_{kl} \left(f\circ \psi^{-1}\right) \left(\psi\left(p\right) +ch \right) \right) \left(\psi^k \left(p\right) -\psi^k  \left(p\right)\right) \left(\psi^l \left(p\right) -\psi^l \left(p\right)\right)\\
    &\quad +\frac{1}{2}\partial_{kl} \left(f\circ \psi^{-1}\right) \left(\psi\left(p\right) +ch\left(p\right) \right) \left(v\left(\psi^k \right)-v\left(\psi^k \left(p\right) \right)\right)  \left(\psi^l \left(p\right) -\psi^l \left(p\right)\right)\\
    &\quad +\frac{1}{2}\partial_{kl} \left(f\circ \psi^{-1}\right) \left(\psi\left(p\right) +ch\left(p\right) \right) \left(\psi^k \left(p\right) -\psi^k  \left(p\right)\right) \left(v\left(\psi^l \right)-v\left(\psi^l \left(p\right) \right)\right) \\
    &=0+\frac{\partial f}{\partial \psi^k } \left(p\right) \left(v\left(\psi^k \right)-0\right) \\
    &\quad +\frac{1}{2}v\left(\partial_{kl} \left(f\circ \psi^{-1}\right) \left(\psi\left(p\right) +ch \right) \right) \cdot 0\cdot 0\\
    &\quad +\frac{1}{2}\partial_{kl} \left(f\circ \psi^{-1}\right) \left(\psi\left(p\right) +ch\left(p\right) \right) \left(v\left(\psi^k \right)-v\left(\psi^k \left(p\right) \right)\right) \cdot 0\\
    &\quad +\frac{1}{2}\partial_{kl} \left(f\circ \psi^{-1}\right) \left(\psi\left(p\right) +ch\left(p\right) \right) \cdot 0 \cdot \left(v\left(\psi^l \right)-v\left(\psi^l \left(p\right) \right)\right) \\
    &=v\left(\psi^k \right) \frac{\partial f}{\partial \psi^k } \left(p\right) =v\left(\psi^k \right) \left. \frac{\partial}{\partial \psi^k } \right|_p \left(f\right)
  \end{align*}
  これにより、次式が成り立つので、
  \begin{align*}
    v=v\left(\psi^k \right) \left. \frac{\partial }{\partial \psi^k } \right|_p 
  \end{align*}
  その接vector空間$T_p \mathcal{M}$はそれらの接vectorたち$\left. \frac{\partial }{\partial \psi^i } \right|_p $によって張られている。\par
  よって、その組$\left\langle \left. \frac{\partial}{\partial \psi^i }\right|_p \right\rangle_{i\in \varLambda_n} $はその接vector空間$T_p \mathcal{M}$の基底で$k\in \varLambda_n $として、$\forall v\in T_p \mathcal{M}$に対し、次式が成り立つ。
  \begin{align*}
    v=v\left(\psi^k \right) \left. \frac{\partial }{\partial \psi^k } \right|_p 
  \end{align*}
  特に、$\dim T_p \mathcal{M} =n$が成り立つ。
\end{proof}
\begin{thm}\label{8.3.3.11}
  $n$次元多様体$\left(\mathcal{M},\mathfrak{O}\right)$の台集合$\mathcal{M}$の元$p$が与えられたとき、その元$p$における座標近傍$\left(U_\alpha,\psi_\alpha \right)$での自然標構$\left\langle \left. \frac{\partial}{\partial \psi^i_\alpha }\right|_p \right\rangle_{i\in \varLambda_n} $、その元$p$における座標近傍$\left(U_\beta ,\psi_\beta \right)$での自然標構$\left\langle \left. \frac{\partial}{\partial \psi^i_\beta }\right|_p \right\rangle_{i\in \varLambda_n} $について、$\forall v\in T_p \mathcal{M} \forall i\in \varLambda_n $に対し、次式が成り立つ。
  \begin{align*}
    v\left( \psi^i_\beta \right)=\frac{\partial \psi^i_\beta }{\partial \psi^k_\alpha } \left(p\right) v\left(\psi^k_\alpha \right) 
  \end{align*}
\end{thm}
\begin{proof}
  $n$次元多様体$\left(\mathcal{M},\mathfrak{O}\right)$の台集合$\mathcal{M}$の元$p$が与えられたとき、その元$p$における座標近傍$\left(U_\alpha,\psi_\alpha \right)$での自然標構$\left\langle \left. \frac{\partial}{\partial \psi^i_\alpha }\right|_p \right\rangle_{i\in \varLambda_n} $、その元$p$における座標近傍$\left(U_\beta ,\psi_\beta \right)$での自然標構$\left\langle \left. \frac{\partial}{\partial \psi^i_\beta }\right|_p \right\rangle_{i\in \varLambda_n} $について、$\forall v\in T_p \mathcal{M} \forall i\in \varLambda_n $に対し、定理\ref{8.3.3.10}より次のようになる。
  \begin{align*}
    v=v\left(\psi^l_\beta \right) \left. \frac{\partial}{\partial \psi_\beta^l } \right|_p 
  \end{align*}
  一方で、$\forall f\in C^\infty_p \mathcal{M}$に対し、定理\ref{8.3.2.2}より次のようになるので、
  \begin{align*}
    v\left(f\right) &= v\left(\psi^k_\alpha \right) \left. \frac{\partial }{\partial \psi^k_\alpha } \right|_p \left(f\right) \\
    &= v\left(\psi^k_\alpha \right) \frac{\partial f}{\partial \psi^k_\alpha } \left(p\right) \\
    &= v\left(\psi^k_\alpha \right) \frac{\partial \psi^l_\beta }{\partial \psi^k_\alpha } \left(p\right) \frac{\partial f}{\partial \psi^l_\beta } \left(p\right) \\
    &= v\left(\psi^k_\alpha \right) \frac{\partial \psi^l_\beta }{\partial \psi^k_\alpha } \left(p\right) \left. \frac{\partial }{\partial \psi^l_\beta } \right|_p \left(f\right) 
  \end{align*}
  したがって、次式が成り立つ。
  \begin{align*}
    v=v\left(\psi^k_\alpha \right) \frac{\partial \psi^l_\beta }{\partial \psi^k_\alpha } \left(p\right) \left. \frac{\partial }{\partial \psi^l_\beta } \right|_p =\frac{\partial \psi^l_\beta }{\partial \psi^k_\alpha } \left(p\right) v\left(\psi^k_\alpha \right) \left. \frac{\partial }{\partial \psi^l_\beta } \right|_p
  \end{align*}
  定理\ref{8.3.3.10}より成分を比較して、よって、$\forall i\in \varLambda_n $に対し、次式が成り立つ。
  \begin{align*}
    v\left( \psi^i_\beta \right)=\frac{\partial \psi^i_\beta }{\partial \psi^k_\alpha } \left(p\right) v\left(\psi^k_\alpha \right)  
  \end{align*}
\end{proof}
\subsubsection{自然な線形同型写像$\iota :T_{\mathbf{p}} \mathbb{R}^n \rightarrow \mathbb{R}^n $}
\begin{thm}\label{8.3.3.12}
  $n$次元Euclid空間$E^n$における位相空間$\left(\mathbb{R}^n ,\mathfrak{O}_{d_{E^n}}\right)$は$n$次元$C^\infty $級多様体である。
\end{thm}
\begin{proof}
  $n$次元Euclid空間$E^n$における位相空間$\left(\mathbb{R}^n ,\mathfrak{O}_{d_{E^n}}\right)$が与えられたとき、これはもちろん$n$次元位相多様体である。そこで、例えば、$\varepsilon =\left\langle \mathbf{e}_i \right\rangle_{i\in \varLambda_n } $、$\mathbf{e}_j =\left( \delta_{ij} \right)_{i\in \varLambda_n }$なるその$n$次元数空間$\mathbb{R}^n $の基底$\varepsilon$を用いた次のような座標近傍系$\left\{ \left(\mathbb{R}^n ,1_\varepsilon \right) \right\} $はその位相多様体$\left(\mathbb{R}^n ,\mathfrak{O}_{d_{E^n}}\right)$に$C^\infty $可微分構造を入れている。
  \begin{align*}
    1_\varepsilon =\left( 1_\varepsilon^i \right)_{i\in \varLambda_n },\ \ 1_\varepsilon^i :\mathbb{R}^n \rightarrow \mathbb{R} ;a_k \mathbf{e}_k \mapsto a_i 
  \end{align*}
\end{proof}
\begin{thm}\label{8.3.3.13}
  $n$次元Euclid空間$E^n$における位相空間$\left(\mathbb{R}^n ,\mathfrak{O}_{d_{E^n}}\right)$は$n$次元$C^\infty $級多様体であった。そこで、$\forall \mathbf{p} \in \mathbb{R}^n $に対し、$\varepsilon =\left\langle \mathbf{e}_i \right\rangle_{i\in \varLambda_n } $なるその$n$次元数空間$\mathbb{R}^n $の基底$\varepsilon$を用いた次のような写像$\iota $は自然な線形同型写像である\footnote{線形代数学で線形同型写像を特徴づけるのに基底を予め決めておく必要がないようなものを自然な線形同型写像などといったりします。}。
  \begin{align*}
    \iota :T_\mathbf{p} \mathbb{R}^n \rightarrow \mathbb{R}^n &;v\mapsto v\left(1_\varepsilon^k \right) \mathbf{e}_k ,\\
    1_\varepsilon =\left( 1_\varepsilon^i \right)_{i\in \varLambda_n },\ \ 1_\varepsilon^i &:\mathbb{R}^n \rightarrow \mathbb{R} ;a_k \mathbf{e}_k \mapsto a_i 
  \end{align*}
\end{thm}\par
このことから、しばしば$T_\mathbf{p} \mathbb{R}^n =\mathbb{R}^n $と同一視することもある。このとき、$\forall v\in T_\mathbf{p} \mathbb{R}^n $に対し、次のようになる。
\begin{align*}
  v=v\left(1_\varepsilon^k \right) \left. \frac{\partial }{\partial 1_\varepsilon^k } \right|_\mathbf{p} =v\left(1_\varepsilon^k \right) \mathbf{e}_k
\end{align*}
\begin{proof}
  $n$次元Euclid空間$E^n$における位相空間$\left(\mathbb{R}^n ,\mathfrak{O}_{d_{E^n}}\right)$は$n$次元$C^\infty $級多様体であった。そこで、$\forall \mathbf{p} \in \mathbb{R}^n $に対し、$\varepsilon =\left\langle \mathbf{e}_i \right\rangle_{i\in \varLambda_n } $なるその$n$次元数空間$\mathbb{R}^n $の基底$\varepsilon$を用いた次のような写像$\iota $について、
  \begin{align*}
    \iota :T_\mathbf{p} \mathbb{R}^n \rightarrow \mathbb{R}^n &;v\mapsto v\left(1_\varepsilon^l \right) \mathbf{e}_l ,\\
    1_\varepsilon =\left( 1_\varepsilon^i \right)_{i\in \varLambda_n },\ \ 1_\varepsilon^i &:\mathbb{R}^n \rightarrow \mathbb{R} ;b_l \mathbf{e}_l \mapsto b_i 
  \end{align*}
  その組$\left(\mathbb{R}^n ,1_\varepsilon \right)$はその点$\mathbf{p}$における座標近傍で$1_\varepsilon^i \in C^\infty_p \mathcal{M}$が成り立つ。ゆえに、$\forall v\in T_\mathbf{p} \mathcal{M}$に対し、定理\ref{8.3.3.10}より次式が成り立つ。
  \begin{align*}
    v=v\left(1_\varepsilon^l \right) \left. \frac{\partial }{\partial 1_\varepsilon^l } \right|_\mathbf{p}
  \end{align*}\par
  $\forall \xi ,\eta \in \mathbb{R} \forall v,w\in T_\mathbf{p} \mathcal{M}$に対し、次のようになるので、
  \begin{align*}
    \iota \left(\xi v+\eta w\right) &=\left(\xi v+\eta w\right) \left(1_\varepsilon^l \right) \mathbf{e}_l \\
    &=\xi v \left(1_\varepsilon^l \right) \mathbf{e}_l+\eta w \left(1_\varepsilon^l \right) \mathbf{e}_l \\
    &=\xi \iota \left(v\right)+\eta \iota \left(w\right)
  \end{align*}
  その写像$\iota $は線形写像である。\par
  さらに、定理\ref{8.3.3.10}より$\dim T_\mathbf{p} \mathbb{R}^n =\dim \mathbb{R}^n =n$が成り立つので、その線形写像$\iota $が線形同型写像であることを示すにはその線形写像$\iota $が単射であることを示せばよい。$\forall v,w\in T_\mathbf{p} \mathbb{R}^n $に対し、$v\ne w$が成り立つなら、定理\ref{8.3.3.10}より次式が成り立つので、
  \begin{align*}
    v=v\left(1_\varepsilon^l \right) \left. \frac{\partial }{\partial 1_\varepsilon^l } \right|_\mathbf{p},\ \ w=w\left(1_\varepsilon^l \right) \left. \frac{\partial }{\partial 1_\varepsilon^l } \right|_\mathbf{p}
  \end{align*}
  $\exists i\in \varLambda_n $に対し、$v\left(1_\varepsilon^i \right) \ne w\left(1_\varepsilon^i \right)$が成り立つ。これにより、次のようになるので、
  \begin{align*}
    \iota \left(v\right) =v\left(1_\varepsilon^l \right) \mathbf{e}_l \ne w\left(1_\varepsilon^l \right) \mathbf{e}_l =\iota \left(w\right)
  \end{align*}
  その線形写像$\iota $は単射である。よって、その線形写像$\iota $は線形同型写像である。\par
  さらに、その$n$次元数空間$\mathbb{R}^n $の$\delta =\left\langle \mathbf{d}_i \right\rangle_{i\in \varLambda_n }$なる基底$\delta $を用いて、$\forall i\in \varLambda_n $に対し、$\mathbf{d}_i =P_{ik} \mathbf{e}_k $とおくと、その行列$\left(P_{ij}\right)_{\left(i,j\right)\in \varLambda_n^2 } $の逆行列$\left(Q_{ij}\right)_{\left(i,j\right)\in \varLambda_n^2 }$が存在して、次式が成り立つ。
  \begin{align*}
    \mathbf{e}_i =Q_{ik} \mathbf{d}_k ,\ \ P_{ik} Q_{kj} =Q_{il} P_{lj} =\delta_{ij}
  \end{align*}
  ここで、次のように写像$1_\delta $が定義されれば、
  \begin{align*}
    1_\delta =\left( 1_\delta^i \right)_{i\in \varLambda_n },\ \ 1_\delta^i &:\mathbb{R}^n \rightarrow \mathbb{R} ;a_k \mathbf{d}_k \mapsto a_i 
  \end{align*}
  その組$\left(\mathbb{R}^n ,1_\delta \right)$はその点$\mathbf{p}$における座標近傍で$1_\delta^i \in C^\infty_p \mathcal{M}$が成り立つ。$\forall a_k \mathbf{d}_k \in \mathbb{R}^n $に対し、$1_\delta^i \left(a_k \mathbf{d}_k \right) =a_i $に注意すれば、次のようになる。
  \begin{align*}
    1_\varepsilon^i \left(a_k \mathbf{d}_k \right) =1_\varepsilon^i \left( a_k P_{kl} \mathbf{e}_l \right)= a_k P_{ki} =1_\delta^k \left( a_k \mathbf{d}_k \right) P_{ki}
  \end{align*}
  したがって、$1_\varepsilon^i =1_\delta^k P_{ki}$が成り立つ。このとき、次のようになるので、
  \begin{align*}
    \iota \left( v\right) &= v\left(1_\varepsilon^l \right) \mathbf{e}^l \\
    &= v\left(1_\delta^k P_{kl} \right) Q_{lm} \mathbf{d}_m \\
    &= v\left(1_\delta^k \right) P_{kl} Q_{lm} \mathbf{d}_m \\
    &= v\left(1_\delta^k \right) \delta_{km} \mathbf{d}_m \\
    &= v\left(1_\delta^k \right) \mathbf{d}_k
  \end{align*}
  よって、その線形同型写像$\iota $は自然な線形同型写像でもある。
\end{proof}
\subsubsection{接束}
\begin{dfn}
  $n$次元多様体$\left(\mathcal{M},\mathfrak{O}\right)$が与えられたとき、次のように定義される$\left(p,v\right)\in \left\{p\right\}\times T_p \mathcal{M}$なる組$\left(p,v\right)$全体の和集合をその多様体$\left(\mathcal{M},\mathfrak{O}\right)$の接束、接bundleという。
  \begin{align*}
    T \mathcal{M} =\bigsqcup_{p\in \mathcal{M}} \left\{p\right\} \times T_p \mathcal{M}
  \end{align*}
\end{dfn}
\begin{thm}\label{8.3.3.14}
  $n$次元多様体$\left(\mathcal{M},\mathfrak{O}\right)$が与えられたとき、$\forall p\in \mathcal{M}$に対し、その集合$\left\{ p\right\} \times T_p \mathcal{M}$について、次のように和とscalar倍が定義されれば、
  \begin{align*}
    +&:\left( \left\{ p\right\} \times T_p \mathcal{M} \right) \times \left( \left\{ p\right\} \times T_p \mathcal{M} \right) \rightarrow \left\{ p\right\} \times T_p \mathcal{M} ;\left(\left(p,v\right),\left(p,w\right) \right) \mapsto \left(p,v+w\right)\\
    \cdot &: \mathbb{R} \times \left( \left\{ p\right\} \times T_p \mathcal{M} \right) \rightarrow \left\{ p\right\} \times T_p \mathcal{M}; \left(\xi ,\left(p,v\right)\right) \mapsto \left(p,\xi v\right)
  \end{align*}
  その集合$\left\{ p\right\} \times T_p \mathcal{M}$は体$\mathbb{R}$上のvector空間をなす。さらに、そのvector空間$\left\{ p\right\} \times T_p \mathcal{M}$の基底としてその元$p$における座標近傍$\left(U,\psi\right)$での自然標構$\left\langle \left. \frac{\partial}{\partial \psi^i }\right|_p \right\rangle_{i\in \varLambda_n } $を用いた組$\left\langle \left(p,\left.\frac{\partial }{\partial \psi^i } \right|_p \right)\right\rangle_{i\in \varLambda_n }$が挙げられる。特に、$\dim \left( \left\{ p\right\} \times T_p \mathcal{M} \right) =n$が成り立つ。
\end{thm}
\begin{proof}
  $n$次元多様体$\left(\mathcal{M},\mathfrak{O}\right)$が与えられたとき、$\forall p\in \mathcal{M}$に対し、その集合$\left\{ p\right\} \times T_p \mathcal{M}$について、次のように和とscalar倍が定義されれば、
  \begin{align*}
    +&:\left( \left\{ p\right\} \times T_p \mathcal{M} \right) \times \left( \left\{ p\right\} \times T_p \mathcal{M} \right) \rightarrow \left\{ p\right\} \times T_p \mathcal{M} ;\left(\left(p,v\right),\left(p,w\right) \right) \mapsto \left(p,v+w\right)\\
    \cdot &: \mathbb{R} \times \left( \left\{ p\right\} \times T_p \mathcal{M} \right) \rightarrow \left\{ p\right\} \times T_p \mathcal{M}; \left(\xi ,\left(p,v\right)\right) \mapsto \left(p,\xi v\right)
  \end{align*}
  定理\ref{8.3.3.7}で述べられたその写像$0:C^\infty_p \mathcal{M} \rightarrow \mathbb{R}$を用いて次のことが示される。
  \begin{itemize}
    \item その組$\left(\left\{ p\right\} \times T_p \mathcal{M} ,+\right)$は可換群をなす。
    \item $\forall \xi \in \mathbb{R} \forall \left(p,v\right) ,\left(p,w\right) \in \left\{ p\right\} \times T_p \mathcal{M}$に対し、$\xi \left(\left(p,v\right)+\left(p,w\right)\right)=\xi \left(p,v\right)+\xi \left(p,w\right)$が成り立つ。
    \item $\forall \xi ,\eta \in \mathbb{R} \forall \left(p,v\right)\in \left\{ p\right\} \times T_p \mathcal{M}$に対し、$\left(\xi +\eta\right)\left(p,v\right)=\xi \left(p,v\right)+\eta \left(p,v\right)$が成り立つ。
    \item $\forall \xi ,\eta \in \mathbb{R} \forall \left(p,v\right)\in \left\{ p\right\} \times T_p \mathcal{M}$に対し、$\left(\xi \eta\right)\left(p,v\right)=\xi \left( \eta \left(p,v\right)\right) $が成り立つ。
    \item $\forall \left(p,v\right)\in \left\{ p\right\} \times T_p \mathcal{M}$に対し、$1\left(p,v\right)=\left(p,v\right)$が成り立つ。
  \end{itemize}
  よって、その集合$\left\{ p\right\} \times T_p \mathcal{M}$は体$\mathbb{R}$上のvector空間をなす。\par
  また、もちろん、その元$p$における座標近傍$\left(U,\psi\right)$での自然標構$\left\langle \left. \frac{\partial}{\partial \psi^i }\right|_p \right\rangle_{i\in \varLambda_n } $を用いた組$\left\langle \left(p,\left.\frac{\partial }{\partial \psi^i } \right|_p \right)\right\rangle_{i\in \varLambda_n }$が与えられたとき、$\forall i\in \varLambda_n $に対し、$\left(p,\left.\frac{\partial }{\partial \psi^i } \right|_p \right) \in \left\{ p\right\} \times T_p \mathcal{M}$が成り立つ。さらに、$c_k \left(p,\left.\frac{\partial }{\partial \psi^k } \right|_p \right) =\left(p,0\right) $のとき次のようになることから、
  \begin{align*}
    \left(p,0\right) = c_k \left(p,\left.\frac{\partial }{\partial \psi^k } \right|_p \right) =\left(p,c_k \left.\frac{\partial }{\partial \psi^k } \right|_p \right)
  \end{align*}
  $c_k \left.\frac{\partial }{\partial \psi^k } \right|_p =0$が成り立つ。定理\ref{8.3.3.10}より、$\forall i\in \varLambda_n $に対し、$c_i =0$が成り立つ。これにより、その組$\left\langle \left(p,\left.\frac{\partial }{\partial \psi^i } \right|_p \right)\right\rangle_{i\in \varLambda_n }$は線形独立である。さらに、$\forall \left(p,v\right) \in \left\{ p\right\} \times T_p \mathcal{M}$に対し、$v\in T_p \mathcal{M}$が成り立つので、定理\ref{8.3.3.10}より$v=v\left(\psi^k \right) \left. \frac{\partial }{\partial \psi^k }\right|_p $が成り立つので、次のようになる。
  \begin{align*}
    \left(p,v\right) =\left(p,v\left(\psi^k \right) \left. \frac{\partial }{\partial \psi^k }\right|_p \right) =v\left(\psi^k \right) \left(p,\left. \frac{\partial }{\partial \psi^k }\right|_p \right)
  \end{align*}
  したがって、そのvector空間$\left\{p\right\} \times T_p \mathcal{M}$はその組$\left\langle \left(p,\left.\frac{\partial }{\partial \psi^i } \right|_p \right)\right\rangle_{i\in \varLambda_n }$によって張られている。\par
  よって、そのvector空間$\left\{ p\right\} \times T_p \mathcal{M}$の基底としてその組$\left\langle \left(p,\left.\frac{\partial }{\partial \psi^i } \right|_p \right)\right\rangle_{i\in \varLambda_n }$が挙げられる。特に、$\dim \left( \left\{ p\right\} \times T_p \mathcal{M} \right) =n$が成り立つ。
\end{proof}\begin{thm}\label{8.3.3.15}
  $n$次元多様体$\left(\mathcal{M},\mathfrak{O}\right)$が与えられたとき、$\forall p\in \mathcal{M}$に対し、次の写像$\rho $はそのvector空間$\left\{ p\right\} \times T_p \mathcal{M}$からそのvector空間$T_p \mathcal{M}$への自然な線形同型写像である。
  \begin{align*}
    \rho : \left\{ p\right\} \times T_p \mathcal{M} \rightarrow T_p \mathcal{M} ;\left(p,v\right) \mapsto v
  \end{align*}
\end{thm}\par
このことから、しばしば$\left\{p\right\} \times T_p \mathcal{M} =T_p \mathcal{M}$と同一視することもある。このとき、接束$T\mathcal{M}$は次のようになる。
\begin{align*}
  T \mathcal{M} =\bigsqcup_{p\in \mathcal{M}} \left\{p\right\} \times T_p \mathcal{M} =\bigsqcup_{p\in \mathcal{M}} T_p \mathcal{M}
\end{align*}
\begin{proof}
  $n$次元多様体$\left(\mathcal{M},\mathfrak{O}\right)$が与えられたとき、$\forall p\in \mathcal{M}$に対し、次の写像$\rho $について、
  \begin{align*}
    \rho : \left\{ p\right\} \times T_p \mathcal{M} \rightarrow T_p \mathcal{M} ;\left(p,v\right) \mapsto v
  \end{align*}
  $\forall \xi ,\eta \in \mathbb{R} \forall \left(p,v\right),\left(p,w\right) \in \left\{ p\right\} \times T_p \mathcal{M}$に対し、次のようになる。
  \begin{align*}
    \rho \left(\xi \left(p,v\right) +\eta \left(p,w\right)\right) &= \rho \left(p,\xi v+\eta w\right) =\xi v+\eta w =\xi \rho \left(p,v\right)+\eta \rho \left(p,w\right)
  \end{align*}
  ゆえに、その写像$\rho $はそのvector空間$\left\{ p\right\} \times T_p \mathcal{M}$からそのvector空間$T_p \mathcal{M}$への線形写像である。\par
  定理\ref{8.3.3.10}、定理\ref{8.3.3.14}より$\dim \left(\left\{ p\right\} \times T_p \mathcal{M} \right) =\dim T_p \mathcal{M} =n$が成り立つので、その線形写像$\rho $が線形同型写像であることをいうにはその写像$\rho $が単射であることを示せばよいことになる。$\forall \left(p,v\right),\left(p,w\right)\in \left\{ p\right\} \times T_p \mathcal{M} $に対し、$\left(p,v\right)\ne \left(p,w\right)$が成り立つなら、$v\ne w$が成り立つので、$\rho \left(p,v\right) \ne \rho \left(p,w\right)$が成り立つ。これにより、その写像$\rho $は単射であるので、その写像$\rho $はそのvector空間$\left\{ p\right\} \times T_p \mathcal{M}$からそのvector空間$T_p \mathcal{M}$への線形同型写像である。\par
  その線形同型写像$\rho $が自然な線形同型写像であることは定義よりただちにわかる。
\end{proof}
\begin{dfn}
  $n$次元多様体$\left(\mathcal{M},\mathfrak{O}\right)$が与えられたとき、次のような写像$\pi $をその接束$T\mathcal{M}$からその多様体$\left(\mathcal{M},\mathfrak{O}\right)$への射影という。
  \begin{align*}
    \pi :T\mathcal{M} \rightarrow \mathcal{M} ;\left(p,v\right) \mapsto p
  \end{align*}
\end{dfn}
\begin{thm}\label{8.3.3.16}
  $n$次元多様体$\left(\mathcal{M},\mathfrak{O}\right)$が与えられたとき、その接束$T\mathcal{M}$からその多様体$\left(\mathcal{M},\mathfrak{O}\right)$への射影$\pi $によるその多様体$\left(\mathcal{M},\mathfrak{O}\right)$からの誘導位相空間$\left(T\mathcal{M},\mathfrak{O}_\mathrm{bund}\right)$は$2n$次元$C^\infty $級多様体となる。
\end{thm}
\begin{proof}
  $n$次元多様体$\left(\mathcal{M},\mathfrak{O}\right)$が与えられたとき、その接束$T\mathcal{M}$からその多様体$\left(\mathcal{M},\mathfrak{O}\right)$への射影$\pi $によるその多様体$\left(\mathcal{M},\mathfrak{O}\right)$からの誘導位相空間$\left(T\mathcal{M},\mathfrak{O}_\mathrm{bund}\right)$について、その多様体$\left(\mathcal{M},\mathfrak{O}\right)$の$C^\infty$級座標近傍系$\left\{ \left(U_\alpha ,\psi_\alpha \right)\right\}_{\alpha \in A}$を用いれば、$\mathcal{M}=\bigcup_{\alpha \in A} U_\alpha $が成り立つ。そこで、次のようになり、
  \begin{align*}
    T\mathcal{M} &=V\left(\pi^{-1}\right) =V\left(\pi^{-1}|\mathcal{M}\right) \\
    &=V\left(\pi^{-1}|\bigcup_{\alpha \in A} U_\alpha \right) =\bigcup_{\alpha \in A} V\left(\pi^{-1} |U_\alpha \right)
  \end{align*}
  その写像$\pi $が連続で、$\forall \alpha \in A$に対し、$U_\alpha \in \mathfrak{O}$が成り立つことから、$V\left(\pi^{-1} |U_\alpha \right) \in \mathfrak{O}_\mathrm{bund}$が成り立つので、その族$\left\{ V\left(\pi^{-1} |U_\alpha \right)\right\}_{\alpha \in A}$はその集合$\mathcal{M}$の開被覆である。そこで、$\psi_\alpha =\left(\psi_\alpha^i \right)_{i\in \varLambda_n }$とおくと、$\forall p\in U_\alpha \forall v\in T_p \mathcal{M}$に対し、定理\ref{8.3.3.10}よりその接vector空間$T_p \mathcal{M}$における自然標構$\left\langle \left. \frac{\partial}{\partial \psi_\alpha^i }\right|_p \right\rangle_{i\in \varLambda_n }$を用いて、$v=v\left(\psi_\alpha^k \right) \left. \frac{\partial}{\partial \psi_\alpha^k }\right|_p $が成り立つ。その自然標構$\left\langle \left. \frac{\partial}{\partial \psi_\alpha^i }\right|_p \right\rangle_{i\in \varLambda_n }$に関する基底変換における線形同型写像$\varphi : \mathbb{R}^n \rightarrow T_p \mathcal{M} $を用いれば、そのvector空間$T_p \mathcal{M}$はその$n$次元数空間$\mathbb{R}^n $と線形同型である。\par
  そこで、次のような写像$\rho $が定義されれば、
  \begin{align*}
    \rho : \left\{ p\right\} \times T_p \mathcal{M} \rightarrow T_p \mathcal{M} ;\left(p,v\right) \mapsto v
  \end{align*}
  次のようにして定義される写像$\widetilde{\psi }_\alpha $は同相写像となる。
  \begin{align*}
    \widetilde{\psi }_\alpha =\left( \begin{matrix}
      \psi_\alpha \circ \pi \\
      \varphi^{-1} \circ \rho 
    \end{matrix} \right) : V\left(\pi^{-1} |U_\alpha \right) \rightarrow V\left(\psi_\alpha |U_\alpha \right) \times \mathbb{R}^n ; \left(p,v\right) \mapsto \left( \begin{matrix}
      \psi_\alpha \left(p\right) \\
      \varphi^{-1} \left(v\right) 
    \end{matrix} \right)
  \end{align*}
  実際、次のような写像$\widetilde{\psi}_\alpha'$が考えられれば、
  \begin{align*}
    \widetilde{\psi }_\alpha' : V\left(\psi_\alpha |U_\alpha \right) \times \mathbb{R}^n \rightarrow V\left(\pi^{-1} |U_\alpha \right) ; \left( \begin{matrix}
      \mathbf{p} \\
      \mathbf{q}
    \end{matrix} \right) \mapsto \left(\psi_\alpha^{-1} \left(\mathbf{p}\right),\varphi \left(\mathbf{q}\right)\right)
  \end{align*}
  $\forall \left(p,v\right)\in T \mathcal{M}$に対し、次のようになるかつ、
  \begin{align*}
    \widetilde{\psi }_\alpha' \circ \widetilde{\psi }_\alpha \left(p,v\right) = \widetilde{\psi }_\alpha' \left( \begin{matrix}
      \psi_\alpha \left(p\right) \\
      \varphi^{-1} \left(v\right) 
    \end{matrix} \right) =\left(\psi_\alpha^{-1} \left(\psi_\alpha \left(p\right) \right),\varphi \left(\varphi^{-1} \left(v\right) \right)\right) =\left(p,v\right) 
  \end{align*}
  $\forall \left( \begin{matrix}
    \mathbf{p} \\
    \mathbf{q}
  \end{matrix} \right) \in V\left(\psi_\alpha |U_\alpha \right) \times \mathbb{R}^n $に対し、次のようになるので、
  \begin{align*}
    \widetilde{\psi }_\alpha \circ \widetilde{\psi }_\alpha' \left( \begin{matrix}
      \mathbf{p} \\
      \mathbf{q}
    \end{matrix} \right) = \widetilde{\psi }_\alpha \left(\psi_\alpha^{-1} \left(\mathbf{p}\right),\varphi \left(\mathbf{q}\right)\right) =\left( \begin{matrix}
      \psi_\alpha \left(\psi_\alpha^{-1} \left(\mathbf{p}\right)\right) \\
      \varphi^{-1} \left(\varphi \left(\mathbf{q}\right)\right) 
    \end{matrix} \right) =\left( \begin{matrix}
      \mathbf{p} \\
      \mathbf{q}
    \end{matrix} \right) 
  \end{align*}
  $\widetilde{\psi}_\alpha' =\widetilde{\psi}_\alpha^{-1} $が成り立つ。ゆえに、その写像$\widetilde{\psi}_\alpha $は全単射である。さらに、その誘導位相空間$\left(T\mathcal{M},\mathfrak{O}_\mathrm{bund} \right)$の部分位相空間$\left(V\left(\pi^{-1} | U_\alpha \right) ,\left(\mathfrak{O}_\mathrm{bund}\right)_{V\left(\pi^{-1} |U_\alpha \right)}\right)$、その$2n$次元Euclid空間における位相空間$\left(\mathbb{R}^{2n} ,\mathfrak{O}_{d_{E^{2n}}} \right)$の部分位相空間$\left(V\left(\psi_\alpha |U_\alpha \right)\times\mathbb{R}^n ,\left(\mathfrak{O}_{d_{E^{2n}}} \right)_{V\left(\psi_\alpha |U_\alpha \right)\times\mathbb{R}^n} \right)$において、$\forall U\in \left(\mathfrak{O}_\mathrm{bund}\right)_{V\left(\pi^{-1} |U_\alpha \right)} \forall \left(p,v\right) \in U$に対し、$p\in U_\alpha $より$\psi_\alpha \left(p\right) \in V\left(\psi_\alpha |U_\alpha \right)$が成り立つかつ、$\varphi^{-1} \left(v\right) \in \mathbb{R}^n $が成り立つので、$\widetilde{\psi}_\alpha \left(p,v\right) \in V\left(\psi_\alpha |U_\alpha \right) \times \mathbb{R}^n $が成り立つ。さらに、
  
  
  $\forall E\in \left(\mathfrak{O}_{d_{E^{2n}}} \right)_{V\left(\psi_\alpha |U_\alpha \right)\times\mathbb{R}^n} $に対し、


\end{proof}
\begin{thebibliography}{50}
\bibitem{1}
  松島与三, 多様体入門, 裳華房, 1965. 第36刷 p24-31 ISBN978-4-7853-1305-0
\bibitem{2}
  新井朝雄, 相対性理論の数理, 日本評論会, 2021. 第1版第1刷 p164-166, 175-183 ISBN978-4-535-78928-9
\bibitem{3}
  志甫淳, 層とホモロジー代数, 共立出版, 2016. 初版2刷 p277-303 ISBN978-4-320-11160-8
\end{thebibliography}
\end{document}
