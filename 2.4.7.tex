\documentclass[dvipdfmx]{jsarticle}
\setcounter{section}{4}
\setcounter{subsection}{6}
\usepackage{xr}
\externaldocument{2.4.5}
\usepackage{amsmath,amsfonts,amssymb,array,comment,mathtools,url,docmute}
\usepackage{longtable,booktabs,dcolumn,tabularx,mathtools,multirow,colortbl,xcolor}
\usepackage[dvipdfmx]{graphics}
\usepackage{bmpsize}
\usepackage{amsthm}
\usepackage{enumitem}
\setlistdepth{20}
\renewlist{itemize}{itemize}{20}
\setlist[itemize]{label=•}
\renewlist{enumerate}{enumerate}{20}
\setlist[enumerate]{label=\arabic*.}
\setcounter{MaxMatrixCols}{20}
\setcounter{tocdepth}{3}
\newcommand{\rotin}{\text{\rotatebox[origin=c]{90}{$\in $}}}
\newcommand{\amap}[6]{\text{\raisebox{-0.7cm}{\begin{tikzpicture} 
  \node (a) at (0, 1) {$\textstyle{#2}$};
  \node (b) at (#6, 1) {$\textstyle{#3}$};
  \node (c) at (0, 0) {$\textstyle{#4}$};
  \node (d) at (#6, 0) {$\textstyle{#5}$};
  \node (x) at (0, 0.5) {$\rotin $};
  \node (x) at (#6, 0.5) {$\rotin $};
  \draw[->] (a) to node[xshift=0pt, yshift=7pt] {$\textstyle{\scriptstyle{#1}}$} (b);
  \draw[|->] (c) to node[xshift=0pt, yshift=7pt] {$\textstyle{\scriptstyle{#1}}$} (d);
\end{tikzpicture}}}}
\newcommand{\twomaps}[9]{\text{\raisebox{-0.7cm}{\begin{tikzpicture} 
  \node (a) at (0, 1) {$\textstyle{#3}$};
  \node (b) at (#9, 1) {$\textstyle{#4}$};
  \node (c) at (#9+#9, 1) {$\textstyle{#5}$};
  \node (d) at (0, 0) {$\textstyle{#6}$};
  \node (e) at (#9, 0) {$\textstyle{#7}$};
  \node (f) at (#9+#9, 0) {$\textstyle{#8}$};
  \node (x) at (0, 0.5) {$\rotin $};
  \node (x) at (#9, 0.5) {$\rotin $};
  \node (x) at (#9+#9, 0.5) {$\rotin $};
  \draw[->] (a) to node[xshift=0pt, yshift=7pt] {$\textstyle{\scriptstyle{#1}}$} (b);
  \draw[|->] (d) to node[xshift=0pt, yshift=7pt] {$\textstyle{\scriptstyle{#2}}$} (e);
  \draw[->] (b) to node[xshift=0pt, yshift=7pt] {$\textstyle{\scriptstyle{#1}}$} (c);
  \draw[|->] (e) to node[xshift=0pt, yshift=7pt] {$\textstyle{\scriptstyle{#2}}$} (f);
\end{tikzpicture}}}}
\renewcommand{\thesection}{第\arabic{section}部}
\renewcommand{\thesubsection}{\arabic{section}.\arabic{subsection}}
\renewcommand{\thesubsubsection}{\arabic{section}.\arabic{subsection}.\arabic{subsubsection}}
\everymath{\displaystyle}
\allowdisplaybreaks[4]
\usepackage{vtable}
\theoremstyle{definition}
\newtheorem{thm}{定理}[subsection]
\newtheorem*{thm*}{定理}
\newtheorem{dfn}{定義}[subsection]
\newtheorem*{dfn*}{定義}
\newtheorem{axs}[dfn]{公理}
\newtheorem*{axs*}{公理}
\renewcommand{\headfont}{\bfseries}
\makeatletter
  \renewcommand{\section}{%
    \@startsection{section}{1}{\z@}%
    {\Cvs}{\Cvs}%
    {\normalfont\huge\headfont\raggedright}}
\makeatother
\makeatletter
  \renewcommand{\subsection}{%
    \@startsection{subsection}{2}{\z@}%
    {0.5\Cvs}{0.5\Cvs}%
    {\normalfont\LARGE\headfont\raggedright}}
\makeatother
\makeatletter
  \renewcommand{\subsubsection}{%
    \@startsection{subsubsection}{3}{\z@}%
    {0.4\Cvs}{0.4\Cvs}%
    {\normalfont\Large\headfont\raggedright}}
\makeatother
\makeatletter
\renewenvironment{proof}[1][\proofname]{\par
  \pushQED{\qed}%
  \normalfont \topsep6\p@\@plus6\p@\relax
  \trivlist
  \item\relax
  {
  #1\@addpunct{.}}\hspace\labelsep\ignorespaces
}{%
  \popQED\endtrivlist\@endpefalse
}
\makeatother
\renewcommand{\proofname}{\textbf{証明}}
\usepackage{tikz,graphics}
\usepackage[dvipdfmx]{hyperref}
\usepackage{pxjahyper}
\hypersetup{
 setpagesize=false,
 bookmarks=true,
 bookmarksdepth=tocdepth,
 bookmarksnumbered=true,
 colorlinks=false,
 pdftitle={},
 pdfsubject={},
 pdfauthor={},
 pdfkeywords={}}
\begin{document}
%\hypertarget{kroneckerux7a4d}{%
\subsection{Kronecker積}%\label{kroneckerux7a4d}}
\subsubsection{Kronecker積}
\begin{dfn}
体$K$上の$A_{mn} \in M_{mn}(K)$、$B_{op} \in M_{op}(K)$なる行列たち$A_{mn}$、$B_{op}$が与えられたとき、$A_{mn} = \left( a_{ij} \right)_{(i,j) \in \varLambda_{m} \times \varLambda_{n}}$、$B_{op} = \left( b_{kl} \right)_{(k,l) \in \varLambda_{o} \times \varLambda_{p}}$と成分表示されれば、次式のように行列たち$A_{mn}$、$B_{op}$の二項演算$\otimes$が定義される。この二項演算$\otimes$をKronecker積という。
\begin{align*}
A_{mn} \otimes B_{op} = \begin{pmatrix}
a_{11}b_{11} & \cdots & a_{11}b_{1p} & \  & a_{1n}b_{11} & \cdots & a_{1n}b_{1p} \\
 \vdots & \ddots & \vdots & \cdots & \vdots & \ddots & \vdots \\
a_{11}b_{o1} & \cdots & a_{11}b_{op} & \  & a_{1n}b_{o1} & \cdots & a_{1n}b_{op} \\
\  & \vdots & \  & \ddots & \  & \vdots & \  \\
a_{m1}b_{11} & \cdots & a_{m1}b_{1p} & \  & a_{mn}b_{11} & \cdots & a_{mn}b_{1p} \\
 \vdots & \ddots & \vdots & \cdots & \vdots & \ddots & \vdots \\
a_{m1}b_{o1} & \cdots & a_{m1}b_{op} & \  & a_{mn}b_{o1} & \cdots & a_{mn}b_{op} \\
\end{pmatrix}
\end{align*}
\end{dfn}
\begin{thm}\label{2.4.7.1}
体$K$上の適切な行列たち$A_{mn}$、$B_{op}$、$C_{qr}$が与えられたとき、$\forall k \in K$に対し、次式が成り立つ。
\begin{align*}
A_{mn} \otimes \left( B_{op} + C_{op} \right) &= A_{mn} \otimes B_{op} + A_{mn} \otimes C_{op}\\
\left( A_{mn} + B_{mn} \right) \otimes C_{op} &= A_{mn} \otimes C_{op} + B_{mn} \otimes C_{op}\\
\left( kA_{mn} \right) \otimes B_{op} &= A_{mn} \otimes \left( kB_{op} \right) = kA_{mn} \otimes B_{op}\\
\left( A_{mn} \otimes B_{op} \right) \otimes C_{qr} &= A_{mn} \otimes \left( B_{op} \otimes C_{qr} \right)
\end{align*}
\end{thm}
\begin{proof}
体$K$上の適切な行列たち$A_{mn}$、$B_{op}$、$C_{qr}$が与えられたとき、$A_{mn} = \left( a_{ij} \right)_{(i,j) \in \varLambda_{m} \times \varLambda_{n}}$、$B_{op} = \left( b_{ij} \right)_{(i,j) \in \varLambda_{o} \times \varLambda_{p}}$、$C_{qr} = \left( c_{ij} \right)_{(i,j) \in \varLambda_{q} \times \varLambda_{r}}$と成分表示されれば、$\forall k \in K$に対し、次のようになる。
\begin{align*}
&\quad A_{mn} \otimes \left( B_{op} + C_{op} \right)\\
&= \begin{pmatrix}
a_{11}\left( b_{11} + c_{11} \right) & \cdots & a_{11}\left( b_{1p} + c_{1p} \right) & \  & a_{1n}\left( b_{11} + c_{11} \right) & \cdots & a_{1n}\left( b_{1p} + c_{1p} \right) \\
 \vdots & \ddots & \vdots & \cdots & \vdots & \ddots & \vdots \\
a_{11}\left( b_{o1} + c_{o1} \right) & \cdots & a_{11}\left( b_{op} + c_{op} \right) & \  & a_{1n}\left( b_{o1} + c_{o1} \right) & \cdots & a_{1n}\left( b_{op} + c_{op} \right) \\
\  & \vdots & \  & \ddots & \  & \vdots & \  \\
a_{m1}\left( b_{11} + c_{11} \right) & \cdots & a_{m1}\left( b_{1p} + c_{1p} \right) & \  & a_{mn}\left( b_{11} + c_{11} \right) & \cdots & a_{mn}\left( b_{1p} + c_{1p} \right) \\
 \vdots & \ddots & \vdots & \cdots & \vdots & \ddots & \vdots \\
a_{m1}\left( b_{o1} + c_{o1} \right) & \cdots & a_{m1}\left( b_{op} + c_{op} \right) & \  & a_{mn}\left( b_{o1} + c_{o1} \right) & \cdots & a_{mn}\left( b_{op} + c_{op} \right) \\
\end{pmatrix}\\
&= \begin{pmatrix}
a_{11}b_{11} + a_{11}c_{11} & \cdots & a_{11}b_{1p} + a_{11}c_{1p} & \  & a_{1n}b_{11} + a_{1n}c_{11} & \cdots & a_{1n}b_{1p} + a_{1n}c_{1p} \\
 \vdots & \ddots & \vdots & \cdots & \vdots & \ddots & \vdots \\
a_{11}b_{o1} + a_{11}c_{o1} & \cdots & a_{11}b_{op} + a_{11}c_{op} & \  & a_{1n}b_{o1} + a_{1n}c_{o1} & \cdots & a_{1n}b_{op} + a_{1n}c_{op} \\
\  & \vdots & \  & \ddots & \  & \vdots & \  \\
a_{m1}b_{11} + a_{m1}c_{11} & \cdots & a_{m1}b_{1p} + a_{m1}c_{1p} & \  & a_{mn}b_{11} + a_{mn}c_{11} & \cdots & a_{mn}b_{1p} + a_{mn}c_{1p} \\
 \vdots & \ddots & \vdots & \cdots & \vdots & \ddots & \vdots \\
a_{m1}b_{o1} + a_{m1}c_{o1} & \cdots & a_{m1}b_{op} + a_{m1}c_{op} & \  & a_{mn}b_{o1} + a_{mn}c_{o1} & \cdots & a_{mn}b_{op} + a_{mn}c_{op} \\
\end{pmatrix}\\
&= \begin{pmatrix}
a_{11}b_{11} & \cdots & a_{11}b_{1p} & \  & a_{1n}b_{11} & \cdots & a_{1n}b_{1p} \\
 \vdots & \ddots & \vdots & \cdots & \vdots & \ddots & \vdots \\
a_{11}b_{o1} & \cdots & a_{11}b_{op} & \  & a_{1n}b_{o1} & \cdots & a_{1n}b_{op} \\
\  & \vdots & \  & \ddots & \  & \vdots & \  \\
a_{m1}b_{11} & \cdots & a_{m1}b_{1p} & \  & a_{mn}b_{11} & \cdots & a_{mn}b_{1p} \\
 \vdots & \ddots & \vdots & \cdots & \vdots & \ddots & \vdots \\
a_{m1}b_{o1} & \cdots & a_{m1}b_{op} & \  & a_{mn}b_{o1} & \cdots & a_{mn}b_{op} \\
\end{pmatrix} + \begin{pmatrix}
a_{11}c_{11} & \cdots & a_{11}c_{1p} & \  & a_{1n}c_{11} & \cdots & a_{1n}c_{1p} \\
 \vdots & \ddots & \vdots & \cdots & \vdots & \ddots & \vdots \\
a_{11}c_{o1} & \cdots & a_{11}c_{op} & \  & a_{1n}c_{o1} & \cdots & a_{1n}c_{op} \\
\  & \vdots & \  & \ddots & \  & \vdots & \  \\
a_{m1}c_{11} & \cdots & a_{m1}c_{1p} & \  & a_{mn}c_{11} & \cdots & a_{mn}c_{1p} \\
 \vdots & \ddots & \vdots & \cdots & \vdots & \ddots & \vdots \\
a_{m1}c_{o1} & \cdots & a_{m1}c_{op} & \  & a_{mn}c_{o1} & \cdots & a_{mn}c_{op} \\
\end{pmatrix}\\
&= A_{mn} \otimes B_{op} + A_{mn} \otimes C_{op}\\
&\quad \left( A_{mn} + B_{mn} \right) \otimes C_{op}\\
&= \begin{pmatrix}
\left( a_{11} + b_{11} \right)c_{11} & \cdots & \left( a_{11} + b_{11} \right)c_{1p} & \  & \left( a_{1n} + b_{1n} \right)c_{11} & \cdots & \left( a_{1n} + b_{1n} \right)c_{1p} \\
 \vdots & \ddots & \vdots & \cdots & \vdots & \ddots & \vdots \\
\left( a_{11} + b_{11} \right)c_{o1} & \cdots & \left( a_{11} + b_{11} \right)c_{op} & \  & \left( a_{1n} + b_{1n} \right)c_{o1} & \cdots & \left( a_{1n} + b_{1n} \right)c_{op} \\
\  & \vdots & \  & \ddots & \  & \vdots & \  \\
\left( a_{m1} + b_{m1} \right)c_{11} & \cdots & \left( a_{m1} + b_{m1} \right)c_{1p} & \  & \left( a_{mn} + b_{mn} \right)c_{11} & \cdots & \left( a_{mn} + b_{mn} \right)c_{1p} \\
 \vdots & \ddots & \vdots & \cdots & \vdots & \ddots & \vdots \\
\left( a_{m1} + b_{m1} \right)c_{o1} & \cdots & \left( a_{m1} + b_{m1} \right)c_{op} & \  & \left( a_{mn} + b_{mn} \right)c_{o1} & \cdots & \left( a_{mn} + b_{mn} \right)c_{op} \\
\end{pmatrix}\\
&= \begin{pmatrix}
a_{11}c_{11} + b_{11}c_{11} & \cdots & a_{11}c_{1p} + b_{11}c_{1p} & \  & a_{1n}c_{11} + b_{1n}c_{11} & \cdots & a_{1n}c_{1p} + b_{1n}c_{1p} \\
 \vdots & \ddots & \vdots & \cdots & \vdots & \ddots & \vdots \\
a_{11}c_{o1} + b_{11}c_{o1} & \cdots & a_{11}c_{op} + b_{11}c_{op} & \  & a_{1n}c_{o1} + b_{1n}c_{o1} & \cdots & a_{1n}c_{op} + b_{1n}c_{op} \\
\  & \vdots & \  & \ddots & \  & \vdots & \  \\
a_{m1}c_{11} + b_{m1}c_{11} & \cdots & a_{m1}c_{1p} + b_{m1}c_{1p} & \  & a_{mn}c_{11} + b_{mn}c_{11} & \cdots & a_{mn}c_{1p} + b_{mn}c_{1p} \\
 \vdots & \ddots & \vdots & \cdots & \vdots & \ddots & \vdots \\
a_{m1}c_{o1} + b_{m1}c_{o1} & \cdots & a_{m1}c_{op} + b_{m1}c_{op} & \  & a_{mn}c_{o1} + b_{mn}c_{o1} & \cdots & a_{mn}c_{op} + b_{mn}c_{op} \\
\end{pmatrix}\\
&= \begin{pmatrix}
a_{11}c_{11} & \cdots & a_{11}c_{1p} & \  & a_{1n}c_{11} & \cdots & a_{1n}c_{1p} \\
 \vdots & \ddots & \vdots & \cdots & \vdots & \ddots & \vdots \\
a_{11}c_{o1} & \cdots & a_{11}c_{op} & \  & a_{1n}c_{o1} & \cdots & a_{1n}c_{op} \\
\  & \vdots & \  & \ddots & \  & \vdots & \  \\
a_{m1}c_{11} & \cdots & a_{m1}c_{1p} & \  & a_{mn}c_{11} & \cdots & a_{mn}c_{1p} \\
 \vdots & \ddots & \vdots & \cdots & \vdots & \ddots & \vdots \\
a_{m1}c_{o1} & \cdots & a_{m1}c_{op} & \  & a_{mn}c_{o1} & \cdots & a_{mn}c_{op} \\
\end{pmatrix} + \begin{pmatrix}
b_{11}c_{11} & \cdots & b_{11}c_{1p} & \  & b_{1n}c_{11} & \cdots & b_{1n}c_{1p} \\
 \vdots & \ddots & \vdots & \cdots & \vdots & \ddots & \vdots \\
b_{11}c_{o1} & \cdots & b_{11}c_{op} & \  & b_{1n}c_{o1} & \cdots & b_{1n}c_{op} \\
\  & \vdots & \  & \ddots & \  & \vdots & \  \\
b_{m1}c_{11} & \cdots & b_{m1}c_{1p} & \  & b_{mn}c_{11} & \cdots & b_{mn}c_{1p} \\
 \vdots & \ddots & \vdots & \cdots & \vdots & \ddots & \vdots \\
b_{m1}c_{o1} & \cdots & b_{m1}c_{op} & \  & b_{mn}c_{o1} & \cdots & b_{mn}c_{op} \\
\end{pmatrix}\\
&= A_{mn} \otimes C_{op} + B_{mn} \otimes C_{op}\\
&\quad \left( kA_{mn} \right) \otimes B_{op}\\
&= \begin{pmatrix}
ka_{11}b_{11} & \cdots & ka_{11}b_{1p} & \  & ka_{1n}b_{11} & \cdots & ka_{1n}b_{1p} \\
 \vdots & \ddots & \vdots & \cdots & \vdots & \ddots & \vdots \\
ka_{11}b_{o1} & \cdots & ka_{11}b_{op} & \  & ka_{1n}b_{o1} & \cdots & ka_{1n}b_{op} \\
\  & \vdots & \  & \ddots & \  & \vdots & \  \\
ka_{m1}b_{11} & \cdots & ka_{m1}b_{1p} & \  & ka_{mn}b_{11} & \cdots & ka_{mn}b_{1p} \\
 \vdots & \ddots & \vdots & \cdots & \vdots & \ddots & \vdots \\
ka_{m1}b_{o1} & \cdots & ka_{m1}b_{op} & \  & ka_{mn}b_{o1} & \cdots & ka_{mn}b_{op} \\
\end{pmatrix}\\
&= k\begin{pmatrix}
a_{11}b_{11} & \cdots & a_{11}b_{1p} & \  & a_{1n}b_{11} & \cdots & a_{1n}b_{1p} \\
 \vdots & \ddots & \vdots & \cdots & \vdots & \ddots & \vdots \\
a_{11}b_{o1} & \cdots & a_{11}b_{op} & \  & a_{1n}b_{o1} & \cdots & a_{1n}b_{op} \\
\  & \vdots & \  & \ddots & \  & \vdots & \  \\
a_{m1}b_{11} & \cdots & a_{m1}b_{1p} & \  & a_{mn}b_{11} & \cdots & a_{mn}b_{1p} \\
 \vdots & \ddots & \vdots & \cdots & \vdots & \ddots & \vdots \\
a_{m1}b_{o1} & \cdots & a_{m1}b_{op} & \  & a_{mn}b_{o1} & \cdots & a_{mn}b_{op} \\
\end{pmatrix}\\
&= kA_{mn} \otimes B_{op}\\
&\quad A_{mn} \otimes \left( kB_{op} \right)\\
&= \begin{pmatrix}
a_{11}kb_{11} & \cdots & a_{11}kb_{1p} & \  & a_{1n}kb_{11} & \cdots & a_{1n}kb_{1p} \\
 \vdots & \ddots & \vdots & \cdots & \vdots & \ddots & \vdots \\
a_{11}kb_{o1} & \cdots & a_{11}kb_{op} & \  & a_{1n}kb_{o1} & \cdots & a_{1n}kb_{op} \\
\  & \vdots & \  & \ddots & \  & \vdots & \  \\
a_{m1}kb_{11} & \cdots & a_{m1}kb_{1p} & \  & a_{mn}kb_{11} & \cdots & a_{mn}kb_{1p} \\
 \vdots & \ddots & \vdots & \cdots & \vdots & \ddots & \vdots \\
a_{m1}kb_{o1} & \cdots & a_{m1}kb_{op} & \  & a_{mn}kb_{o1} & \cdots & a_{mn}kb_{op} \\
\end{pmatrix}\\
&= k\begin{pmatrix}
a_{11}b_{11} & \cdots & a_{11}b_{1p} & \  & a_{1n}b_{11} & \cdots & a_{1n}b_{1p} \\
 \vdots & \ddots & \vdots & \cdots & \vdots & \ddots & \vdots \\
a_{11}b_{o1} & \cdots & a_{11}b_{op} & \  & a_{1n}b_{o1} & \cdots & a_{1n}b_{op} \\
\  & \vdots & \  & \ddots & \  & \vdots & \  \\
a_{m1}b_{11} & \cdots & a_{m1}b_{1p} & \  & a_{mn}b_{11} & \cdots & a_{mn}b_{1p} \\
 \vdots & \ddots & \vdots & \cdots & \vdots & \ddots & \vdots \\
a_{m1}b_{o1} & \cdots & a_{m1}b_{op} & \  & a_{mn}b_{o1} & \cdots & a_{mn}b_{op} \\
\end{pmatrix}\\
&= kA_{mn} \otimes B_{op}\\
&\quad \left( A_{mn} \otimes B_{op} \right) \otimes C_{qr}\\
&= \begin{pmatrix}
a_{11}b_{11} & \cdots & a_{11}b_{1p} & \  & a_{1n}b_{11} & \cdots & a_{1n}b_{1p} \\
 \vdots & \ddots & \vdots & \cdots & \vdots & \ddots & \vdots \\
a_{11}b_{o1} & \cdots & a_{11}b_{op} & \  & a_{1n}b_{o1} & \cdots & a_{1n}b_{op} \\
\  & \vdots & \  & \ddots & \  & \vdots & \  \\
a_{m1}b_{11} & \cdots & a_{m1}b_{1p} & \  & a_{mn}b_{11} & \cdots & a_{mn}b_{1p} \\
 \vdots & \ddots & \vdots & \cdots & \vdots & \ddots & \vdots \\
a_{m1}b_{o1} & \cdots & a_{m1}b_{op} & \  & a_{mn}b_{o1} & \cdots & a_{mn}b_{op} \\
\end{pmatrix} \otimes \begin{pmatrix}
c_{11} & \cdots & c_{1r} \\
 \vdots & \ddots & \vdots \\
c_{q1} & \cdots & c_{qr} \\
\end{pmatrix}\\
&= \left( \begin{matrix}
a_{11}b_{11}c_{11} & \cdots & a_{11}b_{11}c_{1r} & \cdots & a_{11}b_{1p}c_{11} & \cdots & a_{11}b_{1p}c_{1r} & \cdots \\
 \vdots & \ddots & \vdots & \ddots & \vdots & \ddots & \vdots & \ddots \\
a_{11}b_{11}c_{q1} & \cdots & a_{11}b_{11}c_{qr} & \cdots & a_{11}b_{1p}c_{q1} & \cdots & a_{11}b_{1p} & \cdots \\
 \vdots & \ddots & \vdots & \ddots & \vdots & \ddots & \vdots & \ddots \\
a_{11}b_{o1}c_{11} & \cdots & a_{11}b_{o1}c_{1r} & \cdots & a_{11}b_{op}c_{11} & \cdots & a_{11}b_{op}c_{1r} & \cdots \\
 \vdots & \ddots & \vdots & \ddots & \vdots & \ddots & \vdots & \ddots \\
a_{11}b_{o1}c_{q1} & \cdots & a_{11}b_{o1}c_{qr} & \cdots & a_{11}b_{op}c_{q1} & \cdots & a_{11}b_{op}c_{qr} & \cdots \\
 \vdots & \ddots & \vdots & \ddots & \vdots & \ddots & \vdots & \ddots \\
a_{m1}b_{11}c_{11} & \cdots & a_{m1}b_{11}c_{1r} & \cdots & a_{m1}b_{1p}c_{11} & \cdots & a_{m1}b_{1p}c_{1r} & \cdots \\
 \vdots & \ddots & \vdots & \ddots & \vdots & \ddots & \vdots & \ddots \\
a_{m1}b_{11}c_{q1} & \cdots & a_{m1}b_{11}c_{qr} & \cdots & a_{m1}b_{1p}c_{q1} & \cdots & a_{m1}b_{1p}c_{qr} & \cdots \\
 \vdots & \ddots & \vdots & \ddots & \vdots & \ddots & \vdots & \ddots \\
a_{m1}b_{o1}c_{11} & \cdots & a_{m1}b_{o1}c_{1r} & \cdots & a_{m1}b_{op}c_{11} & \cdots & a_{m1}b_{op}c_{1r} & \cdots \\
 \vdots & \ddots & \vdots & \ddots & \vdots & \ddots & \vdots & \ddots \\
a_{m1}b_{o1}c_{q1} & \cdots & a_{m1}b_{o1}c_{qr} & \cdots & a_{m1}b_{op}c_{q1} & \cdots & a_{m1}b_{op}c_{qr} & \cdots \\
\end{matrix} \right.\\
&\quad  \left. \begin{matrix}
a_{1n}b_{11}c_{11} & \cdots & a_{1n}b_{11}c_{1r} & \cdots & a_{1n}b_{1p}c_{11} & \cdots & a_{1n}b_{1p}c_{1r} \\
 \vdots & \ddots & \vdots & \ddots & \vdots & \ddots & \vdots \\
a_{1n}b_{11}c_{q1} & \cdots & a_{1n}b_{11}c_{qr} & \cdots & a_{1n}b_{1p}c_{q1} & \cdots & a_{1n}b_{1p}c_{qr} \\
 \vdots & \ddots & \vdots & \ddots & \vdots & \ddots & \vdots \\
a_{1n}b_{o1}c_{11} & \cdots & a_{1n}b_{o1}c_{1r} & \cdots & a_{1n}b_{op}c_{11} & \cdots & a_{1n}b_{op}c_{1r} \\
 \vdots & \ddots & \vdots & \ddots & \vdots & \ddots & \vdots \\
a_{1n}b_{o1}c_{q1} & \cdots & a_{1n}b_{o1}c_{qr} & \cdots & a_{1n}b_{op}c_{q1} & \cdots & a_{1n}b_{op}c_{qr} \\
 \vdots & \ddots & \vdots & \ddots & \vdots & \ddots & \vdots \\
a_{mn}b_{11}c_{11} & \cdots & a_{mn}b_{11}c_{1r} & \cdots & a_{mn}b_{1p}c_{11} & \cdots & a_{mn}b_{1p}c_{1r} \\
 \vdots & \ddots & \vdots & \ddots & \vdots & \ddots & \vdots \\
a_{mn}b_{11}c_{q1} & \cdots & a_{mn}b_{11}c_{qr} & \cdots & a_{mn}b_{1p}c_{q1} & \cdots & a_{mn}b_{1p}c_{qr} \\
 \vdots & \ddots & \vdots & \ddots & \vdots & \ddots & \vdots \\
a_{mn}b_{o1}c_{11} & \cdots & a_{mn}b_{o1}c_{1r} & \cdots & a_{mn}b_{op}c_{11} & \cdots & a_{mn}b_{op}c_{1r} \\
 \vdots & \ddots & \vdots & \ddots & \vdots & \ddots & \vdots \\
a_{mn}b_{o1}c_{q1} & \cdots & a_{mn}b_{o1}c_{qr} & \cdots & a_{mn}b_{op}c_{q1} & \cdots & a_{mn}b_{op}c_{qr} \\
\end{matrix} \right)\\
&= \begin{pmatrix}
a_{11} & \cdots & a_{1n} \\
 \vdots & \ddots & \vdots \\
a_{m1} & \cdots & a_{mn} \\
\end{pmatrix} \otimes \begin{pmatrix}
b_{11}c_{11} & \cdots & b_{11}c_{1r} & \  & b_{1p}c_{11} & \cdots & b_{1p}c_{1r} \\
 \vdots & \ddots & \vdots & \cdots & \vdots & \ddots & \vdots \\
b_{11}c_{q1} & \cdots & b_{11}c_{qr} & \  & b_{1p}c_{q1} & \cdots & b_{1p}c_{qr} \\
\  & \vdots & \  & \ddots & \  & \vdots & \  \\
b_{o1}c_{11} & \cdots & b_{o1}c_{1r} & \  & b_{op}c_{11} & \cdots & b_{op}c_{1r} \\
 \vdots & \ddots & \vdots & \cdots & \vdots & \ddots & \vdots \\
b_{o1}c_{q1} & \cdots & b_{o1}c_{qr} & \  & b_{op}c_{q1} & \cdots & b_{op}c_{qr} \\
\end{pmatrix}\\
&= A_{mn} \otimes \left( B_{op} \otimes C_{qr} \right)
\end{align*}
\end{proof}
\begin{dfn}
体$K$上の$A_{i} \in M_{m_{i}n_{i}}(K)$なる行列たち$A_{i}$が与えられたとき、次式のようにして$\bigotimes_{i \in \varLambda_{n}} A_{i} \in M_{\prod_{i \in \varLambda_{n}} m_{i}\prod_{i \in \varLambda_{n}} n_{i}}(K)$なる行列$\bigotimes_{i \in \varLambda_{n}} A_{i}$が定義される。その組$\left( A_{i} \right)_{i \in \varLambda_{n}}$からその行列$\bigotimes_{i \in \varLambda_{n}} A_{i}$へ写す写像を一般化されたKronecker積、または単に、Kronecker積ということにする。
\begin{align*}
\bigotimes_{i \in \varLambda_{n}} A_{i} = \left\{ \begin{matrix}
A_{1} & \mathrm{if} & n = 1 \\
\bigotimes_{i \in \varLambda_{n - 1}} A_{i} \otimes A_{n} & \mathrm{if} & 2 \leq n \\
\end{matrix} \right.\ 
\end{align*}
\end{dfn}
%\hypertarget{tensorux7a4dux3068kroneckerux7a4d}{%
\subsubsection{tensor積とKronecker積}%\label{tensorux7a4dux3068kroneckerux7a4d}}
\begin{thm}\label{2.4.7.2}
体$K$上の$m$次元vector空間$V$、$n$次元vector空間$W$、そのvector空間$V$の基底たち$\left\langle \mathbf{t}_{i} \right\rangle_{i \in \varLambda_{m}}$、$\left\langle \mathbf{v}_{i} \right\rangle_{i \in \varLambda_{m}}$、そのvector空間$W$の基底たち$\left\langle \mathbf{u}_{j} \right\rangle_{j \in \varLambda_{n}}$、$\left\langle \mathbf{w}_{j} \right\rangle_{j \in \varLambda_{n}}$が与えられたとき、これらの基底たちをそれぞれ$\alpha$、$\gamma$、$\beta$、$\delta$とおかれると、定理\ref{2.4.5.2}よりそのvector空間$V \otimes W$の基底として$\alpha \otimes \beta$、$\gamma \otimes \delta$があげられるのであった。このとき、そのvector空間$V \otimes W$のそれらの基底たち$\alpha \otimes \beta$、$\gamma \otimes \delta$に関する基底変換行列$\left[ I_{V \otimes W} \right]^{\gamma \otimes \delta}_{\alpha \otimes \beta}$は次式を満たす\footnote{なお、vector$\left( a_{ij} \right)_{(i,j) \in \varLambda_{m} \times \varLambda_{n}}$は$\begin{pmatrix}
  a_{11} & a_{12} & \cdots & a_{1n} \\
  a_{21} & a_{22} & \cdots & a_{2n} \\
   \vdots & \vdots & \ddots & \vdots \\
  a_{m1} & a_{m2} & \cdots & a_{mn} \\
  \end{pmatrix} = \begin{pmatrix}
  a_{11} \\
  a_{12} \\
   \vdots \\
  a_{1n} \\
  a_{21} \\
  a_{22} \\
   \vdots \\
  a_{2n} \\
   \vdots \\
  a_{m1} \\
  a_{m2} \\
   \vdots \\
  a_{mn} \\
  \end{pmatrix}$とみなしていることに注意されたい。もちろん、これは2つの添字の並べ方の問題なので、これが気に入らなければ各自好きな並べ方で議論されてもよいと思う。}。
\begin{align*}
\left[ I_{V \otimes W} \right]^{\gamma \otimes \delta}_{\alpha \otimes \beta} = \left[ I_{V} \right]^{\gamma}_{\alpha} \otimes \left[ I_{W} \right]^{\delta}_{\beta}
\end{align*}
\end{thm}
\begin{proof}
体$K$上の$m$次元vector空間$V$、$n$次元vector空間$W$、そのvector空間$V$の基底たち$\left\langle \mathbf{t}_{i} \right\rangle_{i \in \varLambda_{m}}$、$\left\langle \mathbf{v}_{i} \right\rangle_{i \in \varLambda_{m}}$、そのvector空間$W$の基底たち$\left\langle \mathbf{u}_{j} \right\rangle_{j \in \varLambda_{n}}$、$\left\langle \mathbf{w}_{j} \right\rangle_{j \in \varLambda_{n}}$が与えられたとき、これらの基底たちをそれぞれ$\alpha$、$\gamma$、$\beta$、$\delta$とおかれると、定理\ref{2.4.5.2}よりそのvector空間$V \otimes W$の基底として$\alpha \otimes \beta$、$\gamma \otimes \delta$があげられるのであった。このとき、そのvector空間$V \otimes W$のそれらの基底たち$\alpha \otimes \beta$、$\gamma \otimes \delta$に関する基底変換行列$\left[ I_{V \otimes W} \right]^{\gamma \otimes \delta}_{\alpha \otimes \beta}$について、vector空間たち$K^{m}$、$K^{n}$、$K^{mn}$の標準直交基底たちが$\left\langle \mathbf{d}_{i} \right\rangle_{i \in \varLambda_{m}}$、$\left\langle \mathbf{e}_{j} \right\rangle_{j \in \varLambda_{n}}$、$\left\langle \mathbf{f}_{ij} \right\rangle_{(i,j) \in \varLambda_{m} \times \varLambda_{n}}$、それらの基底たち$\alpha \otimes \beta$、$\gamma \otimes \delta$に関する基底変換における線形同型写像たちが$\varphi_{\alpha \otimes \beta}$、$\varphi_{\gamma \otimes \delta}$、それらの基底たち$\alpha$、$\gamma$に関する基底変換行列$\left[ I_{V} \right]^{\gamma}_{\alpha}$、それらの基底たち$\beta$、$\delta$に関する基底変換行列$\left[ I_{W} \right]^{\delta}_{\beta}$がそれぞれ$\left( a_{ij} \right)_{(i,j) \in \varLambda_{m}^{2}}$、$\left( b_{ij} \right)_{(i,j) \in \varLambda_{n}^{2}}$とおかれると、次のようになる。
\begin{align*}
\varphi_{\gamma \otimes \delta}^{- 1} \circ I_{V \otimes W} \circ \varphi_{\alpha \otimes \beta}\left( \mathbf{f}_{ij} \right) &= \varphi_{\gamma \otimes \delta}^{- 1}\left( \varphi_{\alpha \otimes \beta}\left( \mathbf{f}_{ij} \right) \right)\\
&= \varphi_{\gamma \otimes \delta}^{- 1}\left( \mathbf{t}_{i} \otimes \mathbf{u}_{j} \right)\\
&= \varphi_{\gamma \otimes \delta}^{- 1}\left( \varphi_{\gamma} \circ \varphi_{\gamma}^{- 1} \circ I_{V} \circ \varphi_{\alpha} \circ \varphi_{\alpha}^{- 1}\left( \mathbf{t}_{i} \right) \otimes \varphi_{\delta} \circ \varphi_{\delta}^{- 1} \circ I_{W} \circ \varphi_{\beta} \circ \varphi_{\beta}^{- 1}\left( \mathbf{u}_{j} \right) \right)\\
&= \varphi_{\gamma \otimes \delta}^{- 1}\left( \varphi_{\gamma}\left( \varphi_{\gamma}^{- 1} \circ I_{V} \circ \varphi_{\alpha}\left( \varphi_{\alpha}^{- 1}\left( \mathbf{t}_{i} \right) \right) \right) \otimes \varphi_{\delta}\left( \varphi_{\delta}^{- 1} \circ I_{W} \circ \varphi_{\beta}\left( \varphi_{\beta}^{- 1}\left( \mathbf{u}_{j} \right) \right) \right) \right)\\
&= \varphi_{\gamma \otimes \delta}^{- 1}\left( \varphi_{\gamma}\left( \varphi_{\gamma}^{- 1} \circ I_{V} \circ \varphi_{\alpha}\left( \mathbf{d}_{i} \right) \right) \otimes \varphi_{\delta}\left( \varphi_{\delta}^{- 1} \circ I_{W} \circ \varphi_{\beta}\left( \mathbf{e}_{j} \right) \right) \right)\\
&= \varphi_{\gamma \otimes \delta}^{- 1}\left( \varphi_{\gamma}\left( \left[ I_{V} \right]^{\gamma}_{\alpha}\mathbf{d}_{i} \right) \otimes \varphi_{\delta}\left( \left[ I_{W} \right]^{\delta}_{\beta}\mathbf{e}_{j} \right) \right)\\
&= \varphi_{\gamma \otimes \delta}^{- 1}\left( \varphi_{\gamma}\left( \sum_{k \in \varLambda_{m}} {a_{ki}\mathbf{d}_{k}} \right) \otimes \varphi_{\delta}\left( \sum_{l \in \varLambda_{n}} {b_{lj}\mathbf{e}_{l}} \right) \right)\\
&= \sum_{k \in \varLambda_{m}} {\sum_{l \in \varLambda_{n}} {a_{ki}b_{lj}\varphi_{\gamma \otimes \delta}^{- 1}\left( \varphi_{\gamma}\left( \mathbf{d}_{k} \right) \otimes \varphi_{\delta}\left( \mathbf{e}_{l} \right) \right)}}\\
&= \sum_{k \in \varLambda_{m}} {\sum_{l \in \varLambda_{n}} {a_{ki}b_{lj}\varphi_{\gamma \otimes \delta}^{- 1}\left( \mathbf{v}_{k} \otimes \mathbf{w}_{l} \right)}}\\
&= \sum_{k \in \varLambda_{m}} {\sum_{l \in \varLambda_{n}} {a_{ki}b_{lj}\mathbf{f}_{kl}}}\\
&= \sum_{k \in \varLambda_{m}} {\sum_{l \in \varLambda_{n}} {a_{ki}b_{lj}\left( \delta_{gk}\delta_{hl} \right)_{(g,h) \in \varLambda_{m} \times \varLambda_{n}}}}\\
&= \sum_{k \in \varLambda_{m}} {\sum_{l \in \varLambda_{n}} \begin{pmatrix}
\begin{pmatrix}
0 \\
 \vdots \\
0 \\
 \vdots \\
0 \\
\end{pmatrix} \\
 \vdots \\
\begin{pmatrix}
0 \\
 \vdots \\
a_{ki}b_{lj} \\
 \vdots \\
0 \\
\end{pmatrix} \\
 \vdots \\
\begin{pmatrix}
0 \\
 \vdots \\
0 \\
 \vdots \\
0 \\
\end{pmatrix} \\
\end{pmatrix}}\\
&= \begin{matrix}
\  & \begin{pmatrix}
\begin{pmatrix}
a_{1i}b_{1j} \\
 \vdots \\
0 \\
\end{pmatrix} \\
 \vdots \\
\begin{pmatrix}
0 \\
 \vdots \\
0 \\
\end{pmatrix} \\
\end{pmatrix} & + & \cdots & + & \begin{pmatrix}
\begin{pmatrix}
0 \\
 \vdots \\
a_{1i}b_{nj} \\
\end{pmatrix} \\
 \vdots \\
\begin{pmatrix}
0 \\
 \vdots \\
0 \\
\end{pmatrix} \\
\end{pmatrix} \\
\  & \  & \  & \cdots & \  & \  \\
 + & \begin{pmatrix}
\begin{pmatrix}
0 \\
 \vdots \\
0 \\
\end{pmatrix} \\
 \vdots \\
\begin{pmatrix}
a_{mi}b_{1j} \\
 \vdots \\
0 \\
\end{pmatrix} \\
\end{pmatrix} & + & \cdots & + & \begin{pmatrix}
\begin{pmatrix}
0 \\
 \vdots \\
0 \\
\end{pmatrix} \\
 \vdots \\
\begin{pmatrix}
0 \\
 \vdots \\
a_{mi}b_{nj} \\
\end{pmatrix} \\
\end{pmatrix} \\
\end{matrix}\\
&= \begin{pmatrix}
\begin{pmatrix}
a_{1i}b_{1j} \\
 \vdots \\
a_{1i}b_{nj} \\
\end{pmatrix} \\
 \vdots \\
\begin{pmatrix}
a_{mi}b_{1j} \\
 \vdots \\
a_{mi}b_{nj} \\
\end{pmatrix} \\
\end{pmatrix} = \begin{pmatrix}
a_{1i}b_{1j} \\
 \vdots \\
a_{1i}b_{nj} \\
 \vdots \\
a_{mi}b_{1j} \\
 \vdots \\
a_{mi}b_{nj} \\
\end{pmatrix}
\end{align*}
これにより、したがって、次のようになる。
\begin{align*}
\left[ I_{V \otimes W} \right]^{\gamma \otimes \delta}_{\alpha \otimes \beta} &= \begin{pmatrix}
a_{11}b_{11} & \cdots & a_{11}b_{1n} & \  & a_{1m}b_{11} & \cdots & a_{1m}b_{1n} \\
 \vdots & \ddots & \vdots & \cdots & \vdots & \ddots & \vdots \\
a_{11}b_{n1} & \cdots & a_{11}b_{nn} & \  & a_{1m}b_{n1} & \cdots & a_{1m}b_{nn} \\
\  & \vdots & \  & \ddots & \  & \vdots & \  \\
a_{m1}b_{11} & \cdots & a_{m1}b_{1n} & \  & a_{mm}b_{11} & \cdots & a_{mm}b_{1n} \\
 \vdots & \ddots & \vdots & \cdots & \vdots & \ddots & \vdots \\
a_{m1}b_{n1} & \cdots & a_{m1}b_{nn} & \  & a_{mm}b_{n1} & \cdots & a_{mm}b_{nn} \\
\end{pmatrix}\\
&= \begin{pmatrix}
a_{11} & \cdots & a_{1m} \\
 \vdots & \ddots & \vdots \\
a_{m1} & \cdots & a_{mm} \\
\end{pmatrix} \otimes \begin{pmatrix}
b_{11} & \cdots & b_{1n} \\
 \vdots & \ddots & \vdots \\
b_{n1} & \cdots & b_{nn} \\
\end{pmatrix} = \left[ I_{V} \right]^{\gamma}_{\alpha} \otimes \left[ I_{W} \right]^{\delta}_{\beta}
\end{align*}
\end{proof}
\begin{thm}\label{2.4.7.3}
$n$つの体$K$上$m_{i}$次元vector空間たち$V_{i}$、これらの基底たち$\left\langle \mathbf{v}_{ij_{i}} \right\rangle_{j_{i} \in \varLambda_{m_{i}}}$、$\left\langle \mathbf{w}_{ij_{i}} \right\rangle_{j_{i} \in \varLambda_{m_{i}}}$が与えられたとき、これらの基底たちをそれぞれ$\alpha_{i}$、$\beta_{i}$とおかれると、定理\ref{2.4.5.2}よりそのtensor空間$\bigotimes_{i \in \varLambda_{n}} V_{i}$の基底として$\bigotimes_{i \in \varLambda_{n}} \alpha_{i}$、$\bigotimes_{i \in \varLambda_{n}} \beta_{i}$があげられるのであった。このとき、そのtensor空間$\bigotimes_{i \in \varLambda_{n}} V_{i}$のそれらの基底たち$\bigotimes_{i \in \varLambda_{n}} \alpha_{i}$、$\bigotimes_{i \in \varLambda_{n}} \beta_{i}$に関する基底変換行列$\left[ I_{\bigotimes_{i \in \varLambda_{n}} V_{i}} \right]^{\bigotimes_{i \in \varLambda_{n}} \beta_{i}}_{\bigotimes_{i \in \varLambda_{n}} \alpha_{i}}$は次式を満たす。
\begin{align*}
\left[ I_{\bigotimes_{i \in \varLambda_{n}} V_{i}} \right]^{\bigotimes_{i \in \varLambda_{n}} \beta_{i}}_{\bigotimes_{i \in \varLambda_{n}} \alpha_{i}} = \bigotimes_{i \in \varLambda_{n}} \left[ I_{V_{i}} \right]^{\beta_{i}}_{\alpha_{i}}
\end{align*}
\end{thm}
\begin{proof} 定理\ref{2.4.7.2}と数学的帰納法により直ちにわかる。
\end{proof}
\begin{thm}\label{2.4.7.4}
体$K$上の$m$次元vector空間$T$、$n$次元vector空間$U$、$o$次元vector空間$V$、$p$次元vector空間$W$が与えられたとき、それらのvector空間たち$T$、$U$、$V$、$W$の基底がそれぞれ$\left\langle \mathbf{t}_{i} \right\rangle_{i \in \varLambda_{m}}$、$\left\langle \mathbf{u}_{j} \right\rangle_{j \in \varLambda_{n}}$、$\left\langle \mathbf{v}_{i} \right\rangle_{i \in \varLambda_{o}}$、$\left\langle \mathbf{w}_{j} \right\rangle_{j \in \varLambda_{p}}$と与えられたらば、$\left\langle \mathbf{t}_{i} \right\rangle_{i \in \varLambda_{m}} = \alpha$、$\left\langle \mathbf{u}_{j} \right\rangle_{j \in \varLambda_{n}} = \beta$、$\left\langle \mathbf{v}_{i} \right\rangle_{i \in \varLambda_{o}} = \gamma$、$\left\langle \mathbf{w}_{j} \right\rangle_{j \in \varLambda_{p}} = \delta$とおいて、vector空間$L(T \otimes U,V \otimes W)$と、$\forall\varphi \in L(T,V)\forall\chi \in L(U,W)\forall(i,j) \in \varLambda_{m} \times \varLambda_{n}$に対し、$\rho(\varphi,\chi)\left( \mathbf{t}_{i} \otimes \mathbf{u}_{j} \right) = \varphi\left( \mathbf{t}_{i} \right) \otimes \chi\left( \mathbf{u}_{j} \right)$なる線形写像$\rho(\varphi,\chi):T \otimes U \rightarrow V \otimes W$を用いた次式のような写像$\rho$との組$\left( L(T \otimes U,V \otimes W),\rho \right)$はtensor積であった。
\begin{align*}
\rho:L(T,V) \times L(U,W) \rightarrow L(T \otimes U,V \otimes W);(\varphi,\chi) \mapsto \rho(\varphi,\chi)
\end{align*}
そこで、それらの線形写像たち$\varphi$、$\chi$のそれらの基底たち$\alpha$、$\gamma$、$\beta$、$\delta$に関する表現行列たち$[\varphi]^{\gamma}_{\alpha}$、$[\chi]^{\delta}_{\beta}$とvector空間たち$T \otimes U$、$V \otimes W$の基底たち$\alpha \otimes \beta$、$\gamma \otimes \delta$に関するtensor積$\left( L(T \otimes U,V \otimes W),\rho \right)$で定義される線形写像$\rho(\varphi,\chi):T \otimes U \rightarrow V \otimes W$の表現行列$\left[ \rho(\varphi,\chi) \right]^{\gamma \otimes \delta}_{\alpha \otimes \beta}$について、次式が成り立つ。
\begin{align*}
\left[ \rho(\varphi,\chi) \right]^{\gamma \otimes \delta}_{\alpha \otimes \beta} = [\varphi]^{\gamma}_{\alpha} \otimes [\chi]^{\delta}_{\beta}
\end{align*}
\end{thm}
\begin{proof}
体$K$上の$m$次元vector空間$T$、$n$次元vector空間$U$、$o$次元vector空間$V$、$p$次元vector空間$W$が与えられたとき、それらのvector空間たち$T$、$U$、$V$、$W$の基底がそれぞれ$\left\langle \mathbf{t}_{i} \right\rangle_{i \in \varLambda_{m}}$、$\left\langle \mathbf{u}_{j} \right\rangle_{j \in \varLambda_{n}}$、$\left\langle \mathbf{v}_{i} \right\rangle_{i \in \varLambda_{o}}$、$\left\langle \mathbf{w}_{j} \right\rangle_{j \in \varLambda_{p}}$と与えられたらば、$\left\langle \mathbf{t}_{i} \right\rangle_{i \in \varLambda_{m}} = \alpha$、$\left\langle \mathbf{u}_{j} \right\rangle_{j \in \varLambda_{n}} = \beta$、$\left\langle \mathbf{v}_{i} \right\rangle_{i \in \varLambda_{o}} = \gamma$、$\left\langle \mathbf{w}_{j} \right\rangle_{j \in \varLambda_{p}} = \delta$とおいて、vector空間$L(T \otimes U,V \otimes W)$と、$\forall\varphi \in L(T,V)\forall\chi \in L(U,W)\forall(i,j) \in \varLambda_{m} \times \varLambda_{n}$に対し、$\rho(\varphi,\chi)\left( \mathbf{t}_{i} \otimes \mathbf{u}_{j} \right) = \varphi\left( \mathbf{t}_{i} \right) \otimes \chi\left( \mathbf{u}_{j} \right)$なる線形写像$\rho(\varphi,\chi):T \otimes U \rightarrow V \otimes W$を用いた次式のような写像$\rho$との組$\left( L(T \otimes U,V \otimes W),\rho \right)$はtensor積であった。
\begin{align*}
\rho:L(T,V) \times L(U,W) \rightarrow L(T \otimes U,V \otimes W);(\varphi,\chi) \mapsto \rho(\varphi,\chi)
\end{align*}
そこで、それらの線形写像たち$\varphi$、$\chi$のそれらの基底たち$\alpha$、$\gamma$、$\beta$、$\delta$に関する表現行列たち$[\varphi]^{\gamma}_{\alpha}$、$[\chi]^{\delta}_{\beta}$と、定理\ref{2.4.5.2}よりそれらの組々$\left\langle \mathbf{t}_{i} \otimes \mathbf{u}_{j} \right\rangle_{(i,j) \in \varLambda_{m} \times \varLambda_{n}}$、$\left\langle \mathbf{v}_{i} \otimes \mathbf{w}_{j} \right\rangle_{(i,j) \in \varLambda_{o} \times \varLambda_{p}}$はそれぞれvector空間たち$T \otimes U$、$V \otimes W$の基底となるので、vector空間たち$T \otimes U$、$V \otimes W$の基底たち$\alpha \otimes \beta$、$\gamma \otimes \delta$に関するtensor積$\left( L(T \otimes U,V \otimes W),\rho \right)$で定義される線形写像$\rho(\varphi,\chi):T \otimes U \rightarrow V \otimes W$の表現行列$\left[ \rho(\varphi,\chi) \right]^{\gamma \otimes \delta}_{\alpha \otimes \beta}$について、それらの表現行列たち$[\varphi]^{\gamma}_{\alpha}$、$[\chi]^{\delta}_{\beta}$がそれぞれ$[\varphi]^{\gamma}_{\alpha} = \left( a_{ij} \right)_{(i,j) \in \varLambda_{o} \times \varLambda_{m}}$、$[\chi]^{\delta}_{\beta} = \left( b_{ij} \right)_{(i,j) \in \varLambda_{p} \times \varLambda_{n}}$とおかれれば、それらのvector空間たち$T \otimes U$、$V \otimes W$の基底たち$\alpha \otimes \beta$、$\gamma \otimes \delta$に関するtensor積$\left( L(T \otimes U,V \otimes W),\rho \right)$で定義される線形写像$\rho(\varphi,\chi):T \otimes U \rightarrow V \otimes W$の表現行列$\left[ \rho(\varphi,\chi) \right]^{\gamma \otimes \delta}_{\alpha \otimes \beta}$について、vector空間たち$K^{m}$、$K^{n}$、$K^{o}$、$K^{p}$の標準直交基底たち$\left\langle \mathbf{d}_{k} \right\rangle_{k \in \varLambda_{m}}$、$\left\langle \mathbf{e}_{l} \right\rangle_{l \in \varLambda_{n}}$、$\left\langle \mathbf{f}_{i} \right\rangle_{i \in \varLambda_{o}}$、$\left\langle \mathbf{g}_{j} \right\rangle_{j \in \varLambda_{p}}$、それらの基底たち$\alpha \otimes \beta$、$\gamma \otimes \delta$に関する基底変換における線形同型写像たち$\varphi_{\alpha \otimes \beta}$、$\varphi_{\gamma \otimes \delta}$を用いて次のようになる。
\begin{align*}
\varphi_{\gamma \otimes \delta}^{- 1} \circ \rho(\varphi,\chi) \circ \varphi_{\alpha \otimes \beta}\left( \delta_{ik}\delta_{jl} \right)_{(i,j) \in \varLambda_{m} \times \varLambda_{n}} &= \varphi_{\gamma \otimes \delta}^{- 1}\left( \rho(\varphi,\chi)\left( \varphi_{\alpha \otimes \beta}\left( \delta_{ik}\delta_{jl} \right)_{(i,j) \in \varLambda_{m} \times \varLambda_{n}} \right) \right)\\
&= \varphi_{\gamma \otimes \delta}^{- 1}\left( \rho(\varphi,\chi)\left( \mathbf{t}_{k} \otimes \mathbf{u}_{l} \right) \right)\\
&= \varphi_{\gamma \otimes \delta}^{- 1}\left( \varphi\left( \mathbf{t}_{k} \right) \otimes \chi\left( \mathbf{u}_{l} \right) \right)\\
&= \varphi_{\gamma \otimes \delta}^{- 1}\left( \varphi_{\gamma}^{- 1} \circ \varphi_{\gamma} \circ \varphi \circ \varphi_{\alpha}^{- 1} \circ \varphi_{\alpha}\left( \mathbf{t}_{k} \right) \otimes \varphi_{\delta}^{- 1} \circ \varphi_{\delta} \circ \chi \circ \varphi_{\beta}^{- 1} \circ \varphi_{\beta}\left( \mathbf{u}_{l} \right) \right)\\
&= \varphi_{\gamma \otimes \delta}^{- 1}\left( \varphi_{\gamma}^{- 1}\left( \varphi_{\gamma} \circ \varphi \circ \varphi_{\alpha}^{- 1}\left( \varphi_{\alpha}\left( \mathbf{t}_{k} \right) \right) \right) \otimes \varphi_{\delta}^{- 1}\left( \varphi_{\delta} \circ \chi \circ \varphi_{\beta}^{- 1}\left( \varphi_{\beta}\left( \mathbf{u}_{l} \right) \right) \right) \right)\\
&= \varphi_{\gamma \otimes \delta}^{- 1}\left( \varphi_{\gamma}^{- 1}\left( \varphi_{\gamma} \circ \varphi \circ \varphi_{\alpha}^{- 1}\left( \mathbf{d}_{k} \right) \right) \otimes \varphi_{\delta}^{- 1}\left( \varphi_{\delta} \circ \chi \circ \varphi_{\beta}^{- 1}\left( \mathbf{e}_{l} \right) \right) \right)\\
&= \varphi_{\gamma \otimes \delta}^{- 1}\left( \varphi_{\gamma}^{- 1}\left( [\varphi]^{\gamma}_{\alpha}\mathbf{d}_{k} \right) \otimes \varphi_{\delta}^{- 1}\left( [\chi]^{\delta}_{\beta}\mathbf{e}_{l} \right) \right)\\
&= \varphi_{\gamma \otimes \delta}^{- 1}\left( \varphi_{\gamma}^{- 1}\left( \sum_{g \in \varLambda_{o}} {a_{gk}\mathbf{f}_{g}} \right) \otimes \varphi_{\delta}^{- 1}\left( \sum_{h \in \varLambda_{p}} {b_{hl}\mathbf{g}_{h}} \right) \right)\\
&= \varphi_{\gamma \otimes \delta}^{- 1}\left( \sum_{g \in \varLambda_{o}} {a_{gk}\varphi_{\gamma}^{- 1}\left( \mathbf{f}_{g} \right)} \otimes \sum_{h \in \varLambda_{p}} {b_{hl}\varphi_{\delta}^{- 1}\left( \mathbf{g}_{h} \right)} \right)\\
&= \varphi_{\gamma \otimes \delta}^{- 1}\left( \sum_{g \in \varLambda_{o}} {a_{gk}\mathbf{v}_{g}} \otimes \sum_{h \in \varLambda_{p}} {b_{hl}\mathbf{w}_{h}} \right)\\
&= \sum_{g \in \varLambda_{o}} {\sum_{h \in \varLambda_{p}} {a_{gk}b_{hl}\varphi_{\gamma \otimes \delta}^{- 1}\left( \mathbf{v}_{g} \otimes \mathbf{w}_{h} \right)}}\\
&= \sum_{g \in \varLambda_{o}} {\sum_{h \in \varLambda_{p}} {a_{gk}b_{hl}\left( \delta_{ig}\delta_{jh} \right)_{(i,j) \in \varLambda_{o} \times \varLambda_{p}}}}\\
&= \sum_{g \in \varLambda_{o}} {\sum_{h \in \varLambda_{p}} \begin{pmatrix}
\begin{pmatrix}
0 \\
 \vdots \\
0 \\
 \vdots \\
0 \\
\end{pmatrix} \\
 \vdots \\
\begin{pmatrix}
0 \\
 \vdots \\
a_{gk}b_{hl} \\
 \vdots \\
0 \\
\end{pmatrix} \\
 \vdots \\
\begin{pmatrix}
0 \\
 \vdots \\
0 \\
 \vdots \\
0 \\
\end{pmatrix} \\
\end{pmatrix}}\\
&= \begin{matrix}
\  & \begin{pmatrix}
\begin{pmatrix}
a_{1k}b_{1l} \\
 \vdots \\
0 \\
\end{pmatrix} \\
 \vdots \\
\begin{pmatrix}
0 \\
 \vdots \\
0 \\
\end{pmatrix} \\
\end{pmatrix} & + & \cdots & + & \begin{pmatrix}
\begin{pmatrix}
0 \\
 \vdots \\
a_{1k}b_{pl} \\
\end{pmatrix} \\
 \vdots \\
\begin{pmatrix}
0 \\
 \vdots \\
0 \\
\end{pmatrix} \\
\end{pmatrix} \\
\  & \  & \  & \cdots & \  & \  \\
 + & \begin{pmatrix}
\begin{pmatrix}
0 \\
 \vdots \\
0 \\
\end{pmatrix} \\
 \vdots \\
\begin{pmatrix}
a_{ok}b_{1l} \\
 \vdots \\
0 \\
\end{pmatrix} \\
\end{pmatrix} & + & \cdots & + & \begin{pmatrix}
\begin{pmatrix}
0 \\
 \vdots \\
0 \\
\end{pmatrix} \\
 \vdots \\
\begin{pmatrix}
0 \\
 \vdots \\
a_{ok}b_{pl} \\
\end{pmatrix} \\
\end{pmatrix} \\
\end{matrix}\\
&= \begin{pmatrix}
\begin{pmatrix}
a_{1k}b_{1l} \\
 \vdots \\
a_{1k}b_{pl} \\
\end{pmatrix} \\
 \vdots \\
\begin{pmatrix}
a_{ok}b_{1l} \\
 \vdots \\
a_{ok}b_{pl} \\
\end{pmatrix} \\
\end{pmatrix} = \begin{pmatrix}
a_{1k}b_{1l} \\
 \vdots \\
a_{1k}b_{pl} \\
 \vdots \\
a_{ok}b_{1l} \\
 \vdots \\
a_{ok}b_{pl} \\
\end{pmatrix}
\end{align*}
これにより、したがって、次のようになる。
\begin{align*}
\left[ \rho(\varphi,\chi) \right]^{\gamma \otimes \delta}_{\alpha \otimes \beta} &= \begin{pmatrix}
a_{11}b_{11} & \cdots & a_{11}b_{1n} & \  & a_{1m}b_{11} & \cdots & a_{1m}b_{1n} \\
 \vdots & \ddots & \vdots & \cdots & \vdots & \ddots & \vdots \\
a_{11}b_{p1} & \cdots & a_{11}b_{pn} & \  & a_{1m}b_{p1} & \cdots & a_{1m}b_{pn} \\
\  & \vdots & \  & \ddots & \  & \vdots & \  \\
a_{o1}b_{11} & \cdots & a_{o1}b_{1n} & \  & a_{om}b_{11} & \cdots & a_{om}b_{1n} \\
 \vdots & \ddots & \vdots & \cdots & \vdots & \ddots & \vdots \\
a_{o1}b_{p1} & \cdots & a_{o1}b_{pn} & \  & a_{om}b_{p1} & \cdots & a_{om}b_{pn} \\
\end{pmatrix}\\
&= \begin{pmatrix}
a_{11} & \cdots & a_{1m} \\
 \vdots & \ddots & \vdots \\
a_{o1} & \cdots & a_{om} \\
\end{pmatrix} \otimes \begin{pmatrix}
b_{11} & \cdots & b_{1n} \\
 \vdots & \ddots & \vdots \\
b_{p1} & \cdots & b_{pn} \\
\end{pmatrix} = [\varphi]^{\gamma}_{\alpha} \otimes [\chi]^{\delta}_{\beta}
\end{align*}
\end{proof}
\begin{thebibliography}{50}
  \bibitem{1}
  佐武一郎, 線型代数学, 裳華房, 1958. 第53版 p206-207 ISBN4-7853-1301-3
  \bibitem{2}
  Wikipedia. クロネッカー積. Wikipedia. \url{https://ja.wikipedia.org/wiki/%E3%82%AF%E3%83%AD%E3%83%8D%E3%83%83%E3%82%AB%E3%83%BC%E7%A9%8D} (2022-2-18 2:25 閲覧)
\end{thebibliography}
\end{document}
