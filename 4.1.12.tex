\documentclass[dvipdfmx]{jsarticle}
\setcounter{section}{1}
\setcounter{subsection}{11}
\usepackage{xr}
\externaldocument{4.1.7}
\externaldocument{4.1.10}
\usepackage{amsmath,amsfonts,amssymb,array,comment,mathtools,url,docmute}
\usepackage{longtable,booktabs,dcolumn,tabularx,mathtools,multirow,colortbl,xcolor}
\usepackage[dvipdfmx]{graphics}
\usepackage{bmpsize}
\usepackage{amsthm}
\usepackage{enumitem}
\setlistdepth{20}
\renewlist{itemize}{itemize}{20}
\setlist[itemize]{label=•}
\renewlist{enumerate}{enumerate}{20}
\setlist[enumerate]{label=\arabic*.}
\setcounter{MaxMatrixCols}{20}
\setcounter{tocdepth}{3}
\newcommand{\rotin}{\text{\rotatebox[origin=c]{90}{$\in $}}}
\newcommand{\amap}[6]{\text{\raisebox{-0.7cm}{\begin{tikzpicture} 
  \node (a) at (0, 1) {$\textstyle{#2}$};
  \node (b) at (#6, 1) {$\textstyle{#3}$};
  \node (c) at (0, 0) {$\textstyle{#4}$};
  \node (d) at (#6, 0) {$\textstyle{#5}$};
  \node (x) at (0, 0.5) {$\rotin $};
  \node (x) at (#6, 0.5) {$\rotin $};
  \draw[->] (a) to node[xshift=0pt, yshift=7pt] {$\textstyle{\scriptstyle{#1}}$} (b);
  \draw[|->] (c) to node[xshift=0pt, yshift=7pt] {$\textstyle{\scriptstyle{#1}}$} (d);
\end{tikzpicture}}}}
\newcommand{\twomaps}[9]{\text{\raisebox{-0.7cm}{\begin{tikzpicture} 
  \node (a) at (0, 1) {$\textstyle{#3}$};
  \node (b) at (#9, 1) {$\textstyle{#4}$};
  \node (c) at (#9+#9, 1) {$\textstyle{#5}$};
  \node (d) at (0, 0) {$\textstyle{#6}$};
  \node (e) at (#9, 0) {$\textstyle{#7}$};
  \node (f) at (#9+#9, 0) {$\textstyle{#8}$};
  \node (x) at (0, 0.5) {$\rotin $};
  \node (x) at (#9, 0.5) {$\rotin $};
  \node (x) at (#9+#9, 0.5) {$\rotin $};
  \draw[->] (a) to node[xshift=0pt, yshift=7pt] {$\textstyle{\scriptstyle{#1}}$} (b);
  \draw[|->] (d) to node[xshift=0pt, yshift=7pt] {$\textstyle{\scriptstyle{#2}}$} (e);
  \draw[->] (b) to node[xshift=0pt, yshift=7pt] {$\textstyle{\scriptstyle{#1}}$} (c);
  \draw[|->] (e) to node[xshift=0pt, yshift=7pt] {$\textstyle{\scriptstyle{#2}}$} (f);
\end{tikzpicture}}}}
\renewcommand{\thesection}{第\arabic{section}部}
\renewcommand{\thesubsection}{\arabic{section}.\arabic{subsection}}
\renewcommand{\thesubsubsection}{\arabic{section}.\arabic{subsection}.\arabic{subsubsection}}
\everymath{\displaystyle}
\allowdisplaybreaks[4]
\usepackage{vtable}
\theoremstyle{definition}
\newtheorem{thm}{定理}[subsection]
\newtheorem*{thm*}{定理}
\newtheorem{dfn}{定義}[subsection]
\newtheorem*{dfn*}{定義}
\newtheorem{axs}[dfn]{公理}
\newtheorem*{axs*}{公理}
\renewcommand{\headfont}{\bfseries}
\makeatletter
  \renewcommand{\section}{%
    \@startsection{section}{1}{\z@}%
    {\Cvs}{\Cvs}%
    {\normalfont\huge\headfont\raggedright}}
\makeatother
\makeatletter
  \renewcommand{\subsection}{%
    \@startsection{subsection}{2}{\z@}%
    {0.5\Cvs}{0.5\Cvs}%
    {\normalfont\LARGE\headfont\raggedright}}
\makeatother
\makeatletter
  \renewcommand{\subsubsection}{%
    \@startsection{subsubsection}{3}{\z@}%
    {0.4\Cvs}{0.4\Cvs}%
    {\normalfont\Large\headfont\raggedright}}
\makeatother
\makeatletter
\renewenvironment{proof}[1][\proofname]{\par
  \pushQED{\qed}%
  \normalfont \topsep6\p@\@plus6\p@\relax
  \trivlist
  \item\relax
  {
  #1\@addpunct{.}}\hspace\labelsep\ignorespaces
}{%
  \popQED\endtrivlist\@endpefalse
}
\makeatother
\renewcommand{\proofname}{\textbf{証明}}
\usepackage{tikz,graphics}
\usepackage[dvipdfmx]{hyperref}
\usepackage{pxjahyper}
\hypersetup{
 setpagesize=false,
 bookmarks=true,
 bookmarksdepth=tocdepth,
 bookmarksnumbered=true,
 colorlinks=false,
 pdftitle={},
 pdfsubject={},
 pdfauthor={},
 pdfkeywords={}}
\begin{document}
%\hypertarget{ux4e2dux9593ux5024ux306eux5b9aux7406}{%
\subsection{中間値の定理}%\label{ux4e2dux9593ux5024ux306eux5b9aux7406}}
この周辺の議論のさらなる一般化については位相空間論に詳しいので、そちらのほうを参照されたい。ここでは主に、いわゆる$n$次元Euclid空間というやや特殊化された条件下であるものの、$n$次元Euclid空間特有の性質について述べよう。なお、一般的な位相空間論の知識は仮定していない。
%\hypertarget{ux6700ux5927ux5024ux6700ux5c0fux5024ux306eux5b9aux7406}{%
\subsubsection{最大値最小値の定理}%\label{ux6700ux5927ux5024ux6700ux5c0fux5024ux306eux5b9aux7406}}
\begin{dfn}
$D(f) \subseteq \mathbb{R}^{n}$なる関数$f:D(f) \rightarrow \mathbb{R}$が与えられたとき\footnote{終集合が$\mathbb{R}$と大小関係が比較できるようになっていることが重要です。}、実数$\max{V(f)}$をその集合$D(f)$におけるその関数$f$の最大値といい$\max_{\mathbf{x} \in D(f)}{f\left( \mathbf{x} \right)}$、$\underset{\mathbf{x} \in D(f)}{\mathrm{Max}}{f\left( \mathbf{x} \right)}$などと、特にその集合$D(f)$が明らかな場合では、$\max f$、$\mathrm{Max}f$などとも書く。さらに、$f\left( \mathbf{x} \right) = \max f$なる集合$D(f)$の元$\mathbf{x}$をその関数$f$の最大点といい、$f\left( \mathbf{x} \right) = \max f$が成り立つことをその関数$f$はその元$\mathbf{x}$において最大に達する、最大値に達するなどという。\par
同様に、$D(f) \subseteq \mathbb{R}^{n}$なる関数$f:D(f) \rightarrow \mathbb{R};\mathbf{x} \mapsto f\left( \mathbf{x} \right)$が与えられたとき、実数$\min{V(f)}$をその集合$D(f)$におけるその関数$f$の最小値といい$\min_{\mathbf{x} \in D(f)}{f\left( \mathbf{x} \right)}$、$\underset{\mathbf{x} \in D(f)}{\mathrm{Min}}{f\left( \mathbf{x} \right)}$などと、特にその集合$D(f)$が明らかな場合では、$\min f$、$\mathrm{Min}f$などとも書く。さらに、$f\left( \mathbf{x} \right) = \min f$なる集合$D(f)$の元$\mathbf{x}$をその関数$f$の最小点といい、$f\left( \mathbf{x} \right) = \min f$が成り立つことをその関数$f$はその元$\mathbf{x}$において最小に達する、最小値に達するなどという。
\end{dfn}
\begin{thm}\label{4.1.12.1}
$K \subseteq D(f) \subseteq R \subseteq \mathbb{R}^{m}$、$S \subseteq \mathbb{R}_{\infty}^{n}$なる関数$f:D(f) \rightarrow S$が与えられたとき、その集合$K$がその集合$R$で点列compactでその関数$f:D(f) \rightarrow S$がその集合$K$で連続であるとき、即ち、その関数$f|K:K \rightarrow S$がその集合$K$で連続であるとき、その集合$V\left( f|K \right)$は点列compactであり、特に、$S \subseteq \mathbb{R}^{n}$のとき、その関数$f|K$はその集合$S$で有界である。
\end{thm}\par
この定理の証明のplotを述べよう。その集合$V\left( f|K \right)$の任意の点列$\left( b_{l} \right)_{l \in \mathbb{N}}$に対し、$b_{l} = f\left( \mathbf{a}_{l} \right)$なるその集合$K$の元$\mathbf{a}_{l}$を1つとると、その集合$K$の点列$\left( \mathbf{a}_{l} \right)_{l \in \mathbb{N}}$が得られる。その点列$\left( \mathbf{a}_{l} \right)_{l \in \mathbb{N}}$に収束する部分列$\left( \mathbf{a}_{l_{k}} \right)_{k \in \mathbb{N}}$が存在し、この部分列$\left( \mathbf{a}_{l_{k}} \right)_{k \in \mathbb{N}}$について、$\lim_{k \rightarrow \infty}\mathbf{a}_{l_{k}} \in K$が成り立つ。このこととその関数$f:D(f) \rightarrow S$はその集合$K$で連続であることに注意すれば、$\lim_{k \rightarrow \infty}b_{l_{k}} \in V\left( f|K \right)$が成り立つ。その後半も定理\ref{4.1.7.6}より数行ほどで示される。拡張$n$次元数空間$\mathbb{R}_{\infty}^{n}$のかわりに補完数直線${}^{*}\mathbb{R}$でおきかえても同様にして示される。
\begin{proof}
$K \subseteq D(f) \subseteq R \subseteq \mathbb{R}^{m}$、$S \subseteq \mathbb{R}_{\infty}^{n}$なる関数$f:D(f) \rightarrow S$が与えられたとき、その集合$K$がその集合$R$で点列compactでその関数$f:D(f) \rightarrow S$がその集合$K$で連続であるとき、即ち、その関数$f|K:K \rightarrow S$がその集合$K$で連続であるとき、その集合$V\left( f|K \right)$の任意の点列$\left( b_{l} \right)_{l \in \mathbb{N}}$に対し、$b_{l} = f\left( \mathbf{a}_{l} \right)$なるその集合$K$の元$\mathbf{a}_{l}$を1つとると、その集合$K$の点列$\left( \mathbf{a}_{l} \right)_{l \in \mathbb{N}}$が得られる。ここで、その集合$K$は点列compactであったので、その点列$\left( \mathbf{a}_{l} \right)_{l \in \mathbb{N}}$にその集合$R$で収束する部分列$\left( \mathbf{a}_{l_{k}} \right)_{k \in \mathbb{N}}$が存在し、この部分列$\left( \mathbf{a}_{l_{k}} \right)_{k \in \mathbb{N}}$についてこの極限値$\lim_{k \rightarrow \infty}\mathbf{a}_{l_{k}}$が$\mathbf{a}$とおかれると、$\mathbf{a} = \lim_{k \rightarrow \infty}\mathbf{a}_{l_{k}} \in K$が成り立ち、さらに、その関数$f:D(f) \rightarrow S$はその集合$K$で連続であったので、定理\ref{4.1.10.2}より次式が成り立つ。
\begin{align*}
\lim_{k \rightarrow \infty}b_{l_{k}} = \lim_{k \rightarrow \infty}{f\left( \mathbf{a}_{l_{k}} \right)} = \lim_{\scriptsize \begin{matrix}
\mathbf{x} \rightarrow \mathbf{a} \\
R \rightarrow S \\
\end{matrix}}{f\left( \mathbf{x} \right)} = f\left( \mathbf{a} \right) \in V\left( f|K \right)
\end{align*}
したがって、任意の点列$\left( b_{l} \right)_{l \in \mathbb{N}}$に対し、収束する部分列$\left( b_{l_{k}} \right)_{k \in \mathbb{N}}$が存在し、この部分列$\left( b_{l_{k}} \right)_{k \in \mathbb{N}}$について、$\lim_{k \rightarrow \infty}b_{l_{k}} \in V\left( f|K \right)$が成り立つ。よって、その集合$V\left( f|K \right)$もその集合$S$で点列compactである。\par
特に、$S \subseteq \mathbb{R}^{n}$のとき、定理\ref{4.1.7.6}よりその集合$K$がその集合$S$で点列compactであるならそのときに限り、その集合$K$はその集合$S$で有界な閉集合であったので、その集合$V\left( f|K \right)$もその集合$S$で有界である。
\end{proof}
\begin{thm}\label{4.1.12.2}
$K \subseteq D(f) \subseteq R \subseteq \mathbb{R}^{m}$、$S \subseteq{}^{*}\mathbb{R}$なる関数$f:D(f) \rightarrow S$が与えられたとき、その集合$K$がその集合$R$で点列compactでその関数$f:D(f) \rightarrow S$がその集合$K$で連続であるとき、その関数$f$はその集合$K$で最大値$\max f$、最小値$\min f$をとることができる、即ち、その関数$f|K:K \rightarrow S$の最大値$\max{f|K}$と最小値$\min{f|K}$が存在する。
\end{thm}\par
この定理の証明のplotを述べよう。定理\ref{4.1.12.1}よりその集合$V\left( f|K \right)$は点列compactで上限$\sup{V\left( f|K \right)}$が存在する。また、$\forall\varepsilon \in \mathbb{R}^{+}$に対し、次式が成り立つので、
\begin{align*}
\sup{V\left( f|K \right)} - \varepsilon < f\left( \mathbf{k} \right) < \sup{V\left( f|K \right)} + \varepsilon
\end{align*}
$\sup{V\left( f|K \right)} \in \mathrm{cl}_{S}{V\left( f|K \right)}$が成り立つ。そこで、定理\ref{4.1.7.2}、定理\ref{4.1.7.3}よりその集合$K$は閉集合なので、$\sup{V\left( f|K \right)} \in V\left( f|K \right)$が成り立つ。よって、その関数$f$はその集合$K$で最大値$\max f$をとることができる。
\begin{proof}
$K \subseteq D(f) \subseteq R \subseteq \mathbb{R}^{m}$、$S \subseteq{}^{*}\mathbb{R}$なる関数$f:D(f) \rightarrow S$が与えられたとき、その集合$K$がその集合$R$で点列compactでその関数$f:D(f) \rightarrow S$がその集合$K$で連続であるとき、定理\ref{4.1.12.1}よりその集合$V\left( f|K \right)$は点列compactであった。上限性質よりその集合$V\left( f|K \right)$が上に有界であるとき、実数として上限$\sup{V\left( f|K \right)}$が存在するし、その集合$V\left( f|K \right)$が上に有界でないとき、$\sup{V\left( f|K \right)} = \infty$とすればよい。いづれも拡大実数として上限$\sup{V\left( f|K \right)}$が存在する。また、$\forall\varepsilon \in \mathbb{R}^{+}$に対し、その実数$\sup{V\left( f|K \right)} - \varepsilon$はその集合$V\left( f|K \right)$の上界でありえないので、$\exists\mathbf{k} \in K$に対し、$\sup{V\left( f|K \right)} - \varepsilon \leq f\left( \mathbf{k} \right)$が成り立つ。正の実数$\varepsilon$のおき方に工夫すれば、特に、$\sup{V\left( f|K \right)} - \varepsilon < f\left( \mathbf{k} \right)$が成り立つとしてもよい。また、上限の定義より$f\left( \mathbf{k} \right) < \sup{V\left( f|K \right)} + \varepsilon$が成り立つので、これにより、次のようになる。
\begin{align*}
\sup{V\left( f|K \right)} - \varepsilon < f\left( \mathbf{k} \right) < \sup{V\left( f|K \right)} + \varepsilon &\Leftrightarrow - \varepsilon < f\left( \mathbf{k} \right) - \sup{V\left( f|K \right)} < \varepsilon\\
&\Leftrightarrow \left| f\left( \mathbf{k} \right) - \sup{V\left( f|K \right)} \right| < \varepsilon\\
&\Leftrightarrow f\left( \mathbf{k} \right) \in U\left( \sup{V\left( f|K \right)},\varepsilon \right)
\end{align*}
したがって、$U\left( \sup{V\left( f|K \right)},\varepsilon \right) \cap S \neq \emptyset$が成り立つので、$\sup{V\left( f|K \right)} \in \mathrm{cl}_{S}{V\left( f|K \right)}$が成り立つ。そこで、定理\ref{4.1.7.2}、定理\ref{4.1.7.3}よりその集合$K$はその集合$S$で閉集合であるので、$\sup{V\left( f|K \right)} \in V\left( f|K \right)$が成り立つ。したがって、$\sup{V\left( f|K \right)} = \max{V\left( f|K \right)} = \max{f|K}$が成り立つ。よって、その関数$f$はその集合$K$で最大値$\max f$をとることができる。\par
同様にして、その関数$f$はその集合$K$で最小値$\min f$をとることが示される。
\end{proof}
\begin{thm}[最大値最小値の定理]\label{4.1.12.3}
$K \subseteq D(f) \subseteq R \subseteq \mathbb{R}^{m}$、$S \subseteq{}^{*}\mathbb{R}$なる関数$f:D(f) \rightarrow S$が与えられたとき、その集合$K$が空集合$\emptyset$でない有界な閉集合でその関数$f:D(f) \rightarrow S$がその集合$K$で連続であるとき、その関数$f$はその集合$K$で最大値$\max f$、最小値$\min f$をとることができる。
\end{thm}\par
この定理はよく使われる形である定理\ref{4.1.12.2}の系であり、これを最大値最小値の定理、最大・最小の定理などという。
\begin{proof} 定理\ref{4.1.7.6}より明らかである。
\end{proof}
%\hypertarget{ux9023ux7d50}{%
\subsubsection{連結}%\label{ux9023ux7d50}}
\begin{dfn}
$R \subseteq \mathbb{R}_{\infty}^{n}$なる集合$R$における開集合であるかつ閉集合でもあるようなその集合$R$の部分集合がその集合$R$自身か空集合以外に存在しないようなとき、その集合$R$は連結であるという。
\end{dfn}
\begin{dfn}
$R \subseteq \mathbb{R}_{\infty}^{n}$なる集合$R$における開集合であるかつ連結であるようなその集合$R$の部分集合$D$をその集合$R$における領域という\footnote{日常語とのズレ注意です。}。
\end{dfn}
\begin{thm}\label{4.1.12.4}
$R \subseteq \mathbb{R}_{\infty}^{n}$なる集合$R$が与えられたとき、次のことは同値である。
\begin{itemize}
\item
  その集合$R$は連結である。
\item
  2つの空集合でないその集合$R$における任意の開集合たち$U$、$V$は$R \neq U \sqcup V$を満たす。
\item
  2つの空集合でないその集合$R$における任意の閉集合たち$C$、$D$は$R \neq C \sqcup D$を満たす。
\end{itemize}
\end{thm}
\begin{proof}
$R \subseteq \mathbb{R}_{\infty}^{n}$なる集合$R$が与えられたとき、2つの空集合でないその集合$R$におけるある開集合たち$U$、$V$が存在して、$R = U \sqcup V$が成り立つなら、それらの集合たち$U$、$V$がそれぞれ開集合$V$、$U$の補集合でもあるので、それらの集合たち$U$、$V$は閉集合でもある。ゆえに、その集合$R$は連結でない。逆に、その集合$R$が連結でないなら、その集合$R$自身か空集合以外に集合$R$における開集合であるかつ閉集合でもあるような部分集合$U$が存在する。そこで、$V = R \setminus U$とすれば、その集合$V$が閉集合でもあるので、その集合$V$は空集合でない開集合となっており、さらに、$R = U \sqcup V$が成り立つ。以上の議論により、次のことは同値であることが示された。
\begin{itemize}
\item
  その集合$R$は連結である。
\item
  2つの空集合でないその集合$R$における任意の開集合たち$U$、$V$は$R \neq U \sqcup V$を満たす。
\end{itemize}
同様にして次のことは同値であることも示される。
\begin{itemize}
\item
  その集合$R$は連結である。
\item
  2つの空集合でないその集合$R$における任意の閉集合たち$C$、$D$は$R \neq C \sqcup D$を満たす。
\end{itemize}
\end{proof}
\begin{thm}\label{4.1.12.5}
$A \subseteq D(f) \subseteq R \subseteq \mathbb{R}^{m}$、$S \subseteq \mathbb{R}_{\infty}^{n}$なる関数$f:D(f) \rightarrow S$が与えられたとき、この関数$f$がその集合$A$で連続であるかつ、その集合$A$が連結であるなら、その値域$V\left( f|A \right)$は連結である。
\end{thm}
\begin{proof}
$A \subseteq D(f) \subseteq R \subseteq \mathbb{R}^{m}$、$S \subseteq \mathbb{R}_{\infty}^{n}$なる関数$f:D(f) \rightarrow S$が与えられたとき、この関数$f$がその集合$A$で連続であるかつ、その集合$A$が連結であるとする。その値域$V\left( f|A \right)$が連結でないと仮定すると、定理\ref{4.1.12.4}よりその値域$V\left( f|A \right)$の空集合でない開集合たち$U$、$V$が存在して、$V\left( f|A \right) = U \sqcup V$が成り立つ。このとき、次のようになるかつ、
\begin{align*}
V\left( f|A \right) = U \cup V &\Leftrightarrow A \subseteq V\left( f^{- 1}|V\left( f|A \right) \right) = V\left( f^{- 1}|U \cup V \right)\\
&\Rightarrow A \subseteq V\left( f^{- 1}|U \right) \cup V\left( f^{- 1}|V \right)
\end{align*}
値域の定義より$V\left( f^{- 1}|U \right) \subseteq A$かつ$V\left( f^{- 1}|V \right) \subseteq A$が成り立つので、$A = V\left( f^{- 1}|U \right) \cup V\left( f^{- 1}|V \right)$が成り立つ。また、次のようになるので、
\begin{align*}
U \cap V = \emptyset &\Rightarrow V\left( f^{- 1}|U \cap V \right) = V\left( f^{- 1}|\emptyset \right) = \emptyset\\
&\Leftrightarrow V\left( f^{- 1}|U \right) \cap V\left( f^{- 1}|V \right) = \emptyset
\end{align*}
よって、$A = V\left( f^{- 1}|U \right) \sqcup V\left( f^{- 1}|V \right)$が成り立つ。もちろん、$V\left( f^{- 1}|U \right) \neq \emptyset$かつ$V\left( f^{- 1}|V \right) \neq \emptyset$が成り立つ。\par
そこで、定理\ref{4.1.10.17}よりそれらの集合たち$V\left( f^{- 1}|U \right) \cap A$、$V\left( f^{- 1}|V \right) \cap A$もその集合$A$で開集合であるので、$A = V\left( f^{- 1}|U \right) \sqcup V\left( f^{- 1}|V \right)$に注意すれば、それらの集合たち$V\left( f^{- 1}|U \right)$、$V\left( f^{- 1}|V \right)$もその集合$A$での空集合でない開集合であることになる。このとき、$A = V\left( f^{- 1}|U \right) \sqcup V\left( f^{- 1}|V \right)$が成り立っているので、その集合$A$は連結でないことになるが、これは仮定に矛盾している。
\end{proof}
%\hypertarget{ux4e2dux9593ux5024ux306eux5b9aux7406-1}{%
\subsubsection{中間値の定理}%\label{ux4e2dux9593ux5024ux306eux5b9aux7406-1}}\par
中間値の定理を示す前に次の補題を考えよう。
\begin{thm}\label{4.1.12.6}
$R \subseteq \mathbb{R}$なる集合$R$が連結であるとき、$\forall a,b \in R$に対し、$[ a,b] \subseteq R$が成り立つ。
\end{thm}
\begin{proof}
$R \subseteq \mathbb{R}$なる集合$R$が連結であるとき、$\forall a,b \in R$に対し、ある実数$c$が存在して、$a \leq c \leq b$が成り立つかつ、$c \in R$が成り立たないものとする。このとき、$a = b$のときは明らかに成り立ちえないので、$a < b$が成り立つことになる。$a = c$または$b = c$が成り立つことは仮定よりありえないので、$a < c < b$が成り立つことになる。このとき、開区間たち$( - \infty,c)$、$(c,\infty)$が考えられれば、これらの集合たち$( - \infty,c) \cap R$、$(c,\infty) \cap R$はその集合$R$での開集合である。さらに、次式が成り立つ。
\begin{align*}
R = \left( ( - \infty,c) \cap R \right) \sqcup \left( (c,\infty) \cap R \right)
\end{align*}
実際、$R \subseteq \mathbb{R} \setminus \left\{ c \right\}$より次のようになることから従う。
\begin{align*}
\left( ( - \infty,c) \cap R \right) \cup \left( (c,\infty) \cap R \right) &= \left( ( - \infty,c) \cup (c,\infty) \right) \cap R\\
&= \left( \mathbb{R} \setminus \left\{ c \right\} \right) \cap R = R\\
&\left( ( - \infty,c) \cap R \right) \cap \left( (c,\infty) \cap R \right) = ( - \infty,c) \cap (c,\infty) \cap R\\
&= \emptyset \cap R = \emptyset
\end{align*}
ここで、$a \in ( - \infty,c) \cap R$かつ$b \in (c,\infty)$が成り立つので、これらの集合たち$( - \infty,c) \cap R$、$(c,\infty) \cap R$は空集合でない。以上の議論により、その集合$R$は連結でないことになるが、これは仮定に矛盾している。したがって、$\forall a,b \in R\forall c \in \mathbb{R}$に対し、$a \leq c \leq b$が成り立つなら、$c \in R$が成り立つことになる、即ち、$[ a,b] \subseteq R$が成り立つ。
\end{proof}
\begin{thm}[中間値の定理]\label{4.1.12.7}
$A \subseteq D(f) \subseteq R \subseteq \mathbb{R}^{m}$、$S \subseteq \mathbb{R}$なる関数$f:D(f) \rightarrow S$が与えられたとし、さらに、この関数$f$がその集合$A$で連続であるかつ、その集合$A$が連結であるとする。$\forall\mathbf{a},\mathbf{b} \in A\forall\gamma \in \mathbb{R}$に対し、$f\left( \mathbf{a} \right) \lesseqgtr \gamma \lesseqgtr f\left( \mathbf{b} \right)$が成り立つなら、$f\left( \mathbf{c} \right) = \gamma$なる点$\mathbf{c}$がその集合$A$に存在する。この定理を中間値の定理という。
\end{thm}
\begin{proof}
$A \subseteq D(f) \subseteq R \subseteq \mathbb{R}^{m}$、$S \subseteq \mathbb{R}$なる関数$f:D(f) \rightarrow S$が与えられたとし、さらに、この関数$f$がその集合$A$で連続であるかつ、その集合$A$が連結であるとする。$\forall\mathbf{a},\mathbf{b} \in A$に対し、$f\left( \mathbf{a} \right) \lesseqgtr \gamma \lesseqgtr f\left( \mathbf{b} \right)$が成り立つとき、$f\left( \mathbf{a} \right) \leq \gamma \leq f\left( \mathbf{b} \right)$としても一般性は失われない。さらに、$\gamma = f\left( \mathbf{a} \right)$または$\gamma = f\left( \mathbf{b} \right)$のときは明らかなので、$f\left( \mathbf{a} \right) < \gamma < f\left( \mathbf{b} \right)$が成り立つとしてもよい。このとき、その値域$V\left( f|A \right)$も連結であるので、$\forall f\left( \mathbf{a} \right),f\left( \mathbf{b} \right) \in V\left( f|A \right)$が成り立つことに注意すれば、定理\ref{4.1.12.6}より$\left[ f\left( \mathbf{a} \right),f\left( \mathbf{b} \right) \right] \subseteq V\left( f|A \right)$が成り立つ。ここで、$\gamma \in \left[ f\left( \mathbf{a} \right),f\left( \mathbf{b} \right) \right]$が成り立つので、$\gamma \in V\left( f|A \right)$となりよって、$f\left( \mathbf{c} \right) = \gamma$なる点$\mathbf{c}$がその集合$A$に存在する。
\end{proof}
\begin{thm}\label{4.1.12.8}
$A \subseteq D(f) \subseteq R \subseteq \mathbb{R}^{m}$、$S \subseteq \mathbb{R}$なる関数$f:D(f) \rightarrow S$が与えられたとし、さらに、この関数$f$がその集合$A$で連続であるかつ、その集合$A$が連結でその集合$R$での有界な閉集合であるとき、次式が成り立つ。
\begin{align*}
V\left( f|A \right) = \left[ \min{f|A},\max{f|A} \right] = \left[ \inf{V\left( f|A \right)},\sup{V\left( f|A \right)} \right]
\end{align*}
\end{thm}
\begin{proof}
$A \subseteq D(f) \subseteq R \subseteq \mathbb{R}^{m}$、$S \subseteq \mathbb{R}$なる関数$f:D(f) \rightarrow S$が与えられたとし、さらに、この関数$f$がその集合$A$で連続であるかつ、その集合$A$が連結でその集合$R$での有界な閉集合であるとき、最大値最小値の定理よりその関数$f$はその集合$A$で最大値$\max{f|A}$、最小値$\min{f|A}$をとることができる。したがって、$f\left( \mathbf{a} \right) = \max{f|A}$、$f\left( \mathbf{b} \right) = \min{f|A}$なる点々$\mathbf{a}$、$\mathbf{b}$がその集合$A$に存在する。ここで、$\max{f|A} = \max{V\left( f|A \right)} = \sup{V\left( f|A \right)}$、$\min{f|A} = \min{V\left( f|A \right)} = \inf{V\left( f|A \right)}$が成り立ち、したがって、$V\left( f|A \right) \subseteq \left[ \inf{V\left( f|A \right)},\sup{V\left( f|A \right)} \right]$が成り立つ。また、中間値の定理より$\inf{V\left( f|A \right)} \leq \gamma \leq \sup{V\left( f|A \right)}$なる任意の実数$\gamma$に対し$f\left( \mathbf{c} \right) = \gamma$なる点$\mathbf{c}$がその有界閉区間$D(f)$に存在するので、$V\left( f|A \right) \supseteq \left[ \inf{V\left( f|A \right)},\sup{V\left( f|A \right)} \right]$が成り立つ。よって、次式が成り立つ。
\begin{align*}
V\left( f|A \right) = \left[ \min{f|A},\max{f|A} \right] = \left[ \inf{V\left( f|A \right)},\sup{V\left( f|A \right)} \right]
\end{align*}
\end{proof}\par
中間値の定理のclassicな主張と証明も次に述べよう\footnote{つまり、上記のように現代的な位相空間論に基づく幾何学的な議論をしたものでなく解析学的な感じにしたものです。}。
\begin{thm}[中間値の定理]\label{4.1.12.9}
$[ a,b] \subseteq D(f) \subseteq \mathbb{R}$なる関数$f:D(f) \rightarrow \mathbb{R}$が与えられたとし、さらに、この関数$f$がその有界閉区間$[ a,b]$で連続であるとする。$\forall a,b \in A\forall\gamma \in \mathbb{R}$に対し、$f(a) \lesseqgtr \gamma \lesseqgtr f(b)$が成り立つなら、$f(c) = \gamma$なる実数$c$がその集合$A$に存在する。
\end{thm}\par
これは次のようにして示される。
\begin{enumerate}
\item
  $a_{1} = a$、$b_{1} = b$とおき次式のように帰納的に定義する。
\begin{align*}
a_{n + 1} &= \left\{ \begin{matrix}
a_{n} & \mathrm{if} & \gamma < f\left( \frac{a_{n} + b_{n}}{2} \right) \\
\frac{a_{n} + b_{n}}{2} & \mathrm{if} & f\left( \frac{a_{n} + b_{n}}{2} \right) \leq \gamma \\
\end{matrix} \right.\ ,\\
b_{n + 1} &= \left\{ \begin{matrix}
\frac{a_{n} + b_{n}}{2} & \mathrm{if} & \gamma < f\left( \frac{a_{n} + b_{n}}{2} \right) \\
b_{n} & \mathrm{if} & f\left( \frac{a_{n} + b_{n}}{2} \right) \leq \gamma \\
\end{matrix} \right.\ 
\end{align*}
\item
  数学的帰納法により、$\forall n \in \mathbb{N}$に対し、$f\left( a_{n} \right) \leq \gamma \leq f\left( b_{n} \right)$が成り立つ。
\item
  1. より$\lim_{n \rightarrow \infty}\left( b_{n} - a_{n} \right) = 0$が成り立つ。
\item
  区間縮小法よりその共通部分$\bigcap_{n \in \mathbb{N}} \left[ a_{n},b_{n} \right]$は1つの$\lim_{n \rightarrow \infty}a_{n} = \lim_{n \rightarrow \infty}b_{n} = c$なる実数$c$を含む。
\item
  その関数$f$はその有界閉区間$[ a,b]$で連続であることに注意すれば、2. より$\lim_{n \rightarrow \infty}{f\left( a_{n} \right)} = \lim_{n \rightarrow \infty}{f\left( b_{n} \right)} = f(c) = \gamma$が成り立つ。
\end{enumerate}
\begin{proof}
$[ a,b] \subseteq D(f) \subseteq \mathbb{R}$なる関数$f:D(f) \rightarrow \mathbb{R}$が与えられたとし、さらに、この関数$f$がその有界閉区間$[ a,b]$で連続であるとする。$\forall a,b \in A\forall\gamma \in \mathbb{R}$に対し、$f(a) \lesseqgtr \gamma \lesseqgtr f(b)$が成り立つとき、$f(a) < 0 < f(b)$が成り立つとしてもよい。そこで、$a_{1} = a$、$b_{1} = b$とおき次式のように実数列たち$\left( a_{n} \right)_{n \in \mathbb{N}}$、$\left( b_{n} \right)_{n \in \mathbb{N}}$を帰納的に定義する。
\begin{align*}
a_{n + 1} &= \left\{ \begin{matrix}
a_{n} & \mathrm{if} & \gamma < f\left( \frac{a_{n} + b_{n}}{2} \right) \\
\frac{a_{n} + b_{n}}{2} & \mathrm{if} & f\left( \frac{a_{n} + b_{n}}{2} \right) \leq \gamma \\
\end{matrix} \right.\ ,\\
b_{n + 1} &= \left\{ \begin{matrix}
\frac{a_{n} + b_{n}}{2} & \mathrm{if} & \gamma < f\left( \frac{a_{n} + b_{n}}{2} \right) \\
b_{n} & \mathrm{if} & f\left( \frac{a_{n} + b_{n}}{2} \right) \leq \gamma \\
\end{matrix} \right.\ 
\end{align*}
このとき、$n = 1$のとき、定義より明らかに$f\left( a_{1} \right) \leq \gamma \leq f\left( b_{1} \right)$が成り立ち、$n = k$のとき、$f\left( a_{k} \right) \leq \gamma \leq f\left( b_{k} \right)$が成り立つと仮定すれば、$n = k + 1$のとき、次のようになる。
\begin{align*}
&\quad \left\{ \begin{matrix}
f\left( a_{k + 1} \right) = \left\{ \begin{matrix}
f\left( a_{k} \right) & \mathrm{if} & \gamma < f\left( \frac{a_{k} + b_{k}}{2} \right) \\
f\left( \frac{a_{k} + b_{k}}{2} \right) & \mathrm{if} & f\left( \frac{a_{k} + b_{k}}{2} \right) \leq \gamma \\
\end{matrix} \right.\  \\
f\left( b_{k + 1} \right) = \left\{ \begin{matrix}
f\left( \frac{a_{k} + b_{k}}{2} \right) & \mathrm{if} & \gamma < f\left( \frac{a_{k} + b_{k}}{2} \right) \\
f\left( b_{k} \right) & \mathrm{if} & f\left( \frac{a_{k} + b_{k}}{2} \right) \leq \gamma \\
\end{matrix} \right.\  \\
\end{matrix} \right.\ \\
&\Leftrightarrow \left\{ \begin{matrix}
f\left( a_{k + 1} \right) = f\left( a_{k} \right) & \mathrm{if} & \gamma < f\left( \frac{a_{k} + b_{k}}{2} \right) \\
f\left( a_{k + 1} \right) = f\left( \frac{a_{k} + b_{k}}{2} \right) & \mathrm{if} & f\left( \frac{a_{k} + b_{k}}{2} \right) \leq \gamma \\
f\left( b_{k + 1} \right) = f\left( \frac{a_{k} + b_{k}}{2} \right) & \mathrm{if} & \gamma < f\left( \frac{a_{k} + b_{k}}{2} \right) \\
f\left( b_{k + 1} \right) = f\left( b_{k} \right) & \mathrm{if} & f\left( \frac{a_{k} + b_{k}}{2} \right) \leq \gamma \\
\end{matrix} \right.\ \\
&\Leftrightarrow \left\{ \begin{matrix}
\left\{ \begin{matrix}
f\left( a_{k + 1} \right) = f\left( a_{k} \right) \\
f\left( b_{k + 1} \right) = f\left( \frac{a_{k} + b_{k}}{2} \right) \\
\end{matrix} \right.\  & \mathrm{if} & \gamma < f\left( \frac{a_{k} + b_{k}}{2} \right) \\
\left\{ \begin{matrix}
f\left( a_{k + 1} \right) = f\left( \frac{a_{k} + b_{k}}{2} \right) \\
f\left( b_{k + 1} \right) = f\left( b_{k} \right) \\
\end{matrix} \right.\  & \mathrm{if} & f\left( \frac{a_{k} + b_{k}}{2} \right) \leq \gamma \\
\end{matrix} \right.\ \\
&\Leftrightarrow \left\{ \begin{matrix}
f\left( a_{k + 1} \right) = f\left( a_{k} \right) \\
\gamma < f\left( b_{k + 1} \right) = f\left( \frac{a_{k} + b_{k}}{2} \right) \\
\end{matrix} \right.\  \vee \left\{ \begin{matrix}
f\left( a_{k + 1} \right) = f\left( \frac{a_{k} + b_{k}}{2} \right) \leq \gamma \\
f'\left( b_{k + 1} \right) = f\left( b_{k} \right) \\
\end{matrix} \right.\ \\
&\Leftrightarrow \left\{ \begin{matrix}
f\left( a_{k + 1} \right) = f\left( a_{k} \right) \leq \gamma \\
\gamma < f\left( b_{k + 1} \right) = f\left( \frac{a_{k} + b_{k}}{2} \right) \\
\end{matrix} \right.\  \vee \left\{ \begin{matrix}
f\left( a_{k + 1} \right) = f\left( \frac{a_{k} + b_{k}}{2} \right) \leq \gamma \\
\gamma \leq f\left( b_{k + 1} \right) = f\left( b_{k} \right) \\
\end{matrix} \right.\ \\
&\Rightarrow \left\{ \begin{matrix}
f\left( a_{k + 1} \right) \leq \gamma \\
\gamma \leq f\left( b_{k + 1} \right) \\
\end{matrix} \right.\  \vee \left\{ \begin{matrix}
f\left( a_{k + 1} \right) \leq \gamma \\
\gamma \leq f\left( b_{k + 1} \right) \\
\end{matrix} \right.\ \\
&\Leftrightarrow \left\{ \begin{matrix}
f\left( a_{k + 1} \right) \leq \gamma \leq f\left( b_{k + 1} \right) \\
f\left( a_{k + 1} \right) \leq \gamma \leq f\left( b_{k + 1} \right) \\
\end{matrix} \right.\ \\
&\Leftrightarrow f\left( a_{k + 1} \right) \leq \gamma \leq f\left( b_{k + 1} \right)
\end{align*}
以上より、数学的帰納法によって、$\forall n \in \mathbb{N}$に対し、$f\left( a_{n} \right) \leq \gamma \leq f\left( b_{n} \right)$が成り立つ。\par
一方、明らかに$\forall n \in \mathbb{N}$に対し、$\left[ a_{n + 1},b_{n + 1} \right] \subseteq \left[ a_{n},b_{n} \right]$が成り立つかつ、実数$b_{n + 1} - a_{n + 1}$について次のようになる。
\begin{align*}
b_{n + 1} - a_{n + 1} &= \left\{ \begin{matrix}
\frac{a_{n} + b_{n}}{2} - a_{n} & \mathrm{if} & 0 < f\left( \frac{a_{n} + b_{n}}{2} \right) \\
b_{n} - \frac{a_{n} + b_{n}}{2} & \mathrm{if} & f\left( \frac{a_{n} + b_{n}}{2} \right) \leq 0 \\
\end{matrix} \right.\ \\
&= \left\{ \begin{matrix}
\frac{a_{n} + b_{n} - 2a_{n}}{2} & \mathrm{if} & 0 < f\left( \frac{a_{n} + b_{n}}{2} \right) \\
\frac{2b_{n} - a_{n} - b_{n}}{2} & \mathrm{if} & f\left( \frac{a_{n} + b_{n}}{2} \right) \leq 0 \\
\end{matrix} \right.\ \\
&= \left\{ \begin{matrix}
\frac{b_{n} - a_{n}}{2} & \mathrm{if} & 0 < f\left( \frac{a_{n} + b_{n}}{2} \right) \\
\frac{b_{n} - a_{n}}{2} & \mathrm{if} & f\left( \frac{a_{n} + b_{n}}{2} \right) \leq 0 \\
\end{matrix} \right.\ \\
&= \frac{b_{n} - a_{n}}{2} = \frac{1}{2}\left( b_{n} - a_{n} \right)
\end{align*}
数学的帰納法によって明らかに次のようになる。
\begin{align*}
\lim_{n \rightarrow \infty}\left( b_{n} - a_{n} \right) &= \lim_{n \rightarrow \infty}{\left( \frac{1}{2} \right)^{n - 1}\left( b_{1} - a_{1} \right)}\\
&= 2\left( b_{1} - a_{1} \right)\lim_{n \rightarrow \infty}\left( \frac{1}{2} \right)^{n}\\
&= 2\left( b_{1} - a_{1} \right) \cdot 0 = 0
\end{align*}
以上より、区間縮小法より全ての有界閉区間たち$\left[ a_{n},b_{n} \right]$の共通部分$\bigcap_{n \in \mathbb{N}} \left[ a_{n},b_{n} \right]$は1つの$\lim_{n \rightarrow \infty}a_{n} = \lim_{n \rightarrow \infty}b_{n} = c$なる実数$c$を含む。\par
ここで、その関数$f$は有界閉区間$[ a,b]$で連続であるかつ、$\forall n \in \mathbb{N}$に対し、$f\left( a_{n} \right) \leq \gamma \leq f\left( b_{n} \right)$が成り立つのであったので、はさみうちの原理より次のようになる。
\begin{align*}
\left\{ \begin{matrix}
\lim_{n \rightarrow \infty}{f\left( a_{n} \right)} = \lim_{a_{n} \rightarrow c}{f\left( a_{n} \right)} = f(c) \\
\lim_{n \rightarrow \infty}{f\left( b_{n} \right)} = \lim_{b_{n} \rightarrow c}{f\left( b_{n} \right)} = f(c) \\
\end{matrix} \right.\  \Rightarrow \lim_{n \rightarrow \infty}{f\left( a_{n} \right)} = \lim_{n \rightarrow \infty}{f\left( b_{n} \right)} = f(c) = \gamma
\end{align*}
よって、$f(c) = \gamma$なる実数$c$がその有界閉区間$[ a,b]$に存在する。
\end{proof}
\begin{thm}\label{4.1.12.10}
$D(f) \subseteq \mathbb{R}$で$a,b \in{}^{*}\mathbb{R}$なる開区間$(a,b)$で連続な関数$f:D(f)\mathbf{\rightarrow R}$が与えられたとき、互いに異なる2つの実数たち$\lim_{x \rightarrow a + 0}{f(x)}、\lim_{x \rightarrow b - 0}{f(x)}$が存在するなら、$\lim_{x \rightarrow a + 0}{f(x)} \lessgtr \gamma \lessgtr \lim_{x \rightarrow b - 0}{f(x)}$なる任意の実数$\gamma$に対し$f(z) = \gamma$なる実数$z$がその開区間$(a,b)$に存在する。
\end{thm}
\begin{proof}
$D(f) \subseteq \mathbb{R}$で$a,b \in{}^{*}\mathbb{R}$なる開区間$(a,b)$で連続な関数$f:D(f)\mathbf{\rightarrow R}$が与えられたとき、互いに異なる2つの実数たち$\lim_{x \rightarrow a + 0}{f(x)}、\lim_{x \rightarrow b - 0}{f(x)}$が存在するとき、任意の実数$\gamma$に対し、$\lim_{x \rightarrow a + 0}{f(x)} < \gamma < \lim_{x \rightarrow b - 0}{f(x)}$が成り立つなら、$f(x) < \gamma < f(y)$なる2つの実数たち$x$、$y$がその開区間$(a,b)$に存在する。ここで、有界閉区間$[ x,y]$または$[ y,x]$を考え中間値の定理より$f(x) < \gamma < f(y)$なるその実数$\gamma$に対し、$f(z) = \gamma$なる実数$z$がその有界閉区間$[ x,y]$または$[ y,x]$に存在する。ここで、$x,y \in (a,b)$が成り立つことよりその実数$z$がその開区間$(a,b)$に存在する。
\end{proof}
%\hypertarget{ux5f27ux72b6ux9023ux7d50}{%
\subsubsection{弧状連結}%\label{ux5f27ux72b6ux9023ux7d50}}
\begin{dfn}
$n$次元数空間$\mathbb{R}^{n}$の部分集合$A$が与えられたとき、$\forall\mathbf{a},\mathbf{b} \in A$に対し、次を満たすような集合$\mathbb{R}$の部分集合となる有界閉区間$[\alpha,\beta]$と写像$f:[\alpha,\beta] \rightarrow \mathbb{R}^{n}$が存在するとき、その集合$A$は弧状連結であるといいその写像$f$をそれらの元々$\mathbf{a}$、$\mathbf{b}$を結ぶその集合$A$内で結ぶ連続曲線という。
\begin{itemize}
\item
  その写像$f$が有界閉区間$[\alpha,\beta]$で連続である。
\item
  $f(\alpha) = \mathbf{a}$かつ$f(\beta) = \mathbf{b}$が成り立つ。
\item
  $V(f) \subseteq A$が成り立つ。
\end{itemize}
\end{dfn}
\begin{dfn}
$n$次元数空間$\mathbb{R}^{n}$の部分集合$A$が与えられたとき、$\forall\mathbf{a},\mathbf{b} \in A$に対し、これらの2点$\mathbf{a}$、$\mathbf{b}$を端点とする線分$l = \left\{ \mathbf{a} + t\left( \mathbf{b} - \mathbf{a} \right) \middle| t \in [ 0,1] \right\}$が$l \subseteq A$を満たすとき、その集合$A$は凸集合であるなどという。
\end{dfn}
\begin{dfn}
$n$次元数空間$\mathbb{R}^{n}$の部分集合$A$が与えられたとき、$\forall\mathbf{b} \in A$に対し、2点$\mathbf{a}$、$\mathbf{b}$を端点とする線分$l = \left\{ \mathbf{a} + t\left( \mathbf{b} - \mathbf{a} \right) \middle| t \in [ 0,1] \right\}$が$l \subseteq A$を満たすようなその集合$A$の元$\mathbf{a}$が存在するとき、この集合$A$はその点$\mathbf{a}$に関して星形であるという。
\end{dfn}
\begin{thm}\label{4.1.12.11}
凸集合であるかある1点$\mathbf{a}$に関して星形であるような集合$A$は弧状連結である。
\end{thm}
\begin{proof}
凸集合であるかある1点$\mathbf{a}$に関して星形であるような集合$A$が与えられたとき、その集合$A$が凸集合であるとき、$\forall\mathbf{a},\mathbf{b} \in A$に対し、これらの2点$\mathbf{a}$、$\mathbf{b}$を端点とする線分$l = \left\{ \mathbf{a} + t\left( \mathbf{b} - \mathbf{a} \right) \middle| t \in [ 0,1] \right\}$は$l \subseteq A$を満たすのであった。このとき、写像$f:[ 0,1] \rightarrow l;t \mapsto \mathbf{a} + t\left( \mathbf{b} - \mathbf{a} \right)$が考えられれば、明らかにその写像$f$が有界閉区間$[ 0,1]$で連続であるかつ、$f(0) = \mathbf{a}$かつ$f(1) = \mathbf{b}$が成り立つかつ、$V(f) = l \subseteq A$が成り立つので、その集合$A$は弧状連結である。\par
その集合$A$がある1点$\mathbf{a}$に関して星形であるとき、$\forall\mathbf{b}_{1},\mathbf{b}_{2} \in A$に対し、これらの2点$\mathbf{a}$、$\mathbf{b}_{1}$を端点とする線分$l_{1} = \left\{ \mathbf{a} + t\left( \mathbf{b}_{1} - \mathbf{a} \right) \middle| t \in [ 0,1] \right\}$、これらの2点$\mathbf{a}$、$\mathbf{b}_{2}$を端点とする線分$l_{2} = \left\{ \mathbf{a} + t\left( \mathbf{b}_{2} - \mathbf{a} \right) \middle| t \in [ 0,1] \right\}$は$l_{1},l_{2} \subseteq A$を満たすのであった。ここで、写像たち$g:[ - 1,0] \rightarrow [ 0,1];t \rightarrow - t$、$f_{1}:[ 0,1] \rightarrow l;t \mapsto \mathbf{a} + t\left( \mathbf{b}_{1} - \mathbf{a} \right)$、$f_{2}:[ 0,1] \rightarrow l;t \mapsto \mathbf{a} + t\left( \mathbf{b}_{2} - \mathbf{a} \right)$を用いて写像$f:[ - 1,1] \rightarrow l_{1} \cup l_{2};t \mapsto \left\{ \begin{matrix}
\mathbf{a} + t\left( \mathbf{b}_{1} - \mathbf{a} \right) & \mathrm{if} & t \geq 0 \\
\mathbf{a} - t\left( \mathbf{b}_{2} - \mathbf{a} \right) & \mathrm{if} & t < 0 \\
\end{matrix} \right.\ $が考えられれば、明らかにその写像$f$が有界閉区間$[ - 1,1]$で連続であるかつ、$f(0) = \mathbf{a}$かつ$f(1) = \mathbf{b}$が成り立つかつ、$V(f) = l_{1} \cup l_{2} \subseteq A$が成り立つので、その集合$A$は弧状連結である。
\end{proof}
\begin{dfn}
集合$\mathbb{R}$の有界閉区間$[\alpha,\beta]$から$n$次元数空間$\mathbb{R}^{n}$への次を満たすような写像$f$を折線という。
\begin{itemize}
\item
  その写像$f$はその有界閉区間$[\alpha,\beta]$で連続である。
\item
  有限個の$\left\{ \begin{matrix}
  a_{1} = \alpha \\
  \forall i \in \varLambda_{m}\left[ a_{i} \leq a_{i + 1} \right] \\
  a_{m + 1} = \beta \\
  \end{matrix} \right.\ $なる自然数$m$と実数たち$a_{i}$が存在して、$\forall i \in \varLambda_{m}\exists\mathbf{c}_{1},\mathbf{c}_{2} \in \mathbb{R}^{n}$に対し、$f|\left[ a_{i},a_{i + 1} \right]:\left[ a_{i},a_{i + 1} \right] \rightarrow \mathbb{R}^{n};t \mapsto \mathbf{c}_{1}t + \mathbf{c}_{2}$が成り立つ。
\end{itemize}
\end{dfn}
\begin{thm}\label{4.1.12.12}
集合$\mathbb{R}$の有界閉区間$[\alpha,\beta]$から$n$次元数空間$\mathbb{R}^{n}$への折線$f$はそれらの元々$f(\alpha)$、$f(\beta)$を結ぶその集合$A$内で結ぶ連続曲線である。
\end{thm}
\begin{proof}
集合$\mathbb{R}$の有界閉区間$[\alpha,\beta]$から$n$次元数空間$\mathbb{R}^{n}$への折線$f$において、その写像$f$はその閉区間$[\alpha,\beta]$で連続であるかつ、$V(f) \subseteq \mathbb{R}^{n}$が成り立つので、明らかにその写像$f$はその$n$次元数空間$\mathbb{R}^{n}$の元々$f(\alpha)$、$f(\beta)$を結ぶその集合$A$内で結ぶ連続曲線である。
\end{proof}
\begin{thm}\label{4.1.12.13}
$n$次元数空間$\mathbb{R}^{n}$の空でない開集合$U$について、次のことは同値である。
\begin{itemize}
\item
  その集合$U$は連結である。
\item
  その集合$U$の任意の2点$\mathbf{a}$、$\mathbf{b}$はその集合$U$内の折線で結べる。
\item
  その集合$U$は弧状連結である。
\end{itemize}
\end{thm}\par
これは次のようにして示される。
\begin{enumerate}
\item
  まず、その集合$U$が連結であるなら、その集合$U$の任意の2点$\mathbf{a}$、$\mathbf{b}$はその集合$U$内の折線で結べることを示す。
\item
  $\forall\mathbf{a} \in U$に対し、その点$\mathbf{a}$と折線で結べる点全体の集合を$A$とおきそうでない点全体の集合を$B$とおく。
\item
  その集合$A$は空でないかつ、その集合$U$が開集合であるかつ、$\forall\mathbf{a}' \in U\exists\varepsilon \in \mathbb{R}^{+}\forall\mathbf{b}' \in U\left( \mathbf{a}',\varepsilon \right) \subseteq U$に対し、これらの2点$\mathbf{a}'$、$\mathbf{b}'$を端点とする線分$l$は$l \subseteq U\left( \mathbf{a}',\varepsilon \right)$を満たす。
\item
  その開球$U\left( \mathbf{a}',\varepsilon \right)$内で線分が結ばれることができるので、それらの集合たち$A、B$は開集合である。
\item
  4. よりその集合$B$は空集合である。
\item
  2. と5. より、その集合$U$の任意の2点$\mathbf{a}$、$\mathbf{b}$はその集合$U$内の折線で結べる。
\item
  その集合$U$の任意の2点$\mathbf{a}$、$\mathbf{b}$はその集合$U$内の折線で結べるなら、明らかにその集合$U$は弧状連結である。
\item
  最後にその集合$U$は弧状連結であるなら、その集合$U$は連結であることを背理法で示す。
\item
  その集合$U$は弧状連結であるかつ、その集合$U$は連結でないと仮定する。
\item
  $U = A \sqcup B$かつ$A,B \neq \emptyset$なる開集合たち$A$、$B$が与えられたとき、$\forall\mathbf{a} \in A\forall\mathbf{b} \in B$に対し、それらの元々$\mathbf{a}$、$\mathbf{b}$をその集合$U$内で結ぶ始集合が有界閉区間$[\alpha,\beta]$であるような連続曲線$f$が存在できる。
\item
  10. より$\alpha < \beta$が成り立つ。
\item
  $K = \left\{ t \in [\alpha,\beta] \middle| f(t) \in A \right\}$なる集合$K$は空集合でない。
\item
  $U(K) \neq \emptyset$が成り立つ。
\item
  $\alpha \leq \sup K \leq \beta$が成り立つ。
\item
  その写像$f$はその閉区間$[\alpha,\beta]$で連続であることから、$\varepsilon$-$\delta$論法を開球を用いた式に書き換える。
\item
  その集合$A$は開集合であることから、$\exists\varepsilon \in \mathbb{R}^{+}\exists\delta \in \mathbb{R}^{+}$に対し、$V\left( f|[\alpha,\alpha + \delta) \right) \subseteq U(a,\varepsilon) \subseteq A$が成り立つ。
\item
  16. より$\alpha \neq \sup K$が成り立つ。
\item
  15. から17. までと同様にして、$\beta \neq \sup K$が成り立つ。
\item
  14. から18. より$\alpha < \sup K < \beta$が成り立つ。
\item
  $f\left( \sup K \right) \in A$のとき、その写像$f$はその閉区間$[\alpha,\beta]$で連続であることから、$\varepsilon$-$\delta$論法を開球を用いた式に書き換える。
\item
  $\exists\varepsilon \in \mathbb{R}^{+}\exists\delta \in \mathbb{R}^{+}$に対し、$V\left( f|\left( \sup K - \delta,\sup K + \delta \right) \right) \subseteq U\left( f\left( \sup K \right),\varepsilon \right) \subseteq A$が成り立つ。
\item
  $0 < \delta' < \delta$なる実数$\delta'$をとり$\sup K + \delta' \in K$が成り立つことに注意すると、その実数$\sup K$が上限であることに矛盾している。
\item
  $f\left( \sup K \right) \in B$のとき、20. から21. までと同様にして、$\exists\varepsilon \in \mathbb{R}^{+}\exists\delta \in \mathbb{R}^{+}$に対し、$V\left( f|\left( \sup K - \delta,\sup K + \delta \right) \right) \subseteq U\left( f\left( \sup K \right),\varepsilon \right) \subseteq B$が成り立つ。
\item
  $0 < \delta' < \delta$なる実数$\delta'$をとり$\sup K - \delta' \in K$が成り立つことに注意すれば、$f\left( \sup K - \delta' \right) \in A \cap B$が成り立つ。
\item
  24. は$U = A \sqcup B$が成り立つことに矛盾している。
\item
  以上より、その集合$U$は弧状連結であるなら、その集合$U$は連結である。
\end{enumerate}
\begin{proof}
$n$次元数空間$\mathbb{R}^{n}$の空でない連結な開集合$U$が与えられたとき、$\forall\mathbf{a} \in U$に対し、その点$\mathbf{a}$と折線で結べる点全体の集合を$A$とおきそうでない点全体の集合を$B$とおくと、明らかに$U = A \sqcup B$が成り立ち、$\mathbf{a} \in A$が成り立つので、その集合$A$は空でない。ここで、その集合$U$は開集合であるから、$\forall\mathbf{a}' \in U\exists\varepsilon \in \mathbb{R}^{+}$に対し、中心がその点$\mathbf{a}'$で半径が実数$\varepsilon$であるような開球$U\left( \mathbf{a}',\varepsilon \right)$が$U\left( \mathbf{a}',\varepsilon \right) \subseteq U$を満たす。$\forall\mathbf{b}' \in U\left( \mathbf{a}',\varepsilon \right)$に対し、明らかにこれらの2点$\mathbf{a}'$、$\mathbf{b}'$を端点とする線分$l = \left\{ \mathbf{a}' + t\left( \mathbf{b}' - \mathbf{a}' \right) \middle| t \in [ 0,1] \right\}$は$l \subseteq U\left( \mathbf{a}',\varepsilon \right)$を満たす。明らかに$l \subseteq U\left( \mathbf{a}',\varepsilon \right) \subseteq U$が成り立つので、その点$\mathbf{a}'$がその点$\mathbf{a}$とその集合$U$内で折線で結べるならそのときに限り、その点$\mathbf{b}'$がその点$\mathbf{a}$とその集合$U$内で折線で結べる。ここで、$\mathbf{a}' \in A$なら先ほどの議論で開球$U\left( \mathbf{a}',\varepsilon \right)$内で線分が結ばれることができていたので、$U\left( \mathbf{a}',\varepsilon \right) \subseteq A$が成り立つ。$\mathbf{a}' \in B$なら同様にして$U\left( \mathbf{a}',\varepsilon \right) \subseteq B$が成り立つ。したがって、それらの集合たち$A$、$B$は開集合である。ここで、連結の定義より空でない2つの開集合たちの直和とならないかつ、$U = A \sqcup B$が成り立つかつ、その集合$A$は空でないので、その集合$B$は空集合である。これにより、その集合$U$の任意の2点$\mathbf{a}$、$\mathbf{b}$はその集合$U$内の折線で結べる。\par
その集合$U$の任意の2点$\mathbf{a}$、$\mathbf{b}$はその集合$U$内の折線で結べるなら、その折線はその集合$U$内で結ぶ連続曲線であったので、その集合$U$は弧状連結である。\par
その集合$U$は弧状連結であるかつ、その集合$U$は連結でないと仮定しよう。このとき、$U = A \sqcup B$かつ$A,B \neq \emptyset$なる開集合たち$A$、$B$が存在することになる。$\forall\mathbf{a} \in A\forall\mathbf{b} \in B$に対し、仮定よりそれらの元々$\mathbf{a}$、$\mathbf{b}$をその集合$U$内で結ぶ連続曲線$f$が存在できるのであったので、そうするとき、集合$\mathbb{R}$のある閉区間$[\alpha,\beta]$を用いて$U = A \sqcup B$より$A \cap B = \emptyset$が成り立ち$\mathbf{a} \neq \mathbf{b}$が成り立つかつ、対応$f$は写像で、その閉区間$[\alpha,\beta]$の元が1つ決まると、その集合$U$の元がただ1つ決まるかつ、$f(\alpha) = \mathbf{a}$かつ$f(\beta) = \mathbf{b}$が成り立つので、$\alpha \neq \beta$が得られ、したがって、$\alpha < \beta$が成り立つ。$K = \left\{ t \in [\alpha,\beta] \middle| f(t) \in A \right\}$なる集合$K$が与えられたとき、$\alpha \in K$よりその集合$K$は空集合でなく、$\beta \notin K$かつ$\alpha < \beta$よりその元$\beta$がその集合$K$の上界となるので、$U(K) \neq \emptyset$が成り立ちその集合$K$は上に有界となり、したがって、$\sup K \in \mathbb{R}$なる実数$\sup K$が存在し、定義より明らかに、$\alpha \leq \sup K \leq \beta$が成り立つ。ここで、これらの集合たち$A$、$B$は開集合であるから、$U\left( \mathbf{a},\varepsilon \right) \subseteq A$かつ$U\left( \mathbf{b},\varepsilon \right) \subseteq B$なる実数$\varepsilon$が集合$\mathbb{R}^{+}$に存在する。ここで、その写像$f$はその閉区間$[\alpha,\beta]$で連続であるから、次式が成り立ち、
\begin{align*}
\lim_{t \rightarrow \alpha}{f(t)} = f(\alpha) = \mathbf{a}
\end{align*}
したがって、開球を用いれば、$\varepsilon$-$\delta$論法は、$\forall\varepsilon \in \mathbb{R}^{+}\exists\delta \in \mathbb{R}^{+}$に対し、$V\left( f|U(\alpha,\delta) \cap [\alpha,\beta] \right) \subseteq U\left( \mathbf{a},\varepsilon \right)$が成り立つことと同値である。ここで、次のようになり、
\begin{align*}
t \in U(\alpha,\delta) \cap [\alpha,\beta] &\Leftrightarrow |t - \alpha| < \delta \land \alpha \leq t \leq \beta\\
&\Leftrightarrow \alpha - \delta < t < \alpha + \delta \land \alpha \leq t \leq \beta\\
&\Leftrightarrow \alpha \leq t < \alpha + \delta \land \alpha \leq t \leq \beta
\end{align*}
さらに、その実数$\delta$をもっと小さくとることができ、そうすれば、$\alpha + \delta \leq \beta$が成り立つので、次のようになる。
\begin{align*}
t \in U(\alpha,\delta) \cap [\alpha,\beta] &\Leftrightarrow \alpha \leq t < \alpha + \delta\\
&\Leftrightarrow t \in [\alpha,\alpha + \delta)
\end{align*}
したがって、$\forall\varepsilon \in \mathbb{R}^{+}\exists\delta \in \mathbb{R}^{+}$に対し、$V\left( f|[\alpha,\alpha + \delta) \right) \subseteq U\left( \mathbf{a},\varepsilon \right)$が成り立つ。また、その集合$A$は開集合であるので、$\exists\varepsilon \in \mathbb{R}^{+}\exists\delta \in \mathbb{R}^{+}$に対し、$V\left( f|[\alpha,\alpha + \delta) \right) \subseteq U\left( \mathbf{a},\varepsilon \right) \subseteq A$が成り立つ。これにより、$\alpha + \delta \in K$となりその実数$\alpha$より大きいその集合$K$の元が必ず存在できるので、$\alpha \neq \sup K$が得られる。同様にして、$\sup K \neq \beta$が得られる。先ほどで$\alpha \leq \sup K \leq \beta$が成り立つのであったので、$\alpha < \sup K < \beta$が成り立つ。\par
$\sup K \in [\alpha,\beta]$より$f\left( \sup K \right) \in U = A \sqcup B$が成り立つので、$f\left( \sup K \right) \in A$または$f\left( \sup K \right) \in B$が成り立つかつ、$f\left( \sup K \right) \notin A \cap B$が成り立たないことになる。\par
$f\left( \sup K \right) \in A$のとき、その写像$f$はその閉区間$[\alpha,\beta]$で連続であるから、次式が成り立ち
\begin{align*}
\lim_{t \rightarrow \sup K}{f(t)} = f\left( \sup K \right)
\end{align*}
したがって、開球を用いれば、$\varepsilon$-$\delta$論法は、$\forall\varepsilon \in \mathbb{R}^{+}\exists\delta \in \mathbb{R}^{+}$に対し、$V\left( f|U\left( \sup K,\delta \right) \cap [\alpha,\beta] \right) \subseteq U\left( f\left( \sup K \right),\varepsilon \right)$が成り立つことと同値である。その集合$A$は開集合であるので、$\exists\varepsilon \in \mathbb{R}^{+}\exists\delta \in \mathbb{R}^{+}$に対し、$V\left( f|U\left( \sup K,\delta \right) \cap [\alpha,\beta] \right) \subseteq U\left( f\left( \sup K \right),\varepsilon \right) \subseteq A$が成り立つ。ここで、さらに、その実数$\delta$をもっと小さくとることができ、そうすれば、$\alpha \leq \sup K - \delta$かつ$\sup K + \delta \leq \beta$が成り立つので、
\begin{align*}
t \in U\left( \sup K,\delta \right) \cap [\alpha,\beta] &\Leftrightarrow \left| t - \sup K \right| < \delta \land \alpha \leq t \leq \beta\\
&\Leftrightarrow \sup K - \delta < t < \sup K + \delta \land \alpha \leq t \leq \beta\\
&\Leftrightarrow \sup K - \delta < t < \sup K + \delta\\
&\Leftrightarrow t \in \left( \sup K - \delta,\sup K + \delta \right)
\end{align*}
したがって、$\exists\varepsilon \in \mathbb{R}^{+}\exists\delta \in \mathbb{R}^{+}$に対し、$V\left( f|\left( \sup K - \delta,\sup K + \delta \right) \right) \subseteq U\left( f\left( \sup K \right),\varepsilon \right) \subseteq A$が成り立つ。これにより、$0 < \delta' < \delta$なる実数$\delta'$がとられれば、$\sup K + \delta' \in K$が成り立つが、$\sup K < \sup K + \delta'$が成り立つので、その実数$\sup K$が上限であることに矛盾する。\par
$f\left( \sup K \right) \in B$のとき、同様にして、$\forall\varepsilon \in \mathbb{R}^{+}\exists\delta \in \mathbb{R}^{+}$に対し、$V\left( f|U\left( \sup K,\delta \right) \cap [\alpha,\beta] \right) \subseteq U\left( f\left( \sup K \right),\varepsilon \right)$が成り立つ。その集合$B$は開集合であるので、$\exists\varepsilon \in \mathbb{R}^{+}\exists\delta \in \mathbb{R}^{+}$に対し、$V\left( f|U\left( \sup K,\delta \right) \cap [\alpha,\beta] \right) \subseteq U\left( f\left( \sup K \right),\varepsilon \right) \subseteq B$が成り立つ。ここで、同様にして、$\exists\varepsilon \in \mathbb{R}^{+}\exists\delta \in \mathbb{R}^{+}$に対し、$V\left( f|\left( \sup K - \delta,\sup K + \delta \right) \right) \subseteq U\left( f\left( \sup K \right),\varepsilon \right) \subseteq B$が成り立つ。これにより、$0 < \delta' < \delta$なる実数$\delta'$をとれば、$\sup K - \delta' \in \left( \sup K - \delta,\sup K + \delta \right)$が成り立つかつ、$\sup K - \delta' \in K$が成り立つので、$f\left( \sup K - \delta' \right) \in A$が成り立つが、$f\left( \sup K - \delta' \right) \in V\left( f|U\left( \sup K,\delta \right) \cap [\alpha,\beta] \right) \subseteq B$が成り立つので、次式が成り立ち
\begin{align*}
f\left( \sup K - \delta' \right) \in A \land f\left( \sup K - \delta' \right) \in B \Leftrightarrow f\left( \sup K - \delta' \right) \in A \cap B
\end{align*}
その集合$A \cap B$は空集合でなくなり$U = A \sqcup B$に矛盾する。\par
以上より、その集合$U$は弧状連結であるなら、その集合$U$は連結である。
\end{proof}
\begin{thebibliography}{50}
\bibitem{1}
  杉浦光夫, 解析入門I, 東京大学出版社, 1980. 第34刷 p68-78 ISBN978-4-13-062005-5
\bibitem{2}
  松坂和夫, 集合・位相入門, 岩波書店, 1968. 新装版第2刷 p200-206 ISBN978-4-00-029871-1
\end{thebibliography}
\end{document}
  