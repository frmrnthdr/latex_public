\documentclass[dvipdfmx]{jsarticle}
\setcounter{section}{6}
\setcounter{subsection}{5}
\usepackage{xr}
\externaldocument{4.1.8}
\externaldocument{4.5.4}
\externaldocument{4.5.5}
\externaldocument{4.6.1}
\externaldocument{4.6.3}
\usepackage{amsmath,amsfonts,amssymb,array,comment,mathtools,url,docmute}
\usepackage{longtable,booktabs,dcolumn,tabularx,mathtools,multirow,colortbl,xcolor}
\usepackage[dvipdfmx]{graphics}
\usepackage{bmpsize}
\usepackage{amsthm}
\usepackage{enumitem}
\setlistdepth{20}
\renewlist{itemize}{itemize}{20}
\setlist[itemize]{label=•}
\renewlist{enumerate}{enumerate}{20}
\setlist[enumerate]{label=\arabic*.}
\setcounter{MaxMatrixCols}{20}
\setcounter{tocdepth}{3}
\newcommand{\rotin}{\text{\rotatebox[origin=c]{90}{$\in $}}}
\renewcommand{\thesection}{第\arabic{section}部}
\renewcommand{\thesubsection}{\arabic{section}.\arabic{subsection}}
\renewcommand{\thesubsubsection}{\arabic{section}.\arabic{subsection}.\arabic{subsubsection}}
\everymath{\displaystyle}
\allowdisplaybreaks[4]
\usepackage{vtable}
\theoremstyle{definition}
\newtheorem{thm}{定理}[subsection]
\newtheorem*{thm*}{定理}
\newtheorem{dfn}{定義}[subsection]
\newtheorem*{dfn*}{定義}
\newtheorem{axs}[dfn]{公理}
\newtheorem*{axs*}{公理}
\renewcommand{\headfont}{\bfseries}
\makeatletter
  \renewcommand{\section}{%
    \@startsection{section}{1}{\z@}%
    {\Cvs}{\Cvs}%
    {\normalfont\huge\headfont\raggedright}}
\makeatother
\makeatletter
  \renewcommand{\subsection}{%
    \@startsection{subsection}{2}{\z@}%
    {0.5\Cvs}{0.5\Cvs}%
    {\normalfont\LARGE\headfont\raggedright}}
\makeatother
\makeatletter
  \renewcommand{\subsubsection}{%
    \@startsection{subsubsection}{3}{\z@}%
    {0.4\Cvs}{0.4\Cvs}%
    {\normalfont\Large\headfont\raggedright}}
\makeatother
\makeatletter
\renewenvironment{proof}[1][\proofname]{\par
  \pushQED{\qed}%
  \normalfont \topsep6\p@\@plus6\p@\relax
  \trivlist
  \item\relax
  {
  #1\@addpunct{.}}\hspace\labelsep\ignorespaces
}{%
  \popQED\endtrivlist\@endpefalse
}
\makeatother
\renewcommand{\proofname}{\textbf{証明}}
\usepackage{tikz,graphics}
\usepackage[dvipdfmx]{hyperref}
\usepackage{pxjahyper}
\hypersetup{
 setpagesize=false,
 bookmarks=true,
 bookmarksdepth=tocdepth,
 bookmarksnumbered=true,
 colorlinks=false,
 pdftitle={},
 pdfsubject={},
 pdfauthor={},
 pdfkeywords={}}
\begin{document}
%\hypertarget{lebesgueux7a4dux5206}{%
\subsection{Lebesgue積分}%\label{lebesgueux7a4dux5206}}
%\hypertarget{riemannux7a4dux5206ux306eux7b2c1ux5e73ux5747ux5024ux306eux5b9aux7406}{%
\subsubsection{Riemann積分の第1平均値の定理}%\label{riemannux7a4dux5206ux306eux7b2c1ux5e73ux5747ux5024ux306eux5b9aux7406}}
\begin{thm}[Riemann積分の第1平均値の定理]\label{4.6.6.1}
$\forall E \in \mathfrak{F}_{m}$に対し、連続な関数$f:\mathrm{cl}E \rightarrow \mathbb{R}$、Lebesgue測度空間においてその閉包$\mathrm{cl}E$で定積分可能で$0 \leq g$な関数$g:\mathbb{R}^{n} \rightarrow \mathbb{R}$が与えられたとき、$\exists\mathbf{x} \in \mathrm{cl}E$に対し、次式が成り立つ。
\begin{align*}
\int_{\mathrm{cl}E} {fg\lambda} = f\left( \mathbf{x} \right)\int_{\mathrm{cl}E} {g\lambda}
\end{align*}
特に、$g = 1$とすれば次式が成り立つ。
\begin{align*}
\int_{\mathrm{cl}E} {f\lambda} = f\left( \mathbf{x} \right)\lambda\left( \mathrm{cl}E \right)
\end{align*}
この定理をRiemann積分の第1平均値の定理という。
\end{thm}
\begin{proof}
$\forall E \in \mathfrak{F}_{m}$に対し、連続な関数$f:\mathrm{cl}E \rightarrow \mathbb{R}$、Lebesgue測度空間で定積分可能で$0 \leq g$な関数$g:\mathbb{R}^{n} \rightarrow \mathbb{R}$が与えられたとき、定理\ref{4.1.8.7}、即ち、最大値最小値の定理よりその関数$f$の最大値$\max f$、最小値$\min f$が存在するので、その関数$f$は有界である。したがって、定理\ref{4.6.1.33}、即ち、積分の第1平均値の定理よりその関数$fg$はその集合$\mathrm{cl}E$で積分可能で$\inf{f|\mathrm{cl}E} \leq c \leq \sup{f|\mathrm{cl}E}$なるある実数$c$が存在して、$0 \leq g$が成り立つので、次式が成り立つ。
\begin{align*}
\int_{\mathrm{cl}E} {fg\lambda} = c\int_{\mathrm{cl}E} {g\lambda}
\end{align*}
このとき、$\inf{f|\mathrm{cl}E} = \min f$、$\sup{f|\mathrm{cl}E} = \max f$が成り立つので、中間値の定理の拡張より$\exists\mathbf{x} \in \mathrm{cl}E$に対し、$c = f\left( \mathbf{x} \right)$が成り立つ\footnote{中間値の定理の拡張は次のことを主張する定理です。
\begin{quote}
連結な位相空間$\left( S,\mathfrak{O} \right)$から1次元Euclid空間$E$における位相空間$\left( \mathbb{R},\mathfrak{O}_{d_{E}} \right)$への連続写像$f:S \rightarrow \mathbb{R}$について、$\forall a,b \in S$に対し、$f(a) < f(b)$が成り立つなら、$\forall\gamma \in \left[ f(a),f(b) \right]$に対し、$f(c) = \gamma$なるその集合$S$の元$c$が存在する。
\end{quote}
なお、定理\ref{4.1.8.8}より強いことを主張していることに注意です。証明は位相空間論の内容をかなり要求するので、省略いたします。}。したがって、次のようになる。
\begin{align*}
\int_{\mathrm{cl}E} {fg\lambda} = f\left( \mathbf{x} \right)\int_{\mathrm{cl}E} {g\lambda}
\end{align*}
特に、$g = 1$とすれば次式が成り立つ。
\begin{align*}
\int_{\mathrm{cl}E} {f\lambda} = f\left( \mathbf{x} \right)\lambda\left( \mathrm{cl}E \right)
\end{align*}
\end{proof}
%\hypertarget{lebesgueux7a4dux5206ux3068urysohnux306eux88dcux984c}{%
\subsubsection{Lebesgue積分とUrysohnの補題}%\label{lebesgueux7a4dux5206ux3068urysohnux306eux88dcux984c}}
\begin{thm}\label{4.6.6.2}
Lebesgue測度空間において$E \in \mathfrak{M}_{C}\left( \lambda^{*} \right)$なる集合$E$で定積分可能な関数$f:\mathbb{R}^{n} \rightarrow \mathbb{R}$が与えられたとき、$\forall\varepsilon \in \mathbb{R}^{+}$に対し、連続であり、ある有界な開集合$O$と、$\mathbf{x} \notin O$が成り立つなら、$f_{\varepsilon}\left( \mathbf{x} \right) = 0$が成り立つようなその関数$f_{\varepsilon}:\mathbb{R}^{n} \rightarrow \mathbb{R}$が存在して、次式が成り立つ。
\begin{align*}
\int_{E} {\left| f - f_{\varepsilon} \right|\lambda} < \varepsilon
\end{align*}
\end{thm}
\begin{proof}
Lebesgue測度空間において、$E \in \mathfrak{M}_{C}\left( \lambda^{*} \right)$なる集合$E$で定積分可能な関数$f:\mathbb{R}^{n} \rightarrow \mathbb{R}$が与えられたとき、$0 \leq f$と仮定すれば、定理\ref{4.5.5.18}における非負可測関数の非負単関数の列による近似での列$\left( (f)_{m} \right)_{m \in \mathbb{N}}$を用いて、点$\mathbf{0}$を中心とした半径$m$の開球が$U\left( \mathbf{0},m \right)$とおかれれば、$\forall m \in \mathbb{N}$に対し、$U\left( \mathbf{0},m \right) \subseteq U\left( \mathbf{0},m + 1 \right)$が成り立つので、$0 \leq \chi_{U\left( \mathbf{0},m \right)} \leq \chi_{U\left( \mathbf{0},m + 1 \right)}$が成り立つかつ、$0 \leq (f)_{m} \leq (f)_{m + 1}$も成り立つので、$(f)_{m}\chi_{U\left( \mathbf{0},m \right)} \leq (f)_{m + 1}\chi_{U\left( \mathbf{0},m + 1 \right)}$が成り立つ。これにより、その関数の列$\left( (f)_{m}\chi_{U\left( \mathbf{0},m \right)} \right)_{m \in \mathbb{N}}$は単調増加する。さらに、前述のことに加えて$\lim_{m \rightarrow \infty}\chi_{U\left( \mathbf{0},m \right)} = 1$が成り立つことに注意すれば、次のようになる。
\begin{align*}
(f)_{m}\chi_{U\left( \mathbf{0},m \right)} &\leq \sup\left\{ (f)_{m}\chi_{U\left( \mathbf{0},m \right)} \right\}_{m \in \mathbb{N}}\\
&= \lim_{m \rightarrow \infty}{(f)_{m}\chi_{U\left( \mathbf{0},m \right)}}\\
&= \lim_{m \rightarrow \infty}(f)_{m}\lim_{m \rightarrow \infty}\chi_{U\left( \mathbf{0},m \right)}\\
&= f \cdot 1 = f
\end{align*}
したがって、定理\ref{4.6.1.26}、即ち、単調収束定理より次のようになる。
\begin{align*}
\lim_{m \rightarrow \infty}{\int_{E} {\left| f - (f)_{m}\chi_{U\left( \mathbf{0},m \right)} \right|\lambda}} &= \lim_{m \rightarrow \infty}{\int_{E} {\left| \sup\left\{ (f)_{m}\chi_{U\left( \mathbf{0},m \right)} \right\}_{m \in \mathbb{N}} - (f)_{m}\chi_{U\left( \mathbf{0},m \right)} \right|\lambda}}\\
&= \lim_{m \rightarrow \infty}\left( \int_{E} {\sup\left\{ (f)_{m}\chi_{U\left( \mathbf{0},m \right)} \right\}_{m \in \mathbb{N}}\lambda} - \int_{E} {(f)_{m}\chi_{U\left( \mathbf{0},m \right)}\lambda} \right)\\
&= \int_{E} {\sup\left\{ (f)_{m}\chi_{U\left( \mathbf{0},m \right)} \right\}_{m \in \mathbb{N}}\lambda} - \lim_{m \rightarrow \infty}{\int_{E} {(f)_{m}\chi_{U\left( \mathbf{0},m \right)}\lambda}}\\
&= \lim_{m \rightarrow \infty}{\int_{E} {(f)_{m}\chi_{U\left( \mathbf{0},m \right)}\lambda}} - \lim_{m \rightarrow \infty}{\int_{E} {(f)_{m}\chi_{U\left( \mathbf{0},m \right)}\lambda}} = 0
\end{align*}\par
これにより、$\forall\varepsilon \in \mathbb{R}^{+}$に対し、ある自然数$m_{0}$が存在して、$m_{0} \leq m$が成り立つなら、次式が成り立つ。
\begin{align*}
0 \leq \int_{E} {\left| f - (f)_{m}\chi_{U\left( \mathbf{0},m \right)} \right|\lambda} = \left| \int_{E} {\left| f - (f)_{m}\chi_{U\left( \mathbf{0},m \right)} \right|\lambda} \right| < \frac{\varepsilon}{2}
\end{align*}
そこで、その関数$(f)_{m}\chi_{U\left( \mathbf{0},m \right)}$は単関数同士の積であるから、単関数となって$0 < \alpha_{i}$、$E_{i} \subseteq U\left( \mathbf{0},m \right)$として次のようにおかれることができる。
\begin{align*}
(f)_{m}\chi_{U\left( \mathbf{0},m \right)} = \sum_{i \in \varLambda_{k}} {\alpha_{i}\chi_{E_{i}}}
\end{align*}
なお、$E = \bigsqcup_{i \in \varLambda_{k}} E_{i}$が成り立つ。ここで、$\forall i \in \varLambda_{k}$に対し、定理\ref{4.5.4.14}と定理\ref{4.5.4.16}より開集合$O_{i}$と閉集合$A_{i}$が存在して、次式が成り立つ。
\begin{align*}
A_{i} \subseteq E_{i} \subseteq O_{i},\ \ \lambda\left( O_{i} \setminus E_{i} \right) < \frac{\varepsilon}{4k\alpha_{i}},\ \ \lambda\left( E_{i} \setminus A_{i} \right) < \frac{\varepsilon}{4k\alpha_{i}}
\end{align*}
ここで、$A_{i} \subseteq E_{i} \subseteq O_{i}$かつその集合$E_{i}$が有界であることにより、$\lambda\left( E_{i} \right) < \infty$が成り立つので、次のようになる。
\begin{align*}
\lambda\left( O_{i} \setminus E_{i} \right) + \lambda\left( E_{i} \setminus A_{i} \right) &= \lambda\left( O_{i} \right) - \lambda\left( E_{i} \right) + \lambda\left( E_{i} \right) - \lambda\left( A_{i} \right)\\
&= \lambda\left( O_{i} \right) - \lambda\left( A_{i} \right)\\
&= \lambda\left( O_{i} \setminus A_{i} \right)\\
&< \frac{\varepsilon}{4k\alpha_{i}} + \frac{\varepsilon}{4k\alpha_{i}} = \frac{\varepsilon}{2k\alpha_{i}}
\end{align*}
これにより、$\lambda\left( O_{i} \setminus A_{i} \right) < \frac{\varepsilon}{2k\alpha_{i}}$が得られる。\par
そこで、差集合$\mathbb{R}^{n} \setminus O_{i}$は閉集合で$A_{i} \subseteq O_{i}$より$\forall\mathbf{a} \in \mathbb{R}^{n}$に対し、$\mathbf{a} \in A_{i}$が成り立つなら、$\mathbf{a} \in O_{i}$が成り立つことになり、これの否定が$\mathbf{a} \in A_{i}$かつ$\mathbf{a} \notin O_{i}$が成り立つことなので、$A_{i} \cap \mathbb{R}^{n} \setminus O_{i} = \emptyset$が成り立つ。もちろん、$n$次元Euclid空間$E^{n}$における位相空間$\left( \mathbb{R}^{n},\mathfrak{O}_{d_{E^{n}}} \right)$は$\mathrm{T}_{4}$-空間なので、Urysohnの補題よりその位相空間$\left( \mathbb{R}^{n},\mathfrak{O}_{d_{E^{n}}} \right)$から1次元Euclid空間における位相空間$\left( \mathbb{R},\mathfrak{O}_{d_{E}} \right)$への連続な関数$g_{i}:\mathbb{R}^{n} \rightarrow \mathbb{R}$で次のことを満たすようなものが存在する\footnote{Urysohnの補題は次のことを主張する定理です。
  \begin{quote}
  $\mathrm{T}_{4}$-空間$\left( S,\mathfrak{O} \right)$の閉集合系を$\mathfrak{A}$とおくとき、$\forall A,B \in \mathfrak{A}$に対し、$A \cap B = \emptyset$が成り立つなら、その位相空間$\left( S,\mathfrak{O} \right)$から1次元Euclid空間における位相空間$\left( \mathbb{R},\mathfrak{O}_{d_{E}} \right)$への連続写像$f:S \rightarrow \mathbb{R}$で次のことを満たすようなものが存在する。
  \begin{itemize}
  \item
    $\forall a \in A$に対し、$f(a) = 0$が成り立つかつ、$\forall b \in B$に対し、$f(b) = 1$が成り立つ。
  \item
    $\forall c \in S$に対し、$0 \leq f(c) \leq 1$が成り立つ。
  \end{itemize}
  \end{quote}
なお、証明は大変なので割愛させていただきます。}。
\begin{itemize}
\item
  $\forall\mathbf{x} \in A_{i}$に対し、$g_{i}\left( \mathbf{x} \right) = 0$が成り立つかつ、$\forall\mathbf{x} \in \mathbb{R}^{n} \setminus O_{i}$に対し、$g_{i}\left( \mathbf{x} \right) = 1$が成り立つ。
\item
  $\forall\mathbf{x} \in \mathbb{R}^{n}$に対し、$0 \leq g_{i}\left( \mathbf{x} \right) \leq 1$が成り立つ。
\end{itemize}\par
そこで、次のように関数$f_{\varepsilon}$がおかれれば、
\begin{align*}
f_{\varepsilon} = \sum_{i \in \varLambda_{k}} {\alpha_{i}g_{i}}
\end{align*}
これは連続で、$\mathbf{x} \notin O_{i} \setminus A_{i}$のとき、$\mathbf{x} \notin O_{i} \setminus A_{i}$が成り立つならそのときに限り、$\mathbf{x} \in A_{i} \sqcup \mathbb{R}^{n} \setminus O_{i}$が成り立つので、$\mathbf{x} \in A_{i}$のとき、$A_{i} \subseteq E_{i}$より$\mathbf{x} \in E_{i}$が成り立つかつ、その関数$g_{i}$の定義より$\chi_{E_{i}}\left( \mathbf{x} \right) = g_{i}\left( \mathbf{x} \right) = 1$が成り立つ。$\mathbf{x} \in \mathbb{R}^{n} \setminus O_{i}$のとき、$E_{i} \subseteq O_{i}$より$\mathbf{x} \in \mathbb{R}^{n} \setminus E_{i}$が成り立つかつ、その関数$g_{i}$の定義より$\chi_{E_{i}}\left( \mathbf{x} \right) = g_{i}\left( \mathbf{x} \right) = 0$が成り立つ。以上の議論により、$\mathbf{x} \notin O_{i} \setminus A_{i}$のとき、$\chi_{E_{i}} = g_{i}$が成り立つことになる。\par
したがって、$\left| \chi_{E_{i}} - g_{i} \right| \leq I_{\mathbb{R}^{n}}$が成り立つことに注意すれば、次のようになる。
\begin{align*}
\int_{E} {\left| (f)_{m}\chi_{U\left( \mathbf{0},m \right)} - f_{\varepsilon} \right|\lambda} &= \int_{E} {\left| \sum_{i \in \varLambda_{k}} {\alpha_{i}\chi_{E_{i}}} - \sum_{i \in \varLambda_{k}} {\alpha_{i}g_{i}} \right|\lambda}\\
&= \int_{E} {\left| \sum_{i \in \varLambda_{k}} {\alpha_{i}\left( \chi_{E_{i}} - g_{i} \right)} \right|\lambda}\\
&\leq \int_{E} {\sum_{i \in \varLambda_{k}} {\alpha_{i}\left| \chi_{E_{i}} - g_{i} \right|}\lambda}\\
&= \sum_{i \in \varLambda_{k}} {\alpha_{i}\int_{E} {\left| \chi_{E_{i}} - g_{i} \right|\lambda}}\\
&= \sum_{i \in \varLambda_{k}} {\alpha_{i}\int_{E \setminus \left( O_{i} \setminus A_{i} \right) \sqcup O_{i} \setminus A_{i}} {\left| \chi_{E_{i}} - g_{i} \right|\lambda}}\\
&= \sum_{i \in \varLambda_{k}} {\alpha_{i}\left( \int_{E \setminus \left( O_{i} \setminus A_{i} \right)} {\left| \chi_{E_{i}} - g_{i} \right|\lambda} + \int_{O_{i} \setminus A_{i}} {\left| \chi_{E_{i}} - g_{i} \right|\lambda} \right)}\\
&= \sum_{i \in \varLambda_{k}} {\alpha_{i}\left( \int_{E \setminus \left( O_{i} \setminus A_{i} \right)} {0\lambda} + \int_{O_{i} \setminus A_{i}} {\left| \chi_{E_{i}} - g_{i} \right|\lambda} \right)}\\
&= \sum_{i \in \varLambda_{k}} {\alpha_{i}\int_{O_{i} \setminus A_{i}} {\left| \chi_{E_{i}} - g_{i} \right|\lambda}}\\
&\leq \sum_{i \in \varLambda_{k}} {\alpha_{i}\int_{O_{i} \setminus A_{i}} {I_{\mathbb{R}^{n}}\lambda}}\\
&= \sum_{i \in \varLambda_{k}} {\alpha_{i}\int_{O_{i} \setminus A_{i}} \lambda}\\
&= \sum_{i \in \varLambda_{k}} {\alpha_{i}\lambda\left( O_{i} \setminus A_{i} \right)}\\
&< \sum_{i \in \varLambda_{k}} {\alpha_{i} \cdot \frac{\varepsilon}{2k\alpha_{i}}}\\
&= \sum_{i \in \varLambda_{k}} \frac{\varepsilon}{2k} = \frac{\varepsilon}{2}
\end{align*}\par
以上の議論により、次のようになる。
\begin{align*}
\int_{E} {\left| f - f_{\varepsilon} \right|\lambda} &= \int_{E} {\left| f - (f)_{m}\chi_{U\left( \mathbf{0},m \right)} + (f)_{m}\chi_{U\left( \mathbf{0},m \right)} - f_{\varepsilon} \right|\lambda}\\
&\leq \int_{E} {\left( \left| f - (f)_{m}\chi_{U\left( \mathbf{0},m \right)} \right| + \left| (f)_{m}\chi_{U\left( \mathbf{0},m \right)} - f_{\varepsilon} \right| \right)\lambda}\\
&= \int_{E} {\left| f - (f)_{m}\chi_{U\left( \mathbf{0},m \right)} \right|\lambda} + \int_{E} {\left| (f)_{m}\chi_{U\left( \mathbf{0},m \right)} - f_{\varepsilon} \right|\lambda}\\
&< \frac{\varepsilon}{2} + \frac{\varepsilon}{2} = \varepsilon
\end{align*}\par
このとき、$O = \bigcup_{i \in \varLambda_{k}} O_{i}$とおかれれば、これは有界な開集合で、$\mathbf{x} \notin O$が成り立つなら、$\forall i \in \varLambda_{k}$に対し、$\mathbf{x} \notin O_{i}$が成り立つことになるので、次のようになる。
\begin{align*}
f_{\varepsilon}\left( \mathbf{x} \right) = \sum_{i \in \varLambda_{k}} {\alpha_{i}g_{i}}\left( \mathbf{x} \right) = \sum_{i \in \varLambda_{k}} {\alpha_{i}g_{i}\left( \mathbf{x} \right)} = \sum_{i \in \varLambda_{k}} {\alpha_{i} \cdot 0} = 0
\end{align*}\par
$E \in \mathfrak{M}_{C}\left( \lambda^{*} \right)$なる集合$E$で定積分可能な関数$f:\mathbb{R}^{n} \rightarrow \mathbb{R}$が与えられたとき、$f = (f)_{+} - (f)_{-}$が成り立つかつ、$0 \leq (f)_{+}$かつ$0 \leq (f)_{-}$が成り立つので、上記の議論により$\forall\varepsilon \in \mathbb{R}^{+}$に対し、連続であり、ある有界な開集合たち$O_{+}$、$O_{-}$と、$\mathbf{x} \notin O_{+}$が成り立つなら、$\left( f_{\varepsilon} \right)_{+}\left( \mathbf{x} \right) = 0$、$\mathbf{x} \notin O_{-}$が成り立つなら、$\left( f_{\varepsilon} \right)_{-}\left( \mathbf{x} \right) = 0$が成り立つようなそれらの関数たち$\left( f_{\varepsilon} \right)_{+}:\mathbb{R}^{n} \rightarrow \mathbb{R}$、$\left( f_{\varepsilon} \right)_{-}:\mathbb{R}^{n} \rightarrow \mathbb{R}$が存在して、次式が成り立つ。
\begin{align*}
\int_{E} {\left| (f)_{+} - \left( f_{\varepsilon} \right)_{+} \right|\lambda} < \frac{\varepsilon}{2},\ \ \int_{E} {\left| (f)_{-} - \left( f_{\varepsilon} \right)_{-} \right|\lambda} < \frac{\varepsilon}{2}
\end{align*}
そこで、$O = O_{+} \cup O_{-}$、$f_{\varepsilon} = \left( f_{\varepsilon} \right)_{+} - \left( f_{\varepsilon} \right)_{-}$とおかれれば、次のようになる。
\begin{align*}
\int_{E} {\left| f - f_{\varepsilon} \right|\lambda} &= \int_{E} {\left| \left( (f)_{+} - (f)_{-} \right) - \left( \left( f_{\varepsilon} \right)_{+} - \left( f_{\varepsilon} \right)_{-} \right) \right|\lambda}\\
&= \int_{E} {\left| (f)_{+} - \left( f_{\varepsilon} \right)_{+} - (f)_{-} + \left( f_{\varepsilon} \right)_{-} \right|\lambda}\\
&\leq \int_{E} {\left( \left| (f)_{+} - \left( f_{\varepsilon} \right)_{+} \right| + \left| - (f)_{-} + \left( f_{\varepsilon} \right)_{-} \right| \right)\lambda}\\
&= \int_{E} {\left| (f)_{+} - \left( f_{\varepsilon} \right)_{+} \right|\lambda} + \int_{E} {\left| (f)_{-} - \left( f_{\varepsilon} \right)_{-} \right|\lambda}\\
&< \frac{\varepsilon}{2} + \frac{\varepsilon}{2} = \varepsilon
\end{align*}
さらに、その集合$O$は有界な開集合で、$\mathbf{x} \notin O$が成り立つなら、$\mathbf{x} \notin O_{+}$かつ$\mathbf{x} \notin O_{-}$が成り立つので、次のようになる。
\begin{align*}
f_{\varepsilon}\left( \mathbf{x} \right) = \left( \left( f_{\varepsilon} \right)_{+} - \left( f_{\varepsilon} \right)_{-} \right)\left( \mathbf{x} \right) = \left( f_{\varepsilon} \right)_{+}\left( \mathbf{x} \right) - \left( f_{\varepsilon} \right)_{-}\left( \mathbf{x} \right) = 0 - 0 = 0
\end{align*}
\end{proof}
%\hypertarget{lebesgueux7a4dux5206ux3068affineux5909ux63db}{%
\subsubsection{Lebesgue積分とaffine変換}%\label{lebesgueux7a4dux5206ux3068affineux5909ux63db}}
\begin{thm}\label{4.6.6.3}
Lebesgue測度空間において$E \in \mathfrak{M}_{C}\left( \lambda^{*} \right)$なる集合$E$で定積分をもつ関数$f:\mathbb{R}^{n} \rightarrow \mathbb{R}$が与えられたとき、$\forall A_{nn} \in \mathrm{GL}_{n}\left( \mathbb{R} \right)\forall\mathbf{b} \in \mathbb{R}^{n}$に対し、関数$f \circ \left( A_{nn}I_{\mathbb{R}^{n}} + \mathbf{b} \right)$も可測で定積分をもち次式が成り立つ。
\begin{align*}
\int_{A_{nn}^{- 1}\left( E - \mathbf{b} \right)} {f \circ \left( A_{nn}I_{\mathbb{R}^{n}} + \mathbf{b} \right)\lambda} = \frac{1}{\left| \det A_{nn} \right|}\int_{E} {f\lambda}
\end{align*}
特に、その行列$A_{nn}$が直交行列であるなら、次式が成り立つ。
\begin{align*}
\int_{A_{nn}^{- 1}\left( E - \mathbf{b} \right)} {f \circ \left( A_{nn}I_{\mathbb{R}^{n}} + \mathbf{b} \right)\lambda} = \int_{E} {f\lambda}
\end{align*}
\end{thm}
\begin{proof}
Lebesgue測度空間において$E \in \mathfrak{M}_{C}\left( \lambda^{*} \right)$なる集合$E$で定積分をもつ関数$f:\mathbb{R}^{n} \rightarrow \mathbb{R}$が与えられたとき、$\forall A_{nn} \in \mathrm{GL}_{n}\left( \mathbb{R} \right)\forall\mathbf{b} \in \mathbb{R}^{n}$に対し、$0 \leq f$と仮定すれば、定理\ref{4.5.5.18}における非負可測関数の非負単関数の列による近似での列$\left( (f)_{m} \right)_{m \in \mathbb{N}}$を用いて、$(f)_{m} = \sum_{i \in \varLambda_{k}} {\alpha_{i}\chi_{E_{i}}}$とおかれれば、$\forall\mathbf{x} \in \mathbb{R}^{n}$に対し、次のようになることから、
\begin{align*}
(f)_{m} \circ \left( A_{nn}I_{\mathbb{R}^{n}} + \mathbf{b} \right)\left( \mathbf{x} \right) &= \sum_{i \in \varLambda_{k}} {\alpha_{i}\chi_{E_{i}}} \circ \left( A_{nn}I_{\mathbb{R}^{n}} + \mathbf{b} \right)\left( \mathbf{x} \right)\\
&= \sum_{i \in \varLambda_{k}} {\alpha_{i}\chi_{E_{i}}\left( \left( A_{nn}I_{\mathbb{R}^{n}} + \mathbf{b} \right)\left( \mathbf{x} \right) \right)}\\
&= \sum_{i \in \varLambda_{k}} {\alpha_{i}\chi_{E_{i}}\left( A_{nn}\mathbf{x} + \mathbf{b} \right)}\\
&= \sum_{i \in \varLambda_{k}} \left\{ \begin{matrix}
\alpha_{i} & \mathrm{if} & A_{nn}\mathbf{x} + \mathbf{b} \in E_{i} \\
0 & \mathrm{if} & A_{nn}\mathbf{x} + \mathbf{b} \notin E_{i} \\
\end{matrix} \right.\ \\
&= \sum_{i \in \varLambda_{k}} \left\{ \begin{matrix}
\alpha_{i} & \mathrm{if} & \mathbf{x} \in A_{nn}^{- 1}\left( E_{i} - \mathbf{b} \right) \\
0 & \mathrm{if} & \mathbf{x} \notin A_{nn}^{- 1}\left( E_{i} - \mathbf{b} \right) \\
\end{matrix} \right.\ \\
&= \sum_{i \in \varLambda_{k}} {\alpha_{i}\chi_{A_{nn}^{- 1}\left( E_{i} - \mathbf{b} \right)}\left( \mathbf{x} \right)}\\
&= \left( \sum_{i \in \varLambda_{k}} {\alpha_{i}\chi_{A_{nn}^{- 1}\left( E_{i} - \mathbf{b} \right)}} \right)\left( \mathbf{x} \right)
\end{align*}
次式が成り立つ。
\begin{align*}
(f)_{m} \circ \left( A_{nn}I_{\mathbb{R}^{n}} + \mathbf{b} \right) = \sum_{i \in \varLambda_{k}} {\alpha_{i}\chi_{A_{nn}^{- 1}\left( E_{i} - \mathbf{b} \right)}}
\end{align*}\par
このとき、その関数$(f)_{m} \circ \left( A_{nn}I_{\mathbb{R}^{n}} + \mathbf{b} \right)$が単関数であることから可測なので、次のようになることにより
\begin{align*}
f \circ \left( A_{nn}I_{\mathbb{R}^{n}} + \mathbf{b} \right) &= \sup\left\{ (f)_{m} \right\}_{m \in \mathbb{N}} \circ \left( A_{nn}I_{\mathbb{R}^{n}} + \mathbf{b} \right)\\
&= \lim_{m \rightarrow \infty}(f)_{m} \circ \left( A_{nn}I_{\mathbb{R}^{n}} + \mathbf{b} \right)\\
&= \sup\left\{ (f)_{m} \circ \left( A_{nn}I_{\mathbb{R}^{n}} + \mathbf{b} \right) \right\}_{m \in \mathbb{N}}
\end{align*}
その関数$f \circ \left( A_{nn}I_{\mathbb{R}^{n}} + \mathbf{b} \right)$も可測である。また、定理\ref{4.5.4.22}より$A_{nn}^{- 1}\left( E_{i} - \mathbf{b} \right) \in \mathfrak{M}_{C}\left( \lambda^{*} \right)$が成り立つ。\par
さらに、$\lambda\left( A_{nn}^{- 1}\left( E_{i} - \mathbf{b} \right) \right) = \frac{\lambda\left( E_{i} \right)}{\left| \det A_{nn} \right|}$が成り立つので、定理\ref{4.6.1.26}、即ち、単調収束定理より次のようになる。
\begin{align*}
\int_{A_{nn}^{- 1}\left( E - \mathbf{b} \right)} {f \circ \left( A_{nn}I_{\mathbb{R}^{n}} + \mathbf{b} \right)\lambda} &= \int_{A_{nn}^{- 1}\left( E - \mathbf{b} \right)} {\sup\left\{ (f)_{m} \right\}_{m \in \mathbb{N}} \circ \left( A_{nn}I_{\mathbb{R}^{n}} + \mathbf{b} \right)\lambda}\\
&= \int_{A_{nn}^{- 1}\left( E - \mathbf{b} \right)} {\lim_{m \rightarrow \infty}(f)_{m} \circ \left( A_{nn}I_{\mathbb{R}^{n}} + \mathbf{b} \right)\lambda}\\
&= \int_{A_{nn}^{- 1}\left( E - \mathbf{b} \right)} {\sup\left\{ (f)_{m} \circ \left( A_{nn}I_{\mathbb{R}^{n}} + \mathbf{b} \right) \right\}_{m \in \mathbb{N}}\lambda}\\
&= \lim_{m \rightarrow \infty}{\int_{A_{nn}^{- 1}\left( E - \mathbf{b} \right)} {(f)_{m} \circ \left( A_{nn}I_{\mathbb{R}^{n}} + \mathbf{b} \right)\lambda}}\\
&= \lim_{m \rightarrow \infty}{\int_{A_{nn}^{- 1}\left( E - \mathbf{b} \right)} {\sum_{i \in \varLambda_{k}} {\alpha_{i}\chi_{A_{nn}^{- 1}\left( E_{i} - \mathbf{b} \right)}}\lambda}}\\
&= \lim_{m \rightarrow \infty}{\sum_{i \in \varLambda_{k}} {\alpha_{i}\lambda\left( A_{nn}^{- 1}\left( E_{i} - \mathbf{b} \right) \right)}}\\
&= \lim_{m \rightarrow \infty}{\sum_{i \in \varLambda_{k}} \frac{\alpha_{i}\lambda\left( E_{i} \right)}{\left| \det A_{nn} \right|}}\\
&= \frac{1}{\left| \det A_{nn} \right|}\lim_{m \rightarrow \infty}{\sum_{i \in \varLambda_{k}} {\alpha_{i}\lambda\left( E_{i} \right)}}\\
&= \frac{1}{\left| \det A_{nn} \right|}\lim_{m \rightarrow \infty}{\int_{E} {\sum_{i \in \varLambda_{k}} {\alpha_{i}\chi_{E_{i}}}\lambda}}\\
&= \frac{1}{\left| \det A_{nn} \right|}\lim_{m \rightarrow \infty}{\int_{E} {(f)_{m}\lambda}}\\
&= \frac{1}{\left| \det A_{nn} \right|}\int_{E} {\sup\left\{ (f)_{m} \right\}_{m \in \mathbb{N}}\lambda}\\
&= \frac{1}{\left| \det A_{nn} \right|}\int_{E} {f\lambda}
\end{align*}\par
$E \in \mathfrak{M}_{C}\left( \lambda^{*} \right)$なる集合$E$で定積分可能な関数$f:\mathbb{R}^{n} \rightarrow \mathbb{R}$が与えられたとき、$f = (f)_{+} - (f)_{-}$が成り立つかつ、$0 \leq (f)_{+}$かつ$0 \leq (f)_{-}$が成り立つので、上記の議論により次のようになる。
\begin{align*}
\int_{A_{nn}^{- 1}\left( E - \mathbf{b} \right)} {f \circ \left( A_{nn}I_{\mathbb{R}^{n}} + \mathbf{b} \right)\lambda} &= \int_{A_{nn}^{- 1}\left( E - \mathbf{b} \right)} {\left( (f)_{+} - (f)_{-} \right) \circ \left( A_{nn}I_{\mathbb{R}^{n}} + \mathbf{b} \right)\lambda}\\
&= \int_{A_{nn}^{- 1}\left( E - \mathbf{b} \right)} {(f)_{+} \circ \left( A_{nn}I_{\mathbb{R}^{n}} + \mathbf{b} \right)\lambda} \\
&\quad - \int_{A_{nn}^{- 1}\left( E - \mathbf{b} \right)} {(f)_{-} \circ \left( A_{nn}I_{\mathbb{R}^{n}} + \mathbf{b} \right)\lambda}\\
&= \frac{1}{\left| \det A_{nn} \right|}\int_{E} {(f)_{+}\lambda} - \frac{1}{\left| \det A_{nn} \right|}\int_{E} {(f)_{-}\lambda}\\
&= \frac{1}{\left| \det A_{nn} \right|}\int_{E} {\left( (f)_{+} - (f)_{-} \right)\lambda}\\
&= \frac{1}{\left| \det A_{nn} \right|}\int_{E} {f\lambda}
\end{align*}\par
特に、その行列$A_{nn}$が直交行列であるなら、unitary行列でもあることに注意してその行列$A_{nn}$のJordan標準形が$J$とおかれれば、次のようになることから\footnote{$K \subseteq \mathbb{C}$なる体$K$上の内積空間$(V,\varPhi)$、等長変換$f:V \rightarrow V$が与えられたとき、その等長変換$f$の任意の固有値$\lambda$は、もしこれが存在するなら、$|\lambda| = 1$を満たすという定理を用いた。ただし、体$\mathbb{C}$は代数的閉体であることに注意されたい。}、
\begin{align*}
\left| \det A_{nn} \right| = \left| \det J \right| = 1
\end{align*}
次式が成り立つ。
\begin{align*}
\int_{A_{nn}^{- 1}\left( E - \mathbf{b} \right)} {f \circ \left( A_{nn}I_{\mathbb{R}^{n}} + \mathbf{b} \right)\lambda} = \int_{E} {f\lambda}
\end{align*}
\end{proof}
\begin{thm}\label{4.6.6.4}
Lebesgue測度空間において$E \in \mathfrak{M}_{C}\left( \lambda^{*} \right)$なる集合$E$で定積分可能な関数$f:\mathbb{R}^{n} \rightarrow \mathbb{R}$が与えられたとき、次式が成り立つ。
\begin{align*}
\lim_{\left\| \mathbf{h} \right\| \rightarrow 0}{\int_{E - \mathbf{h}} {\left| f \circ \left( I_{\mathbb{R}^{n}} + \mathbf{h} \right) - f \right|\lambda}} = 0
\end{align*}
\end{thm}
\begin{proof}
Lebesgue測度空間において$E \in \mathfrak{M}_{C}\left( \lambda^{*} \right)$なる集合$E$で定積分可能な関数$f:\mathbb{R}^{n} \rightarrow \mathbb{R}$が与えられたとき、$\forall\varepsilon \in \mathbb{R}^{+}$に対し、定理\ref{4.6.6.2}より連続であり、ある有界な開集合$O$と、$\mathbf{x} \notin O$が成り立つなら、$f_{\varepsilon} \circ \left( I_{\mathbb{R}^{n}} + \mathbf{h} \right)\left( \mathbf{x} \right) = 0$が成り立つようなその関数$f_{\varepsilon} \circ \left( I_{\mathbb{R}^{n}} + \mathbf{h} \right):\mathbb{R}^{n} \rightarrow \mathbb{R}$が存在して、次式が成り立つ。
\begin{align*}
\int_{E - \mathbf{h}} {\left| f \circ \left( I_{\mathbb{R}^{n}} + \mathbf{h} \right) - f_{\varepsilon} \circ \left( I_{\mathbb{R}^{n}} + \mathbf{h} \right) \right|\lambda} &= \int_{E - \mathbf{h}} {\left| f - f_{\varepsilon} \right| \circ \left( I_{\mathbb{R}^{n}} + \mathbf{h} \right)\lambda}\\
&= \int_{E} {\left| f - f_{\varepsilon} \right|\lambda} < \varepsilon
\end{align*}
そこで、その集合$O$は有界なので、ある自然数$m$が存在して、次のようになる。
\begin{align*}
\left\{ \mathbf{x} \in \mathbb{R}^{n} \middle| \left\| \mathbf{h} \right\| < 1 \land f_{\varepsilon} \circ \left( I_{\mathbb{R}^{n}} + \mathbf{h} \right)\left( \mathbf{x} \right) \neq 0 \right\} &= \left\{ \mathbf{x} \in \mathbb{R}^{n} \middle| \left\| \mathbf{h} \right\| < 1 \land f_{\varepsilon}\left( \mathbf{x} + \mathbf{h} \right) \neq 0 \right\}\\
&\subseteq \left\{ \mathbf{x} \in \mathbb{R}^{n} \middle| \left\| \mathbf{h} \right\| < 1 \land \mathbf{x} + \mathbf{h} \in O \right\}\\
&= \left\{ \mathbf{x} \in \mathbb{R}^{n} \middle| \mathbf{x} + \mathbf{h} \in O + U\left( \mathbf{0},1 \right) \right\} \subseteq U\left( \mathbf{0},m \right)
\end{align*}
このとき、その関数$f_{\varepsilon} \circ \left( I_{\mathbb{R}^{n}} + \mathbf{h} \right)$は有界な閉集合$\mathrm{cl}{U\left( \mathbf{0},m \right)}$で連続であるので、定理\ref{4.1.8.4}と定理\ref{4.1.8.5}、即ち、Heine-Borelの被覆定理よりその集合$\mathrm{cl}{U\left( \mathbf{0},m \right)}$がcompactであることに注意すれば、その写像$f_{\varepsilon}$は一様連続であるので\footnote{次のことを主張する位相幾何学の定理から従う。
  \begin{quote}
  2つの距離空間たち$(S,d)$、$(T,e)$とこれらの間の任意の写像$f:S \rightarrow T$が与えられたとする。その距離空間$(S,d)$における位相空間$\left( S,\mathfrak{O}_{d} \right)$がcompact空間であるとき、その写像$f$が連続であるなら、その写像$f$は一様連続である。
  \end{quote}
}、$\forall\varepsilon \in \mathbb{R}^{+}\exists\delta \in \mathbb{R}^{+}\forall\mathbf{x},\mathbf{h} \in \mathrm{cl}{U\left( \mathbf{0},m \right)}$に対し、$\left\| \mathbf{h} \right\| < \delta \Rightarrow \left| f_{\varepsilon}\left( \mathbf{x} + \mathbf{h} \right) - f_{\varepsilon}\left( \mathbf{x} \right) \right| < \varepsilon$が成り立つ。このとき、$\forall\varepsilon \in \mathbb{R}^{+}\exists\delta \in \mathbb{R}^{+}$に対し、$\left\| \mathbf{h} \right\| < \delta \Rightarrow \sup_{\mathbf{x} \in U\left( \mathbf{0},m \right)}\left| f_{\varepsilon} \circ \left( I_{\mathbb{R}^{n}} + \mathbf{h} \right) - f_{\varepsilon} \right| < \varepsilon$が成り立つので、次のようになる。
\begin{align*}
0 &\leq \lim_{\left\| \mathbf{h} \right\| \rightarrow 0}{\int_{E - \mathbf{h}} {\left| f \circ \left( I_{\mathbb{R}^{n}} + \mathbf{h} \right) - f \right|\lambda}}\\
&\leq \lim_{\left\| \mathbf{h} \right\| \rightarrow 0}{\int_{E - \mathbf{h}} {\sup_{\mathbf{x} \in U\left( \mathbf{0},m \right)}\left| f_{\varepsilon} \circ \left( I_{\mathbb{R}^{n}} + \mathbf{h} \right) - f_{\varepsilon} \right|\lambda}}\\
&= \lim_{\left\| \mathbf{h} \right\| \rightarrow 0}{\sup_{\mathbf{x} \in U\left( \mathbf{0},m \right)}\left| f_{\varepsilon} \circ \left( I_{\mathbb{R}^{n}} + \mathbf{h} \right) - f_{\varepsilon} \right|}\int_{E - \mathbf{h}} \lambda\\
&= \lim_{\left\| \mathbf{h} \right\| \rightarrow 0}{\sup_{\mathbf{x} \in U\left( \mathbf{0},m \right)}\left| f_{\varepsilon} \circ \left( I_{\mathbb{R}^{n}} + \mathbf{h} \right) - f_{\varepsilon} \right|}\lambda\left( E - \mathbf{h} \right) = 0
\end{align*}
したがって、次式が成り立つ。
\begin{align*}
\lim_{\left\| \mathbf{h} \right\| \rightarrow 0}{\int_{E - \mathbf{h}} {\left| f \circ \left( I_{\mathbb{R}^{n}} + \mathbf{h} \right) - f \right|\lambda}} = 0
\end{align*}\par
このとき、次のようになることから\footnote{$\lim_{\left\| \mathbf{h} \right\| \rightarrow 0}{\int_{E - \mathbf{h}} {\left| f \circ \left( I_{\mathbb{R}^{n}} + \mathbf{h} \right) - f \right|\lambda}}$ではその極限が存在するかどうか疑わしくなるため。}、
\begin{align*}
\limsup_{\left\| \mathbf{h} \right\| \rightarrow 0}{\int_{E - \mathbf{h}} {\left| f \circ \left( I_{\mathbb{R}^{n}} + \mathbf{h} \right) - f \right|\lambda}} &= \limsup_{\left\| \mathbf{h} \right\| \rightarrow 0}\int_{E - \mathbf{h}} \left| f \circ \left( I_{\mathbb{R}^{n}} + \mathbf{h} \right) - f_{\varepsilon} \circ \left( I_{\mathbb{R}^{n}} + \mathbf{h} \right) \right. \\
&\quad \left. + f_{\varepsilon} \circ \left( I_{\mathbb{R}^{n}} + \mathbf{h} \right) - f_{\varepsilon} + f_{\varepsilon} - f \right|\lambda\\
&\leq \limsup_{\left\| \mathbf{h} \right\| \rightarrow 0} \left( \int_{E - \mathbf{h}}  \left| f \circ \left( I_{\mathbb{R}^{n}} + \mathbf{h} \right) - f_{\varepsilon} \circ \left( I_{\mathbb{R}^{n}} + \mathbf{h} \right) \right| \lambda \right. \\
&\quad \left. + \int_{E - \mathbf{h}} {\left| f_{\varepsilon} \circ \left( I_{\mathbb{R}^{n}} + \mathbf{h} \right) - f_{\varepsilon} \right|\lambda} + \int_{E - \mathbf{h}}  \left| f_{\varepsilon} - f \right|\lambda \right)\\
&< \limsup_{\left\| \mathbf{h} \right\| \rightarrow 0}\left( \varepsilon + \int_{E - \mathbf{h}} {\left| f_{\varepsilon} \circ \left( I_{\mathbb{R}^{n}} + \mathbf{h} \right) - f_{\varepsilon} \right|\lambda} + \varepsilon \right)\\
&= \varepsilon + \limsup_{\left\| \mathbf{h} \right\| \rightarrow 0}{\int_{E - \mathbf{h}} {\left| f_{\varepsilon} \circ \left( I_{\mathbb{R}^{n}} + \mathbf{h} \right) - f_{\varepsilon} \right|\lambda}} + \varepsilon\\
&= 2\varepsilon + \lim_{\left\| \mathbf{h} \right\| \rightarrow 0}{\int_{E - \mathbf{h}} {\left| f_{\varepsilon} \circ \left( I_{\mathbb{R}^{n}} + \mathbf{h} \right) - f_{\varepsilon} \right|\lambda}} = 2\varepsilon
\end{align*}
その正の実数$\varepsilon$の任意性により次式が成り立つ。
\begin{align*}
\limsup_{\left\| \mathbf{h} \right\| \rightarrow 0}{\int_{E - \mathbf{h}} {\left| f \circ \left( I_{\mathbb{R}^{n}} + \mathbf{h} \right) - f \right|\lambda}} = 0
\end{align*}
このとき、次式が成り立つことから、
\begin{align*}
\liminf_{\left\| \mathbf{h} \right\| \rightarrow 0}{\int_{E - \mathbf{h}} {\left| f \circ \left( I_{\mathbb{R}^{n}} + \mathbf{h} \right) - f \right|\lambda}} = - \limsup_{\left\| \mathbf{h} \right\| \rightarrow 0}\left( - \int_{E - \mathbf{h}} {\left| f \circ \left( I_{\mathbb{R}^{n}} + \mathbf{h} \right) - f \right|\lambda} \right) = 0
\end{align*}
次式が成り立つ。
\begin{align*}
\lim_{\left\| \mathbf{h} \right\| \rightarrow 0}{\int_{E - \mathbf{h}} {\left| f \circ \left( I_{\mathbb{R}^{n}} + \mathbf{h} \right) - f \right|\lambda}} = 0
\end{align*}
\end{proof}
%\hypertarget{lebesgueux7a4dux5206ux3068ux9023ux7d9aux95a2ux6570}{%
\subsubsection{Lebesgue積分と連続関数}%\label{lebesgueux7a4dux5206ux3068ux9023ux7d9aux95a2ux6570}}
\begin{thm}\label{4.6.6.5}
Lebesgue測度空間において、$\forall E,F \in \mathfrak{M}_{C}\left( \lambda^{*} \right)$に対し、$\lambda(F) = 0$が成り立つかつ、その集合$E \setminus F$上で連続な関数$f:\mathbb{R}^{n} \rightarrow{}^{*}\mathbb{R}$はその集合$E$で定積分可能で次式が成り立つ\footnote{ちゃんというのなら、その集合$E \setminus F$上で連続な関数$f:\mathbb{R}^{n} \rightarrow{}^{*}\mathbb{R}$が与えられたとき、その関数$f\chi_{E}$はその集合$E$で定積分可能で次式が成り立つ。
\begin{align*}
\int_{E} {f\chi_{E}\lambda} = \int_{E \setminus F} {f\chi_{E}\lambda}
\end{align*}}。
\begin{align*}
\int_{E} {f\lambda} = \int_{E \setminus F} {f\lambda}
\end{align*}
\end{thm}
\begin{proof}
Lebesgue測度空間において、$\forall E,F \in \mathfrak{M}_{C}\left( \lambda^{*} \right)$に対し、$\lambda(F) = 0$が成り立つかつ、その集合$E \setminus F$上で連続な関数$f:\mathbb{R}^{n} \rightarrow{}^{*}\mathbb{R}$が与えられたとき、定理\ref{4.5.5.25}よりその関数$f\chi_{E \setminus F}$は可測である。このとき、$- \infty < f\chi_{E \setminus F} < \infty$なので、その関数$f\chi_{E \setminus F}$はその集合$\mathbb{R}^{n}$上で定積分可能でありその集合$E \setminus F$上で定積分可能で次のようになる。
\begin{align*}
\int_{\mathbb{R}^{n}} {f\chi_{E \setminus F}\lambda} = \int_{E \setminus F} {f\chi_{E \setminus F}\lambda}
\end{align*}
そこで、定理\ref{4.6.1.21}よりその関数$f\chi_{E \setminus F}$はその集合$E$上で定積分可能であり次のようになるかつ、
\begin{align*}
\int_{\mathbb{R}^{n}} {f\chi_{E \setminus F}\lambda} = \int_{E \setminus F} {f\chi_{E \setminus F}\lambda} = \int_{E} {f\chi_{E \setminus F}\lambda}
\end{align*}
$f\chi_{E}\left( \mathbf{x} \right) = f\chi_{E \setminus F}\left( \mathbf{x} \right)\ \left( \mathbb{R}^{n},\ \ \mathfrak{M}_{C}\left( \lambda^{*} \right),\ \ \lambda \right) \ \text{-} \ \mathrm{a.e.}\ \mathbf{x} \in \mathbb{R}^{n}$が成り立ち定理\ref{4.6.3.5}より次のようになることから、
\begin{align*}
\int_{E} {f\chi_{E}\lambda} = \int_{\mathbb{R}^{n}} {f\chi_{E}\lambda} = \int_{\mathbb{R}^{n}} {f\chi_{E \setminus F}\lambda} = \int_{E \setminus F} {f\chi_{E \setminus F}\lambda} = \int_{E} {f\chi_{E \setminus F}\lambda}
\end{align*}
定理\ref{4.6.1.21}よりその関数$f\chi_{E}$はそれらの集合たち$E$、$E \setminus F$上で定積分可能で次のようになる。
\begin{align*}
\int_{E} {f\lambda} = \int_{E} {f\chi_{E}\lambda} = \int_{\mathbb{R}^{n}} {f\chi_{E}\lambda} = \int_{\mathbb{R}^{n}} {f\chi_{E \setminus F}\lambda} = \int_{E \setminus F} {f\chi_{E \setminus F}\lambda} = \int_{E \setminus F} {f\chi_{E}\lambda} = \int_{E \setminus F} {f\lambda}
\end{align*}
\end{proof}
%\hypertarget{ux7a4dux5206ux306eux5f37ux5358ux8abfux6027}{%
\subsubsection{積分の強単調性}%\label{ux7a4dux5206ux306eux5f37ux5358ux8abfux6027}}
\begin{thm}[積分の強単調性]\label{4.6.6.6}
Lebesgue測度空間において、$\forall E \in \mathfrak{M}_{C}\left( \lambda^{*} \right)$に対し、その集合$E$上で連続な関数たち$f:\mathbb{R}^{n} \rightarrow \mathrm{cl}\mathbb{R}$、$g:\mathbb{R}^{n} \rightarrow{}^{*}\mathbb{R}$が$f|E \leq g|E$を満たし、$\exists\mathbf{x} \in E$に対し、$f\left( \mathbf{x} \right) < g\left( \mathbf{x} \right)$が成り立つなら、次式が成り立つ。
\begin{align*}
\int_{E} {f\lambda} < \int_{E} {g\lambda}
\end{align*}
この定理を積分の強単調性という。
\end{thm}
\begin{proof}
Lebesgue測度空間において、$\forall E \in \mathfrak{M}_{C}\left( \lambda^{*} \right)$に対し、その集合$E$上で連続な関数たち$f:\mathbb{R}^{n} \rightarrow{}^{*}\mathbb{R}$、$g:\mathbb{R}^{n} \rightarrow{}^{*}\mathbb{R}$が$f|E \leq g|E$を満たし、$\exists\mathbf{x} \in E$に対し、$f\left( \mathbf{x} \right) < g\left( \mathbf{x} \right)$が成り立つなら、$0 \leq g - f$かつ$0 < (g - f)\left( \mathbf{x} \right)$が成り立つので、$a = \frac{1}{2}(g - f)\left( \mathbf{x} \right)$とおかれれば、その関数$g - f$は連続なので、$\exists\delta \in \mathbb{R}^{+}\forall\mathbf{y} \in E$に対し、$\left\| \mathbf{x} - \mathbf{y} \right\| < \delta$が成り立つ、即ち、$\mathbf{y} \in U\left( \mathbf{x},\delta \right) \cap E$が成り立つなら、$\left| (g - f)\left( \mathbf{x} \right) - (g - f)\left( \mathbf{y} \right) \right| < a$が成り立つ。これにより、次のようになる。
\begin{align*}
\left| (g - f)\left( \mathbf{x} \right) - (g - f)\left( \mathbf{y} \right) \right| < a &\Leftrightarrow \left| 2a - (g - f)\left( \mathbf{y} \right) \right| < a\\
&\Leftrightarrow - a < 2a - (g - f)\left( \mathbf{y} \right) < a\\
&\Leftrightarrow - a < (g - f)\left( \mathbf{y} \right) - 2a < a\\
&\Leftrightarrow a < (g - f)\left( \mathbf{y} \right) < 3a
\end{align*}
ここで、定理\ref{4.5.4.11}より次式が成り立つことから、
\begin{align*}
U\left( \mathbf{x},\delta \right) \in \mathfrak{O}_{d_{E^{n}}} \subseteq \varSigma\left( \mathfrak{O}_{d_{E^{n}}} \right) = \mathfrak{B}_{\mathfrak{T}_{n}} \subseteq \mathfrak{M}_{C}\left( \lambda^{*} \right)
\end{align*}
次のようになり、
\begin{align*}
\int_{U\left( \mathbf{x},\delta \right) \cap E} {a\lambda} = a\int_{U\left( \mathbf{x},\delta \right) \cap E} \lambda = a\lambda\left( U\left( \mathbf{x},\delta \right) \cap E \right) \leq \int_{U\left( \mathbf{x},\delta \right) \cap E} {(g - f)\lambda}
\end{align*}
$0 < \lambda\left( U\left( \mathbf{x},\delta \right) \cap E \right)$が成り立つことに注意すれば、次式が成り立つので、
\begin{align*}
0 < \int_{U\left( \mathbf{x},\delta \right) \cap E} {(g - f)\lambda}
\end{align*}
次のようになる。
\begin{align*}
\int_{E} {f\lambda} &= \int_{E} {f\lambda} + \int_{E} {g\lambda} - \int_{E} {g\lambda}\\
&= - \int_{E} {(g - f)\lambda} + \int_{E} {g\lambda}\\
&= - \int_{\left( E \setminus U\left( \mathbf{x},\delta \right) \right) \sqcup \left( E \cap U\left( \mathbf{x},\delta \right) \right)} {(g - f)\lambda} + \int_{E} {g\lambda}\\
&= - \left( \int_{E \setminus U\left( \mathbf{x},\delta \right)} {(g - f)\lambda} + \int_{E \cap U\left( \mathbf{x},\delta \right)} {(g - f)\lambda} \right) + \int_{E} {g\lambda}\\
&= - \int_{E \setminus U\left( \mathbf{x},\delta \right)} {(g - f)\lambda} - \int_{E \cap U\left( \mathbf{x},\delta \right)} {(g - f)\lambda} + \int_{E} {g\lambda}\\
&< - \int_{E \setminus U\left( \mathbf{x},\delta \right)} {(g - f)\lambda} + \int_{E} {g\lambda}\\
&\leq \int_{E} {g\lambda}
\end{align*}
\end{proof}
%\hypertarget{ux7d2fux6b21ux7a4dux5206}{%
\subsubsection{累次積分}%\label{ux7d2fux6b21ux7a4dux5206}}
\begin{thm}[累次積分]\label{4.6.6.7}
$n$次元Lebesgue測度空間$\left( \mathbb{R}^{n},\ \ \mathfrak{M}_{C}\left( \gamma_{l^{n}} \right),\ \ \lambda^{n} \right)$、$n + 1$次元Lebesgue測度空間$\left( \mathbb{R}^{n + 1},\ \ \mathfrak{M}_{C}\left( \gamma_{l^{n + 1}} \right),\ \ \lambda^{n + 1} \right)$において、$\forall E \in \mathfrak{M}_{C}\left( \gamma_{l^{n}} \right)$に対し、可測な関数$f:E \rightarrow \mathrm{cl}\mathbb{R}^{+}$が与えられたとき、次式のような集合$I_{E,f}$は
\begin{align*}
I_{E,f} = \left\{ \begin{pmatrix}
\mathbf{x} \\
y \\
\end{pmatrix} \in \mathbb{R}^{n + 1} \middle| \mathbf{x} \in E \land 0 \leq y < f\left( \mathbf{x} \right) \right\}
\end{align*}
$I_{E,f} \in \mathfrak{M}_{C}\left( \gamma_{l^{n + 1}} \right)$を満たし、さらに、次式が成り立つ。
\begin{align*}
\lambda^{n + 1}\left( I_{E,f} \right) = \int_{E} {f\lambda^{n}}
\end{align*}
この定理を累次積分という。
\end{thm}
\begin{proof}
$n$次元Lebesgue測度空間$\left( \mathbb{R}^{n},\ \ \mathfrak{M}_{C}\left( \gamma_{l^{n}} \right),\ \ \lambda^{n} \right)$、$n + 1$次元Lebesgue測度空間$\left( \mathbb{R}^{n + 1},\ \ \mathfrak{M}_{C}\left( \gamma_{l^{n + 1}} \right),\ \ \lambda^{n + 1} \right)$において、$\forall E \in \mathfrak{M}_{C}\left( \gamma_{l^{n}} \right)$に対し、可測な関数$f:E \rightarrow \mathrm{cl}\mathbb{R}^{+}$が与えられたとき、定理\ref{4.5.5.18}における非負可測関数の非負単関数の列による近似での列$\left( (f)_{m} \right)_{m \in \mathbb{N}}$を用いて、$(f)_{m} = \sum_{i \in \varLambda_{k}} {\alpha_{i}\chi_{E_{i}}}$とおかれれば、次のようになるので、
\begin{align*}
I_{E,(f)_{m}} &= \left\{ \begin{pmatrix}
\mathbf{x} \\
y \\
\end{pmatrix} \in \mathbb{R}^{n + 1} \middle| \mathbf{x} \in E \land 0 \leq y < (f)_{m}\left( \mathbf{x} \right) \right\}\\
&= \left\{ \begin{pmatrix}
\mathbf{x} \\
y \\
\end{pmatrix} \in \mathbb{R}^{n + 1} \middle| \exists i \in \varLambda_{k}\left[ \mathbf{x} \in E_{i} \land 0 \leq y < (f)_{m}\left( \mathbf{x} \right) = \sum_{i \in \varLambda_{k}} {\alpha_{i}\chi_{E_{i}}}\left( \mathbf{x} \right) \right] \right\}\\
&= \left\{ \begin{pmatrix}
\mathbf{x} \\
y \\
\end{pmatrix} \in \mathbb{R}^{n + 1} \middle| \exists i \in \varLambda_{k}\left[ \mathbf{x} \in E_{i} \land 0 \leq y < \alpha_{i} \right] \right\}\\
&= \bigsqcup_{i \in \varLambda_{k}} \left\{ \begin{pmatrix}
\mathbf{x} \\
y \\
\end{pmatrix} \in \mathbb{R}^{n + 1} \middle| \mathbf{x} \in E_{i} \land 0 \leq y < \alpha_{i} \right\}\\
&= \bigsqcup_{i \in \varLambda_{k}} {E_{i} \times \left[ 0,\alpha_{i} \right)}
\end{align*}
その集合$I_{E,(f)_{m}}$は$I_{E,(f)_{m}} \in \mathfrak{M}_{C}\left( \gamma_{l^{n + 1}} \right)$を満たす。このとき、その列$\left( (f)_{m} \right)_{m \in \mathbb{N}}$は単調増加するので、その列$\left( I_{E,(f)_{m}} \right)_{m \in \mathbb{N}}$も単調増加し次式が成り立つ。
\begin{align*}
\bigcup_{m \in \mathbb{N}} I_{E,(f)_{m}} = \lim_{m \rightarrow \infty}I_{E,(f)_{m}} = I_{E,f}
\end{align*}
これにより、その集合$I_{E,f}$は$I_{E,f} \in \mathfrak{M}_{C}\left( \gamma_{l^{n + 1}} \right)$を満たす。\par
したがって、単調収束定理より次のようになる。
\begin{align*}
\lambda^{n + 1}\left( I_{E,f} \right) &= \lambda^{n + 1}\left( \lim_{m \rightarrow \infty}I_{E,(f)_{m}} \right)\\
&= \lim_{m \rightarrow \infty}{\lambda^{n + 1}\left( I_{E,(f)_{m}} \right)}\\
&= \lim_{m \rightarrow \infty}{\lambda^{n + 1}\left( \bigsqcup_{i \in \varLambda_{k}} {E_{i} \times \left[ 0,\alpha_{i} \right)} \right)}\\
&= \lim_{m \rightarrow \infty}{\sum_{i \in \varLambda_{k}} {\lambda^{n + 1}\left( E_{i} \times \left[ 0,\alpha_{i} \right) \right)}}\\
&= \lim_{m \rightarrow \infty}{\sum_{i \in \varLambda_{k}} {\lambda^{n}\left( E_{i} \right)\lambda\left( \left[ 0,\alpha_{i} \right) \right)}}\\
&= \lim_{m \rightarrow \infty}{\sum_{i \in \varLambda_{k}} {\alpha_{i}\lambda^{n}\left( E_{i} \right)}}\\
&= \lim_{m \rightarrow \infty}{\int_{E} {\sum_{i \in \varLambda_{k}} {\alpha_{i}\chi_{E_{i}}}\lambda^{n}}}\\
&= \lim_{m \rightarrow \infty}{\int_{E} {(f)_{m}\lambda^{n}}}\\
&= \int_{E} {\lim_{m \rightarrow \infty}(f)_{m}\lambda^{n}} = \int_{E} {f\lambda^{n}}
\end{align*}
\end{proof}
\begin{thebibliography}{50}
\bibitem{1}
  伊藤清三, ルベーグ積分入門, 裳華房, 1963. 第24刷 p82-86,111-114 ISBN4-7853-1304-8
\bibitem{2}
  日野正訓. "解析学 I(Lebesgue 積分論)". 京都大学. \url{https://www.math.kyoto-u.ac.jp/~hino/jugyoufile/AnalysisI210710.pdf} (2022年4月4日4:05 取得)
\bibitem{3}
  岩田耕一郎, ルベーグ積分, 森北出版, 2015. 第1版第2刷 p58-59 ISBN978-4-627-05431-8
\end{thebibliography}
\end{document}
