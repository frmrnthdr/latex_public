\documentclass[dvipdfmx]{jsarticle}
\setcounter{section}{1}
\setcounter{subsection}{5}
\usepackage{xr}
\externaldocument{4.1.1}
\externaldocument{4.1.4}
\usepackage{amsmath,amsfonts,amssymb,array,comment,mathtools,url,docmute}
\usepackage{longtable,booktabs,dcolumn,tabularx,mathtools,multirow,colortbl,xcolor}
\usepackage[dvipdfmx]{graphics}
\usepackage{bmpsize}
\usepackage{amsthm}
\usepackage{enumitem}
\setlistdepth{20}
\renewlist{itemize}{itemize}{20}
\setlist[itemize]{label=•}
\renewlist{enumerate}{enumerate}{20}
\setlist[enumerate]{label=\arabic*.}
\setcounter{MaxMatrixCols}{20}
\setcounter{tocdepth}{3}
\newcommand{\rotin}{\text{\rotatebox[origin=c]{90}{$\in $}}}
\newcommand{\amap}[6]{\text{\raisebox{-0.7cm}{\begin{tikzpicture} 
  \node (a) at (0, 1) {$\textstyle{#2}$};
  \node (b) at (#6, 1) {$\textstyle{#3}$};
  \node (c) at (0, 0) {$\textstyle{#4}$};
  \node (d) at (#6, 0) {$\textstyle{#5}$};
  \node (x) at (0, 0.5) {$\rotin $};
  \node (x) at (#6, 0.5) {$\rotin $};
  \draw[->] (a) to node[xshift=0pt, yshift=7pt] {$\textstyle{\scriptstyle{#1}}$} (b);
  \draw[|->] (c) to node[xshift=0pt, yshift=7pt] {$\textstyle{\scriptstyle{#1}}$} (d);
\end{tikzpicture}}}}
\newcommand{\twomaps}[9]{\text{\raisebox{-0.7cm}{\begin{tikzpicture} 
  \node (a) at (0, 1) {$\textstyle{#3}$};
  \node (b) at (#9, 1) {$\textstyle{#4}$};
  \node (c) at (#9+#9, 1) {$\textstyle{#5}$};
  \node (d) at (0, 0) {$\textstyle{#6}$};
  \node (e) at (#9, 0) {$\textstyle{#7}$};
  \node (f) at (#9+#9, 0) {$\textstyle{#8}$};
  \node (x) at (0, 0.5) {$\rotin $};
  \node (x) at (#9, 0.5) {$\rotin $};
  \node (x) at (#9+#9, 0.5) {$\rotin $};
  \draw[->] (a) to node[xshift=0pt, yshift=7pt] {$\textstyle{\scriptstyle{#1}}$} (b);
  \draw[|->] (d) to node[xshift=0pt, yshift=7pt] {$\textstyle{\scriptstyle{#2}}$} (e);
  \draw[->] (b) to node[xshift=0pt, yshift=7pt] {$\textstyle{\scriptstyle{#1}}$} (c);
  \draw[|->] (e) to node[xshift=0pt, yshift=7pt] {$\textstyle{\scriptstyle{#2}}$} (f);
\end{tikzpicture}}}}
\renewcommand{\thesection}{第\arabic{section}部}
\renewcommand{\thesubsection}{\arabic{section}.\arabic{subsection}}
\renewcommand{\thesubsubsection}{\arabic{section}.\arabic{subsection}.\arabic{subsubsection}}
\everymath{\displaystyle}
\allowdisplaybreaks[4]
\usepackage{vtable}
\theoremstyle{definition}
\newtheorem{thm}{定理}[subsection]
\newtheorem*{thm*}{定理}
\newtheorem{dfn}{定義}[subsection]
\newtheorem*{dfn*}{定義}
\newtheorem{axs}[dfn]{公理}
\newtheorem*{axs*}{公理}
\renewcommand{\headfont}{\bfseries}
\makeatletter
  \renewcommand{\section}{%
    \@startsection{section}{1}{\z@}%
    {\Cvs}{\Cvs}%
    {\normalfont\huge\headfont\raggedright}}
\makeatother
\makeatletter
  \renewcommand{\subsection}{%
    \@startsection{subsection}{2}{\z@}%
    {0.5\Cvs}{0.5\Cvs}%
    {\normalfont\LARGE\headfont\raggedright}}
\makeatother
\makeatletter
  \renewcommand{\subsubsection}{%
    \@startsection{subsubsection}{3}{\z@}%
    {0.4\Cvs}{0.4\Cvs}%
    {\normalfont\Large\headfont\raggedright}}
\makeatother
\makeatletter
\renewenvironment{proof}[1][\proofname]{\par
  \pushQED{\qed}%
  \normalfont \topsep6\p@\@plus6\p@\relax
  \trivlist
  \item\relax
  {
  #1\@addpunct{.}}\hspace\labelsep\ignorespaces
}{%
  \popQED\endtrivlist\@endpefalse
}
\makeatother
\renewcommand{\proofname}{\textbf{証明}}
\usepackage{tikz,graphics}
\usepackage[dvipdfmx]{hyperref}
\usepackage{pxjahyper}
\hypersetup{
 setpagesize=false,
 bookmarks=true,
 bookmarksdepth=tocdepth,
 bookmarksnumbered=true,
 colorlinks=false,
 pdftitle={},
 pdfsubject={},
 pdfauthor={},
 pdfkeywords={}}
\begin{document}
%\hypertarget{ux4e0aux6975ux9650ux3068ux4e0bux6975ux9650}{%
\subsection{上極限と下極限}%\label{ux4e0aux6975ux9650ux3068ux4e0bux6975ux9650}}
%\hypertarget{ux4e0aux6975ux9650ux3068ux4e0bux6975ux9650-1}{%
\subsubsection{上極限と下極限}%\label{ux4e0aux6975ux9650ux3068ux4e0bux6975ux9650-1}}
\begin{dfn}
実数列$\left( a_{n} \right)_{n \in \mathbb{N}}$に対し、$\left\{ a_{m} \right\}_{m = n}^{\infty} = \left\{ a_{m} \right\}_{m \in \mathbb{N} \setminus \varLambda_{n - 1}}$とおくと、次式のように極限たち$\limsup_{n \rightarrow \infty}a_{n}$、$\liminf_{n \rightarrow \infty}a_{n}$が定義される。
\begin{align*}
\limsup_{n \rightarrow \infty}a_{n} &= \lim_{n \rightarrow \infty}{\sup\left\{ a_{m} \right\}_{m = n}^{\infty}}\\
\liminf_{n \rightarrow \infty}a_{n} &= \lim_{n \rightarrow \infty}{\inf\left\{ a_{m} \right\}_{m = n}^{\infty}}
\end{align*}
このような極限たち$\limsup_{n \rightarrow \infty}a_{n}$、$\liminf_{n \rightarrow \infty}a_{n}$をそれぞれその実数列$\left( a_{n} \right)_{n \in \mathbb{N}}$の上極限、下極限という。
\end{dfn}
\begin{thm}\label{4.1.6.1}
任意の実数列$\left( a_{n} \right)_{n \in \mathbb{N}}$に対し、これの上極限$\limsup_{n \rightarrow \infty}a_{n}$、下極限$\liminf_{n \rightarrow \infty}a_{n}$が補完数直線${}^{*}\mathbb{R}$で必ず存在し、次式が成り立つ。
\begin{align*}
\limsup_{n \rightarrow \infty}a_{n} &= \inf\left\{ \sup\left\{ a_{m} \right\}_{m = n}^{\infty} \right\}_{n \in \mathbb{N}}\\
\liminf_{n \rightarrow \infty}a_{n} &= \sup\left\{ \inf\left\{ a_{m} \right\}_{m = n}^{\infty} \right\}_{n \in \mathbb{N}}
\end{align*}
\end{thm}
\begin{proof}
任意の実数列$\left( a_{n} \right)_{n \in \mathbb{N}}$に対し、$\left\{ a_{m} \right\}_{m = n}^{\infty} = \left\{ a_{m} \right\}_{m \in \mathbb{N} \setminus \varLambda_{n - 1}}$とおくと、定義より明らかに$\forall n \in \mathbb{N}$に対し、$\left\{ a_{n},a_{n + 1},a_{n + 2}\cdots \right\} \supseteq \left\{ a_{n + 1},a_{n + 2},a_{n + 3}\cdots \right\}$、即ち、$\left\{ a_{m} \right\}_{m = n}^{\infty} \supseteq \left\{ a_{m} \right\}_{m = n + 1}^{\infty}$が成り立つので、次式が成り立つ。
\begin{align*}
\sup\left\{ a_{m} \right\}_{m = n}^{\infty} \geq \sup\left\{ a_{m} \right\}_{m = n + 1}^{\infty}
\end{align*}
このとき、その元の列$\left( \sup\left\{ a_{m} \right\}_{m = n}^{\infty} \right)_{n \in \mathbb{N}}$は単調減少するので、定理\ref{4.1.4.18}より次式が成り立つ。
\begin{align*}
\limsup_{n \rightarrow \infty}a_{n} = \lim_{n \rightarrow \infty}{\sup\left\{ a_{m} \right\}_{m = n}^{\infty}} = \inf\left\{ \sup\left\{ a_{m} \right\}_{m = n}^{\infty} \right\}_{n \in \mathbb{N}} \in{}^{*}\mathbb{R}
\end{align*}
よって、その上極限$\limsup_{n \rightarrow \infty}a_{n}$が補完数直線${}^{*}\mathbb{R}$で必ず存在する。その下極限$\limsup_{n \rightarrow \infty}a_{n}$についても同様にして示される。
\end{proof}
\begin{thm}\label{4.1.6.2}
任意の実数列$\left( a_{n} \right)_{n \in \mathbb{N}}$に対し、次式が成り立つ。
\begin{align*}
\liminf_{n \rightarrow \infty}a_{n} \leq \limsup_{n \rightarrow \infty}a_{n}
\end{align*}
\end{thm}
\begin{proof}
任意の実数列$\left( a_{n} \right)_{n \in \mathbb{N}}$に対し、$\left\{ a_{m} \right\}_{m = n}^{\infty} = \left\{ a_{m} \right\}_{m \in \mathbb{N} \setminus \varLambda_{n - 1}}$とおくと、定義より明らかに次式が成り立つ。
\begin{align*}
\inf\left\{ a_{m} \right\}_{m = n}^{\infty} \leq \sup\left\{ a_{m} \right\}_{m = n}^{\infty}
\end{align*}
あとは$n \rightarrow \infty$とすれば、定理\ref{4.1.4.12}より次式が成り立つ。
\begin{align*}
\liminf_{n \rightarrow \infty}a_{n} = \lim_{n \rightarrow \infty}{\inf\left\{ a_{m} \right\}_{m = n}^{\infty}} \leq \lim_{n \rightarrow \infty}{\sup\left\{ a_{m} \right\}_{m = n}^{\infty}} = \limsup_{n \rightarrow \infty}a_{n}
\end{align*}
\end{proof}
\begin{thm}\label{4.1.6.3}
任意の実数列$\left( a_{n} \right)_{n \in \mathbb{N}}$に対し、ある部分列たち$\left( a_{m_{k}} \right)_{k \in \mathbb{N}}$、$\left( a_{n_{k}} \right)_{k \in \mathbb{N}}$が存在して、次式が成り立つ。
\begin{align*}
\lim_{k \rightarrow \infty}a_{m_{k}} = \liminf_{n \rightarrow \infty}a_{n},\ \ \lim_{k \rightarrow \infty}a_{n_{k}} = \limsup_{n \rightarrow \infty}a_{n}
\end{align*}
\end{thm}
\begin{proof}
任意の実数列$\left( a_{n} \right)_{n \in \mathbb{N}}$に対し、定理\ref{4.1.6.1}よりこの上極限$\limsup_{n \rightarrow \infty}a_{n}$が補完数直線${}^{*}\mathbb{R}$で必ず存在するので、これが$a$とおかれよう。$a \in \mathbb{R}$のとき、$n_{1} = 1$として、$\forall k \in \mathbb{N}$に対し、自然数$n_{k}$が与えられたとする。このとき、$\exists N \in \mathbb{N}\forall n \in \mathbb{N}$に対し、$N \leq n$が成り立つなら、$\left\{ a_{m} \right\}_{m = n}^{\infty} = \left\{ a_{m} \right\}_{m \in \mathbb{N} \setminus \varLambda_{n - 1}}$とおくと、次式が成り立ち、
\begin{align*}
\left| \sup\left\{ a_{m} \right\}_{m = n}^{\infty} - a \right| < \frac{1}{k + 1}
\end{align*}
特に、$N' = \max\left\{ N,n_{k} \right\}$とおけば、$N \leq N'$より次式が成り立つ。
\begin{align*}
\left| \sup\left\{ a_{m} \right\}_{m = N'}^{\infty} - a \right| < \frac{1}{k + 1}
\end{align*}
そこで、上限が上界のうち最小なので、実数$\sup\left\{ a_{m} \right\}_{m = N'}^{\infty} - \frac{1}{k + 1}$がその集合$\left\{ a_{m} \right\}_{m = N'}^{\infty}$の上界でなく、$\exists K \in \mathbb{N} \setminus \varLambda_{N'}$に対し、$\sup\left\{ a_{m} \right\}_{m = N'}^{\infty} - \frac{1}{k + 1} \leq a_{K}$が成り立つ。この自然数$K$を$n_{k + 1}$とおくと、その集合$\mathbb{N}$の元の列$\left( n_{k} \right)_{k \in \mathbb{N}}$が得られる。このとき、$\forall k \in \mathbb{N}$に対し、$n_{k} \leq N'$より$n_{k + 1} \in \mathbb{N} \setminus \varLambda_{N'} \subseteq \mathbb{N} \setminus \varLambda_{n_{k}}$が成り立つので、$n_{k} < n_{k + 1}$となりその元の列$\left( n_{k} \right)_{k \in \mathbb{N}}$は狭義単調増加する。これにより、その実数列$\left( a_{n} \right)_{n \in \mathbb{N}}$の部分列$\left( a_{n_{k}} \right)_{k \in \mathbb{N}}$が得られた。このとき、$\forall\varepsilon \in \mathbb{R}^{+}$に対し、定理\ref{4.1.1.22}、即ち、Archimedesの性質よりある自然数$K$が存在して、$\frac{1}{K + 1} < \varepsilon$が成り立ち、$\forall k \in \mathbb{N}$に対し、$K + 1 \leq k$が成り立つとき、定理\ref{4.1.6.1}より$\exists N \in \mathbb{N}\forall l \in \mathbb{N}$に対し、$N \leq l$が成り立つなら、次式が成り立つ。
\begin{align*}
\left| \sup\left\{ a_{m} \right\}_{m = l}^{\infty} - a \right| < \frac{1}{k}
\end{align*}
特に、$N' = \max\left\{ N,n_{k - 1} \right\}$とおけば、$N \leq N'$より次式が成り立つ。
\begin{align*}
\left| \sup\left\{ a_{m} \right\}_{m = N'}^{\infty} - a \right| < \frac{1}{k},\ \ \sup\left\{ a_{m} \right\}_{m = N'}^{\infty} - \frac{1}{k} \leq a_{n_{k}}
\end{align*}
$n_{k} < N'$より$\left\{ a_{m} \right\}_{m = n_{k}}^{\infty} \subset \left\{ a_{m} \right\}_{m = N'}^{\infty}$が成り立つので、$a_{n_{k}} \leq \sup\left\{ a_{m} \right\}_{m = N'}^{\infty}$が得られる。したがって、次のようになる。
\begin{align*}
\left| a_{n_{k}} - a \right| &= \left| a_{n_{k}} - \sup\left\{ a_{m} \right\}_{m = N'}^{\infty} + \sup\left\{ a_{m} \right\}_{m = N'}^{\infty} - a \right|\\
&\leq \left| a_{n_{k}} - \sup\left\{ a_{m} \right\}_{m = N'}^{\infty} \right| + \left| \sup\left\{ a_{m} \right\}_{m = N'}^{\infty} - a \right|\\
&= \sup\left\{ a_{m} \right\}_{m = N'}^{\infty} - a_{n_{k}} + \left| \sup\left\{ a_{m} \right\}_{m = N'}^{\infty} - a \right|\\
&\leq \frac{1}{k} + \frac{1}{k} \leq \frac{2}{k} < \frac{2}{K + 1} < 2\varepsilon
\end{align*}
よって、次式が成り立つ。
\begin{align*}
\lim_{k \rightarrow \infty}a_{n_{k}} = \limsup_{n \rightarrow \infty}a_{n}
\end{align*}
$a = \infty$のとき、$n_{1} = 1$として、$\forall k \in \mathbb{N}$に対し、自然数$n_{k}$が与えられたとする。このとき、$\exists N \in \mathbb{N}\forall n \in \mathbb{N}$に対し、$N \leq n$が成り立つなら、次式が成り立ち、
\begin{align*}
k + 1 < \sup\left\{ a_{m} \right\}_{m = n}^{\infty}
\end{align*}
特に、$N' = \max\left\{ N,n_{k} \right\}$とおいて$\forall n \in \mathbb{N}$に対し、$N' \leq n$が成り立つなら、次式が成り立つ。
\begin{align*}
k + 1 < \sup\left\{ a_{m} \right\}_{m = N'}^{\infty}
\end{align*}
そこで、$\forall m \in \mathbb{N} \setminus \varLambda_{N' - 1}$に対し、$a_{m} \leq k + 1$が成り立つとすれば、その自然数$k + 1$はその集合$\left\{ a_{m} \right\}_{m = N'}^{\infty}$の上界なので、$\sup\left\{ a_{m} \right\}_{m = N'}^{\infty} \leq k + 1$が得られるが、これは矛盾している。ゆえに、$\exists K \in \mathbb{N} \setminus \varLambda_{N' - 1}$に対し、$k + 1 < a_{K}$が成り立つ。この自然数$K$を$n_{k + 1}$とおくと、その集合$\mathbb{N}$の元の列$\left( n_{k} \right)_{k \in \mathbb{N}}$が得られる。このとき、$\forall k \in \mathbb{N}$に対し、$n_{k} \leq N'$より$n_{k + 1} \in \mathbb{N} \setminus \varLambda_{N'} \subseteq \mathbb{N} \setminus \varLambda_{n_{k}}$が成り立つので、$n_{k} < n_{k + 1}$となりその元の列$\left( n_{k} \right)_{k \in \mathbb{N}}$は狭義単調増加する。これにより、その実数列$\left( a_{n} \right)_{n \in \mathbb{N}}$の部分列$\left( a_{n_{k}} \right)_{k \in \mathbb{N}}$が得られた。このとき、$\forall\varepsilon \in \mathbb{R}^{+}$に対し、定理\ref{4.1.1.22}、即ち、Archimedesの性質よりある自然数$K$が存在して、$\varepsilon < K + 1$が成り立ち、$\forall k \in \mathbb{N}$に対し、$K + 1 \leq k$が成り立つとき、定理\ref{4.1.6.1}より$\exists N \in \mathbb{N}\forall l \in \mathbb{N}$に対し、$N \leq l$が成り立つなら、次式が成り立ち、
\begin{align*}
k < \sup\left\{ a_{m} \right\}_{m = l}^{\infty}
\end{align*}
特に、$N' = \max\left\{ N,n_{k - 1} \right\}$とおけば、$k < a_{n_{k}}$が成り立つ。したがって、次のようになる。
\begin{align*}
\varepsilon < K + 1 \leq k < a_{n_{k}}
\end{align*}
よって、次式が成り立つ。
\begin{align*}
\lim_{k \rightarrow \infty}a_{n_{k}} = \limsup_{n \rightarrow \infty}a_{n}
\end{align*}
$a = - \infty$のときも同様にして示される。\par
下極限についても同様にして示される。
\end{proof}
\begin{thm}\label{4.1.6.4}
任意の実数列$\left( a_{n} \right)_{n \in \mathbb{N}}$と任意の拡大実数$a$について、次のことは同値である。
\begin{itemize}
\item
  $a = \limsup_{n \rightarrow \infty}a_{n}$が成り立つ。
\item
  $\forall b \in{}^{*}\mathbb{R}$に対し、$a < b$のとき、$\exists N \in \mathbb{N}\forall n \in \mathbb{N}$に対し、$N \leq n$が成り立つなら、$a_{n} < b$が成り立ち、$b < a$のとき、$b < a_{n}$が成り立つような自然数$n$が無限に存在する。
\item
  その拡大実数$a$に収束するその実数列$\left( a_{n} \right)_{n \in \mathbb{N}}$の部分列が存在するかつ、$a < b$なる拡大実数$b$に収束するその実数列$\left( a_{n} \right)_{n \in \mathbb{N}}$の部分列は存在しない。
\item
  その拡大実数$a$はその実数列$\left( a_{n} \right)_{n \in \mathbb{N}}$のその補完数直線${}^{*}\mathbb{R}$における集積値のうち最大のものである。
\end{itemize}
さらに、次のことも同値である。
\begin{itemize}
\item
  $a = \liminf_{n \rightarrow \infty}a_{n}$が成り立つ。
\item
  $\forall b \in{}^{*}\mathbb{R}$に対し、$b < a$のとき、$\exists N \in \mathbb{N}\forall n \in \mathbb{N}$に対し、$N \leq n$が成り立つなら、$b < a_{n}$が成り立ち、$a < b$のとき、$a_{n} < b$が成り立つような自然数$n$が無限に存在する。
\item
  その拡大実数$a$に収束するその実数列$\left( a_{n} \right)_{n \in \mathbb{N}}$の部分列が存在するかつ、$a < b$なる拡大実数$b$に収束するその実数列$\left( a_{n} \right)_{n \in \mathbb{N}}$の部分列は存在しない。
\item
  その拡大実数$a$はその実数列$\left( a_{n} \right)_{n \in \mathbb{N}}$の補完数直線${}^{*}\mathbb{R}$における集積値のうち最小のものである。
\end{itemize}
\end{thm}
\begin{proof}
任意の実数列$\left( a_{n} \right)_{n \in \mathbb{N}}$と任意の拡大実数$a$について、$a = \limsup_{n \rightarrow \infty}a_{n}$が成り立つとき、$\left\{ a_{m} \right\}_{m = n}^{\infty} = \left\{ a_{m} \right\}_{m \in \mathbb{N} \setminus \varLambda_{n - 1}}$とおくと、$\forall b \in{}^{*}\mathbb{R}$に対し、$a < b$のとき、$a = \inf\left\{ \sup\left\{ a_{m} \right\}_{m = n}^{\infty} \right\}_{n \in \mathbb{N}}$が成り立つので、$\exists N \in \mathbb{N}$に対し、$\sup\left\{ a_{m} \right\}_{m = N}^{\infty} < b$が成り立ち、その元の列$\left( \sup\left\{ a_{m} \right\}_{m = n}^{\infty} \right)_{n \in \mathbb{N}}$は単調減少するので、$\forall n \in \mathbb{N}$に対し、$N \leq n$が成り立つなら、$\sup\left\{ a_{m} \right\}_{m = n}^{\infty} \leq \sup\left\{ a_{m} \right\}_{m = N}^{\infty} < b$が成り立ち、したがって、$a_{n} \leq \sup\left\{ a_{m} \right\}_{m = n}^{\infty} < b$が成り立つ。$b < a$のとき、$\forall n \in \mathbb{N}$に対し、次式が成り立つので、
\begin{align*}
b < a = \inf\left\{ \sup\left\{ a_{m} \right\}_{m = n}^{\infty} \right\}_{n \in \mathbb{N}} \leq \sup\left\{ a_{m} \right\}_{m = n}^{\infty}
\end{align*}
$b < a_{m_{n}} \leq \sup\left\{ a_{m} \right\}_{m = n}^{\infty}$かつ$n < m_{n}$なる自然数$m_{n}$が存在する。このことはすべての自然数$n$に対し、成り立つので、$b < a_{n}$が成り立つような自然数$n$が無限に存在する。\par
逆に、$\forall b \in{}^{*}\mathbb{R}$に対し、$a < b$のとき、$\exists N \in \mathbb{N}\forall n \in \mathbb{N}$に対し、$N \leq n$が成り立つなら、$a_{n} < b$が成り立ち、$b < a$のとき、$b < a_{n}$が成り立つような自然数$n$が無限に存在するとする。$\forall b \in{}^{*}\mathbb{R}$に対し、$a < b$のとき、$\exists N \in \mathbb{N}\forall n \in \mathbb{N}$に対し、$N \leq n$が成り立つなら、$a_{n} < b$が成り立つので、$a_{n} \leq \sup\left\{ a_{m} \right\}_{m = n}^{\infty} \leq b$が成り立ちその元の列$\left( \sup\left\{ a_{m} \right\}_{m = n}^{\infty} \right)_{n \in \mathbb{N}}$は単調減少するので、次式のようになる。
\begin{align*}
\limsup_{n \rightarrow \infty}a_{n} = \inf\left\{ \sup\left\{ a_{m} \right\}_{m = n}^{\infty} \right\}_{n \in \mathbb{N}} \leq \sup\left\{ a_{m} \right\}_{m = n}^{\infty} \leq \sup\left\{ a_{m} \right\}_{m = N}^{\infty} \leq b
\end{align*}
ここで、$a < b$が成り立つので、$a < \limsup_{n \rightarrow \infty}a_{n}$が成り立つようであれば、稠密性より$\exists b \in{}^{*}\mathbb{R}$に対し、$a < b < \limsup_{n \rightarrow \infty}a_{n}$が成り立つことになるが、これは上記の$\limsup_{n \rightarrow \infty}a_{n} < b$が成り立つことに矛盾する。したがって、$\limsup_{n \rightarrow \infty}a_{n} \leq a$が成り立つ。次に、$\forall b \in{}^{*}\mathbb{R}$に対し、$b < a$が成り立つなら、$b < a_{n}$が成り立つような自然数$n$は無限に存在することになる。$b > \limsup_{n \rightarrow \infty}a_{n}$が成り立つと仮定すると、$\exists N \in \mathbb{N}$に対し、$\sup\left\{ a_{m} \right\}_{m = N}^{\infty} < b$が成り立ち、その元の列$\left( \sup\left\{ a_{m} \right\}_{m = n}^{\infty} \right)_{n \in \mathbb{N}}$は単調減少するので、$\forall n \in \mathbb{N}$に対し、$N \leq n$が成り立つなら、次式が成り立ち、
\begin{align*}
\sup\left\{ a_{m} \right\}_{m = N}^{\infty} \leq \sup\left\{ a_{m} \right\}_{m = n}^{\infty} \leq \limsup_{n \rightarrow \infty}a_{n} < b
\end{align*}
したがって、$a_{n} \leq \sup\left\{ a_{m} \right\}_{m = n}^{\infty} < \limsup_{n \rightarrow \infty}a_{n} < b$が成り立つが、これは$b < a_{n}$が成り立つような自然数$n$が無限に存在することに矛盾する。したがって、$b \leq \limsup_{n \rightarrow \infty}a_{n}$が成り立つ。ここで、$a > \limsup_{n \rightarrow \infty}a_{n}$が成り立つと仮定すると、稠密性より$\exists b \in{}^{*}\mathbb{R}$に対し、$a > b > \limsup_{n \rightarrow \infty}a_{n}$が成り立つことになるが、これは$b \leq \limsup_{n \rightarrow \infty}a_{n}$が成り立つことに矛盾する。したがって、$a \leq \limsup_{n \rightarrow \infty}a_{n}$が成り立つ。以上の議論により、$\limsup_{n \rightarrow \infty}a_{n} \leq a$かつ$a \leq \limsup_{n \rightarrow \infty}a_{n}$が成り立つので、$a = \limsup_{n \rightarrow \infty}a_{n}$が得られる。\par
$\forall b \in{}^{*}\mathbb{R}$に対し、$a < b$のとき、$\exists N \in \mathbb{N}\forall n \in \mathbb{N}$に対し、$N \leq n$が成り立つなら、$a_{n} < b$が成り立ち、$b < a$のとき、$b < a_{n}$が成り立つような自然数$n$が無限に存在するとする。$a = \infty$のとき、$1 < a_{n}$なる自然数$n$を$n_{1}$とおくことにし、$\forall k \in \mathbb{N}$に対し、次のような自然数$n$のうち1つを$n_{k + 1}$とおくことにすると、
\begin{align*}
\max\left\{ a_{m} \right\}_{m \in \varLambda_{n_{k}}} < a_{n}
\end{align*}
このようにして得られるその集合$\mathbb{N}$の元の列$\left( n_{k} \right)_{k \in \mathbb{N}}$は狭義単調増加している。実際、$\exists k \in \mathbb{N}$に対し、$n_{k} \geq n_{k + 1}$が成り立つと仮定すると、$\exists m \in \varLambda_{n_{k}}$に対し、$n_{k + 1} = m$が成り立つので、次式が得られるが、
\begin{align*}
a_{n_{k + 1}} = a_{m} \leq \max\left\{ a_{m} \right\}_{m \in \varLambda_{n_{k}}} < a_{n_{k + 1}}
\end{align*}
これは矛盾している。これにより、その実数列$\left( a_{n_{k}} \right)_{k \in \mathbb{N}}$はその実数列$\left( a_{n} \right)_{n \in \mathbb{N}}$の部分列となっている。さらに、その部分列$\left( a_{n_{k}} \right)_{k \in \mathbb{N}}$は単調増加している。実際、$\exists k \in \mathbb{N}$に対し、$a_{n_{k}} > a_{n_{k + 1}}$が成り立つと仮定すると、その集合$\mathbb{N}$の元の列$\left( n_{k} \right)_{k \in \mathbb{N}}$のおき方より次のようになるが、
\begin{align*}
a_{n_{k + 1}} < a_{n_{k}} \leq \max\left\{ a_{m} \right\}_{m \in \varLambda_{n_{k}}} < a_{n_{k + 1}}
\end{align*}
これは矛盾している。ここで、仮定より$\forall M \in \mathbb{R}^{+}\exists N \in \mathbb{N}$に対し、$M < a_{N}$が成り立ち、定理\ref{4.1.4.10}より$N + 1 \leq n_{N + 1}$が成り立ちその集合$\mathbb{N}$の元の列$\left( n_{k} \right)_{k \in \mathbb{N}}$のおき方より次のようになるので、
\begin{align*}
M < a_{N} \leq \max\left\{ a_{m} \right\}_{m \in \varLambda_{n_{N}}} < a_{n_{N + 1}}
\end{align*}
$\forall k \in \mathbb{N}$に対し、$N + 1 \leq k$が成り立つなら、その部分列$\left( a_{n_{k}} \right)_{k \in \mathbb{N}}$は単調増加しているので、次のようになる。
\begin{align*}
M < a_{N} \leq \max\left\{ a_{m} \right\}_{m \in \varLambda_{n_{N}}} < a_{n_{N + 1}} \leq a_{n_{k}}
\end{align*}
これは$\varepsilon$-$N$論法そのものだから、$\lim_{k \rightarrow \infty}a_{n_{k}} = \infty$が成り立ち、正の無限大に収束するその元の列$\left( a_{n} \right)_{n \in \mathbb{N}}$の部分列が存在する。また、明らかに$\infty < b$なる拡大実数$b$に収束するその元の列$\left( a_{n} \right)_{n \in \mathbb{N}}$の部分列は存在しない。$a = - \infty$のとき、$\forall b \in{}^{*}\mathbb{R}$に対し、$- \infty < b$が成り立つなら、仮定より$\exists N \in \mathbb{N}\forall n \in \mathbb{N}$に対し、$N \leq n$が成り立つなら、$a_{n} < b$が成り立つことになるので、特に、$\forall M \in \mathbb{R}^{+}\exists N \in \mathbb{N}\forall n \in \mathbb{N}$に対し、$N < n$が成り立つなら、$a_{n} < - M$が成り立ち、これは$\varepsilon$-$N$論法そのものだから、$\lim_{n \rightarrow \infty}a_{n} = - \infty$が成り立つ。これにより、その拡大実数$a$に収束するその元の列$\left( a_{n} \right)_{n \in \mathbb{N}}$の部分列が存在する。また、定理\ref{4.1.4.2}、定理\ref{4.1.4.11}より$a < b$なる拡大実数$b$に収束するその元の列$\left( a_{n} \right)_{n \in \mathbb{N}}$の部分列は存在しない。$a \in \mathbb{R}$のとき、仮定より$\forall\varepsilon \in \mathbb{R}^{+}$に対し、$a - \varepsilon < a_{n}$が成り立つような自然数$n$は無限に存在するので、このような自然数たち全体の集合を$A$とおくと、その集合$A$は無限集合であるかつ、$A \subseteq \mathbb{N}$が成り立つので、定理\ref{4.1.1.21}に注意すれば、狭義単調増加する元の列$\left( n_{k} \right)_{k \in \mathbb{N}}$が存在して、$\left\{ n_{k} \right\}_{k \in \mathbb{N}} = A$が成り立つ。仮定より$\exists N \in \mathbb{N}\forall k \in \mathbb{N}$に対し、$N \leq k$が成り立つなら、$a_{k} < a + \varepsilon$が成り立つかつ、定理\ref{4.1.4.10}より$k \leq n_{k}$が成り立つので、もちろん、$N \leq k \leq n_{k}$が成り立ち、このとき、$a_{n_{k}} < a + \varepsilon$が成り立つ。さらに、$n_{k} \in A$なので、$a - \varepsilon < a_{n_{k}}$が成り立つ。したがって、$\left| a_{n_{k}} - a \right| < \varepsilon$が得られる。これは$\varepsilon$-$N$論法そのものなので、$\lim_{k \rightarrow \infty}a_{n_{k}} = a$が成り立つ。また、$\forall b \in{}^{*}\mathbb{R}$に対し、$a < b$が成り立つとすると、$\exists N \in \mathbb{N}\forall n \in \mathbb{N}$に対し、$N \leq n$が成り立つなら、$a_{n} < b$が成り立つことから、稠密性より$a_{n} < c < b$なる元$c$をとると、$c < a_{n}$を満たすような自然数は多くとも$N$つしかないので、$a < b$なる拡大実数$b$に収束するその実数列$\left( a_{n} \right)_{n \in \mathbb{N}}$の部分列は存在しない。\par
逆に、その拡大実数$a$に収束するその実数列$\left( a_{n} \right)_{n \in \mathbb{N}}$の部分列が存在するかつ、$a < b$なる拡大実数$b$に収束するその実数列$\left( a_{n} \right)_{n \in \mathbb{N}}$の部分列は存在しないとする。$\forall b \in{}^{*}\mathbb{R}$に対し、$a < b$が成り立つなら、$b \leq a_{n}$なる自然数$n$が無限にあれば、このような自然数たち全体の集合を$A$とおくと、その集合$A$は無限集合であるかつ、$A \subseteq \mathbb{N}$が成り立つので、定理\ref{4.1.1.21}に注意すれば、狭義単調増加する元の列$\left( n_{k} \right)_{k \in \mathbb{N}}$が存在して、$\left\{ n_{k} \right\}_{k \in \mathbb{N}} = A$が成り立つ。この部分列$\left( a_{n_{k}} \right)_{k \in \mathbb{N}}$が定理\ref{4.1.5.9}よりさらに補完数直線${}^{*}\mathbb{R}$で広い意味で収束する部分列をもち、定理\ref{4.1.4.12}よりこれの極限$c$は$a < b \leq c$を満たすことになるが、これは$a < b$なる元$b$に収束するその元の列$\left( a_{n} \right)_{n \in \mathbb{N}}$の部分列は存在しないことに矛盾する。したがって、$b \leq a_{n}$なる自然数$n$が有限にしかなく、これの最大なものを$N$とおくと、$\exists N + 1 \in \mathbb{N}\forall n \in \mathbb{N}$に対し、$N + 1 \leq n$が成り立つなら、$a_{n} < b$が成り立つ。さらに、$\forall b \in{}^{*}\mathbb{R}$に対し、$b < a$が成り立つなら、その拡大実数$a$に収束するその実数列$\left( a_{n} \right)_{n \in \mathbb{N}}$の部分列が存在するので、この部分列のうち$b < a_{n}$なるもので考えれば、これもその拡大実数$a$に収束するので、$\varepsilon$-$N$論法より$b < a_{n}$なる自然数$n$が無限にある。\par
最後に、その拡大実数$a$に収束するその元の列$\left( a_{n} \right)_{n \in \mathbb{N}}$の部分列が存在するかつ、$a < b$なる拡大実数$b$に収束するその元の列$\left( a_{n} \right)_{n \in \mathbb{N}}$の部分列が存在しないならそのときに限り、集積値の定義より明らかにその拡大実数$a$はその元の列$\left( a_{n} \right)_{n \in \mathbb{N}}$のその集合${}^{*}\mathbb{R}$における集積値のうち最大のものである。\par
下極限についても同様にして示される。
\end{proof}
%\hypertarget{ux4e0aux6975ux9650ux3068ux4e0bux6975ux9650ux3068ux6975ux9650}{%
\subsubsection{上極限と下極限と極限}%\label{ux4e0aux6975ux9650ux3068ux4e0bux6975ux9650ux3068ux6975ux9650}}
\begin{thm}\label{4.1.6.5}
任意の実数列$\left( a_{n} \right)_{n \in \mathbb{N}}$について、これの極限$\lim_{n \rightarrow \infty}a_{n}$が補完数直線${}^{*}\mathbb{R}$に存在するならそのときに限り、次式が成り立つ。
\begin{align*}
\liminf_{n \rightarrow \infty}a_{n} = \limsup_{n \rightarrow \infty}a_{n}
\end{align*}
さらに、これが成り立つなら、次式も成り立つ。
\begin{align*}
\liminf_{n \rightarrow \infty}a_{n} = \lim_{n \rightarrow \infty}a_{n} = \limsup_{n \rightarrow \infty}a_{n}
\end{align*}
\end{thm}
\begin{proof}
任意の実数列$\left( a_{n} \right)_{n \in \mathbb{N}}$について、$\left\{ a_{m} \right\}_{m = n}^{\infty} = \left\{ a_{m} \right\}_{m \in \mathbb{N} \setminus \varLambda_{n - 1}}$とおくと、これの極限$\lim_{n \rightarrow \infty}a_{n}$が補完数直線${}^{*}\mathbb{R}$に存在するなら、任意の部分列もその極限$\lim_{n \rightarrow \infty}a_{n}$に収束し、$\forall a \in{}^{*}\mathbb{R}$に対し、$\lim_{n \rightarrow \infty}a_{n} < a$が成り立つなら、これに収束するような部分列が存在しないので、定理\ref{4.1.6.4}より$\limsup_{n \rightarrow \infty}a_{n} = \lim_{n \rightarrow \infty}a_{n}$が成り立つ。同様にして、$\liminf_{n \rightarrow \infty}a_{n} = \lim_{n \rightarrow \infty}a_{n}$が成り立つので、したがって、次のようになる。
\begin{align*}
\limsup_{n \rightarrow \infty}a_{n} = \lim_{n \rightarrow \infty}a_{n} = \liminf_{n \rightarrow \infty}a_{n}
\end{align*}\par
逆に、次式が成り立つなら、
\begin{align*}
\limsup_{n \rightarrow \infty}a_{n} = \liminf_{n \rightarrow \infty}a_{n}
\end{align*}
$\forall n \in \mathbb{N}$に対し、$\inf\left\{ a_{m} \right\}_{m = n}^{\infty} \leq a_{n} \leq \sup\left\{ a_{m} \right\}_{m = n}^{\infty}$が成り立つので、はさみうちの原理より極限が必ず補完数直線${}^{*}\mathbb{R}$に存在して次式が成り立つ。
\begin{align*}
\lim_{n \rightarrow \infty}{\inf A_{n}} = \lim_{n \rightarrow \infty}a_{n} = \lim_{n \rightarrow \infty}{\sup A_{n}}
\end{align*}
\end{proof}
%\hypertarget{ux4e0aux6975ux9650ux3068ux4e0bux6975ux9650ux306eux4e0dux7b49ux5f0f}{%
\subsubsection{上極限と下極限の不等式}%\label{ux4e0aux6975ux9650ux3068ux4e0bux6975ux9650ux306eux4e0dux7b49ux5f0f}}
\begin{thm}\label{4.1.6.6}
任意の実数列たち$\left( a_{n} \right)_{n \in \mathbb{N}}$、$\left( b_{n} \right)_{n \in \mathbb{N}}$に対し、次の不等式たちが成り立つ。
\begin{align*}
\liminf_{n \rightarrow \infty}a_{n} &+ \liminf_{n \rightarrow \infty}b_{n} \leq \liminf_{n \rightarrow \infty}\left( a_{n} + b_{n} \right) \leq \liminf_{n \rightarrow \infty}a_{n} + \limsup_{n \rightarrow \infty}b_{n}\\
&\leq \limsup_{n \rightarrow \infty}\left( a_{n} + b_{n} \right) \leq \limsup_{n \rightarrow \infty}a_{n} + \limsup_{n \rightarrow \infty}b_{n}
\end{align*}
特に、極限$\lim_{n \rightarrow \infty}b_{n}$が存在するとき、次式が成り立つ。
\begin{align*}
\limsup_{n \rightarrow \infty}\left( a_{n} + b_{n} \right) &= \limsup_{n \rightarrow \infty}a_{n} + \lim_{n \rightarrow \infty}b_{n}\\
\liminf_{n \rightarrow \infty}\left( a_{n} + b_{n} \right) &= \liminf_{n \rightarrow \infty}a_{n} + \lim_{n \rightarrow \infty}b_{n}
\end{align*}
\end{thm}
\begin{proof}
任意の実数列たち$\left( a_{n} \right)_{n \in \mathbb{N}}$、$\left( b_{n} \right)_{n \in \mathbb{N}}$が与えられたとする。$\forall n \in \mathbb{N}$に対し、次式が成り立つので、
\begin{align*}
\sup\left\{ a_{m} + b_{m} \right\}_{m = n}^{\infty} \leq \sup\left\{ a_{m} \right\}_{m = n}^{\infty} + \sup\left\{ b_{m} \right\}_{m = n}^{\infty}
\end{align*}
両辺に$n \rightarrow \infty$とすれば、次のようになる。
\begin{align*}
\lim_{n \rightarrow \infty}{\sup\left\{ a_{m} + b_{m} \right\}_{m = n}^{\infty}} \leq \lim_{n \rightarrow \infty}{\sup\left\{ a_{m} \right\}_{m = n}^{\infty}} + \lim_{n \rightarrow \infty}{\sup\left\{ b_{m} \right\}_{m = n}^{\infty}}
\end{align*}
したがって、次式が成り立つ。
\begin{align*}
\limsup_{n \rightarrow \infty}\left( a_{n} + b_{n} \right) \leq \limsup_{n \rightarrow \infty}a_{n} + \limsup_{n \rightarrow \infty}b_{n}
\end{align*}
同様にして、次式が成り立つことが示される。
\begin{align*}
\liminf_{n \rightarrow \infty}a_{n} + \liminf_{n \rightarrow \infty}b_{n} \leq \liminf_{n \rightarrow \infty}\left( a_{n} + b_{n} \right)
\end{align*}
また、$\forall n \in \mathbb{N}$に対し、次式が成り立つので、
\begin{align*}
\inf\left\{ a_{m} \right\}_{m = n}^{\infty} + \sup\left\{ b_{m} \right\}_{m = n}^{\infty} \leq \sup\left\{ a_{m} + b_{m} \right\}_{m = n}^{\infty}
\end{align*}
両辺に$n \rightarrow \infty$とすれば、次のようになる。
\begin{align*}
\lim_{n \rightarrow \infty}{\inf\left\{ a_{m} \right\}_{m = n}^{\infty}} + \lim_{n \rightarrow \infty}{\sup\left\{ b_{m} \right\}_{m = n}^{\infty}} \leq \lim_{n \rightarrow \infty}{\sup\left\{ a_{m} + b_{m} \right\}_{m = n}^{\infty}}
\end{align*}
したがって、次式が成り立つ。
\begin{align*}
\liminf_{n \rightarrow \infty}a_{n} + \limsup_{n \rightarrow \infty}b_{n} \leq \limsup_{n \rightarrow \infty}\left( a_{n} + b_{n} \right)
\end{align*}
同様にして、次式が成り立つことが示される。
\begin{align*}
\liminf_{n \rightarrow \infty}\left( a_{n} + b_{n} \right) \leq \liminf_{n \rightarrow \infty}a_{n} + \limsup_{n \rightarrow \infty}b_{n}
\end{align*}
以上の議論により次式が成り立つ。
\begin{align*}
\liminf_{n \rightarrow \infty}a_{n} &+ \liminf_{n \rightarrow \infty}b_{n} \leq \liminf_{n \rightarrow \infty}\left( a_{n} + b_{n} \right) \leq \liminf_{n \rightarrow \infty}a_{n} + \limsup_{n \rightarrow \infty}b_{n}\\
&\leq \limsup_{n \rightarrow \infty}\left( a_{n} + b_{n} \right) \leq \limsup_{n \rightarrow \infty}a_{n} + \limsup_{n \rightarrow \infty}b_{n}
\end{align*}\par
特に、極限$\lim_{n \rightarrow \infty}b_{n}$が存在するとき、$\liminf_{n \rightarrow \infty}b_{n} = \limsup_{n \rightarrow \infty}b_{n} = \lim_{n \rightarrow \infty}b_{n}$が成り立つので、次式は上記の議論により成り立ち、
\begin{align*}
\limsup_{n \rightarrow \infty}a_{n} + \liminf_{n \rightarrow \infty}b_{n} \leq \limsup_{n \rightarrow \infty}\left( a_{n} + b_{n} \right) \leq \limsup_{n \rightarrow \infty}a_{n} + \limsup_{n \rightarrow \infty}b_{n}
\end{align*}
さらに、次のように書き換えられることができる。
\begin{align*}
\limsup_{n \rightarrow \infty}a_{n} + \lim_{n \rightarrow \infty}b_{n} \leq \limsup_{n \rightarrow \infty}\left( a_{n} + b_{n} \right) \leq \limsup_{n \rightarrow \infty}a_{n} + \lim_{n \rightarrow \infty}b_{n}
\end{align*}
よって、次式が成り立つ。
\begin{align*}
\limsup_{n \rightarrow \infty}\left( a_{n} + b_{n} \right) = \limsup_{n \rightarrow \infty}a_{n} + \lim_{n \rightarrow \infty}b_{n}
\end{align*}
同様にして、次式が成り立つことも示される。
\begin{align*}
\liminf_{n \rightarrow \infty}\left( a_{n} + b_{n} \right) = \liminf_{n \rightarrow \infty}a_{n} + \lim_{n \rightarrow \infty}b_{n}
\end{align*}
\end{proof}
\begin{thm}\label{4.1.6.7}
任意の実数列たち$\left( a_{n} \right)_{n \in \mathbb{N}}$、$\left( b_{n} \right)_{n \in \mathbb{N}}$が与えられとき、$0 \leq \left( a_{n} \right)_{n \in \mathbb{N}}$かつ$0 \leq \left( b_{n} \right)_{n \in \mathbb{N}}$が成り立つなら、次の不等式が成り立つ。
\begin{align*}
\liminf_{n \rightarrow \infty}a_{n}\liminf_{n \rightarrow \infty}b_{n} \leq \liminf_{n \rightarrow \infty}\left( a_{n}b_{n} \right) \leq \limsup_{n \rightarrow \infty}\left( a_{n}b_{n} \right) \leq \limsup_{n \rightarrow \infty}a_{n}\limsup_{n \rightarrow \infty}b_{n}
\end{align*}
特に、$\forall c \in \mathbb{R}^{+}$に対し、次式が成り立ち、
\begin{align*}
\limsup_{n \rightarrow \infty}\left( ca_{n} \right) &= c\limsup_{n \rightarrow \infty}a_{n}\\
\liminf_{n \rightarrow \infty}\left( ca_{n} \right) &= c\liminf_{n \rightarrow \infty}a_{n}
\end{align*}
$\forall c \in \mathbb{R}$に対し、$c < 0$のとき、次式が成り立つ。
\begin{align*}
\limsup_{n \rightarrow \infty}\left( ca_{n} \right) &= c\liminf_{n \rightarrow \infty}a_{n}\\
\liminf_{n \rightarrow \infty}\left( ca_{n} \right) &= c\limsup_{n \rightarrow \infty}a_{n}
\end{align*}
\end{thm}
\begin{proof}
任意の実数列たち$\left( a_{n} \right)_{n \in \mathbb{N}}$、$\left( b_{n} \right)_{n \in \mathbb{N}}$が与えられとき、$0 \leq \left( a_{n} \right)_{n \in \mathbb{N}}$かつ$0 \leq \left( b_{n} \right)_{n \in \mathbb{N}}$が成り立つなら、$\forall n \in \mathbb{N}$に対し、次式が成り立つので、
\begin{align*}
\sup\left\{ a_{m}b_{m} \right\}_{m = n}^{\infty} \leq \sup\left\{ a_{m} \right\}_{m = n}^{\infty}\sup\left\{ b_{m} \right\}_{m = n}^{\infty}
\end{align*}
両辺に$n \rightarrow \infty$とすれば、次のようになる。
\begin{align*}
\lim_{n \rightarrow \infty}{\sup\left\{ a_{m}b_{m} \right\}_{m = n}^{\infty}} \leq \lim_{n \rightarrow \infty}{\sup\left\{ a_{m} \right\}_{m = n}^{\infty}}\lim_{n \rightarrow \infty}{\sup\left\{ b_{m} \right\}_{m = n}^{\infty}}
\end{align*}
したがって、次式が成り立つ。
\begin{align*}
\limsup_{n \rightarrow \infty}\left( a_{n}b_{n} \right) \leq \limsup_{n \rightarrow \infty}a_{n}\limsup_{n \rightarrow \infty}b_{n}
\end{align*}
同様にして、次式が成り立つことが示される。
\begin{align*}
\liminf_{n \rightarrow \infty}a_{n}\liminf_{n \rightarrow \infty}b_{n} \leq \liminf_{n \rightarrow \infty}\left( a_{n}b_{n} \right)
\end{align*}
また、定義より$\forall n \in \mathbb{N}$に対し、次式が成り立つので、
\begin{align*}
\inf\left\{ a_{m}b_{m} \right\}_{m = n}^{\infty} \leq \sup\left\{ a_{m}b_{m} \right\}_{m = n}^{\infty}
\end{align*}
両辺に$n \rightarrow \infty$とすれば、次のようになる。
\begin{align*}
\lim_{n \rightarrow \infty}{\inf\left\{ a_{m}b_{m} \right\}_{m = n}^{\infty}} \leq \lim_{n \rightarrow \infty}{\sup\left\{ a_{m}b_{m} \right\}_{m = n}^{\infty}}
\end{align*}
したがって、次式が成り立つ。
\begin{align*}
\liminf_{n \rightarrow \infty}\left( a_{n}b_{n} \right) \leq \limsup_{n \rightarrow \infty}\left( a_{n}b_{n} \right)
\end{align*}
以上の議論により次式が成り立つ。
\begin{align*}
\liminf_{n \rightarrow \infty}a_{n}\liminf_{n \rightarrow \infty}b_{n} \leq \liminf_{n \rightarrow \infty}\left( a_{n}b_{n} \right) \leq \limsup_{n \rightarrow \infty}\left( a_{n}b_{n} \right) \leq \limsup_{n \rightarrow \infty}a_{n}\limsup_{n \rightarrow \infty}b_{n}
\end{align*}\par
特に、$\forall c \in \mathbb{R}^{+}$に対し、上記の議論により次のようになるかつ、
\begin{align*}
\limsup_{n \rightarrow \infty}\left( ca_{n} \right) \leq \limsup_{n \rightarrow \infty}c\limsup_{n \rightarrow \infty}a_{n} = c\limsup_{n \rightarrow \infty}a_{n}
\end{align*}
次のようになるので、
\begin{align*}
c\limsup_{n \rightarrow \infty}a_{n} = c\limsup_{n \rightarrow \infty}\left( \frac{ca_{n}}{c} \right) \leq \frac{c}{c}\limsup_{n \rightarrow \infty}\left( ca_{n} \right) = \limsup_{n \rightarrow \infty}\left( ca_{n} \right)
\end{align*}
次式が成り立つ。
\begin{align*}
\limsup_{n \rightarrow \infty}\left( ca_{n} \right) = c\limsup_{n \rightarrow \infty}a_{n}
\end{align*}
同様にして、次式が成り立つことが示される。
\begin{align*}
\liminf_{n \rightarrow \infty}\left( ca_{n} \right) = c\liminf_{n \rightarrow \infty}a_{n}
\end{align*}
また、$\forall c \in \mathbb{R}$に対し、$c < 0$のとき、$- c \in \mathbb{R}^{+}$より次式が成り立つ。
\begin{align*}
\liminf_{n \rightarrow \infty}\left( - ca_{n} \right) = - c\liminf_{n \rightarrow \infty}a_{n}
\end{align*}
両辺に$- 1$をかけると、次のようになり、
\begin{align*}
- \liminf_{n \rightarrow \infty}\left( - ca_{n} \right) = c\liminf_{n \rightarrow \infty}a_{n}
\end{align*}
ここで、次のようになることから、
\begin{align*}
- \liminf_{n \rightarrow \infty}\left( - ca_{n} \right) &= \lim_{n \rightarrow \infty}\left( - \inf\left\{ - ca_{m} \right\}_{m = n}^{\infty} \right)\\
&= \lim_{n \rightarrow \infty}{\sup\left\{ ca_{m} \right\}_{m = n}^{\infty}}\\
&= \limsup_{n \rightarrow \infty}\left( ca_{n} \right)
\end{align*}
次式が成り立つ。
\begin{align*}
\limsup_{n \rightarrow \infty}\left( ca_{n} \right) = c\liminf_{n \rightarrow \infty}a_{n}
\end{align*}
同様にして、次式が成り立つことが示される。
\begin{align*}
\liminf_{n \rightarrow \infty}\left( ca_{n} \right) = c\limsup_{n \rightarrow \infty}a_{n}
\end{align*}
\end{proof}
\begin{thm}\label{4.1.6.8}
任意の実数列$\left( a_{n} \right)_{n \in \mathbb{N}}$が与えられたとき、$0 < \left( a_{n} \right)_{n \in \mathbb{N}}$が成り立つなら、次の不等式が成り立つ。
\begin{align*}
\liminf_{n \rightarrow \infty}\frac{a_{n + 1}}{a_{n}} \leq \liminf_{n \rightarrow \infty}\sqrt[n]{a_{n}} \leq \limsup_{n \rightarrow \infty}\sqrt[n]{a_{n}} \leq \limsup_{n \rightarrow \infty}\frac{a_{n + 1}}{a_{n}}
\end{align*}
\end{thm}
\begin{proof}
任意の実数列$\left( a_{n} \right)_{n \in \mathbb{N}}$が与えられたとき、$0 < \left( a_{n} \right)_{n \in \mathbb{N}}$が成り立つなら、$\limsup_{n \rightarrow \infty}\frac{a_{n + 1}}{a_{n}} = \infty$のとき、$\limsup_{n \rightarrow \infty}\sqrt[n]{a_{n}} \leq \limsup_{n \rightarrow \infty}\frac{a_{n + 1}}{a_{n}}$が成り立つことは自明である。そこで、$\limsup_{n \rightarrow \infty}\frac{a_{n + 1}}{a_{n}} = a \in \mathbb{R}$が成り立つときで考えよう。このとき、$\forall\varepsilon \in \mathbb{R}^{+}\exists N \in \mathbb{N}$に対し、$N < n$が成り立つなら、次のようになる。
\begin{align*}
\left| \sup\left\{ \frac{a_{m + 1}}{a_{m}} \right\}_{m = n}^{\infty} - a \right| < \varepsilon &\Rightarrow \frac{a_{n + 1}}{a_{n}} \leq \sup\left\{ \frac{a_{m + 1}}{a_{m}} \right\}_{m = n}^{\infty} < a + \varepsilon\\
&\Rightarrow \frac{a_{n + 1}}{a_{n}} < a + \varepsilon\\
&\Leftrightarrow a_{n + 1} < (a + \varepsilon)a_{n}
\end{align*}
ここで、数学的帰納法により$a_{n + 1} < (a + \varepsilon)a_{n} < (a + \varepsilon)^{n - N + 1}a_{N}$が成り立つので、次のようになる。
\begin{align*}
\left| \sup\left\{ a_{m} \right\}_{m = n}^{\infty} - a \right| < \varepsilon &\Rightarrow a_{n + 1} < (a + \varepsilon)^{n - N + 1}a_{N}\\
&\Leftrightarrow \sqrt[{n + 1}]{a_{n + 1}} < (a + \varepsilon)^{1 - \frac{N}{n + 1}}\sqrt[{n + 1}]{a_{N}}\\
&\Leftrightarrow \sqrt[{n + 1}]{a_{n + 1}} < (a + \varepsilon)\left\{ (a + \varepsilon)^{- N}a_{N} \right\}^{\frac{1}{n + 1}}\\
&\Rightarrow \sup\left\{ \sqrt[m]{a_{m}} \right\}_{m = n + 1}^{\infty} \leq (a + \varepsilon)\left\{ (a + \varepsilon)^{- N}a_{N} \right\}^{\frac{1}{n + 1}}
\end{align*}
ここで、次式について、
\begin{align*}
\sup\left\{ \sqrt[m]{a_{m}} \right\}_{m = n + 1}^{\infty} \leq (a + \varepsilon)\left\{ (a + \varepsilon)^{- N}a_{N} \right\}^{\frac{1}{n + 1}}
\end{align*}
$n \rightarrow \infty$とすれば、次のようになり、
\begin{align*}
\limsup_{n \rightarrow \infty}\sqrt[n]{a_{n}} &= \lim_{n \rightarrow \infty}{\sup\left\{ \sqrt[m]{a_{m}} \right\}_{m = n + 1}^{\infty}}\\
&\leq \lim_{n \rightarrow \infty}{(a + \varepsilon)\left\{ (a + \varepsilon)^{- N}a_{N} \right\}^{\frac{1}{n + 1}}}\\
&= (a + \varepsilon)\lim_{n \rightarrow \infty}\left\{ (a + \varepsilon)^{- N}a_{N} \right\}^{\frac{1}{n + 1}}\\
&= a + \varepsilon = \limsup_{n \rightarrow \infty}\frac{a_{n + 1}}{a_{n}} + \varepsilon
\end{align*}
その正の実数$\varepsilon$の任意性よりよって、次式が得られる。
\begin{align*}
\limsup_{n \rightarrow \infty}\sqrt[n]{a_{n}} \leq \limsup_{n \rightarrow \infty}\frac{a_{n + 1}}{a_{n}}
\end{align*}\par
同様にして、次式が成り立つことが示される。
\begin{align*}
\liminf_{n \rightarrow \infty}\frac{a_{n + 1}}{a_{n}} \leq \liminf_{n \rightarrow \infty}\sqrt[n]{a_{n}}
\end{align*}
次式が成り立つことはすでに示されているのであった。
\begin{align*}
\liminf_{n \rightarrow \infty}\sqrt[n]{a_{n}} \leq \limsup_{n \rightarrow \infty}\sqrt[n]{a_{n}}
\end{align*}
よって、次式が成り立つ。
\begin{align*}
\liminf_{n \rightarrow \infty}\frac{a_{n + 1}}{a_{n}} \leq \liminf_{n \rightarrow \infty}\sqrt[n]{a_{n}} \leq \limsup_{n \rightarrow \infty}\sqrt[n]{a_{n}} \leq \limsup_{n \rightarrow \infty}\frac{a_{n + 1}}{a_{n}}
\end{align*}
\end{proof}
\begin{thebibliography}{50}
  \bibitem{1}
  杉浦光夫, 解析入門I, 東京大学出版社, 1985. 第34刷 p362-366 ISBN978-4-13-062005-5
  \bibitem{2}
  数学の景色. 上極限,下極限(limsup,liminf)の定義と例と性質2つ". 数学の景色. \url{https://mathlandscape.com/limsup-liminf/} (2022-8-3 15:47 閲覧)
\end{thebibliography}
\end{document}
