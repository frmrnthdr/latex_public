\documentclass[dvipdfmx]{jsarticle}
\setcounter{section}{1}
\setcounter{subsection}{10}
\usepackage{xr}
\externaldocument{4.1.7}
\externaldocument{4.1.8}
\externaldocument{4.1.10}
\usepackage{amsmath,amsfonts,amssymb,array,comment,mathtools,url,docmute}
\usepackage{longtable,booktabs,dcolumn,tabularx,mathtools,multirow,colortbl,xcolor}
\usepackage[dvipdfmx]{graphics}
\usepackage{bmpsize}
\usepackage{amsthm}
\usepackage{enumitem}
\setlistdepth{20}
\renewlist{itemize}{itemize}{20}
\setlist[itemize]{label=•}
\renewlist{enumerate}{enumerate}{20}
\setlist[enumerate]{label=\arabic*.}
\setcounter{MaxMatrixCols}{20}
\setcounter{tocdepth}{3}
\newcommand{\rotin}{\text{\rotatebox[origin=c]{90}{$\in $}}}
\newcommand{\amap}[6]{\text{\raisebox{-0.7cm}{\begin{tikzpicture} 
  \node (a) at (0, 1) {$\textstyle{#2}$};
  \node (b) at (#6, 1) {$\textstyle{#3}$};
  \node (c) at (0, 0) {$\textstyle{#4}$};
  \node (d) at (#6, 0) {$\textstyle{#5}$};
  \node (x) at (0, 0.5) {$\rotin $};
  \node (x) at (#6, 0.5) {$\rotin $};
  \draw[->] (a) to node[xshift=0pt, yshift=7pt] {$\textstyle{\scriptstyle{#1}}$} (b);
  \draw[|->] (c) to node[xshift=0pt, yshift=7pt] {$\textstyle{\scriptstyle{#1}}$} (d);
\end{tikzpicture}}}}
\newcommand{\twomaps}[9]{\text{\raisebox{-0.7cm}{\begin{tikzpicture} 
  \node (a) at (0, 1) {$\textstyle{#3}$};
  \node (b) at (#9, 1) {$\textstyle{#4}$};
  \node (c) at (#9+#9, 1) {$\textstyle{#5}$};
  \node (d) at (0, 0) {$\textstyle{#6}$};
  \node (e) at (#9, 0) {$\textstyle{#7}$};
  \node (f) at (#9+#9, 0) {$\textstyle{#8}$};
  \node (x) at (0, 0.5) {$\rotin $};
  \node (x) at (#9, 0.5) {$\rotin $};
  \node (x) at (#9+#9, 0.5) {$\rotin $};
  \draw[->] (a) to node[xshift=0pt, yshift=7pt] {$\textstyle{\scriptstyle{#1}}$} (b);
  \draw[|->] (d) to node[xshift=0pt, yshift=7pt] {$\textstyle{\scriptstyle{#2}}$} (e);
  \draw[->] (b) to node[xshift=0pt, yshift=7pt] {$\textstyle{\scriptstyle{#1}}$} (c);
  \draw[|->] (e) to node[xshift=0pt, yshift=7pt] {$\textstyle{\scriptstyle{#2}}$} (f);
\end{tikzpicture}}}}
\renewcommand{\thesection}{第\arabic{section}部}
\renewcommand{\thesubsection}{\arabic{section}.\arabic{subsection}}
\renewcommand{\thesubsubsection}{\arabic{section}.\arabic{subsection}.\arabic{subsubsection}}
\everymath{\displaystyle}
\allowdisplaybreaks[4]
\usepackage{vtable}
\theoremstyle{definition}
\newtheorem{thm}{定理}[subsection]
\newtheorem*{thm*}{定理}
\newtheorem{dfn}{定義}[subsection]
\newtheorem*{dfn*}{定義}
\newtheorem{axs}[dfn]{公理}
\newtheorem*{axs*}{公理}
\renewcommand{\headfont}{\bfseries}
\makeatletter
  \renewcommand{\section}{%
    \@startsection{section}{1}{\z@}%
    {\Cvs}{\Cvs}%
    {\normalfont\huge\headfont\raggedright}}
\makeatother
\makeatletter
  \renewcommand{\subsection}{%
    \@startsection{subsection}{2}{\z@}%
    {0.5\Cvs}{0.5\Cvs}%
    {\normalfont\LARGE\headfont\raggedright}}
\makeatother
\makeatletter
  \renewcommand{\subsubsection}{%
    \@startsection{subsubsection}{3}{\z@}%
    {0.4\Cvs}{0.4\Cvs}%
    {\normalfont\Large\headfont\raggedright}}
\makeatother
\makeatletter
\renewenvironment{proof}[1][\proofname]{\par
  \pushQED{\qed}%
  \normalfont \topsep6\p@\@plus6\p@\relax
  \trivlist
  \item\relax
  {
  #1\@addpunct{.}}\hspace\labelsep\ignorespaces
}{%
  \popQED\endtrivlist\@endpefalse
}
\makeatother
\renewcommand{\proofname}{\textbf{証明}}
\usepackage{tikz,graphics}
\usepackage[dvipdfmx]{hyperref}
\usepackage{pxjahyper}
\hypersetup{
 setpagesize=false,
 bookmarks=true,
 bookmarksdepth=tocdepth,
 bookmarksnumbered=true,
 colorlinks=false,
 pdftitle={},
 pdfsubject={},
 pdfauthor={},
 pdfkeywords={}}
\begin{document}
%\hypertarget{ux95a2ux6570ux5217}{%
\subsection{関数列}%\label{ux95a2ux6570ux5217}}
%\hypertarget{ux95a2ux6570ux5217-1}{%
\subsubsection{関数列}%\label{ux95a2ux6570ux5217-1}}
\begin{dfn}
$A \subseteq \mathbb{R}^{m}$、$B \subseteq \mathbb{R}_{\infty}^{n}$なる関数$f:A \rightarrow B$全体の集合$\mathfrak{F}(A,B)$を関数空間といいこれの元の列を関数列という。
\end{dfn}
\begin{dfn}
$A \subseteq \mathbb{R}^{m}$、$B \subseteq \mathbb{R}^{n}$なる有界な関数$f:A \rightarrow B$全体の集合を有界関数空間といいこれを$\mathfrak{B}(A,B)$と書くことにする。
\end{dfn}
\begin{thm}\label{4.1.11.1}
$A \subseteq \mathbb{R}^{m}$、$B \subseteq \mathbb{R}^{n}$なる有界関数空間$\mathfrak{B}(A,B)$は体$\mathbb{R}$上のvector空間をなす\footnote{$\mathbb{R}$のかわりに$\mathbb{C}$としても同様にして示される。}。
\end{thm}
\begin{proof}
$\forall k,l \in \mathbb{R}\forall f,g \in \mathfrak{B}(A,B)$に対し、$\exists M,N \in \mathbb{R}^{+}$に対し、$\left\| f \right\| < M$、$\left\| g \right\| < N$が成り立つので、三角不等式より次のようになる。
\begin{align*}
\left\| kf + lg \right\| \leq \left\| kf \right\| + \left\| \lg \right\| = |k|\left\| f \right\| + |l|\left\| g \right\| \leq |k|M + |l|N
\end{align*}
ゆえに、$kf + lg \in \mathfrak{B}(A,B)$が成り立つ。あとは明らかである。
\end{proof}
\begin{dfn}
$D(f) \subseteq \mathbb{R}^{n}$なる関数$f:D(f) \rightarrow \mathbb{R}$が与えられたとき、上限$\sup{V(f)}$をその集合$D(f)$におけるその関数$f$の上限といい$\sup_{\mathbf{x} \in D(f)}{f\left( \mathbf{x} \right)}$、特にその集合$D(f)$が明らかな場合では、$\sup f$などと書く。
\end{dfn}
\begin{dfn}
同様に、$D(f) \subseteq \mathbb{R}^{n}$なる関数$f:D(f) \rightarrow \mathbb{R}$が与えられたとき、下限$\inf{V(f)}$をその集合$D(f)$におけるその関数$f$の下限といい$\inf_{\mathbf{x} \in D(f)}{f\left( \mathbf{x} \right)}$、特にその集合$D(f)$が明らかな場合では、$\inf f$などと書く。
\end{dfn}
\begin{dfn}
$A \subseteq \mathbb{R}^{m}$、$B \subseteq \mathbb{R}^{n}$なる有界関数空間$\mathfrak{B}(A,B)$が与えられたとき、次のような写像$\left\| \bullet \right\|_{A,\infty}$をその有界関数空間$\mathfrak{B}(A,B)$の一様normという。
\begin{align*}
\left\| \bullet \right\|_{A,\infty}\mathfrak{:B}(A,B) \rightarrow \mathbb{R}^{+} \cup \left\{ 0 \right\};f \mapsto \sup\left\| f \right\| = \sup_{\mathbf{x} \in A}\left\| f\left( \mathbf{x} \right) \right\|
\end{align*}
\end{dfn}
\begin{thm}\label{4.1.11.2}
$A \subseteq \mathbb{R}^{m}$、$B \subseteq \mathbb{R}^{n}$なる有界関数空間$\mathfrak{B}(A,B)$の一様norm$\left\| \bullet \right\|_{A,\infty}$を用いた組$\left( \mathfrak{B}(A,B),\left\| \bullet \right\|_{A,\infty} \right)$は体$\mathbb{R}$上のnorm空間をなす、即ち、次のことが成り立つ。
\begin{itemize}
\item
  $\forall f \in \mathfrak{B}(A,B)$に対し、$\left\| f \right\|_{\infty} = 0$が成り立つならそのときに限り、$f = 0$が成り立つ。
\item
  $\forall f \in \mathfrak{B}(A,B)\forall k \in \mathbb{R}$に対し、$\left\| kf \right\|_{\infty} = |k|\left\| f \right\|_{\infty}$が成り立つ。
\item
  $\forall f,g \in \mathfrak{B}(A,B)$に対し、$\left\| f + g \right\|_{\infty} \leq \left\| f \right\|_{\infty} + \left\| g \right\|_{\infty}$が成り立つ。
\end{itemize}
\end{thm}
\begin{proof}
$A \subseteq \mathbb{R}^{m}$、$B \subseteq \mathbb{R}^{n}$なる有界関数空間$\mathfrak{B}(A,B)$の一様norm$\left\| \bullet \right\|_{A,\infty}$を用いた組$\left( \mathfrak{B}(A,B),\left\| \bullet \right\|_{A,\infty} \right)$が与えられたとき、$\forall f \in \mathfrak{B}(A,B)$に対し、$\left\| f \right\|_{A,\infty} = 0$が成り立つなら、$0 \leq \left\| f \right\| \leq \sup\left\| f \right\| = \left\| f \right\|_{A,\infty} = 0$が成り立つので、$f = 0$が得られる。逆は明らかである。\par
$\forall f \in \mathfrak{B}(A,B)\forall k \in \mathbb{R}$に対し、$\left\| kf \right\| = |k|\left\| f \right\|$が成り立つので、$\left\| kf \right\|_{A,\infty} = \sup\left\| kf \right\| = \sup{|k|\left\| f \right\|} = |k|\sup\left\| f \right\| = |k|\left\| f \right\|_{A,\infty}$が成り立つ。\par
$\forall f,g \in \mathfrak{B}(A,B)$に対し、次式が成り立つので、
\begin{align*}
\left\| f + g \right\| \leq \left\| f \right\| + \left\| g \right\| \leq \left\| f \right\|_{A,\infty} + \left\| g \right\|_{A,\infty}
\end{align*}
$\left\| f + g \right\|_{A,\infty} \leq \left\| f \right\|_{A,\infty} + \left\| g \right\|_{A,\infty}$が成り立つ。\par
よって、その組$\left( \mathfrak{B}(A,B),\left\| \bullet \right\|_{A,\infty} \right)$は体$\mathbb{R}$上のnorm空間をなす。
\end{proof}
\begin{thm}\label{4.1.11.3}
$A \subseteq \mathbb{R}^{m}$、$B \subseteq \mathbb{R}^{n}$なる有界関数空間$\mathfrak{B}(A,B)$が与えられたとき、$\forall D,E \in \mathfrak{P}(A)\forall f \in \mathfrak{B}(A,B)$に対し、$D \subseteq E$が成り立つなら、$\left\| f|D \right\|_{D,\infty} \leq \left\| f|E \right\|_{E,\infty}$が成り立つ。
\end{thm}
\begin{proof}
$A \subseteq \mathbb{R}^{m}$、$B \subseteq \mathbb{R}^{n}$なる有界関数空間$\mathfrak{B}(A,B)$が与えられたとき、$\forall D,E \in \mathfrak{P}(A)\forall f \in \mathfrak{B}(A,B)$に対し、$D \subseteq E$が成り立つなら、もちろん、$f|D:D \rightarrow B \in \mathfrak{B}(D,B)$かつ$f|E:E \rightarrow B \in \mathfrak{B}(E,B)$が成り立つ。このとき、$V\left( f|D \right) \subseteq V\left( f|E \right)$が成り立つので、次のようになる。
\begin{align*}
\left\| f|D \right\|_{D,\infty} = \sup\left\| f|D \right\| = \sup{V\left( f|D \right)} \leq \sup{V\left( f|E \right)} = \sup\left\| f|E \right\| = \left\| f|E \right\|_{E,\infty}
\end{align*}
\end{proof}
%\hypertarget{ux5404ux70b9ux53ceux675f}{%
\subsubsection{各点収束}%\label{ux5404ux70b9ux53ceux675f}}
\begin{dfn}
$A \subseteq R \subseteq \mathbb{R}^{m}$、$S \subseteq \mathbb{R}_{\infty}^{n}$なる関数空間$\mathfrak{F}(A,S)$の関数列$\left( f_{k} \right)_{k \in \mathbb{N}}$が与えられたとする。$\forall\mathbf{x} \in A$に対し、極限値$\lim_{k \rightarrow \infty}{f_{k}\left( \mathbf{x} \right)}$がその集合$S$に存在するとき、$f\left( \mathbf{x} \right) = \lim_{k \rightarrow \infty}{f_{k}\left( \mathbf{x} \right)}$とおくと、この関数列$\left( f_{k} \right)_{k \in \mathbb{N}}$はその集合$A$上でその関数$f:A \rightarrow S$に各点収束するといい、その関数$f$をその関数列$\left( f_{k} \right)_{k \in \mathbb{N}}$の極限関数といい$\lim_{k \rightarrow \infty}f_{k}$と書くことにする。
\end{dfn}
\begin{dfn}
$A \subseteq R \subseteq \mathbb{R}^{m}$、$S \subseteq \mathbb{R}_{\infty}^{n}$なる関数空間$\mathfrak{F}(A,S)$の関数列$\left( f_{k} \right)_{k \in \mathbb{N}}$が与えられたとする。$\forall\mathbf{x} \in A$に対し、その点列$\left( f_{k}\left( \mathbf{x} \right) \right)_{k \in \mathbb{N}}$から誘導される級数$\left( \sum_{i \in \varLambda_{k}} {f_{i}\left( \mathbf{x} \right)} \right)_{k \in \mathbb{N}}$が収束するとき、$f\left( \mathbf{x} \right) = \sum_{k \in \mathbb{N}} {f_{k}\left( \mathbf{x} \right)}$とおくと、この級数$\left( \sum_{i \in \varLambda_{k}} f_{i} \right)_{k \in \mathbb{N}}$はその集合$A$上でその極限関数$f:A \rightarrow S$に各点収束するといい、その極限関数$f$を$\sum_{k \in \mathbb{N}} f_{k}$と書くことにする。
\end{dfn}\par
さて、$A \subseteq R \subseteq \mathbb{R}^{m}$、$S \subseteq \mathbb{R}_{\infty}^{n}$なる関数空間$\mathfrak{F}(A,S)$の関数列$\left( f_{k} \right)_{k \in \mathbb{N}}$が与えられたとき、$\forall k \in \mathbb{N}$に対し、それらの関数たち$f_{k}:A \rightarrow S$がその集合$A$で連続であっても、その極限関数$f$はその集合$A$で連続であるとは限らないことに注意されたい。例えば、関数列$\left( f_{k}:[ 0,1] \rightarrow \mathbb{R};x \mapsto x^{k} \right)_{k \in \mathbb{N}}$が挙げられる。実際、$\forall x \in [ 0,1)$に対し、$f_{k}(x) = x^{k}$なので、$k \rightarrow \infty$とすれば、$f = \lim_{k \rightarrow \infty}f_{k}$として$f(x) = 0$が成り立つかつ、$f_{k}(1) = 1$なので、$k \rightarrow \infty$としても$f(1) = 1$が成り立つ\footnote{杉浦先生の解析入門Iによれば、これはAbel先生がはじめて指摘したらしいです。}。
\begin{thm}\label{4.1.11.4}
$A \subseteq R \subseteq \mathbb{R}^{m}$、$S \subseteq \mathbb{R}^{n}$なる関数空間$\mathfrak{F}(A,S)$の関数列$\left( f_{k} \right)_{k \in \mathbb{N}}$が与えられたとき、次のことは同値である。
\begin{itemize}
\item
  その関数列$\left( f_{k} \right)_{k \in \mathbb{N}}$がその集合$A$上でその極限関数$f:A \rightarrow S$に各点収束する。
\item
  $\forall\varepsilon \in \mathbb{R}^{+}\forall\mathbf{x} \in A\exists N \in \mathbb{N}\forall k \in \mathbb{N}$に対し、$N \leq k$が成り立つなら、$\left\| f_{k}\left( \mathbf{x} \right) - f\left( \mathbf{x} \right) \right\| < \varepsilon$が成り立つ。
\end{itemize}
\end{thm}
\begin{proof}
$A \subseteq R \subseteq \mathbb{R}^{m}$、$S \subseteq \mathbb{R}^{n}$なる関数空間$\mathfrak{F}(A,S)$の関数列$\left( f_{k} \right)_{k \in \mathbb{N}}$が与えられたとき、その関数列$\left( f_{k} \right)_{k \in \mathbb{N}}$がその集合$A$上でその関数$f:A \rightarrow S$に各点収束するならそのときに限り、$\forall\mathbf{x} \in A$に対し、$f\left( \mathbf{x} \right) = \lim_{k \rightarrow \infty}{f_{k}\left( \mathbf{x} \right)}$が成り立つので、これが成り立つならそのときに限り、$\forall\mathbf{x} \in A\forall\varepsilon \in \mathbb{R}^{+}\exists N \in \mathbb{N}\forall k \in \mathbb{N}$に対し、$N \leq k$が成り立つなら、$\left\| f_{k}\left( \mathbf{x} \right) - f\left( \mathbf{x} \right) \right\| < \varepsilon$が成り立つ。これが成り立つならそのときに限り、$\forall\varepsilon \in \mathbb{R}^{+}\forall\mathbf{x} \in A\exists N \in \mathbb{N}\forall k \in \mathbb{N}$に対し、$N \leq k$が成り立つなら、$\left\| f_{k}\left( \mathbf{x} \right) - f\left( \mathbf{x} \right) \right\| < \varepsilon$が成り立つ。よって、次のことは同値であることが示された。
\begin{itemize}
\item
  その関数列$\left( f_{k} \right)_{k \in \mathbb{N}}$がその集合$A$上でその極限関数$f:A \rightarrow S$に各点収束する。
\item
  $\forall\varepsilon \in \mathbb{R}^{+}\forall\mathbf{x} \in A\exists N \in \mathbb{N}\forall k \in \mathbb{N}$に対し、$N \leq k$が成り立つなら、$\left\| f_{k}\left( \mathbf{x} \right) - f\left( \mathbf{x} \right) \right\| < \varepsilon$が成り立つ。
\end{itemize}
\end{proof}
\begin{thm}\label{4.1.11.5}
$A \subseteq R \subseteq \mathbb{R}^{m}$なる関数空間$\mathfrak{F}\left( A,\mathbb{R}^{n} \right)$の関数列$\left( f_{k} \right)_{k \in \mathbb{N}}$が与えられたとする。その級数$\left( \sum_{i \in \varLambda_{k}} \left\| f_{i} \right\| \right)_{k \in \mathbb{N}}$が収束するとき、その級数$\left( \sum_{i \in \varLambda_{k}} f_{i} \right)_{k \in \mathbb{N}}$も各点収束する。
\end{thm}
\begin{dfn}
$A \subseteq R \subseteq \mathbb{R}^{m}$なる関数空間$\mathfrak{F}\left( A,\mathbb{R}^{n} \right)$の関数列$\left( f_{k} \right)_{k \in \mathbb{N}}$が与えられたとする。その級数$\left( \sum_{i \in \varLambda_{k}} \left\| f_{i} \right\| \right)_{k \in \mathbb{N}}$が収束するとき、その級数$\left( \sum_{i \in \varLambda_{k}} f_{i} \right)_{k \in \mathbb{N}}$はその集合$A$上で絶対収束するという。逆に、その級数$\left( \sum_{i \in \varLambda_{k}} f_{i} \right)_{k \in \mathbb{N}}$が収束するかつその級数$\left( \sum_{i \in \varLambda_{k}} \left\| f_{i} \right\| \right)_{k \in \mathbb{N}}$が収束しないとき、その級数$\left( \sum_{i \in \varLambda_{k}} f_{i} \right)_{k \in \mathbb{N}}$はその集合$A$上で条件収束するという。
\end{dfn}
\begin{proof} 定理\ref{4.1.8.13}より明らかである。実際、その級数$\left( \sum_{i \in \varLambda_{k}} \left\| f_{i} \right\| \right)_{k \in \mathbb{N}}$が収束するならそのときに限り、$\forall\mathbf{x} \in A$に対し、その級数$\left( \sum_{i \in \varLambda_{k}} \left\| f_{i}\left( \mathbf{x} \right) \right\| \right)_{k \in \mathbb{N}}$が収束することになる。そこで、定理\ref{4.1.8.13}よりその級数$\left( \sum_{i \in \varLambda_{k}} {f_{i}\left( \mathbf{x} \right)} \right)_{k \in \mathbb{N}}$も収束するので、その級数$\left( \sum_{i \in \varLambda_{k}} f_{i} \right)_{k \in \mathbb{N}}$も各点収束する。
\end{proof}
%\hypertarget{ux4e00ux69d8ux53ceux675f}{%
\subsubsection{一様収束}%\label{ux4e00ux69d8ux53ceux675f}}
\begin{dfn}
$A \subseteq R \subseteq \mathbb{R}^{m}$、$S \subseteq \mathbb{R}^{n}$なる有界関数空間$\mathfrak{B}(A,S)$の関数列$\left( f_{k} \right)_{k \in \mathbb{N}}$が与えられたとする。ある関数$f:A \rightarrow S$が存在して、$\lim_{k \rightarrow \infty}\left\| f_{k} - f \right\|_{A,\infty} = 0$が成り立つとき、この関数列$\left( f_{k} \right)_{k \in \mathbb{N}}$はその集合$A$上でその関数$f:A \rightarrow S$に一様収束するという。
\end{dfn}
\begin{dfn}
$A \subseteq R \subseteq \mathbb{R}^{m}$、$S \subseteq \mathbb{R}^{n}$なる有界関数空間$\mathfrak{B}(A,S)$の関数列$\left( f_{k} \right)_{k \in \mathbb{N}}$が与えられたとする。その関数列$\left( f_{k} \right)_{k \in \mathbb{N}}$から誘導される級数$\left( \sum_{i \in \varLambda_{k}} f_{i} \right)_{k \in \mathbb{N}}$について、ある関数$f:A \rightarrow S$が存在して、$\lim_{k \rightarrow \infty}\left\| \sum_{i \in \varLambda_{k}} f_{i} - f \right\|_{A,\infty} = 0$が成り立つとき、この級数$\left( \sum_{i \in \varLambda_{k}} f_{i} \right)_{k \in \mathbb{N}}$はその集合$A$上でその関数$f:A \rightarrow S$に一様収束するという。
\end{dfn}
\begin{thm}\label{4.1.11.6}
$A \subseteq R \subseteq \mathbb{R}^{m}$、$S \subseteq \mathbb{R}^{n}$なる有界関数空間$\mathfrak{B}(A,S)$の関数列$\left( f_{k} \right)_{k \in \mathbb{N}}$が与えられたとき、次のことは同値である。
\begin{itemize}
\item
  その関数列$\left( f_{k} \right)_{k \in \mathbb{N}}$がその集合$A$上でその関数$f:A \rightarrow S$に一様収束する。
\item
  $\forall\varepsilon \in \mathbb{R}^{+}\exists N \in \mathbb{N}\forall\mathbf{x} \in A\forall k \in \mathbb{N}$に対し、$N \leq k$が成り立つなら、$\left\| f_{k}\left( \mathbf{x} \right) - f\left( \mathbf{x} \right) \right\| < \varepsilon$が成り立つ。
\end{itemize}
\end{thm}\par
この定理と定理\ref{4.1.11.4}によりその関数列$\left( f_{k} \right)_{k \in \mathbb{N}}$がその集合$A$上でその関数$f:A \rightarrow S$に各点収束するときの違いがよく表れているのであろう。
\begin{proof}
$A \subseteq R \subseteq \mathbb{R}^{m}$、$S \subseteq \mathbb{R}^{n}$なる有界関数空間$\mathfrak{B}(A,S)$の関数列$\left( f_{k} \right)_{k \in \mathbb{N}}$が与えられたとき、その関数列$\left( f_{k} \right)_{k \in \mathbb{N}}$がその集合$A$上でその関数$f:A \rightarrow S$に一様収束するならそのときに限り、$\lim_{k \rightarrow \infty}\left\| f_{k} - f \right\|_{A,\infty} = 0$が成り立つ。これが成り立つならそのときに限り、$\forall\varepsilon \in \mathbb{R}^{+}\exists N \in \mathbb{N}\forall k \in \mathbb{N}$に対し、$N \leq k$が成り立つなら、$\left\| f_{k} - f \right\|_{A,\infty} < \varepsilon$が成り立つ。ここで、$\forall\mathbf{x} \in A$に対し、$\left\| f_{k}\left( \mathbf{x} \right) - f\left( \mathbf{x} \right) \right\| \leq \sup_{\mathbf{x} \in A}\left\| f_{k}\left( \mathbf{x} \right) - f\left( \mathbf{x} \right) \right\| = \left\| f_{k} - f \right\|_{A,\infty} < \varepsilon$が成り立つので、$\forall\varepsilon \in \mathbb{R}^{+}\exists N \in \mathbb{N}\forall\mathbf{x} \in A\forall k \in \mathbb{N}$に対し、$N \leq k$が成り立つなら、$\left\| f_{k}\left( \mathbf{x} \right) - f\left( \mathbf{x} \right) \right\| < \varepsilon$が成り立つ。\par
逆に、$\forall\varepsilon \in \mathbb{R}^{+}\exists N \in \mathbb{N}\forall\mathbf{x} \in A\forall k \in \mathbb{N}$に対し、$N \leq k$が成り立つなら、$\left\| f_{k}\left( \mathbf{x} \right) - f\left( \mathbf{x} \right) \right\| < \varepsilon$が成り立つとき、$\forall\varepsilon \in \mathbb{R}^{+}\exists N \in \mathbb{N}\forall k \in \mathbb{N}$に対し、$N \leq k$が成り立つなら、$\sup_{\mathbf{x} \in A}\left\| f_{k}\left( \mathbf{x} \right) - f\left( \mathbf{x} \right) \right\| \leq \varepsilon$が成り立つので、$\forall\varepsilon \in \mathbb{R}^{+}\exists N \in \mathbb{N}\forall k \in \mathbb{N}$に対し、$N \leq k$が成り立つなら、$\left\| f_{k} - f \right\|_{A,\infty} < \varepsilon$が成り立ち、これが成り立つならそのときに限り、$\lim_{k \rightarrow \infty}\left\| f_{k} - f \right\|_{A,\infty} = 0$が成り立つ。よって、その関数列$\left( f_{k} \right)_{k \in \mathbb{N}}$がその集合$A$上でその関数$f:A \rightarrow S$に一様収束する。
\end{proof}\par
先ほどの定理の注意により次の系が得られる。
\begin{thm}\label{4.1.11.7}
$A \subseteq R \subseteq \mathbb{R}^{m}$、$S \subseteq \mathbb{R}^{n}$なる有界関数空間$\mathfrak{B}(A,S)$の関数列$\left( f_{k} \right)_{k \in \mathbb{N}}$が与えられたとき、その関数列$\left( f_{k} \right)_{k \in \mathbb{N}}$がその集合$A$上でその関数$f:A \rightarrow S$に一様収束するなら、その関数列$\left( f_{k} \right)_{k \in \mathbb{N}}$がその集合$A$上でその極限関数$f:A \rightarrow S$に各点収束する。
\end{thm}\par
ただし、これの逆は成り立たないことに注意されたい。
\begin{proof} 定理\ref{4.1.11.4}と定理\ref{4.1.11.6}より明らかである。
\end{proof}\par
ここで、一様収束する関数列の例を挙げておこう。例えば、関数列$\left( f_{k}:\left[ 0,\frac{1}{2} \right] \rightarrow \mathbb{R};x \mapsto x^{k} \right)_{k \in \mathbb{N}}$が挙げられる。実際、$\sup_{x \in \left[ 0,\frac{1}{2} \right]}\left| f_{k}(x) - 0 \right| = \frac{1}{2^{k}}$が成り立つので、$k \rightarrow \infty$とすれば、その関数列$\left( f_{k} \right)_{k \in \mathbb{N}}$は関数$0$に一様収束する。これ以外にも関数列$\left( f_{k}:\mathbb{R}^{+} \rightarrow \mathbb{R};x \mapsto \frac{1}{k + x} \right)_{k \in \mathbb{N}}$が挙げられる。実際、$\sup_{x \in \mathbb{R}^{+}}\left| f_{k}(x) - 0 \right| = \frac{1}{k}$が成り立つので、$k \rightarrow \infty$とすれば、その関数列$\left( f_{k} \right)_{k \in \mathbb{N}}$は関数$0$に一様収束する。逆に、一様収束しない例も挙げておこう。例えば、関数列$\left( f_{k}:[ 0,1] \rightarrow \mathbb{R};x \mapsto x^{k} \right)_{k \in \mathbb{N}}$について、上の例により関数$f:[ 0,1] \rightarrow \mathbb{R};x \mapsto \left\{ \begin{matrix}
0 & \mathrm{if} & x < 1 \\
1 & \mathrm{if} & x = 1 \\
\end{matrix} \right.\ $に各点収束するのであったが、$\sup_{x \in [ 0,1]}\left| f_{k}(x) - f(x) \right| = 1$が成り立つので、$k \rightarrow \infty$としても$0$に収束しない。
\begin{thm}\label{4.1.11.8}
$A \subseteq R \subseteq \mathbb{R}^{m}$、$S \subseteq \mathbb{R}^{n}$なる有界関数空間$\mathfrak{B}(A,S)$の関数列$\left( f_{k} \right)_{k \in \mathbb{N}}$が与えられたとき、$\forall k \in \mathbb{N}$に対し、その関数$f_{k}$がその集合$A$で連続であるかつ、その関数列$\left( f_{k} \right)_{k \in \mathbb{N}}$がその集合$A$上でその関数$f:A \rightarrow S$に一様収束するなら、その極限関数$f$はその集合$A$で連続である。
\end{thm}
\begin{proof}
$A \subseteq R \subseteq \mathbb{R}^{m}$、$S \subseteq \mathbb{R}^{n}$なる有界関数空間$\mathfrak{B}(A,S)$の関数列$\left( f_{k} \right)_{k \in \mathbb{N}}$が与えられたとき、$\forall k \in \mathbb{N}$に対し、その関数$f_{k}$がその集合$A$で連続であるかつ、その関数列$\left( f_{k} \right)_{k \in \mathbb{N}}$がその集合$A$上でその関数$f:A \rightarrow S$に一様収束するとする。このとき、$\forall\varepsilon \in \mathbb{R}^{+}\exists N \in \mathbb{N}\forall k \in \mathbb{N}$に対し、$N \leq k$が成り立つなら、$\left\| f_{k} - f \right\|_{A,\infty} < \varepsilon$が成り立つ。ここで、その関数$f_{k}$がその集合$A$で連続であるので、$\forall\mathbf{x} \in A\exists\delta \in \mathbb{R}^{+}\forall\mathbf{y} \in R$に対し、$\mathbf{y} \in U\left( \mathbf{x},\delta \right) \cap A$が成り立つなら、$f_{k}\left( \mathbf{y} \right) \in U\left( f_{k}\left( \mathbf{x} \right),\varepsilon \right) \cap S$、即ち、$\left\| f_{k}\left( \mathbf{y} \right) - f_{k}\left( \mathbf{x} \right) \right\| < \varepsilon$が成り立つ。このとき、次のようになることから、
\begin{align*}
\left\| f\left( \mathbf{x} \right) - f\left( \mathbf{y} \right) \right\| &= \left\| f\left( \mathbf{x} \right) - f_{k}\left( \mathbf{x} \right) + f_{k}\left( \mathbf{x} \right) - f_{k}\left( \mathbf{y} \right) + f_{k}\left( \mathbf{y} \right) - f\left( \mathbf{y} \right) \right\|\\
&\leq \left\| f\left( \mathbf{x} \right) - f_{k}\left( \mathbf{x} \right) \right\| + \left\| f_{k}\left( \mathbf{x} \right) - f_{k}\left( \mathbf{y} \right) \right\| + \left\| f_{k}\left( \mathbf{y} \right) - f\left( \mathbf{y} \right) \right\|\\
&= \left\| f\left( \mathbf{x} \right) - f_{k}\left( \mathbf{x} \right) \right\| + \left\| f_{k}\left( \mathbf{y} \right) - f_{k}\left( \mathbf{x} \right) \right\| + \left\| f\left( \mathbf{y} \right) - f_{k}\left( \mathbf{y} \right) \right\|\\
&\leq \sup_{\mathbf{x} \in A}\left\| f\left( \mathbf{x} \right) - f_{k}\left( \mathbf{x} \right) \right\| + \left\| f_{k}\left( \mathbf{y} \right) - f_{k}\left( \mathbf{x} \right) \right\| + \sup_{\mathbf{x} \in A}\left\| f\left( \mathbf{x} \right) - f_{k}\left( \mathbf{x} \right) \right\|\\
&= 2\left\| f_{k} - f \right\|_{A,\infty} + \left\| f_{k}\left( \mathbf{y} \right) - f_{k}\left( \mathbf{x} \right) \right\| < 3\varepsilon
\end{align*}
$f\left( \mathbf{y} \right) \in U\left( f\left( \mathbf{x} \right),3\varepsilon \right) \cap S$が得られる。これにより、$\forall\mathbf{x} \in A\forall\varepsilon \in \mathbb{R}^{+}\exists\delta \in \mathbb{R}^{+}\forall\mathbf{y} \in R$に対し、$\mathbf{y} \in U\left( \mathbf{x},\delta \right) \cap A$が成り立つなら、$f\left( \mathbf{y} \right) \in U\left( f\left( \mathbf{x} \right),\varepsilon \right) \cap S$が成り立つことになるので、その極限関数$f$はその集合$A$で連続である。
\end{proof}
%\hypertarget{ux5e83ux7fa9ux4e00ux69d8ux53ceux675f}{%
\subsubsection{広義一様収束}%\label{ux5e83ux7fa9ux4e00ux69d8ux53ceux675f}}
\begin{dfn}
$A \subseteq R \subseteq \mathbb{R}^{m}$、$S \subseteq \mathbb{R}^{n}$なる有界関数空間$\mathfrak{B}(A,S)$の関数列$\left( f_{k} \right)_{k \in \mathbb{N}}$が与えられたとする。$\forall K \in \mathfrak{P}(A)$に対し、その集合$K$がその集合$R$でcompactであるなら、ある関数$f:A \rightarrow S$が存在して、$\lim_{k \rightarrow \infty}\left\| \left( f_{k} - f \right)|K \right\|_{K,\infty} = 0$が成り立つとき、この関数列$\left( f_{k} \right)_{k \in \mathbb{N}}$はその集合$A$上でその関数$f:A \rightarrow S$に広義一様収束する、compact一様収束するという。
\end{dfn}
\begin{dfn}
$A \subseteq R \subseteq \mathbb{R}^{m}$、$S \subseteq \mathbb{R}^{n}$なる有界関数空間$\mathfrak{B}(A,S)$の関数列$\left( f_{k} \right)_{k \in \mathbb{N}}$が与えられたとする。その関数列$\left( f_{k} \right)_{k \in \mathbb{N}}$から誘導される級数$\left( \sum_{i \in \varLambda_{k}} f_{i} \right)_{k \in \mathbb{N}}$について、$\forall K \in \mathfrak{P}(A)$に対し、その集合$K$がその集合$R$でcompactであるなら、ある関数$f:A \rightarrow S$が存在して、$\lim_{k \rightarrow \infty}\left\| \left( \sum_{i \in \varLambda_{k}} f_{i} - f \right)|K \right\|_{K,\infty} = 0$が成り立つとき、この級数$\left( \sum_{i \in \varLambda_{k}} f_{i} \right)_{k \in \mathbb{N}}$はその集合$A$上でその関数$f:A \rightarrow S$に広義一様収束する、compact一様収束するという。
\end{dfn}
\begin{thm}\label{4.1.11.9}
$A \subseteq R \subseteq \mathbb{R}^{m}$、$S \subseteq \mathbb{R}^{n}$なる有界関数空間$\mathfrak{B}(A,S)$の関数列$\left( f_{k} \right)_{k \in \mathbb{N}}$が与えられたとき、その集合$A$がその集合$R$での開集合、または、閉集合であり、$\forall k \in \mathbb{N}$に対し、その関数$f_{k}$がその集合$A$で連続であるかつ、その関数列$\left( f_{k} \right)_{k \in \mathbb{N}}$がその集合$A$上でその関数$f:A \rightarrow S$に広義一様収束するなら、その極限関数$f$はその集合$A$で連続である。
\end{thm}
\begin{proof}
$A \subseteq R \subseteq \mathbb{R}^{m}$、$S \subseteq \mathbb{R}^{n}$なる有界関数空間$\mathfrak{B}(A,S)$の関数列$\left( f_{k} \right)_{k \in \mathbb{N}}$が与えられたとする。その集合$A$がその集合$R$での開集合であり、$\forall k \in \mathbb{N}$に対し、その関数$f_{k}$がその集合$A$で連続であるかつ、その関数列$\left( f_{k} \right)_{k \in \mathbb{N}}$がその集合$A$上でその関数$f:A \rightarrow S$に広義一様収束するとき、$\forall\mathbf{x} \in A\exists\varepsilon \in \mathbb{R}^{+}$に対し、$U\left( \mathbf{x},\varepsilon \right) \cap R \subseteq A$が成り立つ。$K = \mathrm{cl}_{R}\left( U\left( \mathbf{x},\frac{\varepsilon}{2} \right) \cap R \right)$とおかれるとき、$\exists\mathbf{a} \in R$に対し、$\mathbf{a} \in K$が成り立つ、即ち、$\forall\delta \in \mathbb{R}^{+}$に対し、$U\left( \mathbf{a},\delta \right) \cap U\left( \mathbf{x},\frac{\varepsilon}{2} \right) \cap R \neq \emptyset$が成り立つかつ、$\mathbf{a} \notin U\left( \mathbf{x},\varepsilon \right) \cap R$が成り立つと仮定すると、$\exists\mathbf{b} \in R$に対し、$\left\| \mathbf{b} - \mathbf{a} \right\| < \frac{\varepsilon}{4}$かつ$\left\| \mathbf{b} - \mathbf{x} \right\| < \frac{\varepsilon}{2}$が成り立つかつ、$\varepsilon \leq \left\| \mathbf{x} - \mathbf{a} \right\|$が成り立つので、次のようになる。
\begin{align*}
\varepsilon &\leq \left\| \mathbf{x} - \mathbf{a} \right\|\\
&= \left\| \mathbf{x} - \mathbf{b} + \mathbf{b} - \mathbf{a} \right\|\\
&\leq \left\| \mathbf{x} - \mathbf{b} \right\| + \left\| \mathbf{b} - \mathbf{a} \right\|\\
&= \left\| \mathbf{b} - \mathbf{x} \right\| + \left\| \mathbf{b} - \mathbf{a} \right\|\\
&< \frac{\varepsilon}{2} + \frac{\varepsilon}{4} = \frac{3\varepsilon}{4}
\end{align*}
しかしながら、これは矛盾している。ゆえに、$\forall\mathbf{a} \in R$に対し、$\mathbf{a} \in K$が成り立つなら、$\mathbf{a} \in U\left( \mathbf{x},\varepsilon \right) \cap R$が成り立つので、次のようになる。
\begin{align*}
K = \mathrm{cl}_{R}\left( U\left( \mathbf{x},\frac{\varepsilon}{2} \right) \cap R \right) \subseteq U\left( \mathbf{x},\varepsilon \right) \cap R \subseteq A
\end{align*}
ここで、その集合$K$はその集合$R$で有界な閉集合であるので、定理\ref{4.1.7.7}、即ち、Heine-Borelの被覆定理よりその集合$K$はその集合$R$でcompactなその集合$A$の部分集合であることになる。仮定よりある関数$f:A \rightarrow S$が存在して、$\lim_{k \rightarrow \infty}\left\| \left( f_{k} - f \right)|K \right\|_{K,\infty} = 0$が成り立つ。特に、関数列$\left( f_{k}|K \right)_{k \in \mathbb{N}}$について考えれば、ある関数$f|K:K \rightarrow S$が存在して、$\lim_{k \rightarrow \infty}\left\| f_{k}|K - f|K \right\|_{K,\infty} = 0$が成り立つので、この関数列$\left( f_{k}|K \right)_{k \in \mathbb{N}}$はその集合$K$上でその関数$f|K:K \rightarrow S$に一様収束する。そこで、$\forall k \in \mathbb{N}$に対し、その関数$f_{k}|K$がその集合$K$で連続であることから、定理\ref{4.1.11.8}よりその極限関数$f|K$はその集合$K$で連続である。特に、その極限関数$f$はその点$\mathbf{x}$で連続であるので、よって、その極限関数$f$はその集合$A$で連続である。\par
その集合$A$がその集合$R$での閉集合であり、$\forall k \in \mathbb{N}$に対し、その関数$f_{k}$がその集合$A$で連続であるかつ、その関数列$\left( f_{k} \right)_{k \in \mathbb{N}}$がその集合$A$上でその関数$f:A \rightarrow S$に広義一様収束するなら、$\forall\mathbf{x} \in A\forall\varepsilon \in \mathbb{R}^{+}$に対し、$L = \mathrm{cl}_{R}\left( U\left( \mathbf{x},\varepsilon \right) \cap R \right) \cap A$とおかれるとき、その集合$L$はその集合$R$で閉集合であり、上と同じ議論により、$\mathrm{cl}_{R}\left( U\left( \mathbf{x},\varepsilon \right) \cap R \right) \subseteq U\left( \mathbf{x},2\varepsilon \right) \cap R$が成り立つので、その集合$L$は有界である。定理\ref{4.1.7.7}、即ち、Heine-Borelの被覆定理よりその集合$L$はその集合$R$でcompactなその集合$A$の部分集合であることになる。あとはその集合$A$が開集合のときと同様にして示される。実際、仮定よりある関数$f:A \rightarrow S$が存在して、$\lim_{k \rightarrow \infty}\left\| \left( f_{k} - f \right)|L \right\|_{L,\infty} = 0$が成り立つ。特に、関数列$\left( f_{k}|L \right)_{k \in \mathbb{N}}$について考えれば、ある関数$f|L:L \rightarrow S$が存在して、$\lim_{k \rightarrow \infty}\left\| f_{k}|L - f|L \right\|_{L,\infty} = 0$が成り立つので、この関数列$\left( f_{k}|L \right)_{k \in \mathbb{N}}$はその集合$L$上でその関数$f|L:L \rightarrow S$に一様収束する。そこで、$\forall k \in \mathbb{N}$に対し、その関数$f_{k}|L$がその集合$L$で連続であることから、定理\ref{4.1.11.8}よりその極限関数$f|L$はその集合$L$で連続である。特に、その極限関数$f$はその点$\mathbf{x}$で連続であるので、よって、その極限関数$f$はその集合$A$で連続である。
\end{proof}
%\hypertarget{ux95a2ux6570ux65cf}{%
\subsubsection{関数族}%\label{ux95a2ux6570ux65cf}}
\begin{dfn}
$\varLambda \subseteq \mathbb{R}^{l}$、$A \subseteq \mathbb{R}^{m}$、$B \subseteq \mathbb{R}^{n}$なる有界な関数$f:\varLambda \times A \rightarrow B$が与えられたとき、次のような写像$\left( f_{\lambda} \right)_{\lambda \in \varLambda}$を有界関数空間$\mathfrak{B}(A,B)$における関数族という。
\begin{align*}
\left( f_{\lambda} \right)_{\lambda \in \varLambda}:\varLambda \rightarrow \mathfrak{B}(A,B);\lambda \mapsto \left( f_{\lambda}:A \rightarrow B;\mathbf{x} \mapsto f\left( \lambda,\mathbf{x} \right) \right)
\end{align*}
\end{dfn}
\begin{dfn}
$\varLambda \subseteq T \subseteq \mathbb{R}^{l}$、$A \subseteq R \subseteq \mathbb{R}^{m}$、$S \subseteq \mathbb{R}^{n}$なる有界関数空間$\mathfrak{B}(A,S)$における関数族$\left( f_{\lambda} \right)_{\lambda \in \varLambda}$が与えられたとき、$\mathbf{b} \in \mathrm{cl}_{T}\varLambda$なる点$\mathbf{b}$について、$\forall\mathbf{x} \in A$に対し、極限値$\lim_{\scriptsize \begin{matrix} \lambda \rightarrow \mathbf{b} \\ T \rightarrow \mathbb{R} \end{matrix}}{f_{\lambda}\left( \mathbf{x} \right)}$がその集合$S$に存在するとき、$f\left( \mathbf{x} \right) = \lim_{\scriptsize \begin{matrix} \lambda \rightarrow \mathbf{b} \\ T \rightarrow \mathbb{R} \end{matrix}}{f_{\lambda}\left( \mathbf{x} \right)}$とおくと、この関数族$\left( f_{\lambda} \right)_{\lambda \in \varLambda}$がその集合$A$上で$\lambda \rightarrow \mathbf{b}$のときにその関数$f:A \rightarrow S$に各点収束するといい、その関数$f$をその関数族$\left( f_{\lambda} \right)_{\lambda \in \varLambda}$の極限関数といい$\lim_{\scriptsize \begin{matrix} \lambda \rightarrow \mathbf{b} \\ T \rightarrow \mathbb{R} \end{matrix}}f_{\lambda}$と書くことにする。
\end{dfn}
\begin{dfn}
$\varLambda \subseteq T \subseteq \mathbb{R}^{l}$、$A \subseteq R \subseteq \mathbb{R}^{m}$、$S \subseteq \mathbb{R}^{n}$なる有界関数空間$\mathfrak{B}(A,S)$における関数族$\left( f_{\lambda} \right)_{\lambda \in \varLambda}$が与えられたとき、$\mathbf{b} \in \mathrm{cl}_{T}\varLambda$なる点$\mathbf{b}$に対し、ある関数$f:A \rightarrow S$が存在して、$\lim_{\scriptsize \begin{matrix} \lambda \rightarrow \mathbf{b} \\ T \rightarrow \mathbb{R} \end{matrix}}\left\| f_{\lambda} - f \right\|_{A,\infty} = 0$が成り立つとき、この関数族$\left( f_{\lambda} \right)_{\lambda \in \varLambda}$はその集合$A$上で$\lambda \rightarrow \mathbf{b}$のときにその関数$f:A \rightarrow S$に一様収束するという。
\end{dfn}
\begin{thm}\label{4.1.11.10}
$\varLambda \subseteq T \subseteq \mathbb{R}^{l}$、$A \subseteq R \subseteq \mathbb{R}^{m}$、$S \subseteq \mathbb{R}^{n}$なる有界関数空間$\mathfrak{B}(A,S)$における関数族$\left( f_{\lambda} \right)_{\lambda \in \varLambda}$が与えられたとき、次のことは同値である。
\begin{itemize}
\item
  その関数族$\left( f_{\lambda} \right)_{\lambda \in \varLambda}$がその集合$A$上で$\lambda \rightarrow \mathbf{b}$のときにその関数$f:A \rightarrow S$に一様収束する。
\item
  $\forall\varepsilon \in \mathbb{R}^{+}\exists\delta \in \mathbb{R}^{+}\forall\lambda \in \varLambda$に対し、$\left\| \lambda - \mathbf{b} \right\| < \delta$が成り立つなら、$\left\| f_{\lambda} - f \right\|_{A,\infty} < \varepsilon$が成り立つ。
\item
  任意のその集合$\varLambda$の点列$\left( \lambda_{k} \right)_{k \in \mathbb{N}}$に対し、その集合$T$で$\lim_{k \rightarrow \infty}\lambda_{k} = \mathbf{b}$が成り立つなら、その関数列$\left( f_{\lambda_{k}} \right)_{k \in \mathbb{N}}$はその集合$A$上でその関数$f:A \rightarrow S$に一様収束する。
\end{itemize}
\end{thm}
\begin{proof}
$\varLambda \subseteq T \subseteq \mathbb{R}^{l}$、$A \subseteq R \subseteq \mathbb{R}^{m}$、$S \subseteq \mathbb{R}^{n}$なる有界関数空間$\mathfrak{B}(A,S)$における関数族$\left( f_{\lambda} \right)_{\lambda \in \varLambda}$が与えられたとき、次のことは同値であることは明らかである。
\begin{itemize}
\item
  その関数族$\left( f_{\lambda} \right)_{\lambda \in \varLambda}$がその集合$A$上で$\lambda \rightarrow \mathbf{b}$のときにその関数$f:A \rightarrow S$に一様収束する。
\item
  $\forall\varepsilon \in \mathbb{R}^{+}\exists\delta \in \mathbb{R}^{+}\forall\lambda \in \varLambda$に対し、$\left\| \lambda - \mathbf{b} \right\| < \delta$が成り立つなら、$\left\| f_{\lambda} - f \right\|_{A,\infty} < \varepsilon$が成り立つ。
\end{itemize}\par
また、その関数族$\left( f_{\lambda} \right)_{\lambda \in \varLambda}$がその集合$A$上で$\lambda \rightarrow \mathbf{b}$のときにその関数$f:A \rightarrow S$に一様収束するならそのときに限り、$\lim_{\scriptsize \begin{matrix} \lambda \rightarrow \mathbf{b} \\ T \rightarrow \mathbb{R} \end{matrix}}\left\| f_{\lambda} - f \right\|_{A,\infty} = 0$が成り立つ。そこで、関数$F:\varLambda \rightarrow \mathbb{R};\lambda \mapsto \left\| f_{\lambda} - f \right\|_{A,\infty}$が考えられれば、$F(\lambda) \rightarrow 0\ \left( \lambda \rightarrow \mathbf{b},\ \ T \rightarrow \mathbb{R} \right)$が成り立つ。そこで、定理\ref{4.1.10.2}よりこれが成り立つならそのときに限り、任意のその集合$\varLambda$の点列$\left( \lambda_{k} \right)_{k \in \mathbb{N}}$に対し、その集合$T$で$\lim_{k \rightarrow \infty}\lambda_{k} = \mathbf{b}$が成り立つなら、$\lim_{k \rightarrow \infty}{F\left( \lambda_{k} \right)} = 0$が成り立つ。そこで、次式が成り立つことから、
\begin{align*}
\lim_{k \rightarrow \infty}{F\left( \lambda_{k} \right)} = \lim_{k \rightarrow \infty}\left\| f_{\lambda_{k}} - f \right\|_{A,\infty} = 0
\end{align*}
これが成り立つならそのときに限り、その関数列$\left( f_{\lambda_{k}} \right)_{k \in \mathbb{N}}$はその集合$A$上でその関数$f:A \rightarrow S$に一様収束する。したがって、次のことは同値である。
\begin{itemize}
\item
  その関数族$\left( f_{\lambda} \right)_{\lambda \in \varLambda}$がその集合$A$上で$\lambda \rightarrow \mathbf{b}$のときにその関数$f:A \rightarrow S$に一様収束する。
\item
  任意のその集合$\varLambda$の点列$\left( \lambda_{k} \right)_{k \in \mathbb{N}}$に対し、その集合$T$で$\lim_{k \rightarrow \infty}\lambda_{k} = \mathbf{b}$が成り立つなら、その関数列$\left( f_{\lambda_{k}} \right)_{k \in \mathbb{N}}$はその集合$A$上でその関数$f:A \rightarrow S$に一様収束する。
\end{itemize}
\end{proof}
\begin{thm}\label{4.1.11.11}
$\varLambda \subseteq T \subseteq \mathbb{R}^{l}$、$A \subseteq R \subseteq \mathbb{R}^{m}$、$S \subseteq \mathbb{R}^{n}$なる有界関数空間$\mathfrak{B}(A,S)$における関数族$\left( f_{\lambda} \right)_{\lambda \in \varLambda}$が与えられたとき、その関数族$\left( f_{\lambda} \right)_{\lambda \in \varLambda}$がその集合$A$上で$\lambda \rightarrow \mathbf{b}$のときにその関数$f:A \rightarrow S$に一様収束するなら、その関数族$\left( f_{\lambda} \right)_{\lambda \in \varLambda}$はその集合$A$上で$\lambda \rightarrow \mathbf{b}$のときその極限関数$f:A \rightarrow S$に各点収束する。
\end{thm}
\begin{proof} 定理\ref{4.1.11.7}と定理\ref{4.1.11.10}より明らかである。
\end{proof}
\begin{thm}\label{4.1.11.12}
$\varLambda \subseteq T \subseteq \mathbb{R}^{l}$、$A \subseteq R \subseteq \mathbb{R}^{m}$、$S \subseteq \mathbb{R}^{n}$なる有界関数空間$\mathfrak{B}(A,S)$における関数族$\left( f_{\lambda} \right)_{\lambda \in \varLambda}$が与えられたとき、$\forall\lambda \in \varLambda$に対し、その関数$f_{\lambda}$がその集合$A$で連続であるかつ、その関数族$\left( f_{\lambda} \right)_{\lambda \in \varLambda}$がその集合$A$上で$\lambda \rightarrow \mathbf{b}$のときにその関数$f:A \rightarrow S$に一様収束するなら、その極限関数$f$はその集合$A$で連続である。
\end{thm}
\begin{proof} 定理\ref{4.1.11.8}と定理\ref{4.1.11.10}より明らかである。
\end{proof}
\begin{dfn}
$\varLambda \subseteq T \subseteq \mathbb{R}^{l}$、$A \subseteq R \subseteq \mathbb{R}^{m}$、$S \subseteq \mathbb{R}^{n}$なる有界関数空間$\mathfrak{B}(A,S)$における関数族$\left( f_{\lambda} \right)_{\lambda \in \varLambda}$が与えられたとする。$\mathbf{b} \in \mathrm{cl}_{T}\varLambda$なる点$\mathbf{b}$に対し、$\forall K \in \mathfrak{P}(A)$に対し、その集合$K$がその集合$R$でcompactであるなら、ある関数$f:A \rightarrow S$が存在して、$\lim_{\scriptsize \begin{matrix} \lambda \rightarrow \mathbf{b} \\ T \rightarrow \mathbb{R} \end{matrix}}\left\| \left( f_{\lambda} - f \right)|K \right\|_{K,\infty} = 0$が成り立つとき、この関数族$\left( f_{\lambda} \right)_{\lambda \in \varLambda}$はその集合$A$上で$\lambda \rightarrow \mathbf{b}$のときにその関数$f:A \rightarrow S$に広義一様収束する、compact一様収束するという。
\end{dfn}
\begin{thm}\label{4.1.11.13}
$\varLambda \subseteq T \subseteq \mathbb{R}^{l}$、$A \subseteq R \subseteq \mathbb{R}^{m}$、$S \subseteq \mathbb{R}^{n}$なる有界関数空間$\mathfrak{B}(A,S)$における関数族$\left( f_{\lambda} \right)_{\lambda \in \varLambda}$が与えられたとき、その集合$A$がその集合$R$での開集合、または、閉集合であり、$\forall\lambda \in \varLambda$に対し、その関数$f_{\lambda}$がその集合$A$で連続であるかつ、その関数族$\left( f_{\lambda} \right)_{\lambda \in \varLambda}$がその集合$A$上で$\lambda \rightarrow \mathbf{b}$のときにその関数$f:A \rightarrow S$に広義一様収束するなら、その極限関数$f$はその集合$A$で連続である。
\end{thm}
\begin{proof} 定理\ref{4.1.11.9}と定理\ref{4.1.11.10}より明らかである。
\end{proof}
%\hypertarget{cauchyux306eux4e00ux69d8ux53ceux675fux6761ux4ef6}{%
\subsubsection{Cauchyの一様収束条件}%\label{cauchyux306eux4e00ux69d8ux53ceux675fux6761ux4ef6}}
\begin{thm}[関数列に関するCauchyの一様収束条件]\label{4.1.11.14}
$A \subseteq R \subseteq \mathbb{R}^{m}$なる有界関数空間$\mathfrak{B}\left( A,\mathbb{R}^{n} \right)$の関数列$\left( f_{k} \right)_{k \in \mathbb{N}}$が与えられたとき、次のことは同値である。
\begin{itemize}
\item
  その関数列$\left( f_{k} \right)_{k \in \mathbb{N}}$がその集合$A$上でその関数$f:A \rightarrow \mathbb{R}^{n}$に一様収束する。
\item
  $\forall\varepsilon \in \mathbb{R}^{+}\exists N \in \mathbb{N}\forall k,l \in \mathbb{N}$に対し、$N \leq k$かつ$N \leq l$が成り立つなら、$\left\| f_{k} - f_{l} \right\|_{A,\infty} < \varepsilon$が成り立つ。
\end{itemize}
この定理を関数列に関するCauchyの一様収束条件という。
\end{thm}
\begin{proof}
$A \subseteq R \subseteq \mathbb{R}^{m}$なる有界関数空間$\mathfrak{B}\left( A,\mathbb{R}^{n} \right)$の関数列$\left( f_{k} \right)_{k \in \mathbb{N}}$が与えられたとき、その関数列$\left( f_{k} \right)_{k \in \mathbb{N}}$がその集合$A$上でその関数$f:A \rightarrow \mathbb{R}^{n}$に一様収束するならそのときに限り、$\forall\varepsilon \in \mathbb{R}^{+}\exists N \in \mathbb{N}\forall k \in \mathbb{N}$に対し、$N \leq k$が成り立つなら、$\left\| f_{k} - f \right\|_{A,\infty} < \varepsilon$が成り立つことになるので、$\forall\varepsilon \in \mathbb{R}^{+}\exists N \in \mathbb{N}\forall k,l \in \mathbb{N}$に対し、$N \leq k$かつ$N \leq l$が成り立つなら、三角不等式より次のようになる。
\begin{align*}
\left\| f_{k} - f_{l} \right\|_{A,\infty} &= \left\| f_{k} - f + f - f_{l} \right\|_{A,\infty}\\
&\leq \left\| f_{k} - f \right\|_{A,\infty} + \left\| f - f_{l} \right\|_{A,\infty}\\
&= \left\| f_{k} - f \right\|_{A,\infty} + \left\| f_{l} - f \right\|_{A,\infty} < 2\varepsilon
\end{align*}\par
逆に、$\forall\varepsilon \in \mathbb{R}^{+}\exists N \in \mathbb{N}\forall k,l \in \mathbb{N}$に対し、$N \leq k$かつ$N \leq l$が成り立つなら、$\left\| f_{k} - f_{l} \right\|_{A,\infty} < \varepsilon$が成り立つとき、$\forall\mathbf{x} \in A$に対し、$\left\| f_{k}\left( \mathbf{x} \right) - f_{l}\left( \mathbf{x} \right) \right\| \leq \sup_{\mathbf{x} \in A}\left\| f_{k}\left( \mathbf{x} \right) - f_{l}\left( \mathbf{x} \right) \right\| = \left\| f_{k} - f_{l} \right\|_{A,\infty} < \varepsilon$が成り立つので、その点列$\left( f_{k}\left( \mathbf{x} \right) \right)_{k \in \mathbb{N}}$はCauchy列である。Cauchyの収束条件よりその点列$\left( f_{k}\left( \mathbf{x} \right) \right)_{k \in \mathbb{N}}$は収束しその極限値$f\left( \mathbf{x} \right)$が$n$次元数空間$\mathbb{R}^{n}$に存在する。このとき、$l \rightarrow \infty$とすれば、$\forall\varepsilon \in \mathbb{R}^{+}\exists N \in \mathbb{N}\forall k \in \mathbb{N}$に対し、$N \leq k$が成り立つなら、$\left\| f_{k}\left( \mathbf{x} \right) - f\left( \mathbf{x} \right) \right\| \leq \varepsilon$が成り立つ。これにより、$\left\| f_{k} - f \right\|_{A,\infty} \leq \varepsilon$が得られるので、その関数列$\left( f_{k} \right)_{k \in \mathbb{N}}$はその集合$A$上でその関数$f:A \rightarrow \mathbb{R}^{n}$に一様収束する。
\end{proof}
\begin{thm}[関数族に関するCauchyの一様収束条件]\label{4.1.11.15}
$\varLambda \subseteq T \subseteq \mathbb{R}^{l}$、$A \subseteq R \subseteq \mathbb{R}^{m}$なる有界関数空間$\mathfrak{B}\left( A,\mathbb{R}^{n} \right)$における関数族$\left( f_{\lambda} \right)_{\lambda \in \varLambda}$が与えられたとき、次のことは同値である。
\begin{itemize}
\item
  その関数族$\left( f_{\lambda} \right)_{\lambda \in \varLambda}$がその集合$A$上で$\lambda \rightarrow \mathbf{b}$のときにその関数$f:A \rightarrow \mathbb{R}^{n}$に一様収束する。
\item
  $\forall\varepsilon \in \mathbb{R}^{+}\exists\delta \in \mathbb{R}^{+}\forall\kappa,\lambda \in \varLambda$に対し、$\kappa,\lambda \in U\left( \mathbf{b},\delta \right) \cap T$が成り立つなら、$\left\| f_{\kappa} - f_{\lambda} \right\|_{A,\infty} < \varepsilon$が成り立つ。
\end{itemize}
この定理を関数族に関するCauchyの一様収束条件という。
\end{thm}
\begin{proof}
$\varLambda \subseteq T \subseteq \mathbb{R}^{l}$、$A \subseteq R \subseteq \mathbb{R}^{m}$なる有界関数空間$\mathfrak{B}\left( A,\mathbb{R}^{n} \right)$における関数族$\left( f_{\lambda} \right)_{\lambda \in \varLambda}$が与えられたとき、その関数族$\left( f_{\lambda} \right)_{\lambda \in \varLambda}$がその集合$A$上で$\lambda \rightarrow \mathbf{b}$のときにその関数$f:A \rightarrow \mathbb{R}^{n}$に一様収束するならそのときに限り、$\forall\varepsilon \in \mathbb{R}^{+}\exists\delta \in \mathbb{R}^{+}\forall\lambda \in \varLambda$に対し、$\lambda \in U\left( \mathbf{b},\delta \right) \cap T$が成り立つなら、$\left\| f_{\lambda} - f \right\|_{A,\infty} < \varepsilon$が成り立つことになるので、$\forall\varepsilon \in \mathbb{R}^{+}\exists\delta \in \mathbb{R}^{+}\forall\kappa,\lambda \in \varLambda$に対し、$\kappa,\lambda \in U\left( \mathbf{b},\delta \right) \cap T$が成り立つなら、三角不等式より次のようになる。
\begin{align*}
\left\| f_{\kappa} - f_{\lambda} \right\|_{A,\infty} &= \left\| f_{\kappa} - f + f - f_{\lambda} \right\|_{A,\infty}\\
&\leq \left\| f_{\kappa} - f \right\|_{A,\infty} + \left\| f - f_{\lambda} \right\|_{A,\infty}\\
&= \left\| f_{\kappa} - f \right\|_{A,\infty} + \left\| f_{\lambda} - f \right\|_{A,\infty} < 2\varepsilon
\end{align*}\par
逆に、$\forall\varepsilon \in \mathbb{R}^{+}\exists\delta \in \mathbb{R}^{+}\forall\kappa,\lambda \in \varLambda$に対し、$\kappa,\lambda \in U\left( \mathbf{b},\delta \right) \cap T$が成り立つなら、$\left\| f_{\kappa} - f_{\lambda} \right\|_{A,\infty} < \varepsilon$が成り立つとき、$\forall\mathbf{x} \in A$に対し、$\left\| f_{\kappa}\left( \mathbf{x} \right) - f_{\lambda}\left( \mathbf{x} \right) \right\| \leq \sup_{\mathbf{x} \in A}\left\| f_{\kappa}\left( \mathbf{x} \right) - f_{\lambda}\left( \mathbf{x} \right) \right\| = \left\| f_{\kappa} - f_{\lambda} \right\|_{A,\infty} < \varepsilon$が成り立つので、定理\ref{4.1.10.14}、即ち、関数の極限に関するCauchyの収束条件よりその極限値$\lim_{\scriptsize \begin{matrix} \lambda \rightarrow \mathbf{b} \\ T \rightarrow \mathbb{R} \end{matrix}}{f_{\lambda}\left( \mathbf{x} \right)}$がその$n$次元数空間$\mathbb{R}^{n}$に存在する。このとき、$\lambda \rightarrow \mathbf{b}$とすれば、$\forall\varepsilon \in \mathbb{R}^{+}\exists\delta \in \mathbb{R}^{+}\forall\kappa \in \varLambda$に対し、$\kappa \in U\left( \mathbf{b},\delta \right) \cap T$が成り立つなら、$\left\| f_{\kappa}\left( \mathbf{x} \right) - f\left( \mathbf{x} \right) \right\| \leq \varepsilon$が成り立つ。これにより、$\left\| f_{\kappa} - f \right\|_{A,\infty} \leq \varepsilon$が得られるので、その関数族$\left( f_{\lambda} \right)_{\lambda \in \varLambda}$がその集合$A$上で$\lambda \rightarrow \mathbf{b}$のときにその関数$f:A \rightarrow \mathbb{R}^{n}$に一様収束する。
\end{proof}
%\hypertarget{ux512aux7d1aux6570ux5b9aux7406}{%
\subsubsection{優級数定理}%\label{ux512aux7d1aux6570ux5b9aux7406}}
\begin{thm}[優級数定理]\label{4.1.11.16}
$A \subseteq R \subseteq \mathbb{R}^{m}$なる関数空間$\mathfrak{F}\left( A,\mathbb{R}^{n} \right)$の関数列$\left( f_{k} \right)_{k \in \mathbb{N}}$が与えられたとする。その級数$\left( \sum_{i \in \varLambda_{k}} f_{i} \right)_{k \in \mathbb{N}}$が次の条件たちいづれも満たすとき、
\begin{itemize}
\item
  $\forall k \in \mathbb{N}$に対し、その関数$f_{k}$はその集合$A$で連続である。
\item
  $\forall k \in \mathbb{N}\exists M_{k} \in \mathbb{R}^{+} \cup \left\{ 0 \right\}$に対し、$\left\| f_{k} \right\| \leq M_{k}$が成り立つ\footnote{もちろん、$\forall\mathbf{x} \in A\forall k \in \mathbb{N}\exists M_{k} \in \mathbb{R}^{+} \cup \left\{ 0 \right\}$に対し、$\left\| f_{k}\left( \mathbf{x} \right) \right\| \leq M_{k}$が成り立つという意味である。}。
\item
  その級数$\left( \sum_{i \in \varLambda_{k}} M_{i} \right)_{k \in \mathbb{N}}$が収束する。
\end{itemize}
その級数$\left( \sum_{i \in \varLambda_{k}} f_{i} \right)_{k \in \mathbb{N}}$が絶対収束しその極限関数$\sum_{k \in \mathbb{N}} f_{k}$はその集合$A$で連続である。この定理を優級数定理という。\par
なお、この定理におけるその級数$\left( \sum_{i \in \varLambda_{k}} M_{i} \right)_{k \in \mathbb{N}}$をその級数$\left( \sum_{i \in \varLambda_{k}} f_{i} \right)_{k \in \mathbb{N}}$の優級数という。
\end{thm}
\begin{proof}
$A \subseteq R \subseteq \mathbb{R}^{m}$なる関数空間$\mathfrak{F}\left( A,\mathbb{R}^{n} \right)$の関数列$\left( f_{k} \right)_{k \in \mathbb{N}}$が与えられたとする。その級数$\left( \sum_{i \in \varLambda_{k}} f_{i} \right)_{k \in \mathbb{N}}$が次の条件たちいづれも満たすとき、
\begin{itemize}
\item
  $\forall k \in \mathbb{N}$に対し、その関数$f_{k}$はその集合$A$で連続である。
\item
  $\forall k \in \mathbb{N}\exists M_{k} \in \mathbb{R}^{+} \cup \left\{ 0 \right\}$に対し、$\left\| f_{k} \right\| \leq M_{k}$が成り立つ。
\item
  その級数$\left( \sum_{i \in \varLambda_{k}} M_{i} \right)_{k \in \mathbb{N}}$が収束する。
\end{itemize}
比較定理より、その級数$\left( \sum_{i \in \varLambda_{k}} M_{i} \right)_{k \in \mathbb{N}}$が収束するかつ、$\forall k \in \mathbb{N}$に対し、$\left\| f_{k} \right\| \leq M_{k}$が成り立つなら、その級数$\left( \sum_{i \in \varLambda_{k}} \left\| f_{i} \right\| \right)_{k \in \mathbb{N}}$も収束する。したがって、定理\ref{4.1.11.5}よりその級数$\left( \sum_{i \in \varLambda_{k}} f_{i} \right)_{k \in \mathbb{N}}$も絶対収束する。\par
ここで、仮定より級数$\left( \sum_{i \in \varLambda_{k}} M_{i} \right)_{k \in \mathbb{N}}$が収束することから、極限値$\lim_{k \rightarrow \infty}{\sum_{i \in \varLambda_{k}} M_{i}} = \sum_{k \in \mathbb{N}} M_{k}$が存在する。これが$S$とおかれると、$\forall\varepsilon \in \mathbb{R}^{+}\exists p \in \mathbb{N}\forall k \in \mathbb{N}$に対し、$p \leq k$が成り立つなら、$\left| \sum_{i \in \varLambda_{k}} M_{i} - S \right| < \varepsilon$が成り立つ。ここで、仮定よりそれらの定数たち$M_{k}$は負でない実数であるから、次のようになる。
\begin{align*}
\left| \sum_{i \in \varLambda_{k}} M_{i} - S \right| &= \left| \sum_{i \in \varLambda_{k}} M_{i} - \sum_{i \in \mathbb{N}} M_{i} \right|\\
&= \left| \sum_{i \in \varLambda_{k}} M_{i} - \left( \sum_{i \in \varLambda_{k}} M_{i} + \sum_{i \in \mathbb{N} \setminus \varLambda_{k}} M_{i} \right) \right|\\
&= \left| \sum_{i \in \varLambda_{k}} M_{i} - \sum_{i \in \varLambda_{k}} M_{i} - \sum_{i \in \mathbb{N} \setminus \varLambda_{k}} M_{i} \right|\\
&= \left| - \sum_{i \in \mathbb{N} \setminus \varLambda_{k}} M_{i} \right| = \sum_{i \in \mathbb{N} \setminus \varLambda_{k}} M_{i}
\end{align*}
また、関数$\left\| \sum_{i \in \mathbb{N}} f_{i} - \sum_{i \in \varLambda_{k}} f_{i} \right\|$において三角不等式より次のようになり、
\begin{align*}
\left\| \sum_{i \in \mathbb{N}} f_{i} - \sum_{i \in \varLambda_{k}} f_{i} \right\| &= \left\| \sum_{i \in \varLambda_{k}} f_{i} + \sum_{i \in \mathbb{N} \setminus \varLambda_{k}} f_{i} - \sum_{i \in \varLambda_{k}} f_{i} \right\|\\
&= \left\| \sum_{i \in \mathbb{N} \setminus \varLambda_{k}} f_{i} \right\| \leq \sum_{i \in \mathbb{N} \setminus \varLambda_{k}} \left\| f_{i} \right\|
\end{align*}
仮定より$\forall i \in \mathbb{N}$に対し、$\left\| f_{i} \right\| \leq M_{i}$が成り立つのであったので、$\sum_{i \in \mathbb{N} \setminus \varLambda_{k}} \left\| f_{i} \right\| \leq \sum_{i \in \mathbb{N} \setminus \varLambda_{k}} M_{i}$が成り立つ。ここで、$\forall\varepsilon \in \mathbb{R}^{+}\exists p \in \mathbb{N}\forall k \in \mathbb{N}$に対し、$p < k$が成り立つなら、$\sum_{i \in \mathbb{N} \setminus \varLambda_{k}} M_{i} < \varepsilon$が成り立つことによりしたがって、次式が成り立つ。
\begin{align*}
\left\| \sum_{i \in \mathbb{N}} f_{i} - \sum_{i \in \varLambda_{k}} f_{i} \right\| = \left\| \sum_{i \in \mathbb{N} \setminus \varLambda_{k}} f_{i} \right\| \leq \sum_{i \in \mathbb{N} \setminus \varLambda_{k}} \left\| f_{i} \right\| \leq \sum_{i \in \mathbb{N} \setminus \varLambda_{k}} M_{i} < \varepsilon
\end{align*}
したがって、$k = p$とすれば、$\forall\varepsilon \in \mathbb{R}^{+}\exists p \in \mathbb{N}$に対し、$\left\| \sum_{i \in \mathbb{N}} f_{i} - \sum_{i \in \varLambda_{p}} f_{i} \right\| < \varepsilon$が成り立つ。\par
また、$\forall k \in \mathbb{N}$に対し、それらの関数たち$f_{k}$はその集合$A$で連続であったので、$\forall k \in \mathbb{N}\forall\mathbf{a} \in A$に対し、$\lim_{\scriptsize \begin{matrix} \mathbf{x} \rightarrow \mathbf{a} \\ R \rightarrow S \end{matrix}}{f_{k}\left( \mathbf{x} \right)} = f_{k}\left( \mathbf{a} \right)$が成り立つ。したがって、$\forall\mathbf{a} \in A$に対し、次式が成り立つ。
\begin{align*}
\lim_{\scriptsize \begin{matrix} \mathbf{x} \rightarrow \mathbf{a} \\ R \rightarrow S \end{matrix}}{\sum_{i \in \varLambda_{p}} {f_{i}\left( \mathbf{x} \right)}} = \sum_{i \in \varLambda_{p}} {\lim_{\scriptsize \begin{matrix} \mathbf{x} \rightarrow \mathbf{a} \\ R \rightarrow S \end{matrix}}{f_{i}\left( \mathbf{x} \right)}} = \sum_{i \in \varLambda_{p}} {f_{i}\left( \mathbf{a} \right)}
\end{align*}
これは関数$\sum_{i \in \varLambda_{p}} f_{i}$がその集合$A$で連続であるということを表している。したがって、$\forall\mathbf{a} \in A\forall\varepsilon \in \mathbb{R}^{+}\exists\delta \in \mathbb{R}^{+}\forall\mathbf{x} \in A$に対し、$\mathbf{x} \in U\left( \mathbf{a},\delta \right) \cap R$、即ち、$\left\| \mathbf{x} - \mathbf{a} \right\| < \delta$が成り立つなら、次式が成り立つ。
\begin{align*}
\left\| \sum_{i \in \varLambda_{p}} {f_{i}\left( \mathbf{x} \right)} - \sum_{i \in \varLambda_{p}} {f_{i}\left( \mathbf{a} \right)} \right\| < \varepsilon
\end{align*}\par
以上より、$\forall\mathbf{a} \in A\forall\varepsilon \in \mathbb{R}^{+}\exists p \in \mathbb{N}\forall\mathbf{x} \in A$に対し、$\left\| \mathbf{x} - \mathbf{a} \right\| < \delta$が成り立つなら、次式が成り立つので、
\begin{align*}
\left\| \sum_{i \in \mathbb{N}} {f_{i}\left( \mathbf{x} \right)} - \sum_{i \in \varLambda_{p}} {f_{i}\left( \mathbf{x} \right)} \right\| < \varepsilon,\ \ \left\| \sum_{i \in \mathbb{N}} {f_{i}\left( \mathbf{a} \right)} - \sum_{i \in \varLambda_{p}} {f_{i}\left( \mathbf{a} \right)} \right\| < \varepsilon,\ \ \left\| \sum_{i \in \varLambda_{p}} {f_{i}\left( \mathbf{x} \right)} - \sum_{i \in \varLambda_{p}} {f_{i}\left( \mathbf{a} \right)} \right\| < \varepsilon
\end{align*}
三角不等式より次のようになる。
\begin{align*}
\left\| \sum_{k \in \mathbb{N}} {f_{k}\left( \mathbf{x} \right)} - \sum_{k \in \mathbb{N}} {f_{k}\left( \mathbf{a} \right)} \right\| &= \left\| \sum_{i \in \mathbb{N}} {f_{i}\left( \mathbf{x} \right)} - \sum_{i \in \mathbb{N}} {f_{i}\left( \mathbf{a} \right)} \right\|\\
&= \left\| \sum_{i \in \mathbb{N}} {f_{i}\left( \mathbf{x} \right)} - \sum_{i \in \varLambda_{p}} {f_{i}\left( \mathbf{x} \right)} + \sum_{i \in \varLambda_{p}} {f_{i}\left( \mathbf{x} \right)} - \sum_{i \in \varLambda_{p}} {f_{i}\left( \mathbf{a} \right)} + \sum_{i \in \varLambda_{p}} {f_{i}\left( \mathbf{a} \right)} - \sum_{i \in \mathbb{N}} {f_{i}\left( \mathbf{a} \right)} \right\|\\
&\leq \left\| \sum_{i \in \mathbb{N}} {f_{i}\left( \mathbf{x} \right)} - \sum_{i \in \varLambda_{p}} {f_{i}\left( \mathbf{x} \right)} \right\| + \left\| \sum_{i \in \mathbb{N}} {f_{i}\left( \mathbf{a} \right)} - \sum_{i \in \varLambda_{p}} {f_{i}\left( \mathbf{a} \right)} \right\| + \left\| \sum_{i \in \varLambda_{p}} {f_{i}\left( \mathbf{x} \right)} - \sum_{i \in \varLambda_{p}} {f_{i}\left( \mathbf{a} \right)} \right\| \\
&< 3\varepsilon
\end{align*}
よって、その極限関数$\sum_{i \in \mathbb{N}} f_{i}$はその集合$A$で連続である。
\end{proof}
\begin{thm}[Weierstrassの$M$判定法]\label{4.1.11.17}
$A \subseteq R \subseteq \mathbb{R}^{m}$なる有界関数空間$\mathfrak{B}\left( A,\mathbb{R}^{n} \right)$の関数列$\left( f_{k} \right)_{k \in \mathbb{N}}$が与えられたとする。次の条件たちいづれも満たされるとき、
\begin{itemize}
\item
  $\forall k \in \mathbb{N}\exists M_{k} \in \mathbb{R}^{+} \cup \left\{ 0 \right\}$に対し、$\left\| f_{k} \right\|_{A,\infty} \leq M_{k}$が成り立つ。
\item
  その級数$\left( \sum_{i \in \varLambda_{k}} M_{i} \right)_{k \in \mathbb{N}}$が収束する。
\end{itemize}
その級数$\left( \sum_{i \in \varLambda_{k}} f_{i} \right)_{k \in \mathbb{N}}$はその極限関数$\sum_{k \in \mathbb{N}} f_{k}$に絶対収束するかつ一様収束する。この定理をWeierstrassの$M$判定法という。\par
この定理におけるその級数$\left( \sum_{i \in \varLambda_{k}} M_{i} \right)_{k \in \mathbb{N}}$もその級数$\left( \sum_{i \in \varLambda_{k}} f_{i} \right)_{k \in \mathbb{N}}$の優級数という。
\end{thm}
\begin{proof}
$A \subseteq R \subseteq \mathbb{R}^{m}$なる有界関数空間$\mathfrak{B}\left( A,\mathbb{R}^{n} \right)$の関数列$\left( f_{k} \right)_{k \in \mathbb{N}}$が与えられたとする。次の条件たちいづれも満たされるとする。
\begin{itemize}
\item
  $\forall k \in \mathbb{N}\exists M_{k} \in \mathbb{R}^{+} \cup \left\{ 0 \right\}$に対し、$\left\| f_{k} \right\|_{A,\infty} \leq M_{k}$が成り立つ。
\item
  その級数$\left( \sum_{i \in \varLambda_{k}} M_{i} \right)_{k \in \mathbb{N}}$が収束する。
\end{itemize}
$\forall\varepsilon \in \mathbb{R}^{+}\exists N \in \mathbb{N}\forall l,m \in \mathbb{N}$に対し、$N \leq l$かつ$N \leq m$が成り立つとき、$l < m$としても一般性は失われないのでそうすると、$\forall\mathbf{x} \in A$に対し、定理\ref{4.1.8.4}、即ち、級数に関するCauchyの収束条件より次のようになる。
\begin{align*}
\left| \sum_{i \in \varLambda_{m} \setminus \varLambda_{l}} \left\| f_{i}\left( \mathbf{x} \right) \right\| \right| &\leq \sum_{i \in \varLambda_{m} \setminus \varLambda_{l}} \left\| f_{i}\left( \mathbf{x} \right) \right\|\\
&\leq \sum_{i \in \varLambda_{m} \setminus \varLambda_{l}} {\sup_{\mathbf{x} \in A}\left\| f_{i}\left( \mathbf{x} \right) \right\|}\\
&= \sum_{i \in \varLambda_{m} \setminus \varLambda_{l}} \left\| f_{i} \right\|_{A,\infty}\\
&\leq \sum_{i \in \varLambda_{m} \setminus \varLambda_{l}} M_{i}\\
&= \left| \sum_{i \in \varLambda_{m} \setminus \varLambda_{l}} M_{i} \right| < \varepsilon
\end{align*}
定理\ref{4.1.8.4}、即ち、級数に関するCauchyの収束条件より$\forall\mathbf{x} \in A$に対し、その級数$\left( \sum_{i \in \varLambda_{k}} {f_{i}\left( \mathbf{x} \right)} \right)_{k \in \mathbb{N}}$は絶対収束する、即ち、その級数$\left( \sum_{i \in \varLambda_{k}} f_{i} \right)_{k \in \mathbb{N}}$は絶対収束する。定理\ref{4.1.11.5}よりその級数$\left( \sum_{i \in \varLambda_{k}} {f_{i}\left( \mathbf{x} \right)} \right)_{k \in \mathbb{N}}$は収束するので、その極限関数が$\sum_{k \in \mathbb{N}} f_{k}$とおかれることができる。\par
また、定理\ref{4.1.8.4}、即ち、級数に関するCauchyの収束条件より次のようになり、
\begin{align*}
\left\| \sum_{i \in \varLambda_{m}} f_{i} - \sum_{i \in \varLambda_{l}} f_{i} \right\|_{A,\infty} &= \left\| \sum_{i \in \varLambda_{m} \setminus \varLambda_{l}} f_{i} \right\|_{A,\infty}\\
&\leq \sum_{i \in \varLambda_{m} \setminus \varLambda_{l}} \left\| f_{i} \right\|_{A,\infty}\\
&\leq \sum_{i \in \varLambda_{m} \setminus \varLambda_{l}} M_{i}\\
&= \left| \sum_{i \in \varLambda_{m} \setminus \varLambda_{l}} M_{i} \right| < \varepsilon
\end{align*}
定理\ref{4.1.11.7}と定理\ref{4.1.11.14}、即ち、関数列に関するCauchyの一様収束条件よりよって、その級数$\left( \sum_{i \in \varLambda_{k}} f_{i} \right)_{k \in \mathbb{N}}$はその極限関数$\sum_{k \in \mathbb{N}} f_{k}$に一様収束する。
\end{proof}
%\hypertarget{ux6975ux9650ux306eux9806ux5e8fux4ea4ux63db}{%
\subsubsection{極限の順序交換}%\label{ux6975ux9650ux306eux9806ux5e8fux4ea4ux63db}}
\begin{thm}\label{4.1.11.18}
$A \subseteq R \subseteq \mathbb{R}^{l}$、$B \subseteq S \subseteq \mathbb{R}^{m}$、$T \subseteq \mathbb{R}^{n}$なる関数$f:A \times B \rightarrow T$が与えられたとき、次のことが満たされるなら、
\begin{itemize}
\item
  その関数族$\left( A \rightarrow T;\mathbf{x} \mapsto f\begin{pmatrix}
  \mathbf{x} \\
  \mathbf{y} \\
  \end{pmatrix} \right)_{\mathbf{y} \in B}$がその集合$A$上で$\mathbf{y} \rightarrow \mathbf{b}$のときにある極限関数$X:A \rightarrow T$に一様収束する。
\item
  その関数族$\left( B \rightarrow T;\mathbf{y} \mapsto f\begin{pmatrix}
  \mathbf{x} \\
  \mathbf{y} \\
  \end{pmatrix} \right)_{\mathbf{x} \in A}$がその集合$B$上で$\mathbf{x} \rightarrow \mathbf{a}$のときにある極限関数$Y:B \rightarrow T$に各点収束する。
\item
  極限値$\lim_{\mathbf{x} \rightarrow \mathbf{a} ,\  R \rightarrow T }{X\left( \mathbf{x} \right)}$がその集合$T$に存在する。
\end{itemize}
次式が成り立つ\footnote{仮定が満たされてなく成り立たない例として、次のような関数$f$が挙げられる。
\begin{align*}
  f:\mathbb{R}^+ \times \mathbb{R}^+ \rightarrow \mathbb{R} ;\begin{pmatrix}
    x \\ y
  \end{pmatrix} \mapsto \frac{x}{x+y}
\end{align*}
実際、次のようになる。
\begin{align*}
  \lim_{y\rightarrow \infty } \lim_{x\rightarrow \infty } f\begin{pmatrix}
    x \\ y
  \end{pmatrix} &= \lim_{y\rightarrow \infty } \lim_{x\rightarrow \infty } \frac{x}{x+y} = \lim_{y\rightarrow \infty } \lim_{x\rightarrow \infty } \frac{1}{1+\frac{y}{x}} = \lim_{y\rightarrow \infty } 1 =1 \\
  \lim_{x\rightarrow \infty } \lim_{y\rightarrow \infty } f\begin{pmatrix}
    x \\ y
  \end{pmatrix} &= \lim_{x\rightarrow \infty } \lim_{y\rightarrow \infty } \frac{x}{x+y} = \lim_{x\rightarrow \infty } \lim_{y\rightarrow \infty } \frac{\frac{x}{y}}{\frac{x}{y}+1} = \lim_{x\rightarrow \infty } 0 =0 \\
\end{align*}}。
\begin{align*}
\lim_{\scriptsize \begin{matrix} \begin{pmatrix}
\mathbf{x} \\
\mathbf{y} \\
\end{pmatrix} \rightarrow \begin{pmatrix}
\mathbf{a} \\
\mathbf{b} \\
\end{pmatrix} \\ R \times S \rightarrow T \end{matrix}}{f\begin{pmatrix}
\mathbf{x} \\
\mathbf{y} \\
\end{pmatrix}} = \lim_{\scriptsize \begin{matrix} \mathbf{x} \rightarrow \mathbf{a} \\ R \rightarrow T \end{matrix}}{\lim_{\scriptsize \begin{matrix} \mathbf{y} \rightarrow \mathbf{b} \\ S \rightarrow T \end{matrix}}{f\begin{pmatrix}
\mathbf{x} \\
\mathbf{y} \\
\end{pmatrix}}} = \lim_{\scriptsize \begin{matrix} \mathbf{y} \rightarrow \mathbf{b} \\ S \rightarrow T \end{matrix}}{\lim_{\scriptsize \begin{matrix} \mathbf{x} \rightarrow \mathbf{a} \\ R \rightarrow T \end{matrix}}{f\begin{pmatrix}
\mathbf{x} \\
\mathbf{y} \\
\end{pmatrix}}}
\end{align*}
\end{thm}
\begin{proof}
$A \subseteq R \subseteq \mathbb{R}^{l}$、$B \subseteq S \subseteq \mathbb{R}^{m}$、$T \subseteq \mathbb{R}^{n}$なる関数$f:A \times B \rightarrow T$が与えられたとき、次のことが満たされるとする。
\begin{itemize}
\item
  その関数族$\left( A \rightarrow T;\mathbf{x} \mapsto f\begin{pmatrix}
  \mathbf{x} \\
  \mathbf{y} \\
  \end{pmatrix} \right)_{\mathbf{y} \in B}$がその集合$A$上で$\mathbf{y} \rightarrow \mathbf{b}$のときにある極限関数$X:A \rightarrow T$に一様収束する。
\item
  その関数族$\left( B \rightarrow T;\mathbf{y} \mapsto f\begin{pmatrix}
  \mathbf{x} \\
  \mathbf{y} \\
  \end{pmatrix} \right)_{\mathbf{x} \in A}$がその集合$B$上で$\mathbf{x} \rightarrow \mathbf{a}$のときにある極限関数$Y:B \rightarrow T$に各点収束する。
\item
  極限値$\lim_{\mathbf{x} \rightarrow \mathbf{a} ,\  R \rightarrow T }{X\left( \mathbf{x} \right)}$がその集合$T$に存在する。
\end{itemize}
そこで、関数たち$A \rightarrow T;\mathbf{x} \mapsto f\begin{pmatrix}
\mathbf{x} \\
\mathbf{y} \\
\end{pmatrix}$、$B \rightarrow T;\mathbf{y} \mapsto f\begin{pmatrix}
\mathbf{x} \\
\mathbf{y} \\
\end{pmatrix}$がそれぞれ$X_{\mathbf{y}}$、$Y_{\mathbf{x}}$とおかれ、さらに、その極限値$\lim_{\scriptsize \begin{matrix} \mathbf{x} \rightarrow \mathbf{a} \\ R \rightarrow T \end{matrix}}{X\left( \mathbf{x} \right)}$が$\mathbf{c}$とおかれよう。このとき、上の仮定は次のようになる。
\begin{itemize}
\item
  $\forall\varepsilon \in \mathbb{R}^{+}\exists\delta \in \mathbb{R}^{+}\forall\mathbf{x} \in A\forall\mathbf{y} \in B$に対し、$\mathbf{y} \in U\left( \mathbf{b},\delta \right) \cap S$が成り立つなら、$\left\| X_{\mathbf{y}}\left( \mathbf{x} \right) - X\left( \mathbf{x} \right) \right\| < \varepsilon$が成り立つ。
\item
  $\forall\mathbf{y} \in B\forall\varepsilon \in \mathbb{R}^{+}\exists\gamma' \in \mathbb{R}^{+}\forall\mathbf{x} \in A$に対し、$\mathbf{x} \in U\left( \mathbf{a},\gamma' \right) \cap R$が成り立つなら、$\left\| Y_{\mathbf{x}}\left( \mathbf{y} \right) - Y\left( \mathbf{y} \right) \right\| < \varepsilon$が成り立つ。
\item
  $\forall\varepsilon \in \mathbb{R}^{+}\exists\gamma'' \in \mathbb{R}^{+}\forall\mathbf{x} \in A$に対し、$\mathbf{x} \in U\left( \mathbf{a},\gamma'' \right) \cap R$が成り立つなら、$\left\| X\left( \mathbf{x} \right) - \mathbf{c} \right\| < \varepsilon$かつ$\mathbf{c} \in T$が成り立つ。
\end{itemize}
そこで、$\forall\varepsilon \in \mathbb{R}^{+}\exists\delta \in \mathbb{R}^{+}\forall\mathbf{y} \in B$に対し、$\mathbf{y} \in U\left( \mathbf{b},\delta \right) \cap S$が成り立つなら、$\exists\gamma \in \mathbb{R}^{+}\forall\mathbf{x} \in A$に対し、$U\left( \mathbf{a},\gamma \right) \cap R \subseteq U\left( \mathbf{a},\gamma' \right) \cap U\left( \mathbf{a},\gamma'' \right) \cap R$が成り立たせることができると注意すれば、$\mathbf{x} \in U\left( \mathbf{a},\gamma \right) \cap R$が成り立つとき、次のようになる。
\begin{align*}
\left\| Y\left( \mathbf{y} \right) - \mathbf{c} \right\| &= \left\| Y\left( \mathbf{y} \right) - Y_{\mathbf{x}}\left( \mathbf{y} \right) + Y_{\mathbf{x}}\left( \mathbf{y} \right) - X\left( \mathbf{x} \right) + X\left( \mathbf{x} \right) - \mathbf{c} \right\|\\
&= \left\| f\begin{pmatrix}
\mathbf{x} \\
\mathbf{y} \\
\end{pmatrix} - X\left( \mathbf{x} \right) + Y\left( \mathbf{y} \right) - f\begin{pmatrix}
\mathbf{x} \\
\mathbf{y} \\
\end{pmatrix} + X\left( \mathbf{x} \right) - \mathbf{c} \right\|\\
&\leq \left\| f\begin{pmatrix}
\mathbf{x} \\
\mathbf{y} \\
\end{pmatrix} - X\left( \mathbf{x} \right) \right\| + \left\| Y\left( \mathbf{y} \right) - f\begin{pmatrix}
\mathbf{x} \\
\mathbf{y} \\
\end{pmatrix} \right\| + \left\| X\left( \mathbf{x} \right) - \mathbf{c} \right\|\\
&= \left\| X_{\mathbf{y}}\left( \mathbf{x} \right) - X\left( \mathbf{x} \right) \right\| + \left\| Y_{\mathbf{x}}\left( \mathbf{y} \right) - Y\left( \mathbf{y} \right) \right\| + \left\| X\left( \mathbf{x} \right) - \mathbf{c} \right\|\\
&< 3\varepsilon
\end{align*}
特に、$\forall\varepsilon \in \mathbb{R}^{+}\exists\delta \in \mathbb{R}^{+}\forall\mathbf{y} \in B$に対し、$\mathbf{y} \in U\left( \mathbf{b},\delta \right) \cap S$が成り立つなら、$\left\| Y\left( \mathbf{y} \right) - \mathbf{c} \right\| < 3\varepsilon$が成り立つ。これにより、$\lim_{\scriptsize \begin{matrix} \mathbf{y} \rightarrow \mathbf{b} \\ S \rightarrow T \end{matrix}}{Y\left( \mathbf{y} \right)} = \mathbf{c} =\lim_{\scriptsize \begin{matrix} \mathbf{x} \rightarrow \mathbf{a} \\ R \rightarrow T \end{matrix}}{X\left( \mathbf{x} \right)}$が成り立つので、次のようになる。
\begin{align*}
\lim_{\scriptsize \begin{matrix} \mathbf{y} \rightarrow \mathbf{b} \\ S \rightarrow T \end{matrix}}{\lim_{\scriptsize \begin{matrix} \mathbf{x} \rightarrow \mathbf{a} \\ R \rightarrow T \end{matrix}}{f\begin{pmatrix}
\mathbf{x} \\
\mathbf{y} \\
\end{pmatrix}}} &= \lim_{\scriptsize \begin{matrix} \mathbf{y} \rightarrow \mathbf{b} \\ S \rightarrow T \end{matrix}}{\lim_{\scriptsize \begin{matrix} \mathbf{x} \rightarrow \mathbf{a} \\ R \rightarrow T \end{matrix}}{Y_{\mathbf{x}}\left( \mathbf{y} \right)}}\\
&= \lim_{\scriptsize \begin{matrix} \mathbf{y} \rightarrow \mathbf{b} \\ S \rightarrow T \end{matrix}}{Y\left( \mathbf{y} \right)}\\
&= \lim_{\scriptsize \begin{matrix} \mathbf{x} \rightarrow \mathbf{a} \\ R \rightarrow T \end{matrix}}{X\left( \mathbf{x} \right)}\\
&= \lim_{\scriptsize \begin{matrix} \mathbf{x} \rightarrow \mathbf{a} \\ R \rightarrow T \end{matrix}}{\lim_{\scriptsize \begin{matrix} \mathbf{y} \rightarrow \mathbf{b} \\ S \rightarrow T \end{matrix}}{X_{\mathbf{y}}\left( \mathbf{x} \right)}}\\
&= \lim_{\scriptsize \begin{matrix} \mathbf{x} \rightarrow \mathbf{a} \\ R \rightarrow T \end{matrix}}{\lim_{\scriptsize \begin{matrix} \mathbf{y} \rightarrow \mathbf{b} \\ S \rightarrow T \end{matrix}}{f\begin{pmatrix}
\mathbf{x} \\
\mathbf{y} \\
\end{pmatrix}}}
\end{align*}\par
$\forall\gamma,\delta \in \mathbb{R}^{+}$に対し、$r = \min\left\{ \gamma,\delta \right\}$として、$U\left( \begin{pmatrix}
\mathbf{a} \\
\mathbf{b} \\
\end{pmatrix},r \right) \subseteq U\left( \mathbf{a},\gamma \right) \times U\left( \mathbf{b},\delta \right)$が成り立つことを示そう。$\forall\begin{pmatrix}
\mathbf{x} \\
\mathbf{y} \\
\end{pmatrix} \in \mathbb{R}^{l} \times \mathbb{R}^{m}$に対し、$\begin{pmatrix}
\mathbf{x} \\
\mathbf{y} \\
\end{pmatrix} \in U\left( \begin{pmatrix}
\mathbf{a} \\
\mathbf{b} \\
\end{pmatrix},r \right)$が成り立つなら、$\left\| \begin{pmatrix}
\mathbf{x} \\
\mathbf{y} \\
\end{pmatrix} - \begin{pmatrix}
\mathbf{a} \\
\mathbf{b} \\
\end{pmatrix} \right\| < r$が成り立つので、次のようになる。
\begin{align*}
\left\| \mathbf{x} - \mathbf{a} \right\|^{2} &\leq \left\| \mathbf{x} - \mathbf{a} \right\|^{2} + \left\| \mathbf{y} - \mathbf{b} \right\|^{2}\\
&= \left\| \begin{pmatrix}
\mathbf{x - a} \\
\mathbf{y - b} \\
\end{pmatrix} \right\|^{2} = \left\| \begin{pmatrix}
\mathbf{x} \\
\mathbf{y} \\
\end{pmatrix} - \begin{pmatrix}
\mathbf{a} \\
\mathbf{b} \\
\end{pmatrix} \right\|^{2}\\
&< r^{2} \leq \gamma^{2}\\
\left\| \mathbf{y} - \mathbf{b} \right\|^{2} &\leq \left\| \mathbf{x} - \mathbf{a} \right\|^{2} + \left\| \mathbf{y} - \mathbf{b} \right\|^{2}\\
&= \left\| \begin{pmatrix}
\mathbf{x - a} \\
\mathbf{y - b} \\
\end{pmatrix} \right\|^{2} = \left\| \begin{pmatrix}
\mathbf{x} \\
\mathbf{y} \\
\end{pmatrix} - \begin{pmatrix}
\mathbf{a} \\
\mathbf{b} \\
\end{pmatrix} \right\|^{2}\\
&< r^{2} \leq \delta^{2}
\end{align*}
したがって、$\mathbf{x} \in U\left( \mathbf{a},\gamma \right)$かつ$\mathbf{y} \in U\left( \mathbf{b},\delta \right)$が成り立つので、$\begin{pmatrix}
\mathbf{x} \\
\mathbf{y} \\
\end{pmatrix} \in U\left( \mathbf{a},\gamma \right) \times U\left( \mathbf{b},\delta \right)$が成り立つ。\par
$\forall\varepsilon \in \mathbb{R}^{+}\exists\gamma,\delta \in \mathbb{R}^{+}$に対し、$r = \min\left\{ \gamma,\delta \right\}$とすれば、$\forall\begin{pmatrix}
\mathbf{x} \\
\mathbf{y} \\
\end{pmatrix} \in A \times B$に対し、$\begin{pmatrix}
\mathbf{x} \\
\mathbf{y} \\
\end{pmatrix} \in U\left( \begin{pmatrix}
\mathbf{a} \\
\mathbf{b} \\
\end{pmatrix},r \right) \cap (R \times S)$が成り立つなら、次のようになるので、
\begin{align*}
U\left( \begin{pmatrix}
\mathbf{a} \\
\mathbf{b} \\
\end{pmatrix},r \right) \cap (R \times S) &\subseteq \left( U\left( \mathbf{a},\gamma \right) \times U\left( \mathbf{b},\delta \right) \right) \cap (R \times S)\\
&= \left( U\left( \mathbf{a},\gamma \right) \cap R \right) \times \left( U\left( \mathbf{b},\delta \right) \cap S \right)
\end{align*}
$\mathbf{x} \in U\left( \mathbf{a},\gamma \right) \cap R$かつ$\mathbf{y} \in U\left( \mathbf{b},\delta \right) \cap S$が成り立つ。したがって、次のようになる。
\begin{align*}
\left\| f\begin{pmatrix}
\mathbf{x} \\
\mathbf{y} \\
\end{pmatrix} - \mathbf{c} \right\| &= \left\| f\begin{pmatrix}
\mathbf{x} \\
\mathbf{y} \\
\end{pmatrix} - X\left( \mathbf{x} \right) + X\left( \mathbf{x} \right) - \mathbf{c} \right\|\\
&\leq \left\| f\begin{pmatrix}
\mathbf{x} \\
\mathbf{y} \\
\end{pmatrix} - X\left( \mathbf{x} \right) \right\| + \left\| X\left( \mathbf{x} \right) - \mathbf{c} \right\|\\
&< 2\varepsilon
\end{align*}
これにより、$\lim_{\scriptsize \begin{matrix} \begin{pmatrix}
\mathbf{x} \\
\mathbf{y} \\
\end{pmatrix} \rightarrow \begin{pmatrix}
\mathbf{a} \\
\mathbf{b} \\
\end{pmatrix} \\ R \times S \rightarrow T \end{matrix}}{f\begin{pmatrix}
\mathbf{x} \\
\mathbf{y} \\
\end{pmatrix}} = \mathbf{c} =\lim_{\scriptsize \begin{matrix} \mathbf{x} \rightarrow \mathbf{a} \\ R \rightarrow T \end{matrix}}{X\left( \mathbf{x} \right)}$が成り立つので、次のようになる。
\begin{align*}
\lim_{\scriptsize \begin{matrix} \begin{pmatrix}
\mathbf{x} \\
\mathbf{y} \\
\end{pmatrix} \rightarrow \begin{pmatrix}
\mathbf{a} \\
\mathbf{b} \\
\end{pmatrix} \\ R \times S \rightarrow T \end{matrix}}{f\begin{pmatrix}
\mathbf{x} \\
\mathbf{y} \\
\end{pmatrix}} &= \lim_{\scriptsize \begin{matrix} \mathbf{x} \rightarrow \mathbf{a} \\ R \rightarrow T \end{matrix}}{X\left( \mathbf{x} \right)}\\
&= \lim_{\scriptsize \begin{matrix} \mathbf{x} \rightarrow \mathbf{a} \\ R \rightarrow T \end{matrix}}{\lim_{\scriptsize \begin{matrix} \mathbf{y} \rightarrow \mathbf{b} \\ S \rightarrow T \end{matrix}}{X_{\mathbf{y}}\left( \mathbf{x} \right)}}\\
&= \lim_{\scriptsize \begin{matrix} \mathbf{x} \rightarrow \mathbf{a} \\ R \rightarrow T \end{matrix}}{\lim_{\scriptsize \begin{matrix} \mathbf{y} \rightarrow \mathbf{b} \\ S \rightarrow T \end{matrix}}{f\begin{pmatrix}
\mathbf{x} \\
\mathbf{y} \\
\end{pmatrix}}}
\end{align*}\par
よって、次式が成り立つ。
\begin{align*}
\lim_{\scriptsize \begin{matrix} \begin{pmatrix}
\mathbf{x} \\
\mathbf{y} \\
\end{pmatrix} \rightarrow \begin{pmatrix}
\mathbf{a} \\
\mathbf{b} \\
\end{pmatrix} \\ R \times S \rightarrow T \end{matrix}}{f\begin{pmatrix}
\mathbf{x} \\
\mathbf{y} \\
\end{pmatrix}} = \lim_{\scriptsize \begin{matrix} \mathbf{x} \rightarrow \mathbf{a} \\ R \rightarrow T \end{matrix}}{\lim_{\scriptsize \begin{matrix} \mathbf{y} \rightarrow \mathbf{b} \\ S \rightarrow T \end{matrix}}{f\begin{pmatrix}
\mathbf{x} \\
\mathbf{y} \\
\end{pmatrix}}} = \lim_{\scriptsize \begin{matrix} \mathbf{y} \rightarrow \mathbf{b} \\ S \rightarrow T \end{matrix}}{\lim_{\scriptsize \begin{matrix} \mathbf{x} \rightarrow \mathbf{a} \\ R \rightarrow T \end{matrix}}{f\begin{pmatrix}
\mathbf{x} \\
\mathbf{y} \\
\end{pmatrix}}}
\end{align*}
\end{proof}
%\hypertarget{diniux306eux5b9aux7406}{%
\subsubsection{Diniの定理}%\label{diniux306eux5b9aux7406}}
\begin{thm}[Diniの定理]\label{4.1.11.19}
$A \subseteq R \subseteq \mathbb{R}^{m}$、$S \subseteq \mathbb{R}$なる有界関数空間$\mathfrak{B}(A,S)$の関数列$\left( f_{k} \right)_{k \in \mathbb{N}}$が与えられたとき、次のことが成り立つなら、
\begin{itemize}
\item
  その集合$A$がその集合$R$でcompactである。
\item
  $\forall k \in \mathbb{N}$に対し、その関数$f_{k}$はその集合$A$で連続である。
\item
  その実数列$\left( f_{k} \right)_{k \in \mathbb{N}}$は単調増加している、即ち、$\forall k \in \mathbb{N}$に対し、$f_{k} \leq f_{k + 1}$が成り立つ\footnote{もちろん、$f_{k} \geq f_{k + 1}$が成り立つとしても一般性は失われない。}。
\item
  その関数列$\left( f_{k} \right)_{k \in \mathbb{N}}$が極限関数$f:A \rightarrow S$にその集合$A$上で各点収束する。
\item
  その極限関数$f:A \rightarrow S$がその集合$A$で連続である。
\end{itemize}
その関数列$\left( f_{k} \right)_{k \in \mathbb{N}}$はその集合$A$上でその極限関数$f:A \rightarrow S$に一様収束する。この定理をDiniの定理という。
\end{thm}
\begin{proof}
$A \subseteq R \subseteq \mathbb{R}^{m}$、$S \subseteq \mathbb{R}$なる有界関数空間$\mathfrak{B}(A,S)$の関数列$\left( f_{k} \right)_{k \in \mathbb{N}}$が与えられたとき、次のことが成り立つとき、
\begin{itemize}
\item
  その集合$A$がその集合$R$でcompactである。
\item
  $\forall k \in \mathbb{N}$に対し、その関数$f_{k}$はその集合$A$で連続である。
\item
  $\forall k \in \mathbb{N}$に対し、$f_{k} \leq f_{k + 1}$が成り立つ。
\item
  その関数列$\left( f_{k} \right)_{k \in \mathbb{N}}$が極限関数$f:A \rightarrow S$にその集合$A$上で各点収束する。
\item
  その極限関数$f:A \rightarrow S$がその集合$A$で連続である。
\end{itemize}
上から4つ目の仮定より$\forall\mathbf{x} \in A\forall\varepsilon \in \mathbb{R}^{+}\exists N_{\mathbf{x}} \in \mathbb{N}$に対し、次式が成り立つ。
\begin{align*}
0 \leq \left| f\left( \mathbf{x} \right) - f_{N_{\mathbf{x}}}\left( \mathbf{x} \right) \right| < \varepsilon
\end{align*}
上から2つ目、5つ目の仮定よりその関数$f_{N_{\mathbf{x}}}:A \rightarrow S$とその極限関数$f:A \rightarrow S$どちらもその点$\mathbf{x}$で連続であるので、$\exists\delta_{\mathbf{x}} \in \mathbb{R}^{+}\forall\mathbf{y} \in A$に対し、$\mathbf{y} \in U\left( \mathbf{x},\delta_{\mathbf{x}} \right) \cap R$が成り立つなら、次式が成り立つ。
\begin{align*}
0 \leq \left| f_{N_{\mathbf{x}}}\left( \mathbf{y} \right) - f_{N_{\mathbf{x}}}\left( \mathbf{x} \right) \right| < \varepsilon,\ \ 0 \leq \left| f\left( \mathbf{y} \right) - f\left( \mathbf{x} \right) \right| < \varepsilon
\end{align*}
これにより、次のようになる。
\begin{align*}
\left| f\left( \mathbf{y} \right) - f_{N_{\mathbf{x}}}\left( \mathbf{y} \right) \right| &= \left| f\left( \mathbf{x} \right) - f\left( \mathbf{x} \right) + f\left( \mathbf{y} \right) - f_{N_{\mathbf{x}}}\left( \mathbf{x} \right) + f_{N_{\mathbf{x}}}\left( \mathbf{x} \right) - f_{N_{\mathbf{x}}}\left( \mathbf{y} \right) \right|\\
&= \left| f\left( \mathbf{x} \right) - f_{N_{\mathbf{x}}}\left( \mathbf{x} \right) + f_{N_{\mathbf{x}}}\left( \mathbf{x} \right) - f_{N_{\mathbf{x}}}\left( \mathbf{y} \right) + f\left( \mathbf{y} \right) - f\left( \mathbf{x} \right) \right|\\
&\leq \left| f\left( \mathbf{x} \right) - f_{N_{\mathbf{x}}}\left( \mathbf{x} \right) \right| + \left| f_{N_{\mathbf{x}}}\left( \mathbf{x} \right) - f_{N_{\mathbf{x}}}\left( \mathbf{y} \right) \right| + \left| f\left( \mathbf{y} \right) - f\left( \mathbf{x} \right) \right|\\
&= \left| f\left( \mathbf{x} \right) - f_{N_{\mathbf{x}}}\left( \mathbf{x} \right) \right| + \left| f_{N_{\mathbf{x}}}\left( \mathbf{y} \right) - f_{N_{\mathbf{x}}}\left( \mathbf{x} \right) \right| + \left| f\left( \mathbf{y} \right) - f\left( \mathbf{x} \right) \right|\\
&< 3\varepsilon
\end{align*}\par
そこで、その族$\left\{ U\left( \mathbf{x},\delta_{\mathbf{x}} \right) \cap R \right\}_{\mathbf{x} \in A}$がその集合$A$の開被覆であるので、1つ目の仮定よりその集合$A$のある有限集合な部分集合$A'$が存在して、その族$\left\{ U\left( \mathbf{x},\delta_{\mathbf{x}} \right) \cap R \right\}_{\mathbf{x} \in A'}$がその集合$A$の開被覆であることができる。そこで、$N = \max\left\{ N_{\mathbf{x}} \right\}_{\mathbf{x} \in A'}$とすれば、$\forall\varepsilon \in \mathbb{R}^{+}\exists N \in \mathbb{N}\forall\mathbf{y} \in A\forall k \in \mathbb{N}$に対し、$N \leq k$が成り立つなら、その族$\left\{ U\left( \mathbf{x},\delta_{\mathbf{x}} \right) \cap R \right\}_{\mathbf{x} \in A'}$がその集合$A$の開被覆であるので、$\exists\mathbf{x} \in A'\exists\delta_{\mathbf{x}} \in \mathbb{R}^{+}$に対し、$\mathbf{y} \in U\left( \mathbf{x},\delta_{\mathbf{x}} \right) \cap R$が成り立つので、上記の議論により、次のようになる。
\begin{align*}
0 \leq \left| f\left( \mathbf{y} \right) - f_{N_{\mathbf{x}}}\left( \mathbf{y} \right) \right| < 3\varepsilon
\end{align*}
3つ目の仮定よりしたがって、次のようになる。
\begin{align*}
0 &\leq \left| f_{N}\left( \mathbf{y} \right) - f\left( \mathbf{y} \right) \right|\\
&= f\left( \mathbf{y} \right) - f_{N}\left( \mathbf{y} \right)\\
&= f\left( \mathbf{y} \right) - f_{N_{\mathbf{x}}}\left( \mathbf{y} \right) - f_{N}\left( \mathbf{y} \right) + f_{N_{\mathbf{x}}}\left( \mathbf{y} \right)\\
&= \left| f\left( \mathbf{y} \right) - f_{N_{\mathbf{x}}}\left( \mathbf{y} \right) \right| - \left| f_{N}\left( \mathbf{y} \right) - f_{N_{\mathbf{x}}}\left( \mathbf{y} \right) \right|\\
&\leq \left| f\left( \mathbf{y} \right) - f_{N_{\mathbf{x}}}\left( \mathbf{y} \right) \right| < 3\varepsilon
\end{align*}
よって、その関数列$\left( f_{k} \right)_{k \in \mathbb{N}}$はその集合$A$上でその極限関数$f:A \rightarrow S$に一様収束する。
\end{proof}
%\hypertarget{ux95a2ux6570ux5217ux3068abelux5909ux5f62}{%
\subsubsection{関数列とAbel変形}%\label{ux95a2ux6570ux5217ux3068abelux5909ux5f62}}
\begin{thm}[関数列に関するDirichletの収束判定法]\label{4.1.11.20}
$A \subseteq R \subseteq \mathbb{R}^{m}$、$S \subseteq \mathbb{R}$なる有界関数空間$\mathfrak{B}(A,S)$の関数列たち$\left( f_{k} \right)_{k \in \mathbb{N}}$、$\left( g_{k} \right)_{k \in \mathbb{N}}$が与えられたとき、次のことが成り立つなら、
\begin{itemize}
\item
  $\exists C \in \mathbb{R}^{+} \cup \left\{ 0 \right\}\forall n \in \mathbb{N}$に対し、次式が成り立つ。
\begin{align*}
\left\| \sum_{k \in \varLambda_{n}} f_{k} \right\|_{A,\infty} \leq C
\end{align*}
\item
  $0 \leq \left( g_{k} \right)_{k \in \mathbb{N}}$が成り立つ、即ち、$\forall k \in \mathbb{N}$に対し、$0 \leq g_{k}$が成り立つ。
\item
  その関数列$\left( g_{k} \right)_{k \in \mathbb{N}}$は単調減少している、即ち、$\forall k \in \mathbb{N}$に対し、$g_{k} \geq g_{k + 1}$が成り立つ。
\item
  その関数列$\left( g_{k} \right)_{k \in \mathbb{N}}$は極限関数$0:A \rightarrow S$にその集合$A$上で一様収束する。
\end{itemize}
その級数$\left( \sum_{k \in \varLambda_{n}} {f_{k}g_{k}} \right)_{n \in \mathbb{N}}$はその集合$A$上で一様収束する。この定理を関数列に関するDirichletの収束判定法という。
\end{thm}
\begin{proof}
$A \subseteq R \subseteq \mathbb{R}^{m}$、$S \subseteq \mathbb{R}$なる有界関数空間$\mathfrak{B}(A,S)$の関数列たち$\left( f_{k} \right)_{k \in \mathbb{N}}$、$\left( g_{k} \right)_{k \in \mathbb{N}}$が与えられたとき、次のことが成り立つなら、
\begin{itemize}
\item
  $\exists C \in \mathbb{R}^{+} \cup \left\{ 0 \right\}\forall n \in \mathbb{N}$に対し、次式が成り立つ。
\begin{align*}
\left\| \sum_{k \in \varLambda_{n}} f_{k} \right\|_{A,\infty} \leq C
\end{align*}
\item
  $0 \leq \left( g_{k} \right)_{k \in \mathbb{N}}$が成り立つ。
\item
  その関数列$\left( g_{k} \right)_{k \in \mathbb{N}}$は単調減少している。
\item
  その関数列$\left( g_{k} \right)_{k \in \mathbb{N}}$は極限関数$0:A \rightarrow S$にその集合$A$上で一様収束する。
\end{itemize}
$\forall\mathbf{x} \in A$に対し、次のことが成り立つので、
\begin{itemize}
\item
  $\exists C \in \mathbb{R}^{+} \cup \left\{ 0 \right\}\forall n \in \mathbb{N}$に対し、次式が成り立つ。
\begin{align*}
\left| \sum_{k \in \varLambda_{n}} {f_{k}\left( \mathbf{x} \right)} \right| \leq \sup_{\mathbf{x} \in A}\left| \sum_{k \in \varLambda_{n}} {f_{k}\left( \mathbf{x} \right)} \right| = \left\| \sum_{k \in \varLambda_{n}} f_{k} \right\|_{A,\infty} \leq C
\end{align*}
\item
  $0 \leq \left( g_{k}\left( \mathbf{x} \right) \right)_{k \in \mathbb{N}}$が成り立つ。
\item
  その実数列$\left( g_{k}\left( \mathbf{x} \right) \right)_{k \in \mathbb{N}}$が単調減少している。
\end{itemize}
定理\ref{4.1.8.26}、即ち、Abelの変形より、$\forall m,n \in \mathbb{N}$に対し、$m + 1 \leq n$が成り立つなら、次式が成り立つ。
\begin{align*}
\left| \sum_{k \in \varLambda_{n} \setminus \varLambda_{m}} {f_{k}\left( \mathbf{x} \right)g_{k}\left( \mathbf{x} \right)} \right| \leq 2Cg_{m + 1}\left( \mathbf{x} \right)
\end{align*}
したがって、次のようになる。
\begin{align*}
\left\| \sum_{k \in \varLambda_{n} \setminus \varLambda_{m}} {f_{k}g_{k}} \right\|_{A,\infty} &= \sup_{\mathbf{x} \in A}\left| \sum_{k \in \varLambda_{n} \setminus \varLambda_{m}} {f_{k}\left( \mathbf{x} \right)g_{k}\left( \mathbf{x} \right)} \right|\\
&\leq \sup_{\mathbf{x} \in A}{2Cg_{m + 1}\left( \mathbf{x} \right)}\\
&= 2C\sup_{\mathbf{x} \in A}\left| g_{m + 1}\left( \mathbf{x} \right) \right|\\
&= 2C\left\| g_{m + 1} \right\|_{A,\infty}
\end{align*}\par
ここで、その関数列$\left( g_{k} \right)_{k \in \mathbb{N}}$は極限関数$0:A \rightarrow S$にその集合$A$上で一様収束することから、$\forall\varepsilon \in \mathbb{R}^{+}\exists N \in \mathbb{N}\forall m \in \mathbb{N}$に対し、$N \leq m + 1$が成り立つなら、$\left\| g_{m + 1} \right\|_{A,\infty} < \varepsilon$が成り立つ。これにより、$\forall m,n \in \mathbb{N}$に対し、$N \leq m$かつ$N \leq n$が成り立つなら、$m + 1 \leq n$のとき、次のようになる。
\begin{align*}
\left\| \sum_{k \in \varLambda_{n}} {f_{k}g_{k}} - \sum_{k \in \varLambda_{m}} {f_{k}g_{k}} \right\|_{A,\infty} &= \left\| \sum_{k \in \varLambda_{n} \setminus \varLambda_{m}} {f_{k}g_{k}} \right\|_{A,\infty}\\
&\leq 2C\left\| g_{m + 1} \right\|_{A,\infty} \leq 2C\varepsilon
\end{align*}
定理\ref{4.1.11.14}、即ち、関数列に関するCauchyの一様収束条件より\footnote{詳しくいえば、$m + 1 \leq n$のとき、次のようになる。
\begin{align*}
\left\| \sum_{k \in \varLambda_{n}} {f_{k}g_{k}} - \sum_{k \in \varLambda_{m}} {f_{k}g_{k}} \right\|_{A,\infty} = \left\| \sum_{k \in \varLambda_{n} \setminus \varLambda_{m}} {f_{k}g_{k}} \right\|_{A,\infty} \leq 2C\left\| g_{m + 1} \right\|_{A,\infty} \leq 2C\varepsilon
\end{align*}
$m = n$のときは明らかに次式が成り立つ。
\begin{align*}
\left\| \sum_{k \in \varLambda_{n}} {f_{k}g_{k}} - \sum_{k \in \varLambda_{m}} {f_{k}g_{k}} \right\|_{A,\infty} = 0 \leq 2C\varepsilon
\end{align*}
$n + 1 \leq m$のとき、同様にして$\left\| \sum_{k \in \varLambda_{m} \setminus \varLambda_{n}} {f_{k}g_{k}} \right\|_{A,\infty} \leq 2C\left\| g_{n + 1} \right\|_{A,\infty}$が成り立つことが示されるので、次のようになる。
\begin{align*}
\left\| \sum_{k \in \varLambda_{n}} {f_{k}g_{k}} - \sum_{k \in \varLambda_{m}} {f_{k}g_{k}} \right\|_{A,\infty} = \left\| \sum_{k \in \varLambda_{m} \setminus \varLambda_{n}} {f_{k}g_{k}} \right\|_{A,\infty} \leq 2C\left\| g_{n + 1} \right\|_{A,\infty} < 2C\varepsilon
\end{align*}
いづれの場合でも、次式が成り立つ。
\begin{align*}
\left\| \sum_{k \in \varLambda_{n}} {f_{k}g_{k}} - \sum_{k \in \varLambda_{m}} {f_{k}g_{k}} \right\|_{A,\infty} < 2C\varepsilon
\end{align*}}その級数$\left( \sum_{k \in \varLambda_{n}} {f_{k}g_{k}} \right)_{n \in \mathbb{N}}$はその集合$A$上で一様収束する。
\end{proof}
\begin{thm}[関数列に関するAbelの収束判定法]\label{4.1.11.21}
$A \subseteq R \subseteq \mathbb{R}^{m}$、$S \subseteq \mathbb{R}$なる有界関数空間$\mathfrak{B}(A,S)$の関数列たち$\left( f_{k} \right)_{k \in \mathbb{N}}$、$\left( g_{k} \right)_{k \in \mathbb{N}}$が与えられたとき、次のことが成り立つなら、
\begin{itemize}
\item
  その関数列$\left( \sum_{k \in \varLambda_{n}} f_{k} \right)_{n \in \mathbb{N}}$はその集合$A$上で一様収束する。
\item
  $0 \leq \left( g_{k} \right)_{k \in \mathbb{N}}$が成り立つ、即ち、$\forall k \in \mathbb{N}$に対し、$0 \leq g_{k}$が成り立つ。
\item
  その関数列$\left( g_{k} \right)_{k \in \mathbb{N}}$は単調減少している、即ち、$\forall k \in \mathbb{N}$に対し、$g_{k} \geq g_{k + 1}$が成り立つ。
\item
  その関数$g_{1}$はその集合$A$上で有界である。
\end{itemize}
その級数$\left( \sum_{k \in \varLambda_{n}} {f_{k}g_{k}} \right)_{n \in \mathbb{N}}$はその集合$A$上で一様収束する。この定理を関数列に関するAbelの収束判定法という。
\end{thm}
\begin{proof}
$A \subseteq R \subseteq \mathbb{R}^{m}$、$S \subseteq \mathbb{R}$なる有界関数空間$\mathfrak{B}(A,S)$の関数列たち$\left( f_{k} \right)_{k \in \mathbb{N}}$、$\left( g_{k} \right)_{k \in \mathbb{N}}$が与えられたとき、次のことが成り立つなら、
\begin{itemize}
\item
  その関数列$\left( \sum_{k \in \varLambda_{n}} f_{k} \right)_{n \in \mathbb{N}}$はその集合$A$上で一様収束する。
\item
  $0 \leq \left( g_{k} \right)_{k \in \mathbb{N}}$が成り立つ。
\item
  その関数列$\left( g_{k} \right)_{k \in \mathbb{N}}$は単調減少している。
\item
  その関数$g_{1}$はその集合$A$上で有界である。
\end{itemize}
その関数列$\left( \sum_{k \in \varLambda_{n}} f_{k} \right)_{n \in \mathbb{N}}$がその集合$A$上で一様収束するので、定理\ref{4.1.11.14}、即ち、関数列に関するCauchyの一様収束条件より$\forall\varepsilon \in \mathbb{R}^{+}\exists N \in \mathbb{N}\forall n \in \mathbb{N}$に対し、$N < n + N$が成り立ち、したがって、次式が成り立つ。
\begin{align*}
\left\| \sum_{k \in \varLambda_{n}} f_{k + N} \right\|_{A,\infty} &= \left\| \sum_{k \in \varLambda_{n + N} \setminus \varLambda_{N}} f_{k} \right\|_{A,\infty}\\
&= \left\| \sum_{k \in \varLambda_{n + N}} f_{k} - \sum_{k \in \varLambda_{N}} f_{k} \right\|_{A,\infty} < \varepsilon
\end{align*}\par
ここで、$\forall\mathbf{x} \in A$に対し、次のことが成り立つので、
\begin{itemize}
\item
  $\exists C \in \mathbb{R}^{+} \cup \left\{ 0 \right\}\forall n \in \mathbb{N}$に対し、次式が成り立つ。
\begin{align*}
\left| \sum_{k \in \varLambda_{n}} {f_{k + N}\left( \mathbf{x} \right)} \right| \leq \sup_{\mathbf{x} \in A}\left| \sum_{k \in \varLambda_{n}} {f_{k + N}\left( \mathbf{x} \right)} \right| = \left\| \sum_{k \in \varLambda_{n}} f_{k + N} \right\|_{A,\infty} \leq \varepsilon
\end{align*}
\item
  $0 \leq \left( g_{k + N}\left( \mathbf{x} \right) \right)_{k \in \mathbb{N}}$が成り立つ。
\item
  その実数列$\left( g_{k + N}\left( \mathbf{x} \right) \right)_{k \in \mathbb{N}}$が単調減少している。
\end{itemize}
定理\ref{4.1.8.26}、即ち、Abelの変形より、$\forall m,n \in \mathbb{N}$に対し、$N \leq m < n$が成り立つなら、次式が成り立つ。
\begin{align*}
\left| \sum_{k \in \varLambda_{n - N} \setminus \varLambda_{m - N}} {f_{k + N}\left( \mathbf{x} \right)g_{k + N}\left( \mathbf{x} \right)} \right| \leq 2\varepsilon g_{m + 1}\left( \mathbf{x} \right)
\end{align*}
したがって、次のようになる。
\begin{align*}
\left\| \sum_{k \in \varLambda_{n - N} \setminus \varLambda_{m - N}} {f_{k + N}g_{k + N}} \right\|_{A,\infty} &= \sup_{\mathbf{x} \in A}\left| \sum_{k \in \varLambda_{n - N} \setminus \varLambda_{m - N}} {f_{k + N}\left( \mathbf{x} \right)g_{k + N}\left( \mathbf{x} \right)} \right|\\
&\leq \sup_{\mathbf{x} \in A}{2\varepsilon g_{m + 1}\left( \mathbf{x} \right)}\\
&= 2\varepsilon\sup_{\mathbf{x} \in A}{g_{m + 1}\left( \mathbf{x} \right)}\\
&\leq 2\varepsilon\sup_{\mathbf{x} \in A}{g_{1}\left( \mathbf{x} \right)}\\
&= 2\varepsilon\left\| g_{1} \right\|_{A,\infty}
\end{align*}
そこで、その関数$g_{1}$はその集合$A$上で有界であるので、$\exists M \in \mathbb{R}^{+}\forall\mathbf{x} \in A$に対し、$\left| g_{1}\left( \mathbf{x} \right) \right| < M$が成り立つので、次のようになる。
\begin{align*}
\left| g_{1}\left( \mathbf{x} \right) \right| \leq \sup_{\mathbf{x} \in A}\left| g_{1}\left( \mathbf{x} \right) \right| = \left\| g_{1} \right\|_{A,\infty} \leq M
\end{align*}
これにより、次のようになる。
\begin{align*}
\left\| \sum_{k \in \varLambda_{n}} {f_{k}g_{k}} - \sum_{k \in \varLambda_{m}} {f_{k}g_{k}} \right\|_{A,\infty} &= \left\| \sum_{k \in \varLambda_{n} \setminus \varLambda_{m}} {f_{k}g_{k}} \right\|_{A,\infty}\\
&= \left\| \sum_{k \in \varLambda_{n - N} \setminus \varLambda_{m - N}} {f_{k + N}g_{k + N}} \right\|_{A,\infty}\\
&= 2\varepsilon\left\| g_{1} \right\|_{A,\infty} \leq 2\varepsilon M
\end{align*}
定理\ref{4.1.11.14}、即ち、関数列に関するCauchyの一様収束条件よりその級数$\left( \sum_{k \in \varLambda_{n}} {f_{k}g_{k}} \right)_{n \in \mathbb{N}}$はその集合$A$上で一様収束する。
\end{proof}
\begin{thebibliography}{50}
\bibitem{1}
  杉浦光夫, 解析入門I, 東京大学出版社, 1985. 第34刷 p62-63,301-314,377-378 ISBN978-4-13-062005-5
\bibitem{2}
  数学の景色. "広義一様収束の定義と具体例". 数学の景色. \url{https://mathlandscape.com/unif-conv-on-cpts/} (2022-8-16 6:28 閲覧)
\bibitem{3}
  数学の景色. "一様収束と各点収束の違いを4つの例とともに理解する". 数学の景色. \url{https://mathlandscape.com/uniformly-pointwise-convergence/} (2022-8-16 6:29 閲覧)
\bibitem{4}
  数学の景色. "連続関数列の一様収束極限は必ず連続関数になることの証明". 数学の景色. \url{https://mathlandscape.com/unif-conv-to-continuous/} (2022-8-16 6:31 閲覧)
\bibitem{5}
  数学の景色. "ワイエルシュトラスのM判定法(優級数定理)とは~証明と具体例~". 数学の景色. \url{https://mathlandscape.com/m-test/} (2022-8-16 6:34 閲覧)
\bibitem{6}
  へんなの(@Notes\_JP). 極限操作(微分・積分・lim)の交換:定理と反例 - Notes\_JP. Hatena Blog. \url{https://www.mynote-jp.com/entry/interchange-of-limiting-operations} (2022年12月24日16:24 閲覧)
\end{thebibliography}
\end{document}
