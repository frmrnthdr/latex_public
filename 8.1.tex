\documentclass[a4paper]{jsarticle}
\setcounter{section}{0}
\usepackage{amsmath,amsfonts,amssymb,array,comment,mathtools,url,docmute}
\usepackage{longtable,booktabs,dcolumn,tabularx,mathtools,multirow,colortbl,xcolor}
\usepackage[dvipdfmx]{graphics}
\usepackage{bmpsize}
\usepackage{amsthm}
\usepackage{enumitem}
\setlistdepth{20}
\renewlist{itemize}{itemize}{20}
\setlist[itemize]{label=•}
\renewlist{enumerate}{enumerate}{20}
\setlist[enumerate]{label=\arabic*.}
\setcounter{MaxMatrixCols}{20}
\setcounter{tocdepth}{3}
\newcommand{\rotin}{\text{\rotatebox[origin=c]{90}{$\in $}}}
\renewcommand{\thesection}{第\arabic{section}部}
\renewcommand{\thesubsection}{\arabic{section}.\arabic{subsection}}
\renewcommand{\thesubsubsection}{\arabic{section}.\arabic{subsection}.\arabic{subsubsection}}
\everymath{\displaystyle}
\allowdisplaybreaks[4]
\usepackage{vtable}
\theoremstyle{definition}
\newtheorem{thm}{定理}[subsection]
\newtheorem*{thm*}{定理}
\newtheorem{dfn}{定義}[subsection]
\newtheorem*{dfn*}{定義}
\newtheorem{axs}[dfn]{公理}
\newtheorem*{axs*}{公理}
\renewcommand{\headfont}{\bfseries}
\makeatletter
  \renewcommand{\section}{%
    \@startsection{section}{1}{\z@}%
    {\Cvs}{\Cvs}%
    {\normalfont\huge\headfont\raggedright}}
\makeatother
\makeatletter
  \renewcommand{\subsection}{%
    \@startsection{subsection}{2}{\z@}%
    {0.5\Cvs}{0.5\Cvs}%
    {\normalfont\LARGE\headfont\raggedright}}
\makeatother
\makeatletter
  \renewcommand{\subsubsection}{%
    \@startsection{subsubsection}{3}{\z@}%
    {0.4\Cvs}{0.4\Cvs}%
    {\normalfont\Large\headfont\raggedright}}
\makeatother
\makeatletter
\renewenvironment{proof}[1][\proofname]{\par
  \pushQED{\qed}%
  \normalfont \topsep6\p@\@plus6\p@\relax
  \trivlist
  \item\relax
  {
  #1\@addpunct{.}}\hspace\labelsep\ignorespaces
}{%
  \popQED\endtrivlist\@endpefalse
}
\makeatother
\renewcommand{\proofname}{\textbf{証明}}
\usepackage{tikz,graphics}
\usepackage[dvipdfmx]{hyperref}
\usepackage{pxjahyper}
\hypersetup{
 setpagesize=false,
 bookmarks=true,
 bookmarksdepth=tocdepth,
 bookmarksnumbered=true,
 colorlinks=false,
 pdftitle={},
 pdfsubject={},
 pdfauthor={},
 pdfkeywords={}}
\begin{document}
\section{位相空間論}
ここでは、位相空間論の初歩的な内容を扱うことにする。この分野の特徴としてはかなり抽象的な内容であり心象するのがかなり困難で、実際、議論するときは記号論理学や集合論の知識が駆使されることになる\footnote{俗にいえば、これは数学科でかなり人を選ぶ魔窟ともいわれており、図学的なセンスがあってもこの分野ですっかり幾何学が嫌いになってしまう人もいれば、逆に、図学アレルギーの人でもこの分野を通じてすっかり幾何学に慣れ親しむ人もいたり…(独断と偏見を含む)。あと、しばしばみかける標語としては、初学の時点ではイメージしたり語句の名前の由来を考えだしたら負けというのがあって…。}。なぜ位相空間を考えようとしたかの動機を説明する。諸説あるが、微分積分学でお馴染みである極限の厳密な議論である$\varepsilon$-$\delta$論法を$n$次元に拡張してここで位相空間に該当するものが後に述べる$n$次元Euclid空間そのものでありこれを一般的な集合に対して適用できないか考えだした結果である。これは連続写像の章でみれば、いかに位相空間がうまく定義されているかがより顕在化するであろう。ここで、微分可能な集合を考えた場合、開球体という概念が導入されるはずであるが、実は属されるどの点を中心とするある開球体を含むような集合全体がEuclid空間における位相となっている。この議論は距離空間における位相空間の議論でよく分かるであろう。さて、なぜ一般的な集合に位相を持ち込みたかった理由について述べれば、1つの理由としては、一般的な集合に縁があるかないかで班分けしたかったものが挙げられる。この章の最後のほうに境界という概念が出てくるのだが、これは位相空間があったおかげで生じた概念である。これによって、後に述べるように一般的な集合に点列の極限の概念が定式化でき、さらに連結の章で詳しく述べられることになると思われるが、集合が1つのかたまりになっているか否かという違いも定式化できる。やや解析学よりの理由になってしまったが、いづれにせよ、幾何学や解析学以外にも代数学にもあらわれるさまざまな状況に現れる共通した概念であるゆえ、このようにかなり抽象化されているのが特徴である\footnote{ほかの説明があったら、ぜひ教えてください!}。なお、記号論理学や集合論、実数論に関する初歩的な知識のみ仮定しておいた。
\end{document}