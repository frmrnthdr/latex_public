\documentclass[dvipdfmx]{jsarticle}
\setcounter{section}{1}
\setcounter{subsection}{9}
\usepackage{xr}
\externaldocument{8.1.1}
\externaldocument{8.1.3}
\externaldocument{8.1.4}
\externaldocument{8.1.6}
\usepackage{amsmath,amsfonts,amssymb,array,comment,mathtools,url,docmute}
\usepackage{longtable,booktabs,dcolumn,tabularx,mathtools,multirow,colortbl,xcolor}
\usepackage[dvipdfmx]{graphics}
\usepackage{bmpsize}
\usepackage{amsthm}
\usepackage{enumitem}
\setlistdepth{20}
\renewlist{itemize}{itemize}{20}
\setlist[itemize]{label=•}
\renewlist{enumerate}{enumerate}{20}
\setlist[enumerate]{label=\arabic*.}
\setcounter{MaxMatrixCols}{20}
\setcounter{tocdepth}{3}
\newcommand{\rotin}{\text{\rotatebox[origin=c]{90}{$\in $}}}
\newcommand{\amap}[6]{\text{\raisebox{-0.7cm}{\begin{tikzpicture} 
  \node (a) at (0, 1) {$\textstyle{#2}$};
  \node (b) at (#6, 1) {$\textstyle{#3}$};
  \node (c) at (0, 0) {$\textstyle{#4}$};
  \node (d) at (#6, 0) {$\textstyle{#5}$};
  \node (x) at (0, 0.5) {$\rotin $};
  \node (x) at (#6, 0.5) {$\rotin $};
  \draw[->] (a) to node[xshift=0pt, yshift=7pt] {$\textstyle{\scriptstyle{#1}}$} (b);
  \draw[|->] (c) to node[xshift=0pt, yshift=7pt] {$\textstyle{\scriptstyle{#1}}$} (d);
\end{tikzpicture}}}}
\newcommand{\twomaps}[9]{\text{\raisebox{-0.7cm}{\begin{tikzpicture} 
  \node (a) at (0, 1) {$\textstyle{#3}$};
  \node (b) at (#9, 1) {$\textstyle{#4}$};
  \node (c) at (#9+#9, 1) {$\textstyle{#5}$};
  \node (d) at (0, 0) {$\textstyle{#6}$};
  \node (e) at (#9, 0) {$\textstyle{#7}$};
  \node (f) at (#9+#9, 0) {$\textstyle{#8}$};
  \node (x) at (0, 0.5) {$\rotin $};
  \node (x) at (#9, 0.5) {$\rotin $};
  \node (x) at (#9+#9, 0.5) {$\rotin $};
  \draw[->] (a) to node[xshift=0pt, yshift=7pt] {$\textstyle{\scriptstyle{#1}}$} (b);
  \draw[|->] (d) to node[xshift=0pt, yshift=7pt] {$\textstyle{\scriptstyle{#2}}$} (e);
  \draw[->] (b) to node[xshift=0pt, yshift=7pt] {$\textstyle{\scriptstyle{#1}}$} (c);
  \draw[|->] (e) to node[xshift=0pt, yshift=7pt] {$\textstyle{\scriptstyle{#2}}$} (f);
\end{tikzpicture}}}}
\renewcommand{\thesection}{第\arabic{section}部}
\renewcommand{\thesubsection}{\arabic{section}.\arabic{subsection}}
\renewcommand{\thesubsubsection}{\arabic{section}.\arabic{subsection}.\arabic{subsubsection}}
\everymath{\displaystyle}
\allowdisplaybreaks[4]
\usepackage{vtable}
\theoremstyle{definition}
\newtheorem{thm}{定理}[subsection]
\newtheorem*{thm*}{定理}
\newtheorem{dfn}{定義}[subsection]
\newtheorem*{dfn*}{定義}
\newtheorem{axs}[dfn]{公理}
\newtheorem*{axs*}{公理}
\renewcommand{\headfont}{\bfseries}
\makeatletter
  \renewcommand{\section}{%
    \@startsection{section}{1}{\z@}%
    {\Cvs}{\Cvs}%
    {\normalfont\huge\headfont\raggedright}}
\makeatother
\makeatletter
  \renewcommand{\subsection}{%
    \@startsection{subsection}{2}{\z@}%
    {0.5\Cvs}{0.5\Cvs}%
    {\normalfont\LARGE\headfont\raggedright}}
\makeatother
\makeatletter
  \renewcommand{\subsubsection}{%
    \@startsection{subsubsection}{3}{\z@}%
    {0.4\Cvs}{0.4\Cvs}%
    {\normalfont\Large\headfont\raggedright}}
\makeatother
\makeatletter
\renewenvironment{proof}[1][\proofname]{\par
  \pushQED{\qed}%
  \normalfont \topsep6\p@\@plus6\p@\relax
  \trivlist
  \item\relax
  {
  #1\@addpunct{.}}\hspace\labelsep\ignorespaces
}{%
  \popQED\endtrivlist\@endpefalse
}
\makeatother
\renewcommand{\proofname}{\textbf{証明}}
\usepackage{tikz,graphics}
\usepackage[dvipdfmx]{hyperref}
\usepackage{pxjahyper}
\hypersetup{
 setpagesize=false,
 bookmarks=true,
 bookmarksdepth=tocdepth,
 bookmarksnumbered=true,
 colorlinks=false,
 pdftitle={},
 pdfsubject={},
 pdfauthor={},
 pdfkeywords={}}
\begin{document}
%\hypertarget{ux7b2c1ux53efux7b97ux516cux7406ux306bux304aux3051ux308bux70b9ux5217}{%
\subsection{第1可算公理における点列}%\label{ux7b2c1ux53efux7b97ux516cux7406ux306bux304aux3051ux308bux70b9ux5217}}
%\hypertarget{ux7b2c1ux53efux7b97ux516cux7406ux306eux57faux672cux88dcux984c}{%
\subsubsection{第1可算公理の基本補題}%\label{ux7b2c1ux53efux7b97ux516cux7406ux306eux57faux672cux88dcux984c}}
\begin{dfn*}[定義\ref{第1可算公理}の再掲]
位相空間$\left( S,\mathfrak{O} \right)$が与えられたとき、$\forall a \in S$に対し、たかだか可算集合である、即ち、${\#}{\mathbf{V}^{*}(a)} \leq \aleph_{0}$が成り立つようなその元$a$の基本近傍系$\mathbf{V}^{*}(a)$が存在することを第1可算公理という。
\end{dfn*}
\begin{dfn*}[定義\ref{第2可算公理}の再掲]
位相空間$\left( S,\mathfrak{O} \right)$の開基$\mathfrak{B}$がたかだか可算集合であること、即ち、${\#}\mathfrak{B} \leq \aleph_{0}$が成り立つことを第2可算公理という。
\end{dfn*}
\begin{thm*}[定理\ref{8.1.2.16}の再掲]
位相空間$\left( S,\mathfrak{O} \right)$が第2可算公理を満たすなら、第1可算公理も満たす。
\end{thm*}
\begin{thm*}[第1可算公理の基本補題\ref{8.1.2.17}の再掲]
位相空間$\left( S,\mathfrak{O} \right)$が第1可算公理を満たすとき、$\forall a \in S$に対し、その全近傍系$\mathbf{V}(a)$のある元の列$\left( U_{n} \right)_{n \in \mathbb{N}}$が存在して、次のことが成り立つ。
\begin{itemize}
\item
  $\forall V \in \mathbf{V}(a)\exists n \in \mathbb{N}$に対し、$U_{n} \subseteq V$が成り立つ。
\item
  $\forall n \in \mathbb{N}$に対し、$U_{n + 1} \subseteq U_{n}$が成り立つ。
\end{itemize}
この定理を第1可算公理の基本補題という。
\end{thm*}
%\hypertarget{ux7b2c1ux53efux7b97ux516cux7406ux3092ux6e80ux305fux3059ux4f4dux76f8ux7a7aux9593ux3068ux70b9ux5217}{%
\subsubsection{第1可算公理を満たす位相空間と点列}%\label{ux7b2c1ux53efux7b97ux516cux7406ux3092ux6e80ux305fux3059ux4f4dux76f8ux7a7aux9593ux3068ux70b9ux5217}}
\begin{thm}\label{8.1.10.1}
位相空間$\left( S,\mathfrak{O} \right)$が第1可算公理を満たすとき、その集合$S$の元$a$と点列$\left( a_{n} \right)_{n \in \mathbb{N}}$について、次のことは同値である。
\begin{itemize}
\item
  その元$a$がその点列$\left( a_{n} \right)_{n \in \mathbb{N}}$の堆積点である。
\item
  その点列$\left( a_{n} \right)_{n \in \mathbb{N}}$の部分列でその元$a$に収束するものが存在する。
\end{itemize}
\end{thm}
\begin{proof}
位相空間$\left( S,\mathfrak{O} \right)$が第1可算公理を満たすとき、その集合$S$の元$a$と点列$\left( a_{n} \right)_{n \in \mathbb{N}}$について、その元$a$がその点列$\left( a_{n} \right)_{n \in \mathbb{N}}$の堆積点であるなら、$\forall V \in \mathbf{V}(a)\forall n \in \mathbb{N}\exists n_{0} \in \mathbb{N}$に対し、$n \leq n_{0}$が成り立つかつ、$a_{n_{0}} \in V$が成り立つ。そこで、定理\ref{8.1.2.17}、即ち、第1可算公理の基本補題よりその全近傍系$\mathbf{V}(a)$のある元の列$\left( U_{n} \right)_{n \in \mathbb{N}}$が存在して、次のことが成り立つ。
\begin{itemize}
\item
  $\forall V \in \mathbf{V}(a)\exists n \in \mathbb{N}$に対し、$U_{n} \subseteq V$が成り立つ。
\item
  $\forall n \in \mathbb{N}$に対し、$U_{n + 1} \subseteq U_{n}$が成り立つ。
\end{itemize}
そこで、$\exists m' \in \mathbb{N}$に対し、$1 \leq m'$かつ$a_{m'} \in U_{1}$が成り立つので、このような自然数$m'$を$n_{1}$とおき、さらに、$\forall k \in \mathbb{N}\exists n_{k} \in \mathbb{N}$に対し、$k \leq n_{k}$かつ$a_{n_{k}} \in U_{k}$が成り立つことから、$n_{k} < m'$なる任意の自然数$m'$に対し、ある自然数$n'$が存在して、$m' \leq n'$が成り立つかつ、$a_{n'} \in U_{k + 1}$が成り立つので、このような自然数$n'$を$n_{k + 1}$とおけば、その点列$\left( a_{n_{k}} \right)_{k \in \mathbb{N}}$は帰納的に定義されることができてその点列$\left( a_{n} \right)_{n \in \mathbb{N}}$の部分列となっている。このとき、$\forall V \in \mathbf{V}(a)\exists k_{0} \in \mathbb{N}\forall k \in \mathbb{N}$に対し、$k_{0} \leq k$が成り立つなら、$U_{k_{0}} \subseteq V$かつ$U_{k} \subseteq U_{k_{0}}$が成り立つかつ、$a_{k} \in U_{k}$が成り立つので、$a_{k} \in V$が成り立つ。これにより、その点列$\left( a_{n} \right)_{n \in \mathbb{N}}$の部分列$\left( a_{n_{k}} \right)_{k \in \mathbb{N}}$はその元$a$に収束する。\par
その点列$\left( a_{n} \right)_{n \in \mathbb{N}}$の部分列でその元$a$に収束するもの$\left( a_{n_{k}} \right)_{k \in \mathbb{N}}$が存在するなら、$\forall V \in \mathbf{V}(a)\exists k_{0} \in \mathbb{N}\forall k \in \mathbb{N}$に対し、$k_{0} \leq k$が成り立つなら、$a_{n_{k}} \in V$が成り立つ。これにより、$\forall n \in \mathbb{N}$に対し、$n \leq n_{k_{0}}$が成り立つとき、もちろん、$n \leq n_{k_{0}}$かつ$a_{n_{k_{0}}} \in V$が成り立つし、$n_{k_{0}} < n$が成り立つとき、次式が成り立つので、
\begin{align*}
V\left( \left( n_{k} \right)_{k \in \mathbb{N}}|V\left( \left( n_{k} \right)_{k \in \mathbb{N}}^{- 1}|\varLambda_{n} \right) \right) \subseteq \varLambda_{n}
\end{align*}
その集合$V\left( \left( n_{k} \right)_{k \in \mathbb{N}}|V\left( \left( n_{k} \right)_{k \in \mathbb{N}}^{- 1}|\varLambda_{n} \right) \right)$は有限集合である。しかも、その元$n_{k_{0}}$がこれに属するので、空集合でない。これは自然数全体の集合$\mathbb{N}$の部分集合であることから、最大値が存在するので\footnote{これは数学的帰納法によって示すと分かりやすいかもしれない。}、これを$m$とおくと、$\exists k' \in \mathbb{N}$に対し、$n_{k'} \leq n$が成り立つかつ、$m = n_{k'}$が成り立つ。そこで、$\forall k \in \mathbb{N}$に対し、$k' < k$が成り立つなら、$m = n_{k'} < n_{k}$なので、$k \notin V\left( \left( n_{k} \right)_{k \in \mathbb{N}}^{- 1}|\varLambda_{n} \right)$が成り立つことになり、よって、$n_{k'} \leq n < n_{k}$が成り立つ。このような自然数$k$を$k_{0}'$と1つとれば、$n_{k_{0}} < n < n_{k_{0}'}$が成り立つので、$k_{0} < k_{0}'$が得られ、したがって、$n < n_{k_{0}'}$かつ$a_{n_{k_{0}}} \in V$が成り立つ。以上の議論により、その元$a$がその点列$\left( a_{n} \right)_{n \in \mathbb{N}}$の堆積点である。
\end{proof}
\begin{thm}\label{8.1.10.2}
位相空間$\left( S,\mathfrak{O} \right)$が第1可算公理を満たすとき、その集合$S$の元$a$と部分集合$M$について、次のことは同値である。
\begin{itemize}
\item
  $a \in {\mathrm{cl}}M$が成り立つ。
\item
  その元$a$に収束するその集合$M$の点列が存在する。
\end{itemize}
\end{thm}
\begin{proof}
位相空間$\left( S,\mathfrak{O} \right)$が第1可算公理を満たすとき、その集合$S$の元$a$と部分集合$M$について、$a \in {\mathrm{cl}}M$が成り立つとする。定理\ref{8.1.2.17}、即ち、第1可算公理の基本補題よりその全近傍系$\mathbf{V}(a)$のある元の列$\left( U_{n} \right)_{n \in \mathbb{N}}$が存在して、次のことが成り立つ。
\begin{itemize}
\item
  $\forall V \in \mathbf{V}(a)\exists n \in \mathbb{N}$に対し、$U_{n} \subseteq V$が成り立つ。
\item
  $\forall n \in \mathbb{N}$に対し、$U_{n + 1} \subseteq U_{n}$が成り立つ。
\end{itemize}
定理\ref{8.1.1.26}より$\forall n \in \mathbb{N}$に対し、$M \cap U_{n} \neq \emptyset$が成り立つので、これに属する元を$a_{n}$とおけば、その集合$M$の点列$\left( a_{n} \right)_{n \in \mathbb{N}}$が得られる。このとき、$\forall V \in \mathbf{V}(a)\exists n_{0} \in \mathbb{N}\forall n \in \mathbb{N}$に対し、$n_{0} \leq n$が成り立つなら、$U_{n} \subseteq U_{n_{0}} \subseteq V$かつ$a_{n} \in U_{n}$が成り立つので、$a_{n} \in V$が得られ、よって、その集合$M$の点列$\left( a_{n} \right)_{n \in \mathbb{N}}$が存在してその元$a$に収束する。\par
逆に、その元$a$に収束するその集合$M$の点列が存在するなら、$a \in {\mathrm{cl}}(M)$が成り立つことは定理\ref{8.1.9.7}そのものである。
\end{proof}
\begin{thm}\label{8.1.10.3}
2つの位相空間たち$\left( S,\mathfrak{O} \right)$、$\left( T,\mathfrak{P} \right)$とこれらの間の写像$f:S \rightarrow T$が与えられ、さらに、位相空間$\left( S,\mathfrak{O} \right)$が第1可算公理を満たすとき、その集合$S$の元$a$と部分集合$M$について、次のことは同値である。
\begin{itemize}
\item
  その写像$f$はその元$a$で連続である。
\item
  その集合$S$の任意の点列$\left( a_{n} \right)_{n \in \mathbb{N}}$がその元$a$に収束するなら、その集合$T$の点列$\left( f\left( a_{n} \right) \right)_{n \in \mathbb{N}}$もその元$f(a)$に収束する。
\end{itemize}
\end{thm}
\begin{proof}
2つの位相空間たち$\left( S,\mathfrak{O} \right)$、$\left( T,\mathfrak{P} \right)$とこれらの間の写像$f:S \rightarrow T$が与えられ、さらに、位相空間$\left( S,\mathfrak{O} \right)$が第1可算公理を満たすとき、その集合$S$の元$a$と部分集合$M$について、その写像$f$はその元$a$で連続であるなら、その集合$S$の任意の点列$\left( a_{n} \right)_{n \in \mathbb{N}}$がその元$a$に収束するなら、その集合$T$の点列$\left( f\left( a_{n} \right) \right)_{n \in \mathbb{N}}$もその元$f(a)$に収束することは定理\ref{8.1.9.9}そのものである。\par
逆に、その写像$f$がその元$a$で連続でないなら、それらの元々$a$、$f(a)$の全近傍系をそれぞれ$\mathbf{V}(a)$、$\mathbf{W}\left( f(a) \right)$として、定理\ref{8.1.3.1}より$\exists V \in \mathbf{W}\left( f(a) \right)$に対し、$V\left( f^{- 1}|V \right) \in \mathbf{V}(a)$が成り立たない。ここで、定理\ref{8.1.1.22}より$\forall O \in \mathfrak{O}$に対し、$a \in O$が成り立つなら、$O \setminus V\left( f^{- 1}|V \right) \neq \emptyset$が成り立つ。定理\ref{8.1.2.17}、即ち、第1可算公理の基本補題よりその全近傍系$\mathbf{V}(a)$のある元の列$\left( U_{n} \right)_{n \in \mathbb{N}}$が存在して、次のことが成り立ち、
\begin{itemize}
\item
  $\forall V \in \mathbf{V}(a)\exists n \in \mathbb{N}$に対し、$U_{n} \subseteq V$が成り立つ。
\item
  $\forall n \in \mathbb{N}$に対し、$U_{n + 1} \subseteq U_{n}$が成り立つ。
\end{itemize}
したがって、$\forall n \in \mathbb{N}$に対し、${\mathrm{int}}U_{n}\in \mathfrak{O}$かつ$a \in {\mathrm{int}}U_{n}$より$\emptyset \neq {\mathrm{int}}U_{n} \setminus V\left( f^{- 1}|V \right)$が成り立ち${\mathrm{int}}U_{n} \setminus V\left( f^{- 1}|V \right) \subseteq U_{n} \setminus V\left( f^{- 1}|V \right)$が成り立つので、$U_{n} \setminus V\left( f^{- 1}|V \right) \neq \emptyset$が成り立つことになり$a_{n} \in U_{n} \setminus V\left( f^{- 1}|V \right)$なる元$a_{n}$がとられることができその集合$S$の点列$\left( a_{n} \right)_{n \in \mathbb{N}}$が得られる。このとき、$\forall W \in \mathbf{V}(a)\exists n_{0} \in \mathbb{N}\forall n \in \mathbb{N}$に対し、$n_{0} \leq n$が成り立つなら、$U_{n} \subseteq U_{n_{0}} \subseteq W$が成り立つので、$a_{n} \in W$が成り立つ。これはまさしくその点列$\left( a_{n} \right)_{n \in \mathbb{N}}$がほとんどその近傍$U$に属することになる、即ち、その点列$\left( a_{n} \right)_{n \in \mathbb{N}}$がその元$a$に収束する。一方で、$\forall n \in \mathbb{N}$に対し、$a_{n} \in U_{n} \setminus V\left( f^{- 1}|V \right)$より$a_{n} \notin V\left( f^{- 1}|V \right)$が成り立つので、$f\left( a_{n} \right) \notin V$が成り立つ、即ち、$\exists V \in \mathbf{W}\left( f(a) \right)\forall n \in \mathbb{N}\exists n \in \mathbb{N}$に対し、$n \leq n$が成り立つかつ、$f\left( a_{n} \right) \notin V$が成り立つので、その点列$\left( f\left( a_{n} \right) \right)_{n \in \mathbb{N}}$がその元$f(a)$に収束しない。対偶律によりその集合$S$の任意の点列$\left( a_{n} \right)_{n \in \mathbb{N}}$がその元$a$に収束するなら、その集合$T$の点列$\left( f\left( a_{n} \right) \right)_{n \in \mathbb{N}}$もその元$f(a)$に収束するなら、その写像$f:S \rightarrow T$が連続である。
\end{proof}
%\hypertarget{ux7b2c1ux53efux7b97ux516cux7406ux3092ux6e80ux305fux3059compactux7a7aux9593ux3068ux70b9ux5217}{%
\subsubsection{第1可算公理を満たすcompact空間と点列}%\label{ux7b2c1ux53efux7b97ux516cux7406ux3092ux6e80ux305fux3059compactux7a7aux9593ux3068ux70b9ux5217}}
\begin{dfn}
位相空間$\left( S,\mathfrak{O} \right)$が与えられたとき、その台集合$S$の任意の元の列$\left( a_{n} \right)_{n \in \mathbb{N}}$に対し、ある部分列$\left( a_{n_{k}} \right)_{k \in \mathbb{N}}$が存在して、その部分列$\left( a_{n_{k}} \right)_{k \in \mathbb{N}}$が収束するとき、その位相空間$\left( S,\mathfrak{O} \right)$を点列compact空間という。
\end{dfn}
\begin{thm}\label{8.1.10.4}
位相空間$\left( S,\mathfrak{O} \right)$が第1可算公理を満たすとき、その位相空間$\left( S,\mathfrak{O} \right)$がcompact空間であるなら、その位相空間$\left( S,\mathfrak{O} \right)$も点列compact空間である。
\end{thm}
\begin{proof}
位相空間$\left( S,\mathfrak{O} \right)$が第1可算公理を満たすとき、その位相空間$\left( S,\mathfrak{O} \right)$がcompact空間であるなら、その集合$S$の任意の点列$\left( a_{n} \right)_{n \in \mathbb{N}}$に対し、定理\ref{8.1.9.11}よりその点列$\left( a_{n} \right)_{n \in \mathbb{N}}$は堆積点をもつので、これを$a$とおくと、定理\ref{8.1.10.1}よりその点列$\left( a_{n} \right)_{n \in \mathbb{N}}$の部分列でその元$a$に収束するものが存在する。よって、その位相空間$\left( S,\mathfrak{O} \right)$も点列compact空間である。
\end{proof}
\begin{thm}\label{8.1.10.5}
位相空間$\left( S,\mathfrak{O} \right)$がLindelöfの性質をもっているとき、その位相空間$\left( S,\mathfrak{O} \right)$が点列compact空間であるなら、その位相空間$\left( S,\mathfrak{O} \right)$はcompact空間でもある。
\end{thm}
\begin{proof}
位相空間$\left( S,\mathfrak{O} \right)$がLindelöfの性質をもっているとき、その位相空間$\left( S,\mathfrak{O} \right)$が点列compact空間であるかつ、その位相空間$\left( S,\mathfrak{O} \right)$がcompact空間でないと仮定しよう。その集合$S$のある開被覆$\mathfrak{U}$が存在して、これの有限な部分集合をどのようにとってもその集合$S$の開被覆でありえない。しかしながら、その位相空間$\left( S,\mathfrak{O} \right)$はLindelöfの性質をもっているので、その台集合$S$の開被覆$\mathfrak{U}'$が存在して、その開被覆$\mathfrak{U}$の部分集合でたかだか可算である。これ$\mathfrak{U}'$が$\left\{ U_{n} \right\}_{n \in \mathbb{N}}$とおかれれば、その集合$S$の点列$\left( a_{n} \right)_{n \in \mathbb{N}}$で$a_{n} \in S \setminus \bigcup_{k \in \varLambda_{n}} U_{k}$が成り立つようにすると、その位相空間$\left( S,\mathfrak{O} \right)$は点列compact空間なので、その点列$\left( a_{n} \right)_{n \in \mathbb{N}}$のある部分列$\left( a_{n_{k}} \right)_{k \in \mathbb{N}}$が存在して、これが収束する。したがって、定理\ref{8.1.10.1}よりその収束点を$a$とおけば、これはその点列$\left( a_{n} \right)_{n \in \mathbb{N}}$の堆積点である。したがって、$a \in U_{n_{0}}$なる開集合$U_{n_{0}}$がとられれば、$U_{n_{0}} \in \mathbf{V}(a)$が成り立つので、$\exists n_{0}' \in \mathbb{N}$に対し、$n_{0} \leq n_{0}'$が成り立つかつ、$a_{n_{0}'} \in U_{n_{0}}$が成り立つ。したがって、$a_{n_{0}'} \in U_{n_{0}} \subseteq \bigcup_{} U_{k \in \varLambda_{n_{0}'}}$が成り立つことになるが、これはその点列$\left( a_{n} \right)_{n \in \mathbb{N}}$のおき方で$a_{n} \in S \setminus \bigcup_{k \in \varLambda_{n}} U_{k}$が成り立つことに矛盾する。ゆえに、その位相空間$\left( S,\mathfrak{O} \right)$はcompact空間でもある。
\end{proof}
\begin{thm}\label{8.1.10.5s}
位相空間$\left( S,\mathfrak{O} \right)$が第2可算公理を満たすとき、その位相空間$\left( S,\mathfrak{O} \right)$が点列compact空間であるならそのときに限り、その位相空間$\left( S,\mathfrak{O} \right)$はcompact空間である。
\end{thm}
\begin{proof}
位相空間$\left( S,\mathfrak{O} \right)$が第2可算公理を満たすとき、定理\ref{8.1.6.16}よりその位相空間$\left( S,\mathfrak{O} \right)$はLindelöfの性質をもっているので、定理\ref{8.1.10.5}よりその位相空間$\left( S,\mathfrak{O} \right)$が点列compact空間であるなら、その位相空間$\left( S,\mathfrak{O} \right)$もcompact空間である。逆に、その位相空間$\left( S,\mathfrak{O} \right)$がcompact空間であるなら、定理\ref{8.1.2.16}より位相空間$\left( S,\mathfrak{O} \right)$は第1可算公理も満たすので、定理\ref{8.1.10.4}よりその位相空間$\left( S,\mathfrak{O} \right)$も点列compact空間である。
\end{proof}
\begin{thm}\label{8.1.10.6}
位相空間$\left( S,\mathfrak{O} \right)$がLindelöfの性質をもっており第1可算公理を満たすとき、次のことは同値である。
\begin{itemize}
\item
  その位相空間$\left( S,\mathfrak{O} \right)$はcompact空間である。
\item
  その集合$S$の任意の点列は堆積点をもつ。
\item
  その集合$S$の任意の点列に対し、ある部分列が存在して、これが収束する。
\end{itemize}
\end{thm}
\begin{proof}
位相空間$\left( S,\mathfrak{O} \right)$がLindelöfの性質をもっており第1可算公理を満たすとき、その位相空間$\left( S,\mathfrak{O} \right)$がcompact空間であるなら、その集合$S$の任意の点列$\left( a_{n} \right)_{n \in \mathbb{N}}$に対し、定理\ref{8.1.9.11}よりその点列$\left( a_{n} \right)_{n \in \mathbb{N}}$は堆積点をもつ。また、これが成り立つなら、定理\ref{8.1.10.1}よりその集合$S$の任意の点列に対し、ある部分列が存在して、これが収束する。これが成り立つなら、定義より明らかにその位相空間$\left( S,\mathfrak{O} \right)$は点列compact空間であり、さらに、定理\ref{8.1.10.5}よりその位相空間$\left( S,\mathfrak{O} \right)$はcompact空間でもある。
\end{proof}
\begin{thm}\label{8.1.10.7}
位相空間$\left( S,\mathfrak{O} \right)$が第2可算公理を満たすとき、次のことは同値である。
\begin{itemize}
\item
  その位相空間$\left( S,\mathfrak{O} \right)$はcompact空間である。
\item
  その集合$S$の任意の点列は堆積点をもつ。
\item
  その集合$S$の任意の点列に対し、ある部分列が存在して、これが収束する。
\end{itemize}
\end{thm}
\begin{proof}
位相空間$\left( S,\mathfrak{O} \right)$が第2可算公理を満たすとき、これはLindelöfの性質をもっているかつ、第1可算公理を満たすことになるので、定理\ref{8.1.10.6}より次のことは同値である。
\begin{itemize}
\item
  その位相空間$\left( S,\mathfrak{O} \right)$はcompact空間である。
\item
  その集合$S$の任意の点列は堆積点をもつ。
\item
  その集合$S$の任意の点列に対し、ある部分列が存在して、これが収束する。
\end{itemize}
\end{proof}
\begin{thebibliography}{50}
\bibitem{1}
  Mathpedia. "ネットによる位相空間論". Mathpedia. \url{https://math.jp/wiki/%E3%83%8D%E3%83%83%E3%83%88%E3%81%AB%E3%82%88%E3%82%8B%E4%BD%8D%E7%9B%B8%E7%A9%BA%E9%96%93%E8%AB%96} (2022-5-4 1:12 閲覧)
\end{thebibliography}
\end{document}
