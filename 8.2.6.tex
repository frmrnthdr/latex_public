\documentclass[dvipdfmx]{jsarticle}
\setcounter{section}{2}
\setcounter{subsection}{5}
\usepackage{xr}
\externaldocument{8.2.1}
\externaldocument{8.2.4}
\externaldocument{8.2.5}
\usepackage{amsmath,amsfonts,amssymb,array,comment,mathtools,url,docmute}
\usepackage{longtable,booktabs,dcolumn,tabularx,mathtools,multirow,colortbl,xcolor}
\usepackage[dvipdfmx]{graphics}
\usepackage{bmpsize}
\usepackage{amsthm}
\usepackage{enumitem}
\setlistdepth{20}
\renewlist{itemize}{itemize}{20}
\setlist[itemize]{label=•}
\renewlist{enumerate}{enumerate}{20}
\setlist[enumerate]{label=\arabic*.}
\setcounter{MaxMatrixCols}{20}
\setcounter{tocdepth}{3}
\newcommand{\rotin}{\text{\rotatebox[origin=c]{90}{$\in $}}}
\newcommand{\amap}[6]{\text{\raisebox{-0.7cm}{\begin{tikzpicture} 
  \node (a) at (0, 1) {$\textstyle{#2}$};
  \node (b) at (#6, 1) {$\textstyle{#3}$};
  \node (c) at (0, 0) {$\textstyle{#4}$};
  \node (d) at (#6, 0) {$\textstyle{#5}$};
  \node (x) at (0, 0.5) {$\rotin $};
  \node (x) at (#6, 0.5) {$\rotin $};
  \draw[->] (a) to node[xshift=0pt, yshift=7pt] {$\textstyle{\scriptstyle{#1}}$} (b);
  \draw[|->] (c) to node[xshift=0pt, yshift=7pt] {$\textstyle{\scriptstyle{#1}}$} (d);
\end{tikzpicture}}}}
\newcommand{\twomaps}[9]{\text{\raisebox{-0.7cm}{\begin{tikzpicture} 
  \node (a) at (0, 1) {$\textstyle{#3}$};
  \node (b) at (#9, 1) {$\textstyle{#4}$};
  \node (c) at (#9+#9, 1) {$\textstyle{#5}$};
  \node (d) at (0, 0) {$\textstyle{#6}$};
  \node (e) at (#9, 0) {$\textstyle{#7}$};
  \node (f) at (#9+#9, 0) {$\textstyle{#8}$};
  \node (x) at (0, 0.5) {$\rotin $};
  \node (x) at (#9, 0.5) {$\rotin $};
  \node (x) at (#9+#9, 0.5) {$\rotin $};
  \draw[->] (a) to node[xshift=0pt, yshift=7pt] {$\textstyle{\scriptstyle{#1}}$} (b);
  \draw[|->] (d) to node[xshift=0pt, yshift=7pt] {$\textstyle{\scriptstyle{#2}}$} (e);
  \draw[->] (b) to node[xshift=0pt, yshift=7pt] {$\textstyle{\scriptstyle{#1}}$} (c);
  \draw[|->] (e) to node[xshift=0pt, yshift=7pt] {$\textstyle{\scriptstyle{#2}}$} (f);
\end{tikzpicture}}}}
\renewcommand{\thesection}{第\arabic{section}部}
\renewcommand{\thesubsection}{\arabic{section}.\arabic{subsection}}
\renewcommand{\thesubsubsection}{\arabic{section}.\arabic{subsection}.\arabic{subsubsection}}
\everymath{\displaystyle}
\allowdisplaybreaks[4]
\usepackage{vtable}
\theoremstyle{definition}
\newtheorem{thm}{定理}[subsection]
\newtheorem*{thm*}{定理}
\newtheorem{dfn}{定義}[subsection]
\newtheorem*{dfn*}{定義}
\newtheorem{axs}[dfn]{公理}
\newtheorem*{axs*}{公理}
\renewcommand{\headfont}{\bfseries}
\makeatletter
  \renewcommand{\section}{%
    \@startsection{section}{1}{\z@}%
    {\Cvs}{\Cvs}%
    {\normalfont\huge\headfont\raggedright}}
\makeatother
\makeatletter
  \renewcommand{\subsection}{%
    \@startsection{subsection}{2}{\z@}%
    {0.5\Cvs}{0.5\Cvs}%
    {\normalfont\LARGE\headfont\raggedright}}
\makeatother
\makeatletter
  \renewcommand{\subsubsection}{%
    \@startsection{subsubsection}{3}{\z@}%
    {0.4\Cvs}{0.4\Cvs}%
    {\normalfont\Large\headfont\raggedright}}
\makeatother
\makeatletter
\renewenvironment{proof}[1][\proofname]{\par
  \pushQED{\qed}%
  \normalfont \topsep6\p@\@plus6\p@\relax
  \trivlist
  \item\relax
  {
  #1\@addpunct{.}}\hspace\labelsep\ignorespaces
}{%
  \popQED\endtrivlist\@endpefalse
}
\makeatother
\renewcommand{\proofname}{\textbf{証明}}
\usepackage{tikz,graphics}
\usepackage[dvipdfmx]{hyperref}
\usepackage{pxjahyper}
\hypersetup{
 setpagesize=false,
 bookmarks=true,
 bookmarksdepth=tocdepth,
 bookmarksnumbered=true,
 colorlinks=false,
 pdftitle={},
 pdfsubject={},
 pdfauthor={},
 pdfkeywords={}}
\begin{document}
%\hypertarget{ux5b8cux5099ux5316}{%
\subsection{完備化}%\label{ux5b8cux5099ux5316}}
%\hypertarget{ux5b8cux5099ux5316-1}{%
\subsubsection{完備化}%\label{ux5b8cux5099ux5316-1}}
\begin{thm}\label{8.2.6.1}
距離空間$(S,d)$が与えられたとき、この距離空間$(S,d)$におけるCauchy列$\left( a_{n} \right)_{n \in \mathbb{N}}$全体の集合が$C(S,d)$とおかれると、$\forall\left( a_{n} \right)_{n \in \mathbb{N}},\left( b_{n} \right)_{n \in \mathbb{N}} \in C(S,d)$に対し、実数列$\left( d\left( a_{n},b_{n} \right) \right)_{n \in \mathbb{N}}$は収束する。
\end{thm}
\begin{proof}
距離空間$(S,d)$が与えられたとき、この距離空間$(S,d)$におけるCauchy列$\left( a_{n} \right)_{n \in \mathbb{N}}$全体の集合が$C(S,d)$とおかれると、$\forall\left( a_{n} \right)_{n \in \mathbb{N}},\left( b_{n} \right)_{n \in \mathbb{N}} \in C(S,d)$に対し、実数列$\left( d\left( a_{n},b_{n} \right) \right)_{n \in \mathbb{N}}$が考えられれば、$\forall m,n \in \mathbb{N}$に対し、次式が成り立つ。
\begin{align*}
d\left( a_{m},b_{m} \right) - d\left( a_{n},b_{n} \right) &\leq d\left( a_{m},a_{n} \right) + d\left( a_{n},b_{n} \right) + d\left( b_{n},b_{m} \right) - d\left( a_{n},b_{n} \right)\\
&= d\left( a_{m},a_{n} \right) + d\left( b_{m},b_{n} \right)
\end{align*}
同様にして、次式が成り立つので、
\begin{align*}
d\left( a_{n},b_{n} \right) - d\left( a_{m},b_{m} \right) \leq d\left( a_{m},a_{n} \right) + d\left( b_{m},b_{n} \right)
\end{align*}
したがって、次式が得られる。
\begin{align*}
\left| d\left( a_{n},b_{n} \right) - d\left( a_{m},b_{m} \right) \right| \leq d\left( a_{m},a_{n} \right) + d\left( b_{m},b_{n} \right)
\end{align*}
ここで、それらの元の列たち$\left( a_{n} \right)_{n \in \mathbb{N}}$、$\left( b_{n} \right)_{n \in \mathbb{N}}$はいづれもCauchy列であるから、$\forall\varepsilon \in \mathbb{R}^{+}$に対し、ある自然数$n_{0}$が存在して、任意の自然数たち$m$、$n$に対し、$n_{0} < m$、$n_{0} < n$が成り立つなら、$d\left( a_{m},a_{n} \right) < \varepsilon$かつ$d\left( b_{m},b_{n} \right) < \varepsilon$が成り立つので、次式のようになる。
\begin{align*}
\left| d\left( a_{n},b_{n} \right) - d\left( a_{m},b_{m} \right) \right| &\leq d\left( a_{m},a_{n} \right) + d\left( b_{m},b_{n} \right)\\
&< \varepsilon + \varepsilon = 2\varepsilon
\end{align*}
これにより、その実数列$\left( d\left( a_{n},b_{n} \right) \right)_{n \in \mathbb{N}}$はCauchy列であるから、定理\ref{8.2.5.9}よりその実数列$\left( d\left( a_{n},b_{n} \right) \right)_{n \in \mathbb{N}}$は収束する。
\end{proof}
\begin{thm}\label{8.2.6.2}
距離空間$(S,d)$が与えられたとき、この距離空間$(S,d)$におけるCauchy列$\left( a_{n} \right)_{n \in \mathbb{N}}$全体の集合$C(S,d)$の元々$\left( a_{n} \right)_{n \in \mathbb{N}}$、$\left( b_{n} \right)_{n \in \mathbb{N}}$に対し、$\lim_{n \rightarrow \infty}{d\left( a_{n},b_{n} \right)} = 0$が成り立つことを$\left( a_{n} \right)_{n \in \mathbb{N}}R_{C(S,d)}\left( b_{n} \right)_{n \in \mathbb{N}}$とおかれると、その関係$R_{C(S,d)}$はその集合$C(S,d)$における同値関係となる。
\end{thm}
\begin{proof}
距離空間$(S,d)$が与えられたとき、この距離空間$(S,d)$におけるCauchy列$\left( a_{n} \right)_{n \in \mathbb{N}}$全体の集合$C(S,d)$の元々$\left( a_{n} \right)_{n \in \mathbb{N}}$、$\left( b_{n} \right)_{n \in \mathbb{N}}$に対し、$\lim_{n \rightarrow \infty}{d\left( a_{n},b_{n} \right)} = 0$が成り立つことを$\left( a_{n} \right)_{n \in \mathbb{N}}R_{C(S,d)}\left( b_{n} \right)_{n \in \mathbb{N}}$とおかれるとする。\par
$\forall\left( a_{n} \right)_{n \in \mathbb{N}} \in C(S,d)$に対し、その元の列$\left( a_{n} \right)_{n \in \mathbb{N}}$はCauchy列であるから、$\forall\varepsilon \in \mathbb{R}^{+}\exists n_{0} \in \mathbb{N}\forall n \in \mathbb{N}$に対し、$n_{0} < n$が成り立つなら、$d\left( a_{n},a_{n} \right) < \varepsilon$が成り立つので、$\lim_{n \rightarrow \infty}{d\left( a_{n},a_{n} \right)} = 0$が成り立つ。ゆえに、$\left( a_{n} \right)_{n \in \mathbb{N}}R_{C(S,d)}\left( a_{n} \right)_{n \in \mathbb{N}}$が成り立つ。\par
$\forall\left( a_{n} \right)_{n \in \mathbb{N}},\left( b_{n} \right)_{n \in \mathbb{N}} \in C(S,d)$に対し、$\left( a_{n} \right)_{n \in \mathbb{N}}R_{C(S,d)}\left( b_{n} \right)_{n \in \mathbb{N}}$が成り立つなら、$\lim_{n \rightarrow \infty}{d\left( a_{n},\ \ b_{n} \right)} = 0$が成り立つ。ここで、距離空間の定義より$\lim_{n \rightarrow \infty}{d\left( b_{n},a_{n} \right)} = 0$が成り立つので、$\left( b_{n} \right)_{n \in \mathbb{N}}R_{C(S,d)}\left( a_{n} \right)_{n \in \mathbb{N}}$が成り立つ。\par
$\forall\left( a_{n} \right)_{n \in \mathbb{N}},\left( b_{n} \right)_{n \in \mathbb{N}},\left( c_{n} \right)_{n \in \mathbb{N}} \in C(S,d)$に対し、$\left( a_{n} \right)_{n \in \mathbb{N}}R_{C(S,d)}\left( b_{n} \right)_{n \in \mathbb{N}}$かつ$\left( b_{n} \right)_{n \in \mathbb{N}}R_{C(S,d)}\left( c_{n} \right)_{n \in \mathbb{N}}$が成り立つなら、$\lim_{n \rightarrow \infty}{d\left( a_{n},b_{n} \right)} = 0$かつ$\lim_{n \rightarrow \infty}{d\left( b_{n},c_{n} \right)} = 0$が成り立つ、即ち、$\forall\varepsilon \in \mathbb{R}^{+}\exists n_{0} \in \mathbb{N}\forall n \in \mathbb{N}$に対し、$n_{0} < n$が成り立つなら、$d\left( a_{n},b_{n} \right) < \varepsilon$かつ$d\left( b_{n},c_{n} \right) < \varepsilon$が成り立つ。したがって、次のようになる。
\begin{align*}
d\left( a_{n},c_{n} \right) \leq d\left( a_{n},b_{n} \right) + d\left( b_{n},c_{n} \right) < 2\varepsilon
\end{align*}
これにより、$\lim_{n \rightarrow \infty}{d\left( a_{n},c_{n} \right)} = 0$が成り立つので、$\left( a_{n} \right)_{n \in \mathbb{N}}R_{C(S,d)}\left( c_{n} \right)_{n \in \mathbb{N}}$が成り立つ。\par
以上より、その関係$R_{C(S,d)}$はその集合$C(S,d)$における同値関係となる。
\end{proof}
\begin{thm}\label{8.2.6.3}
距離空間$(S,d)$が与えられたとき、上で定義された集合$C(S,d)$とこれにおける同値関係$R_{C(S,d)}$において、その商集合${C(S,d)}/{R_{C(S,d)}}$が$S^{*}$と、その集合$C(S,d)$からその商集合$S^{*}$への自然な写像$C_{R_{C(R,d)}}$が$\pi$とおかれると、$\forall\left( a_{n} \right)_{n \in \mathbb{N}},\left( b_{n} \right)_{n \in \mathbb{N}},\left( a_{n}' \right)_{n \in \mathbb{N}},\left( b_{n}' \right)_{n \in \mathbb{N}} \in C(S,d)$に対し、次式が成り立つなら、
\begin{align*}
\pi\left( a_{n} \right)_{n \in \mathbb{N}} = \pi\left( a_{n}' \right)_{n \in \mathbb{N}},\ \ \pi\left( b_{n} \right)_{n \in \mathbb{N}} = \pi\left( b_{n}' \right)_{n \in \mathbb{N}}
\end{align*}
次式が成り立つ。
\begin{align*}
\lim_{n \rightarrow \infty}{d\left( a_{n},b_{n} \right)} = \lim_{n \rightarrow \infty}{d\left( a_{n}',b_{n}' \right)}
\end{align*}
これにより、次式のような写像$d^{*}$が定義されることができる。
\begin{align*}
d^{*}:S^{*} \times S^{*} \rightarrow \mathbb{R};\left( \pi\left( a_{n} \right)_{n \in \mathbb{N}},\pi\left( b_{n} \right)_{n \in \mathbb{N}} \right) \mapsto \lim_{n \rightarrow \infty}{d\left( a_{n},b_{n} \right)}
\end{align*}
\end{thm}
\begin{proof}
距離空間$(S,d)$が与えられたとき、上で定義された集合$C(S,d)$とこれにおける同値関係$R_{C(S,d)}$において、その商集合${C(S,d)}/{R_{C(S,d)}}$が$S^{*}$と、その集合$C(S,d)$からその商集合$S^{*}$への自然な写像$C_{R_{C(R,d)}}$が$\pi$とおかれると、$\forall\left( a_{n} \right)_{n \in \mathbb{N}},\left( b_{n} \right)_{n \in \mathbb{N}},\left( a_{n}' \right)_{n \in \mathbb{N}},\left( b_{n}' \right)_{n \in \mathbb{N}} \in C(S,d)$に対し、次式が成り立つなら、
\begin{align*}
\pi\left( a_{n} \right)_{n \in \mathbb{N}} = \pi\left( a_{n}' \right)_{n \in \mathbb{N}},\ \ \pi\left( b_{n} \right)_{n \in \mathbb{N}} = \pi\left( b_{n}' \right)_{n \in \mathbb{N}}
\end{align*}
定義よりもちろん次式が成り立つ、
\begin{align*}
\left( a_{n} \right)_{n \in \mathbb{N}}R_{C(S,d)}\left( a_{n}' \right)_{n \in \mathbb{N}},\ \ \left( b_{n} \right)_{n \in \mathbb{N}}R_{C(S,d)}\left( b_{n}' \right)_{n \in \mathbb{N}}
\end{align*}
即ち、次式が成り立つ。
\begin{align*}
\lim_{n \rightarrow \infty}{d\left( a_{n},a_{n}' \right)} = 0,\ \ \lim_{n \rightarrow \infty}{d\left( b_{n},b_{n}' \right)} = 0
\end{align*}
ここで、次のようになることから、
\begin{align*}
d\left( a_{n},b_{n} \right) \leq d\left( a_{n},a_{n}' \right) + d\left( a_{n}',b_{n}' \right) + d\left( b_{n}',b_{n} \right)
\end{align*}
したがって、次のようになる\footnote{ここで、「ねぇ、これって解析学でやった定理だよね? これって距離空間に対して成り立つの? 」と気になるかもしれませんが、いま考えている元の列はただの実数列になっておりそのまま使えます! ですので、ご安心ください…。}。
\begin{align*}
\lim_{n \rightarrow \infty}{d\left( a_{n},b_{n} \right)} &\leq \lim_{n \rightarrow \infty}\left( d\left( a_{n},a_{n}' \right) + d\left( a_{n}',b_{n}' \right) + d\left( b_{n}',b_{n} \right) \right)\\
&= \lim_{n \rightarrow \infty}{d\left( a_{n},a_{n}' \right)} + \lim_{n \rightarrow \infty}{d\left( a_{n}',b_{n}' \right)} + \lim_{n \rightarrow \infty}{d\left( b_{n}',b_{n} \right)}\\
&= 0 + \lim_{n \rightarrow \infty}{d\left( a_{n}',b_{n}' \right)} + 0\\
&= \lim_{n \rightarrow \infty}{d\left( a_{n}',b_{n}' \right)}
\end{align*}
同様にして$\lim_{n \rightarrow \infty}{d\left( a_{n}',b_{n}' \right)} \leq \lim_{n \rightarrow \infty}{d\left( a_{n},b_{n} \right)}$が成り立つので、次式が成り立つなら、
\begin{align*}
\pi\left( a_{n} \right)_{n \in \mathbb{N}} = \pi\left( a_{n}' \right)_{n \in \mathbb{N}},\ \ \pi\left( b_{n} \right)_{n \in \mathbb{N}} = \pi\left( b_{n}' \right)_{n \in \mathbb{N}}
\end{align*}
次式が成り立つ。
\begin{align*}
\lim_{n \rightarrow \infty}{d\left( a_{n},b_{n} \right)} = \lim_{n \rightarrow \infty}{d\left( a_{n}',b_{n}' \right)}
\end{align*}
ことが示された。
\end{proof}
\begin{thm}\label{8.2.6.4}
距離空間$(S,d)$が与えられたとき、上で次式のように定義された写像$d^{*}$を用いた組$\left( S^{*},d^{*} \right)$は距離空間をなす。
\begin{align*}
d^{*}:S^{*} \times S^{*} \rightarrow \mathbb{R};\left( \pi\left( a_{n} \right)_{n \in \mathbb{N}},\pi\left( b_{n} \right)_{n \in \mathbb{N}} \right) \mapsto \lim_{n \rightarrow \infty}{d\left( a_{n},b_{n} \right)}
\end{align*}
\end{thm}
\begin{proof}
距離空間$(S,d)$が与えられたとき、定理\ref{8.2.6.3}で次式のように定義された写像$d^{*}$を用いた組$\left( S^{*},d^{*} \right)$について、
\begin{align*}
d^{*}:S^{*} \times S^{*} \rightarrow \mathbb{R};\left( \pi\left( a_{n} \right)_{n \in \mathbb{N}},\pi\left( b_{n} \right)_{n \in \mathbb{N}} \right) \mapsto \lim_{n \rightarrow \infty}{d\left( a_{n},b_{n} \right)}
\end{align*}
$\forall\pi\left( a_{n} \right)_{n \in \mathbb{N}},\pi\left( b_{n} \right)_{n \in \mathbb{N}} \in S^{*}$に対し、$d^{*}\left( \pi\left( a_{n} \right)_{n \in \mathbb{N}},\pi\left( b_{n} \right)_{n \in \mathbb{N}} \right) = 0$が成り立つならそのときに限り、$\lim_{n \rightarrow \infty}{d\left( a_{n},b_{n} \right)} = 0$が成り立つ。これにより、$\left( a_{n} \right)_{n \in \mathbb{N}}R_{C(S,d)}\left( b_{n} \right)_{n \in \mathbb{N}}$が成り立つので、$\pi\left( a_{n} \right)_{n \in \mathbb{N}} = \pi\left( b_{n} \right)_{n \in \mathbb{N}}$が得られる。逆も同様にして示される。\par
$\forall\pi\left( a_{n} \right)_{n \in \mathbb{N}},\pi\left( b_{n} \right)_{n \in \mathbb{N}} \in S^{*}$に対し、$\lim_{n \rightarrow \infty}{d\left( a_{n},b_{n} \right)} = \lim_{n \rightarrow \infty}{d\left( b_{n},a_{n} \right)}$が成り立つので、$d^{*}\left( \pi\left( a_{n} \right)_{n \in \mathbb{N}},\pi\left( b_{n} \right)_{n \in \mathbb{N}} \right) = d^{*}\left( \pi\left( b_{n} \right)_{n \in \mathbb{N}},\pi\left( a_{n} \right)_{n \in \mathbb{N}} \right)$が成り立つ。\par
$\forall\pi\left( a_{n} \right)_{n \in \mathbb{N}},\pi\left( b_{n} \right)_{n \in \mathbb{N}},\pi\left( c_{n} \right)_{n \in \mathbb{N}} \in S^{*}$に対し、次のようになることから、
\begin{align*}
d\left( a_{n},c_{n} \right) \leq d\left( a_{n},b_{n} \right) + d\left( b_{n},c_{n} \right)
\end{align*}
両辺に$n \rightarrow \infty$とすれば、次のようになる。
\begin{align*}
\lim_{n \rightarrow \infty}{d\left( a_{n},c_{n} \right)} \leq \lim_{n \rightarrow \infty}{d\left( a_{n},b_{n} \right)} + \lim_{n \rightarrow \infty}{d\left( b_{n},c_{n} \right)}
\end{align*}
よって、次式が成り立つ。
\begin{align*}
d^{*}\left( \pi\left( a_{n} \right)_{n \in \mathbb{N}},\pi\left( c_{n} \right)_{n \in \mathbb{N}} \right) \leq d^{*}\left( \pi\left( a_{n} \right)_{n \in \mathbb{N}},\pi\left( b_{n} \right)_{n \in \mathbb{N}} \right) + d^{*}\left( \pi\left( b_{n} \right)_{n \in \mathbb{N}},\pi\left( c_{n} \right)_{n \in \mathbb{N}} \right)
\end{align*}\par
以上より、その組$\left( S^{*},d^{*} \right)$は距離空間をなす。
\end{proof}
\begin{thm}\label{8.2.6.5}
距離空間$(S,d)$が与えられたとき、$\forall a \in S$に対し、元の列$(a)_{n \in \mathbb{N}}$は明らかに$(a)_{n \in \mathbb{N}} \in C(S,d)$を満たす。これにより、上で定義された写像$\pi$を用いて次式のように写像$\varphi$が定義されれば、
\begin{align*}
\varphi:S \rightarrow S^{*};a \mapsto \pi(a)_{n \in \mathbb{N}}
\end{align*}
次のことが成り立つ。
\begin{itemize}
\item
  その距離空間$\left( S^{*},d^{*} \right)$は完備である。
\item
  $\forall a,b \in S$に対し、$d(a,b) = d^{*}\left( \varphi(a),\varphi(b) \right)$が成り立つ。
\item
  その写像$\varphi$の値域$V(\varphi)$はその台集合$S^{*}$において密である、即ち、${\mathrm{cl}}{V(\varphi)} = S^{*}$が成り立つ。
\end{itemize}
\end{thm}
\begin{proof}
距離空間$(S,d)$が与えられたとき、$\forall a \in S$に対し、元の列$(a)_{n \in \mathbb{N}}$は明らかに$(a)_{n \in \mathbb{N}} \in C(S,d)$を満たす。これにより、上で定義された写像$\pi$を用いて次式のように写像$\varphi$が定義されれば、
\begin{align*}
\varphi:S \rightarrow S^{*};a \mapsto \pi(a)_{n \in \mathbb{N}}
\end{align*}
$\forall a,b \in S$に対し、明らかに次のようになる。
\begin{align*}
d(a,b) &= \lim_{n \rightarrow \infty}{d(a,b)}\\
&= d^{*}\left( \pi(a)_{n \in \mathbb{N}},\pi(b)_{n \in \mathbb{N}} \right)\\
&= d^{*}\left( \varphi(a),\varphi(b) \right)
\end{align*}\par
また、$\forall\pi\left( a_{n} \right)_{n \in \mathbb{N}} \in S^{*}$に対し、距離空間$\left( S^{*},d^{*} \right)$において、次のようになる。
\begin{align*}
d^{*}\left( \pi\left( a_{m} \right)_{m \in \mathbb{N}},\varphi\left( a_{n} \right) \right) = d^{*}\left( \pi\left( a_{m} \right)_{m \in \mathbb{N}},\pi\left( a_{n} \right)_{m \in \mathbb{N}} \right) = \lim_{m \rightarrow \infty}{d\left( a_{m},a_{n} \right)}
\end{align*}
ここで、$\left( a_{n} \right)_{n \in \mathbb{N}} \in C(S,d)$が成り立つので、$\forall\varepsilon \in \mathbb{R}^{+}\exists\delta \in \mathbb{N}\forall m,n \in \mathbb{N}$に対し、$\delta \leq m$かつ$\delta \leq n$が成り立つなら、$d\left( a_{m},a_{n} \right) < \varepsilon$が成り立つ。したがって、$\forall\varepsilon \in \mathbb{R}^{+}\exists\delta \in \mathbb{N}\forall n \in \mathbb{N}$に対し、$\delta \leq n$が成り立つなら、$\lim_{m \rightarrow \infty}{d\left( a_{m},a_{n} \right)} < \varepsilon$が成り立つので、次式が得られる。
\begin{align*}
\lim_{n \rightarrow \infty}{d^{*}\left( \pi\left( a_{m} \right)_{m \in \mathbb{N}},\varphi\left( a_{n} \right) \right)} = \lim_{n \rightarrow \infty}{\lim_{m \rightarrow \infty}{d\left( a_{m},a_{n} \right)}} = 0
\end{align*}
定理\ref{8.2.1.12}より$\pi\left( a_{n} \right)_{n \in \mathbb{N}} = \pi\left( a_{m} \right)_{m \in \mathbb{N}} = \lim_{n \rightarrow \infty}{\varphi\left( a_{n} \right)}$が成り立つ。\par
さらに、定理\ref{8.2.1.13}よりその距離空間$\left( S^{*},d^{*} \right)$における位相空間$\left( S^{*},\mathfrak{O}_{d^{*}} \right)$において、$\pi\left( a_{m} \right)_{m \in \mathbb{N}} \in {\mathrm{cl}}{V(\varphi)}$が成り立つ。以上より、$S^{*} = {\mathrm{cl}}{V(\varphi)}$が得られたので、その写像$\varphi$の値域$V(\varphi)$はその台集合$S^{*}$において密である。\par
これにより、その距離空間$\left( S^{*},d^{*} \right)$の任意のCauchy列$\left( a_{n}^{*} \right)_{n \in \mathbb{N}}$がとられれば、$\forall n \in \mathbb{N}$に対し、$a_{n}^{*} \in S^{*} = {\mathrm{cl}}{V(\varphi)}$が成り立つので、定理\ref{8.2.1.13}よりあるその集合$S$の元の列$\left( a_{n}' \right)_{n \in \mathbb{N}}$が存在して$a_{n}^{*} = \lim_{n \rightarrow \infty}{\varphi\left( a_{n}' \right)}$が成り立つ。したがって、$\exists\delta \in \mathbb{N}\forall m \in \mathbb{N}$に対し、$\delta \leq m$が成り立つなら、$d^{*}\left( a_{n}^{*},\varphi\left( a_{m}' \right) \right) < \frac{1}{n}$が成り立つので、$a_{\delta}' = a_{n}$とおくと、$\exists a_{n} \in S$に対し、$d^{*}\left( a_{n}^{*},\varphi\left( a_{n} \right) \right) < \frac{1}{n}$が成り立つ。このとき、$\forall\varepsilon \in \mathbb{R}^{+}\exists\delta \in \mathbb{N}\forall m,n \in \mathbb{N}$に対し、$\delta \leq m$かつ$\delta \leq n$が成り立つなら、その元の列$\left( a_{n}^{*} \right)_{n \in \mathbb{N}}$は仮定よりCauchy列であるから、$d^{*}\left( a_{m}^{*},a_{n}^{*} \right) < \varepsilon$が成り立つ。さらに、$0 < \frac{1}{m} < \varepsilon$かつ$0 < \frac{1}{n} < \varepsilon$が成り立つようにとられればよいので、したがって、上記の議論により次式が成り立つことから、
\begin{align*}
d\left( a_{m},a_{n} \right) &= d^{*}\left( \varphi\left( a_{m} \right),\varphi\left( a_{n} \right) \right)\\
&\leq d^{*}\left( \varphi\left( a_{m} \right),a_{m}^{*} \right) + d^{*}\left( a_{m}^{*},a_{n}^{*} \right) + d^{*}\left( a_{n}^{*},\varphi\left( a_{n} \right) \right)\\
&< \varepsilon + \varepsilon + \varepsilon = 3\varepsilon
\end{align*}
その元の列$\left( a_{n} \right)_{n \in \mathbb{N}}$はCauchy列である、即ち、$\left( a_{n} \right)_{n \in \mathbb{N}} \in C(S,d)$が成り立つ。したがって、上記の議論により$\pi\left( a_{n} \right)_{n \in \mathbb{N}} = \lim_{n \rightarrow \infty}{\varphi\left( a_{n} \right)}$が成り立つ、即ち、定理\ref{8.2.1.12}より$\lim_{n \rightarrow \infty}{d^{*}\left( \pi\left( a_{m} \right)_{m \in \mathbb{N}},\varphi\left( a_{n} \right) \right)} = 0$が成り立つ。したがって、次式が成り立つことから、
\begin{align*}
0 &\leq d^{*}\left( a_{n}^{*},\pi\left( a_{m} \right)_{m \in \mathbb{N}} \right)\\
&\leq d^{*}\left( \varphi\left( a_{n} \right),a_{n}^{*} \right) + d^{*}\left( \pi\left( a_{m} \right)_{m \in \mathbb{N}},\varphi\left( a_{n} \right) \right)
\end{align*}
$n \rightarrow \infty$とすれば、はさみうちの原理より$\lim_{n \rightarrow \infty}{d^{*}\left( a_{n}^{*},\pi\left( a_{m} \right)_{m \in \mathbb{N}} \right)} = 0$が成り立つ、即ち、定理\ref{8.2.1.12}より$\lim_{n \rightarrow \infty}a_{n}^{*} = \pi\left( a_{n} \right)_{n \in \mathbb{N}}$が成り立つ、即ち、その元の列$\left( a_{n}^{*} \right)_{n \in \mathbb{N}}$は収束するので、その距離空間$\left( S^{*},d^{*} \right)$は完備である。
\end{proof}
\begin{thm}\label{8.2.6.6}
距離空間$(S,d)$が与えられたとき、次のことを満たす距離空間$\left( S^{*},d^{*} \right)$と写像$\varphi:S \rightarrow S^{*}$が存在するのであった。
\begin{itemize}
\item
  その距離空間$\left( S^{*},d^{*} \right)$は完備である。
\item
  $\forall a,b \in S$に対し、$d(a,b) = d^{*}\left( \varphi(a),\varphi(b) \right)$が成り立つ。
\item
  その写像$\varphi$の値域$V(\varphi)$はその台集合$S^{*}$において密である、即ち、${\mathrm{cl}}{V(\varphi)} = S^{*}$が成り立つ。
\end{itemize}
このとき、そのような距離空間$\left( S^{*},d^{*} \right)$と写像$\varphi$の組が$\left( \left( S^{*},d^{*} \right),\varphi \right)$、$\left( \left( T^{*},e^{*} \right),\chi \right)$と2つ与えられたとき、ある全単射$f:S^{*} \rightarrow T^{*}$が存在して、次のことを満たす。
\begin{itemize}
\item
  $f \circ \varphi = \chi$が成り立つ。
\item
  $\forall a,b \in S^{*}$に対し、$d^{*}(a,b) = e^{*}\left( f(a),f(b) \right)$が成り立つ。
\end{itemize}
\end{thm}
\begin{proof}
距離空間$(S,d)$が与えられたとき、次のことを満たす距離空間$\left( S^{*},d^{*} \right)$と写像$\varphi:S \rightarrow S^{*}$が存在するのであった。
\begin{itemize}
\item
  その距離空間$\left( S^{*},d^{*} \right)$は完備である。
\item
  $\forall a,b \in S$に対し、$d(a,b) = d^{*}\left( \varphi(a),\varphi(b) \right)$が成り立つ。
\item
  その写像$\varphi$の値域$V(\varphi)$はその台集合$S^{*}$において密である、即ち、${\mathrm{cl}}{V(\varphi)} = S^{*}$が成り立つ。
\end{itemize}
このとき、そのような距離空間$\left( S^{*},d^{*} \right)$と写像$\varphi$の組が$\left( \left( S^{*},d^{*} \right),\varphi \right)$、$\left( \left( T^{*},e^{*} \right),\chi \right)$と2つ与えられたとき、$\forall a^{*} \in S^{*}$に対し、$a^{*} \in {\mathrm{cl}}{V(\varphi)}$が成り立つので、定理\ref{8.2.1.13}よりあるその集合$S$の元の列$\left( a_{n} \right)_{n \in \mathbb{N}}$が存在して、$a^{*} = \lim_{n \rightarrow \infty}{\varphi\left( a_{n} \right)}$が成り立つ。ここで、その集合$T^{*}$における元の列$\left( \chi\left( a_{n} \right) \right)_{n \in \mathbb{N}}$において、$\forall\varepsilon \in \mathbb{R}^{+}\exists\delta \in \mathbb{N}\forall m,n \in \mathbb{N}$に対し、$\delta \leq m$かつ$\delta \leq n$が成り立つなら、定理\ref{8.2.4.7}よりその元の列$\left( \varphi\left( a_{n} \right) \right)_{n \in \mathbb{N}}$はCauchy列となり、したがって、$d^{*}\left( \varphi\left( a_{m} \right),\varphi\left( a_{n} \right) \right) < \varepsilon$が成り立つ。ここで、上記の議論により次のようになる。
\begin{align*}
e^{*}\left( \chi\left( a_{m} \right),\chi\left( a_{n} \right) \right) &= d\left( a_{m},a_{n} \right)\\
&= d^{*}\left( \varphi\left( a_{m} \right),\varphi\left( a_{n} \right) \right) < \varepsilon
\end{align*}
ゆえに、その元の列$\left( \chi\left( a_{n} \right) \right)_{n \in \mathbb{N}}$はCauchy列である。ここで、その距離空間$\left( T^{*},e^{*} \right)$は完備であるので、その元の列$\left( \chi\left( a_{n} \right) \right)_{n \in \mathbb{N}}$は収束することになり、その極限点を$b^{*}$とおく。ここで、このような元の列$\left( a_{n} \right)_{n \in \mathbb{N}}$が$\left( a_{n} \right)_{n \in \mathbb{N}}$、$\left( b_{n} \right)_{n \in \mathbb{N}}$と2つと与えられたとき、$\forall\varepsilon \in \mathbb{R}^{+}\exists\delta \in \mathbb{N}\forall n \in \mathbb{N}$に対し、$\delta \leq n$が成り立つなら、$e^{*}\left( \chi\left( a_{n} \right),b^{*} \right) < \varepsilon$が成り立つ。したがって、次のようになるので、
\begin{align*}
e^{*}\left( \chi\left( b_{n} \right),b^{*} \right) &\leq e^{*}\left( \chi\left( a_{n} \right),\chi\left( b_{n} \right) \right) + e^{*}\left( \chi\left( a_{n} \right),b^{*} \right)\\
&= d\left( a_{n},b_{n} \right) + e^{*}\left( \chi\left( a_{n} \right),b^{*} \right)\\
&= d^{*}\left( \varphi\left( a_{n} \right),\varphi\left( b_{n} \right) \right) + e^{*}\left( \chi\left( a_{n} \right),b^{*} \right)\\
&= d^{*}\left( \varphi\left( a_{n} \right),\varphi\left( b_{n} \right) \right) + e^{*}\left( \chi\left( a_{n} \right),b^{*} \right)\\
&\leq d^{*}\left( \varphi\left( a_{n} \right),a^{*} \right) + d^{*}\left( \varphi\left( b_{n} \right),a^{*} \right) + e^{*}\left( \chi\left( a_{n} \right),b^{*} \right)\\
&< \varepsilon + \varepsilon + \varepsilon = 3\varepsilon
\end{align*}
その極限点$b^{*}$はその元の列$\left( a_{n} \right)_{n \in \mathbb{N}}$に依らないことが示された。\par
以上より、次のような写像$f$が定義される。
\begin{align*}
f:S^{*} \rightarrow T^{*};a^{*} \mapsto b^{*}
\end{align*}
このとき、先ほどの元の列$\left( a_{n} \right)_{n \in \mathbb{N}}$を用いて$b^{*} = \lim_{n \rightarrow \infty}{\chi\left( a_{n} \right)}$が成り立つので、その元の列$\left( \varphi\left( a_{n} \right) \right)_{n \in \mathbb{N}}$は極限点としてその元$a^{*}$に収束することになる。したがって、上記の議論と同様にして次のような写像$g$が定義されれば、
\begin{align*}
g:T^{*} \rightarrow S^{*};b^{*} \mapsto a^{*}
\end{align*}
$g \circ f = I_{S^{*}}$が成り立つ。同様にして、$f \circ g = I_{T^{*}}$が成り立つことが示されるので、その写像$f$は全単射である。\par
$\forall a \in S$に対し、明らかに$\varphi(a) = \lim_{n \rightarrow \infty}{\varphi(a)}$が成り立つので、その写像$f$の定義よりその元の列$\left( \chi(a) \right)_{n \in \mathbb{N}}$は収束することになり、その極限点を$b^{*}$とおくと、$\chi(a) = b^{*} = f \circ \varphi(a)$が成り立つ。よって、$f \circ \varphi = \chi$が成り立つ。\par
$\forall a,b \in S^{*}$に対し、$a,b \in {\mathrm{cl}}{V(\varphi)}$が成り立つので、定理\ref{8.2.1.13}よりあるその集合$S$の元の列々$\left( a_{n} \right)_{n \in \mathbb{N}}$、$\left( b_{n} \right)_{n \in \mathbb{N}}$が存在して、$a = \lim_{n \rightarrow \infty}{\varphi\left( a_{n} \right)}$、$b = \lim_{n \rightarrow \infty}{\varphi\left( b_{n} \right)}$が成り立つ。このとき、定理\ref{8.2.1.24}より次のようになる。
\begin{align*}
d^{*}(a,b) &= d^{*}\left( \lim_{n \rightarrow \infty}{\varphi\left( a_{n} \right)},\lim_{n \rightarrow \infty}{\varphi\left( b_{n} \right)} \right)\\
&= \lim_{n \rightarrow \infty}{d^{*}\left( \varphi\left( a_{n} \right),\varphi\left( b_{n} \right) \right)}\\
&= \lim_{n \rightarrow \infty}{d\left( a_{n},b_{n} \right)}\\
&= \lim_{n \rightarrow \infty}{e^{*}\left( \chi\left( a_{n} \right),\chi\left( b_{n} \right) \right)}\\
&= e^{*}\left( \lim_{n \rightarrow \infty}{\chi\left( a_{n} \right)},\lim_{n \rightarrow \infty}{\chi\left( b_{n} \right)} \right)\\
&= e^{*}\left( f(a),f(b) \right)
\end{align*}
\end{proof}
\begin{thm}\label{8.2.6.7} 距離空間$(S,d)$が与えられたとき、定理\ref{8.2.6.6}における写像$\varphi$は単射である。
\end{thm}
\begin{proof} 距離空間$(S,d)$が与えられたとき、定理\ref{8.2.6.6}における距離空間$\left( S^{*},d^{*} \right)$と写像$\varphi:S \rightarrow S^{*}$において、$\forall a,b \in S$に対し、$a \neq b$が成り立つなら、$0 < d(a,b)$が成り立つ。ここで、$d(a,b) = d^{*}\left( \varphi(a),\varphi(b) \right)$が成り立つことにより$0 < d^{*}\left( \varphi(a),\varphi(b) \right)$が成り立つので、$\varphi(a) \neq \varphi(b)$が成り立つ。よって、その写像$\varphi$は単射である。
\end{proof}
\begin{dfn}距離空間$(S,d)$が与えられたとき、定理\ref{8.2.6.6}より次のことを満たす距離空間$\left( S^{*},d^{*} \right)$と写像$\varphi:S \rightarrow S^{*}$が存在し、
\begin{itemize}
\item
  その距離空間$\left( S^{*},d^{*} \right)$は完備である。
\item
  $\forall a,b \in S$に対し、$d(a,b) = d^{*}\left( \varphi(a),\varphi(b) \right)$が成り立つ。
\item
  その写像$\varphi$の値域$V(\varphi)$はその台集合$S^{*}$において密である、即ち、${\mathrm{cl}}{V(\varphi)} = S^{*}$が成り立つ。
\end{itemize}
さらに、そのような距離空間$\left( S^{*},d^{*} \right)$と写像$\varphi$の組が$\left( \left( S^{*},d^{*} \right),\varphi \right)$、$\left( \left( T^{*},e^{*} \right),\chi \right)$と2つ与えられたとき、ある全単射$f:S^{*} \rightarrow T^{*}$が存在して、次のことを満たすのであった。
\begin{itemize}
\item
  $f \circ \varphi = \chi$が成り立つ。
\item
  $\forall a,b \in S^{*}$に対し、$d^{*}(a,b) = e^{*}\left( f(a),f(b) \right)$が成り立つ。
\end{itemize}
このときの距離空間$\left( S^{*},d^{*} \right)$をその距離空間$(S,d)$の完備化、完備拡大という。
\end{dfn}\par
なお、定理\ref{8.2.6.7}に注意すれば、$\forall a \in S$に対し、$a = \varphi(a)$とみなすことで、$S \subseteq S^{*}$が成り立つとみなせることができる。例えば、距離空間$\left( \mathbb{Q},d_{E}|\mathbb{Q} \times \mathbb{Q} \right)$の完備拡大が距離空間$\left( \mathbb{R},d_{E} \right)$、即ち、1次元Euclid空間$E$である。このことは解析学の実数論におけるCauchy列による実数の構成そのものであるので、他の解析学の書籍と併せて読まれるとおもしろいかもしれない。
\begin{thebibliography}{50}
\bibitem{1}
  松坂和夫, 集合・位相入門, 岩波書店, 1968. 新装版第2刷 p111,268-274 ISBN978-4-00-029871-1
\end{thebibliography}
\end{document}
