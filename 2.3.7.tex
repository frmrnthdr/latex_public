\documentclass[dvipdfmx]{jsarticle}
\setcounter{section}{3}
\setcounter{subsection}{6}
\usepackage{xr}
\externaldocument{2.1.1}
\externaldocument{2.1.2}
\externaldocument{2.3.6}
\usepackage{amsmath,amsfonts,amssymb,array,comment,mathtools,url,docmute}
\usepackage{longtable,booktabs,dcolumn,tabularx,mathtools,multirow,colortbl,xcolor}
\usepackage[dvipdfmx]{graphics}
\usepackage{bmpsize}
\usepackage{amsthm}
\usepackage{enumitem}
\setlistdepth{20}
\renewlist{itemize}{itemize}{20}
\setlist[itemize]{label=•}
\renewlist{enumerate}{enumerate}{20}
\setlist[enumerate]{label=\arabic*.}
\setcounter{MaxMatrixCols}{20}
\setcounter{tocdepth}{3}
\newcommand{\rotin}{\text{\rotatebox[origin=c]{90}{$\in $}}}
\renewcommand{\thesection}{第\arabic{section}部}
\renewcommand{\thesubsection}{\arabic{section}.\arabic{subsection}}
\renewcommand{\thesubsubsection}{\arabic{section}.\arabic{subsection}.\arabic{subsubsection}}
\everymath{\displaystyle}
\allowdisplaybreaks[4]
\usepackage{vtable}
\theoremstyle{definition}
\newtheorem{thm}{定理}[subsection]
\newtheorem*{thm*}{定理}
\newtheorem{dfn}{定義}[subsection]
\newtheorem*{dfn*}{定義}
\newtheorem{axs}[dfn]{公理}
\newtheorem*{axs*}{公理}
\renewcommand{\headfont}{\bfseries}
\makeatletter
  \renewcommand{\section}{%
    \@startsection{section}{1}{\z@}%
    {\Cvs}{\Cvs}%
    {\normalfont\huge\headfont\raggedright}}
\makeatother
\makeatletter
  \renewcommand{\subsection}{%
    \@startsection{subsection}{2}{\z@}%
    {0.5\Cvs}{0.5\Cvs}%
    {\normalfont\LARGE\headfont\raggedright}}
\makeatother
\makeatletter
  \renewcommand{\subsubsection}{%
    \@startsection{subsubsection}{3}{\z@}%
    {0.4\Cvs}{0.4\Cvs}%
    {\normalfont\Large\headfont\raggedright}}
\makeatother
\makeatletter
\renewenvironment{proof}[1][\proofname]{\par
  \pushQED{\qed}%
  \normalfont \topsep6\p@\@plus6\p@\relax
  \trivlist
  \item\relax
  {
  #1\@addpunct{.}}\hspace\labelsep\ignorespaces
}{%
  \popQED\endtrivlist\@endpefalse
}
\makeatother
\renewcommand{\proofname}{\textbf{証明}}
\usepackage{tikz,graphics}
\usepackage[dvipdfmx]{hyperref}
\usepackage{pxjahyper}
\hypersetup{
 setpagesize=false,
 bookmarks=true,
 bookmarksdepth=tocdepth,
 bookmarksnumbered=true,
 colorlinks=false,
 pdftitle={},
 pdfsubject={},
 pdfauthor={},
 pdfkeywords={}}
\begin{document}
%\hypertarget{ux7b49ux9577ux5199ux50cf}{%
\subsection{等長写像}%\label{ux7b49ux9577ux5199ux50cf}}
%\hypertarget{ux7b49ux9577ux5199ux50cf-1}{%
\subsubsection{等長写像}%\label{ux7b49ux9577ux5199ux50cf-1}}
\begin{axs}[等長写像の公理]
$K \subseteq \mathbb{C}$なる体$K$上の内積空間たち$(V,\varPhi )$、$(W,X)$が与えられたとする。写像$f:V \rightarrow W$のうち次のことを満たすとき、その写像をその内積空間$(V,\varPhi )$からその内積空間$(W,X)$への等長写像、計量同型写像、内積空間として線形同型写像、unitary写像などという。
\begin{itemize}
\item
  その写像$f$はそのvector空間$V$からそのvector空間$W$への線形同型写像である。
\item
  $\forall\mathbf{v},\mathbf{w} \in V$に対し、$X\left( f\left( \mathbf{v} \right),f\left( \mathbf{w} \right) \right) = \varPhi \left( \mathbf{v},\mathbf{w} \right)$が成り立つ。
\end{itemize}
\end{axs}
\begin{dfn}
$K \subseteq \mathbb{C}$なる体$K$上の内積空間たち$(V,\varPhi )$、$(W,X)$が与えられたとする。その内積空間$(V,\varPhi )$からその内積空間$(W,X)$への等長写像が存在するとき、これらの内積空間たち$(V,\varPhi )$、$(W,X)$は計量同型である、内積空間として同型であるなどといい、ここでは、$(V,\varPhi ) \cong_{\mathrm{isometry}}(W,X)$と書くことにする。
\end{dfn}
\begin{thm}\label{2.3.7.1}
$K \subseteq \mathbb{C}$なる体$K$上の内積空間たち$(U,\varPhi )$、$(V,X)$、$(W,\varPsi )$が与えられたとする。その内積空間$(U,\varPhi )$からその内積空間$(V,X)$への等長写像$f$、その内積空間$(V,X)$からその内積空間$(W,\varPsi )$への等長写像$g$との合成写像$g \circ f$もその内積空間$(U,\varPhi )$からその内積空間$(W,\varPsi )$への等長写像となる。
\end{thm}
\begin{proof}
$K \subseteq \mathbb{C}$なる体$K$上の内積空間たち$(U,\varPhi )$、$(V,X)$、$(W,\varPsi )$が与えられたとする。その内積空間$(U,\varPhi )$からその内積空間$(V,X)$への等長写像$f$、その内積空間$(V,X)$からその内積空間$(W,\varPsi )$への等長写像$g$との合成写像$g \circ f$について、もちろん、その写像$g \circ f$はそのvector空間$U$からそのvector空間$W$への線形同型写像である。そこで、$\forall\mathbf{v},\mathbf{w} \in U$に対し、次のようになることから、
\begin{align*}
\varPsi \left( g \circ f\left( \mathbf{v} \right),g \circ f\left( \mathbf{w} \right) \right) &= \varPsi \left( g\left( f\left( \mathbf{v} \right) \right),g\left( f\left( \mathbf{w} \right) \right) \right)\\
&= X\left( f\left( \mathbf{v} \right),f\left( \mathbf{w} \right) \right)\\
&= \varPhi \left( \mathbf{v},\mathbf{w} \right)
\end{align*}
その合成写像$g \circ f$もその内積空間$(U,\varPhi )$からその内積空間$(W,\varPsi )$への等長写像となる。
\end{proof}
\begin{thm}\label{2.3.7.2}
$K \subseteq \mathbb{C}$なる体$K$上の内積空間たち$(V,\varPhi )$、$(W,X)$が与えられたとする。その内積空間$(V,\varPhi )$からその内積空間$(W,X)$への等長写像$f$が存在するとき、これの逆写像$f^{- 1}$が存在して、これがその内積空間$(W,X)$からその内積空間$(V,\varPhi )$への等長写像となる。
\end{thm}
\begin{proof}
$K \subseteq \mathbb{C}$なる体$K$上の内積空間たち$(V,\varPhi )$、$(W,X)$が与えられたとする。その内積空間$(V,\varPhi )$からその内積空間$(W,X)$への等長写像$f$が存在するとき、その写像$f$は線形同型写像であるので、逆写像$f^{- 1}$が存在して、これも線形同型写像となる。さらに、$\forall\mathbf{v},\mathbf{w} \in W$に対し、次のようになることから、
\begin{align*}
\varPhi \left( f^{- 1}\left( \mathbf{v} \right),f^{- 1}\left( \mathbf{w} \right) \right) &= X\left( f\left( f^{- 1}\left( \mathbf{v} \right) \right),f\left( f^{- 1}\left( \mathbf{w} \right) \right) \right)\\
&= X\left( f \circ f^{- 1}\left( \mathbf{v} \right),f \circ f^{- 1}\left( \mathbf{w} \right) \right)\\
&= X\left( \mathbf{v},\mathbf{w} \right)
\end{align*}
その逆写像$f^{- 1}$がその内積空間$(W,X)$からその内積空間$(V,\varPhi )$への等長写像となる。
\end{proof}
\begin{thm}\label{2.3.7.3}
$K \subseteq \mathbb{C}$なる体$K$上の内積空間たちが与えられたとき、その関係$\cong_{\mathrm{isometry}}$は同値関係となる。
\end{thm}
\begin{proof}
$K \subseteq \mathbb{C}$なる体$K$上の内積空間$(V,\varPhi )$が与えられたとき、恒等写像$I_{V}$が考えられれば、$(V,\varPhi ) \cong_{\mathrm{isometry}}(V,\varPhi )$が成り立つ。内積空間たち$(V,\varPhi )$、$(W,X)$が与えられたとき、定理\ref{2.3.7.2}より$(V,\varPhi ) \cong_{\mathrm{isometry}}(W,X)$が成り立つなら、$(W,X) \cong_{\mathrm{isometry}}(V,\varPhi )$も成り立つ。内積空間たち$(U,\varPhi )$、$(V,X)$、$(W,\varPsi )$が与えられたとき、$(U,\varPhi ) \cong_{\mathrm{isometry}}(V,X)$かつ$(V,X) \cong_{\mathrm{isometry}}(W,\varPsi )$が成り立つなら、定理\ref{2.3.7.1}より$(U,\varPhi ) \cong_{\mathrm{isometry}}(W,\varPsi )$が成り立つ。以上の議論により、その関係$\cong_{\mathrm{isometry}}$は同値関係となる。
\end{proof}
\begin{thm}\label{2.3.7.4}
$K \subseteq \mathbb{C}$なる体$K$上の内積空間たち$(V,\varPhi )$、$(W,X)$、これらから誘導されるnorm空間たち$\left( V,\varphi_{\varPhi } \right)$、$\left( W,\varphi_{X} \right)$、線形写像$f:V \rightarrow W$が与えられ、さらに、$\dim V = \dim W = n$が成り立つとき、次のことは同値である。
\begin{itemize}
\item
  その写像$f$が等長写像である。
\item
  $\forall\mathbf{v},\mathbf{w} \in V$に対し、$\varPhi \left( \mathbf{v},\mathbf{w} \right) = X\left( f\left( \mathbf{v} \right),f\left( \mathbf{w} \right) \right)$が成り立つ。
\item
  $\varphi_{\varPhi } = \varphi_{X} \circ f$が成り立つ。
\item
  $\forall\mathbf{v} \in V$に対し、$\varphi_{\varPhi }\left( \mathbf{v} \right) = 1$が成り立つなら、$\varphi_{X} \circ f\left( \mathbf{v} \right) = 1$が成り立つ。
\end{itemize}
\end{thm}
\begin{proof}
$K \subseteq \mathbb{C}$なる体$K$上の内積空間たち$(V,\varPhi )$、$(W,X)$、これらから誘導されるnorm空間たち$\left( V,\varphi_{\varPhi } \right)$、$\left( W,\varphi_{X} \right)$、線形写像$f:V \rightarrow W$が与えられ、さらに、$\dim V = \dim W = n$が成り立つとき、その写像$f$が等長写像であるなら、定義より直ちに、$\forall\mathbf{v},\mathbf{w} \in V$に対し、$\varPhi \left( \mathbf{v},\mathbf{w} \right) = X\left( f\left( \mathbf{v} \right),f\left( \mathbf{w} \right) \right)$が成り立つ。\par
$\forall\mathbf{v},\mathbf{w} \in V$に対し、$\varPhi \left( \mathbf{v},\mathbf{w} \right) = X\left( f\left( \mathbf{v} \right),f\left( \mathbf{w} \right) \right)$が成り立つなら、$\forall\mathbf{v} \in V$に対し、次のようになることから、
\begin{align*}
\varphi_{\varPhi }\left( \mathbf{v} \right) = \sqrt{\varPhi \left( \mathbf{v},\mathbf{v} \right)} = \sqrt{X\left( f\left( \mathbf{v} \right),f\left( \mathbf{v} \right) \right)} = \varphi_{X}\left( f\left( \mathbf{v} \right) \right) = \varphi_{X} \circ f\left( \mathbf{v} \right)
\end{align*}
$\varphi_{\varPhi } = \varphi_{X} \circ f$が成り立つ。\par
逆に、$\varphi_{\varPhi } = \varphi_{X} \circ f$が成り立つなら、$\forall\mathbf{v},\mathbf{w} \in V$に対し、次のようになることから、
\begin{align*}
{\varphi_{\varPhi }\left( \mathbf{v} + \mathbf{w} \right)}^{2} &= \varPhi \left( \mathbf{v} + \mathbf{w},\mathbf{v} + \mathbf{w} \right)\\
&= \varPhi \left( \mathbf{v},\mathbf{v} \right) + \varPhi \left( \mathbf{v},\mathbf{w} \right) + \varPhi \left( \mathbf{w},\mathbf{v} \right) + \varPhi \left( \mathbf{w},\mathbf{w} \right)\\
&= \varPhi \left( \mathbf{v},\mathbf{v} \right) + \varPhi \left( \mathbf{v},\mathbf{w} \right) + \overline{\varPhi \left( \mathbf{v},\mathbf{w} \right)} + \varPhi \left( \mathbf{w},\mathbf{w} \right)\\
&= {\varphi_{\varPhi }\left( \mathbf{v} \right)}^{2} + 2\mathrm{Re}{\varPhi \left( \mathbf{v},\mathbf{w} \right)} + {\varphi_{\varPhi }\left( \mathbf{w} \right)}^{2}\\
{\varphi_{\varPhi }\left( \mathbf{v} + \mathbf{w} \right)}^{2} &= {\varphi_{X} \circ f\left( \mathbf{v} + \mathbf{w} \right)}^{2}\\
&= {\varphi_{X}\left( f\left( \mathbf{v} + \mathbf{w} \right) \right)}^{2}\\
&= {\varphi_{X}\left( f\left( \mathbf{v} \right) + f\left( \mathbf{w} \right) \right)}^{2}\\
&= X\left( f\left( \mathbf{v} \right) + f\left( \mathbf{w} \right),f\left( \mathbf{v} \right) + f\left( \mathbf{w} \right) \right)\\
&= X\left( f\left( \mathbf{v} \right),f\left( \mathbf{v} \right) \right) + X\left( f\left( \mathbf{v} \right),f\left( \mathbf{w} \right) \right) + X\left( f\left( \mathbf{w} \right),f\left( \mathbf{v} \right) \right) + X\left( f\left( \mathbf{w} \right),f\left( \mathbf{w} \right) \right)\\
&= X\left( f\left( \mathbf{v} \right),f\left( \mathbf{v} \right) \right) + X\left( f\left( \mathbf{v} \right),f\left( \mathbf{w} \right) \right) + \overline{X\left( f\left( \mathbf{v} \right),f\left( \mathbf{w} \right) \right)} + X\left( f\left( \mathbf{w} \right),f\left( \mathbf{w} \right) \right)\\
&= {\varphi_{X}\left( f\left( \mathbf{v} \right) \right)}^{2} + 2\mathrm{Re}{X\left( f\left( \mathbf{v} \right),f\left( \mathbf{w} \right) \right)} + {\varphi_{X}\left( f\left( \mathbf{w} \right) \right)}^{2}\\
&= {\varphi_{X} \circ f\left( \mathbf{v} \right)}^{2} + 2\mathrm{Re}{X\left( f\left( \mathbf{v} \right),f\left( \mathbf{w} \right) \right)} + {\varphi_{X} \circ f\left( \mathbf{w} \right)}^{2}\\
&= {\varphi_{\varPhi }\left( \mathbf{v} \right)}^{2} + 2\mathrm{Re}{X\left( f\left( \mathbf{v} \right),f\left( \mathbf{w} \right) \right)} + {\varphi_{\varPhi }\left( \mathbf{w} \right)}^{2}
\end{align*}
${\varphi_{\varPhi }\left( \mathbf{v} \right)}^{2} + 2\mathrm{Re}{\varPhi \left( \mathbf{v},\mathbf{w} \right)} + {\varphi_{\varPhi }\left( \mathbf{w} \right)}^{2} = {\varphi_{\varPhi }\left( \mathbf{v} \right)}^{2} + 2\mathrm{Re}{X\left( f\left( \mathbf{v} \right),f\left( \mathbf{w} \right) \right)} + {\varphi_{\varPhi }\left( \mathbf{w} \right)}^{2}$が得られ、したがって、$\mathrm{Re}{\varPhi \left( \mathbf{v},\mathbf{w} \right)} = \mathrm{Re}{X\left( f\left( \mathbf{v} \right),f\left( \mathbf{w} \right) \right)}$が成り立つ。$K \subseteq \mathbb{R}$のとき、次のようになる。
\begin{align*}
\varPhi \left( \mathbf{v},\mathbf{w} \right) = \mathrm{Re}{\varPhi \left( \mathbf{v},\mathbf{w} \right)} = \mathrm{Re}{X\left( f\left( \mathbf{v} \right),f\left( \mathbf{w} \right) \right)} = X\left( f\left( \mathbf{v} \right),f\left( \mathbf{w} \right) \right)
\end{align*}
$\mathbb{R} \subset K \subseteq \mathbb{C}$のとき、次のようになることから、
\begin{align*}
{\varphi_{\varPhi }\left( i\mathbf{v} + \mathbf{w} \right)}^{2} &= \varPhi \left( i\mathbf{v} + \mathbf{w},i\mathbf{v} + \mathbf{w} \right)\\
&= \varPhi \left( i\mathbf{v},i\mathbf{v} \right) + \varPhi \left( i\mathbf{v},\mathbf{w} \right) + \varPhi \left( \mathbf{w},i\mathbf{v} \right) + \varPhi \left( \mathbf{w},\mathbf{w} \right)\\
&= \varPhi \left( \mathbf{v},\mathbf{v} \right) + \varPhi \left( i\mathbf{v},\mathbf{w} \right) + \overline{\varPhi \left( i\mathbf{v},\mathbf{w} \right)} + \varPhi \left( \mathbf{w},\mathbf{w} \right)\\
&= {\varphi_{\varPhi }\left( \mathbf{v} \right)}^{2} + 2\mathrm{Re}{\varPhi \left( i\mathbf{v},\mathbf{w} \right)} + {\varphi_{\varPhi }\left( \mathbf{w} \right)}^{2}\\
&= {\varphi_{\varPhi }\left( \mathbf{v} \right)}^{2} + 2\mathrm{Re}\left( - i\varPhi \left( \mathbf{v},\mathbf{w} \right) \right) + {\varphi_{\varPhi }\left( \mathbf{w} \right)}^{2}\\
&= {\varphi_{\varPhi }\left( \mathbf{v} \right)}^{2} + 2\mathrm{Im}{\varPhi \left( \mathbf{v},\mathbf{w} \right)} + {\varphi_{\varPhi }\left( \mathbf{w} \right)}^{2}\\
{\varphi_{\varPhi }\left( i\mathbf{v} + \mathbf{w} \right)}^{2} &= {\varphi_{X} \circ f\left( i\mathbf{v} + \mathbf{w} \right)}^{2}\\
&= {\varphi_{X}\left( f\left( i\mathbf{v} + \mathbf{w} \right) \right)}^{2}\\
&= {\varphi_{X}\left( f\left( i\mathbf{v} \right) + f\left( \mathbf{w} \right) \right)}^{2}\\
&= X\left( f\left( i\mathbf{v} \right) + f\left( \mathbf{w} \right),f\left( i\mathbf{v} \right) + f\left( \mathbf{w} \right) \right)\\
&= X\left( f\left( i\mathbf{v} \right),f\left( i\mathbf{v} \right) \right) + X\left( f\left( i\mathbf{v} \right),f\left( \mathbf{w} \right) \right) + X\left( f\left( \mathbf{w} \right),f\left( i\mathbf{v} \right) \right) + X\left( f\left( \mathbf{w} \right),f\left( \mathbf{w} \right) \right)\\
&= X\left( \mathrm{if}\left( \mathbf{v} \right),\mathrm{if}\left( \mathbf{v} \right) \right) + X\left( \mathrm{if}\left( \mathbf{v} \right),f\left( \mathbf{w} \right) \right) + \overline{X\left( \mathrm{if}\left( \mathbf{v} \right),f\left( \mathbf{w} \right) \right)} + X\left( f\left( \mathbf{w} \right),f\left( \mathbf{w} \right) \right)\\
&= {\varphi_{X}\left( f\left( \mathbf{v} \right) \right)}^{2} + 2\mathrm{Re}{X\left( \mathrm{if}\left( \mathbf{v} \right),f\left( \mathbf{w} \right) \right)} + {\varphi_{X}\left( f\left( \mathbf{w} \right) \right)}^{2}\\
&= {\varphi_{X} \circ f\left( \mathbf{v} \right)}^{2} + 2\mathrm{Re}{X\left( - i\left( f\left( \mathbf{v} \right),f\left( \mathbf{w} \right) \right) \right)} + {\varphi_{X} \circ f\left( \mathbf{w} \right)}^{2}\\
&= {\varphi_{\varPhi }\left( \mathbf{v} \right)}^{2} + 2\mathrm{Im}{X\left( f\left( \mathbf{v} \right),f\left( \mathbf{w} \right) \right)} + {\varphi_{\varPhi }\left( \mathbf{w} \right)}^{2}
\end{align*}
${\varphi_{\varPhi }\left( \mathbf{v} \right)}^{2} + 2\mathrm{Im}{\varPhi \left( \mathbf{v},\mathbf{w} \right)} + {\varphi_{\varPhi }\left( \mathbf{w} \right)}^{2} = {\varphi_{\varPhi }\left( \mathbf{v} \right)}^{2} + 2\mathrm{Im}{X\left( f\left( \mathbf{v} \right),f\left( \mathbf{w} \right) \right)} + {\varphi_{\varPhi }\left( \mathbf{w} \right)}^{2}$が得られ、したがって、$\mathrm{Im}{\varPhi \left( \mathbf{v},\mathbf{w} \right)} = \mathrm{Im}{X\left( f\left( \mathbf{v} \right),f\left( \mathbf{w} \right) \right)}$が成り立つ。以上の議論により、次のようになる。
\begin{align*}
\varPhi \left( \mathbf{v},\mathbf{w} \right) &= \mathrm{Re}{\varPhi \left( \mathbf{v},\mathbf{w} \right)} + i\mathrm{Im}{\varPhi \left( \mathbf{v},\mathbf{w} \right)}\\
&= \mathrm{Re}{X\left( f\left( \mathbf{v} \right),f\left( \mathbf{w} \right) \right)} + i\mathrm{Im}{X\left( f\left( \mathbf{v} \right),f\left( \mathbf{w} \right) \right)}\\
&= X\left( f\left( \mathbf{v} \right),f\left( \mathbf{w} \right) \right)
\end{align*}\par
$\varphi_{\varPhi } = \varphi_{X} \circ f$が成り立つとき、直ちに、$\forall\mathbf{v} \in V$に対し、$\varphi_{\varPhi }\left( \mathbf{v} \right) = 1$が成り立つなら、$\varphi_{X} \circ f\left( \mathbf{v} \right) = 1$が成り立つ。\par
逆に、$\forall\mathbf{v} \in V$に対し、$\varphi_{\varPhi }\left( \mathbf{v} \right) = 1$が成り立つなら、$\varphi_{X} \circ f\left( \mathbf{v} \right) = 1$が成り立つとき、$\mathbf{v} = \mathbf{0}$のときでは明らかなので、$\mathbf{v} \neq \mathbf{0}$のとき、$\varphi_{\varPhi }\left( \mathbf{v} \right) \neq 0$が成り立つので、次のようになり、
\begin{align*}
\varphi_{\varPhi }\left( \frac{\mathbf{v}}{\varphi_{\varPhi }\left( \mathbf{v} \right)} \right) = \frac{1}{\varphi_{\varPhi }\left( \mathbf{v} \right)} \cdot \varphi_{\varPhi }\left( \mathbf{v} \right) = 1
\end{align*}
したがって、次のようになる。
\begin{align*}
\varphi_{X} \circ f\left( \mathbf{v} \right) &= \frac{\varphi_{\varPhi }\left( \mathbf{v} \right)}{\varphi_{\varPhi }\left( \mathbf{v} \right)}\varphi_{X} \circ f\left( \mathbf{v} \right)\\
&= \varphi_{\varPhi }\left( \mathbf{v} \right)\varphi_{X} \circ f\left( \frac{\mathbf{v}}{\varphi_{\varPhi }\left( \mathbf{v} \right)} \right)\\
&= \varphi_{\varPhi }\left( \mathbf{v} \right) \cdot 1 = \varphi_{\varPhi }\left( \mathbf{v} \right)
\end{align*}
これにより、$\varphi_{\varPhi } = \varphi_{X} \circ f$が得られる。\par
$\varphi_{\varPhi } = \varphi_{X} \circ f$が成り立つなら、$\forall\mathbf{v} \in V$に対し、$\mathbf{v} \neq \mathbf{0}$が成り立つなら、$0 \neq \varphi_{\varPhi }\left( \mathbf{v} \right) = \varphi_{X} \circ f\left( \mathbf{v} \right) = \varphi_{X}\left( f\left( \mathbf{v} \right) \right)$が得られるので、$f\left( \mathbf{v} \right) \neq \mathbf{0}$が成り立つ。対偶律により$\ker f = \left\{ \mathbf{0} \right\}$が得られるので、定理\ref{2.1.2.12}よりその線形写像$f$は単射である。そこで、$\dim V = \dim W$が成り立つので、定理\ref{2.1.2.15}よりその線形写像$f$は全単射である、即ち、線形同型写像である。これにより、その写像$f$は等長写像である。
\end{proof}
\begin{thm}\label{2.3.7.5}
$K \subseteq \mathbb{C}$なる体$K$上の内積空間たち$(V,\varPhi )$、$(W,X)$、線形写像$f:V \rightarrow W$、その内積空間$(V,\varPhi )$の正規直交基底$\mathcal{B}$が与えられ、さらに、$\mathcal{B} =\left\langle \mathbf{o}_{i} \right\rangle_{i \in \varLambda_{n}}$とおき、$\dim V = \dim W = n$が成り立つとき、次のことは同値である。
\begin{itemize}
\item
  その写像$f$が等長写像である。
\item
  その組$\left\langle f\left( \mathbf{o}_{i} \right) \right\rangle_{i \in \varLambda_{n}}$がその内積空間$(W,X)$の正規直交基底をなす。
\end{itemize}
\end{thm}
\begin{proof}
$K \subseteq \mathbb{C}$なる体$K$上の内積空間たち$(V,\varPhi )$、$(W,X)$、これらから誘導されるnorm空間たち$\left( V,\varphi_{\varPhi } \right)$、$\left( W,\varphi_{X} \right)$、線形写像$f:V \rightarrow W$、その内積空間$(V,\varPhi )$の正規直交基底$\mathcal{B}$が与えられ、さらに、$\mathcal{B} =\left\langle \mathbf{o}_{i} \right\rangle_{i \in \varLambda_{n}}$とおき、$\dim V = \dim W = n$が成り立つとき、その写像$f$が等長写像であるなら、定理\ref{2.3.7.4}より次のことは同値であるのであった。
\begin{itemize}
\item
  その写像$f$が等長写像である。
\item
  $\forall\mathbf{v},\mathbf{w} \in V$に対し、$\varPhi \left( \mathbf{v},\mathbf{w} \right) = X\left( f\left( \mathbf{v} \right),f\left( \mathbf{w} \right) \right)$が成り立つ。
\item
  $\varphi_{\varPhi } = \varphi_{X} \circ f$が成り立つ。
\end{itemize}
そこで、定理\ref{2.1.2.6}よりその等長写像$f$は線形同型写像でもあるので、その組$\left\langle f\left( \mathbf{o}_{i} \right) \right\rangle_{i \in \varLambda_{n}}$はそのvector空間$W$の基底をなす。さらに、$\forall i,j \in \varLambda_{n}$に対し、$i \neq j$が成り立つなら、次のようになることから、
\begin{align*}
X\left( f\left( \mathbf{o}_{i} \right),f\left( \mathbf{o}_{j} \right) \right) = \varPhi \left( \mathbf{o}_{i},\mathbf{o}_{j} \right) = 0
\end{align*}
その組$\left\langle f\left( \mathbf{o}_{i} \right) \right\rangle_{i \in \varLambda_{n}}$がその内積空間$(V,\varPhi )$の直交基底をなす。さらに、$\forall i \in \varLambda_{n}$に対し、次のようになることから、
\begin{align*}
\varphi_{X}\left( f\left( \mathbf{o}_{i} \right) \right) = \varphi_{X} \circ f\left( \mathbf{o}_{i} \right) = \varphi_{\varPhi }\left( \mathbf{o}_{i} \right) = 1
\end{align*}
その組$\left\langle f\left( \mathbf{o}_{i} \right) \right\rangle_{i \in \varLambda_{n}}$がその内積空間$(W,X)$の正規直交基底をなす。\par
逆に、その組$\left\langle f\left( \mathbf{o}_{i} \right) \right\rangle_{i \in \varLambda_{n}}$がその内積空間$(W,X)$の正規直交基底をなすなら、もちろん、その写像$f$は線形同型写像である。さらに、$\forall\mathbf{v},\mathbf{w} \in V$に対し、次のようにおかれると、
\begin{align*}
\mathbf{v} = \sum_{i \in \varLambda_{n}} {a_{i}\mathbf{o}_{i}},\ \ \mathbf{w} = \sum_{i \in \varLambda_{n}} {b_{i}\mathbf{o}_{i}}
\end{align*}
次のようになり、
\begin{align*}
f\left( \mathbf{v} \right) &= f\left( \sum_{i \in \varLambda_{n}} {a_{i}\mathbf{o}_{i}} \right) = \sum_{i \in \varLambda_{n}} {a_{i}f\left( \mathbf{o}_{i} \right)}\\
&f\left( \mathbf{w} \right) = f\left( \sum_{i \in \varLambda_{n}} {b_{i}\mathbf{o}_{i}} \right) = \sum_{i \in \varLambda_{n}} {b_{i}f\left( \mathbf{o}_{i} \right)}
\end{align*}
定理\ref{2.3.6.17}より次のようになることから、
\begin{align*}
\varPhi \left( \mathbf{v},\mathbf{w} \right) &= \begin{pmatrix}
\overline{a_{1}} & \overline{a_{2}} & \cdots & \overline{a_{n}} \\
\end{pmatrix}\begin{pmatrix}
b_{1} \\
b_{2} \\
 \vdots \\
b_{n} \\
\end{pmatrix}\\
X\left( f\left( \mathbf{v} \right),f\left( \mathbf{w} \right) \right) &= \begin{pmatrix}
\overline{a_{1}} & \overline{a_{2}} & \cdots & \overline{a_{n}} \\
\end{pmatrix}\begin{pmatrix}
b_{1} \\
b_{2} \\
 \vdots \\
b_{n} \\
\end{pmatrix}
\end{align*}
$X\left( f\left( \mathbf{v} \right),f\left( \mathbf{w} \right) \right) = \varPhi \left( \mathbf{v},\mathbf{w} \right)$が成り立つ。
\end{proof}
\begin{thm}\label{2.3.7.6}
$K \subseteq \mathbb{C}$なる体$K$上の$\dim V = \dim W$なる任意の内積空間たち$(V,\varPhi )$、$(W,X)$が与えられたとき、これらは計量同型である、即ち、$(V,\varPhi ) \cong_{\mathrm{isometry}}(W,X)$が成り立つ。
\end{thm}
\begin{proof}
$K \subseteq \mathbb{C}$なる体$K$上の$\dim V = \dim W = n$なる任意の内積空間たち$(V,\varPhi )$、$(W,X)$が与えられたとき、Gram-Schmidtの直交化法によりこれらの正規直交基底たち$\left\langle \mathbf{o}_{i} \right\rangle_{i \in \varLambda_{n}}$、$\left\langle \mathbf{p}_{i} \right\rangle_{i \in \varLambda_{n}}$が存在して、線形写像$f:V \rightarrow W$で、$\forall i \in \varLambda_{n}$に対し、$f\left( \mathbf{o}_{i} \right) = \mathbf{p}_{i}$とすれば、次式が成り立ち、
\begin{align*}
\left\langle \mathbf{p}_{i} \right\rangle_{i \in \varLambda_{n}} = \left\langle f\left( \mathbf{o}_{i} \right) \right\rangle_{i \in \varLambda_{n}}
\end{align*}
定理\ref{2.3.7.5}よりその写像$f$はその内積空間$(V,\varPhi )$からその内積空間$(W,X)$への等長写像である。
\end{proof}
\begin{thm}\label{2.3.7.7}
$K \subseteq \mathbb{C}$なる体$K$上の$\dim V = n$なる任意の内積空間$(V,\varPhi )$が与えられたとき、これは標準内積空間$\left( K^{n},\left\langle \bullet \middle| \bullet \right\rangle \right)$と計量同型である。さらに、等長写像として、その内積空間$(V,\varPhi )$の正規直交基底$\mathcal{B}$に関する基底変換における線形同型写像の逆写像$\varphi_{\mathcal{B}}^{- 1}:V \rightarrow K^{n}$があげられる。
\end{thm}
\begin{proof}
$K \subseteq \mathbb{C}$なる体$K$上の$\dim V = n$なる任意の内積空間$(V,\varPhi )$が与えられたとき、これは標準内積空間$\left( K^{n},\left\langle \bullet \middle| \bullet \right\rangle \right)$と計量同型であることは定理\ref{2.3.7.6}から直ちにわかる。$\forall\mathbf{v},\mathbf{w} \in V$に対し、次のようにおかれると、
\begin{align*}
\mathbf{v} = \sum_{i \in \varLambda_{n}} {a_{i}\mathbf{o}_{i}},\ \ \mathbf{w} = \sum_{i \in \varLambda_{n}} {b_{i}\mathbf{o}_{i}}
\end{align*}
定理\ref{2.3.6.18}より次のようになることから、
\begin{align*}
\varPhi \left( \mathbf{v},\mathbf{w} \right) = \begin{pmatrix}
\overline{a_{1}} & \overline{a_{2}} & \cdots & \overline{a_{n}} \\
\end{pmatrix}\begin{pmatrix}
b_{1} \\
b_{2} \\
 \vdots \\
b_{n} \\
\end{pmatrix} = \left\langle \begin{pmatrix}
a_{1} \\
a_{2} \\
 \vdots \\
a_{n} \\
\end{pmatrix} \middle| \begin{pmatrix}
b_{1} \\
b_{2} \\
 \vdots \\
b_{n} \\
\end{pmatrix} \right\rangle = \left\langle \varphi_{\mathcal{B}}\left( \mathbf{v} \right) \middle| \varphi_{\mathcal{B}}\left( \mathbf{w} \right) \right\rangle
\end{align*}
等長写像として、その内積空間$(V,\varPhi )$の正規直交基底$\mathcal{B}$に関する基底変換における線形同型写像の逆写像$\varphi_{\mathcal{B}}^{- 1}:V \rightarrow K^{n}$があげられる。
\end{proof}
\begin{thm}\label{2.3.7.8}
$K \subseteq \mathbb{C}$なる体$K$上の$\dim V = \dim W = n$なる任意の内積空間たち$(V,\varPhi )$、$(W,X)$が与えられたとき、写像$f:V \rightarrow W$が、$\forall\mathbf{v},\mathbf{w} \in V$に対し、$X\left( f\left( \mathbf{v} \right),f\left( \mathbf{w} \right) \right) = \varPhi \left( \mathbf{v},\mathbf{w} \right)$を満たすなら\footnote{その写像$f$が線形的であるという仮定は入ってないことに注意してね… 汗}、その写像$f$は等長写像である。
\end{thm}
\begin{proof}
$K \subseteq \mathbb{C}$なる体$K$上の$\dim V = \dim W = n$なる任意の内積空間たち$(V,\varPhi )$、$(W,X)$、これらから誘導されるnorm空間たち$\left( V,\varphi_{\varPhi } \right)$、$\left( W,\varphi_{X} \right)$が与えられたとき、写像$f:V \rightarrow W$が、$\forall\mathbf{v},\mathbf{w} \in V$に対し、$X\left( f\left( \mathbf{v} \right),f\left( \mathbf{w} \right) \right) = \varPhi \left( \mathbf{v},\mathbf{w} \right)$を満たすなら、そのvector空間$V$の正規直交基底$\mathcal{B}$が$\mathcal{B} =\left\langle \mathbf{o}_{i} \right\rangle_{i \in \varLambda_{n}}$と与えられたとき、$\forall i,j \in \varLambda_{n}$に対し、$i \neq j$が成り立つかつ、$f\left( \mathbf{o}_{i} \right) = f\left( \mathbf{o}_{j} \right)$が成り立つと仮定しよう。このとき、$\varPhi \left( \mathbf{o}_{i},\mathbf{o}_{j} \right) = 0$が成り立つので、次のようになる。
\begin{align*}
1 &= \varPhi \left( \mathbf{o}_{i},\mathbf{o}_{i} \right)\\
&= X\left( f\left( \mathbf{o}_{i} \right),f\left( \mathbf{o}_{i} \right) \right)\\
&= X\left( f\left( \mathbf{o}_{i} \right),f\left( \mathbf{o}_{j} \right) \right)\\
&= \varPhi \left( \mathbf{o}_{i},\mathbf{o}_{j} \right) = 0
\end{align*}
これは矛盾しているので、$i \neq j$が成り立つなら、$f\left( \mathbf{o}_{i} \right) \neq f\left( \mathbf{o}_{j} \right)$が成り立つ。さらに、$\forall i,j \in \varLambda_{n}$に対し、$i \neq j$が成り立つなら、$X\left( f\left( \mathbf{o}_{i} \right),f\left( \mathbf{o}_{j} \right) \right) = \varPhi \left( \mathbf{o}_{i},\mathbf{o}_{j} \right) = 0$が成り立つので、その族$\left\{ f\left( \mathbf{o}_{i} \right) \right\}_{i \in \varLambda_{n}}$は直交系をなす。定理\ref{2.3.6.9}より直交系$\left\{ f\left( \mathbf{o}_{i} \right) \right\}_{i \in \varLambda_{n}}$をなすこれらのvectors$f\left( \mathbf{o}_{i} \right)$は線形独立である。したがって、その組$\left\langle f\left( \mathbf{o}_{i} \right) \right\rangle_{i \in \varLambda_{n}}$はその部分空間$\mathrm{span}\left\{ f\left( \mathbf{o}_{i} \right) \right\}_{i \in \varLambda_{n}}$の基底をなし、$\mathrm{span}\left\{ f\left( \mathbf{o}_{i} \right) \right\}_{i \in \varLambda_{n}} \subseteq W$かつ$\dim{\mathrm{span}\left\{ f\left( \mathbf{o}_{i} \right) \right\}_{i \in \varLambda_{n}}} = n = \dim W$が成り立つので、$\mathrm{span}\left\{ f\left( \mathbf{o}_{i} \right) \right\}_{i \in \varLambda_{n}} = W$が得られる。これにより、その組$\left\langle f\left( \mathbf{o}_{i} \right) \right\rangle_{i \in \varLambda_{n}}$はそのvector空間$W$の基底をなす。特に、$\forall i,j \in \varLambda_{n}$に対し、$i \neq j$が成り立つなら、$X\left( f\left( \mathbf{o}_{i} \right),f\left( \mathbf{o}_{j} \right) \right) = \varPhi \left( \mathbf{o}_{i},\mathbf{o}_{j} \right) = 0$が成り立つかつ、$\forall i \in \varLambda_{n}$に対し、$\varphi_{X}\left( f\left( \mathbf{o}_{i} \right) \right) = \sqrt{X\left( f\left( \mathbf{o}_{i} \right),f\left( \mathbf{o}_{i} \right) \right)} = \sqrt{\varPhi \left( \mathbf{o}_{i},\mathbf{o}_{i} \right)} = 1$が成り立つので、その基底$\left\langle f\left( \mathbf{o}_{i} \right) \right\rangle_{i \in \varLambda_{n}}$はその内積空間$(W,X)$の正規直交基底をなす。\par
次に、$\forall\mathbf{v} \in W$に対し、次のようにおかれると、
\begin{align*}
\mathbf{v} = \sum_{i \in \varLambda_{n}} {b_{i}f\left( \mathbf{o}_{i} \right)}
\end{align*}
次のようになることから、
\begin{align*}
X\left( f\left( \sum_{i \in \varLambda_{n}} {a_{i}\mathbf{o}_{i}} \right) - \sum_{i \in \varLambda_{n}} {a_{i}f\left( \mathbf{o}_{i} \right)},\mathbf{v} \right) &= X\left( f\left( \sum_{i \in \varLambda_{n}} {a_{i}\mathbf{o}_{i}} \right) - \sum_{i \in \varLambda_{n}} {a_{i}f\left( \mathbf{o}_{i} \right)},\sum_{i \in \varLambda_{n}} {b_{i}f\left( \mathbf{o}_{i} \right)} \right)\\
&= X\left( f\left( \sum_{i \in \varLambda_{n}} {a_{i}\mathbf{o}_{i}} \right),\sum_{i \in \varLambda_{n}} {b_{i}f\left( \mathbf{o}_{i} \right)} \right) \\
&\quad - X\left( \sum_{i \in \varLambda_{n}} {a_{i}f\left( \mathbf{o}_{i} \right)},\sum_{i \in \varLambda_{n}} {b_{i}f\left( \mathbf{o}_{i} \right)} \right)\\
&= \sum_{j \in \varLambda_{n}} {b_{j}X\left( f\left( \sum_{i \in \varLambda_{n}} {a_{i}\mathbf{o}_{i}} \right),f\left( \mathbf{o}_{j} \right) \right)} \\
&\quad - \sum_{i \in \varLambda_{n}} {\sum_{j \in \varLambda_{n}} {\overline{a_{i}}b_{j}X\left( f\left( \mathbf{o}_{i} \right),f\left( \mathbf{o}_{j} \right) \right)}}\\
&= \sum_{j \in \varLambda_{n}} {b_{j}\varPhi \left( \sum_{i \in \varLambda_{n}} {a_{i}\mathbf{o}_{i}},\mathbf{o}_{j} \right)} - \sum_{j \in \varLambda_{n}} {\sum_{i \in \varLambda_{n}} {\overline{a_{i}}b_{j}\varPhi \left( \mathbf{o}_{i},\mathbf{o}_{j} \right)}}\\
&= \sum_{j \in \varLambda_{n}} {b_{j}\sum_{i \in \varLambda_{n}} {\overline{a_{i}}\varPhi \left( \mathbf{o}_{i},\mathbf{o}_{j} \right)}} - \sum_{j \in \varLambda_{n}} {\sum_{i \in \varLambda_{n}} {\overline{a_{i}}b_{j}\varPhi \left( \mathbf{o}_{i},\mathbf{o}_{j} \right)}}\\
&= \sum_{j \in \varLambda_{n}} {b_{j}\left( \sum_{i \in \varLambda_{n}} {\overline{a_{i}}\varPhi \left( \mathbf{o}_{i},\mathbf{o}_{j} \right)} - \sum_{i \in \varLambda_{n}} {\overline{a_{i}}\varPhi \left( \mathbf{o}_{i},\mathbf{o}_{j} \right)} \right)}\\
&= \sum_{j \in \varLambda_{n}} {b_{j} \cdot 0} = 0
\end{align*}
定理\ref{2.3.6.5}より$f\left( \sum_{i \in \varLambda_{n}} {a_{i}\mathbf{o}_{i}} \right) - \sum_{i \in \varLambda_{n}} {a_{i}f\left( \mathbf{o}_{i} \right)} = \mathbf{0}$が成り立つ。そこで、$\forall k,l \in K\forall\mathbf{v},\mathbf{w} \in V$に対し、次のようにおかれると、
\begin{align*}
\mathbf{v} = \sum_{i \in \varLambda_{n}} {a_{i}f\left( \mathbf{o}_{i} \right)},\ \ \mathbf{w} = \sum_{i \in \varLambda_{n}} {b_{i}f\left( \mathbf{o}_{i} \right)}
\end{align*}
次のようになることから、
\begin{align*}
f\left( k\mathbf{v} + l\mathbf{w} \right) &= f\left( k\sum_{i \in \varLambda_{n}} {a_{i}f\left( \mathbf{o}_{i} \right)} + l\sum_{i \in \varLambda_{n}} {b_{i}f\left( \mathbf{o}_{i} \right)} \right)\\
&= f\left( \sum_{i \in \varLambda_{n}} {\left( ka_{i} + lb_{i} \right)f\left( \mathbf{o}_{i} \right)} \right)\\
&= f\left( \sum_{i \in \varLambda_{n}} {\left( ka_{i} + lb_{i} \right)f\left( \mathbf{o}_{i} \right)} \right) \\
&\quad - \sum_{i \in \varLambda_{n}} {\left( ka_{i} + lb_{i} \right)f\left( \mathbf{o}_{i} \right)} + \sum_{i \in \varLambda_{n}} {\left( ka_{i} + lb_{i} \right)f\left( \mathbf{o}_{i} \right)}\\
&= \mathbf{0} + \sum_{i \in \varLambda_{n}} {\left( ka_{i} + lb_{i} \right)f\left( \mathbf{o}_{i} \right)}\\
&= \sum_{i \in \varLambda_{n}} {\left( ka_{i} + lb_{i} \right)f\left( \mathbf{o}_{i} \right)}\\
&= k\sum_{i \in \varLambda_{n}} {a_{i}f\left( \mathbf{o}_{i} \right)} + l\sum_{i \in \varLambda_{n}} {b_{i}f\left( \mathbf{o}_{i} \right)}\\
&= k\mathbf{v} + l\mathbf{w}
\end{align*}
その写像$f$は線形写像である。\par
以上の議論と定理\ref{2.3.7.4}よりその写像$f$は等長写像である。
\end{proof}
%\hypertarget{ux76f4ux4ea4ux7a7aux9593}{%
\subsubsection{直交空間}%\label{ux76f4ux4ea4ux7a7aux9593}}
\begin{dfn}
$K \subseteq \mathbb{C}$なる内積空間$(V,\varPhi )$、$S \neq \emptyset$なるそのvector空間$V$の部分集合$S$が与えられたとき、$\forall\mathbf{s} \in S$に対し、$\varPhi \left( \mathbf{s},\mathbf{v} \right) = 0$なるvector$\mathbf{v}$全体の集合を$S^{\bot}$とおく、即ち、次式のようにおく。
\begin{align*}
S^{\bot} = \left\{ \mathbf{v} \in V \middle| \forall\mathbf{s} \in S\left[ \varPhi \left( \mathbf{s},\mathbf{v} \right) = 0 \right] \right\}
\end{align*}
その集合$S^{\bot}$をその集合$S$のその内積空間$(V,\varPhi )$における直交空間という。
\end{dfn}
\begin{thm}\label{2.3.7.9}
$K \subseteq \mathbb{C}$なる内積空間$(V,\varPhi )$、$S \neq \emptyset$なるそのvector空間$V$の部分集合$S$が与えられたとき、その集合$S$のその内積空間$(V,\varPhi )$における直交空間$S^{\bot}$はそのvector空間$V$の部分空間をなす。
\end{thm}
\begin{proof}
$K \subseteq \mathbb{C}$なる内積空間$(V,\varPhi )$、$S \neq \emptyset$なるそのvector空間$V$の部分集合$S$が与えられたとき、その集合$S$のその内積空間$(V,\varPhi )$における直交空間$S^{\bot}$について、もちろん、$\mathbf{0} \in S^{\bot}$が成り立つ。さらに、$\forall k,l \in K\forall\mathbf{v},\mathbf{w} \in S^{\bot}$に対し、直交空間の定義より$\forall\mathbf{s} \in S$に対し、次のようになることから、
\begin{align*}
\varPhi \left( \mathbf{s},k\mathbf{v} + l\mathbf{w} \right) &= k\varPhi \left( \mathbf{s},\mathbf{w} \right) + l\varPhi \left( \mathbf{s},\mathbf{w} \right)\\
&= k \cdot 0 + l \cdot 0 = 0
\end{align*}
$k\mathbf{v} + l\mathbf{w} \in S^{\bot}$が成り立つ。定理\ref{2.1.1.9}よりその集合$S$のその内積空間$(V,\varPhi )$における直交空間$S^{\bot}$はそのvector空間$V$の部分空間をなす。
\end{proof}
\begin{thm}\label{2.3.7.10}
$K \subseteq \mathbb{C}$かつ$\dim V = n$なる内積空間$(V,\varPhi )$、$\dim W = r$なるそのvector空間$V$の部分空間$W$が与えられたとき、$V = W \oplus W^{\bot}$が成り立つ。さらに、次式が成り立つ。
\begin{align*}
\dim V = \dim W + \dim W^{\bot}
\end{align*}\par
この定理により$V = W \oplus W^{\bot}$が成り立つことから、その直交空間$W^{\bot}$はその部分空間$W$の補空間となっている。そのような意味でその直交空間$W^{\bot}$をその部分空間$W$の直交補空間ともいう。
\end{thm}
\begin{proof}
$K \subseteq \mathbb{C}$かつ$\dim V = n$なる内積空間$(V,\varPhi )$、$\dim W = r$なるそのvector空間$V$の部分空間$W$が与えられたとき、定理\ref{2.3.6.12}よりその部分空間$W$の正規直交基底$\left\langle \mathbf{o}_{i} \right\rangle_{i \in \varLambda_{r}}$が与えられれば、$i \in \varLambda_{n} \setminus \varLambda_{r}$なる適切な$n - r$つのvectors$\mathbf{o}_{i}$を付け加えた組$\left\langle \mathbf{o}_{i} \right\rangle_{i \in \varLambda_{n}}$がそのvector空間$V$の正規直交基底をなすようにすることができる。そこで、次式が成り立つことから、
\begin{align*}
V = \mathrm{span}\left\{ \mathbf{o}_{i} \right\}_{i \in \varLambda_{n}},\ \ W = \mathrm{span}\left\{ \mathbf{o}_{i} \right\}_{i \in \varLambda_{r}},\\
\mathrm{span}\left\{ \mathbf{o}_{i} \right\}_{i \in \varLambda_{n}} = \mathrm{span}\left\{ \mathbf{o}_{i} \right\}_{i \in \varLambda_{r}} \oplus \mathrm{span}\left\{ \mathbf{o}_{i} \right\}_{i \in \varLambda_{n} \setminus \varLambda_{r}}
\end{align*}
$W^{\bot} = \mathrm{span}\left\{ \mathbf{o}_{i} \right\}_{i \in \varLambda_{n} \setminus \varLambda_{r}}$が成り立つことを示せばよい。\par
実際、$\forall\mathbf{v} \in V$に対し、次のようにおかれると、
\begin{align*}
\mathbf{v} = \sum_{i \in \varLambda_{n}} {a_{i}\mathbf{o}_{i}}
\end{align*}
$\mathbf{v} \notin \mathrm{span}\left\{ \mathbf{o}_{i} \right\}_{i \in \varLambda_{n} \setminus \varLambda_{r}}$が成り立つなら、$\exists i \in \varLambda_{r}$に対し、$a_{i} \neq 0$が成り立つ。したがって、この添数を$i'$とすると、$\mathbf{o}_{i'} \in \mathrm{span}\left\{ \mathbf{o}_{i} \right\}_{i \in \varLambda_{r}}$が成り立つかつ、$\forall i,j \in \varLambda_{n}$に対し、$i \neq j$が成り立つなら、$\varPhi \left( \mathbf{o}_{i},\mathbf{o}_{j} \right) = 0$が成り立つかつ、$\forall i \in \varLambda_{n}$に対し、$\varPhi \left( \mathbf{v}_{i},\mathbf{v}_{i} \right) = 1$が成り立つことに注意すれば、次のようになることから、
\begin{align*}
\varPhi \left( \mathbf{o}_{i'},\mathbf{v} \right) &= \varPhi \left( \mathbf{o}_{i'},\sum_{i \in \varLambda_{n}} {a_{i}\mathbf{o}_{i}} \right)\\
&= \sum_{i \in \varLambda_{n}} {a_{i}\varPhi \left( \mathbf{o}_{i'},\mathbf{o}_{i} \right)}\\
&= \sum_{i \in \varLambda_{n} \setminus \left\{ i' \right\}} {a_{i}\varPhi \left( \mathbf{o}_{i'},\mathbf{o}_{i} \right)} + a_{i'}\varPhi \left( \mathbf{o}_{i'},\mathbf{o}_{i'} \right)\\
&= \sum_{i \in \varLambda_{n} \setminus \left\{ i' \right\}} {a_{i} \cdot 0} + a_{i'} \cdot 1 = a_{i'} \neq 0
\end{align*}
$\exists\mathbf{w} \in W$に対し、$\varPhi \left( \mathbf{w},\mathbf{v} \right) \neq 0$が成り立つので、$\mathbf{v} \notin W^{\bot}$が成り立つ。対偶律により、$W^{\bot} \subseteq \mathrm{span}\left\{ \mathbf{o}_{i} \right\}_{i \in \varLambda_{n} \setminus \varLambda_{r}}$が成り立つ。\par
逆に、$\forall\mathbf{v} \in V$に対し、$\mathbf{v} \in \mathrm{span}\left\{ \mathbf{o}_{i} \right\}_{i \in \varLambda_{n} \setminus \varLambda_{r}}$が成り立つなら、次のようにおかれると、
\begin{align*}
\mathbf{v} = \sum_{i \in \varLambda_{n} \setminus \varLambda_{r}} {a_{i}\mathbf{o}_{i}}
\end{align*}
$\forall\mathbf{w} \in W$に対し、$W = \mathrm{span}\left\{ \mathbf{o}_{i} \right\}_{i \in \varLambda_{r}}$が成り立つことに注意すれば、次のようにおかれると、
\begin{align*}
\mathbf{w} = \sum_{i \in \varLambda_{r}} {b_{i}\mathbf{o}_{i}}
\end{align*}
$\forall i,j \in \varLambda_{n}$に対し、$i \neq j$が成り立つなら、$\varPhi \left( \mathbf{o}_{i},\mathbf{o}_{j} \right) = 0$が成り立つことに注意すれば、次のようになるので、
\begin{align*}
\varPhi \left( \mathbf{w},\mathbf{v} \right) &= \varPhi \left( \sum_{i \in \varLambda_{r}} {b_{i}\mathbf{o}_{i}},\sum_{i \in \varLambda_{n} \setminus \varLambda_{r}} {a_{i}\mathbf{o}_{i}} \right)\\
&= \sum_{i \in \varLambda_{r}} {\sum_{j \in \varLambda_{n} \setminus \varLambda_{r}} {a_{j}\overline{b_{i}}\varPhi \left( \mathbf{o}_{i},\mathbf{o}_{j} \right)}}\\
&= \sum_{i \in \varLambda_{r}} {\sum_{j \in \varLambda_{n} \setminus \varLambda_{r}} {a_{j}\overline{b_{i}} \cdot 0}} = 0
\end{align*}
$\mathbf{v} \in W^{\bot}$が成り立つ。これにより、$\mathrm{span}\left\{ \mathbf{o}_{i} \right\}_{i \in \varLambda_{n} \setminus \varLambda_{r}} \subseteq W^{\bot}$が得られる。\par
以上の議論により、$W^{\bot} = \mathrm{span}\left\{ \mathbf{o}_{i} \right\}_{i \in \varLambda_{n} \setminus \varLambda_{r}}$が得られるので、次のようになる。
\begin{align*}
V &= \mathrm{span}\left\{ \mathbf{o}_{i} \right\}_{i \in \varLambda_{n}}\\
&= \mathrm{span}\left\{ \mathbf{o}_{i} \right\}_{i \in \varLambda_{r}} \oplus \mathrm{span}\left\{ \mathbf{o}_{i} \right\}_{i \in \varLambda_{n} \setminus \varLambda_{r}}\\
&= W \oplus W^{\bot}
\end{align*}\par
さらに、次のようになる。
\begin{align*}
\dim V = \dim{W \oplus W^{\bot}} = \dim W + \dim W^{\bot}
\end{align*}
\end{proof}
\begin{thm}\label{2.3.7.11}
$K \subseteq \mathbb{C}$かつ$\dim V = n$なる内積空間$(V,\varPhi )$、$\dim W = r$なるそのvector空間$V$の部分空間$W$が与えられたとき、$W^{\bot\bot} = W$が成り立つ。
\end{thm}
\begin{proof}
$K \subseteq \mathbb{C}$かつ$\dim V = n$なる内積空間$(V,\varPhi )$、$\dim W = r$なるそのvector空間$V$の部分空間$W$が与えられたとき、定理\ref{2.3.7.10}より$V = W \oplus W^{\bot}$が成り立つので、その直交空間$W^{\bot}$の直交補空間が$W$と与えられることから従う。
\end{proof}
\begin{thm}\label{2.3.7.12}
$K \subseteq \mathbb{C}$かつ$\dim V = n$なる内積空間$(V,\varPhi )$が与えられたとき、そのvector空間$V$の部分空間たち$U$、$W$について、$U \subseteq W$が成り立つなら、$W^{\bot} \subseteq U^{\bot}$が成り立つ。
\end{thm}
\begin{proof}
$K \subseteq \mathbb{C}$かつ$\dim V = n$なる内積空間$(V,\varPhi )$が与えられたとき、そのvector空間$V$の部分空間たち$U$、$W$について、$U \subseteq W$が成り立つなら、$\forall\mathbf{v} \in V$に対し、$\mathbf{v} \in W^{\bot}$が成り立つなら、$\forall\mathbf{w} \in W$に対し、$\varPhi \left( \mathbf{w},\mathbf{v} \right) = 0$が成り立つことになる。そこで、$\forall\mathbf{u} \in U$に対し、$\mathbf{u} \in U \subseteq W$が成り立つので、$\varPhi \left( \mathbf{u},\mathbf{v} \right) = 0$が成り立つ。よって、$\mathbf{v} \in U^{\bot}$が成り立つので、$W^{\bot} \subseteq U^{\bot}$が得られる。
\end{proof}
\begin{thm}\label{2.3.7.13}
$K \subseteq \mathbb{C}$かつ$\dim V = n$なる内積空間$(V,\varPhi )$が与えられたとき、そのvector空間$V$の部分空間たち$U$、$W$について、次式が成り立つ。
\begin{align*}
(U + W)^{\bot} &= U^{\bot} \cap W^{\bot}\\
(U \cap W)^{\bot} &= U^{\bot} + W^{\bot}
\end{align*}
\end{thm}
\begin{proof}
$K \subseteq \mathbb{C}$かつ$\dim V = n$なる内積空間$(V,\varPhi )$が与えられたとき、そのvector空間$V$の部分空間たち$U$、$W$について、もちろん、$U \subseteq U + W$が成り立つので、定理\ref{2.3.7.12}より$(U + W)^{\bot} \subseteq U^{\bot}$が成り立つ。同様にして、$(U + W)^{\bot} \subseteq W^{\bot}$が得られるので、次のようになる。
\begin{align*}
(U + W)^{\bot} = (U + W)^{\bot} \cap (U + W)^{\bot} \subseteq U^{\bot} \cap W^{\bot}
\end{align*}
逆に、$\forall\mathbf{v} \in V$に対し、$\mathbf{v} \in U^{\bot} \cap W^{\bot}$が成り立つなら、$\mathbf{u} \in U$、$\mathbf{w} \in W$として、$\forall\mathbf{u} + \mathbf{w} \in U + W$に対し、$\varPhi \left( \mathbf{u},\mathbf{v} \right) = \varPhi \left( \mathbf{w},\mathbf{v} \right) = 0$が成り立つので、次のようになる。
\begin{align*}
\varPhi \left( \mathbf{u} + \mathbf{w},\mathbf{v} \right) = \varPhi \left( \mathbf{u},\mathbf{v} \right) + \varPhi \left( \mathbf{w},\mathbf{v} \right) = 0 + 0 = 0
\end{align*}
ゆえに、$\mathbf{v} \in (U + W)^{\bot}$が成り立つので、$(U + W)^{\bot} = U^{\bot} \cap W^{\bot}$が得られる。\par
また、上記の議論により、$\left( U^{\bot} + W^{\bot} \right)^{\bot} = U^{\bot\bot} \cap W^{\bot\bot}$が成り立つので、定理\ref{2.3.7.11}より$(U \cap W)^{\bot} = U^{\bot} + W^{\bot}$が得られる。
\end{proof}
\begin{thm}[Bessel-Parsevalの不等式]\label{2.3.7.14}
$K \subseteq \mathbb{C}$かつ$\dim V = n$なる内積空間$(V,\varPhi )$から誘導されるnorm空間$\left( V,\varphi_{\varPhi } \right)$、この内積空間$(V,\varPhi )$における正規直交系$\left\{ \mathbf{o}_{i} \right\}_{i \in \varLambda_{r}}$が与えられたとき、$\forall\mathbf{v} \in V$に対し、次式が成り立つ。
\begin{align*}
\sum_{i \in \varLambda_{r}} \left| \varPhi \left( \mathbf{o}_{i},\mathbf{v} \right) \right|^{2} \leq {\varphi_{\varPhi }\left( \mathbf{v} \right)}^{2}
\end{align*}
さらに、その内積空間$(V,\varPhi )$の正規直交基底$\left\langle \mathbf{o}_{i} \right\rangle_{i \in \varLambda_{n}}$が与えられたとき、$\forall\mathbf{v} \in V$に対し、次式が成り立つ。
\begin{align*}
\sum_{i \in \varLambda_{n}} \left| \varPhi \left( \mathbf{o}_{i},\mathbf{v} \right) \right|^{2} = {\varphi_{\varPhi }\left( \mathbf{v} \right)}^{2}
\end{align*}
その不等式をBessel-Parsevalの不等式という。
\end{thm}
\begin{proof}
$K \subseteq \mathbb{C}$かつ$\dim V = n$なる内積空間$(V,\varPhi )$から誘導されるnorm空間$\left( V,\varphi_{\varPhi } \right)$、この内積空間$(V,\varPhi )$における正規直交系$\left\{ \mathbf{o}_{i} \right\}_{i \in \varLambda_{r}}$が与えられたとき、定理\ref{2.3.6.12}より$\i \in \varLambda_{n} \setminus \varLambda_{r}$なる適切な$n - r$つのvectors$\mathbf{o}_{i}$を付け加えた組$\left\langle \mathbf{o}_{i} \right\rangle_{i \in \varLambda_{n}}$がそのvector空間$V$の正規直交基底をなすようにすることができる。$\forall\mathbf{v} \in V$に対し、次のようにおかれると、
\begin{align*}
\mathbf{v} = \sum_{i \in \varLambda_{n}} {a_{i}\mathbf{o}_{i}}
\end{align*}
定理\ref{2.3.6.17}より$\forall i \in \varLambda_{n}$に対し、$v_{i} = \varPhi \left( \mathbf{o}_{i},\mathbf{v} \right)$が成り立つので、次式が得られる。
\begin{align*}
\sum_{i \in \varLambda_{r}} \left| \varPhi \left( \mathbf{o}_{i},\mathbf{v} \right) \right|^{2} \leq \sum_{i \in \varLambda_{r}} \left| \varPhi \left( \mathbf{o}_{i},\mathbf{v} \right) \right|^{2} + {\varphi_{\varPhi }\left( \mathbf{v} - \sum_{i \in \varLambda_{r}} {a_{i}\mathbf{o}_{i}} \right)}^{2}
\end{align*}
そこで、次のようになることから、
\begin{align*}
{\varphi_{\varPhi }\left( \mathbf{v} - \sum_{i \in \varLambda_{r}} {a_{i}\mathbf{o}_{i}} \right)}^{2} &= {\varphi_{\varPhi }\left( \mathbf{v} - \sum_{i \in \varLambda_{r}} {\varPhi \left( \mathbf{o}_{i},\mathbf{v} \right)\mathbf{o}_{i}} \right)}^{2}\\
&= \varPhi \left( \mathbf{v} - \sum_{i \in \varLambda_{r}} {\varPhi \left( \mathbf{o}_{i},\mathbf{v} \right)\mathbf{o}_{i}},\mathbf{v} - \sum_{i \in \varLambda_{r}} {\varPhi \left( \mathbf{o}_{i},\mathbf{v} \right)\mathbf{o}_{i}} \right)\\
&= \varPhi \left( \mathbf{v,v} \right) - \varPhi \left( \mathbf{v},\sum_{i \in \varLambda_{r}} {\varPhi \left( \mathbf{o}_{i},\mathbf{v} \right)\mathbf{o}_{i}} \right) - \varPhi \left( \sum_{i \in \varLambda_{r}} {\varPhi \left( \mathbf{o}_{i},\mathbf{v} \right)\mathbf{o}_{i}},\mathbf{v} \right) \\
&\quad + \varPhi \left( \sum_{i \in \varLambda_{r}} {\varPhi \left( \mathbf{o}_{i},\mathbf{v} \right)\mathbf{o}_{i}},\sum_{i \in \varLambda_{r}} {\varPhi \left( \mathbf{o}_{i},\mathbf{v} \right)\mathbf{o}_{i}} \right)\\
&= \varPhi \left( \mathbf{v,v} \right) - \sum_{i \in \varLambda_{r}} {\varPhi \left( \mathbf{o}_{i},\mathbf{v} \right)\varPhi \left( \mathbf{v},\mathbf{o}_{i} \right)} - \sum_{i \in \varLambda_{r}} {\overline{\varPhi \left( \mathbf{o}_{i},\mathbf{v} \right)}\varPhi \left( \mathbf{o}_{i},\mathbf{v} \right)} \\
&\quad + \sum_{i \in \varLambda_{r}} {\sum_{j \in \varLambda_{r}} {\overline{\varPhi \left( \mathbf{o}_{i},\mathbf{v} \right)}\varPhi \left( \mathbf{o}_{j},\mathbf{v} \right)\varPhi \left( \mathbf{o}_{i},\mathbf{o}_{j} \right)}}\\
&= \varPhi \left( \mathbf{v,v} \right) - \sum_{i \in \varLambda_{r}} {\varPhi \left( \mathbf{o}_{i},\mathbf{v} \right)\overline{\varPhi \left( \mathbf{o}_{i},\mathbf{v} \right)}} - \sum_{i \in \varLambda_{r}} {\varPhi \left( \mathbf{o}_{i},\mathbf{v} \right)\overline{\varPhi \left( \mathbf{o}_{i},\mathbf{v} \right)}} \\
&\quad + \sum_{i,j \in \varLambda_{r}} {\varPhi \left( \mathbf{o}_{j},\mathbf{v} \right)\overline{\varPhi \left( \mathbf{o}_{i},\mathbf{v} \right)}\varPhi \left( \mathbf{o}_{i},\mathbf{o}_{j} \right)}\\
&= \varPhi \left( \mathbf{v,v} \right) - 2\sum_{i \in \varLambda_{r}} \left| \varPhi \left( \mathbf{o}_{i},\mathbf{v} \right) \right|^{2} \\
&\quad + \sum_{\scriptsize \begin{matrix} i,j \in \varLambda_{r} \\i \neq j \\\end{matrix}} {\varPhi \left( \mathbf{o}_{j},\mathbf{v} \right)\overline{\varPhi \left( \mathbf{o}_{i},\mathbf{v} \right)}\varPhi \left( \mathbf{o}_{i},\mathbf{o}_{j} \right)} + \sum_{\scriptsize \begin{matrix} i,j \in \varLambda_{r} \\i = j \end{matrix}} {\varPhi \left( \mathbf{o}_{j},\mathbf{v} \right)\overline{\varPhi \left( \mathbf{o}_{i},\mathbf{v} \right)}\varPhi \left( \mathbf{o}_{i},\mathbf{o}_{j} \right)}\\
&= \varPhi \left( \mathbf{v,v} \right) - 2\sum_{i \in \varLambda_{r}} \left| \varPhi \left( \mathbf{o}_{i},\mathbf{v} \right) \right|^{2} \\
&\quad + \sum_{\scriptsize \begin{matrix} i,j \in \varLambda_{r} \\i \neq j \\\end{matrix}} {\varPhi \left( \mathbf{o}_{j},\mathbf{v} \right)\overline{\varPhi \left( \mathbf{o}_{i},\mathbf{v} \right)} \cdot 0} + \sum_{\scriptsize \begin{matrix} i,j \in \varLambda_{r} \\i = j \end{matrix}} {\varPhi \left( \mathbf{o}_{j},\mathbf{v} \right)\overline{\varPhi \left( \mathbf{o}_{i},\mathbf{v} \right)} \cdot 1}\\
&= \varPhi \left( \mathbf{v,v} \right) - 2\sum_{i \in \varLambda_{r}} \left| \varPhi \left( \mathbf{o}_{i},\mathbf{v} \right) \right|^{2} + \sum_{i \in \varLambda_{r}} \left| \varPhi \left( \mathbf{o}_{i},\mathbf{v} \right) \right|^{2}\\
&= \varPhi \left( \mathbf{v,v} \right) - \sum_{i \in \varLambda_{r}} \left| \varPhi \left( \mathbf{o}_{i},\mathbf{v} \right) \right|^{2}
\end{align*}
次のようになる。
\begin{align*}
\sum_{i \in \varLambda_{r}} \left| \varPhi \left( \mathbf{o}_{i},\mathbf{v} \right) \right|^{2} &\leq \sum_{i \in \varLambda_{r}} \left| \varPhi \left( \mathbf{o}_{i},\mathbf{v} \right) \right|^{2} + \varPhi \left( \mathbf{v,v} \right) - \sum_{i \in \varLambda_{r}} \left| \varPhi \left( \mathbf{o}_{i},\mathbf{v} \right) \right|^{2}\\
&= \varPhi \left( \mathbf{v,v} \right) = {\varphi_{\varPhi }\left( \mathbf{v} \right)}^{2}
\end{align*}
よって、次式が成り立つ。
\begin{align*}
\sum_{i \in \varLambda_{r}} \left| \varPhi \left( \mathbf{o}_{i},\mathbf{v} \right) \right|^{2} \leq {\varphi_{\varPhi }\left( \mathbf{v} \right)}^{2}
\end{align*}\par
さらに、その内積空間$(V,\varPhi )$の正規直交基底$\left\langle \mathbf{o}_{i} \right\rangle_{i \in \varLambda_{n}}$が与えられたとき、$\forall\mathbf{v} \in V$に対し、次のようにおかれると、
\begin{align*}
\mathbf{v} = \sum_{i \in \varLambda_{n}} {a_{i}\mathbf{o}_{i}}
\end{align*}
定理\ref{2.3.6.17}より$\forall i \in \varLambda_{n}$に対し、$v_{i} = \varPhi \left( \mathbf{o}_{i},\mathbf{v} \right)$が成り立つかつ、次のようになるので、
\begin{align*}
\varphi_{\varPhi }\left( \mathbf{v} - \sum_{i \in \varLambda_{n}} {a_{i}\mathbf{o}_{i}} \right) = \varphi_{\varPhi }\left( \mathbf{v} - \mathbf{v} \right) = \varphi_{\varPhi }\left( \mathbf{0} \right) = 0
\end{align*}
次式が得られる。
\begin{align*}
\sum_{i \in \varLambda_{n}} \left| \varPhi \left( \mathbf{o}_{i},\mathbf{v} \right) \right|^{2} = \sum_{i \in \varLambda_{n}} \left| \varPhi \left( \mathbf{o}_{i},\mathbf{v} \right) \right|^{2} + {\varphi_{\varPhi }\left( \mathbf{v} - \sum_{i \in \varLambda_{n}} {a_{i}\mathbf{o}_{i}} \right)}^{2}
\end{align*}
そこで、上記の議論により次のようになることから、
\begin{align*}
{\varphi_{\varPhi }\left( \mathbf{v} - \sum_{i \in \varLambda_{n}} {a_{i}\mathbf{o}_{i}} \right)}^{2} = \varPhi \left( \mathbf{v,v} \right) - \sum_{i \in \varLambda_{n}} \left| \varPhi \left( \mathbf{o}_{i},\mathbf{v} \right) \right|^{2}
\end{align*}
次のようになる。
\begin{align*}
\sum_{i \in \varLambda_{n}} \left| \varPhi \left( \mathbf{o}_{i},\mathbf{v} \right) \right|^{2} &= \sum_{i \in \varLambda_{n}} \left| \varPhi \left( \mathbf{o}_{i},\mathbf{v} \right) \right|^{2} + \varPhi \left( \mathbf{v,v} \right) - \sum_{i \in \varLambda_{n}} \left| \varPhi \left( \mathbf{o}_{i},\mathbf{v} \right) \right|^{2}\\
&= \varPhi \left( \mathbf{v,v} \right) = {\varphi_{\varPhi }\left( \mathbf{v} \right)}^{2}
\end{align*}
よって、次式が成り立つ。
\begin{align*}
\sum_{i \in \varLambda_{n}} \left| \varPhi \left( \mathbf{o}_{i},\mathbf{v} \right) \right|^{2} = {\varphi_{\varPhi }\left( \mathbf{v} \right)}^{2}
\end{align*}
\end{proof}
%\hypertarget{ux6b63ux5c04ux5f71}{%
\subsubsection{正射影}%\label{ux6b63ux5c04ux5f71}}
\begin{dfn}
$K \subseteq \mathbb{C}$かつ$\dim V = n$なる内積空間$(V,\varPhi )$、そのvector空間$V$の部分空間$W$が与えられたとき、$V = W \oplus W^{\bot}$が成り立つのであった。このとき、そのvector空間$V$からその直和因子$W$への直和分解から定まる射影をその直交空間$W^{\bot}$に沿うそのvector空間$V$からその部分空間$W$への正射影という。特に、射影子でもあるような正射影を正射影子という。以下ここでは、その正射影を$P_{W}$と書くことにする。
\end{dfn}
\begin{thm}\label{2.3.7.15}
$K \subseteq \mathbb{C}$かつ$\dim V = n$なる内積空間$(V,\varPhi )$が与えられたとき、$\forall\mathbf{p} \in V$に対し、$\mathbf{p} \neq \mathbf{0}$が成り立つなら、そのvector空間$V$からその部分空間$\mathrm{span}\left\{ \mathbf{p} \right\}$への正射影$P_{\mathrm{span}\left\{ \mathbf{p} \right\}}$について、$\forall\mathbf{v} \in V$に対し、次式が成り立つ。
\begin{align*}
P_{\mathrm{span}\left\{ \mathbf{p} \right\}}\left( \mathbf{v} \right) = \frac{\varPhi \left( \mathbf{p},\mathbf{v} \right)}{\varPhi \left( \mathbf{p},\mathbf{p} \right)}\mathbf{p}
\end{align*}
\end{thm}
\begin{proof}
$K \subseteq \mathbb{C}$かつ$\dim V = n$なる内積空間$(V,\varPhi )$から誘導されるnorm空間$\left( V,\varphi_{\varPhi } \right)$が与えられたとき、$\forall\mathbf{p} \in V$に対し、$\mathbf{p} \neq \mathbf{0}$が成り立つなら、$\varphi_{\varPhi }\left( \mathbf{p} \right) \neq 0$が成り立つので、そのvector空間$V$からその部分空間$\mathrm{span}\left\{ \mathbf{p} \right\}$への正射影$P_{\mathrm{span}\left\{ \mathbf{p} \right\}}$について、$\mathbf{o}_{1} = \frac{\mathbf{p}}{\varphi_{\varPhi }\left( \mathbf{p} \right)}$とおかれれば、その組$\left\langle \mathbf{o}_{1} \right\rangle$がその部分空間$\mathrm{span}\left\{ \mathbf{p} \right\}$の正規直交基底をなす。そこで、定理\ref{2.3.6.12}より$\i \in \varLambda_{n} \setminus \left\{ 1 \right\}$なる適切な$n - 1$つのvectors$\mathbf{o}_{i}$を付け加えた組$\left\langle \mathbf{o}_{i} \right\rangle_{i \in \varLambda_{n}}$がそのvector空間$V$の正規直交基底をなすようにすることができるので、次のようになる。
\begin{align*}
V &= \mathrm{span}\left\{ \mathbf{o}_{i} \right\}_{i \in \varLambda_{n}}\\
&= \mathrm{span}\left\{ \mathbf{o}_{1} \right\} \oplus \mathrm{span}\left\{ \mathbf{o}_{i} \right\}_{i \in \varLambda_{n} \setminus \left\{ 1 \right\}}\\
&= \mathrm{span}\left\{ \mathbf{p} \right\} \oplus \mathrm{span}\left\{ \mathbf{o}_{i} \right\}_{i \in \varLambda_{n} \setminus \left\{ 1 \right\}}
\end{align*}
そこで、$\forall\mathbf{v} \in V$に対し、次のようにおかれると、
\begin{align*}
\mathbf{v} = \sum_{i \in \varLambda_{n}} {a_{i}\mathbf{o}_{i}}
\end{align*}
次のようになる。
\begin{align*}
P_{\mathrm{span}\left\{ \mathbf{p} \right\}}\left( \mathbf{v} \right) &= P_{\mathrm{span}\left\{ \mathbf{p} \right\}}\left( \sum_{i \in \varLambda_{n}} {a_{i}\mathbf{o}_{i}} \right)\\
&= P_{\mathrm{span}\left\{ \mathbf{p} \right\}}\left( a_{1}\mathbf{o}_{1} + \sum_{i \in \varLambda_{n} \setminus \left\{ 1 \right\}} {a_{i}\mathbf{o}_{i}} \right)\\
&= P_{\mathrm{span}\left\{ \mathbf{p} \right\}}\left( \frac{a_{1}}{\varphi_{\varPhi }\left( \mathbf{p} \right)}\mathbf{p} + \sum_{i \in \varLambda_{n} \setminus \left\{ 1 \right\}} {a_{i}\mathbf{o}_{i}} \right)\\
&= \frac{a_{1}}{\varphi_{\varPhi }\left( \mathbf{p} \right)}\mathbf{\ p}
\end{align*}
そこで、定理\ref{2.3.6.17}より$\forall i \in \varLambda_{n}$に対し、$v_{i} = \varPhi \left( \mathbf{o}_{i},\mathbf{v} \right)$が成り立つので、次のようになる。
\begin{align*}
P_{\mathrm{span}\left\{ \mathbf{p} \right\}}\left( \mathbf{v} \right) &= \frac{\varPhi \left( \mathbf{o}_{1},\mathbf{v} \right)}{\varphi_{\varPhi }\left( \mathbf{p} \right)}\mathbf{p}\\
&= \frac{\varPhi \left( \frac{\mathbf{p}}{\varphi_{\varPhi }\left( \mathbf{p} \right)},\mathbf{v} \right)}{\varphi_{\varPhi }\left( \mathbf{p} \right)}\mathbf{p}\\
&= \frac{\varPhi \left( \mathbf{p},\mathbf{v} \right)}{{\varphi_{\varPhi }\left( \mathbf{p} \right)}^{2}}\mathbf{p}\\
&= \frac{\varPhi \left( \mathbf{p},\mathbf{v} \right)}{\varPhi \left( \mathbf{p},\mathbf{p} \right)}\mathbf{p}
\end{align*}
\end{proof}
\begin{thm}\label{2.3.7.16}
$K \subseteq \mathbb{C}$かつ$\dim V = n$なる内積空間$(V,\varPhi )$から誘導されるnorm空間$\left( V,\varphi_{\varPhi } \right)$が与えられたとき、そのvector空間$V$からそのvector空間$V$の部分空間$W$への正射影$P_{W}$において、次のことが成り立つ。
\begin{itemize}
\item
  $\varphi_{\varPhi } \circ P_{W} \leq \varphi_{\varPhi }$が成り立つ。
\item
  $\forall\mathbf{v} \in V\forall\mathbf{w} \in W$に対し、$\varphi_{\varPhi }\left( \mathbf{v} - P_{W}\left( \mathbf{v} \right) \right) \leq \varphi_{\varPhi }\left( \mathbf{v} - \mathbf{w} \right)$が成り立つ。
\item
  $\forall\mathbf{v} \in V\forall\mathbf{w} \in W$に対し、$\varphi_{\varPhi }\left( \mathbf{v} - P_{W}\left( \mathbf{v} \right) \right) = \varphi_{\varPhi }\left( \mathbf{v} - \mathbf{w} \right)$が成り立つならそのときに限り、$\mathbf{w} = P_{W}\left( \mathbf{v} \right)$が成り立つ。
\end{itemize}
\end{thm}
\begin{proof}
$K \subseteq \mathbb{C}$かつ$\dim V = n$なる内積空間$(V,\varPhi )$から誘導されるnorm空間$\left( V,\varphi_{\varPhi } \right)$が与えられたとき、そのvector空間$V$からそのvector空間$V$の部分空間$W$への正射影$P_{W}$において、定理\ref{2.3.6.12}よりその部分空間$W$の正規直交基底$\left\langle \mathbf{o}_{i} \right\rangle_{i \in \varLambda_{r}}$が与えられれば、$\i \in \varLambda_{n} \setminus \varLambda_{r}$なる適切な$n - r$つのvectors$\mathbf{o}_{i}$を付け加えた組$\left\langle \mathbf{o}_{i} \right\rangle_{i \in \varLambda_{n}}$がそのvector空間$V$の正規直交基底をなすようにすることができる。$\forall\mathbf{v} \in V$に対し、次のようにおかれると、
\begin{align*}
\mathbf{v} = \sum_{i \in \varLambda_{n}} {a_{i}\mathbf{o}_{i}}
\end{align*}
次式が成り立つことから、
\begin{align*}
V = \mathrm{span}\left\{ \mathbf{o}_{i} \right\}_{i \in \varLambda_{n}} = \mathrm{span}\left\{ \mathbf{o}_{i} \right\}_{i \in \varLambda_{r}} \oplus \mathrm{span}\left\{ \mathbf{o}_{i} \right\}_{i \in \varLambda_{n} \setminus \varLambda_{r}},\ \ W = \mathrm{span}\left\{ \mathbf{o}_{i} \right\}_{i \in \varLambda_{r}}
\end{align*}
次のようになる。
\begin{align*}
P_{W}\left( \mathbf{v} \right) &= P_{W}\left( \sum_{i \in \varLambda_{n}} {a_{i}\mathbf{o}_{i}} \right)\\
&= P_{W}\left( \sum_{i \in \varLambda_{r}} {a_{i}\mathbf{o}_{i}} + \sum_{i \in \varLambda_{n} \setminus \varLambda_{r}} {a_{i}\mathbf{o}_{i}} \right)\\
&= \sum_{i \in \varLambda_{r}} {a_{i}\mathbf{o}_{i}}
\end{align*}
したがって、定理\ref{2.3.6.17}より次のようになることから、
\begin{align*}
{\varphi_{\varPhi } \circ P_{W}\left( \mathbf{v} \right)}^{2} &= {\varphi_{\varPhi }\left( P_{W}\left( \mathbf{v} \right) \right)}^{2}\\
&= {\varphi_{\varPhi }\left( \sum_{i \in \varLambda_{r}} {a_{i}\mathbf{o}_{i}} \right)}^{2}\\
&= \varPhi \left( \sum_{i \in \varLambda_{r}} {a_{i}\mathbf{o}_{i}},\sum_{i \in \varLambda_{r}} {a_{i}\mathbf{o}_{i}} \right)\\
&= \begin{pmatrix}
\overline{a_{1}} & \cdots & \overline{a_{r}} & 0 & \cdots & 0 \\
\end{pmatrix}\begin{pmatrix}
a_{1} \\
 \vdots \\
a_{r} \\
0 \\
 \vdots \\
0 \\
\end{pmatrix}\\
&= \sum_{i \in \varLambda_{r}} \left| a_{i} \right|^{2}\\
&\leq \sum_{i \in \varLambda_{n}} \left| a_{i} \right|^{2}\\
&= \begin{pmatrix}
\overline{a_{1}} & \overline{a_{2}} & \cdots & \overline{a_{n}} \\
\end{pmatrix}\begin{pmatrix}
a_{1} \\
a_{2} \\
 \vdots \\
a_{n} \\
\end{pmatrix}\\
&= \varPhi \left( \sum_{i \in \varLambda_{n}} {a_{i}\mathbf{o}_{i}},\sum_{i \in \varLambda_{n}} {a_{i}\mathbf{o}_{i}} \right)\\
&= {\varphi_{\varPhi }\left( \sum_{i \in \varLambda_{n}} {a_{i}\mathbf{o}_{i}} \right)}^{2} = {\varphi_{\varPhi }\left( \mathbf{v} \right)}^{2}
\end{align*}
$\varphi_{\varPhi } \circ P_{W} \leq \varphi_{\varPhi }$が成り立つ。\par
さらに、$\forall\mathbf{v},\mathbf{w} \in V$に対し、次のようにおかれると、
\begin{align*}
\mathbf{v} = \sum_{i \in \varLambda_{n}} {a_{i}\mathbf{o}_{i}},\ \ \mathbf{w} = \sum_{i \in \varLambda_{r}} {b_{i}\mathbf{o}_{i}}
\end{align*}
定理\ref{2.3.6.17}より次のようになる。
\begin{align*}
{\varphi_{\varPhi }\left( \mathbf{v} - P_{W}\left( \mathbf{v} \right) \right)}^{2} &= {\varphi_{\varPhi }\left( \sum_{i \in \varLambda_{n}} {a_{i}\mathbf{o}_{i}} - P_{W}\left( \sum_{i \in \varLambda_{n}} {a_{i}\mathbf{o}_{i}} \right) \right)}^{2}\\
&= {\varphi_{\varPhi }\left( \sum_{i \in \varLambda_{n}} {a_{i}\mathbf{o}_{i}} - \sum_{i \in \varLambda_{r}} {a_{i}\mathbf{o}_{i}} \right)}^{2}\\
&= {\varphi_{\varPhi }\left( \sum_{i \in \varLambda_{r}} {a_{i}\mathbf{o}_{i}} + \sum_{i \in \varLambda_{n} \setminus \varLambda_{r}} {a_{i}\mathbf{o}_{i}} - \sum_{i \in \varLambda_{r}} {a_{i}\mathbf{o}_{i}} \right)}^{2}\\
&= {\varphi_{\varPhi }\left( \sum_{i \in \varLambda_{n} \setminus \varLambda_{r}} {a_{i}\mathbf{o}_{i}} \right)}^{2}\\
&= \varPhi \left( \sum_{i \in \varLambda_{n} \setminus \varLambda_{r}} {a_{i}\mathbf{o}_{i}},\sum_{i \in \varLambda_{n} \setminus \varLambda_{r}} {a_{i}\mathbf{o}_{i}} \right)\\
&= \begin{pmatrix}
0 & \cdots & 0 & \overline{a_{r + 1}} & \cdots & \overline{a_{n}} \\
\end{pmatrix}\begin{pmatrix}
0 \\
 \vdots \\
0 \\
a_{r + 1} \\
 \vdots \\
a_{n} \\
\end{pmatrix}\\
&= \sum_{i \in \varLambda_{n} \setminus \varLambda_{r}} \left| a_{i} \right|^{2}\\
&{\varphi_{\varPhi }\left( \mathbf{v} - \mathbf{w} \right)}^{2} = {\varphi_{\varPhi }\left( \sum_{i \in \varLambda_{n}} {a_{i}\mathbf{o}_{i}} - \sum_{i \in \varLambda_{r}} {b_{i}\mathbf{o}_{i}} \right)}^{2}\\
&= {\varphi_{\varPhi }\left( \sum_{i \in \varLambda_{r}} {a_{i}\mathbf{o}_{i}} - \sum_{i \in \varLambda_{r}} {b_{i}\mathbf{o}_{i}} + \sum_{i \in \varLambda_{n} \setminus \varLambda_{r}} {a_{i}\mathbf{o}_{i}} \right)}^{2}\\
&= {\varphi_{\varPhi }\left( \sum_{i \in \varLambda_{r}} {\left( a_{i} - b_{i} \right)\mathbf{o}_{i}} + \sum_{i \in \varLambda_{n} \setminus \varLambda_{r}} {a_{i}\mathbf{o}_{i}} \right)}^{2}\\
&= \varPhi \left( \sum_{i \in \varLambda_{r}} {\left( a_{i} - b_{i} \right)\mathbf{o}_{i}} + \sum_{i \in \varLambda_{n} \setminus \varLambda_{r}} {a_{i}\mathbf{o}_{i}},\sum_{i \in \varLambda_{r}} {\left( a_{i} - b_{i} \right)\mathbf{o}_{i}} + \sum_{i \in \varLambda_{n} \setminus \varLambda_{r}} {a_{i}\mathbf{o}_{i}} \right)\\
&= \begin{pmatrix}
\overline{a_{1} - b_{1}} & \cdots & \overline{a_{r} - b_{r}} & \overline{a_{r + 1}} & \cdots & \overline{a_{n}} \\
\end{pmatrix}\begin{pmatrix}
a_{1} - b_{1} \\
 \vdots \\
a_{r} - b_{r} \\
a_{r + 1} \\
 \vdots \\
a_{n} \\
\end{pmatrix}\\
&= \sum_{i \in \varLambda_{r}} \left| a_{i} - b_{i} \right|^{2} + \sum_{i \in \varLambda_{n} \setminus \varLambda_{r}} \left| a_{i} \right|^{2}
\end{align*}
ここで、$\forall i \in \varLambda_{r}$に対し、$0 \leq \left| a_{i} - b_{i} \right|^{2}$が成り立つことに注意すれば、次のようになるので、
\begin{align*}
{\varphi_{\varPhi }\left( \mathbf{v} - P_{W}\left( \mathbf{v} \right) \right)}^{2} &= \sum_{i \in \varLambda_{n} \setminus \varLambda_{r}} \left| a_{i} \right|^{2}\\
&\leq \sum_{i \in \varLambda_{r}} \left| a_{i} - b_{i} \right|^{2} + \sum_{i \in \varLambda_{n} \setminus \varLambda_{r}} \left| a_{i} \right|^{2}\\
&= {\varphi_{\varPhi }\left( \mathbf{v} - \mathbf{w} \right)}^{2}
\end{align*}
よって、$\varphi_{\varPhi }\left( \mathbf{v} - P_{W}\left( \mathbf{v} \right) \right) \leq \varphi_{\varPhi }\left( \mathbf{v} - \mathbf{w} \right)$が成り立つ。\par
さらに、$\mathbf{w} \neq P_{W}\left( \mathbf{v} \right)$が成り立つなら、次式が成り立つことから、
\begin{align*}
\mathbf{w} = \sum_{i \in \varLambda_{r}} {b_{i}\mathbf{o}_{i}},\ \ P_{W}\left( \mathbf{v} \right) = \sum_{i \in \varLambda_{r}} {a_{i}\mathbf{o}_{i}}
\end{align*}
係数を比較して、$\exists i \in \varLambda_{r}$に対し、$a_{i} \neq b_{i}$が成り立つことから、$0 < \left| a_{i} - b_{i} \right|^{2}$が成り立つ。したがって、上記の議論により次のようになることから、
\begin{align*}
{\varphi_{\varPhi }\left( \mathbf{v} - P_{W}\left( \mathbf{v} \right) \right)}^{2} = \sum_{i \in \varLambda_{n} \setminus \varLambda_{r}} \left| a_{i} \right|^{2} < \sum_{i \in \varLambda_{r}} \left| a_{i} - b_{i} \right|^{2} + \sum_{i \in \varLambda_{n} \setminus \varLambda_{r}} \left| a_{i} \right|^{2} = {\varphi_{\varPhi }\left( \mathbf{v} - \mathbf{w} \right)}^{2}
\end{align*}
$\varphi_{\varPhi }\left( \mathbf{v} - P_{W}\left( \mathbf{v} \right) \right) < \varphi_{\varPhi }\left( \mathbf{v} - \mathbf{w} \right)$が成り立つ。対偶律により$\varphi_{\varPhi }\left( \mathbf{v} - P_{W}\left( \mathbf{v} \right) \right) = \varphi_{\varPhi }\left( \mathbf{v} - \mathbf{w} \right)$が成り立つなら、$\mathbf{w} = P_{W}\left( \mathbf{v} \right)$が成り立つ。逆に、$\mathbf{w} = P_{W}\left( \mathbf{v} \right)$が成り立つなら、明らかに$\varphi_{\varPhi }\left( \mathbf{v} - P_{W}\left( \mathbf{v} \right) \right) = \varphi_{\varPhi }\left( \mathbf{v} - \mathbf{w} \right)$が成り立つので、よって、$\forall\mathbf{v} \in V\forall\mathbf{w} \in W$に対し、$\varphi_{\varPhi }\left( \mathbf{v} - P_{W}\left( \mathbf{v} \right) \right) = \varphi_{\varPhi }\left( \mathbf{v} - \mathbf{w} \right)$が成り立つならそのときに限り、$\mathbf{w} = P_{W}\left( \mathbf{v} \right)$が成り立つ。
\end{proof}
\begin{thebibliography}{50}
  \bibitem{1}
  松坂和夫, 線型代数入門, 岩波書店, 1980. 新装版第2刷 p347-353 ISBN978-4-00-029872-8
\bibitem{2}
  木村一輝. "ベッセルの不等式・パーセバルの等式とは:有限のケースで証明". 趣味の大学数学. \url{https://math-fun.net/20210625/15588/} (2022-3-4 19:24 閲覧)
\end{thebibliography}
\end{document}
  