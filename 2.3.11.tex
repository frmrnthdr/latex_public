\documentclass[dvipdfmx]{jsarticle}
\setcounter{section}{3}
\setcounter{subsection}{10}
\usepackage{xr}
\externaldocument{2.1.5}
\externaldocument{2.2.2}
\externaldocument{2.2.3}
\externaldocument{2.3.4}
\externaldocument{2.3.5}
\externaldocument{2.3.6}
\externaldocument{2.3.7}
\externaldocument{2.3.8}
\externaldocument{2.3.9}
\externaldocument{2.3.10}
\usepackage{amsmath,amsfonts,amssymb,array,comment,mathtools,url,docmute}
\usepackage{longtable,booktabs,dcolumn,tabularx,mathtools,multirow,colortbl,xcolor}
\usepackage[dvipdfmx]{graphics}
\usepackage{bmpsize}
\usepackage{amsthm}
\usepackage{enumitem}
\setlistdepth{20}
\renewlist{itemize}{itemize}{20}
\setlist[itemize]{label=•}
\renewlist{enumerate}{enumerate}{20}
\setlist[enumerate]{label=\arabic*.}
\setcounter{MaxMatrixCols}{20}
\setcounter{tocdepth}{3}
\newcommand{\rotin}{\text{\rotatebox[origin=c]{90}{$\in $}}}
\newcommand{\amap}[6]{\text{\raisebox{-0.7cm}{\begin{tikzpicture} 
  \node (a) at (0, 1) {$\textstyle{#2}$};
  \node (b) at (#6, 1) {$\textstyle{#3}$};
  \node (c) at (0, 0) {$\textstyle{#4}$};
  \node (d) at (#6, 0) {$\textstyle{#5}$};
  \node (x) at (0, 0.5) {$\rotin $};
  \node (x) at (#6, 0.5) {$\rotin $};
  \draw[->] (a) to node[xshift=0pt, yshift=7pt] {$\textstyle{\scriptstyle{#1}}$} (b);
  \draw[|->] (c) to node[xshift=0pt, yshift=7pt] {$\textstyle{\scriptstyle{#1}}$} (d);
\end{tikzpicture}}}}
\newcommand{\twomaps}[9]{\text{\raisebox{-0.7cm}{\begin{tikzpicture} 
  \node (a) at (0, 1) {$\textstyle{#3}$};
  \node (b) at (#9, 1) {$\textstyle{#4}$};
  \node (c) at (#9+#9, 1) {$\textstyle{#5}$};
  \node (d) at (0, 0) {$\textstyle{#6}$};
  \node (e) at (#9, 0) {$\textstyle{#7}$};
  \node (f) at (#9+#9, 0) {$\textstyle{#8}$};
  \node (x) at (0, 0.5) {$\rotin $};
  \node (x) at (#9, 0.5) {$\rotin $};
  \node (x) at (#9+#9, 0.5) {$\rotin $};
  \draw[->] (a) to node[xshift=0pt, yshift=7pt] {$\textstyle{\scriptstyle{#1}}$} (b);
  \draw[|->] (d) to node[xshift=0pt, yshift=7pt] {$\textstyle{\scriptstyle{#2}}$} (e);
  \draw[->] (b) to node[xshift=0pt, yshift=7pt] {$\textstyle{\scriptstyle{#1}}$} (c);
  \draw[|->] (e) to node[xshift=0pt, yshift=7pt] {$\textstyle{\scriptstyle{#2}}$} (f);
\end{tikzpicture}}}}
\renewcommand{\thesection}{第\arabic{section}部}
\renewcommand{\thesubsection}{\arabic{section}.\arabic{subsection}}
\renewcommand{\thesubsubsection}{\arabic{section}.\arabic{subsection}.\arabic{subsubsection}}
\everymath{\displaystyle}
\allowdisplaybreaks[4]
\usepackage{vtable}
\theoremstyle{definition}
\newtheorem{thm}{定理}[subsection]
\newtheorem*{thm*}{定理}
\newtheorem{dfn}{定義}[subsection]
\newtheorem*{dfn*}{定義}
\newtheorem{axs}[dfn]{公理}
\newtheorem*{axs*}{公理}
\renewcommand{\headfont}{\bfseries}
\makeatletter
  \renewcommand{\section}{%
    \@startsection{section}{1}{\z@}%
    {\Cvs}{\Cvs}%
    {\normalfont\huge\headfont\raggedright}}
\makeatother
\makeatletter
  \renewcommand{\subsection}{%
    \@startsection{subsection}{2}{\z@}%
    {0.5\Cvs}{0.5\Cvs}%
    {\normalfont\LARGE\headfont\raggedright}}
\makeatother
\makeatletter
  \renewcommand{\subsubsection}{%
    \@startsection{subsubsection}{3}{\z@}%
    {0.4\Cvs}{0.4\Cvs}%
    {\normalfont\Large\headfont\raggedright}}
\makeatother
\makeatletter
\renewenvironment{proof}[1][\proofname]{\par
  \pushQED{\qed}%
  \normalfont \topsep6\p@\@plus6\p@\relax
  \trivlist
  \item\relax
  {
  #1\@addpunct{.}}\hspace\labelsep\ignorespaces
}{%
  \popQED\endtrivlist\@endpefalse
}
\makeatother
\renewcommand{\proofname}{\textbf{証明}}
\usepackage{tikz,graphics}
\usepackage[dvipdfmx]{hyperref}
\usepackage{pxjahyper}
\hypersetup{
 setpagesize=false,
 bookmarks=true,
 bookmarksdepth=tocdepth,
 bookmarksnumbered=true,
 colorlinks=false,
 pdftitle={},
 pdfsubject={},
 pdfauthor={},
 pdfkeywords={}}
\begin{document}
%\hypertarget{ux7279ux7570ux5024ux5206ux89e3}{%
\subsection{特異値分解}%\label{ux7279ux7570ux5024ux5206ux89e3}}
%\hypertarget{ux7ddaux5f62ux540cux578bux5199ux50cfux306eux6975ux5206ux89e3}{%
\subsubsection{線形同型写像の極分解}%\label{ux7ddaux5f62ux540cux578bux5199ux50cfux306eux6975ux5206ux89e3}}
\begin{thm}\label{2.3.11.1}
$K \subseteq \mathbb{C}$なる体$K$上の$n$次元内積空間$(V,\varPhi)$が与えられたとき、任意の線形写像$f:V \rightarrow V$に対し、線形写像たち$f \circ f^{*}$、$f^{*} \circ f$はいづれもHermite変換であり、これらから誘導されるHermite双線形形式たち$B_{f \circ f^{*}}$、$B_{f^{*} \circ f}$はいづれも半正値である。特に、その線形写像$f$が線形同型写像であるなら、それらのHermite双線形形式たち$B_{f \circ f^{*}}$、$B_{f^{*} \circ f}$はいづれも正値である。
\end{thm}
\begin{proof}
$K \subseteq \mathbb{C}$なる体$K$上の$n$次元内積空間$(V,\varPhi)$が与えられたとき、任意の線形写像$f:V \rightarrow V$に対し、定理\ref{2.3.8.6}より次のようになる。
\begin{align*}
\left( f \circ f^{*} \right)^{*} &= f^{**} \circ f^{*} = f \circ f^{*}\\
\left( f^{*} \circ f \right)^{*} &= f^{*} \circ f^{**} = f^{*} \circ f
\end{align*}
よって、線形写像たち$f \circ f^{*}$、$f^{*} \circ f$はいづれもHermite変換である。このとき、これらから誘導される双線形形式たち$B_{f \circ f^{*}}$、$B_{f^{*} \circ f}$について、定理\ref{2.3.10.2}よりこれらはHermite双線形形式であり、その内積空間$(V,\varPhi)$から誘導されるnorm空間$\left( V,\varphi_{\varPhi} \right)$について、$\forall\mathbf{v} \in V$に対し、次式が成り立つことから、
\begin{align*}
\varPhi\left( \mathbf{v},f \circ f^{*}\left( \mathbf{v} \right) \right) &= \varPhi\left( f^{*}\left( \mathbf{v} \right),f^{*}\left( \mathbf{v} \right) \right) = {\varphi_{\varPhi}\left( f^{*}\left( \mathbf{v} \right) \right)}^{2} = {\varphi_{\varPhi} \circ f^{*}\left( \mathbf{v} \right)}^{2} \geq 0\\
\varPhi\left( \mathbf{v},f^{*} \circ f\left( \mathbf{v} \right) \right) &= \varPhi\left( f\left( \mathbf{v} \right),f\left( \mathbf{v} \right) \right) = {\varphi_{\varPhi}\left( f\left( \mathbf{v} \right) \right)}^{2} = {\varphi_{\varPhi} \circ f\left( \mathbf{v} \right)}^{2} \geq 0
\end{align*}
これらから誘導されるHermite双線形形式たち$B_{f \circ f^{*}}$、$B_{f^{*} \circ f}$はいづれも半正値である。\par
特に、その線形写像$f$が線形同型写像であるなら、その線形写像$f^{*}$も線形同型写像で$\ker f = \ker f^{*} = \left\{ \mathbf{0} \right\}$が成り立つことから、$\forall\mathbf{v} \in V$に対し、$\mathbf{v} \neq \mathbf{0}$が成り立つなら、$f\left( \mathbf{v} \right) \neq \mathbf{0}$かつ$f^{*}\left( \mathbf{v} \right) \neq \mathbf{0}$が成り立つので、次式が成り立つことから、
\begin{align*}
\varPhi\left( \mathbf{v},f \circ f^{*}\left( \mathbf{v} \right) \right) = {\varphi_{\varPhi}\left( f^{*}\left( \mathbf{v} \right) \right)}^{2} > 0,\ \ \varPhi\left( \mathbf{v},f^{*} \circ f\left( \mathbf{v} \right) \right) = {\varphi_{\varPhi}\left( f\left( \mathbf{v} \right) \right)}^{2} > 0
\end{align*}
これらから誘導されるHermite双線形形式たち$B_{f \circ f^{*}}$、$B_{f^{*} \circ f}$はいづれも正値である。
\end{proof}
\begin{thm}\label{2.3.11.2}
体$\mathbb{C}$上の$n$次元内積空間$(V,\varPhi)$が与えられたとき、任意のHermite変換$f:V \rightarrow V$に対し、$i \in \varLambda_{n}$なるその線形写像$f$の固有値たちが$\lambda_{i}$とおかれれば、次のことが成り立つ。
\begin{itemize}
\item
  これから誘導されるHermite双線形形式が半正値であるなら、あるHermite変換$g$が一意的に存在して、これから誘導されるHermite双線形形式が半正値で$i \in \varLambda_{n}$なるそのHermite変換$g$の固有値たちが$\sqrt{\lambda_{i}}$と与えられ$f = g \circ g$が成り立つ。
\item
  これから誘導されるHermite双線形形式が正値であるなら、あるHermite変換$g$が一意的に存在して、これから誘導されるHermite双線形形式が正値で$i \in \varLambda_{n}$なるそのHermite変換$g$の固有値たちが$\sqrt{\lambda_{i}}$と与えられ$f = g \circ g$が成り立つ。
\item
  これから誘導されるHermite双線形形式が半負値であるなら、あるHermite変換$g$が一意的に存在して、これから誘導されるHermite双線形形式が半負値で$i \in \varLambda_{n}$なるそのHermite変換$g$の固有値たちが$- \sqrt{- \lambda_{i}}$と与えられ$f = g \circ g$が成り立つ。
\item
  これから誘導されるHermite双線形形式が負値であるなら、あるHermite変換$g$が一意的に存在して、これから誘導されるHermite双線形形式が負値で$i \in \varLambda_{n}$なるそのHermite変換$g$の固有値たちが$- \sqrt{- \lambda_{i}}$と与えられ$f = g \circ g$が成り立つ。
\end{itemize}
\end{thm}
\begin{proof}
体$\mathbb{C}$上の$n$次元内積空間$(V,\varPhi)$が与えられたとき、任意のHermite変換$f:V \rightarrow V$に対し、$i \in \varLambda_{n}$なるその線形写像$f$の固有値たちが$\lambda_{i}$とおかれれば、これから誘導されるHermite双線形形式$B_{f}$が半正値であるなら、定理\ref{2.3.10.3}より次のことが成り立つようなそのvector空間$V$の基底$\mathcal{B}$が存在する。
\begin{itemize}
\item
  その基底$\mathcal{B}$はその内積空間$(V,\varPhi)$の正規直交基底である。
\item
  その基底$\mathcal{B}$に関するそのHermite変換$f$の表現行列$[ f]_{\mathcal{B}}^{\mathcal{B}}$が対角行列となる。
\item
  その基底$\mathcal{B}$はそのHermite双線形形式$B_{f}$に関する直交基底をなす。
\end{itemize}
そこで、第$(i,i)$成分が$\sqrt{\lambda_{i}}$の対角行列をその基底$\mathcal{B}$に関するあるHermite変換$g$の表現行列$[ g]_{\mathcal{B}}^{\mathcal{B}}$となるようなそのHermite変換$g$が考えられれば、定理\ref{2.3.10.4}よりこれから誘導されるHermite双線形形式$B_{g}$が半正値であり、$\forall\mathbf{v} \in V$に対し、その基底$\mathcal{B}$に関する基底変換における線形同型写像$\varphi_{\mathcal{B}}$を用いて次のようになることから、
\begin{align*}
f\left( \mathbf{v} \right) &= \varphi_{\mathcal{B}} \circ \varphi_{\mathcal{B}}^{- 1} \circ f \circ \varphi_{\mathcal{B}} \circ \varphi_{\mathcal{B}}^{- 1}\left( \mathbf{v} \right)\\
&= \varphi_{\mathcal{B}}\left( \varphi_{\mathcal{B}}^{- 1} \circ f \circ \varphi_{\mathcal{B}}\left( \varphi_{\mathcal{B}}^{- 1}\left( \mathbf{v} \right) \right) \right)\\
&= \varphi_{\mathcal{B}}\left( [ f]_{\mathcal{B}}^{\mathcal{B}}\left( \varphi_{\mathcal{B}}^{- 1}\left( \mathbf{v} \right) \right) \right)\\
&= \varphi_{\mathcal{B}}\left( [ g]_{\mathcal{B}}^{\mathcal{B}}[ g]_{\mathcal{B}}^{\mathcal{B}}\left( \varphi_{\mathcal{B}}^{- 1}\left( \mathbf{v} \right) \right) \right)\\
&= \varphi_{\mathcal{B}}\left( \varphi_{\mathcal{B}}^{- 1} \circ g \circ \varphi_{\mathcal{B}} \circ \varphi_{\mathcal{B}}^{- 1} \circ g \circ \varphi_{\mathcal{B}}\left( \varphi_{\mathcal{B}}^{- 1}\left( \mathbf{v} \right) \right) \right)\\
&= \varphi_{\mathcal{B}} \circ \varphi_{\mathcal{B}}^{- 1} \circ g \circ \varphi_{\mathcal{B}} \circ \varphi_{\mathcal{B}}^{- 1} \circ g \circ \varphi_{\mathcal{B}} \circ \varphi_{\mathcal{B}}^{- 1}\left( \mathbf{v} \right)\\
&= g \circ g\left( \mathbf{v} \right)
\end{align*}
$f = g \circ g$が成り立つ。また、定理\ref{2.2.2.12}よりそのHermite変換$g$の固有値たちが$\sqrt{\lambda_{i}}$と与えられる。\par
そこで、あるHermite変換たち$g$、$h$が存在して、これらから誘導されるHermite双線形形式が半正値で$f = g \circ g = h \circ h$が成り立つとすれば、定理\ref{2.3.8.13}よりこれらは正規変換であるので、定理\ref{2.3.9.8}より次のようにspectrum分解されることができる。
\begin{align*}
g = \sum_{i \in \varLambda_{s}} {\mu_{i}P_{W_{g}\left( \mu_{i} \right)}},\ \ h = \sum_{i \in \varLambda_{t}} {\nu_{i}P_{W_{h}\left( \nu_{i} \right)}}
\end{align*}
このとき、次のようになることから、
\begin{align*}
f &= g \circ g\\
&= \sum_{i \in \varLambda_{s}} {\mu_{i}P_{W_{g}\left( \mu_{i} \right)}} \circ \sum_{i \in \varLambda_{s}} {\mu_{i}P_{W_{g}\left( \mu_{i} \right)}}\\
&= \sum_{i,j \in \varLambda_{s}} {\mu_{i}\mu_{j}P_{W_{g}\left( \mu_{i} \right)} \circ P_{W_{g}\left( \mu_{j} \right)}}\\
&= \sum_{\scriptsize \begin{matrix} i,j \in \varLambda_{s} \\i = j \\\end{matrix}} {\mu_{i}\mu_{j}P_{W_{g}\left( \mu_{i} \right)} \circ P_{W_{g}\left( \mu_{j} \right)}} + \sum_{\scriptsize \begin{matrix} i,j \in \varLambda_{s} \\i \neq j \\\end{matrix}} {\mu_{i}\mu_{j}P_{W_{g}\left( \mu_{i} \right)} \circ P_{W_{g}\left( \mu_{j} \right)}}\\
&= \sum_{i \in \varLambda_{s} } {\mu_{i}^{2}P_{W_{g}\left( \mu_{i} \right)}}
\end{align*}
定理\ref{2.3.9.12}よりその線形写像$f$の固有値たち$\lambda_{i}$が$\mu_{i}^{2}$と与えられる。同様にして、その線形写像$f$の固有値たち$\lambda_{i}$が$\nu_{i}^{2}$とも与えられることが示される。そこで、固有値たちの順序に注意すれば、$\lambda_{i} = \mu_{i}^{2} = \nu_{i}^{2}$が成り立つとしてもよい。そのHermite変換たち$g$、$h$は半正値なので、定理\ref{2.3.10.4}より$\mu_{i} = \nu_{i}$が成り立つ。このとき、定理\ref{2.3.9.12}より次のようにspectrum分解されるなら、
\begin{align*}
g = \sum_{i \in \varLambda_{s}} {\mu_{i}P_{W_{g}\left( \mu_{i} \right)}},\ \ h = \sum_{i \in \varLambda_{t}} {\nu_{i}P_{W_{h}\left( \nu_{i} \right)}}
\end{align*}
$g = h$が成り立つ。\par
以下、同様に第$(i,i)$成分が$\sqrt{\lambda_{i}}$、$\sqrt{\lambda_{i}}$、$- \sqrt{- \lambda_{i}}$、$- \sqrt{- \lambda_{i}}$の対角行列をその基底$\mathcal{B}$に関するあるHermite変換$g$の表現行列$[ g]_{\mathcal{B}}^{\mathcal{B}}$となるようなそのHermite変換$g$が考えられれば、それぞれ次のことが成り立つことが示される。
\begin{itemize}
\item
  これから誘導されるHermite双線形形式が半正値であるなら、あるHermite変換$g$が一意的に存在して、これから誘導されるHermite双線形形式が半正値で$i \in \varLambda_{n}$なるそのHermite変換$g$の固有値たちが$\sqrt{\lambda_{i}}$と与えられ$f = g \circ g$が成り立つ。
\item
  これから誘導されるHermite双線形形式が正値であるなら、あるHermite変換$g$が一意的に存在して、これから誘導されるHermite双線形形式が正値で$i \in \varLambda_{n}$なるそのHermite変換$g$の固有値たちが$\sqrt{\lambda_{i}}$と与えられ$f = g \circ g$が成り立つ。
\item
  これから誘導されるHermite双線形形式が半負値であるなら、あるHermite変換$g$が一意的に存在して、これから誘導されるHermite双線形形式が半負値で$i \in \varLambda_{n}$なるそのHermite変換$g$の固有値たちが$- \sqrt{- \lambda_{i}}$と与えられ$f = g \circ g$が成り立つ。
\item
  これから誘導されるHermite双線形形式が負値であるなら、あるHermite変換$g$が一意的に存在して、これから誘導されるHermite双線形形式が負値で$i \in \varLambda_{n}$なるそのHermite変換$g$の固有値たちが$- \sqrt{- \lambda_{i}}$と与えられ$f = g \circ g$が成り立つ。
\end{itemize}
\end{proof}
\begin{thm}[左極分解]\label{2.3.11.3}
体$\mathbb{C}$上の$n$次元内積空間$(V,\varPhi)$が与えられたとき、任意の線形同型写像$f:V \rightarrow V$に対し、あるHermite変換$H$と等長変換$U$が一意的に存在して、そのHermite変換$H$から誘導されるHermite双線形形式が正値で$f = H \circ U$が成り立つ。\par
この定理をその線形同型写像$f$の左極分解という。
\end{thm}
\begin{proof}
体$\mathbb{C}$上の$n$次元内積空間$(V,\varPhi)$が与えられたとき、任意の線形同型写像$f:V \rightarrow V$に対し、定理\ref{2.3.11.1}よりその線形写像$f \circ f^{*}$はいづれもHermite変換であり、これらから誘導されるHermite双線形形式たち$B_{f \circ f^{*}}$は正値である。定理\ref{2.3.11.2}よりあるHermite変換$H$が一意的に存在して、これから誘導されるHermite双線形形式が正値で$f \circ f^{*} = H \circ H$が成り立つ。ここで、定理\ref{2.3.10.4}よりそのHermite変換の固有値たちいづれも正の実数なので、定理\ref{2.3.8.13}よりそのHermite変換$H$は正規変換で定理\ref{2.3.9.4}、即ち、Toeplitzの定理よりある正規直交基底$\mathcal{B}$が存在して、これに関する表現行列$[ H]_{\mathcal{B}}^{\mathcal{B}}$が対角行列となる。このとき、定理\ref{2.2.2.12}よりその対角行列$[ H]_{\mathcal{B}}^{\mathcal{B}}$の対角成分がそのHermite変換$H$の固有値でありいづれも正の実数なので、その対角行列の対角成分の逆数を対角成分とする対角行列が考えられれば、これがその対角行列$[ H]_{\mathcal{B}}^{\mathcal{B}}$の逆行列となるので、その対角行列$[ H]_{\mathcal{B}}^{\mathcal{B}}$の逆行列${[ H]_{\mathcal{B}}^{\mathcal{B}}}^{- 1}$が存在する。このとき、定理\ref{2.1.5.16}よりそのHermite変換$H$は線形同型写像となって$\left[ H^{- 1} \right]_{\mathcal{B}}^{\mathcal{B}} = {[ H]_{\mathcal{B}}^{\mathcal{B}}}^{- 1}$が成り立つので、$U = H^{- 1} \circ f$とおかれれば、定理\ref{2.3.8.6}より次式が成り立つ。
\begin{align*}
U \circ U^{*} &= H^{- 1} \circ f \circ \left( H^{- 1} \circ f \right)^{*}\\
&= H^{- 1} \circ \left( f \circ f^{*} \right) \circ {H^{- 1}}^{*}\\
&= H^{- 1} \circ (H \circ H) \circ {H^{*}}^{- 1}\\
&= H^{- 1} \circ H \circ H \circ H^{- 1} = I_{V}\\
U^{*} \circ U &= \left( H^{- 1} \circ f \right)^{*} \circ H^{- 1} \circ f\\
&= f^{*} \circ {H^{- 1}}^{*} \circ H^{- 1} \circ f\\
&= f^{*} \circ {H^{*}}^{- 1} \circ H^{- 1} \circ f\\
&= f^{*} \circ H^{- 1} \circ H^{- 1} \circ f\\
&= f^{*} \circ (H \circ H)^{- 1} \circ f\\
&= f^{*} \circ \left( f \circ f^{*} \right)^{- 1} \circ f\\
&= f^{*} \circ {f^{*}}^{- 1} \circ f^{- 1} \circ f = I_{V}
\end{align*}
このとき、定理\ref{2.3.8.7}よりその線形写像$U$は等長変換である。\par
さらに、$f = H \circ T = H \circ U$なる等長変換たち$T$、$U$が存在するなら、そのHermite変換$H$は線形同型写像でもあるので、次のようになる。
\begin{align*}
T &= H^{- 1} \circ H \circ T\\
&= H^{- 1} \circ f\\
&= H^{- 1} \circ H \circ U = U
\end{align*}
あるHermite変換$H$と等長変換$U$が一意的に存在して、そのHermite変換$H$から誘導されるHermite双線形形式が正値で$f = H \circ U$が成り立つ。
\end{proof}
\begin{thm}[右極分解]\label{2.3.11.4}
体$\mathbb{C}$上の$n$次元内積空間$(V,\varPhi)$が与えられたとき、任意の線形同型写像$f:V \rightarrow V$に対し、あるHermite変換$H$と等長変換$U$が一意的に存在して、そのHermite変換$H$から誘導されるHermite双線形形式が正値で$f = U \circ H$が成り立つ。\par
この定理をその線形同型写像$f$の右極分解という。
\end{thm}
\begin{proof}定理\ref{2.3.11.3}と同様にして示される。
\end{proof}
%\hypertarget{ux7ddaux5f62ux540cux578bux5199ux50cfux306eux6975ux5206ux89e3-1}{%
\subsubsection{線形同型写像の極分解}%\label{ux7ddaux5f62ux540cux578bux5199ux50cfux306eux6975ux5206ux89e3-1}}
\begin{thm}\label{2.3.11.5}
$K \subseteq \mathbb{C}$なる体$K$上の$n$次元内積空間$(V,\varPhi)$が与えられたとき、任意の線形写像$f:V \rightarrow V$に対し、$\ker{f^{*} \circ f} = \ker f$が成り立つ。
\end{thm}
\begin{proof}
$K \subseteq \mathbb{C}$なる体$K$上の$n$次元内積空間$(V,\varPhi)$が与えられたとき、任意の線形写像$f:V \rightarrow V$に対し、もちろん、$\ker f \subseteq \ker{f^{*} \circ f}$が成り立つ。一方で、$\forall\mathbf{v} \in \ker{f^{*} \circ f}$に対し、その内積空間$(V,\varPhi)$から誘導されるnorm空間$\left( V,\varphi_{\varPhi} \right)$について、次のようになることから、
\begin{align*}
\varphi_{\varPhi}\left( f\left( \mathbf{v} \right) \right) &= \sqrt{\varPhi\left( f\left( \mathbf{v} \right),f\left( \mathbf{v} \right) \right)}\\
&= \sqrt{\varPhi\left( f^{*} \circ f\left( \mathbf{v} \right),\mathbf{v} \right)}\\
&= \sqrt{\varPhi\left( \mathbf{0},\mathbf{v} \right)} = 0
\end{align*}
$f\left( \mathbf{v} \right) = 0$が成り立ち、したがって、$\mathbf{v} \in \ker f$が得られる。よって、$\ker{f^{*} \circ f} = \ker f$が成り立つ。
\end{proof}
\begin{thm}[左極分解]\label{2.3.11.6}
体$\mathbb{C}$上の$n$次元内積空間$(V,\varPhi)$が与えられたとき、任意の線形写像$f:V \rightarrow V$に対し、あるHermite変換$H$と等長変換$U$が存在して、そのHermite変換$H$から誘導されるHermite双線形形式が正値で$f = H \circ U$が成り立つ。\par
この定理をその線形写像$f$の左極分解という。
\end{thm}
\begin{proof}
体$\mathbb{C}$上の$n$次元内積空間$(V,\varPhi)$が与えられたとき、任意の線形写像$f:V \rightarrow V$に対し、定理\ref{2.3.11.1}よりその線形写像$f \circ f^{*}$はHermite変換であり、これらから誘導されるHermite双線形形式たち$B_{f \circ f^{*}}$は半正値である。このとき、定理\ref{2.3.11.2}よりあるHermite変換$H$が一意的に存在して、これから誘導されるHermite双線形形式が半正値で$f \circ f^{*} = H \circ H$が成り立つ。さらに、定理\ref{2.3.8.13}よりこのHermite変換$H$は正規変換であるので、定理\ref{2.3.9.4}、即ち、Toeplitzの定理よりある正規直交基底$\mathcal{B}$が存在して、これに関するそのHermite変換$H$の表現行列$[ H]_{\mathcal{B}}^{\mathcal{B}}$が対角行列となる。このとき、定理\ref{2.2.2.12}よりその対角行列$[ H]_{\mathcal{B}}^{\mathcal{B}}$の対角成分$\lambda_{i}$がそのHermite変換$H$の固有値であり定理\ref{2.3.10.4}よりいづれも非負実数なので、そのHermite双線形形式$B_{H}$の符号が定理\ref{2.3.5.22}より$(\pi,0)$と与えられたらば、$\forall i \in \varLambda_{\pi}$に対し、$0 < \lambda_{i}$、$\forall i \in \varLambda_{n} \setminus \varLambda_{\pi}$に対し、$0 = \lambda_{i}$と与えられるとしてもよい。以下、そのHermite変換$H$の互いに異なる固有値たちの族が$\left\{ \sqrt{\lambda_{i'}'} \right\}_{i' \in \varLambda_{s}}$と与えられよう。$f \circ f^{*} = H \circ H$が成り立つことから定理\ref{2.2.3.7}、即ち、Frobeniusの定理よりそのHermite変換$f \circ f^{*}$の互いに異なる固有値たちの族が$\left\{ \lambda_{i'}' \right\}_{i' \in \varLambda_{s}}$と与えられる。それらのHermite変換たち$f \circ f^{*}$、$H$は定理\ref{2.3.8.13}より正規変換であるので、定理\ref{2.3.9.8}より次のようにspectrum分解されることができる。
\begin{align*}
f \circ f^{*} = \sum_{i' \in \varLambda_{s}} {\lambda_{i'}'P_{W_{f \circ f^{*}}\left( \lambda_{i'}' \right)}},\ \ H = \sum_{i' \in \varLambda_{s}} {\sqrt{\lambda_{i'}'}P_{W_{H}\left( \sqrt{\lambda_{i'}'} \right)}}
\end{align*}\par
$\mathcal{B} =\left\langle \mathbf{o}_{i} \right\rangle_{i \in \varLambda_{n}}$とおかれれば、$\mathbf{o}_{i} \in W_{H}\left( \sqrt{\lambda_{i'}'} \right)$が成り立つようなその基底$\mathcal{B}$をなすvector$\mathbf{o}_{i}$について、$f \circ f^{*} = H \circ H$が成り立つことから定理\ref{2.2.3.8}より$\mathbf{o}_{i} \in W_{f \circ f^{*}}\left( \lambda_{i'}' \right)$が成り立つ。$\forall i \in \varLambda_{\pi}$に対し、$\mathbf{p}_{i} = \frac{f^{*}\left( \mathbf{o}_{i} \right)}{\sqrt{\lambda_{i}}}$とおかれれば、部分空間$\mathrm{span}\left\{ \mathbf{p}_{i} \right\}_{i \in \varLambda_{\pi}}$について、$\forall i,j \in \varLambda_{\pi}$に対し、$i \neq j$が成り立つなら、次のようになるかつ、
\begin{align*}
\varPhi\left( \mathbf{p}_{i},\mathbf{p}_{j} \right) &= \varPhi\left( \frac{f^{*}\left( \mathbf{o}_{i} \right)}{\sqrt{\lambda_{i}}},\frac{f^{*}\left( \mathbf{o}_{j} \right)}{\sqrt{\lambda_{j}}} \right)\\
&= \frac{\varPhi\left( f^{*}\left( \mathbf{o}_{i} \right),f^{*}\left( \mathbf{o}_{j} \right) \right)}{\sqrt{\lambda_{i}\lambda_{j}}}\\
&= \frac{\varPhi\left( \mathbf{o}_{i},f^{**} \circ f^{*}\left( \mathbf{o}_{j} \right) \right)}{\sqrt{\lambda_{i}\lambda_{j}}}\\
&= \frac{\varPhi\left( \mathbf{o}_{i},f \circ f^{*}\left( \mathbf{o}_{j} \right) \right)}{\sqrt{\lambda_{i}\lambda_{j}}}\\
&= \frac{\varPhi\left( \mathbf{o}_{i},\lambda_{j}\mathbf{o}_{j} \right)}{\sqrt{\lambda_{i}\lambda_{j}}}\\
&= \frac{\lambda_{j}\varPhi\left( \mathbf{o}_{i},\mathbf{o}_{j} \right)}{\sqrt{\lambda_{i}\lambda_{j}}} = 0
\end{align*}
$\forall i \in \varLambda_{\pi}$に対し、次のようになるので、
\begin{align*}
\varphi_{\varPhi}\left( \mathbf{p}_{i} \right) &= \sqrt{\varPhi\left( \frac{f^{*}\left( \mathbf{o}_{i} \right)}{\sqrt{\lambda_{i}}},\frac{f^{*}\left( \mathbf{o}_{i} \right)}{\sqrt{\lambda_{i}}} \right)}\\
&= \sqrt{\frac{\varPhi\left( f^{*}\left( \mathbf{o}_{i} \right),f^{*}\left( \mathbf{o}_{i} \right) \right)}{\lambda_{i}}}\\
&= \sqrt{\frac{\varPhi\left( \mathbf{o}_{i},f^{**} \circ f^{*}\left( \mathbf{o}_{i} \right) \right)}{\lambda_{i}}}\\
&= \sqrt{\frac{\varPhi\left( \mathbf{o}_{i},f \circ f^{*}\left( \mathbf{o}_{i} \right) \right)}{\lambda_{i}}}\\
&= \sqrt{\frac{\varPhi\left( \mathbf{o}_{i},\lambda_{i}\mathbf{o}_{i} \right)}{\lambda_{i}}}\\
&= \sqrt{\frac{\lambda_{i}\varPhi\left( \mathbf{o}_{i},\mathbf{o}_{i} \right)}{\lambda_{i}}}\\
&= \sqrt{\varPhi\left( \mathbf{o}_{i},\mathbf{o}_{i} \right)}\\
&= \varphi_{\varPhi}\left( \mathbf{o}_{i} \right) = 1
\end{align*}
その組$\left\langle \mathbf{p}_{i} \right\rangle_{i \in \varLambda_{\pi}}$はその部分空間$\mathrm{span}\left\{ \mathbf{p}_{i} \right\}_{i \in \varLambda_{\pi}}$の正規直交基底をなす。そこで、定理\ref{2.3.6.12}よりその組$\left\langle \mathbf{p}_{i} \right\rangle_{i \in \varLambda_{n}}$がそのvector空間$V$の正規直交基底をなすようにすることができる。これが$\mathcal{C}$とおかれよう。\par
このとき、それらの基底たち$\mathcal{B}$、$\mathcal{C}$に関する基底変換における線形同型写像たち$\varphi_{\mathcal{B}}$、$\varphi_{\mathcal{C}}$を用いれば、$\mathbf{p}_{i} = \varphi_{\mathcal{C}} \circ \varphi_{\mathcal{B}}^{- 1}\left( \mathbf{o}_{i} \right)$が成り立ち、定理\ref{2.3.7.5}よりその線形写像$\varphi_{\mathcal{C}} \circ \varphi_{\mathcal{B}}^{- 1}$は等長変換である。これが$U^{*}$とおかれよう。\par
$\mathbf{o}_{i} \in W_{H}\left( \sqrt{\lambda_{i'}'} \right)$が成り立つようなその基底$\mathcal{B}$をなすvector$\mathbf{o}_{i}$について、$f \circ f^{*} = H \circ H$が成り立つことから定理\ref{2.2.3.8}より$\mathbf{o}_{i} \in W_{f \circ f^{*}}\left( \lambda_{i'}' \right)$が成り立つ。$\lambda_{i}' \neq 0$のとき、次のようになり、
\begin{align*}
U^{*} \circ H\left( \mathbf{o}_{i} \right) &= U^{*} \circ \sum_{i \in \varLambda_{s}} {\sqrt{\lambda_{i}'}P_{W_{H}\left( \sqrt{\lambda_{i}'} \right)}}\left( \mathbf{o}_{i} \right)\\
&= U^{*}\left( \sum_{i \in \varLambda_{s}} {\sqrt{\lambda_{i}'}P_{W_{H}\left( \sqrt{\lambda_{i}'} \right)}}\left( \mathbf{o}_{i} \right) \right)\\
&= U^{*}\left( \sqrt{\lambda_{i}'}P_{W_{H}\left( \sqrt{\lambda_{i}'} \right)}\left( \mathbf{o}_{i} \right) \right)\\
&= U^{*}\left( \sqrt{\lambda_{i}'}\mathbf{o}_{i} \right)\\
&= \sqrt{\lambda_{i}'}U^{*}\left( \mathbf{o}_{i} \right)\\
&= \sqrt{\lambda_{i}'}\frac{f^{*}\left( \mathbf{o}_{i} \right)}{\sqrt{\lambda_{i}'}} = f^{*}\left( \mathbf{o}_{i} \right)
\end{align*}
$\lambda_{i}' = 0$のとき、次のようになることから、
\begin{align*}
W_{H}(0) &= \ker\left( 0I_{V} - H \right)\\
&= \ker H\\
&= \ker{H^{*} \circ H}\\
&= \ker{H \circ H}\\
&= \ker{f \circ f^{*}}\\
&= \ker{f^{**} \circ f^{*}}\\
&= \ker f^{*}
\end{align*}
$\mathbf{o}_{i} \in \ker f^{*}$が成り立つので、次のようになる。
\begin{align*}
U^{*} \circ H\left( \mathbf{o}_{i} \right) = U^{*}\left( 0\mathbf{o}_{i} \right) = \mathbf{0} = f^{*}\left( \mathbf{o}_{i} \right)
\end{align*}\par
以上の議論により、$\forall\mathbf{v} \in V$に対し、$\mathbf{v} = \sum_{i \in \varLambda_{n}} {a_{i}\mathbf{o}_{i}}$とおかれれば、次のようになることから、
\begin{align*}
U^{*} \circ H\left( \mathbf{v} \right) &= U^{*} \circ H\left( \sum_{i \in \varLambda_{n}} {a_{i}\mathbf{o}_{i}} \right)\\
&= \sum_{i \in \varLambda_{n}} {a_{i}U^{*} \circ H\left( \mathbf{o}_{i} \right)}\\
&= \sum_{i \in \varLambda_{n}} {a_{i}f^{*}\left( \mathbf{o}_{i} \right)}\\
&= f^{*}\left( \sum_{i \in \varLambda_{n}} {a_{i}\mathbf{o}_{i}} \right) = f^{*}\left( \mathbf{v} \right)
\end{align*}
次のようになる。
\begin{align*}
f = f^{**} = \left( U^{*} \circ H \right)^{*} = H^{*} \circ U^{**} = H \circ U
\end{align*}\par
定理\ref{2.3.8.7}より$U^{**} = {U^{*}}^{- 1}$が成り立つので、次のようになる。
\begin{align*}
U^{*} = U^{***} = {{U^{*}}^{- 1}}^{*} = {U^{- 1}}^{**} = U^{- 1}
\end{align*}
再び定理\ref{2.3.8.7}よりその線形写像$U$も等長変換である。よって、任意の線形写像$f:V \rightarrow V$に対し、あるHermite変換$H$と等長変換$U$が存在して、そのHermite変換$H$から誘導されるHermite双線形形式が正値で$f = H \circ U$が成り立つ。
\end{proof}
\begin{thm}[右極分解]\label{2.3.11.7}
体$\mathbb{C}$上の$n$次元内積空間$(V,\varPhi)$が与えられたとき、任意の線形写像$f:V \rightarrow V$に対し、あるHermite変換$H$と等長変換$U$が存在して、そのHermite変換$H$から誘導されるHermite双線形形式が正値で$f = U \circ H$が成り立つ。\par
この定理をその線形写像$f$の右極分解という。
\end{thm}
\begin{proof}
体$\mathbb{C}$上の$n$次元内積空間$(V,\varPhi)$が与えられたとき、任意の線形写像$f:V \rightarrow V$に対し、あるHermite変換$H$と等長変換$U^{*}$が存在して、そのHermite変換$H$から誘導されるHermite双線形形式が正値で$f^{*} = H \circ U^{*}$が成り立つ。したがって、次のようになる。
\begin{align*}
f = f^{**} = \left( H \circ U^{*} \right)^{*} = U^{**} \circ H^{*} = U \circ H
\end{align*}
定理\ref{2.3.8.7}より$U^{**} = {U^{*}}^{- 1}$が成り立つので、次のようになる。
\begin{align*}
U^{*} = U^{***} = {{U^{*}}^{- 1}}^{*} = {U^{- 1}}^{**} = U^{- 1}
\end{align*}
再び定理\ref{2.3.8.7}よりその線形写像$U$も等長変換である。よって、あるHermite変換$H$と等長変換$U$が存在して、そのHermite変換$H$から誘導されるHermite双線形形式が正値で$f = U \circ H$が成り立つ。
\end{proof}
%\hypertarget{ux7279ux7570ux5024ux5206ux89e3-1}{%
\subsubsection{特異値分解}%\label{ux7279ux7570ux5024ux5206ux89e3-1}}
\begin{thm}[特異値分解]\label{2.3.11.8}
体$\mathbb{C}$上の$n$次元内積空間$(V,\varPhi)$が与えられたとき、任意の線形写像$f:V \rightarrow V$に対し、ある等長写像たち$S:K^{n} \rightarrow V$、$T:K^{n} \rightarrow V$が存在して、線形写像$S^{- 1} \circ f \circ T:K^{n} \rightarrow K^{n}$に対応する行列が対角成分が非負実数の対角行列となる。\par
この定理をその線形写像$f$の特異値分解という。
\end{thm}
\begin{proof}
体$\mathbb{C}$上の$n$次元内積空間$(V,\varPhi)$が与えられたとき、任意の線形写像$f:V \rightarrow V$に対し、定理\ref{2.3.11.6}、即ち、左極分解よりあるHermite変換$H$と等長変換$U$が存在して、そのHermite変換$H$から誘導されるHermite双線形形式が正値で$f = H \circ U$が成り立つ。このとき、$H = f \circ U^{*}$が成り立ち、定理\ref{2.3.8.13}よりこのHermite変換$H$は正規変換であるので、定理\ref{2.3.9.4}、即ち、Toeplitzの定理よりある正規直交基底$\mathcal{B}$が存在して、これに関するそのHermite変換$H$の表現行列$[ H]_{\mathcal{B}}^{\mathcal{B}}$が対角行列となる。このとき、定理\ref{2.2.2.12}よりその対角行列$[ H]_{\mathcal{B}}^{\mathcal{B}}$の対角成分$\lambda_{i}$がそのHermite変換$H$の固有値であり定理\ref{2.3.10.4}よりいづれも非負実数となる。そこで、次式が成り立つことから、
\begin{align*}
\varphi_{\mathcal{B}}^{- 1} \circ H \circ \varphi_{\mathcal{B}} = \varphi_{\mathcal{B}}^{- 1} \circ f \circ U^{*} \circ \varphi_{\mathcal{B}}
\end{align*}
$S = \varphi_{\mathcal{B}}$、$T = U^{*} \circ \varphi_{\mathcal{B}}$とおかれれば、これらは線形同型写像でその線形写像$S^{- 1} \circ f \circ T:K^{n} \rightarrow K^{n}$に対応する行列がその行列$[ H]_{\mathcal{B}}^{\mathcal{B}}$であり対角成分が非負実数の対角行列となる。\par
そこで、標準内積空間$\left( \mathbb{C}^{n},\left\langle \bullet \middle| \bullet \right\rangle \right)$について、その基底$\mathcal{B}$に関するその内積$\varPhi$の表現行列が単位行列となっていることに注意すれば、$\forall\mathbf{v},\mathbf{w} \in V$に対し、定理\ref{2.3.4.8}、定理\ref{2.3.8.7}より次のようになるので、
\begin{align*}
\varPhi\left( S\left( \mathbf{v} \right),S\left( \mathbf{w} \right) \right) &= \varPhi\left( \varphi_{\mathcal{B}}\left( \mathbf{v} \right),\varphi_{\mathcal{B}}\left( \mathbf{w} \right) \right)\\
&=^{t}\overline{\varphi_{\mathcal{B}}^{- 1}\left( \varphi_{\mathcal{B}}\left( \mathbf{v} \right) \right)}[\varPhi]_{\mathcal{B}}\varphi_{\mathcal{B}}^{- 1}\left( \varphi_{\mathcal{B}}\left( \mathbf{w} \right) \right)\\
&=^{t}\overline{\varphi_{\mathcal{B}}^{- 1} \circ \varphi_{\mathcal{B}}\left( \mathbf{v} \right)}I_{n}\varphi_{\mathcal{B}}^{- 1} \circ \varphi_{\mathcal{B}}\left( \mathbf{w} \right)\\
&=^{t}\overline{\mathbf{v}}\mathbf{w}=\left\langle \mathbf{v} \middle| \mathbf{w} \right\rangle\\
\varPhi\left( T\left( \mathbf{v} \right),T\left( \mathbf{w} \right) \right) &= \varPhi\left( U^{*} \circ \varphi_{\mathcal{B}}\left( \mathbf{v} \right),U^{*} \circ \varphi_{\mathcal{B}}\left( \mathbf{w} \right) \right)\\
&= \varPhi\left( \varphi_{\mathcal{B}}\left( \mathbf{v} \right),U^{**} \circ U^{*} \circ \varphi_{\mathcal{B}}\left( \mathbf{w} \right) \right)\\
&= \varPhi\left( \varphi_{\mathcal{B}}\left( \mathbf{v} \right),U \circ U^{- 1} \circ \varphi_{\mathcal{B}}\left( \mathbf{w} \right) \right)\\
&= \varPhi\left( \varphi_{\mathcal{B}}\left( \mathbf{v} \right),\varphi_{\mathcal{B}}\left( \mathbf{w} \right) \right)\\
&=^{t}\overline{\varphi_{\mathcal{B}}^{- 1}\left( \varphi_{\mathcal{B}}\left( \mathbf{v} \right) \right)}[\varPhi]_{\mathcal{B}}\varphi_{\mathcal{B}}^{- 1}\left( \varphi_{\mathcal{B}}\left( \mathbf{w} \right) \right)\\
&=^{t}\overline{\varphi_{\mathcal{B}}^{- 1} \circ \varphi_{\mathcal{B}}\left( \mathbf{v} \right)}I_{n}\varphi_{\mathcal{B}}^{- 1} \circ \varphi_{\mathcal{B}}\left( \mathbf{w} \right)\\
&=^{t}\overline{\mathbf{v}}\mathbf{w}=\left\langle \mathbf{v} \middle| \mathbf{w} \right\rangle
\end{align*}
これらの線形同型写像たち$S$、$T$は等長写像である。
\end{proof}
\begin{thebibliography}{50}
  \bibitem{1}
    松坂和夫, 線型代数入門, 岩波書店, 1980. 新装版第2刷 p396-398 ISBN978-4-00-029872-8
  \bibitem{2}
    日野正訓. "解析学 I(Lebesgue 積分論)". 京都大学. \url{https://www.math.kyoto-u.ac.jp/~hino/jugyoufile/AnalysisI210710.pdf} (2022-4-4 4:05 取得)
  \bibitem{3}
    小川朋宏. "量子情報数理特論 (第 8 回) 部分トレース,極分解,特異値分解". 電気通信大学. \url{http://www.quest.lab.uec.ac.jp/ogawa/qmath2020/qmath20200701.pdf} (2022-4-10 4:55 取得)
  \bibitem{4}
    理数アラカルト. "行列の極分解". 理数アラカルト. \url{https://risalc.info/src/polar-decomposition.html}
    (2022-4-10 4:56 閲覧)
\end{thebibliography}
\end{document}
