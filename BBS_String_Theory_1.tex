\documentclass[10pt,a4paper]{jsarticle}
%%%%%%%%%%余白の設定%%%%%%%%%%
%\usepackage[a4paper,truedimen,top=2.5cm,bottom=2.5cm,left=2.5cm,right=2.5cm,headsep=10pt]{geometry}
%\usepackage{fancyhdr}  % フッターやヘッダーをいじるため by 2023年度@k74226197Y126が配属された研究室の先生
%\usepackage{lastpage}  % 最後のページを認識するため by 2023年度@k74226197Y126が配属された研究室の先生

%%%%%%%%%%目次の設定%%%%%%%%%%
\setcounter{tocdepth}{3}
\usepackage{booktabs} %しおり

%%%%%%%%%%sectionの見出しの設定%%%%%%%%%%
\renewcommand{\thesection}{第\arabic{section}部} %sectionの見出しの設定
\renewcommand{\thesubsection}{\arabic{section}.\arabic{subsection}} %subsectionの見出しの設定
\renewcommand{\thesubsubsection}{\arabic{section}.\arabic{subsection}.\arabic{subsubsection}} %subsubsectionの見出しの設定
\renewcommand{\headfont}{\bfseries}
\makeatletter
\renewcommand{\section}{ %sectionの設定
  \@startsection{section}{1}{\z@}%
  {\Cvs}{\Cvs} %上下の余白
  {\normalfont\huge\headfont\raggedright}} %字体など
\renewcommand{\subsection}{ %subsectionの設定
  \@startsection{subsection}{2}{\z@}%
  {0.5\Cvs}{0.5\Cvs} %上下の余白
  {\normalfont\LARGE\headfont\raggedright}} %字体など
\renewcommand{\subsubsection}{ %subsubsectionの設定
  \@startsection{subsubsection}{3}{\z@}%
  {0.4\Cvs}{0.4\Cvs} %上下の余白
  {\normalfont\Large\headfont\raggedright}} %字体など
%\usepackage[compact]{titlesec} %sectionの設定の別の方法 by 2023年度@k74226197Y126が配属された研究室の先生
%  \titlespacing*{\section}{0pt}{3ex}{2ex}     % * を付けると続く文章が indent されない。 by 2023年度@k74226197Y126が配属された研究室の先生
%  \titlespacing*{\subsection}{0pt}{2ex}{1ex}  % {command}{left spaces}{top spaces}{bottom spaces} by 2023年度@k74226197Y126が配属された研究室の先生
%  \titlespacing*{\subsubsection}{0pt}{1ex}{1ex} by 2023年度@k74226197Y126が配属された研究室の先生

%%%%%%%%%%数式の設定%%%%%%%%%%
\usepackage{amsmath,amsfonts,amssymb,bm,mathtools,mathrsfs} %数式
%\usepackage{physics} %物理数学
\usepackage{array} %場合分け
\usepackage{exscale} % 大型数式のsizeがfont sizeに応じてできるようにするため. by 2023年度@k74226197Y126が配属された研究室の先生
%\usepackage{mathbbol} % 数字の白抜き ただし、アルファベットがダサくなる。 by 2023年度@k74226197Y126が配属された研究室の先生
\setcounter{MaxMatrixCols}{20} %行列のsizeの上限を20まで拡張する. 
\everymath{\displaystyle} %文中の数式を大きくする. by 2023年度@k74226197Y126が配属された研究室の先生
\allowdisplaybreaks[4] %数式環境内で改頁させる. 
%\numberwithin{equation}{section}   % 数式番号を section 毎に変更。 amsmath package の後じゃないとエラーが出る。 by 2023年度@k74226197Y126が配属された研究室の先生
%\usepackage{slashed} %Dirac’s slash
%%% rap %%% - make two letters overlap
%\newcommand{\rap}[2] % by 2023年度@k74226197Y126が配属された研究室の先生
%{\setbox1=\hbox{#1} % by 2023年度@k74226197Y126が配属された研究室の先生
%\setbox2=\hbox to\wd1{\hss #2\hss} % by 2023年度@k74226197Y126が配属された研究室の先生
%\mbox{\rlap{\box1}\box2}} % by 2023年度@k74226197Y126が配属された研究室の先生
%\usepackage{simplewick}  % Wick contraction by 2023年度@k74226197Y126が配属された研究室の先生
%\usepackage[vcentermath]{youngtab} %Young tableau by 2023年度@k74226197Y126が配属された研究室の先生

%%%%%%%%%%定理環境の設定%%%%%%%%%%
\usepackage{amsthm} %定理環境
\makeatletter
\theoremstyle{definition} 
\newtheorem{thm}{定理}[subsection] %番号あり
\newtheorem*{thm*}{定理} %番号なし
\newtheorem{dfn}{定義}[subsection] %番号あり
\newtheorem*{dfn*}{定義} %番号なし
\newtheorem{axs}[dfn]{公理} %番号あり
\newtheorem*{axs*}{公理} %番号なし
\renewcommand{\proofname}{\textbf{証明}} %証明の見出し
\renewenvironment{proof}[1][\proofname]{\par
  \pushQED{\qed} %証明記号
  \normalfont \topsep6\p@\@plus6\p@\relax
  \trivlist
  \item\relax
  {\bfseries %[...]で入力した証明の見出しの字体など
  #1\@addpunct{.}}\hspace\labelsep\ignorespaces
}{%
  \popQED\endtrivlist\@endpefalse %証明環境の閉じの設定
}
\makeatother

%%%%%%%%%%箇条書きの設定%%%%%%%%%%
\usepackage{enumitem} %番号あり箇条書き
\setlistdepth{20}
\renewlist{itemize}{itemize}{20} %箇条書きの深さ
\setlist[itemize]{label=•} %箇条書きの記号
\renewlist{enumerate}{enumerate}{20} %番号あり箇条書きの深さ
\setlist[enumerate]{label=\arabic*.,ref=\arabic*.} %番号あり箇条書きの番号の書式

%%%%%%%%%%表の設定%%%%%%%%%%
\usepackage{longtable,dcolumn,tabularx,multirow,colortbl,xcolor} %表

%%%%%%%%%%画像の設定%%%%%%%%%%
\usepackage[dvipdfmx]{graphics} %画像挿入 必要に応じて[dvipdfmx]を消したりする. 
\usepackage{bmpsize} %画像sizeの読み込み 不具合あり

%%%%%%%%%%TikZの設定%%%%%%%%%%
\usepackage{tikz} %TikZ
\usepackage{vtable} %表 ただしあまり先頭に書くと不具合が生じる. 
\usetikzlibrary{arrows.meta}
%\usetikzlibrary{arrows,shapes,patterns,calc,babel}  % babel が無いと onlyamsmath と conflict する by 2023年度@k74226197Y126が配属された研究室の先生
%\input{arrowsnew} % by 2023年度@k74226197Y126が配属された研究室の先生
%\usetikzlibrary{decorations.markings}  % snakes オプションは古いらしい。by 2023年度@k74226197Y126が配属された研究室の先生
%\usetikzlibrary{positioning} % by 2023年度@k74226197Y126が配属された研究室の先生

%%%%%%%%%%字体の設定%%%%%%%%%%
%\usepackage{newtxtext}  % 本文フォントの変更がこれでできる(Times系へ変更?) by 2023年度@k74226197Y126が配属された研究室の先生
%\usepackage{newtxmath}  % 数式フォントの変更がこれでできる by 2023年度@k74226197Y126が配属された研究室の先生
%\usepackage[british]{babel}  % 部分的に言語環境を変えるためのもの by 2023年度@k74226197Y126が配属された研究室の先生

%%%%%%%%%%commandの設定%%%%%%%%%%
\newcommand{\mathbm}[1]{\bm{#1}} %\mathbmでも\bmを出力させる. 
%\newcommand{\sla}[1]{\rap{$#1$}{/}} % by 2023年度@k74226197Y126が配属された研究室の先生
\newcommand{\sla}[1]{\rap{$#1$}{$\backslash$}} % by 2023年度@k74226197Y126が配属された研究室の先生
\newcommand{\nord}[1]{\vcentcolon\mathrel{#1}\vcentcolon} %normal ordering by 2023年度@k74226197Y126が配属された研究室の先生
\providecommand{\vcentcolon}{\mathrel{\mathop{:}}} % by 2023年度@k74226197Y126が配属された研究室の先生
\newcommand{\arccoth}{\mathrm{arccoth}\,}
\newcommand{\Arccoth}{\mathrm{Arccoth}\,}
\newcommand{\arcsinh}{\mathrm{arcsinh}\,}
\newcommand{\arccosh}{\mathrm{arccosh}\,}
\newcommand{\arctanh}{\mathrm{arctanh}\,}
\renewcommand{\arccoth}{\mathrm{arccoth}\,}
\newcommand{\Arcsinh}{\mathrm{Arcsinh}\,}
\newcommand{\Arccosh}{\mathrm{Arccosh}\,}
\newcommand{\Arctanh}{\mathrm{Arctanh}\,}
\renewcommand{\Arccoth}{\mathrm{Arccoth}\,}
\newcommand{\Log}{\mathrm{Log}\,}
\newcommand{\pr}{\mathrm{pr}\,}
\newcommand{\proj}{\mathrm{proj}\,}
\newcommand{\tr}{\mathrm{tr}\,}
\newcommand{\Tr}{\mathrm{Tr}\,}
%\renewcommand{\Im}{\mathrm{Im}\,}
%\renewcommand{\Re}{\mathrm{Re}\,}
\newcommand{\diag}{\mathrm{diag}\,}
\newcommand{\ind}{\mathrm{ind}\,}
\newcommand{\Ker}{\mathrm{Ker}\,}
\newcommand{\sign}{\mathrm{sign}\,}
\newcommand{\sgn}{\mathrm{sgn}\,}
%\renewcommand{\<}{\langle}
%\renewcommand{\>}{\rangle}
\newcommand{\Int}{\mathrm{Int}\,}
\newcommand{\topint}{\mathrm{int}\,}
\newcommand{\Cl}{\mathrm{Cl}\,}
\newcommand{\cl}{\mathrm{cl}\,}
\newcommand{\Ext}{\mathrm{Ext}\,}
\newcommand{\ext}{\mathrm{ext}\,}
\newcommand{\Bd}{\mathrm{Bd}\,}
\newcommand{\bd}{\mathrm{bd}\,}
\newcommand{\im}{\mathrm{im}\,}
\newcommand{\rank}{\mathrm{rank}\,}
\newcommand{\nullity}{\mathrm{nullity}\,}
\newcommand{\Span}{\mathrm{Span}\,}
\newcommand{\linspan}{\mathrm{span}\,}
\newcommand{\Hom}{\mathrm{Hom}\,}
\newcommand{\mapshom}{\mathrm{hom}\,}
\newcommand{\homeo}{\mathrm{homeo}\,}
\newcommand{\diffeo}{\mathrm{diffeo}\,}
\newcommand{\Aut}{\mathrm{Aut}\,}
\newcommand{\aut}{\mathrm{aut}\,}
\newcommand{\End}{\mathrm{End}\,}
\newcommand{\mapsend}{\mathrm{end}\,}
\newcommand{\Coker}{\mathrm{Coker}\,}
\newcommand{\coker}{\mathrm{coker}\,}
\newcommand{\rotin}{\text{\rotatebox[origin=c]{90}{$\in $}}} %90度回転させた\in
\newcommand{\amap}[6]{\text{\raisebox{-0.7cm}{\begin{tikzpicture} %写像
  \node (a) at (0, 1) {$\textstyle{#2}$};
  \node (b) at (#6, 1) {$\textstyle{#3}$};
  \node (c) at (0, 0) {$\textstyle{#4}$};
  \node (d) at (#6, 0) {$\textstyle{#5}$};
  \node (x) at (0, 0.5) {$\rotin $};
  \node (x) at (#6, 0.5) {$\rotin $};
  \draw[->] (a) to node[xshift=0pt, yshift=7pt] {$\textstyle{\scriptstyle{#1}}$} (b);
  \draw[|->] (c) to node[xshift=0pt, yshift=7pt] {$\textstyle{\scriptstyle{#1}}$} (d);
\end{tikzpicture}}}}
\newcommand{\twomaps}[9]{\text{\raisebox{-0.7cm}{\begin{tikzpicture} %2つ並んだ写像
  \node (a) at (0, 1) {$\textstyle{#3}$};
  \node (b) at (#9, 1) {$\textstyle{#4}$};
  \node (c) at (#9+#9, 1) {$\textstyle{#5}$};
  \node (d) at (0, 0) {$\textstyle{#6}$};
  \node (e) at (#9, 0) {$\textstyle{#7}$};
  \node (f) at (#9+#9, 0) {$\textstyle{#8}$};
  \node (x) at (0, 0.5) {$\rotin $};
  \node (x) at (#9, 0.5) {$\rotin $};
  \node (x) at (#9+#9, 0.5) {$\rotin $};
  \draw[->] (a) to node[xshift=0pt, yshift=7pt] {$\textstyle{\scriptstyle{#1}}$} (b);
  \draw[|->] (d) to node[xshift=0pt, yshift=7pt] {$\textstyle{\scriptstyle{#2}}$} (e);
  \draw[->] (b) to node[xshift=0pt, yshift=7pt] {$\textstyle{\scriptstyle{#1}}$} (c);
  \draw[|->] (e) to node[xshift=0pt, yshift=7pt] {$\textstyle{\scriptstyle{#2}}$} (f);
\end{tikzpicture}}}}

%%%%%%%%%%校閲の設定%%%%%%%%%%
%\RequirePackage[l2tabu, orthodox]{nag}  % 古いコマンドやパッケージの利用を警告してくれる by 2023年度@k74226197Y126が配属された研究室の先生
%\usepackage[all, warning]{onlyamsmath}  % amsmath が提供しない数式環境を使用した場合に警告してくれる by 2023年度@k74226197Y126が配属された研究室の先生

%%%%%%%%%%その他の設定%%%%%%%%%%
\usepackage{comment} %comment環境
\usepackage{docmute} %\inputを用いるとき\begin{document}...\end{document}の...のみ抽出するためのpackage
\usepackage{url} %URL
\usepackage{fancybox} %枠囲み文字

%%%%%%%%%%一時的な設定%%%%%%%%%%
\newif\iffigure  %図などの重いものを出力しないようにする. by 2023年度@k74226197Y126が配属された研究室の先生
\figurefalse %by 2023年度@k74226197Y126が配属された研究室の先生
\figuretrue  %これの前に%を付けると図が出力されない. by 2023年度@k74226197Y126が配属された研究室の先生
%\usepackage{showkeys}  %\refなどの名前を表示する. by 2023年度@k74226197Y126が配属された研究室の先生

%%%%%%%%%%hyperreferの設定%%%%%%%%%%
\usepackage[dvipdfmx]{hyperref}
\usepackage{pxjahyper}
\hypersetup{
 setpagesize=false,
 bookmarks=true,
 bookmarksdepth=tocdepth,
 bookmarksnumbered=true,
 colorlinks=false,
 pdftitle={},
 pdfsubject={},
 pdfauthor={},
 pdfkeywords={}}
\title{BBS "String Theory and M-Theory" 第1章の和訳}
\author{@k74226197Y126}
\date{2024年4月}
\begin{document}
\maketitle
\begin{abstract}
    もともとはゼミ用の資料だったものです. 内容が導入だったため, 和訳のみにしています. 
\end{abstract}
\section{導入}
20世紀理論物理学で革命を起こした新たな進歩が主に2つあり, 一般相対性理論と量子力学が挙げられる. 一般相対性理論はスケールの広い宇宙膨張の現代の理解で中心をなし, 惑星の運動や光の逸れに対するNewton重力の予測の微修正を与え重力波とブラックホールの存在を予言している. これにおける重力による時空の曲がり具合の扱いによって, 空間と時間の観点が見直され, 今では力学として同時に扱われている\footnote{dynamicalの訳があやしい. }. 一方で量子力学は微視的な物理学の理解で重要な役割を果たしており, 自然での正確な性質をなす証拠が見つかりつづけている. 特に, これの非常によい妥当性\footnote{exact validityの訳があやしい. }は弦理論の研究上の基本的な前提となっている. \par
自然の基本法則の理解は確かに一般相対性理論と量子力学が無事整合し統合されない限り不完全ではある. これの困難さはさまざまな観点からわかる. その2つを特徴づける観測や計算の方法の扱いが大きく異なるし, さらに, 約1980年までその2つの分野はほぼ独立に発展してきており両方とも精通している物理学者はほとんどいなかった. 両方とも統合されるというゴールで以って, 弦理論は科学的にだけではなく社会学的にも\footnote{sociologyの訳があやしい. }劇的に変えていった. \par
場の量子論こと相対論的量子力学では, 空間的な領域における時空点上で定義される2つの場が可換的である, もしくは, Fermi粒子なら, 交代的である必要がある一方で, 重力の扱いに関しては, その計量の扱いがわからないため, 2つの時空点が空間的な領域にあるようにできるかどうかは不明である. これが力学上の\footnote{dynamicalの訳があやしい. }問題の1つになっている. さらに, 計量もほかの量子場と同様に量子ゆらぎの影響を受ける対象となっており, 明らかに, これらは話を難しくさせている要因になっている. その他に話を難しくされている要因として, ブラックホールの量子的な扱いや初期の宇宙の扱いも関係している. \par
最もわかりやすい試みとして, 場の量子論における摂動で量子力学と一般相対性理論を組み合わせるというものもあるが, これは込み入った収束性の問題になっていく. 重力を考慮した散乱の補正の著しい特徴として, 紫外発散というのがあり, これは摂動論との相性が悪い. 万有引力定数は4次元で長さの2乗に比例した次元をもつ関係上, 摂動次数の議論で場の量子論でみてきた従来のくりこみでは収束させることができないことがわかる. 詳しい計算によって, このような次元解析による問題を解決できるほどうまくいかないこともわかる\footnote{
    実際にその計算をしてみよう. 万有引力定数を$G$とおくと, 
    \begin{align*}
        \left[ G \right] &= \left[ \frac{\text{力} \times \text{長さ}^2 }{\text{質量}^2 } \right] = \frac{\mathrm{L}^2 \times \mathrm{M} \mathrm{L} \mathrm{T}^{-2} }{\mathrm{M}^2 } = \mathrm{L}^3 \mathrm{M}^{-1} \mathrm{T}^{-2}
    \end{align*}
    なので, $c = \hbar = 1$の下でpower countingをすると, 
    \begin{align*}
        \left[ G \right] = - 1 - 3 + 2 = - 2
    \end{align*}
    となってしまう. 
}. (摂動でのくりこみは基礎付けで据えることができないと考えて, くりこみできないにもかかわらず, 一般相対性理論の量子化を試みる物理学者もいる. ループ量子重力はこの方法の一例である. その考えが何であれ, かなりの努力にもかかわらず, あまり実りのある結果が得られていないということも述べたほうが公平であろう.~)\par
\subsection{歴史での起源}
弦理論は1960年代後半に強い力の理解の1つの試みとして現れており, その力はクォークが陽子と中性子のなかに閉じ込めるように陽子と中性子が原子核のなかに閉じ込めるのになすものとして考えられている. この弦理論は弦とよばれる点として考える粒子というより1次元だけ拡張した基礎的なものに基づいており, 強い力の種類の多さや粒子やハドロンの強い相互作用を定性的に説明できる. \par
弦の扱いで基礎となる考えは粒子が弦の調和振動子, あるいは, 量子状態にあたるというもので, これによって, 図1.1のように無数に異なって観測されるハドロンを説明する際, 1つの基礎となるもの, 即ち, 弦を仮定することでよく整合のとれた扱いが得られる. \par
1970年代初頭では, 量子色力学という他の強い力に関する理論が発達し, これの結果やその弦理論のさまざまな技術的な問題で弦理論は淘汰されていった. なお, 現代の観点では, その問題点がうまく機能しているものとなって, 弦理論は研究のある分野で再び活発になっていく. 今では, その問題をどうするかでよりよい理解が得られているものの, 強い相互作用における弦理論はまだ具体的にわかっていない. \par
弦理論は重力など自然の基本的な力を統一的に扱える量子論を作るというより野心的な動機にも適しているということが判明し, 原理的には, 弦理論に素粒子物理学や宇宙論を隅々まで理解できる可能性がある. これはまだ叶えられそうにもない夢であるものの, この魅力的な理論で次々と驚きが得られてきている. 
\subsection{弦理論の一般的な特徴}
弦理論はまだ完全に定式化されておらず, 低エネルギーで素粒子の標準模型がどのように現れるのか, どう宇宙がはじまったかという詳しいことはまだわかっていないものの, この弦理論でよくわかっているいくつかの一般的な特徴がある. これらは弦理論の最終的な定式化に関係なくいえる. 
\subsubsection*{重力}
1つ目の弦理論の一般的な特徴としては, 一般相対性理論と相性がよいというもので, かなり重要なものだと思われる. 一般相対性理論は近距離, もしくは, 高エネルギーで修正されるものの, 通常の距離やエネルギーで弦理論はEinsteinが提唱したものによくあうようになる. これは, 量子論の枠組みでもよく一般相対性理論が現れてくるため, 注目されている. 通常の場の量子論では, 重力の存在を考えていないものの, 弦理論ではそうなっている. 
\subsubsection*{楊-Millsゲージ理論}
素粒子物理学のすべてを述べるというゴールを果たすという目でみれば, 弦のスペクトルにおける重力子の存在はよくわかっておらず, ゲージ群$\mathrm{SU} \left(3\right) \times \mathrm{SU} \left(2\right) \times \mathrm{U} \left(1\right) $に基づく楊とMillsによる理論である標準模型が説明される必要がある. 標準模型を含む楊-Millsゲージ理論が得られるというもの弦理論の一般的な特徴である. さらに, 標準模型の重要な性質である鏡像異性な複素表現も現れる. しかしながら, なぜ自然でクォークとレプトンの三世代で$\mathrm{SU} \left(3\right) \times \mathrm{SU} \left(2\right) \times \mathrm{U} \left(1\right) $によるゲージ理論しか出てこないのかはよくわかっていない. 
\subsubsection*{超対称性}
3つ目の弦理論の一般的な特徴として, 一貫とした超対称性が挙げられる. これはBose粒子とFermi粒子が関係する対称性であり, 第2章, 第3章でみるように超対称性が現れないbosonicなほうの弦理論というものもあるが, Fermi粒子を考えていないため, 現実に則していない. Fermi粒子を考えた弦理論の数学の整合性は局所的な超対称性にかなり依っており, 超対称性は現実だと考えられている弦理論の一般的な特徴となっている. この対称性がみられていないという事実で, 超対称性を特徴づけるエネルギーのスケールになく超対称性におけるよく知られている粒子の対をなすもう一方のほうの質量は実験で確かめられる最小のより下回っていることが示唆されている. \par
実験で可能なエネルギーで確認できる超弦理論による主要な予想の1つとして, 時空超対称性というものもある. 弦理論に限らずさまざまな議論から超対称性にあうよう特徴づけられるエネルギーのスケールにないことは電弱のスケール, 即ち, $100\ \mathrm{GeV}$から数$\mathrm{TeV}$の範囲に関係しているのではといわれている. これが正しければ, 超対称性で対となるもう一方の粒子は2007年稼働予定のCERN Large Hadron Colliderで観測できるということになる. 
\subsubsection*{空間の余剰次元}
多くの物理学の理論とは異なって, 超弦理論は我々のいる時空の次元を述べることができる. この超弦理論は10次元時空で議論されており然るべき状況で11次元もできる. \par
弦理論といつも体験している4次元の世界と結びつけるために, すぐ思いつく可能性としては, 6, 7次元の分は内部空間というみえないくらい大きさが十分小さい多様体\footnote{"internal manifold"を当初"みえないくらい大きさが十分小さい多様体"と訳したが, 内部空間という語句があるので, そちらに直した. }にコンパクト化されるというものがあり, 素粒子物理学のことなので, 残りの4次元は我々がいる4次元時空となる. もちろん, 宇宙論で時間に依存する他の幾何学的構造もあらわれるかもしれない. \par
余剰次元の考えは1920年代KaluzaとKleinによって初めて議論され, そのゴールは5次元の一般相対性理論を円環にコンパクト化することで4次元の電磁気学と重力を統一的に扱うというものであった. 今日では, 電磁気学がこれであらわれることはないことが知られているものの, この有用な方法は弦理論で再び現れる. 現在ではコンパクト化とよばれるそのKaluzaとKleinの考えは図1.2の2つの円筒で図示される. 1つ目の円筒の表面は2次元な一方で, 円の半径がかなり小さいなら, つまり, 円筒を遠くからみると, その円筒は1次元にみえてくる. そこで, 円筒の長い部分を我々の4次元時空, 短い部分を6, 7次元のコンパクトな内部空間におきかえて想像することもできる. 遠距離, もしくは, 低エネルギーでそのコンパクトな空間はみえなくその世界は4次元のようにみえてくる. 第9章, 第10章でみるように, 内部空間がみえていると仮定したとしても, これらの位相的な性質は4次元の粒子の内容や構造によって決定される. 1980年代半ば, 初めて6次元分のコンパクト化されている余剰次元としてCalabi-Yau多様体が考えられており, しかも, 第10章でみるようにモジュール空間に関する問題のように深刻な欠点があるにもかかわらず現象論からみて有力だとみられた. 円環とは異なって, Calabi-Yau多様体は等長写像をもつとは限らなく, これが対称性を高めているというより低めている. 
\subsubsection*{弦の大きさ}
従来の場の量子論における素粒子は数学的にいえば点であった一方で, 弦理論の摂動では, 基礎となる対象は1次元の厚さのないループであり, 弦は次元解析で推量できる$l_{\mathrm{s}}$で表される特徴づけている長さのスケールをもっている. 弦理論は重力を考慮した相対論的量子論でもあるので, 弦理論は光速$c$, Planck定数を$2\pi $で割った量$\hbar$, 万有引力定数$G$といった基礎となる定数を含めているはずである. これらからPlanck長とよばれる長さを構成できて次のようになる: 
\begin{align*}
    l_{\mathrm{p}} = \sqrt{ \frac{\hbar G}{c^3} } \approx 1.6 \times 10^{-33} \ \mathrm{cm}.
\end{align*}
同様に, Planck質量が次のように与えられる: 
\begin{align*}
    m_{\mathrm{p}} = \sqrt{ \frac{\hbar c}{G} } \approx 1.2 \times 10^{19} \ \mathrm{GeV} / c^2.
\end{align*}
そのPlanckのスケールはコンパクトな余剰次元の分を特徴づける大きさと同様に基礎となる弦の長さのスケールを大まかに推量するための自然な候補となっている. Planckエネルギーよりはるかに低いエネルギーでの実験では, Planck長ほど短い距離をみることができなく, そのようなエネルギーの下では, 弦は点粒子によい精度で近似できる. これによって, なぜ場の量子論がうまく自然を説明できてきたかが説明できる. 
\subsection{弦理論の基本事項}
弦は時間によって変化するため, 弦は時空上で2次元曲面を掃くことになる. その2次元曲面をその弦の世界面という. これは弦における点粒子の世界線に対応するものになる. 場の量子論の摂動でみてきたように, 振幅の寄与は世界線の構成可能な部分を表すFeynman図に関連づけており, 特に, 相互作用は世界線の交点に対応している. 同様に, 弦理論の摂動における展開で世界面はさまざまな位相幾何学と絡んでくる. \par
弦理論の相互作用の存在は世界面の孤立特異点というより世界面の位相幾何学の議論で理解できる. この点粒子の場合との差異は2つの重要な意味づけをもっている. 1つ目は弦理論で相互作用の構造は自由場理論で一意的に決まるというもので, 場の量子論の意味で相互作用の任意性がない. 2つ目は, 弦の相互作用は近距離の紫外特異点\footnote{"singularities"は紫外特異点と訳した. 補足すると, 円環は一点とホモトピー同値すらないので, 数学的にまずいことになっている. }によらないので, 弦理論での振幅で紫外発散することがない. その弦のスケール$1/l_{\mathrm{s}}$はUVカットオフとして機能する. 
\subsubsection*{世界体積%\footnote{world-volumeの訳があやしい. }
の作用と臨界次元}
弦は$p$次元の空間と張力, あるいはエネルギー密度$T_p$をもつ対象である$p$-ブレーンの特別な場合とみなせ, 実際, 非摂動励起\footnote{nonpertubative excitationsの訳があやしい. }として超弦理論でさまざまな$p$-ブレーンが現れる. $p$-ブレーンの古典力学での運動は時空で掃く$p+1$次元体積$V$内でとり\footnote{extremizesの訳があやしい. }, したがって, $S_p = - T_p V$という$p$-ブレーンの作用が与えられる. 基本的な$p = 1$なる弦の例でいえば, $V$は世界面の面積でその作用は南部-後藤作用としてよばれる. \par
古典力学的に, 南部-後藤作用\footnote{
    式は次のように与えられる: 
    \begin{align*}
        S_{\text{南部-後藤}} = - \int_{世界面} \sqrt{\det X^* \eta } d \sigma \wedge d \tau .
    \end{align*}
    ここで, $X$は世界面からの埋め込みで$\eta $はMinkowski計量, $X^* \eta $は引き戻しで引き起こされる行列で, 第$\left( \alpha , \beta \right) $成分が$\frac{\partial X^{\mu}}{\partial \sigma^{\alpha}} \frac{\partial X^{\nu}}{\partial \sigma^{\beta}} $となっている. この式から, 量子化するとき, Taylor展開しないといけなく, 項が無限に現れるので, 量子化するのが難しくなる. 
}は次の弦の$\sigma$模型の作用に同等である\footnote{補足すると, 共形変換で不変でないと, 南部-後藤作用に戻せられない. このことがM理論でかなり問題になっている. }: 
\begin{align*}
    S_{\sigma} = - \frac{T}{2} \int_{\text{世界面}} \sqrt{- h } h^{\alpha \beta } \eta_{\mu \nu } \frac{\partial X^{\mu}}{\partial \sigma^{\alpha}} \frac{\partial X^{\nu}}{\partial \sigma^{\beta}} d \sigma \wedge d \tau. 
\end{align*}
ここで, $h_{\alpha \overline{\beta}} d \sigma^{\alpha} \otimes d \overline{\sigma }^{\beta} $\footnote{$h$はKähler計量となる. }は世界面の補助的な\footnote{auxiliaryの訳があやしい. }計量で$h$はこれから局所的に誘導される行列式, $h^{\alpha \beta } $は局所的に誘導される行列の逆行列である. その関数$X^{\mu}$は世界面の埋め込みによって誘導されているものとする. $h_{\alpha \beta } $のEuler-Lagrange方程式はその作用からこれを消去し南部-後藤作用に導くのに用いられる. \par
量子力学的には, その議論の流れはより微妙で場の古典論の方程式で$h_{\alpha \beta} $を消去する代わりに, 局所的な対称性とゲージ固定を扱う標準的な方法を用いてFeynmanの経路積分を計算することになり, これが正しくされたときに, 時空の次元が26なとき以外に, 共形異常が生じることがわかる. このことは第2章, 第3章で詳しくみる. 超弦理論に対する類似した計算でそういった臨界次元が10と与えられる. 
\subsubsection*{閉弦と開弦}
その関数$X^{\mu}$で埋め込みされている変数$\tau$は世界面の局所座標で変数$\sigma$は世界面の弦を特徴づけている. 円環と同相な閉弦の場合, その変数$\sigma$に周期性があり, その周期を$\pi$とすれば, 弦が$X^{\mu} \left( \sigma, \tau \right) = X^{\mu} \left( \sigma + \pi, \tau \right) $と特徴づけされる\footnote{数学的にいえば, 写像$\sigma : \mathbb{R} / \pi \mathbb{Z} \rightarrow \text{世界面}$が自然に誘導されるということになる. }. どの弦理論でも閉弦を含むことになり, 臨界次元における弦理論\footnote{"critical string theory"は臨界次元における弦理論と訳した. }での閉弦のスペクトルのうち質量がないものとして重力子がよく現れる. \par
連結な開区間と同相な開弦の場合, 端は, 任意の添字$\mu$に対し, Neumann境界条件かDirichlet境界条件のいずれかを満たしている必要があり, Dirichlet境界条件は弦の端にある時空多様体を定める. この方法によってのみ, 開弦がD-ブレーンという物理的な対象とみるなら, 意味のあるものとなる. なお, DはDirichletに因んでいる. すべての開弦の境界条件がNeumann境界条件なら, その弦の端は時空上どこでもとれる. 現在の解釈では, これが時空上D-ブレーンでみたされているということを意味することになる. 
\subsubsection*{摂動論}
摂動論は量子電磁気学のように十分小さい無次元の結合係数がでてくる場の量子論で便利で, これによって, その十分小さい変数で展開される物理量が計算できる. 量子電磁気学では, その十分小さい変数は$\alpha \approx 1 / 137$という微細構造定数となる. 物理量$T \left( \alpha \right) $に対し, Feynman図を用いて次のように計算される: 
\begin{align*}
    T \left( \alpha \right) = T_0 + \alpha T_1 + \alpha^2 T_2 + \cdots .
\end{align*}
摂動での級数はしばしば収束半径が0の漸近展開となる. 展開の変数が小さくても, その展開の最初の項が精度のよい近似になっているので, 有用なときもある. \par
Heterotic\footnote{heteroticの訳し方がわからず. }, もしくはII型の超弦理論は向きづけられた閉弦のみ含み, その結果, 摂動展開での世界面は向き付け可能な閉Riemann面に限る. 摂動のどの次数でも独特な世界面の位相幾何学があり, そのおかげで, UVが収束する. 摂動展開のどの次数でも弦理論ではFeynman図が1つのみ現れるという事実は場の量子論でFeynman図が数多く現れるというのと一線を画し, 弦理論では, 結合係数$g_{\mathrm{s}}$が小さくある必要はなくなる. したがって, 現実に則した真空が摂動論によってのみしか精度よく計算できないというわけでもなくなる. このことから, 弦理論での非摂動的効果を理解するのが重要となってくる. 
\subsubsection*{超弦}
第一次超弦革命\footnote{ここでは, "The first superstring revolution"を"第一次超弦革命"と訳した. }は1984年とりうる2つのLie代数$\mathrm{SO} \left( 32 \right) $, $E_8 \times E_8 $のうちどちらかに基づく局所的な楊-Millsゲージ対称性から$\mathcal{N} = 1$超対称性における10次元の理論によって量子力学の整合性がとれるということがわかったことで生じた\footnote{このあたりの訳は微妙である. }. 第5章でみるように, その2つのとり方のみ特定の量子異常が打ち消され, そのとりうる群がその2つのみということから, 弦理論は硬い構造をもつ可能性があり, 予測できることが多くなる可能性がある. (量子異常も$\mathrm{U} \left( 1 \right)^{496} $と$E_8 \times \mathrm{U} \left( 1 \right)^{248} $のみ可能なものの, このようなゲージ群の弦理論はとれない.~)\par
超弦の定式化で左移動と右移動両方ともとる場合, その左移動, 右移動に伴う超対称性は反掌性と掌性どちらもとれて, それぞれIIA型, IIB型超弦理論とよばれる異なったものが引き起こされる. 3つ目のI型超弦理論はorientifold射影とよばれる方法で左右対称性からIIB型超弦理論を改良することによって導出される. この射影によって残る弦は向き付けされていない. I型, II型超弦理論はそれぞれ第4章, 第5章で世界面と時空超対称性の定式化を通してみていく. \par
より驚くべき可能性として, 26次元のbosonicな弦理論では左移動のほうを, 10次元の超弦理論では右移動のほうを用いることになるというのがある. このようにして構成された弦理論をheteroticとよばれ, 第7章で詳しくみていく. 時空次元の不自然さは奇妙に感じられるかもしれないが, 欠かせないものである. 整合性をとるために16次元分の余剰次元では, トーラスのうち非常に特別なものを扱う必要がある. そのような性質をもつようなトーラス\footnote{toriはtorusの複数形となる. }はちょうど2つありLie代数$\mathrm{SO} \left( 32 \right) $, $E_8 \times E_8 $に対応する. \par
まとめると, いずれも10次元の5つの異なった超弦理論があり, そのうち, I型とheteroticの超弦理論3つは10次元の意味で$\mathcal{N} = 1$超対称性をもっている. 10次元の最小スピノールは16実成分をもち, これらは16個の保存される超電荷をもつ. I型超弦理論はゲージ群$\mathrm{SO} \left( 32 \right) $をもつ一方で, heteroticなものは$\mathrm{SO} \left( 32 \right) $と$E_8 \times E_8 $両方ともとれる. 他2つのIIA型, IIB型超弦理論は$\mathcal{N} = 2$の超対称性, もしくは, 32個の超電荷をもっている. 
\subsection{超弦理論の最近の発展}
5種類もの異なる超弦理論があるということはやや不可解で, 宇宙は1つしかないので, とりうる理論が1つしかないというのが最も好ましいかもしれない. 1980年代後半, 2つのII型超弦理論, 2つのheteroticな超弦理論の間にT-双対性として知られる性質があることがわかり, 同じものだと思えるようになった. \par
非摂動的な現象の理解で1990年代に進展がみられ, 非摂動的なS-双対性と特定の状況下での強い結合における11次元の理論の黎明\footnote{この訳はあやしい. }によって, 新しい特徴づけがされた. これらの対応がわかれば, 唯一つの理論があるという可能な限り最良の結論が得られる. このことは次にまとめられ, あとで詳しく議論する. 
\subsubsection*{T-双対性}
弦理論は驚くべきの性質をもっており, そのうち1つはT-双対性とよばれ, 第6章で詳しくみる. T-双対性は, 多くの場合, 余剰次元の2つの異なる幾何学的構造は物理的に同じものだと思える. 簡単な例でいえば, 半径$R$の円環は半径$l_{\mathrm{s}}^2 / R$の円環と同じものだと思える. ここで, 今までと同様に, $l_{\mathrm{s}}$はその基礎となる弦の長さのスケールである. \par
T-双対性は2つの異なる理論, 例えば, 2つのII型, 2つのheteroticな超弦理論を関係づけている. したがって, IIA型, IIB型ないし2つのheteroticな理論は1つものだと思えてくる. より正確にいえば, これらは円環の半径を変化させて得られる幾何学的構造の両端を表している. この半径は基礎となる理論の変数というわけではなく, むしろ, スカラー場の真空期待値として生じ, さまざまにとりうる. \par
その双対性でより洗練された例があり, 例えば, Calabi-Yau多様体でコンパクト化されたIIA型超弦理論に相当するものとミラーCalabi-Yau多様体でコンパクト化されたIIB型超弦理論に相当するものがある. この位相的に異なるCalabi-Yau多様体の間のミラーの組は第9章で議論する. T-双対性と意外な関係がある. 
\subsubsection*{S-双対性}
S-双対性とよばれる別の種類の双対性は1990年代半ばの第二次超弦革命\footnote{"the second superstring revolution"を"第二次超弦革命"と訳した. }のなかでみられ, これは第8章で議論する. S-双対性は, T-双対性で$R$が$l_{\mathrm{s}}^2 / R$に関係づけられるのと同様に, 弦の結合係数$g_{\mathrm{s}} $を$1 / g_{\mathrm{s}} $に関係づけられる. その2つの基本的な例\footnote{T-双対性とS-双対性のこと. }はI型超弦理論を$\mathrm{SO} \left( 32 \right) $のheteroticな超弦理論にIIB型超弦理論をこれ自身に関係づけている. したがって, これらにおける十分小さい$g_{\mathrm{s}}$の振る舞いや摂動論がわかれば, $1 \ll g_{\mathrm{s}} $のとき, これらの3つがどうなるかわかる. 例えば, 強く結合された\footnote{strongly coupledの訳があやしい. }I型超弦理論は弱く結合された\footnote{weakly coupledの訳があやしい. }$\mathrm{SO} \left( 32 \right) $のheteroticな超弦理論と同じものだと思え, IIB型超弦理論の場合, これはこれ自身に関係づけており, 実際に対称性をもっている. その弦の結合係数$g_{\mathrm{s}} $は$\exp \phi $の真空期待値によって与えられる. ここで, $\phi $はディラトン場である. T-双対性と同様にS-双対性は実際のところ場の変換$\phi \mapsto - \phi $であり, 真空期待値に関するものではない. 
\subsubsection*{D-ブレーン}
非摂動的にみると, 基本的な弦に加えて, さまざまな空間次元が$p$の対象であるD-ブレーンも超弦理論に含まれていることがわかり, 1-ブレーンこと基本的な弦という例外を除いて, $g_{\mathrm{s}} \to 0$のとき, 重さが無限大に発散し摂動論に現れないことになる. 一方で, その結合係数$g_{\mathrm{s}}$が十分小さくないとき, この区別\footnote{1-ブレーンと$2 \leq p$なる$p$-ブレーンとの区別のこと? }は意味をなさなくなり, この場合, すべての$p$-ブレーンは基本的な弦と同じくらいの重要さになる. つまり, $p$-ブレーン民主主義がある\footnote{"$p$-brane democracy"を"$p$-ブレーン民主主義"と訳した. }. \par
I型, II型超弦理論はD-ブレーンという$p$-ブレーンのクラスがあり, これらの張力は$1 / g_{\mathrm{s}} $に比例する. 前に述べたように, これらが定める性質として, 基本的な弦が連続したようなものにできる対象となるものがあり, 基本的な弦がD-ブレーンで連続したようなものとできるという事実は標準模型のような楊-Millsの場の量子論がD-ブレーンの世界体積\footnote{world volumeの訳があやしい. }で存在できることを意味して, 楊-Mills場はそのD-ブレーンにくっついた開弦の質量がない模型として生じる. 標準模型に似た理論がD-ブレーンにあるという事実には多くの興味深い意味付けがある. 例えば, 我々のいる世界が4次元な理由がより高次元の時空に埋め込まれている3次元D-ブレーン, 即ち, D3-ブレーンにいることに限られるためとなる. こういう流れでの模型の作り方はときどきbrane-world法, シナリオ\footnote{"the brane-world approach", "scenario"をそれぞれ"brane-world法", "シナリオ"と訳した. }とよばれ, 第10章で議論する. 
\subsubsection*{M理論とは? }
S-双対性は5つの超弦理論のうち3つが強い結合の下でどう扱うか述べている. これから, $g_{\mathrm{s}}$が大きいときのIIA型, $E_8 \times E_8 $のheteroticなもの残り2つの超弦理論はどうなるのかという問題が提起される. とても驚くような解決法は大きさ$g_{\mathrm{s}} l_{\mathrm{s}} $での11次元のものに一般化するというものであり, 11次元のもので結合係数が十分大きいとき\footnote{When the eleventh dimension is large, の訳があやしい. }, 摂動的弦理論では扱いきれないので, 別の新しい方法が必要とされる. 重要なこととして, M理論という11次元の新しい量子論が生じる. 低エネルギーでは, 11次元超重力という場の古典論に近似できるが, M理論はそれ以上のものであり, M理論と上で述べた2つの超弦理論は図1.3のようにT-双対性, S-双対性も含めたいくつかの双対性によって関係づけられ, これらは1つの理論の異なる側面となっている. 超弦理論の大まかなクラスやM理論における真空を特定し, 正確に述べる方法があるものの, これらの真空が得られるようなより基礎的なもので簡潔かつ説得力のある定式化はまだされていない. なお, この定式化は調整可能な無次元変数や恣意性なく一意的でなければいけないとしている. 時空多様体の存在などいつも受け入れている多くのことはより基礎的なものの特徴づけ\footnote{identifiable featuresの訳し方が微妙である. }というより特定の真空のその場しのぎの扱い\footnote{emergent propertiesの訳があやしい. }として理解される可能性もある. これが正しければ, 求めている定式化はこれまでの理論とだいぶ異なってくる. 場の量子論に基づく通常の方法では, 周囲に時空多様体の存在が仮定され, 時空多様体を仮定しないもので自由度の程度がどれくらいかは明らかでない. \par
特定の時空上の背景にあたる行列理論というM理論の正確な量子力学の扱いにおける興味深い提案があり, 行列理論は$N$が十分大きいとき$N \times N$型行列の量子力学における平坦な11次元時空でのM理論の双対的なものとなっている. 空間次元のうち$n$だけトーラスにコンパクト化されると, その行列理論の双対的なものは空間次元が$n$に時間次元を加えた場の量子論になり, その$n$が大きすぎないとき, この予想が正しいと考えられている. しかしながら, その他のコンパクト化へ一般化する方法がよくわかっておらず, 行列理論はM理論の一部にしかなっていない. 
\subsubsection*{F理論}
前にみたように, IIA型, $E_8 \times E_8 $のheteroticな超弦理論はM理論というより基礎となる11次元の理論から得られるとみなせる. すると, 他の超弦理論も同様な方法で導けないかと考えたくなるかもしれない. F理論という第9章でみていく1つの方法があり, これは10次元のIIB型超弦理論が非摂動的な$\mathrm{SL} \left( 2, \mathbb{Z} \right)$対称性をもっているという事実を用いており, さらに, これはトーラスのモジュラー群でIIB型超弦理論はトーラスの概複素構造として$\mathrm{SL} \left( 2, \mathbb{Z} \right)$のもとで変換する複素スカラー場$\tau $を含んでいる. したがって, IIB型超弦理論が概複素構造$\tau $の補助的な\footnote{auxiliaryの訳があやしい. }2-トーラス$\mathbb{T}^2 $をもっているとみなせるとすれば, この$\mathrm{SL} \left( 2, \mathbb{Z} \right)$対称性は幾何学的な解釈ができて, 自然にトーラスの対称性と解釈される. 
\subsubsection*{フラックスコンパクト化\footnote{flux compactificationの訳があやしい. }}
既にKaluzaとKleinが悩ませてきた1つの疑問として, なぜ5次元が丸まっていないといけないのかというのがある. 初期の問題として, その円環の大きさとある値においてこれを安定させているものは何かというもので, これらの問題は弦理論でも似たようなものがあり, モジュラー空間問題とよばれるものの1つになっている. 弦理論では, 内部空間の形状と大きさはスカラー場の真空期待値によってさまざまにとりうる. 最近, 現代の弦理論の研究で急速に発展している分野であるフラックスコンパクト化でその解決法がわかってきている. これは第10章でみていく. \par
M理論という基礎となるものは一意的なものの, 量子真空といったことがかなり多くとりうり, これらの解決法の1つは, 時空で4次元Minkowski空間とコンパクトな多様体の直積\footnote{timesの訳があやしい. }からなっているべきで素粒子物理学で考えている状況を正確に述べていく\footnote{"accurately describes the world of particle physics"をここでは"素粒子物理学で考えている状況を正確に述べていく"と訳した. }というもので, 現在の弦理論での研究で主な課題の1つがこの方法を見つけるというものである. \par
正しい真空を見つけ, 同時に, これがなぜ正しいものになるか理解するのはよいことであるが, これはいくつかの特別な数学的な扱いによって見つかるものなのか, それとも, 宇宙における我々のいるところがたまたまそうなっている\footnote{"an environmental accident of our particular corner of the Universe"を"宇宙における我々のいるところがたまたまそうなっている"と訳した. }のかという疑問が挙がる. この疑問の挙げ方は第一原理から得られる素粒子物理学で観測される範囲を決めるのに重要となる可能性もある. 
\subsubsection*{ブラックホールのエントロピー}
一般相対性理論から巨視的なブラックホールはきちんと定義された温度とエントロピーを伴う熱力学的な対象のように振る舞うことがわかり, そのエントロピーはBekenstein-Hawkingのエントロピー式という事象の地平線の面積の$1 / 4$で重力単位で与えられる. 量子論では, エントロピー$S$は, 通常, 該当する微視的な状況に寄与する多数の量子状態, 即ち, $\exp S$の分だけあるということになり, したがって, 自然にそのことがブラックホールやblack $p$-brane\footnote{black $p$-branesの訳し方がわからず. }というその高次元に一般化されたものにも適用できるのかという疑問が挙がる. D-ブレーンはこの疑問を調べるための準備になる. \par
これに関する初期の研究では, 多くの超対称性をもつブラックホールの特別なものに対してのみしか量子状態を数え上げる確からしい方法がなく, この場合, そのエントロピー式にあうことがわかった. 最近では, どのようにしてより広いブラックホールのクラスやblack $p$-braneを調べるか, その面積の式をどう修正して計算するかということまでわかってきている. これは第11章でみていく. 数多くの例が研究されており, 想定内の修正を除いて矛盾している箇所は見つかっておらず, これらの研究によって, 微視的な物理学での弦理論におけるブラックホールの熱力学的な性質がわかってきているといえる. これは今のところ弦理論でかなりうまくいったものの1つである. 
\subsubsection*{AdS/CFT対応}
1990年代後半注目すべき発見として, 共形変換で不変な場の量子論と時空上の幾何学的構造のうち特別なものでの超弦理論, M理論がちょうど同一視できる, あるいは, これらの間に双対性があるというものがあり, あてはまる$p$-ブレーンの集まりはブラックホールのような地平線での時空上の幾何学的構造をなしており, この幾何学的構造は反de Sitter空間と球面の直積とみなせる. IIB型超弦理論におけるD3-ブレーンにあたる$N$を考えることで引き起こされる例として, 4次元の$\mathcal{N} = 4$超対称性をもつ$\mathrm{SU} \left( N \right) $の楊-Millsゲージ理論の間の双対性や5次元反de Sitter空間$\mathrm{AdS}_5 $と5次元球面$S^5 $の直積で与えられる10次元の幾何学的構造が入っているIIB型超弦理論が挙げられる. 5次元球面から引き起こされる$N$組の平坦な5次微分形式があり\footnote{flux threadingの訳があやしい. }, 類似したM理論の双対性もある. \par
これらの双対性はしばしばAdS/CFT対応とよばれており, AdSは反de Sitter空間という負のスカラー曲率をもつ対称性の高い時空上の幾何学的構造からきていて, CFTは共形場理論という共形変換からなる群構造の下で不変な場の量子論からきている. 2次元の乳濁液のようなもの上の3次元空間の表現をなしていることに類似しているので, これらの等価性はホログラフィー対応\footnote{れっきとした専門用語だけど, ここでは, 日常会話での意味も重ねている. }の例の1つとなっている. これらの双対性の研究で双対をなす場の量子論とともに弦理論やM理論についてかなりわかってきている. 第12章でこれの導入を述べる. 
\subsubsection*{弦理論とM理論での宇宙論}
超弦宇宙論という分野は新しくおもしろい分野として現れてきており, 超弦理論による考察で宇宙論の研究で新しい考えがもたらされている. これには弦理論でしかいえない測定値を予測できる最初の分野になる\footnote{このあたりの訳はあやしい. }可能性もある. \par
弦理論など重力を考慮した量子論では, 真空のエネルギー密度を特徴づける宇宙定数$\Lambda $が原理的に計算できる量となる. ダークエネルギーとよばれるこのエネルギーは最近かなりよい精度で測定されており, 現代の宇宙における質量, エネルギー全体のうち約$70\ \%$占められている. この割合は時間の単調増加する関数となっており, 宇宙定数/ダークエネルギーの観測値は宇宙論にとって重要なものの, Planck単位系で表すと約$10^{-120} $と非常に小さくなる. 時間に依存しないコンパクトな多様体でコンパクト化された11次元超重力に基づく弦理論とM理論で$0 < \Lambda $となることへの説明の試みはあるうれしくない定理によってうまくいかなかったものの, ある非摂動的な効果では, そのうれしくない定理に適用されないようにできている. \par
最近, 流行っている観点として, 弦理論で$\Lambda $のほぼすべての値が説明されることができるものの, $\Lambda $が十分小さいようなもののみで生命を担う宇宙が述べられることができるというものがあり, もし, これが大きければ, 我々はここにいてその疑問を呈することもできないだろう. こういう理由の与え方を人間原理という. これが正しくとも, この理由を用いない別の$\Lambda $が小さいことへの説明が欲しいところである. \par
宇宙論におけるもう1つの重要な問題として, インフレーションというかなり初期の宇宙における加速膨張というのがあり, インフレーションで観測されたものでいえば, かなり強く, 基礎となる理論からどう説明されるべきかという疑問は重要となる. インフレーションの期間前では, 観測可能な宇宙の起源であるビッグバンがありこれを理解するために多くの努力が払われており, 2つの大きく異なる提案として, 何もないところでの量子トンネルとブレーンの衝突が挙げられている. 
\end{document}