\documentclass[dvipdfmx]{jsarticle}
\setcounter{section}{1}
\setcounter{subsection}{0}
\usepackage{xr}
\externaldocument{8.1.1}
\externaldocument{8.1.2}
\externaldocument{8.1.7}
\externaldocument{8.1.9}
\usepackage{amsmath,amsfonts,amssymb,array,comment,mathtools,url,docmute}
\usepackage{longtable,booktabs,dcolumn,tabularx,mathtools,multirow,colortbl,xcolor}
\usepackage[dvipdfmx]{graphics}
\usepackage{bmpsize}
\usepackage{amsthm}
\usepackage{enumitem}
\setlistdepth{20}
\renewlist{itemize}{itemize}{20}
\setlist[itemize]{label=•}
\renewlist{enumerate}{enumerate}{20}
\setlist[enumerate]{label=\arabic*.}
\setcounter{MaxMatrixCols}{20}
\setcounter{tocdepth}{3}
\newcommand{\rotin}{\text{\rotatebox[origin=c]{90}{$\in $}}}
\newcommand{\amap}[6]{\text{\raisebox{-0.7cm}{\begin{tikzpicture} 
  \node (a) at (0, 1) {$\textstyle{#2}$};
  \node (b) at (#6, 1) {$\textstyle{#3}$};
  \node (c) at (0, 0) {$\textstyle{#4}$};
  \node (d) at (#6, 0) {$\textstyle{#5}$};
  \node (x) at (0, 0.5) {$\rotin $};
  \node (x) at (#6, 0.5) {$\rotin $};
  \draw[->] (a) to node[xshift=0pt, yshift=7pt] {$\textstyle{\scriptstyle{#1}}$} (b);
  \draw[|->] (c) to node[xshift=0pt, yshift=7pt] {$\textstyle{\scriptstyle{#1}}$} (d);
\end{tikzpicture}}}}
\newcommand{\twomaps}[9]{\text{\raisebox{-0.7cm}{\begin{tikzpicture} 
  \node (a) at (0, 1) {$\textstyle{#3}$};
  \node (b) at (#9, 1) {$\textstyle{#4}$};
  \node (c) at (#9+#9, 1) {$\textstyle{#5}$};
  \node (d) at (0, 0) {$\textstyle{#6}$};
  \node (e) at (#9, 0) {$\textstyle{#7}$};
  \node (f) at (#9+#9, 0) {$\textstyle{#8}$};
  \node (x) at (0, 0.5) {$\rotin $};
  \node (x) at (#9, 0.5) {$\rotin $};
  \node (x) at (#9+#9, 0.5) {$\rotin $};
  \draw[->] (a) to node[xshift=0pt, yshift=7pt] {$\textstyle{\scriptstyle{#1}}$} (b);
  \draw[|->] (d) to node[xshift=0pt, yshift=7pt] {$\textstyle{\scriptstyle{#2}}$} (e);
  \draw[->] (b) to node[xshift=0pt, yshift=7pt] {$\textstyle{\scriptstyle{#1}}$} (c);
  \draw[|->] (e) to node[xshift=0pt, yshift=7pt] {$\textstyle{\scriptstyle{#2}}$} (f);
\end{tikzpicture}}}}
\renewcommand{\thesection}{第\arabic{section}部}
\renewcommand{\thesubsection}{\arabic{section}.\arabic{subsection}}
\renewcommand{\thesubsubsection}{\arabic{section}.\arabic{subsection}.\arabic{subsubsection}}
\everymath{\displaystyle}
\allowdisplaybreaks[4]
\usepackage{vtable}
\theoremstyle{definition}
\newtheorem{thm}{定理}[subsection]
\newtheorem*{thm*}{定理}
\newtheorem{dfn}{定義}[subsection]
\newtheorem*{dfn*}{定義}
\newtheorem{axs}[dfn]{公理}
\newtheorem*{axs*}{公理}
\renewcommand{\headfont}{\bfseries}
\makeatletter
  \renewcommand{\section}{%
    \@startsection{section}{1}{\z@}%
    {\Cvs}{\Cvs}%
    {\normalfont\huge\headfont\raggedright}}
\makeatother
\makeatletter
  \renewcommand{\subsection}{%
    \@startsection{subsection}{2}{\z@}%
    {0.5\Cvs}{0.5\Cvs}%
    {\normalfont\LARGE\headfont\raggedright}}
\makeatother
\makeatletter
  \renewcommand{\subsubsection}{%
    \@startsection{subsubsection}{3}{\z@}%
    {0.4\Cvs}{0.4\Cvs}%
    {\normalfont\Large\headfont\raggedright}}
\makeatother
\makeatletter
\renewenvironment{proof}[1][\proofname]{\par
  \pushQED{\qed}%
  \normalfont \topsep6\p@\@plus6\p@\relax
  \trivlist
  \item\relax
  {
  #1\@addpunct{.}}\hspace\labelsep\ignorespaces
}{%
  \popQED\endtrivlist\@endpefalse
}
\makeatother
\renewcommand{\proofname}{\textbf{証明}}
\usepackage{tikz,graphics}
\usepackage[dvipdfmx]{hyperref}
\usepackage{pxjahyper}
\hypersetup{
 setpagesize=false,
 bookmarks=true,
 bookmarksdepth=tocdepth,
 bookmarksnumbered=true,
 colorlinks=false,
 pdftitle={},
 pdfsubject={},
 pdfauthor={},
 pdfkeywords={}}
\begin{document}
%\hypertarget{ux8dddux96e2ux7a7aux9593}{%
\subsection{距離空間}%\label{ux8dddux96e2ux7a7aux9593}}\par
%\hypertarget{ux8dddux96e2ux7a7aux9593-1}{%
\subsubsection{距離空間}%\label{ux8dddux96e2ux7a7aux9593-1}}
\begin{dfn}
空集合でない集合$S$と写像$d:S \times S \rightarrow \mathbb{R}$が与えられたとき、次のことが成り立つとき、この組$(S,d)$を距離空間といい、その集合$S$を台集合、その写像$d$を距離関数という。さらに、$a,b \in S$なる元々$a$、$b$の組$(a,b)$のその距離関数$d$による像$d(a,b)$をその元々$a$、$b$の距離という。
\begin{itemize}
\item
  $\forall a,b \in S$に対し、$d(a,b) \geq 0$が成り立つ。
\item
  $\forall a,b \in S$に対し、$d(a,b) = 0$が成り立つならそのときに限り、$a = b$が成り立つ。
\item
  $\forall a,b \in S$に対し、$d(a,b) = d(b,a)$が成り立つ。
\item
  $\forall a,b,c \in S$に対し、$d(a,c) \leq d(a,b) + d(b,c)$が成り立つ。この不等式を三角不等式という。
\end{itemize}
\end{dfn}\par
例えば、絶対値を用いた写像$d_{E}$を用いた組$\left( \mathbb{R},d_{E} \right)$が挙げられる。これは先ほどの定理\ref{8.1.7.18}によって保証されて定義\ref{1次元Euclid空間における位相空間}として定義された1次元Euclid空間における位相空間と関係深い。これは解析学から生じた概念で位相空間の源流となっていると考えてもよい。詳しくは後述する。
\begin{thm}\label{8.2.1.1}
空集合でない集合$S$と写像$d:S \times S \rightarrow \mathbb{R}$が与えられたとき、次のことが成り立つならそのときに限り、
\begin{itemize}
\item
  $\forall a,b \in S$に対し、$d(a,b) \geq 0$が成り立つ。
\item
  $\forall a,b \in S$に対し、$d(a,b) = 0$が成り立つならそのときに限り、$a = b$が成り立つ。
\item
  $\forall a,b \in S$に対し、$d(a,b) = d(b,a)$が成り立つ。
\item
  $\forall a,b,c \in S$に対し、$d(a,c) \leq d(a,b) + d(b,c)$が成り立つ。
\end{itemize}
次のことが成り立つ。
\begin{itemize}
\item
  $\forall a,b \in S$に対し、$d(a,b) \geq 0$が成り立つ。
\item
  $\forall a,b \in S$に対し、$d(a,b) = 0$が成り立つならそのときに限り、$a = b$が成り立つ。
\item
  $\forall a,b,c \in S$に対し、$d(c,a) \leq d(a,b) + d(b,c)$が成り立つ。
\end{itemize}
\end{thm}
\begin{proof}
空集合でない集合$S$と写像$d:S \times S \rightarrow \mathbb{R}$が与えられたとき、次のことが成り立つなら、
\begin{itemize}
\item
  $\forall a,b \in S$に対し、$d(a,b) \geq 0$が成り立つ。
\item
  $\forall a,b \in S$に対し、$d(a,b) = 0$が成り立つならそのときに限り、$a = b$が成り立つ。
\item
  $\forall a,b \in S$に対し、$d(a,b) = d(b,a)$が成り立つ。
\item
  $\forall a,b,c \in S$に対し、$d(a,c) \leq d(a,b) + d(b,c)$が成り立つ。
\end{itemize}
明らかに次のことが成り立つ。
\begin{itemize}
\item
  $\forall a,b \in S$に対し、$d(a,b) \geq 0$が成り立つ。
\item
  $\forall a,b \in S$に対し、$d(a,b) = 0$が成り立つならそのときに限り、$a = b$が成り立つ。
\item
  $\forall a,b,c \in S$に対し、$d(c,a) \leq d(a,b) + d(b,c)$が成り立つ。
\end{itemize}\par
逆に、上のことが成り立つなら、$\forall a,b \in S$に対し、$d(b,a) \leq d(a,a) + d(a,b) = d(a,b)$が成り立つかつ、$d(a,b) \leq d(b,b) + d(b,a) = d(b,a)$が成り立つので、$d(a,b) = d(b,a)$が成り立つ。これにより、次のことが成り立つ。
\begin{itemize}
\item
  $\forall a,b \in S$に対し、$d(a,b) \geq 0$が成り立つ。
\item
  $\forall a,b \in S$に対し、$d(a,b) = 0$が成り立つならそのときに限り、$a = b$が成り立つ。
\item
  $\forall a,b \in S$に対し、$d(a,b) = d(b,a)$が成り立つ。
\item
  $\forall a,b,c \in S$に対し、$d(a,c) \leq d(a,b) + d(b,c)$が成り立つ。
\end{itemize}

\end{proof}
%\hypertarget{ux8dddux96e2ux7a7aux9593ux306bux304aux3051ux308bux4f4dux76f8ux7a7aux9593}{%
\subsubsection{距離空間における位相空間}%\label{ux8dddux96e2ux7a7aux9593ux306bux304aux3051ux308bux4f4dux76f8ux7a7aux9593}}
\begin{dfn}
距離空間$(S,d)$が与えられたとき、$\forall a \in S\forall\varepsilon \in \mathbb{R}^{+}$に対し、次式のように定義される集合$B(a,\varepsilon)$をその元$a$を中心とする半径$\varepsilon$の球体という\footnote{定義\ref{1次元Euclid空間における球体}と比較するとよいかもしれない。}。
\begin{align*}
B(a,\varepsilon) = \left\{ b \in S \middle| d(a,b) < \varepsilon \right\}
\end{align*}
\end{dfn}
\begin{thm}\label{8.2.1.2}
距離空間$(S,d)$が与えられたとき、その台集合$S$の部分集合$O$において、$\forall a \in O\exists\varepsilon \in \mathbb{R}^{+}$に対し、$B(a,\varepsilon) \subseteq O$が成り立つようなその集合$O$と空集合全体の集合が$\mathfrak{O}_{d}$とおかれると、組$\left( S,\mathfrak{O}_{d} \right)$は位相空間をなす。
\end{thm}
\begin{dfn}
この集合$\mathfrak{O}_{d}$をその距離空間$(S,d)$における位相、これの元をその距離空間$\mathfrak{O}_{d}$における開集合、その位相空間$\left( S,\mathfrak{O}_{d} \right)$をその距離空間$(S,d)$における位相空間という。
\end{dfn}
\begin{proof}
距離空間$(S,d)$が与えられたとき、その台集合$S$の部分集合$O$において、$\forall a \in O\exists\varepsilon \in \mathbb{R}^{+}$に対し、$B(a,\varepsilon) \subseteq O$が成り立つようなその集合$O$と空集合全体の集合が$\mathfrak{O}_{d}$とおかれると、$\forall a,b \in S$に対し、$d(a,b) < \varepsilon$となる正の実数$\varepsilon$が存在するので、$S,\emptyset \in \mathfrak{O}_{d}$が成り立つ。\par
ここで、$\forall O,P \in \mathfrak{O}_{d}$に対し、積集合$O \cap P$が空集合であれば、$O \cap P \in \mathfrak{O}_{d}$が成り立つし、空集合でなければ、これの任意の元$a$に対し、ある正の実数たち$\delta$、$\varepsilon$が存在して、$B(a,\delta) \subseteq O$かつ$B(a,\varepsilon) \subseteq P$が成り立つことになる。ここで、$\varepsilon' = \min\left\{ \delta,\varepsilon \right\}$とおかれれば、$B(a,\varepsilon) \subseteq O$かつ$B(a,\varepsilon) \subseteq P$が成り立つので、$B(a,\varepsilon) \subseteq O \cap P$が成り立ち、したがって、$O \cap P \in \mathfrak{O}_{d}$が成り立つ。\par
さらに、添数集合$\varLambda$によって添数づけられたその集合$\mathfrak{O}_{d}$の族$\left\{ O_{\lambda} \right\}_{\lambda \in \varLambda }$が与えられたとき、$\forall a \in \bigcup_{\lambda \in \varLambda} O_{\lambda}$に対し、$a \in O_{\lambda}$なる開集合$O_{\lambda}$とある正の実数$\varepsilon$が存在して$B(a,\varepsilon) \subseteq O_{\lambda}$が成り立つことになる。ここで、$O_{\lambda} \subseteq \bigcup_{\lambda \in \varLambda} O_{\lambda}$が成り立つので、$B(a,\varepsilon) \subseteq \bigcup_{\lambda \in \varLambda} O_{\lambda}$が成り立ち、したがって、$\bigcup_{\lambda \in \varLambda} O_{\lambda} \in \mathfrak{O}_{d}$が成り立つ。\par
以上より、その組$\left( S,\mathfrak{O}_{d} \right)$は位相空間をなす。
\end{proof}
\begin{thm}\label{8.2.1.3}
距離空間$(S,d)$が与えられたとき、$\forall a \in S$に対し、その元$a$を中心とする半径$\varepsilon$の球体$B(a,\varepsilon)$はその距離空間$(S,d)$における開集合である。\end{thm}\par
これにより、球体を開球、開球体ともいう。
\begin{proof} 定義より明らかである。
\end{proof}
\begin{thm}\label{8.2.1.4}
距離空間$(S,d)$の開基として、開球体全体の集合$\mathfrak{U}$が挙げられる。
\end{thm}
\begin{proof}
距離空間$(S,d)$が与えられたとき、これにおける開集合$O$の任意の元$a$に対し、ある正の実数$\varepsilon_{a}$が存在して$B\left( a,\varepsilon_{a} \right) \subseteq O$が成り立つので、$\bigcup_{a \in O} {B\left( a,\varepsilon_{a} \right)} \subseteq O$が成り立つ。また、上記より$O \subseteq \bigcup_{a \in O} {B\left( a,\varepsilon_{a} \right)}$が成り立つので、$O = \bigcup_{a \in O} {B\left( a,\varepsilon_{a} \right)}$が成り立つ。これにより、開球体全体の集合$\mathfrak{U}$は開基をなす。
\end{proof}
\begin{thm}\label{8.2.1.5}
距離空間$(S,d)$の開基として、半径が有理数の開球体全体の集合$\mathfrak{U}'$が挙げられる。
\end{thm}
\begin{proof}
距離空間$(S,d)$が与えられたとき、これにおける開集合$O$の任意の元$a$に対し、ある正の実数$\varepsilon_{a}$が存在して$B\left( a,\varepsilon_{a} \right) \subseteq O$が成り立つ。ここで、$0 < r_{a} < \varepsilon_{a}$が成り立つような有理数$r_{a}$が存在するので、$B\left( a,r_{a} \right) \subseteq B\left( a,\varepsilon_{a} \right) \subseteq O$が成り立つ。また、上記より$O \subseteq \bigcup_{a \in O} {B\left( a,r_{a} \right)}$が成り立つので、$O = \bigcup_{a \in O} {B\left( a,r_{a} \right)}$が成り立つ。これにより、半径が有理数の開球体全体の集合$\mathfrak{U}'$は開基をなす。
\end{proof}
\begin{thm}\label{8.2.1.6}
距離空間$(S,d)$が与えられたとき、$\forall a \in S$に対し、その元$a$の基本近傍系として、その元$a$を中心とする開球体全体の集合$\mathfrak{U}_{a}$が挙げられる。
\end{thm}
\begin{proof}
距離空間$(S,d)$が与えられたとき、$\forall a \in S$に対し、$a \in O$なる開集合$O$が存在するので、このような任意の開集合たち$O$に対し、定理\ref{8.2.1.4}よりその距離空間$(S,d)$の開基として、開球体全体の集合$\mathfrak{U}$が挙げられるので、その元$a$を中心とする開球体全体の集合$\mathfrak{U}_{a}$のある元$B(a,\varepsilon)$が存在して、$B(a,\varepsilon) \subseteq O$が成り立つかつ、$a \in B(a,\varepsilon) = {\mathrm{int}}{B(a,\varepsilon)}$が成り立つので、その集合$\mathfrak{U}_{a}$がその元$a$の基本近傍系をなす。
\end{proof}
\begin{thm}\label{8.2.1.7}
距離空間$(S,d)$が与えられたとき、$\forall a \in S$に対し、その元$a$の基本近傍系として、その元$a$を中心とする半径が有理数の開球体全体の集合$\mathfrak{U}_{a}'$が挙げられる。
\end{thm}
\begin{proof} 距離空間$(S,d)$が与えられたとき、定理\ref{8.2.1.6}より$\forall a \in S$に対し、その元$a$の基本近傍系として、その元$a$を中心とする開球体全体の集合$\mathfrak{U}_{a}$が挙げられる。ここで、このような元$B(a,\varepsilon)$がとられると、$0 < r < \varepsilon$なる有理数$r$が存在するので、$B(a,r) \subseteq B(a,\varepsilon) \subseteq O$が成り立つかつ、$a \in B(a,r) = {\mathrm{int}}{B(a,r)} \subseteq B(a,\varepsilon)$が成り立ち、したがって、その元$a$を中心とする半径が有理数の開球体全体の集合$\mathfrak{U}_{a}'$がその元$a$の基本近傍系をなす。
\end{proof}
\begin{thm}\label{8.2.1.8} 距離空間$(S,d)$は第1可算公理を満たす。
\end{thm}
\begin{proof}
距離空間$(S,d)$が与えられたとき、$\forall a \in S$に対し、定理\ref{8.2.1.7}のような基本近傍系$\mathfrak{U}_{a}'$がとられれば、次のようになることから、
\begin{align*}
{\#}\mathfrak{U}_{a}' \leq {\#}\mathbb{Q} = \aleph_{0}
\end{align*}
任意の距離空間$(S,d)$は第1可算公理を満たす。
\end{proof}
\begin{thm}\label{8.2.1.9}
距離空間$(S,d)$が可分であるならそのときに限り、その距離空間$(S,d)$は第2可算公理を満たす。
\end{thm}
\begin{proof}
距離空間$(S,d)$が与えられたとき、距離空間$(S,d)$が可分であるなら、定義よりあるその集合$S$のその集合$S$自身でないたかだか可算な部分集合$M$が存在して、${\mathrm{cl}}M = S$が成り立つ。$\forall a \in S$に対し、その元$a$を中心とする半径が有理数の開球体全体の集合$\mathfrak{U}_{a}'$を用いた和集合$\bigcup_{a \in M} \mathfrak{U}_{a}'$について、次のようになることから、
\begin{align*}
{\#}{\bigcup_{a \in M} \mathfrak{U}_{a}'} &= {\#}{\bigsqcup_{a \in M} \mathfrak{U}_{a}'}\\
&= \sum_{a \in M} {{\#}\mathfrak{U}_{a}'}\\
&\leq \sum_{a \in M} {{\#}\mathbb{Q}} = \aleph_{0}
\end{align*}
その和集合$\bigcup_{a \in M} \mathfrak{U}_{a}'$はたかだか可算である。任意の空集合でない開集合$O$、任意のこれの元$a$に対し、$B(a,\varepsilon) \subseteq O$なる開球体$B(a,\varepsilon)$が存在し、さらに、${\mathrm{cl}}M = S$が成り立つので、$B\left( a,\frac{\varepsilon}{2} \right) \subseteq S \setminus M$とすれば、$B\left( a,\frac{\varepsilon}{2} \right) \subseteq {\mathrm{int}}(S \setminus M) = S \setminus {\mathrm{cl}}M = \emptyset$となり矛盾している。したがって、$B\left( a,\frac{\varepsilon}{2} \right) \subseteq S \setminus M$が成り立たない、即ち、$B\left( a,\frac{\varepsilon}{2} \right) \cap M \neq \emptyset$が成り立つ。このとき、$a' \in B\left( a,\frac{\varepsilon}{2} \right)$なるその集合$M$の元$a'$が存在して、$d\left( a,a' \right) < \frac{\varepsilon}{2}$が成り立つ。ここで、$d\left( a,a' \right) < r < \frac{\varepsilon}{2}$なる有理数$r$がとられれば、$B\left( a',r \right) \in \bigcup_{a \in M} \mathfrak{U}_{a}'$が成り立つかつ、$a \in B\left( a',r \right) \subseteq B(a,\varepsilon) \subseteq O$が成り立つ。これにより、その和集合$\bigcup_{a \in M} \mathfrak{U}_{a}'$はその距離空間$(S,d)$の開基をなす。以上より、その距離空間$(S,d)$は第2可算公理を満たす。\par
逆は定理\ref{8.1.2.18}より明らかである。
\end{proof}
\begin{thm}\label{8.2.1.10}
距離空間$(S,d)$が与えられたとき、$\forall a \in S\forall\varepsilon \in \mathbb{R}^{+}$に対し、次式のように定義される集合たち$B^{*}(a,\varepsilon)$、$S(a,\varepsilon)$はいづれも閉集合である。
\begin{align*}
B^{*}(a,\varepsilon) &= \left\{ b \in S \middle| d(a,b) \leq \varepsilon \right\}\\
S(a,\varepsilon) &= \left\{ b \in S \middle| d(a,b) = \varepsilon \right\}
\end{align*}
\end{thm}
\begin{dfn}
$a \in S$、$\varepsilon \in \mathbb{R}^{+}$なるこのような集合たち$B^{*}(a,\varepsilon)$、$S(a,\varepsilon)$をそれぞれその元$a$を中心とする半径$\varepsilon$の閉球体、その元$a$を中心とする半径$\varepsilon$の球面という。
\end{dfn}\par
なお、必ずしも${\mathrm{cl}}{B(a,\varepsilon)} = B^{*}(a,\varepsilon)$、$\partial B(a,\varepsilon) = S(a,\varepsilon)$が成り立つとは限らない。
\begin{proof}
距離空間$(S,d)$が与えられたとき、$\forall a \in S\forall\varepsilon \in \mathbb{R}^{+}$に対し、次式のように定義される集合たち$B^{*}(a,\varepsilon)$、$S(a,\varepsilon)$について、
\begin{align*}
B^{*}(a,\varepsilon) &= \left\{ b \in S \middle| d(a,b) \leq \varepsilon \right\}\\
S(a,\varepsilon) &= \left\{ b \in S \middle| d(a,b) = \varepsilon \right\}
\end{align*}
$S \setminus B^{*}(a,\varepsilon) = \left\{ b \in S \middle| d(a,b) > \varepsilon \right\}$が成り立ち、ここで、$\exists a^{*} \in S \setminus B^{*}(a,\varepsilon)\forall\delta \in \mathbb{R}^{+}$に対し、$S \setminus B^{*}(a,\varepsilon) \subset B\left( a^{*},\delta \right)$が成り立つと仮定すると、$B^{*}(a,\varepsilon) \cap B\left( a^{*},\delta \right) \neq \emptyset$が成り立つことになる。ここで、このような元$b$がとられれば、$d(a,b) \leq \varepsilon$が成り立つかつ、$d\left( a^{*},b \right) \leq \delta$が成り立つ。このとき、次のようになり、
\begin{align*}
d\left( a,a^{*} \right) \leq d(a,b) + d\left( a^{*},b \right) \leq \varepsilon + \delta
\end{align*}
$0 \leq d\left( a,a^{*} \right)$が成り立つので、その正の実数$\varepsilon + \delta$の任意性より$d\left( a,a^{*} \right) = 0$が成り立ち、したがって、$a = a^{*}$が成り立つ。しかしながら、このことはその元$a$がその集合$B^{*}(a,\varepsilon)$の元でもあるかつ、その集合$S \setminus B^{*}(a,\varepsilon)$の元でもあることになるので、矛盾している。したがって、$\forall a^{*} \in S \setminus B^{*}(a,\varepsilon)\exists\delta \in \mathbb{R}^{+}$に対し、$B\left( a^{*},\delta \right) \subseteq S \setminus B^{*}(a,\varepsilon)$が成り立つことになる。ゆえに、その集合$S \setminus B^{*}(a,\varepsilon)$は開集合である、即ち、その集合$B^{*}(a,\varepsilon)$は閉集合であることが示された。\par
また、次のようになることから、
\begin{align*}
S \setminus S(a,\varepsilon) &= \left\{ b \in S \middle| d(a,b) \neq \varepsilon \right\}\\
&= \left\{ b \in S \middle| d(a,b) < \varepsilon < d(a,b) \right\}\\
&= \left\{ b \in S \middle| d(a,b) < \varepsilon \right\} \cap \left\{ b \in S \middle| \varepsilon < d(a,b) \right\}\\
&= B(a,\varepsilon) \cap S \setminus B^{*}(a,\varepsilon)
\end{align*}
それらの集合たち$B(a,\varepsilon)$、$S \setminus B^{*}(a,\varepsilon)$はいづれも開集合であるから、その集合$S \setminus S(a,\varepsilon)$も開集合となる。したがって、その集合$S(a,\varepsilon)$は閉集合である。
\end{proof}
%\hypertarget{ux8dddux96e2ux7a7aux9593ux306bux304aux3051ux308bux6975ux9650}{%
\subsubsection{距離空間における極限}%\label{ux8dddux96e2ux7a7aux9593ux306bux304aux3051ux308bux6975ux9650}}
\begin{dfn}
距離空間$(S,d)$が与えられたとき、その集合$S$の元の列$\left( a_{n} \right)_{n \in \mathbb{N}}$について、$\forall\varepsilon \in \mathbb{R}^{+}\exists n_{0} \in \mathbb{N}\forall n \in \mathbb{N}$に対し、$n_{0} < n$が成り立つなら、$d\left( a_{n},a \right) < \varepsilon$が成り立つとき、その元の列$\left( a_{n} \right)_{n \in \mathbb{N}}$はその集合$S$の元$a$に収束するといい、このことを$\lim_{n \rightarrow \infty}a_{n} = a$と書く。また、その元$a$をその元の列$\left( a_{n} \right)_{n \in \mathbb{N}}$の極限、極限点などという\footnote{この定義は定義\ref{有向点族の収束}と矛盾していないことが後に示されるので、そこまで動揺せずに読み進まれてもよいかと思います! }。
\end{dfn}
\begin{thm}\label{8.2.1.11}
距離空間$(S,d)$が与えられたとき、その集合$S$の元の列$\left( a_{n} \right)_{n \in \mathbb{N}}$の極限$a$が存在すれば、これは一意的である。
\end{thm}
\begin{proof}
距離空間$(S,d)$が与えられたとき、その集合$S$の元の列$\left( a_{n} \right)_{n \in \mathbb{N}}$の2つの極限たち$a$、$a'$が存在すれば、$\forall\varepsilon \in \mathbb{R}^{+}\exists n_{0} \in \mathbb{N}\forall n \in \mathbb{N}$に対し、$n_{0} < n$が成り立つなら、$d\left( a_{n},a \right) < \varepsilon$が成り立つかつ、$d\left( a_{n},a' \right) < \varepsilon$が成り立つ。ここで、三角不等式より$d\left( a,a' \right) \leq d\left( a_{n},a \right) + d\left( a_{n},a' \right) < 2\varepsilon$が成り立つので、$d\left( a,a' \right) = 0$が成り立ち、したがって、$a = a'$が成り立つ。よって、その集合$S$の元の列$\left( a_{n} \right)_{n \in \mathbb{N}}$の極限$a$が存在すれば、これは一意的である。
\end{proof}
\begin{thm}\label{8.2.1.12}
距離空間$(S,d)$が与えられたとき、その集合$S$の元の列$\left( a_{n} \right)_{n \in \mathbb{N}}$の極限$a$が存在すれば、$\lim_{n \rightarrow \infty}a_{n} = a$が成り立つならそのときに限り、$\lim_{n \rightarrow \infty}{d\left( a_{n},a \right)} = 0$が成り立つ。
\end{thm}
\begin{proof}
距離空間$(S,d)$が与えられたとき、その集合$S$の元の列$\left( a_{n} \right)_{n \in \mathbb{N}}$の極限$a$が存在すれば、定義より$\lim_{n \rightarrow \infty}a_{n} = a$が成り立つならそのときに限り、$\forall\varepsilon \in \mathbb{R}^{+}\exists n_{0} \in \mathbb{N}\forall n \in \mathbb{N}$に対し、$n_{0} < n$が成り立つなら、$d\left( a_{n},a \right) < \varepsilon$が成り立つ。ここで、$d\left( a_{n},a \right) = \left| d\left( a_{n},a \right) - 0 \right| < \varepsilon$が成り立つので、これが成り立つならそのときに限り、$\forall\varepsilon \in \mathbb{R}^{+}\exists\delta \in \mathbb{N}\forall n \in \mathbb{N}$に対し、$\delta < n$が成り立つなら、$\left| d\left( a_{n},a \right) - 0 \right| < \varepsilon$が成り立つ。これはまさしく$\varepsilon $-$\delta $論法そのものなので\footnote{実数列$\left( a_{n} \right)_{n \in \mathbb{N}}$が与えられたとき、$\lim_{n \rightarrow \infty}a_{n} = \alpha$が成り立つとは、$\forall\varepsilon \in \mathbb{R}^{+}\exists\delta \in \mathbb{N}\forall n \in \mathbb{N}$に対し、$\delta < n$が成り立つなら、$\left| a_{n} - \alpha \right| < \varepsilon$が成り立つことを指すように定義されているんだったと思います。}、$\lim_{n \rightarrow \infty}a_{n} = a$が成り立つならそのときに限り、$\lim_{n \rightarrow \infty}{d\left( a_{n},a \right)} = 0$が成り立つ。
\end{proof}
\begin{thm}\label{8.2.1.13}
距離空間$(S,d)$が与えられたとき、$\forall a \in S\forall M \in \mathfrak{P}(S)$に対し、次のことが成り立つ。
\begin{itemize}
\item
  その元$a$がその部分集合$M$の触点である、即ち、$a \in {\mathrm{cl}}M$が成り立つならそのときに限り、その元$a$がその集合$M$のある元の列$\left( a_{n} \right)_{n \in \mathbb{N}}$が存在してこれの極限となる\footnote{この定理により、距離空間$(S,d)$が与えられたとき、$\forall M \in \mathfrak{P}(S)$に対し、その集合$M$の元の列$\left( a_{n} \right)_{n \in \mathbb{N}}$の極限全体の集合を${\mathrm{cl}}M$とおき、写像$cl\mathfrak{:P}(S)\mathfrak{\rightarrow P}(S);M \mapsto {\mathrm{cl}}M$が与えられれば、その写像$cl$を閉包作用子とする位相が定理\ref{8.1.1.14}より構成されることができるようになります。}。
\item
  その元$a$がその部分集合$M$の内点である、即ち、$a \in {\mathrm{int}}M$が成り立つならそのときに限り、その元$a$が極限であるようなその集合$S$の任意の元の列$\left( a_{n} \right)_{n \in \mathbb{N}}$に対し、ある自然数$n_{0}$が存在して、任意の自然数$n$に対し、$n_{0} < n$が成り立つなら、$a_{n} \in M$が成り立つ。
\item
  その元$a$がその部分集合$M$の集積点であるならそのときに限り、任意の自然数$n$に対し、$a_{n} \neq a$なるその集合$M$のある元の列$\left( a_{n} \right)_{n \in \mathbb{N}}$の極限がその元$a$である。
\item
  その元$a$がその部分集合$M$の孤立点であるならそのときに限り、その元$a$が極限であるようなその集合$M$の任意の元の列$\left( a_{n} \right)_{n \in \mathbb{N}}$に対し、ある自然数$n_{0}$が存在して、任意の自然数$n$に対し、$n_{0} < n$が成り立つなら、$a_{n} = a$が成り立つ。
\end{itemize}
\end{thm}
\begin{proof}
距離空間$(S,d)$が与えられたとき、$\forall a \in S\forall M \in \mathfrak{P}(S)$に対し、その元$a$がその部分集合$M$の触点である、即ち、$a \in {\mathrm{cl}}M$が成り立つなら、定理\ref{8.1.2.17}と定理\ref{8.2.1.6}より$\forall n \in \mathbb{N}$に対し、$B\left( a,\frac{1}{n} \right) \cap M \neq \emptyset$が成り立ち、したがって、その集合$M$のある元の列$\left( a_{n} \right)_{n \in \mathbb{N}}$が存在して、$\forall n \in \mathbb{N}$に対し、$a_{n} \in B\left( a,\frac{1}{n} \right)$が成り立つ。このとき、$d\left( a_{n},a \right) < \frac{1}{n}$が成り立つので、$\lim_{n \rightarrow \infty}a_{n} = a \in {\mathrm{cl}}(M)$が成り立つ。\par
逆に、$\forall a \in S\forall M \in \mathfrak{P}(S)$に対し、その元$a$がその集合$M$のある元の列$\left( a_{n} \right)_{n \in \mathbb{N}}$が存在してこれの極限となるなら、$\forall\varepsilon \in \mathbb{R}^{+}\exists n_{0} \in \mathbb{N}\forall n \in \mathbb{N}$に対し、$n_{0} < n$が成り立つなら、$d\left( a_{n},a \right) < \varepsilon$が成り立つので、$a_{n} \in B(a,\varepsilon)$が成り立つ。これにより、$B(a,\varepsilon) \cap M \neq \emptyset$が成り立つ。定理\ref{8.1.2.17}と定理\ref{8.2.1.6}よりよって、$a \in {\mathrm{cl}}M$が成り立つ。\par
$\forall a \in S\forall M \in \mathfrak{P}(S)$に対し、その元$a$がその部分集合$M$の内点である、即ち、$a \in {\mathrm{int}}M$が成り立つならそのときに限り、その元$a$は集合${\mathrm{cl}}(S \setminus M)$の元でないことになるので、上記の議論によりその元$a$がその集合$M$の任意の元の列$\left( a_{n} \right)_{n \in \mathbb{N}}$の極限とならない。したがって、その元$a$が極限であるようなその集合$S$の任意の元の列$\left( a_{n} \right)_{n \in \mathbb{N}}$に対し、任意の自然数$n$に対し、ある自然数$n_{0}$が存在して、$a_{n} \notin M$が成り立つかつ、$n_{0} < n$が成り立つということにならない、即ち、ある自然数$n_{0}$が存在して、任意の自然数$n$に対し、$n_{0} < n$が成り立つなら、$a_{n} \in M$が成り立つことになる。\par
その元$a$がその部分集合$M$の集積点であるならそのときに限り、$a \in {\mathrm{cl}}\left( M \setminus \left\{ a \right\} \right)$が成り立ち、上記の議論により、これが成り立つならそのときに限り、その元$a$がその集合$M \setminus \left\{ a \right\}$のある元の列$\left( a_{n} \right)_{n \in \mathbb{N}}$が存在してこれの極限となるのであった。これは任意の自然数$n$に対し、$a_{n} \neq a$なるその集合$M$のある元の列$\left( a_{n} \right)_{n \in \mathbb{N}}$の極限がその元$a$であるということを意味する。\par
その元$a$がその部分集合$M$の孤立点であるならそのときに限り、$a \in M$が成り立つかつ、$a \in {\mathrm{cl}}\left( M \setminus \left\{ a \right\} \right)$が成り立たなく、上記の議論により、これが成り立つならそのときに限り、$a \in M$が成り立つかつ、その元$a$がその集合$M \setminus \left\{ a \right\}$の任意の元の列$\left( a_{n} \right)_{n \in \mathbb{N}}$の極限とならないのであった。ゆえに、任意の自然数$n$に対し、$a_{n} \neq a$が成り立つようなその元$a$が極限であるようなその集合$M$の元の列$\left( a_{n} \right)_{n \in \mathbb{N}}$は存在しない、即ち、その元$a$が極限であるようなその集合$M$の任意の元の列$\left( a_{n} \right)_{n \in \mathbb{N}}$に対し、ある自然数$n_{0}$が存在して、任意の自然数$n$に対し、$n_{0} < n$が成り立つなら、$a_{n} = a$が成り立つことになる。
\end{proof}
%\hypertarget{ux8dddux96e2ux7a7aux9593ux306bux304aux3051ux308bux9023ux7d9aux5199ux50cf}{%
\subsubsection{距離空間における連続写像}%\label{ux8dddux96e2ux7a7aux9593ux306bux304aux3051ux308bux9023ux7d9aux5199ux50cf}}
\begin{thm}\label{8.2.1.14}
2つの距離空間たち$(S,d)$、$(T,e)$が与えられたとき、写像$f:S \rightarrow T$について、次のことは同値である\footnote{この辺りでやっと解析学でお馴染みの概念に近づいてきましたね…(*´Д`)}。
\begin{itemize}
\item
  その写像$f$がその集合$S$の元$a$で連続である。
\item
  $\forall\varepsilon \in \mathbb{R}^{+}\exists\delta \in \mathbb{R}^{+}\forall b \in S$に対し、$d(b,a) < \delta$が成り立つなら、$e\left( f(b),f(a) \right) < \varepsilon$が成り立つ。
\item
  その集合$S$の任意の元の列$\left( a_{n} \right)_{n \in \mathbb{N}}$に対し、$\lim_{n \rightarrow \infty}a_{n} = a$が成り立つなら、$\lim_{n \rightarrow \infty}{f\left( a_{n} \right)} = f(a)$が成り立つ。
\end{itemize}
\end{thm}
\begin{proof}
2つの距離空間たち$(S,d)$、$(T,e)$が与えられたとき、写像$f:S \rightarrow T$について、その写像$f$がその集合$S$の元$a$で連続であるなら、$\forall\varepsilon \in \mathbb{R}^{+}\exists\delta \in \mathbb{R}^{+}$に対し、球体$B\left( f(a),\varepsilon \right)$は開集合なので、値域$V\left( f^{- 1}|B\left( f(a),\varepsilon \right) \right)$も開集合であることになる。したがって、$B(a,\delta) \subseteq V\left( f^{- 1}|B\left( f(a),\varepsilon \right) \right)$が成り立つ。これにより、$\forall b \in S$に対し、$d(b,a) < \delta$が成り立つなら、$b \in B(a,\delta)$が成り立ち、したがって、$b \in V\left( f^{- 1}|B\left( f(a),\varepsilon \right) \right)$が成り立つ、即ち、$f(b) \in B\left( f(a),\varepsilon \right)$が成り立つので、$e\left( f(b),f(a) \right) < \varepsilon$が成り立つ。\par
$\forall\varepsilon \in \mathbb{R}^{+}\exists\delta \in \mathbb{R}^{+}\forall b \in S$に対し、$d(b,a) < \delta$が成り立つなら、$e\left( f(b),f(a) \right) < \varepsilon$が成り立つとき、極限がその元$a$であるようなその集合$S$の元の列$\left( a_{n} \right)_{n \in \mathbb{N}}$について、$\lim_{n \rightarrow \infty}a_{n} = a$が成り立つので、$\forall\varepsilon \in \mathbb{R}^{+}\exists\delta \in \mathbb{R}^{+}\exists n_{0} \in \mathbb{N}\forall n \in \mathbb{N}\forall b \in S$に対し、$n_{0} < n$が成り立つなら、$d\left( a_{n},a \right) < \delta$が成り立つので、$e\left( f\left( a_{n} \right),f(a) \right) < \varepsilon$が成り立つ。したがって、その集合$S$の任意の元の列$\left( a_{n} \right)_{n \in \mathbb{N}}$に対し、$\lim_{n \rightarrow \infty}a_{n} = a$が成り立つなら、$\lim_{n \rightarrow \infty}{f\left( a_{n} \right)} = f(a)$が成り立つ。\par
その集合$S$の任意の元の列$\left( a_{n} \right)_{n \in \mathbb{N}}$に対し、$\lim_{n \rightarrow \infty}a_{n} = a$が成り立つなら、$\lim_{n \rightarrow \infty}{f\left( a_{n} \right)} = f(a)$が成り立つかつ、その写像$f$がその集合$S$の元$a$で連続でないとしよう。このとき、その写像$f$がその集合$S$の元$a$で連続であるならそのときに限り、上記の議論と同様にして、$\forall\varepsilon \in \mathbb{R}^{+}\exists\delta \in \mathbb{R}^{+}$に対し、$B(a,\delta) \subseteq V\left( f^{- 1}|B\left( f(a),\varepsilon \right) \right)$が成り立つ。したがって、この仮定の下では、$\exists\varepsilon \in \mathbb{R}^{+}\forall\delta \in \mathbb{R}^{+}$に対し、$B(a,\delta) \supset V\left( f^{- 1}|B\left( f(a),\varepsilon \right) \right)$が成り立つことになる。これにより、$B(a,\delta) \cap S \setminus V\left( f^{- 1}|B\left( f(a),\varepsilon \right) \right) \neq \emptyset$が成り立ち、特に、$\forall n \in \mathbb{N}$に対し、$B\left( a,\frac{1}{n} \right) \cap S \setminus V\left( f^{- 1}|B\left( f(a),\varepsilon \right) \right) \neq \emptyset$が成り立つ。これにより、その集合$B\left( a,\frac{1}{n} \right) \cap S \setminus V\left( f^{- 1}|B\left( f(a),\varepsilon \right) \right)$は空集合でないので、この集合の元の列$\left( a_{n} \right)_{n \in \mathbb{N}}$が存在して、$a_{n} \in B\left( a,\frac{1}{n} \right)$が成り立つ、即ち、$d\left( a_{n},a \right) < \frac{1}{n}$が成り立つことになるので、$\lim_{n \rightarrow \infty}a_{n} = a$が成り立つ。一方で、$a_{n} \in S \setminus V\left( f^{- 1}|B\left( f(a),\varepsilon \right) \right)$も成り立つので、$a_{n} \in V\left( f^{- 1}|B\left( f(a),\varepsilon \right) \right)$が成り立たない、即ち、$f\left( a_{n} \right) \in B\left( f(a),\varepsilon \right)$が成り立たないことになる。したがって、$e\left( f\left( a_{n} \right),f(a) \right) \geq \varepsilon$が成り立ち、これが任意の自然数$n$に対して成り立つので、$\lim_{n \rightarrow \infty}{f\left( a_{n} \right)} = f(a)$が成り立たないことになる。しかしながら、これは仮定に矛盾するので、その集合$S$の任意の元の列$\left( a_{n} \right)_{n \in \mathbb{N}}$に対し、$\lim_{n \rightarrow \infty}a_{n} = a$が成り立つなら、$\lim_{n \rightarrow \infty}{f\left( a_{n} \right)} = f(a)$が成り立つなら、その写像$f$がその集合$S$の元$a$で連続であることが示された。
\end{proof}
%\hypertarget{ux4f4dux76f8ux7684ux306aux540cux5024}{%
\subsubsection{位相的な同値}%\label{ux4f4dux76f8ux7684ux306aux540cux5024}}
\begin{dfn}
距離空間たち$(S,d)$、$(S,e)$が与えられたとき、これらの距離空間における位相たち$\mathfrak{O}_{d}$、$\mathfrak{O}_{e}$が$\mathfrak{O}_{d} = \mathfrak{O}_{e}$を満たすとき、これらの距離関数たち$d$、$e$は位相的に同値であるという。
\end{dfn}
\begin{thm}\label{8.2.1.15}
距離空間たち$(S,d)$、$(S,e)$が与えられたとき、これらの距離関数たち$d$、$e$が位相的に同値であるならそのときに限り、恒等写像$I_{S}$がその位相空間$\left( S,\mathfrak{O}_{d} \right)$からその位相空間$\left( S,\mathfrak{O}_{e} \right)$への同相写像となる。
\end{thm}
\begin{proof}
距離空間たち$(S,d)$、$(S,e)$が与えられたとき、これらの距離関数たち$d$、$e$が位相的に同値であるならそのときに限り、これらの距離空間における位相たち$\mathfrak{O}_{d}$、$\mathfrak{O}_{e}$が$\mathfrak{O}_{d} = \mathfrak{O}_{e}$を満たすのであった。このとき、恒等写像$I_{S}$について、$\forall O \in \mathfrak{O}_{e}$に対し、$V\left( I_{S}^{- 1}|O \right) = O \in \mathfrak{O}_{e} = \mathfrak{O}_{d}$が成り立つかつ、このことはその逆写像$I_{S}^{- 1}$に対しても同じようにいえるので、その恒等写像$I_{S}$がその位相空間$\left( S,\mathfrak{O}_{d} \right)$からその位相空間$\left( S,\mathfrak{O}_{e} \right)$への同相写像となる。\par
逆に、これが成り立つなら、$\forall O \in \mathfrak{O}_{e}$に対し、$V\left( I_{S}^{- 1}|O \right) = O \in \mathfrak{O}_{d}$が成り立つので、$\mathfrak{O}_{e} \subseteq \mathfrak{O}_{d}$が成り立つかつ、逆についても同様なことがいえるので、これらの距離空間における位相たち$\mathfrak{O}_{d}$、$\mathfrak{O}_{e}$が$\mathfrak{O}_{d} = \mathfrak{O}_{e}$を満たす、即ち、これらの距離関数たち$d$、$e$が位相的に同値であることが示された。
\end{proof}
\begin{thm}\label{8.2.1.16}
距離空間たち$(S,d)$、$(S,e)$が与えられたとき、次のことは同値である。
\begin{itemize}
\item
  これらの距離関数たち$d$、$e$が位相的に同値である\footnote{これが定理\ref{8.2.1.15}の上位互換となります。}。
\item
  恒等写像$I_{S}$がその位相空間$\left( S,\mathfrak{O}_{d} \right)$からその位相空間$\left( S,\mathfrak{O}_{e} \right)$への同相写像となる。
\item
  $\forall a \in S\forall\varepsilon \in \mathbb{R}^{+}\exists\delta \in \mathbb{R}^{+}$に対し、$d(b,a) < \delta$なら$e(b,a) < \varepsilon$が成り立つかつ、$e(b,a) < \delta$なら$d(b,a) < \varepsilon$も成り立つ。
\item
  その集合$S$の任意の点列$\left( a_{n} \right)_{n \in \mathbb{N}}$に対し、その元の列$\left( a_{n} \right)_{n \in \mathbb{N}}$の極限がこれらの距離空間$(S,d)$、$(S,e)$いづれにおいてもその集合$S$の元$a$である。
\end{itemize}
\end{thm}
\begin{proof} 距離空間たち$(S,d)$、$(S,e)$が与えられたとき、定理\ref{8.2.1.15}よりこれらの距離関数たち$d$、$e$が位相的に同値であるならそのときに限り、恒等写像$I_{S}$がその位相空間$\left( S,\mathfrak{O}_{d} \right)$からその位相空間$\left( S,\mathfrak{O}_{e} \right)$への同相写像となるのであった。\par
このとき、その恒等写像$I_{S}$はその位相空間$\left( S,\mathfrak{O}_{d} \right)$からその位相空間$\left( S,\mathfrak{O}_{e} \right)$への連続写像であるかつ、その位相空間$\left( S,\mathfrak{O}_{e} \right)$からその位相空間$\left( S,\mathfrak{O}_{d} \right)$への連続写像でもあるので、これが成り立つならそのときに限り、定理\ref{8.2.1.14}より$\forall\varepsilon \in \mathbb{R}^{+}\exists\delta \in \mathbb{R}^{+}\forall b \in S$に対し、$d(b,a) < \delta$なら$e(b,a) < \varepsilon$が成り立つかつ、$e(b,a) < \delta$なら$d(b,a) < \varepsilon$が成り立つ。\par
また、その恒等写像$I_{S}$はその位相空間$\left( S,\mathfrak{O}_{d} \right)$からその位相空間$\left( S,\mathfrak{O}_{e} \right)$への連続写像であるかつ、その位相空間$\left( S,\mathfrak{O}_{e} \right)$からその位相空間$\left( S,\mathfrak{O}_{d} \right)$への連続写像でもあるので、これが成り立つならそのときに限り、定理\ref{8.2.1.14}より$\lim_{n \rightarrow \infty}a_{n} = a$が成り立つなら、これらの距離空間$(S,d)$、$(S,e)$いづれにおいても次式が成り立つことから、
\begin{align*}
\lim_{n \rightarrow \infty}a_{n} = \lim_{n \rightarrow \infty}{I_{S}\left( a_{n} \right)} = I_{S}(a) = a
\end{align*}
その集合$S$の任意の点列$\left( a_{n} \right)_{n \in \mathbb{N}}$と任意の元$a$に対し、その元の列$\left( a_{n} \right)_{n \in \mathbb{N}}$の極限がその元$a$となる。
\end{proof}
%\hypertarget{ux90e8ux5206ux8dddux96e2ux7a7aux9593}{%
\subsubsection{部分距離空間}%\label{ux90e8ux5206ux8dddux96e2ux7a7aux9593}}
\begin{thm}\label{8.2.1.17}
距離空間$(S,d)$が与えられたとき、$\forall M \in \mathfrak{P}(S)$に対し、$d|M \times M = d_{M}$とおかれれば、その組$\left( M,d_{M} \right)$は距離空間をなす。
\end{thm}
\begin{dfn}
上のように構成された距離空間$\left( M,d_{M} \right)$をその距離空間$(S,d)$の部分距離空間という。
\end{dfn}
\begin{proof}
距離空間$(S,d)$が与えられたとき、$\forall M \in \mathfrak{P}(S)$に対し、$d|M \times M = d_{M}$とおかれれば、$\forall a,b \in M$に対し、$a,b \in S$も成り立つから、$d_{M}(a,b) = d(a,b) \geq 0$が成り立つ。$\forall a,b \in M$に対し、$a,b \in S$も成り立つから、$d_{M}(a,b) = d(a,b)$が成り立つならそのときに限り、$a = b$が成り立つ。$\forall a,b \in M$に対し、$a,b \in S$も成り立つから、$d_{M}(a,b) = d(a,b) = d(b,a) = d_{M}(b,a)$が成り立つ。$\forall a,b,c \in M$に対し、$a,b,c \in S$も成り立つから、$d_{M}(a,c) = d(a,c) \leq d(a,b) + d(b,c) = d_{M}(a,b) + d_{M}(b,c)$が成り立つ。これにより、次のことが成り立つ。
\begin{itemize}
\item
  $\forall a,b \in M$に対し、$d_{M}(a,b) \geq 0$が成り立つ。
\item
  $\forall a,b \in M$に対し、$d_{M}(a,b) = 0$が成り立つならそのときに限り、$a = b$が成り立つ。
\item
  $\forall a,b \in M$に対し、$d_{M}(a,b) = d_{M}(b,a)$が成り立つ。
\item
  $\forall a,b,c \in M$に対し、$d_{M}(a,c) \leq d_{M}(a,b) + d_{M}(b,c)$が成り立つ。
\end{itemize}
以上より、その組$\left( M,d_{M} \right)$は距離空間をなす。
\end{proof}
\begin{thm}\label{8.2.1.18}
距離空間$(S,d)$とこれの部分距離空間$\left( M,d_{M} \right)$が与えられたとき、この部分距離空間$\left( M,d_{M} \right)$における位相空間$\left( M,\mathfrak{O}_{d_{M}} \right)$はその距離空間$(S,d)$における位相空間$\left( S,\mathfrak{O} \right)$の部分位相空間$\left( M,\left( \mathfrak{O}_{d} \right)_{M} \right)$に等しい、即ち、$\mathfrak{O}_{d_{M}} = \left( \mathfrak{O}_{d} \right)_{M}$が成り立つ。
\end{thm}
\begin{proof}
距離空間$(S,d)$とこれの部分距離空間$\left( M,d_{M} \right)$が与えられたとき、この部分距離空間$\left( M,d_{M} \right)$における位相空間$\left( M,\mathfrak{O}_{d_{M}} \right)$において、$\forall O \in \mathfrak{O}_{d_{M}}$に対し、$O \in \mathfrak{O}_{d_{M}}$が成り立つならそのときに限り、$\forall a \in O\exists\varepsilon \in \mathbb{R}^{+}$に対し、$B(a,\varepsilon) \subseteq O$が成り立つことになる。このとき、$\forall b \in B(a,\varepsilon)$に対し、$d_{M}(a,b) = d(a,b) < \varepsilon$が成り立つので、その部分距離空間$\left( M,d_{M} \right)$におけるその開球体$B(a,\varepsilon)$はその距離空間$(S,d)$における開球体でもある。したがって、その開集合$O$はその位相空間$\left( S,\mathfrak{O}_{d} \right)$における開集合でもある。このとき、$O = O \cap M$が成り立つので、次式が成り立つ。
\begin{align*}
\mathfrak{O}_{d_{M}} \subseteq \left\{ O'\in \mathfrak{P}(M) \middle| \exists O \in \mathfrak{O}_{d}\left[ O' = O \cap M \right] \right\}
\end{align*}\par
逆に、その集合$M$の部分集合$O'$で$O' = O \cap M$なるその距離空間$(S,d)$における開集合$O$が存在するとき、$\exists a \in O'\forall\varepsilon \in \mathbb{R}^{+}$に対し、$O' \subset B(a,\varepsilon)$が成り立つとすれば、その差集合$B(a,\varepsilon) \setminus O'$は空集合でなくこれの元$b$がとられれば、$0 \leq d(a,b) < \varepsilon$が成り立ち、その正の実数$\varepsilon$の任意性より$0 = d(a,b)$が成り立ち、したがって、$a = b$が成り立つことになる。しかしながら、これはその元$a$がその集合$O'$の元であるかつ、その集合$O'$の元でもないことを意味しており矛盾している。したがって、$\forall a \in O'\exists\varepsilon \in \mathbb{R}^{+}$に対し、$B(a,\varepsilon) \subseteq O'$が成り立つ。このことはその部分距離空間$\left( M,d_{M} \right)$についても同じことがいえるので、次式が成り立つ。
\begin{align*}
\mathfrak{O}_{d_{M}} \supseteq \left\{ O'\in \mathfrak{P}(M) \middle| \exists O \in \mathfrak{O}_{d}\left[ O' = O \cap M \right] \right\}
\end{align*}\par
以上、定理\ref{8.1.4.7}より$\mathfrak{O}_{d_{M}} = \left\{ O'\in \mathfrak{P}(M) \middle| \exists O \in \mathfrak{O}_{d}\left[ O' = O \cap M \right] \right\} = \left( \mathfrak{O}_{d} \right)_{M}$が成り立つので、その部分距離空間$\left( M,d_{M} \right)$における位相空間$\left( M,\mathfrak{O}_{d_{M}} \right)$はその距離空間$(S,d)$における位相空間$\left( S,\mathfrak{O} \right)$の部分位相空間$\left( M,\left( \mathfrak{O}_{d} \right)_{M} \right)$に等しい。
\end{proof}
%\hypertarget{ux76f4ux7a4dux8dddux96e2ux7a7aux9593}{%
\subsubsection{直積距離空間}%\label{ux76f4ux7a4dux8dddux96e2ux7a7aux9593}}
\begin{thm}\label{8.2.1.19}
添数集合$\varLambda_{n}$によって添数づけられた距離空間たちの族$\left\{ \left( S_{i},d_{i} \right) \right\}_{i \in \varLambda_{n}}$が与えられたとき、次式のように写像$d$が定義されると、
\begin{align*}
d:\prod_{i \in \varLambda_{n}} S_{i} \times \prod_{i \in \varLambda_{n}} S_{i} \rightarrow \mathbb{R};\left( \left( a_{i} \right)_{i \in \varLambda_{n}},\left( b_{i} \right)_{i \in \varLambda_{n}} \right) \mapsto \sqrt{\sum_{i \in \varLambda_{n}} {d_{i}\left( a_{i},b_{i} \right)}^{2}}
\end{align*}
その組$\left( \prod_{i \in \varLambda_{n}} S_{i},d \right)$は距離空間をなす。
\end{thm}
\begin{dfn}
上のように構成された距離空間$\left( \prod_{i \in \varLambda_{n}} S_{i},d \right)$をその添数集合$\varLambda_{n}$によって添数づけられた距離空間たちの族$\left\{ \left( S_{i},d_{i} \right) \right\}_{i \in \varLambda_{n}}$の直積距離空間という。
\end{dfn}
\begin{proof}
添数集合$\varLambda_{n}$によって添数づけられた距離空間たちの族$\left\{ \left( S_{i},d_{i} \right) \right\}_{i \in \varLambda_{n}}$が与えられたとき、次式のように写像$d$が定義されると、
\begin{align*}
d:\prod_{i \in \varLambda_{n}} S_{i} \times \prod_{i \in \varLambda_{n}} S_{i} \rightarrow \mathbb{R};\left( \left( a_{i} \right)_{i \in \varLambda_{n}},\left( b_{i} \right)_{i \in \varLambda_{n}} \right) \mapsto \sqrt{\sum_{i \in \varLambda_{n}} {d_{i}\left( a_{i},b_{i} \right)}^{2}}
\end{align*}
明らかに$\forall\left( a_{i} \right)_{i \in \varLambda_{n}},\left( b_{i} \right)_{i \in \varLambda_{n}} \in \prod_{i \in \varLambda_{n}} S_{i}$に対し、$d\left( \left( a_{i} \right)_{i \in \varLambda_{n}},\left( b_{i} \right)_{i \in \varLambda_{n}} \right) \geq 0$が成り立つ。\par
$\forall\left( a_{i} \right)_{i \in \varLambda_{n}},\left( b_{i} \right)_{i \in \varLambda_{n}} \in \prod_{i \in \varLambda_{n}} S_{i}$に対し、$d\left( \left( a_{i} \right)_{i \in \varLambda_{n}},\left( b_{i} \right)_{i \in \varLambda_{n}} \right) = 0$が成り立つならそのときに限り、$\sum_{i \in \varLambda_{n}} {d_{i}\left( a_{i},b_{i} \right)}^{2} = 0$が成り立ち、$\forall i \in \varLambda_{n}$に対し、$0 \leq {d_{i}\left( a_{i},b_{i} \right)}^{2}$が成り立つので、$\forall i \in \varLambda_{n}$に対し、$0 = d_{i}\left( a_{i},b_{i} \right)$が成り立つ。これにより、$a_{i} = b_{i}$が成り立つので、$\left( a_{i} \right)_{i \in \varLambda_{n}} = \left( b_{i} \right)_{i \in \varLambda_{n}}$が成り立つ。\par
また、明らかに$\forall\left( a_{i} \right)_{i \in \varLambda_{n}},\left( b_{i} \right)_{i \in \varLambda_{n}} \in \prod_{i \in \varLambda_{n}} S_{i}$に対し、次式が成り立つ。
\begin{align*}
d\left( \left( a_{i} \right)_{i \in \varLambda_{n}},\left( b_{i} \right)_{i \in \varLambda_{n}} \right) &= \sqrt{\sum_{i \in \varLambda_{n}} {d_{i}\left( a_{i},b_{i} \right)}^{2}}\\
&= \sqrt{\sum_{i \in \varLambda_{n}} {d_{i}\left( b_{i},a_{i} \right)}^{2}}\\
&= d\left( \left( b_{i} \right)_{i \in \varLambda_{n}},\left( a_{i} \right)_{i \in \varLambda_{n}} \right)
\end{align*}\par
最後に、$\forall\left( a_{i} \right)_{i \in \varLambda_{n}},\left( b_{i} \right)_{i \in \varLambda_{n}},\left( c_{i} \right)_{i \in \varLambda_{n}} \in \prod_{i \in \varLambda_{n}} S_{i}$に対し、次のようになることから、
\begin{align*}
&\quad \sum_{i \in \varLambda_{n}} {d_{i}\left( a_{i},b_{i} \right)}^{2}\sum_{i \in \varLambda_{n}} {d_{i}\left( b_{i},c_{i} \right)}^{2} - \left( \sum_{i \in \varLambda_{n}} {d_{i}\left( a_{i},b_{i} \right)d_{i}\left( b_{i},c_{i} \right)} \right)^{2}\\
&= \sum_{i,j \in \varLambda_{n}} {{d_{i}\left( a_{i},b_{i} \right)}^{2}{d_{j}\left( b_{j},c_{j} \right)}^{2}} - \sum_{i \in \varLambda_{n}} {{d_{i}\left( a_{i},b_{i} \right)}^{2}{d_{i}\left( b_{i},c_{i} \right)}^{2}} \\
&\quad - 2\sum_{\scriptsize \begin{matrix}
i,j \in \varLambda_{n} \\
i \neq j \\
\end{matrix}} {d_{i}\left( a_{i},b_{i} \right)d_{i}\left( b_{i},c_{i} \right)d_{j}\left( a_{j},b_{j} \right)d_{j}\left( b_{j},c_{j} \right)}\\
&= \sum_{i \in \varLambda_{n}} {{d_{i}\left( a_{i},b_{i} \right)}^{2}{d_{i}\left( b_{i},c_{i} \right)}^{2}} + \sum_{\scriptsize \begin{matrix}
i,j \in \varLambda_{n} \\
i \neq j \\
\end{matrix}} {{d_{i}\left( a_{i},b_{i} \right)}^{2}{d_{j}\left( b_{j},c_{j} \right)}^{2}} - \sum_{i \in \varLambda_{n}} {{d_{i}\left( a_{i},b_{i} \right)}^{2}{d_{i}\left( b_{i},c_{i} \right)}^{2}} \\
&\quad - 2\sum_{\scriptsize \begin{matrix}
i,j \in \varLambda_{n} \\
i \neq j \\
\end{matrix}} {d_{i}\left( a_{i},b_{i} \right)d_{i}\left( b_{i},c_{i} \right)d_{j}\left( a_{j},b_{j} \right)d_{j}\left( b_{j},c_{j} \right)}\\
&= \sum_{\scriptsize \begin{matrix}
i,j \in \varLambda_{n} \\
i \neq j \\
\end{matrix}} {{d_{i}\left( a_{i},b_{i} \right)}^{2}{d_{j}\left( b_{j},c_{j} \right)}^{2}} - 2\sum_{\scriptsize \begin{matrix}
i,j \in \varLambda_{n} \\
i \neq j \\
\end{matrix}} {d_{i}\left( a_{i},b_{i} \right)d_{i}\left( b_{i},c_{i} \right)d_{j}\left( a_{j},b_{j} \right)d_{j}\left( b_{j},c_{j} \right)}\\
&= \sum_{\scriptsize \begin{matrix}
i,j \in \varLambda_{n} \\
i < j \\
\end{matrix}} {{d_{i}\left( a_{i},b_{i} \right)}^{2}{d_{j}\left( b_{j},c_{j} \right)}^{2}} - 2\sum_{\scriptsize \begin{matrix}
i,j \in \varLambda_{n} \\
i < j \\
\end{matrix}} {d_{i}\left( a_{i},b_{i} \right)d_{i}\left( b_{i},c_{i} \right)d_{j}\left( a_{j},b_{j} \right)d_{j}\left( b_{j},c_{j} \right)} \\
&\quad + \sum_{\scriptsize \begin{matrix}
i,j \in \varLambda_{n} \\
i < j \\
\end{matrix}} {{d_{j}\left( a_{j},b_{j} \right)}^{2}{d_{i}\left( b_{i},c_{i} \right)}^{2}}\\
&= \sum_{\scriptsize \begin{matrix}
i,j \in \varLambda_{n} \\
i < j \\
\end{matrix}} \left( {d_{i}\left( a_{i},b_{i} \right)}^{2}{d_{j}\left( b_{j},c_{j} \right)}^{2} - 2d_{i}\left( a_{i},b_{i} \right)d_{i}\left( b_{i},c_{i} \right)d_{j}\left( a_{j},b_{j} \right)d_{j}\left( b_{j},c_{j} \right) + {d_{j}\left( a_{j},b_{j} \right)}^{2}{d_{i}\left( b_{i},c_{i} \right)}^{2} \right)\\
&= \sum_{\scriptsize \begin{matrix}
i,j \in \varLambda_{n} \\
i < j \\
\end{matrix}} \left( d_{i}\left( a_{i},b_{i} \right)d_{j}\left( b_{j},c_{j} \right) - d_{j}\left( a_{j},b_{j} \right)d_{i}\left( b_{i},c_{i} \right) \right)^{2} \geq 0
\end{align*}
次式が成り立つ。
\begin{align*}
\sqrt{\sum_{i \in \varLambda_{n}} {d_{i}\left( a_{i},b_{i} \right)}^{2}\sum_{i \in \varLambda_{n}} {d_{i}\left( b_{i},c_{i} \right)}^{2}} \geq \sum_{i \in \varLambda_{n}} {d_{i}\left( a_{i},b_{i} \right)d_{i}\left( b_{i},c_{i} \right)}
\end{align*}
これにより、次のようになり、
\begin{align*}
\sum_{i \in \varLambda_{n}} {d_{i}\left( a_{i},c_{i} \right)}^{2} &\leq \sum_{i \in \varLambda_{n}} \left( d_{i}\left( a_{i},b_{i} \right) + d_{i}\left( b_{i},c_{i} \right) \right)^{2}\\
&= \sum_{i \in \varLambda_{n}} \left( {d_{i}\left( a_{i},b_{i} \right)}^{2} + 2d_{i}\left( a_{i},b_{i} \right)d_{i}\left( b_{i},c_{i} \right) + {d_{i}\left( b_{i},c_{i} \right)}^{2} \right)\\
&= \sum_{i \in \varLambda_{n}} {d_{i}\left( a_{i},b_{i} \right)}^{2} + 2\sum_{i \in \varLambda_{n}} {d_{i}\left( a_{i},b_{i} \right)d_{i}\left( b_{i},c_{i} \right)} + \sum_{i \in \varLambda_{n}} {d_{i}\left( b_{i},c_{i} \right)}^{2}\\
&\leq \sum_{i \in \varLambda_{n}} {d_{i}\left( a_{i},b_{i} \right)}^{2} + 2\sqrt{\sum_{i \in \varLambda_{n}} {d_{i}\left( a_{i},b_{i} \right)}^{2}\sum_{i \in \varLambda_{n}} {d_{i}\left( b_{i},c_{i} \right)}^{2}} + \sum_{i \in \varLambda_{n}} {d_{i}\left( b_{i},c_{i} \right)}^{2}\\
&= \left( \sqrt{\sum_{i \in \varLambda_{n}} {d_{i}\left( a_{i},b_{i} \right)}^{2}} + \sqrt{\sum_{i \in \varLambda_{n}} {d_{i}\left( b_{i},c_{i} \right)}^{2}} \right)^{2}
\end{align*}
したがって、次のようになる。
\begin{align*}
d\left( \left( a_{i} \right)_{i \in \varLambda_{n}},\left( c_{i} \right)_{i \in \varLambda_{n}} \right) &= \sqrt{\sum_{i \in \varLambda_{n}} {d_{i}\left( a_{i},c_{i} \right)}^{2}}\\
&\leq \sqrt{\sum_{i \in \varLambda_{n}} {d_{i}\left( a_{i},b_{i} \right)}^{2}} + \sqrt{\sum_{i \in \varLambda_{n}} {d_{i}\left( b_{i},c_{i} \right)}^{2}}\\
&= d\left( \left( a_{i} \right)_{i \in \varLambda_{n}},\left( b_{i} \right)_{i \in \varLambda_{n}} \right) + d\left( \left( b_{i} \right)_{i \in \varLambda_{n}},\left( c_{i} \right)_{i \in \varLambda_{n}} \right)
\end{align*}
以上より、次のことが成り立つので、
\begin{itemize}
\item
  $\forall\left( a_{i} \right)_{i \in \varLambda_{n}},\left( b_{i} \right)_{i \in \varLambda_{n}} \in \prod_{i \in \varLambda_{n}} S_{i}$に対し、$d\left( \left( a_{i} \right)_{i \in \varLambda_{n}},\left( b_{i} \right)_{i \in \varLambda_{n}} \right) \geq 0$が成り立つ。
\item
  $\forall\left( a_{i} \right)_{i \in \varLambda_{n}},\left( b_{i} \right)_{i \in \varLambda_{n}} \in \prod_{i \in \varLambda_{n}} S_{i}$に対し、$d\left( \left( a_{i} \right)_{i \in \varLambda_{n}},\left( b_{i} \right)_{i \in \varLambda_{n}} \right) = 0$が成り立つならそのときに限り、$\left( a_{i} \right)_{i \in \varLambda_{n}} = \left( b_{i} \right)_{i \in \varLambda_{n}}$が成り立つ。
\item
  $\forall\left( a_{i} \right)_{i \in \varLambda_{n}},\left( b_{i} \right)_{i \in \varLambda_{n}} \in \prod_{i \in \varLambda_{n}} S_{i}$に対し、$d\left( \left( a_{i} \right)_{i \in \varLambda_{n}},\left( b_{i} \right)_{i \in \varLambda_{n}} \right) = d\left( \left( b_{i} \right)_{i \in \varLambda_{n}},\left( a_{i} \right)_{i \in \varLambda_{n}} \right)$が成り立つ。
\item
  $\forall\left( a_{i} \right)_{i \in \varLambda_{n}},\left( b_{i} \right)_{i \in \varLambda_{n}},\left( c_{i} \right)_{i \in \varLambda_{n}} \in \prod_{i \in \varLambda_{n}} S_{i}$に対し、$d\left( \left( a_{i} \right)_{i \in \varLambda_{n}},\ \ \left( c_{i} \right)_{i \in \varLambda_{n}} \right) \leq d\left( \left( a_{i} \right)_{i \in \varLambda_{n}},\ \ \left( b_{i} \right)_{i \in \varLambda_{n}} \right) + d\left( \left( b_{i} \right)_{i \in \varLambda_{n}},\ \ \left( c_{i} \right)_{i \in \varLambda_{n}} \right)$が成り立つ。
\end{itemize}
その組$\left( \prod_{i \in \varLambda_{n}} S_{i},d \right)$は距離空間をなす。
\end{proof}
\begin{thm}\label{8.2.1.20}
$\forall\left( a_{i} \right)_{i \in \varLambda_{n}} \in \prod_{i \in \varLambda_{n}} S_{i}$に対し、添数集合$\varLambda_{n}$によって添数づけられた距離空間たちの族$\left\{ \left( S_{i},d_{i} \right) \right\}_{i \in \varLambda_{n}}$の直積距離空間$\left( \prod_{i \in \varLambda_{n}} S_{i},d \right)$における位相空間$\left( \prod_{i \in \varLambda_{n}} S_{i},\mathfrak{O}_{d} \right)$のその元$\left( a_{i} \right)_{i \in \varLambda_{n}}$の基本近傍系として、その元$\left( a_{i} \right)_{i \in \varLambda_{n}}$を中心とする開球体全体の集合が挙げられる。
\end{thm}
\begin{proof}
$\forall\left( a_{i} \right)_{i \in \varLambda_{n}} \in \prod_{i \in \varLambda_{n}} S_{i}$に対し、添数集合$\varLambda_{n}$によって添数づけられた距離空間たちの族$\left\{ \left( S_{i},d_{i} \right) \right\}_{i \in \varLambda_{n}}$の直積距離空間$\left( \prod_{i \in \varLambda_{n}} S_{i},d \right)$における位相空間$\left( \prod_{i \in \varLambda_{n}} S_{i},\mathfrak{O}_{d} \right)$のその元$\left( a_{i} \right)_{i \in \varLambda_{n}}$の基本近傍系について、集合$V$がその元$\left( a_{i} \right)_{i \in \varLambda_{n}}$の近傍であるなら、$\exists\varepsilon \in \mathbb{R}^{+}$に対し、$B\left( \left( a_{i} \right)_{i \in \varLambda_{n}},\varepsilon \right) \subseteq {\mathrm{int}}V \subseteq V$が成り立つので、確かにその元$\left( a_{i} \right)_{i \in \varLambda_{n}}$の基本近傍系として、その元$\left( a_{i} \right)_{i \in \varLambda_{n}}$を中心とする開球体全体の集合が挙げられる。
\end{proof}
\begin{thm}\label{8.2.1.21}
$\forall\left( a_{i} \right)_{i \in \varLambda_{n}} \in \prod_{i \in \varLambda_{n}} S_{i}$に対し、添数集合$\varLambda_{n}$によって添数づけられた距離空間たちの族$\left\{ \left( S_{i},d_{i} \right) \right\}_{i \in \varLambda_{n}}$における位相空間の族$\left\{ \left( S_{i},\mathfrak{O}_{d_{i}} \right) \right\}_{i \in \varLambda_{n}}$の直積位相空間$\left( \prod_{i \in \varLambda_{n}} S_{i},\mathfrak{O} \right)$のその元$\left( a_{i} \right)_{i \in \varLambda_{n}}$の基本近傍系として、その元$a_{i}$の開球体$B\left( a_{i},\varepsilon \right)$を用いた$\prod_{i \in \varLambda_{n}} {B\left( a_{i},\varepsilon \right)}$の形で表される集合全体の集合が挙げられる。
\end{thm}
\begin{proof} 定理\ref{8.1.4.18}と定理\ref{8.2.1.6}より明らかである。
\end{proof}
\begin{thm}\label{8.2.1.22}
添数集合$\varLambda_{n}$によって添数づけられた距離空間たちの族$\left\{ \left( S_{i},d_{i} \right) \right\}_{i \in \varLambda_{n}}$の直積距離空間$\left( \prod_{i \in \varLambda_{n}} S_{i},d \right)$において、$\forall\left( a_{i} \right)_{i \in \varLambda_{n}} \in \prod_{i \in \varLambda_{n}} S_{i}$に対し、次式が成り立つ。
\begin{align*}
\prod_{i \in \varLambda_{n}} {B\left( a_{i},\frac{\varepsilon}{\sqrt{n}} \right)} \subseteq B\left( \left( a_{i} \right)_{i \in \varLambda_{n}},\varepsilon \right) \subseteq \prod_{i \in \varLambda_{n}} {B\left( a_{i},\varepsilon \right)}
\end{align*}
\end{thm}
\begin{proof}
添数集合$\varLambda_{n}$によって添数づけられた距離空間たちの族$\left\{ \left( S_{i},d_{i} \right) \right\}_{i \in \varLambda_{n}}$の直積距離空間$\left( \prod_{i \in \varLambda_{n}} S_{i},d \right)$において、$\forall\left( a_{i} \right)_{i \in \varLambda_{n}} \in \prod_{i \in \varLambda_{n}} S_{i}\forall\left( b_{i} \right)_{i \in \varLambda_{n}} \in \prod_{i \in \varLambda_{n}} {B\left( a_{i},\frac{\varepsilon}{\sqrt{n}} \right)}$に対し、$\left( b_{i} \right)_{i \in \varLambda_{n}} \in \prod_{i \in \varLambda_{n}} {B\left( a_{i},\frac{\varepsilon}{\sqrt{n}} \right)}$が成り立つなら、$\forall i \in \varLambda_{n}$に対し、$b_{i} \in B\left( a_{i},\frac{\varepsilon}{\sqrt{n}} \right)$が成り立ち、したがって、次のようになる。
\begin{align*}
\forall i \in \varLambda_{n}\left[ b_{i} \in B\left( a_{i},\frac{\varepsilon}{\sqrt{n}} \right) \right] &\Leftrightarrow \forall i \in \varLambda_{n}\left[ 0 \leq d_{i}\left( a_{i},b_{i} \right) < \frac{\varepsilon}{\sqrt{n}} \right]\\
&\Leftrightarrow \forall i \in \varLambda_{n}\left[ {d_{i}\left( a_{i},b_{i} \right)}^{2} < \frac{\varepsilon^{2}}{n} \right]\\
&\Rightarrow \sum_{i \in \varLambda_{n}} {d_{i}\left( a_{i},b_{i} \right)}^{2} < \sum_{i \in \varLambda_{n}} \frac{\varepsilon^{2}}{n} = \frac{\varepsilon^{2}}{n} \cdot n = \varepsilon^{2}\\
&\Leftrightarrow \sqrt{\sum_{i \in \varLambda_{n}} {d_{i}\left( a_{i},b_{i} \right)}^{2}} < \varepsilon\\
&\Leftrightarrow d\left( \left( a_{i} \right)_{i \in \varLambda_{n}},\left( b_{i} \right)_{i \in \varLambda_{n}} \right) < \varepsilon\\
&\Leftrightarrow \left( b_{i} \right)_{i \in \varLambda_{n}} \in B\left( \left( a_{i} \right)_{i \in \varLambda_{n}},\varepsilon \right)
\end{align*}
以上より、次式が成り立つことが示された。
\begin{align*}
\prod_{i \in \varLambda_{n}} {B\left( a_{i},\frac{\varepsilon}{\sqrt{n}} \right)} \subseteq B\left( \left( a_{i} \right)_{i \in \varLambda_{n}},\varepsilon \right)
\end{align*}\par
また、$\forall\left( a_{i} \right)_{i \in \varLambda_{n}} \in \prod_{i \in \varLambda_{n}} S_{i}\forall\left( b_{i} \right)_{i \in \varLambda_{n}} \in B\left( \left( a_{i} \right)_{i \in \varLambda_{n}},\varepsilon \right)$に対し、$\left( b_{i} \right)_{i \in \varLambda_{n}} \in B\left( \left( a_{i} \right)_{i \in \varLambda_{n}},\varepsilon \right)$が成り立つなら、次のようになる。
\begin{align*}
\left( b_{i} \right)_{i \in \varLambda_{n}} \in B\left( \left( a_{i} \right)_{i \in \varLambda_{n}},\varepsilon \right) &\Leftrightarrow d\left( \left( a_{i} \right)_{i \in \varLambda_{n}},\left( b_{i} \right)_{i \in \varLambda_{n}} \right) < \varepsilon\\
&\Leftrightarrow \sqrt{\sum_{i \in \varLambda_{n}} {d_{i}\left( a_{i},b_{i} \right)}^{2}} < \varepsilon\\
&\Leftrightarrow \sum_{i \in \varLambda_{n}} {d_{i}\left( a_{i},b_{i} \right)}^{2} < \varepsilon^{2}\\
&\Leftrightarrow \forall i \in \varLambda_{n}\left[ {d_{i}\left( a_{i},b_{i} \right)}^{2} \leq \sum_{i \in \varLambda_{n}} {d_{i}\left( a_{i},b_{i} \right)}^{2} < \varepsilon^{2} \right]\\
&\Rightarrow \forall i \in \varLambda_{n}\left[ {d_{i}\left( a_{i},b_{i} \right)}^{2} < \varepsilon^{2} \right]\\
&\Leftrightarrow \forall i \in \varLambda_{n}\left[ d_{i}\left( a_{i},b_{i} \right) < \varepsilon \right]\\
&\Leftrightarrow \forall i \in \varLambda_{n}\left[ b_{i} \in B\left( a_{i},\varepsilon \right) \right]\\
&\Leftrightarrow \left( b_{i} \right)_{i \in \varLambda_{n}} \in \prod_{i \in \varLambda_{n}} {B\left( a_{i},\varepsilon \right)}
\end{align*}
以上より、次式が成り立つことが示された。
\begin{align*}
B\left( \left( a_{i} \right)_{i \in \varLambda_{n}},\varepsilon \right) \subseteq \prod_{i \in \varLambda_{n}} {B\left( a_{i},\varepsilon \right)}
\end{align*}\par
これにより、$\forall\left( a_{i} \right)_{i \in \varLambda_{n}} \in \prod_{i \in \varLambda_{n}} S_{i}$に対し、次式が成り立つ。
\begin{align*}
\prod_{i \in \varLambda_{n}} {B\left( a_{i},\frac{\varepsilon}{\sqrt{n}} \right)} &\subseteq B\left( \left( a_{i} \right)_{i \in \varLambda_{n}},\varepsilon \right)\\
&\subseteq \prod_{i \in \varLambda_{n}} {B\left( a_{i},\varepsilon \right)}
\end{align*}
\end{proof}
\begin{thm}\label{8.2.1.23}
添数集合$\varLambda_{n}$によって添数づけられた距離空間たちの族$\left\{ \left( S_{i},d_{i} \right) \right\}_{i \in \varLambda_{n}}$の直積距離空間$\left( \prod_{i \in \varLambda_{n}} S_{i},d \right)$における位相空間$\left( \prod_{i \in \varLambda_{n}} S_{i},\mathfrak{O}_{d} \right)$はその族$\left\{ \left( S_{i},d_{i} \right) \right\}_{i \in \varLambda_{n}}$における位相空間の族$\left\{ \left( S_{i},\mathfrak{O}_{d_{i}} \right) \right\}_{i \in \varLambda_{n}}$の直積位相空間$\left( \prod_{i \in \varLambda_{n}} S_{i},\mathfrak{O} \right)$に等しい、即ち、$\mathfrak{O}_{d} = \mathfrak{O}$が成り立つ。
\end{thm}
\begin{proof} 定理\ref{8.2.1.20}より$\forall\left( a_{i} \right)_{i \in \varLambda_{n}} \in \prod_{i \in \varLambda_{n}} S_{i}$に対し、添数集合$\varLambda_{n}$によって添数づけられた距離空間たちの族$\left\{ \left( S_{i},d_{i} \right) \right\}_{i \in \varLambda_{n}}$の直積距離空間$\left( \prod_{i \in \varLambda_{n}} S_{i},d \right)$における位相空間$\left( \prod_{i \in \varLambda_{n}} S_{i},\mathfrak{O}_{d} \right)$のその元$\left( a_{i} \right)_{i \in \varLambda_{n}}$の基本近傍系として、その元$\left( a_{i} \right)_{i \in \varLambda_{n}}$を中心とする開球体全体の集合が挙げられるかつ、定理\ref{8.2.1.21}よりその族$\left\{ \left( S_{i},d_{i} \right) \right\}_{i \in \varLambda_{n}}$における位相空間の族$\left\{ \left( S_{i},\mathfrak{O}_{d_{i}} \right) \right\}_{i \in \varLambda_{n}}$の直積位相空間$\left( \prod_{i \in \varLambda_{n}} S_{i},\mathfrak{O} \right)$のその元$\left( a_{i} \right)_{i \in \varLambda_{n}}$の基本近傍系として、その元$a_{i}$の開球体$B\left( a_{i},\varepsilon \right)$を用いた$\prod_{i \in \varLambda_{n}} {B\left( a_{i},\varepsilon \right)}$の形で表される集合全体の集合が挙げられる。\par
ここで、$\forall O \in \mathfrak{O}_{d}$に対し、定理\ref{8.1.1.23}より$\forall\left( a_{i} \right)_{i \in \varLambda_{n}} \in O$に対し、その集合$O$自身がその元$\left( a_{i} \right)_{i \in \varLambda_{n}}$の近傍である。このとき、基本近傍系の定義より$\exists\varepsilon \in \mathbb{R}^{+}$に対し、$B\left( \left( a_{i} \right)_{i \in \varLambda_{n}},\varepsilon \right) \subseteq O$が成り立ち、定理\ref{8.2.1.22}より$\prod_{i \in \varLambda_{n}} {B\left( a_{i},\frac{\varepsilon}{\sqrt{n}} \right)} \subseteq B\left( \left( a_{i} \right)_{i \in \varLambda_{n}},\varepsilon \right)$が成り立つので、$\exists\varepsilon \in \mathbb{R}^{+}$に対し、$\prod_{i \in \varLambda_{n}} {B\left( a_{i},\frac{\varepsilon}{\sqrt{n}} \right)} \subseteq O$が成り立つ。このとき、定理\ref{8.1.2.16}よりその開集合$O$はその位相空間$\left( \prod_{i \in \varLambda_{n}} S_{i},\mathfrak{O} \right)$での開集合でもあるので、$\mathfrak{O}_{d}\subseteq \mathfrak{O}$が成り立つ。\par
一方で、$\forall O \in \mathfrak{O}$に対し、定理\ref{8.1.1.23}より$\forall\left( a_{i} \right)_{i \in \varLambda_{n}} \in O$に対し、その集合$O$自身がその元$\left( a_{i} \right)_{i \in \varLambda_{n}}$の近傍である。このとき、基本近傍系の定義より$\exists\varepsilon \in \mathbb{R}^{+}$に対し、$\prod_{i \in \varLambda_{n}} {B\left( a_{i},\varepsilon \right)} \subseteq O$が成り立ち、定理\ref{8.2.1.22}より$B\left( \left( a_{i} \right)_{i \in \varLambda_{n}},\varepsilon \right) \subseteq \prod_{i \in \varLambda_{n}} {B\left( a_{i},\varepsilon \right)}$が成り立つので、$\exists\varepsilon \in \mathbb{R}^{+}$に対し、$B\left( \left( a_{i} \right)_{i \in \varLambda_{n}},\varepsilon \right) \subseteq O$が成り立つ。このとき、定理\ref{8.1.2.16}よりその開集合$O$はその位相空間$\left( \prod_{i \in \varLambda_{n}} S_{i},\mathfrak{O}_{d} \right)$での開集合でもあるので、$\mathfrak{O \subseteq}\mathfrak{O}_{d}$が成り立つ。\par
以上より、$\mathfrak{O}_{d} = \mathfrak{O}$が成り立つので、その族$\left\{ \left( S_{i},d_{i} \right) \right\}_{i \in \varLambda_{n}}$の直積距離空間$\left( \prod_{i \in \varLambda_{n}} S_{i},d \right)$における位相空間$\left( \prod_{i \in \varLambda_{n}} S_{i},\mathfrak{O}_{d} \right)$はその族$\left\{ \left( S_{i},d_{i} \right) \right\}_{i \in \varLambda_{n}}$における位相空間の族$\left\{ \left( S_{i},\mathfrak{O}_{d_{i}} \right) \right\}_{i \in \varLambda_{n}}$の直積位相空間$\left( \prod_{i \in \varLambda_{n}} S_{i},\mathfrak{O} \right)$に等しい。
\end{proof}
%\hypertarget{ux76f4ux7a4dux8dddux96e2ux7a7aux9593ux306bux304aux3051ux308bux6975ux9650}{%
\subsubsection{直積距離空間における極限}%\label{ux76f4ux7a4dux8dddux96e2ux7a7aux9593ux306bux304aux3051ux308bux6975ux9650}}
\begin{thm}\label{8.2.1.23}
添数集合$\varLambda_{n}$によって添数づけられた距離空間たちの族$\left\{ \left( S_{i},d_{i} \right) \right\}_{i \in \varLambda_{n}}$の直積距離空間$\left( \prod_{i \in \varLambda_{n}} S_{i},d \right)$が与えられたとき、その集合$\prod_{i \in \varLambda_{n}} S_{i}$の元の列$\left( \left( a_{i,m} \right)_{i \in \varLambda_{n}} \right)_{m \in \mathbb{N}}$の極限が存在してこれがその集合$\prod_{i \in \varLambda_{n}} S_{i}$の元$\left( a_{i} \right)_{i \in \varLambda_{n}}$であるとすれば、$\lim_{m \rightarrow \infty}\left( a_{i,m} \right)_{i \in \varLambda_{n}} = \left( a_{i} \right)_{i \in \varLambda_{n}}$が成り立つならそのときに限り、$\forall i \in \varLambda_{n}$に対し、$\lim_{m \rightarrow \infty}a_{i,m} = a_{i}$が成り立つ。
\end{thm}
\begin{proof}
添数集合$\varLambda_{n}$によって添数づけられた距離空間たちの族$\left\{ \left( S_{i},d_{i} \right) \right\}_{i \in \varLambda_{n}}$の直積距離空間$\left( \prod_{i \in \varLambda_{n}} S_{i},d \right)$が与えられたとき、その集合$\prod_{i \in \varLambda_{n}} S_{i}$の元の列$\left( \left( a_{i,m} \right)_{i \in \varLambda_{n}} \right)_{m \in \mathbb{N}}$の極限が存在してこれがその集合$\prod_{i \in \varLambda_{n}} S_{i}$の元$\left( a_{i} \right)_{i \in \varLambda_{n}}$であるとすれば、$\forall\varepsilon \in \mathbb{R}^{+}\exists m_{0} \in \mathbb{N}\forall m \in \mathbb{N}$に対し、$m_{0} < m$が成り立つなら、$d\left( \left( a_{i,m} \right)_{i \in \varLambda_{n}},\left( a_{i} \right)_{i \in \varLambda_{n}} \right) < \varepsilon \leq \varepsilon\sqrt{n}$が成り立つことになる。ここで、$\forall i \in \varLambda_{n}$に対し、定理\ref{8.2.1.22}より次のようになるかつ、
\begin{align*}
d\left( \left( a_{i,m} \right)_{i \in \varLambda_{n}},\left( a_{i} \right)_{i \in \varLambda_{n}} \right) < \varepsilon &\Leftrightarrow \left( a_{i} \right)_{i \in \varLambda_{n}} \in B\left( \left( a_{i,m} \right)_{i \in \varLambda_{n}},\varepsilon \right)\\
&\Rightarrow \left( a_{i} \right)_{i \in \varLambda_{n}} \in \prod_{i \in \varLambda_{n}} {B\left( a_{i,m},\varepsilon \right)}\\
&\Leftrightarrow \forall i \in \varLambda_{n}\left[ a_{i} \in B\left( a_{i,m},\varepsilon \right) \right]\\
&\Leftrightarrow \forall i \in \varLambda_{n}\left[ d_{i}\left( a_{i,m},a_{i} \right) < \varepsilon \right]
\end{align*}
\ref{8.2.1.22}より次のようになるので、
\begin{align*}
\forall i \in \varLambda_{n}\left[ d_{i}\left( a_{i,m},a_{i} \right) < \varepsilon \right] &\Leftrightarrow \forall i \in \varLambda_{n}\left[ a_{i} \in B\left( a_{i,m},\varepsilon \right) \right]\\
&\Leftrightarrow \left( a_{i} \right)_{i \in \varLambda_{n}} \in \prod_{i \in \varLambda_{n}} {B\left( a_{i,m},\varepsilon \right)}\\
&\Rightarrow \left( a_{i} \right)_{i \in \varLambda_{n}} \in B\left( \left( a_{i,m} \right)_{i \in \varLambda_{n}},\varepsilon\sqrt{n} \right)\\
&\Leftrightarrow d\left( \left( a_{i,m} \right)_{i \in \varLambda_{n}},\left( a_{i} \right)_{i \in \varLambda_{n}} \right) < \varepsilon\sqrt{n}
\end{align*}
その正の実数$\varepsilon$の任意性よりその元の列$\left( \left( a_{im} \right)_{i \in \varLambda_{n}} \right)_{m \in \mathbb{N}}$の極限が存在してこれがその集合$\prod_{i \in \varLambda_{n}} S_{i}$の元$\left( a_{i} \right)_{i \in \varLambda_{n}}$であるならそのときに限り、$\forall\varepsilon \in \mathbb{R}^{+}\exists m_{0} \in \mathbb{N}\forall m \in \mathbb{N}$に対し、$m_{0} < m$が成り立つなら、$\forall i \in \varLambda_{n}$に対し、$d_{i}\left( a_{i,m},a_{i} \right) < \varepsilon$が成り立ち、特に、$d_{i}\left( a_{i,m},a_{i} \right) < \varepsilon$が成り立つ。したがって、これが成り立つならそのときに限り、$\forall i \in \varLambda_{n}$に対し、$\lim_{m \rightarrow \infty}a_{i,m} = a_{i}$が成り立つ。
\end{proof}
\begin{thm}\label{8.2.1.24}
距離空間$(S,d)$が与えられたとき、その集合$S$の元の列たち$\left( a_{n} \right)_{n \in \mathbb{N}}$、$\left( b_{n} \right)_{n \in \mathbb{N}}$の極限が存在するとすれば、次式が成り立つ。
\begin{align*}
\lim_{n \rightarrow \infty}{d\left( a_{n},b_{n} \right)} = d\left( \lim_{n \rightarrow \infty}a_{n},\lim_{n \rightarrow \infty}b_{n} \right)
\end{align*}
\end{thm}
\begin{proof}
距離空間$(S,d)$が与えられたとき、その集合$S$の元の列たち$\left( a_{n} \right)_{n \in \mathbb{N}}$、$\left( b_{n} \right)_{n \in \mathbb{N}}$の極限がそれぞれ$a$、$b$と存在するとすれば、$\forall\varepsilon \in \mathbb{R}^{+}\exists n_{0} \in \mathbb{N}\forall n \in \mathbb{N}$に対し、$n_{0} < n$が成り立つなら、$d\left( a_{n},a \right) < \varepsilon$が成り立つかつ、$d\left( b_{n},b \right) < \varepsilon$が成り立つので、次のようになる。
\begin{align*}
- d\left( a_{n},a \right) - d\left( b_{n},b \right) &= - d\left( a_{n},a \right) - d\left( b_{n},b \right) + d(a,b) - d(a,b)\\
&\leq - d\left( a_{n},a \right) - d\left( b_{n},b \right) + d\left( a,b_{n} \right) + d\left( b_{n},b \right) - d(a,b)\\
&= - d\left( a_{n},a \right) + d\left( a,b_{n} \right) - d(a,b)\\
&\leq - d\left( a_{n},a \right) + d\left( a_{n},b_{n} \right) + d\left( a_{n},a \right) - d(a,b)\\
&= d\left( a_{n},b_{n} \right) - d(a,b)\\
&\leq d\left( a_{n},a \right) + d\left( a,b_{n} \right) - d(a,b)\\
&\leq d\left( a_{n},a \right) + d\left( b_{n},b \right) + d(a,b) - d(a,b)\\
&\leq d\left( a_{n},a \right) + d\left( b_{n},b \right)
\end{align*}
したがって、次式が成り立つ。
\begin{align*}
\left| d\left( a_{n},b_{n} \right) - d\left( \lim_{n \rightarrow \infty}a_{n},\lim_{n \rightarrow \infty}b_{n} \right) \right| &= \left| d\left( a_{n},b_{n} \right) - d(a,b) \right|\\
&\leq d\left( a_{n},a \right) + d\left( b_{n},b \right) < 2\varepsilon
\end{align*}
以上より、次式が成り立つ。
\begin{align*}
\lim_{n \rightarrow \infty}{d\left( a_{n},b_{n} \right)} = d\left( \lim_{n \rightarrow \infty}a_{n},\lim_{n \rightarrow \infty}b_{n} \right)
\end{align*}
\end{proof}
\begin{thebibliography}{50}
\bibitem{1}
  松坂和夫, 集合・位相入門, 岩波書店, 1968. 新装版第2刷 p234-247 ISBN978-4-00-029871-1
\end{thebibliography}
\end{document}
