\documentclass[dvipdfmx]{jsarticle}
\setcounter{section}{2}
\setcounter{subsection}{9}
\usepackage{xr}
\externaldocument{1.2.4}
\usepackage{amsmath,amsfonts,amssymb,array,comment,mathtools,url,docmute}
\usepackage{longtable,booktabs,dcolumn,tabularx,mathtools,multirow,colortbl,xcolor}
\usepackage[dvipdfmx]{graphics}
\usepackage{bmpsize}
\usepackage{amsthm}
\usepackage{enumitem}
\setlistdepth{20}
\renewlist{itemize}{itemize}{20}
\setlist[itemize]{label=•}
\renewlist{enumerate}{enumerate}{20}
\setlist[enumerate]{label=\arabic*.}
\setcounter{MaxMatrixCols}{20}
\setcounter{tocdepth}{3}
\newcommand{\rotin}{\text{\rotatebox[origin=c]{90}{$\in $}}}
\newcommand{\amap}[6]{\text{\raisebox{-0.7cm}{\begin{tikzpicture} 
  \node (a) at (0, 1) {$\textstyle{#2}$};
  \node (b) at (#6, 1) {$\textstyle{#3}$};
  \node (c) at (0, 0) {$\textstyle{#4}$};
  \node (d) at (#6, 0) {$\textstyle{#5}$};
  \node (x) at (0, 0.5) {$\rotin $};
  \node (x) at (#6, 0.5) {$\rotin $};
  \draw[->] (a) to node[xshift=0pt, yshift=7pt] {$\textstyle{\scriptstyle{#1}}$} (b);
  \draw[|->] (c) to node[xshift=0pt, yshift=7pt] {$\textstyle{\scriptstyle{#1}}$} (d);
\end{tikzpicture}}}}
\newcommand{\twomaps}[9]{\text{\raisebox{-0.7cm}{\begin{tikzpicture} 
  \node (a) at (0, 1) {$\textstyle{#3}$};
  \node (b) at (#9, 1) {$\textstyle{#4}$};
  \node (c) at (#9+#9, 1) {$\textstyle{#5}$};
  \node (d) at (0, 0) {$\textstyle{#6}$};
  \node (e) at (#9, 0) {$\textstyle{#7}$};
  \node (f) at (#9+#9, 0) {$\textstyle{#8}$};
  \node (x) at (0, 0.5) {$\rotin $};
  \node (x) at (#9, 0.5) {$\rotin $};
  \node (x) at (#9+#9, 0.5) {$\rotin $};
  \draw[->] (a) to node[xshift=0pt, yshift=7pt] {$\textstyle{\scriptstyle{#1}}$} (b);
  \draw[|->] (d) to node[xshift=0pt, yshift=7pt] {$\textstyle{\scriptstyle{#2}}$} (e);
  \draw[->] (b) to node[xshift=0pt, yshift=7pt] {$\textstyle{\scriptstyle{#1}}$} (c);
  \draw[|->] (e) to node[xshift=0pt, yshift=7pt] {$\textstyle{\scriptstyle{#2}}$} (f);
\end{tikzpicture}}}}
\renewcommand{\thesection}{第\arabic{section}部}
\renewcommand{\thesubsection}{\arabic{section}.\arabic{subsection}}
\renewcommand{\thesubsubsection}{\arabic{section}.\arabic{subsection}.\arabic{subsubsection}}
\everymath{\displaystyle}
\allowdisplaybreaks[4]
\usepackage{vtable}
\theoremstyle{definition}
\newtheorem{thm}{定理}[subsection]
\newtheorem*{thm*}{定理}
\newtheorem{dfn}{定義}[subsection]
\newtheorem*{dfn*}{定義}
\newtheorem{axs}[dfn]{公理}
\newtheorem*{axs*}{公理}
\renewcommand{\headfont}{\bfseries}
\makeatletter
  \renewcommand{\section}{%
    \@startsection{section}{1}{\z@}%
    {\Cvs}{\Cvs}%
    {\normalfont\huge\headfont\raggedright}}
\makeatother
\makeatletter
  \renewcommand{\subsection}{%
    \@startsection{subsection}{2}{\z@}%
    {0.5\Cvs}{0.5\Cvs}%
    {\normalfont\LARGE\headfont\raggedright}}
\makeatother
\makeatletter
  \renewcommand{\subsubsection}{%
    \@startsection{subsubsection}{3}{\z@}%
    {0.4\Cvs}{0.4\Cvs}%
    {\normalfont\Large\headfont\raggedright}}
\makeatother
\makeatletter
\renewenvironment{proof}[1][\proofname]{\par
  \pushQED{\qed}%
  \normalfont \topsep6\p@\@plus6\p@\relax
  \trivlist
  \item\relax
  {
  #1\@addpunct{.}}\hspace\labelsep\ignorespaces
}{%
  \popQED\endtrivlist\@endpefalse
}
\makeatother
\renewcommand{\proofname}{\textbf{証明}}
\usepackage{tikz,graphics}
\usepackage[dvipdfmx]{hyperref}
\usepackage{pxjahyper}
\hypersetup{
 setpagesize=false,
 bookmarks=true,
 bookmarksdepth=tocdepth,
 bookmarksnumbered=true,
 colorlinks=false,
 pdftitle={},
 pdfsubject={},
 pdfauthor={},
 pdfkeywords={}}
\begin{document}
%\hypertarget{ux6fc3ux5ea6ux306eux6f14ux7b97}{%
\subsection{濃度の演算}%\label{ux6fc3ux5ea6ux306eux6f14ux7b97}}
%\hypertarget{ux6fc3ux5ea6ux306eux6f14ux7b97-1}{%
\subsubsection{濃度の演算}%\label{ux6fc3ux5ea6ux306eux6f14ux7b97-1}}
\begin{dfn}
任意の$A \cap B = \emptyset $なる2つの集合たち$A$、$B$について、次式のように定義する。
\begin{align*}
\# A + \# B = \# (A \sqcup B)
\end{align*}
\end{dfn}
\begin{thm}\label{1.2.8.1}
この定義は集合によらない、即ち、$\# A = \# A'$かつ$\# B = \# B'$となるような集合たち$A'$、$B'$が存在するなら、$\# A + \# B = \# A' + \# B'$が成り立つ。
\end{thm}
\begin{proof}
任意の$A \cap B = \emptyset $なる2つの集合たち$A$、$B$について、次式のように定義する。
\begin{align*}
\# A + \# B = \# (A \sqcup B)
\end{align*}
さらに、$A \sim A'$かつ$B \sim B'$かつ$A' \cap B' = \emptyset $なる集合たち$A'$、$B'$を考えると、全単射たち$f_{A}$、$f_{B}$がそれぞれ集合$\mathfrak{F}\left( A,A' \right)$、$\mathfrak{F}\left( B,B' \right)$に存在するのであったので、次式のように写像$g$が定義されると、
\begin{align*}
g:A \sqcup B \rightarrow A' \sqcup B';a \mapsto \left\{ \begin{matrix}
f_{A}(a) & {\mathrm{if}} & a \in A \\
f_{B}(a) & {\mathrm{if}} & a \in B \\
\end{matrix} \right.\ 
\end{align*}
それらの写像たち$f_{A}$、$f_{B}$が全単射であったので、$V\left( f_{A} \right) = A'$かつ$V\left( f_{B} \right) = B'$が成り立つかつ、逆写像たち$f_{A}^{- 1}$、$f_{B}^{- 1}$が存在するかつ、$A' \cap B' = \emptyset $が成り立つことに注意すれば、次式のように逆写像$g^{- 1}$が存在するので、
\begin{align*}
g^{- 1}:A' \sqcup B' \rightarrow A \sqcup B;b \mapsto \left\{ \begin{matrix}
f_{A}^{- 1}(b) & {\mathrm{if}} & b \in A' \\
f_{B}^{- 1}(b) & {\mathrm{if}} & b \in B' \\
\end{matrix} \right.\ 
\end{align*}
その写像$g$は全単射となる。したがって、$A \sqcup B \sim A' \sqcup B'$が成り立つので、次式が成り立つ。
\begin{align*}
\# (A \sqcup B) = \# \left( A' \sqcup B' \right)
\end{align*}
\end{proof}
\begin{thm}\label{1.2.8.2}
任意の濃度たち$\mathfrak{m}$、$\mathfrak{n}$に対し、次式のような集合たち$A$、$B$が存在する。
\begin{align*}
\mathfrak{m} =\# A,\ \ \mathfrak{n} =\# B,\ \ A \cap B = \emptyset 
\end{align*}
\end{thm}
\begin{proof}
任意の濃度たち$\mathfrak{m}$、$\mathfrak{n}$に対し、定義より明らかに次式のような集合たち$A$、$B$が存在する。
\begin{align*}
\mathfrak{m} =\# A,\ \ \mathfrak{n} =\# B
\end{align*}
ここで、$A \cap B = \emptyset $が成り立つなら、示すべきことは示されている。\par
$A \cap B \neq \emptyset $が成り立つなら、次式のような写像たち$f_{A}$、$f_{B}$が考えられれば、
\begin{align*}
f_{A}&:A \rightarrow A \times \left\{ 0 \right\};a \mapsto (a,0)\\
f_{B}&:B \rightarrow B \times \left\{ 1 \right\};b \mapsto (b,1)
\end{align*}
これらの写像たち$f_{A}$、$f_{B}$は明らかに全単射となるので、$A \sim A \times \left\{ 0 \right\}$かつ$B \sim B \times \left\{ 1 \right\}$が成り立つかつ、$0 \neq 1$より$\left( A \times \left\{ 0 \right\} \right) \cap \left( B \times \left\{ 1 \right\} \right) = \emptyset $が成り立つことになるので、次式が成り立ち、
\begin{align*}
\# \left( A \times \left\{ 0 \right\} \right) = \# A = \mathfrak{m},\ \ \# \left( B \times \left\{ 1 \right\} \right) = \# B = \mathfrak{n,\ \ }\left( A \times \left\{ 0 \right\} \right) \cap \left( B \times \left\{ 1 \right\} \right) = \emptyset 
\end{align*}
よって、示すべきことは示された。
\end{proof}
\begin{thm}\label{1.2.8.3}
次のことが成り立つ\footnote{これにより濃度は可換群のように計算されることができる。}。
\begin{itemize}
\item
  $\forall\# A,\# B,\# C \in \mathcal{F} /\sim $に対し、次式が成り立つ。
\begin{align*}
\left( \# A + \# B \right) + \# C = \# A + \left( \# B + \# C \right)
\end{align*}
\item
  $\forall\# A \in \mathcal{F} /\sim $に対し、次式が成り立つ。
\begin{align*}
\# A + \# \emptyset  = \# A
\end{align*}
\item
  $\forall\# A,\# B \in \mathcal{F} /\sim $に対し、次式が成り立つ。
\begin{align*}
\# A + \# B = \# B + \# A
\end{align*}
\item
  $\forall\# A,\# B,\# C,\# D \in \mathcal{F} /\sim $に対し、$\# A \leq \# B$かつ$\# C \leq \# D$が成り立つなら、次式が成り立つ。
\begin{align*}
\# A + \# C \leq \# B + \# D
\end{align*}
\end{itemize}
\end{thm}
\begin{proof}
$\forall\# A,\# B,\# C \in \mathcal{F} /\sim $に対し、次のようになる。
\begin{align*}
\left( \# A + \# B \right) + \# C &= \# (A \sqcup B) + \# C\\
&= \# \left( (A \sqcup B) \sqcup C \right)\\
&= \# \left( A \sqcup (B \sqcup C) \right)\\
&= \# A + \# (B \sqcup C)\\
&= \# A + \left( \# B + \# C \right)
\end{align*}\par
$\forall\# A \in \mathcal{F} /\sim $に対し、次のようになる。
\begin{align*}
\# A + \# \emptyset  = \# (A \sqcup \emptyset ) = \# A
\end{align*}\par
$\forall\# A,\# B \in \mathcal{F} /\sim $に対し、次のようになる。
\begin{align*}
\# A + \# B &= \# (A \sqcup B)\\
&= \# (B \sqcup A)\\
&= \# B + \# A
\end{align*}\par
$\forall\# A,\# B,\# C,\# D \in \mathcal{F} /\sim $に対し、$\# A \leq \# B$かつ$\# C \leq \# D$が成り立つなら、次式のように単射な写像たち$f_{A}$、$f_{B}$が存在する。
\begin{align*}
f_{A}:A \rightarrowtail B,\ \ f_{B}:C \rightarrowtail D
\end{align*}
ここで、次のような写像たち$f_{A}'$、$f_{B}'$、$f_{A}''$、$f_{B}''$が定義されると、
\begin{align*}
f_{A}':A \rightarrow A \times \left\{ 0 \right\};a \mapsto (a,0),\ \ f_{B}':C \rightarrow C \times \left\{ 1 \right\};b \mapsto (b,1),\\
f_{A}'':B \rightarrow B \times \left\{ 0 \right\};a \mapsto (a,0),\ \ f_{B}'':D \rightarrow D \times \left\{ 1 \right\};b \mapsto (b,1)
\end{align*}
明らかにこれらの写像たち$f_{A}'$、$f_{B}'$、$f_{A}''$、$f_{B}''$は全単射となるので、次式たちが成り立つ。
\begin{align*}
\# A = \# \left( A \times \left\{ 0 \right\} \right),\ \ \# C = \# \left( C \times \left\{ 1 \right\} \right),\\
\# B = \# \left( B \times \left\{ 0 \right\} \right),\ \ \# D = \# \left( D \times \left\{ 1 \right\} \right)
\end{align*}
さらに、$\left( A \times \left\{ 0 \right\} \right) \cap \left( C \times \left\{ 1 \right\} \right) = \left( B \times \left\{ 0 \right\} \right) \cap \left( D \times \left\{ 1 \right\} \right) = \emptyset $が成り立つので、次式たちが成り立つ。
\begin{align*}
\# A + \# C &= \# \left( A \times \left\{ 0 \right\} \right) + \# \left( C \times \left\{ 1 \right\} \right)\\
&= \# \left( \left( A \times \left\{ 0 \right\} \right) \sqcup \left( C \times \left\{ 1 \right\} \right) \right)\\
\# B + \# D &= \# \left( B \times \left\{ 0 \right\} \right) + \# \left( D \times \left\{ 1 \right\} \right)\\
&= \# \left( \left( B \times \left\{ 0 \right\} \right) \sqcup \left( D \times \left\{ 1 \right\} \right) \right)
\end{align*}
ここで、次式のように写像$g$が定義されると、
\begin{align*}
g:\left( A \times \left\{ 0 \right\} \right) \sqcup \left( C \times \left\{ 1 \right\} \right) \rightarrow \left( B \times \left\{ 0 \right\} \right) \sqcup \left( D \times \left\{ 1 \right\} \right);(a,b) \mapsto \left\{ \begin{matrix}
\left( f_{A}(a),0 \right) & {\mathrm{if}} & (a,b) \in A \times \left\{ 0 \right\} \\
\left( f_{B}(a),1 \right) & {\mathrm{if}} & (a,b) \in C \times \left\{ 1 \right\} \\
\end{matrix} \right.\ 
\end{align*}
これらの写像たち$f_{A}$、$f_{B}$は単射であったので、明らかにこの写像$g$も単射となる。\par
したがって、次式が成り立つ。
\begin{align*}
\# \left( \left( A \times \left\{ 0 \right\} \right) \sqcup \left( C \times \left\{ 1 \right\} \right) \right) \leq \# \left( \left( B \times \left\{ 0 \right\} \right) \sqcup \left( D \times \left\{ 1 \right\} \right) \right)
\end{align*}
よって、次式が成り立つ。
\begin{align*}
\# A + \# C \leq \# B + \# D
\end{align*}
\end{proof}
\begin{dfn}
任意の2つの集合たち$A$、$B$について、次式のように定義する。
\begin{align*}
\# A \cdot \# B = \# (A \times B)
\end{align*}
\end{dfn}
\begin{thm}\label{1.2.8.4}
この定義は集合によらない、即ち、$\# A = \# A'$かつ$\# B = \# B'$となるような集合たち$A'$、$B'$が存在するなら、$\# A \cdot \# B = \# A' \cdot \# B'$が成り立つ。
\end{thm}
\begin{proof} 任意の2つの集合たち$A$、$B$について、次式のように定義する。
\begin{align*}
\# A \cdot \# B = \# (A \times B)
\end{align*}
さらに、$A \sim A'$かつ$B \sim B'$かつ$A' \cap B' = \emptyset $なる集合たち$A'$、$B'$を考えると、全単射たち$f_{A}$、$f_{B}$がそれぞれ集合$\mathfrak{F}\left( A,A' \right)$、$\mathfrak{F}\left( B,B' \right)$に存在するのであったので、次式のように写像$g$が定義されると、
\begin{align*}
g:A \times B \rightarrow A' \times B';(a,b) \mapsto \left( f_{A}(a),f_{B}(b) \right)
\end{align*}
次式のようにこれの逆写像$g^{- 1}$が存在するので、
\begin{align*}
g^{- 1}:A' \times B' \rightarrow A \times B;\left( a',b' \right) \mapsto \left( f_{A}^{- 1}\left( a' \right),f_{B}^{- 1}\left( b' \right) \right)
\end{align*}
その写像$g$は全単射となる。したがって、$A \times B \sim A' \times B'$が成り立ち、したがって、次のようになる。
\begin{align*}
\# A \cdot \# B &= \# (A \times B)\\
&= \# \left( A' \times B' \right)\\
&= \# A' \cdot \# B'
\end{align*}
\end{proof}
\begin{thm}\label{1.2.8.5}
次のことが成り立つ。
\begin{itemize}
\item
  $\forall\# A,\# B,\# C \in \mathcal{F} /\sim $に対し、次式が成り立つ。
\begin{align*}
\left( \# A \cdot \# B \right) \cdot \# C = \# A \cdot \left( \# B \cdot \# C \right)
\end{align*}
\item
  $\forall\# A \in \mathcal{F} /\sim $に対し、次式が成り立つ。
\begin{align*}
\# A \cdot \# \emptyset  = \# \emptyset 
\end{align*}
\item
  $\forall\# A \in \mathcal{F} /\sim $に対し、次式が成り立つ。
\begin{align*}
\# A \cdot \# \left\{ 1 \right\} = \# A
\end{align*}
\item
  $\forall\# A,\# B \in \mathcal{F} /\sim $に対し、次式が成り立つ。
\begin{align*}
\# A \cdot \# B = \# B \cdot \# A
\end{align*}
\item
  $\forall\# A,\# B,\# C,\# D \in \mathcal{F} /\sim $に対し、$\# A \leq \# B$かつ$\# C \leq \# D$が成り立つなら、次式が成り立つ。
\begin{align*}
\# A \cdot \# C \leq \# B \cdot \# D
\end{align*}
\item
  $\forall\# A,\# B,\# C \in \mathcal{F} /\sim $に対し、次式が成り立つ。
\begin{align*}
\# A \cdot \left( \# B + \# C \right) = \left( \# A \cdot \# B \right) + \left( \# A \cdot \# C \right)
\end{align*}
\end{itemize}
\end{thm}
\begin{proof}
$\forall\# A,\# B,\# C \in \mathcal{F} /\sim $に対し、次のようになる。
\begin{align*}
\left( \# A \cdot \# B \right) \cdot \# C &= \# (A \times B) \cdot \# C\\
&= \# \left( (A \times B) \times C \right)
\end{align*}
ここで、次式のような写像$f$が考えられれば、
\begin{align*}
f:(A \times B) \times C \rightarrow A \times (B \times C);\left( (a,b),c \right) \mapsto \left( a,(b,c) \right)
\end{align*}
これは明らかに全単射となるので、次のようになる。
\begin{align*}
\left( \# A \cdot \# B \right) \cdot \# C &= \# \left( (A \times B) \times C \right)\\
&= \# \left( A \times (B \times C) \right)\\
&= \# A \cdot \# (B \times C)\\
&= \# A \cdot \left( \# B \cdot \# C \right)
\end{align*}\par
$\forall\# A \in \mathcal{F} /\sim $に対し、次のようになる。
\begin{align*}
\# A \cdot \# \emptyset  = \# (A \times \emptyset ) = \# \emptyset 
\end{align*}\par
$\forall\# A \in \mathcal{F} /\sim $に対し、次式のような写像$f$が考えられれば、
\begin{align*}
f:A \rightarrow A \times \left\{ 1 \right\};a \mapsto (a,1)
\end{align*}
これは明らかに全単射となるので、次のようになる。
\begin{align*}
\# A \cdot \# \left\{ 1 \right\} = \# \left( A \times \left\{ 1 \right\} \right) = \# A
\end{align*}\par
$\forall\# A,\# B \in \mathcal{F} /\sim $に対し、次式のような写像$f$が考えられれば、
\begin{align*}
f:A \times B \rightarrow B \times A;(a,b) \mapsto (b,a)
\end{align*}
これは明らかに全単射となるので、次のようになる。
\begin{align*}
\# A \cdot \# B &= \# (A \times B)\\
&= \# (B \times A)\\
&= \# B \cdot \# A
\end{align*}\par
$\forall\# A,\# B,\# C,\# D \in \mathcal{F} /\sim $に対し、$\# A \leq \# B$かつ$\# C \leq \# D$が成り立つなら、次式のように単射な写像たち$f_{A}$、$f_{B}$が存在する。
\begin{align*}
f_{A}:A \rightarrowtail B,\ \ f_{B}:C \rightarrowtail D
\end{align*}
ここで、次式のように写像$g$が定義されると、
\begin{align*}
g:A \times C \rightarrow B \times D;(a,b) \mapsto \left( f_{A}(a),f_{B}(b) \right)
\end{align*}
これらの写像たち$f_{A}$、$f_{B}$は単射であったので、明らかにこの写像$g$も単射となる。したがって、次式が成り立つ。
\begin{align*}
\# (A \times C) \leq \# (B \times D)
\end{align*}
よって、次式が成り立つ。
\begin{align*}
\# A \cdot \# C \leq \# B \cdot \# D
\end{align*}\par
$\forall\# A,\# B,\# C \in \mathcal{F} /\sim $に対し、$A \times (B \sqcup C) = (A \times B) \sqcup (A \times C)$が成り立つことに注意すれば、次のようになる。
\begin{align*}
\# A \cdot \left( \# B + \# C \right) &= \# A \cdot \# (B \sqcup C)\\
&= \# \left( A \times (B \sqcup C) \right)\\
&= \# \left( (A \times B) \sqcup (A \times C) \right)\\
&= \# (A \times B) + \# (A \times C)\\
&= \# A \cdot \# B + \# A \cdot \# C
\end{align*}
\end{proof}
\begin{thm}\label{1.2.8.6}
添数集合$\varLambda$によって添数づけられた集合族$\left( A_{\lambda} \right)_{\lambda \in \varLambda}$が与えられ、$\forall\lambda \in \varLambda$に対し、$\# A_{\lambda} = \mathfrak{n}$が成り立つとき、次式が成り立つ。
\begin{align*}
\# {\bigsqcup_{\lambda \in \varLambda} A_{\lambda}} = \# \varLambda\mathfrak{\cdot n}
\end{align*}
\end{thm}
\begin{proof}
添数集合$\varLambda$によって添数づけられた集合族$\left( A_{\lambda} \right)_{\lambda \in \varLambda}$が与えられ、$\forall\lambda \in \varLambda$に対し、$\# A_{\lambda} = \mathfrak{n}$が成り立つとき、$\lambda' \in \varLambda$なる1つの添数$\lambda'$を用いると、$\forall\lambda \in \varLambda$に対し、次式のような全単射な写像$f_{\lambda}$が存在する。
\begin{align*}
f_{\lambda}:A_{\lambda'}\overset{\sim}{\rightarrow}A_{\lambda}
\end{align*}
ここで、次式のような写像$f$が考えられる。
\begin{align*}
f:A_{\lambda'} \times \varLambda \rightarrow \bigsqcup_{\lambda \in \varLambda} A_{\lambda};(a,\lambda) \mapsto f_{\lambda}(a)
\end{align*}
$\forall b \in \bigsqcup_{\lambda \in \varLambda} A_{\lambda}$に対し、$\exists\lambda \in \varLambda\left[ b \in A_{\lambda} \right]$が成り立つので、$b \in A_{\lambda}$なる添数$\lambda$が存在する。さらに、その写像$f_{\lambda}$は全射であったので、$f_{\lambda}(a) = b$なる元$a$がその集合$A_{\lambda'}$に存在する。したがって、$f(a,\lambda) = f_{\lambda}(a) = b$なる順序づけられた組$(a,\lambda)$がその集合$A_{\lambda'} \times \varLambda$に存在することになるので、その写像$f$は全射となることが示された。\par
$\forall(a,\lambda),(b,\mu) \in A_{\lambda'} \times \varLambda$に対し、$(a,\lambda) \neq (b,\mu)$が成り立つなら、$a \neq b$または$\lambda \neq \mu$が成り立つ。ここで、$\lambda \neq \mu$が成り立つとき、$A_{\lambda} \cap A_{\mu} = \emptyset $が成り立つかつ、$f_{\lambda}(a) \in A_{\lambda}$かつ$f_{\mu}(b) \in A_{\mu}$が成り立つので、$f_{\lambda}(a) = f_{\mu}(b)$が成り立たない。したがって、$f(a,\lambda) \neq f(b,\mu)$が成り立つ。一方で、$\lambda = \mu$かつ$a \neq b$のとき、それらの写像たち$f_{\lambda}$、$f_{\mu}$は単射であったので、$f_{\lambda}(a) \neq f_{\mu}(b)$が成り立ち、したがって、$f(a,\lambda) \neq f(b,\mu)$が成り立つ。以上より、その写像$f$は単射であることが示された。\par
したがって、その写像$f$は全単射となるので、次式が成り立つ。
\begin{align*}
\# \left( A_{\lambda'} \times \varLambda \right) = \# {\bigsqcup_{\lambda \in \varLambda} A_{\lambda}}
\end{align*}
これにより、次のようになる。
\begin{align*}
\# {\bigsqcup_{\lambda \in \varLambda} A_{\lambda}} &= \# \left( A_{\lambda'} \times \varLambda \right)\\
&= \# A_{\lambda'} \cdot \# \varLambda\\
&= \# \varLambda \cdot \# A_{\lambda'}\\
&= \# \varLambda\mathfrak{\cdot n}
\end{align*}
\end{proof}
\begin{dfn}
任意の2つの集合たち$A$、$B$について、次式のように定義する。
\begin{align*}
{\# B}^{\# A} = \# {\mathfrak{F}(A,B)}
\end{align*}
\end{dfn}
\begin{thm}\label{1.2.8.7}
このとき、この定義は集合によらない、即ち、$\# A = \# A'$かつ$\# B = \# B'$となるような集合たち$A'$、$B'$が存在するなら、${\# B}^{\# A} = {\# B'}^{\# A'}$が成り立つ。
\end{thm}
\begin{proof} 任意の2つの集合たち$A$、$B$について、次式のように定義する。
\begin{align*}
{\# B}^{\# A} = \# {\mathfrak{F}(A,B)}
\end{align*}
さらに、$A \sim A'$かつ$B \sim B'$かつ$A' \cap B' = \emptyset $なる集合たち$A'$、$B'$を考えると、全単射たち$f_{A}$、$f_{B}$がそれぞれ集合$\mathfrak{F}\left( A,A' \right)$、$\mathfrak{F}\left( B,B' \right)$に存在するのであったので、次式のように写像$g$が定義されると、
\begin{align*}
g:\mathfrak{F}(A,B)\mathfrak{\rightarrow F}\left( A',B' \right);f \mapsto f_{B} \circ f \circ f_{A}^{- 1}
\end{align*}
即ち、次式のように写像$g$が定義されると、
\begin{center}
  \begin{tikzpicture}[auto] 
    \node (a) at (0, 0) {$A'$};
    \node (b) at (0, 3) {$A$};
    \node (c) at (3, 0) {$B'$};
    \node (d) at (3, 3) {$B$};
    \draw [->] (b) to node {$\scriptstyle f$} (d);
    \draw [>->>] (b) to node {$\scriptstyle f_A $} (a);
    \draw [>->>] (d) to node {$\scriptstyle f_B $} (c);
    \draw [->] (a) to node {$\scriptstyle f_B \circ f\circ f_{A}^{-1} $} (c);
    \end{tikzpicture} 
\end{center}
次式のように逆写像$g^{- 1}$が存在するので、
\begin{align*}
g^{- 1}\mathfrak{:F}\left( A',B' \right)\mathfrak{\rightarrow F}(A,B);f \mapsto f_{B}^{- 1} \circ f \circ f_{A}
\end{align*}
その写像$g$は全単射となる。\par
したがって、$\mathfrak{F}(A,B)\mathfrak{\sim F}\left( A',B' \right)$が成り立ち、したがって、次のようになる。
\begin{align*}
{\# B}^{\# A} &= \# {\mathfrak{F}(A,B)}\\
&= \# {\mathfrak{F}\left( A',B' \right)}\\
&= {\# B'}^{\# A'}
\end{align*}
\end{proof}
\begin{thm}\label{1.2.8.8}
次のことが成り立つ\footnote{これはいわゆる指数法則である。}。
\begin{itemize}
\item
  $\forall\# A,\# B,\# C \in \mathcal{F} /\sim $に対し、$B \cap C = \emptyset $が成り立つとすると、次式が成り立つ。
\begin{align*}
{\# A}^{\# B} \cdot {\# A}^{\# C} = {\# A}^{\# B + \# C}
\end{align*}
\item
  $\forall\# A,\# B,\# C \in \mathcal{F} /\sim $に対し、次式が成り立つ。
\begin{align*}
\left( \# B \cdot \# C \right)^{\# A} = {\# B}^{\# A} \cdot {\# C}^{\# A}
\end{align*}
\item
  $\forall\# A,\# B,\# C \in \mathcal{F} /\sim $に対し、次式が成り立つ。
\begin{align*}
\left( {\# A}^{\# B} \right)^{\# C} = {\# A}^{\# B \cdot \# C}
\end{align*}
\end{itemize}
\end{thm}
\begin{proof}
$\forall\# A,\# B,\# C \in \mathcal{F} /\sim $に対し、$B \cap C = \emptyset $が成り立つとする。$\forall f \in \mathfrak{F}(B \sqcup C,A)$に対し、$f|B \in \mathfrak{F}(B,A)$かつ$f|C \in \mathfrak{F}(C,A)$が成り立つので、その集合$\mathfrak{F}(B,A)\mathfrak{\times F}(C,A)$の元$\left( f|B,f|C \right)$がただ1つ決まる。逆に、このような元$\left( f|B,f|C \right)$が1つ決まると、仮定より$B \cap C = \emptyset $が成り立つので、次式のような写像$f$がただ1つ決まる。
\begin{align*}
f:B \sqcup C \rightarrow A;a \mapsto \left\{ \begin{matrix}
f|B(a) & {\mathrm{if}} & a \in B \\
f|C(a) & {\mathrm{if}} & a \in C \\
\end{matrix} \right.\ 
\end{align*}
このようにして、定められる次式のような写像$\mathfrak{f}$は、
\begin{align*}
\mathfrak{f:F}(B \sqcup C,A)\mathfrak{\rightarrow F}(B,A)\mathfrak{\times F}(C,A);f \mapsto \left( f|B,f|C \right)
\end{align*}
逆写像が構成されているので、全単射でありしたがって、次式が成り立つ。
\begin{align*}
\# {\mathfrak{F}(B \sqcup C,A)} = \# \left( \mathfrak{F}(B,A)\mathfrak{\times F}(C,A) \right)
\end{align*}
ここで、次のようになることから、
\begin{align*}
\# \left( \mathfrak{F}(B,A)\mathfrak{\times F}(C,A) \right) &= \# {\mathfrak{F}(B,A)} \cdot \# {\mathfrak{F}(C,A)}\\
&= {\# A}^{\# B} \cdot {\# A}^{\# C}\\
\# {\mathfrak{F}(B \sqcup C,A)} &= {\# A}^{\# {B \sqcup C}}\\
&= {\# A}^{\# B + \# C}
\end{align*}
よって、次式が成り立つ。
\begin{align*}
{\# A}^{\# B} \cdot {\# A}^{\# C} = {\# A}^{\# B + \# C}
\end{align*}\par
$\forall\# A,\# B,\# C \in \mathcal{F} /\sim $に対し、集合$B \times C$からその集合$B$への射影、その集合$C$への射影をそれぞれ${\mathrm{pr}}_{1}$、${\mathrm{pr}}_{2}$とすると、$\forall f \in \mathfrak{F}(A,B \times C)$に対し、${\mathrm{pr}}_{1} \circ f \in \mathfrak{F}(A,B)$かつ${\mathrm{pr}}_{2} \circ f\in \mathfrak{F}(A,C)$が成り立つので、その集合$\mathfrak{F}(A,B)\mathfrak{\times F}(A,C)$の元$\left( {\mathrm{pr}}_{1} \circ f,{\mathrm{pr}}_{2} \circ f \right)$がただ1つ決まる。逆に、このような元$\left( {\mathrm{pr}}_{1} \circ f,{\mathrm{pr}}_{2} \circ f \right)$が1つ決まると、次式のような写像$f$がただ1つ決まる。
\begin{align*}
f:A \rightarrow B \times C;a \mapsto \left( {\mathrm{pr}}_{1} \circ f(a),{\mathrm{pr}}_{2} \circ f(a) \right) = f(a)
\end{align*}
このようにして、定められる次式のような写像$\mathfrak{f}$は、
\begin{align*}
\mathfrak{f:F}(A,B \times C)\mathfrak{\rightarrow F}(A,B)\mathfrak{\times F}(A,C);f \mapsto \left( {\mathrm{pr}}_{1} \circ f,{\mathrm{pr}}_{2} \circ f \right)
\end{align*}
逆写像が構成されているので、全単射でありしたがって、次式が成り立つ。
\begin{align*}
\# {\mathfrak{F}(A,B \times C)} = \# \left( \mathfrak{F}(A,B)\mathfrak{\times F}(A,C) \right)
\end{align*}
ここで、次のようになることから、
\begin{align*}
\# {\mathfrak{F}(A,B \times C)} &= {\# {B \times C}}^{\# A}\\
&= \left( \# B \cdot \# C \right)^{\# A}\\
\# \left( \mathfrak{F}(A,B)\mathfrak{\times F}(A,C) \right) &= \# {\mathfrak{F}(A,B)} \cdot \# {\mathfrak{F}(A,C)}\\
&= {\# B}^{\# A} \cdot {\# C}^{\# A}
\left( \# B \cdot \# C \right)^{\# A} \\
&= {\# B}^{\# A} \cdot {\# C}^{\# A}
\end{align*}\par
$\forall\# A,\# B,\# C \in \mathcal{F} /\sim \forall f \in \mathfrak{F}(B \times C,A)$に対し、その集合$C$の1つの元$c$を用いて次式のような写像$f_{c}$を考えると、
\begin{align*}
\left( f_{c}:B \rightarrow A;b \mapsto f(b,c) \right)\in \mathfrak{F}(B,A)
\end{align*}
$\forall c \in C$に対し、この写像$f_{c}$がただ1つ決まる。ここで、次式のような写像$\widetilde{f}$が考えられれば、
\begin{align*}
\left( \widetilde{f}:C \mapsto \mathfrak{F}(B,A);c \mapsto f_{c} \right)\in \mathfrak{F}\left( C,\mathfrak{F}(B,A) \right)
\end{align*}
$\forall(b,c) \in B \times C$に対し、$f(b,c) = f_{c}(b) = \left( \widetilde{f}(c) \right)(b)$が成り立つので、その写像$\widetilde{f}$もただ1つ決まる。逆に、このような元$\widetilde{f}$が1つ決まるとき、次式のような写像$f'$が考えられれば、
\begin{align*}
f':B \times C \rightarrow A;(b,c) \mapsto \left( \widetilde{f}(c) \right)(b)
\end{align*}
$\widetilde{f} = \widetilde{f'}$が成り立つので、その写像$f'$はただ1つ決まる。このようにして、定められる次式のような写像$\mathfrak{f}$は、
\begin{align*}
\mathfrak{f:F}(B \times C,A)\mathfrak{\rightarrow F}\left( C,\mathfrak{F}(B,A) \right);f \mapsto \widetilde{f}
\end{align*}
逆写像が構成されているので、全単射でありしたがって、次式が成り立つ。
\begin{align*}
\# {\mathfrak{F}(B \times C,A)} = \# {\mathfrak{F}\left( C,\mathfrak{F}(B,A) \right)}
\end{align*}
ここで、次のようになることから、
\begin{align*}
\# {\mathfrak{F}\left( C,\mathfrak{F}(B,A) \right)} &= {\# {\mathfrak{F}(B,A)}}^{\# C}\\
&= \left( {\# A}^{\# B} \right)^{\# C}\\
\# {\mathfrak{F}(B \times C,A)} &= {\# A}^{\# {B \times C}}\\
&= {\# A}^{\# B \cdot \# C}
\end{align*}
よって、
\begin{align*}
\left( {\# A}^{\# B} \right)^{\# C} = {\# A}^{\# B \cdot \# C}
\end{align*}
\end{proof}
\begin{thm}\label{1.2.8.9}
任意の添数集合$\varLambda$によって添数づけられた集合の族$\left\{ A_{\lambda} \right\}_{\lambda \in \varLambda}$が与えられ、$\forall\lambda \in \varLambda$に対し、$A_{\lambda} = A$が成り立つとする。このとき、次式が成り立つ。
\begin{align*}
\# {\prod_{\lambda \in \varLambda} A_{\lambda}} = {\# A}^{\# \varLambda}
\end{align*}
\end{thm}
\begin{proof}
任意の添数集合$\varLambda$によって添数づけられた集合の族$\left\{ A_{\lambda} \right\}_{\lambda \in \varLambda}$が与えられ、$\forall\lambda \in \varLambda$に対し、$A_{\lambda} = A$が成り立つとする。このとき、直積の定義より次式が成り立つ。
\begin{align*}
\prod_{\lambda \in \varLambda} A_{\lambda} &= \left\{ f \in \mathfrak{F}\left( \varLambda,\bigcup_{\lambda \in \varLambda} A_{\lambda} \right) \middle| \forall\lambda \in \varLambda\left[ f(\lambda) \in A_{\lambda} \right] \right\}\\
&= \left\{ f \in \mathfrak{F}(\varLambda,A) \middle| \forall\lambda \in \varLambda\left[ f(\lambda) \in A \right] \right\} = \mathfrak{F}(\varLambda,A)
\end{align*}
これにより、次のようになる。
\begin{align*}
\# {\prod_{\lambda \in \varLambda} A_{\lambda}} = \# {\mathfrak{F}(\varLambda,A)} = {\# A}^{\# \varLambda}
\end{align*}
\end{proof}
\begin{thm}\label{1.2.8.10}
集合$A$の部分集合系$\mathfrak{P}(A)$について、次式が成り立つ\footnote{なお、無限集合$A$について、次のことが成り立つことが知られているが、この証明にZornの補題を用いる必要があるため、のちに述べることとする。
\begin{align*}
  \# B \leq \# A &\Rightarrow \# A + \# B = \# A\\
  1 \leq \# B \leq \# A &\Rightarrow \# A\# B = \# A\\
  2 \leq \# B \leq \# A &\Rightarrow 2^{\# A} = {\# B}^{\# A}
\end{align*}}。
\begin{align*}
\# {\mathfrak{P}(A)} = \# {\mathfrak{F}\left( A,\left\{ 0,1 \right\} \right)} = 2^{\# A}
\end{align*}
\end{thm}
\begin{proof}
集合$A$の部分集合系$\mathfrak{P}(A)$について、$\forall A'\in \mathfrak{P}(A)$に対し、その集合$A'$をこれの指示関数$\chi_{A'}$へうつす写像$\Phi$を考えると、その写像$\Phi$は全単射であったので、次式が成り立つ。
\begin{align*}
\# {\mathfrak{P}(A)} = \# {\mathfrak{F}\left( A,\left\{ 0,1 \right\} \right)} = 2^{\# A}
\end{align*}
\end{proof}
%\hypertarget{ux6fc3ux5ea6ux306epeanoux7cfb}{%
\subsubsection{濃度のPeano系}%\label{ux6fc3ux5ea6ux306epeanoux7cfb}}
\begin{thm}\label{1.2.8.11}
Peano系$\left( \mathbb{Z}_{\geq 0},0,1 + \right)$が与えられたとき、次式のように集合$\mathfrak{L}$と
\begin{align*}
\mathfrak{L} =\left\{ \mathfrak{n \in}\mathcal{F} /\sim \middle| \exists n \in \mathbb{Z}_{\geq 0}\left[ \mathfrak{n} =\# \varLambda_{n} \right] \right\}
\end{align*}
写像$\mathfrak{s}$が定義されると、
\begin{align*}
\mathfrak{s:L \rightarrow L;}\# \varLambda_{n} \mapsto \# \left( \varLambda_{n} \cup \left\{ n + 1 \right\} \right)
\end{align*}
その組$\left( \mathfrak{L,}\# \emptyset ,\mathfrak{s} \right)$はPeano系となる\footnote{この定理により集合の元の個数が負でない整数であることが分かる。}。
\end{thm}
\begin{proof}
添数集合$\varLambda_{n}$が与えられたとする。このとき、次式のように集合$\mathfrak{L}$と
\begin{align*}
\mathfrak{L} =\left\{ \mathfrak{n \in}\mathcal{F} /\sim \middle| \exists n \in \mathbb{Z}_{\geq 0}\left[ \mathfrak{n}=\# \varLambda_{n} \right] \right\}
\end{align*}
写像$\mathfrak{s}$が定義されると、
\begin{align*}
\mathfrak{s:L \rightarrow L};\# \varLambda_{n} \mapsto \# \varLambda_{n + 1}
\end{align*}
$\emptyset  = \varLambda_{0}$が成り立つので、$\# \emptyset \in \mathfrak{L}$が成り立つかつ、空集合$\emptyset $に属するような元が存在しないので、$\# \emptyset  \notin V\left( \mathfrak{s} \right)$が成り立つ。\par
また、$\# \varLambda_{m} \neq \# \varLambda_{n}$が成り立つなら、それらの添数集合たち$\varLambda_{m}$、$\varLambda_{n}$との間には全単射が存在しないことになる。ここで、次式のように全単射な写像$f$が存在すると仮定すると、
\begin{align*}
f:\varLambda_{m + 1}\overset{\sim}{\rightarrow}\varLambda_{n + 1}
\end{align*}
$f(m + 1) = n + 1$のとき、次式のような写像$f'$は明らかに単射で、
\begin{align*}
f':\varLambda_{m} \rightarrow \varLambda_{n};m' \mapsto f\left( m' \right)
\end{align*}
$f(m + 1) = n + 1$が成り立つことに注意すれば、$\forall n' \in \varLambda_{n}$に対し、$f^{- 1}\left( n' \right) \in \varLambda_{m}$が成り立つので、$V\left( f' \right) = \varLambda_{n}$となりその写像$f'$は全射となり、したがって、それらの添数集合たち$\varLambda_{m}$、$\varLambda_{n}$との間には全単射が存在することになるが、これは仮定に矛盾する。$f(m + 1) \neq n + 1$のとき、次式のような写像$f''$は明らかに全単射であるので、
\begin{align*}
f'':\varLambda_{m} \cup \left\{ m + 1 \right\} \rightarrow \varLambda_{m} \cup \left\{ m + 1 \right\};m' \mapsto \left\{ \begin{matrix}
f^{- 1}(n + 1) & {\mathrm{if}} & m' = m + 1 \\
m + 1 & {\mathrm{if}} & m' = f^{- 1}(n + 1) \\
m' & \mathrm{otherwise} & \  \\
\end{matrix} \right.\ 
\end{align*}
写像$f \circ f''$もまた全単射となる。このとき、$f \circ f''(m + 1) = n + 1$が成り立ち、上記の議論によりそれらの添数集合たち$\varLambda_{m}$、$\varLambda_{n}$との間には全単射が存在することになるが、これは仮定に矛盾する。以上より、次式のような任意の写像$f$は全単射であることができない。
\begin{align*}
f:\varLambda_{m + 1}\overset{\sim}{\rightarrow}\varLambda_{n + 1}
\end{align*}
したがって、$\neg\varLambda_{m + 1} \sim \varLambda_{n + 1}$が成り立つことになり$\mathfrak{s}\left( \# \varLambda_{m} \right)\mathfrak{\neq s}\left( \# \varLambda_{n} \right)$が成り立つので、その写像$\mathfrak{s}$は単射である。\par
また、$\mathfrak{L}'\subseteq \mathfrak{L}$なる集合が$\# \emptyset  \in \mathfrak{L}'$が成り立つかつ、$\forall\# \varLambda_{n}\in \mathfrak{L}$に対し、$\# \varLambda_{n} \in \mathfrak{L}'$が成り立つなら、$\mathfrak{s}\left( \# \varLambda_{n} \right) \in \mathfrak{L}'$が成り立つことを満たすとき、$\forall n \in \mathbb{Z}_{\geq 0}$に対し、$\# \varLambda_{n} \in \mathfrak{L}'$が成り立つことを示そう。$n = 0$のとき、仮定より明らかに次式が成り立つ。
\begin{align*}
\# \varLambda_{0} = \# \emptyset  \in \mathfrak{L}'
\end{align*}
$n = k$のとき、$\# \varLambda_{k} \in \mathfrak{L}'$が成り立つなら、$n = k + 1$のとき、$\# \varLambda_{k + 1} = \mathfrak{s}\left( \# \varLambda_{k} \right) \in \mathfrak{L}'$が成り立つ。以上より、$\forall n \in \mathbb{Z}_{\geq 0}$に対し、$\# \varLambda_{n} \in \mathfrak{L}'$が成り立つ。ここで、$\forall\# \varLambda_{n}\in \mathfrak{L}$に対し、その集合$\mathfrak{L}$の定義より$n \in \mathbb{Z}_{\geq 0}$が成り立ち、したがって、$\# \varLambda_{n} \in \mathfrak{L}'$が成り立つことになるので、$\mathfrak{L \subseteq}\mathfrak{L}'$が成り立つ。以上より、$\mathfrak{L}'\subseteq \mathfrak{L}$なる集合が$\# \emptyset  \in \mathfrak{L}'$が成り立つかつ、$\forall\# \varLambda_{n}\in \mathfrak{L}$に対し、$\# \varLambda_{n} \in \mathfrak{L}'$が成り立つなら、$\mathfrak{L} =\mathfrak{L}'$が成り立つ。
\end{proof}
\begin{thm}\label{1.2.8.12}
Peano系$\left( \mathfrak{L,}\# \emptyset ,\mathfrak{s} \right)$が与えられたとき、次式が成り立つ。
\begin{align*}
\left\{ \begin{matrix}
\# \varLambda_{m} + \# \emptyset  = \# \varLambda_{m} \\
\# \varLambda_{m} + \# \varLambda_{n + 1} = \mathfrak{s}\left( \# \varLambda_{m} + \# \varLambda_{n} \right) \\
\end{matrix} \right.\ 
\end{align*}
\end{thm}\par
これにより、そのPeano系$\left( \mathfrak{L,}\# \emptyset ,\mathfrak{s} \right)$のもとで、その演算$+$は自然数での加法$+$に等しいことになる。
\begin{proof}
Peano系$\left( \mathfrak{L,}\# \emptyset ,\mathfrak{s} \right)$が与えられたとき、明らかに$\varLambda_{m} \sqcup \emptyset  = \varLambda_{m}$が成り立つので、次式が成り立つ。
\begin{align*}
\# \varLambda_{m} + \# \emptyset  = \# \left( \varLambda_{m} \sqcup \emptyset  \right) = \# \varLambda_{m}
\end{align*}
また、$m = 0$のとき、次のようになる。
\begin{align*}
\# \varLambda_{m} + \# \varLambda_{n + 1} &= \# \emptyset  + \# \varLambda_{n + 1}\\
&= \# \varLambda_{n + 1}\\
& = \mathfrak{s}\left( \# \varLambda_{n} \right)\\
& = \mathfrak{s}\left( \# \emptyset  + \# \varLambda_{n} \right)\\
& = \mathfrak{s}\left( \# \varLambda_{m} + \# \varLambda_{n} \right)
\end{align*}\par
$m \neq 0$のとき、次式のような集合$\varLambda'$が与えられたとき、
\begin{align*}
\varLambda' = \left\{ n'' \in \mathbb{N} \middle| n'' = n' + m \land n' \in \varLambda_{n + 1} \right\}
\end{align*}
次式のような写像$f$について、
\begin{align*}
f:\varLambda_{n + 1} \rightarrow \varLambda';n' \mapsto n' + m
\end{align*}
写像$+_{m}:\mathbb{Z}_{\geq 0} \rightarrow \mathbb{Z}_{\geq 0};n \mapsto n + m$が考えられれば、$m \neq 0$が成り立つので、その写像$+_{m}$は単射であるから、その写像$f$も単射となるかつ、その集合$\varLambda'$の定義よりその写像$f$は全射となる。したがって、その写像$f$は全単射となるので、$\# \varLambda_{n + 1} = \# \varLambda'$が成り立つ。ここで、定義より明らかに$\varLambda_{m} \sqcup \varLambda' = \varLambda_{m + n + 1}$が成り立つので、次のようになる。
\begin{align*}
\# \varLambda_{m} + \# \varLambda_{n + 1} &= \# \varLambda_{m} + \# \varLambda'\\
&= \# \left( \varLambda_{m} \sqcup \varLambda' \right)\\
&= \# \varLambda_{m + n + 1}\\
& = \mathfrak{s}\left( \# \varLambda_{m + n} \right)
\end{align*}
ここで、次式のような集合$\varLambda''$が与えられたとき、
\begin{align*}
\varLambda'' = \left\{ n'' \in \mathbb{N} \middle| n'' = n' + m \land n' \in \varLambda_{n} \right\}
\end{align*}
次式のような写像$f'$について、
\begin{align*}
f:\varLambda_{n} \rightarrow \varLambda'';n' \mapsto n' + m
\end{align*}
同様にして、その写像$f''$は全単射であることが示されるので、$\# \varLambda_{n} = \# \varLambda''$が成り立つ。ここで、定義より明らかに$\varLambda_{m} \sqcup \varLambda'' = \varLambda_{m + n}$が成り立つので、次のようになる。
\begin{align*}
\# \varLambda_{m} + \# \varLambda_{n + 1}&=\mathfrak{s}\left( \# \varLambda_{m + n} \right)\\
&=\mathfrak{s}\left( \# \left( \varLambda_{m} \sqcup \varLambda'' \right) \right)\\
&=\mathfrak{s}\left( \# \varLambda_{m} + \# \varLambda'' \right)\\
&=\mathfrak{s}\left( \# \varLambda_{m} + \# \varLambda_{n} \right)
\end{align*}
\end{proof}
\begin{thm}\label{1.2.8.13}
Peano系$\left( \mathfrak{L,}\# \emptyset ,\mathfrak{s} \right)$が与えられたとき、次式が成り立つ。
\begin{align*}
\left\{ \begin{matrix}
\# \varLambda_{m} \cdot \# \emptyset  = \# \emptyset  \\
\# \varLambda_{m} \cdot \# \varLambda_{n + 1} = \left( \# \varLambda_{m} \cdot \# \varLambda_{n} \right) + \# \varLambda_{m} \\
\end{matrix} \right.\ 
\end{align*}
これにより、そのPeano系$\left( \mathfrak{L,}\# \emptyset ,\mathfrak{s} \right)$のもとでその演算$\cdot$は自然数での乗法$\cdot$に等しいことになる。
\end{thm}
\begin{proof}
Peano系$\left( \mathfrak{L,}\# \emptyset ,\mathfrak{f} \right)$が与えられたとき、$\varLambda_{m} \times \emptyset  = \emptyset $が成り立つので、次のようになる。
\begin{align*}
\# \varLambda_{m} \cdot \# \emptyset  = \# \left( \varLambda_{m} \times \emptyset  \right) = \# \emptyset 
\end{align*}
また、$m \neq 0$のとき、次のようになる。
\begin{align*}
\# \varLambda_{m} \cdot \# \varLambda_{n + 1} = \# \left( \varLambda_{m} \times \varLambda_{n + 1} \right)
\end{align*}
ここで、$\varLambda_{n + 1} = \varLambda_{n} \sqcup \left\{ n + 1 \right\}$が成り立つことに注意すれば、次のようになる。
\begin{align*}
\# \varLambda_{m} \cdot \# \varLambda_{n + 1} &= \# \left( \varLambda_{m} \times \left( \varLambda_{n} \sqcup \left\{ n + 1 \right\} \right) \right)\\
&= \# \left( \left( \varLambda_{m} \times \varLambda_{n} \right) \sqcup \left( \varLambda_{m} \times \left\{ n + 1 \right\} \right) \right)\\
&= \# \left( \varLambda_{m} \times \varLambda_{n} \right) + \# \left( \varLambda_{m} \times \left\{ n + 1 \right\} \right)\\
&= \left( \# \varLambda_{m} \cdot \# \varLambda_{n} \right) + \# \left( \varLambda_{m} \times \left\{ n + 1 \right\} \right)
\end{align*}
ここで、次式のような写像$f$が考えられると、
\begin{align*}
f:\varLambda_{m} \rightarrow \varLambda_{m} \times \left\{ n + 1 \right\};m' \mapsto \left( m',n + 1 \right)
\end{align*}
これは明らかに全単射となるので、$\varLambda_{m} \sim \varLambda_{m} \times \left\{ n + 1 \right\}$が成り立ち$\# \varLambda_{m} = \# \left( \varLambda_{m} \times \left\{ n + 1 \right\} \right)$が成り立つ。したがって、次のようになる。
\begin{align*}
\# \varLambda_{m} \cdot \# \varLambda_{n + 1} = \left( \# \varLambda_{m} \cdot \# \varLambda_{n} \right) + \# \varLambda_{m}
\end{align*}
\end{proof}
\begin{thm}\label{1.2.8.14}
Peano系$\left( \mathfrak{L,}\# \emptyset ,\mathfrak{s} \right)$が与えられたとき、$\mathfrak{\forall m,n \in L}$に対し、次のようにおくと、
\begin{align*}
\mathfrak{L}_{\mathfrak{\leq n}} = \left\{ \mathfrak{m \in L} \middle| \exists!\mathfrak{m}'\in \mathfrak{L}\left[ \mathfrak{n = m +}\mathfrak{m}' \right] \right\}
\end{align*}
$\mathfrak{m \in}\mathfrak{L}_{\mathfrak{\leq n}}$が成り立つならそのときに限り、$\mathfrak{m \leq n}$が成り立つ。
\end{thm}
\begin{proof}
Peano系$\left( \mathfrak{L,}\# \emptyset ,\mathfrak{s} \right)$が与えられたとき、$\mathfrak{\forall m,n \in L}$に対し、次のようにおくと、
\begin{align*}
\mathfrak{L}_{\mathfrak{\leq n}} = \left\{ \mathfrak{m \in L} \middle| \exists!\mathfrak{m}'\in \mathfrak{L}\left[ \mathfrak{n = m +}\mathfrak{m}' \right] \right\}
\end{align*}
$\mathfrak{m \in}\mathfrak{L}_{\mathfrak{\leq n}}$が成り立つなら、$\exists!\mathfrak{m}'\in \mathfrak{L}$に対し、$\mathfrak{n = m +}\mathfrak{m}'$が成り立つ。そこで、$\exists m,n \in \mathbb{Z}_{\geq 0}$に対し、$\mathfrak{m} =\# \varLambda_{m}$、$\mathfrak{n} =\# \varLambda_{n}$が成り立つのでそうすると、ある集合$\varLambda'$が存在して、$\varLambda_{n} = \varLambda' \sqcup \varLambda_{m}$が成り立つ。これにより、包含写像$\varLambda_{m} \hookrightarrow \varLambda_{n}$が考えられれば、これは単射なので、$\mathfrak{m} = \# \varLambda_{m} \leq \# \varLambda_{n} = \mathfrak{n}$が成り立つ。\par
逆に、$\mathfrak{m \leq n}$が成り立つなら、ある単射な写像$f:\varLambda_{m} \rightarrow \varLambda_{n}$が存在する。そこで、$\mathfrak{n}' = \# {\varLambda_{n} \setminus V(f)}$とおくと、次の写像$f'$は全単射なので、
\begin{align*}
f':\varLambda_{m} \rightarrow V(f);i \mapsto f(i)
\end{align*}
次のようになる。
\begin{align*}
\mathfrak{n} &= \# \varLambda_{n}\\
&= \# \left( V(f) \sqcup \varLambda_{n} \setminus V(f) \right)\\
&= \# {V(f)} + \# {\varLambda_{n} \setminus V(f)}\\
&= \# \varLambda_{m} + \# {\varLambda_{n} \setminus V(f)}\\
&= \mathfrak{m} + \# {\varLambda_{n} \setminus V(f)}
\end{align*}
定理\ref{1.2.4.19}、即ち、その集合$\mathbb{Z}_{\geq 0}$の比較可能性より$\mathfrak{m \in}\mathfrak{L}_{\mathfrak{\leq n}}$が成り立つ。
\end{proof}
\begin{thebibliography}{50}
  \bibitem{1}
    松坂和夫, 集合・位相入門, 岩波書店, 1968. 新装版第2刷 p78-86 ISBM978-4-00-029871-1
\end{thebibliography}
\end{document}

