\documentclass[dvipdfmx]{jsarticle}
\setcounter{section}{4}
\setcounter{subsection}{3}
\usepackage{xr}
\externaldocument{2.1.1}
\externaldocument{2.1.2}
\externaldocument{2.1.3}
\externaldocument{2.1.5}
\externaldocument{2.4.1}
\externaldocument{2.4.2}
\usepackage{amsmath,amsfonts,amssymb,array,comment,mathtools,url,docmute}
\usepackage{longtable,booktabs,dcolumn,tabularx,mathtools,multirow,colortbl,xcolor}
\usepackage[dvipdfmx]{graphics}
\usepackage{bmpsize}
\usepackage{amsthm}
\usepackage{enumitem}
\setlistdepth{20}
\renewlist{itemize}{itemize}{20}
\setlist[itemize]{label=•}
\renewlist{enumerate}{enumerate}{20}
\setlist[enumerate]{label=\arabic*.}
\setcounter{MaxMatrixCols}{20}
\setcounter{tocdepth}{3}
\newcommand{\rotin}{\text{\rotatebox[origin=c]{90}{$\in $}}}
\newcommand{\amap}[6]{\text{\raisebox{-0.7cm}{\begin{tikzpicture} 
  \node (a) at (0, 1) {$\textstyle{#2}$};
  \node (b) at (#6, 1) {$\textstyle{#3}$};
  \node (c) at (0, 0) {$\textstyle{#4}$};
  \node (d) at (#6, 0) {$\textstyle{#5}$};
  \node (x) at (0, 0.5) {$\rotin $};
  \node (x) at (#6, 0.5) {$\rotin $};
  \draw[->] (a) to node[xshift=0pt, yshift=7pt] {$\textstyle{\scriptstyle{#1}}$} (b);
  \draw[|->] (c) to node[xshift=0pt, yshift=7pt] {$\textstyle{\scriptstyle{#1}}$} (d);
\end{tikzpicture}}}}
\newcommand{\twomaps}[9]{\text{\raisebox{-0.7cm}{\begin{tikzpicture} 
  \node (a) at (0, 1) {$\textstyle{#3}$};
  \node (b) at (#9, 1) {$\textstyle{#4}$};
  \node (c) at (#9+#9, 1) {$\textstyle{#5}$};
  \node (d) at (0, 0) {$\textstyle{#6}$};
  \node (e) at (#9, 0) {$\textstyle{#7}$};
  \node (f) at (#9+#9, 0) {$\textstyle{#8}$};
  \node (x) at (0, 0.5) {$\rotin $};
  \node (x) at (#9, 0.5) {$\rotin $};
  \node (x) at (#9+#9, 0.5) {$\rotin $};
  \draw[->] (a) to node[xshift=0pt, yshift=7pt] {$\textstyle{\scriptstyle{#1}}$} (b);
  \draw[|->] (d) to node[xshift=0pt, yshift=7pt] {$\textstyle{\scriptstyle{#2}}$} (e);
  \draw[->] (b) to node[xshift=0pt, yshift=7pt] {$\textstyle{\scriptstyle{#1}}$} (c);
  \draw[|->] (e) to node[xshift=0pt, yshift=7pt] {$\textstyle{\scriptstyle{#2}}$} (f);
\end{tikzpicture}}}}
\renewcommand{\thesection}{第\arabic{section}部}
\renewcommand{\thesubsection}{\arabic{section}.\arabic{subsection}}
\renewcommand{\thesubsubsection}{\arabic{section}.\arabic{subsection}.\arabic{subsubsection}}
\everymath{\displaystyle}
\allowdisplaybreaks[4]
\usepackage{vtable}
\theoremstyle{definition}
\newtheorem{thm}{定理}[subsection]
\newtheorem*{thm*}{定理}
\newtheorem{dfn}{定義}[subsection]
\newtheorem*{dfn*}{定義}
\newtheorem{axs}[dfn]{公理}
\newtheorem*{axs*}{公理}
\renewcommand{\headfont}{\bfseries}
\makeatletter
  \renewcommand{\section}{%
    \@startsection{section}{1}{\z@}%
    {\Cvs}{\Cvs}%
    {\normalfont\huge\headfont\raggedright}}
\makeatother
\makeatletter
  \renewcommand{\subsection}{%
    \@startsection{subsection}{2}{\z@}%
    {0.5\Cvs}{0.5\Cvs}%
    {\normalfont\LARGE\headfont\raggedright}}
\makeatother
\makeatletter
  \renewcommand{\subsubsection}{%
    \@startsection{subsubsection}{3}{\z@}%
    {0.4\Cvs}{0.4\Cvs}%
    {\normalfont\Large\headfont\raggedright}}
\makeatother
\makeatletter
\renewenvironment{proof}[1][\proofname]{\par
  \pushQED{\qed}%
  \normalfont \topsep6\p@\@plus6\p@\relax
  \trivlist
  \item\relax
  {
  #1\@addpunct{.}}\hspace\labelsep\ignorespaces
}{%
  \popQED\endtrivlist\@endpefalse
}
\makeatother
\renewcommand{\proofname}{\textbf{証明}}
\usepackage{tikz,graphics}
\usepackage[dvipdfmx]{hyperref}
\usepackage{pxjahyper}
\hypersetup{
 setpagesize=false,
 bookmarks=true,
 bookmarksdepth=tocdepth,
 bookmarksnumbered=true,
 colorlinks=false,
 pdftitle={},
 pdfsubject={},
 pdfauthor={},
 pdfkeywords={}}
\begin{document}
%\hypertarget{ux5546vectorux7a7aux9593ux3068ux53ccux5bfeux7a7aux9593}{%
\subsection{商vector空間と双対空間}%\label{ux5546vectorux7a7aux9593ux3068ux53ccux5bfeux7a7aux9593}}
%\hypertarget{ux5546vectorux7a7aux9593ux304bux3089ux306eux81eaux7136ux306aux7ddaux5f62ux5199ux50cf}{%
\subsubsection{商vector空間からの自然な線形写像}%\label{ux5546vectorux7a7aux9593ux304bux3089ux306eux81eaux7136ux306aux7ddaux5f62ux5199ux50cf}}
\begin{thm}\label{2.4.4.1}
体$K$上の$m$次元vector空間$V$、$n$次元vector空間$U$、そのvector空間$V$の部分空間$W$、線形写像$f:V \rightarrow U$が与えられたとき、$W \subseteq \ker f$が成り立つとき、$\mathbf{v} \equiv \mathbf{w}\ \mathrm{mod}W$が成り立つなら、$f\left( \mathbf{v} \right) = f\left( \mathbf{w} \right)$が成り立つ。
\end{thm}
\begin{proof}
体$K$上の$m$次元vector空間$V$、$n$次元vector空間$U$、そのvector空間$V$の部分空間$W$、線形写像$f:V \rightarrow U$が与えられたとき、$W \subseteq \ker f$が成り立つとき、$\mathbf{v} \equiv \mathbf{w}\ \mathrm{mod}W$が成り立つなら、$\mathbf{v} - \mathbf{w} \in W$が成り立つ。そこで、$W \subseteq \ker f$が成り立つので、$f\left( \mathbf{v} - \mathbf{w} \right) = f\left( \mathbf{v} \right) - f\left( \mathbf{w} \right) = \mathbf{0}$が成り立つ。よって、$f\left( \mathbf{v} \right) = f\left( \mathbf{w} \right)$が得られる。
\end{proof}
\begin{thm}\label{2.4.4.2}
体$K$上の$m$次元vector空間$V$、$n$次元vector空間$U$、そのvector空間$V$の部分空間$W$、線形写像$f:V \rightarrow U$が与えられたとき、$W \subseteq \ker f$が成り立つなら、$\forall\mathbf{v} + W \in {V}/{W}\exists!\mathbf{u} \in U$に対し、$\mathbf{u} = f\left( \mathbf{v} \right)$が成り立つ。
\end{thm}
\begin{proof}
体$K$上の$m$次元vector空間$V$、$n$次元vector空間$U$、そのvector空間$V$の部分空間$W$、線形写像$f:V \rightarrow U$が与えられたとき、$W \subseteq \ker f$が成り立つなら、$\forall\mathbf{v} + W \in {V}/{W}\exists\mathbf{u} \in U$に対し、$\mathbf{u} = f\left( \mathbf{v} \right)$が成り立つのは明らかである。そこで、$\forall\mathbf{w} + W \in {V}/{W}$に対し、$\mathbf{v} + W = \mathbf{w} + W$が成り立つかつ、$f\left( \mathbf{v} \right) \neq f\left( \mathbf{w} \right)$が成り立つと仮定すると、$\mathbf{v} + W = \mathbf{w} + W$が成り立つならそのときに限り、$\mathbf{v} \equiv \mathbf{w}\ \mathrm{mod}W$が成り立つことになり、定理\ref{2.4.4.1}より$f\left( \mathbf{v} \right) = f\left( \mathbf{w} \right)$が得られるが、これは仮定に矛盾する。よって、このようなvector$f(\mathbf{v})$はそのvector空間$U$に一意的に存在する。
\end{proof}
\begin{dfn}
体$K$上の$m$次元vector空間$V$、$n$次元vector空間$U$、そのvector空間$V$の部分空間$W$、線形写像$f:V \rightarrow U$が与えられたとする。$W \subseteq \ker f$が成り立つとき、定理\ref{2.4.4.2}により次式のように写像$\psi_{W}(f)$が定義される。
\begin{align*}
\psi_{W}(f):{V}/{W} \rightarrow U;\mathbf{v} + W \mapsto f\left( \mathbf{v} \right)
\end{align*}
その写像$\psi_{W}(f)$をその線形写像$f$によるその部分空間$W$に関する商vector空間からの自然な線形写像という。
\end{dfn}
\begin{thm}\label{2.4.4.3}
体$K$上の$m$次元vector空間$V$、$n$次元vector空間$U$、そのvector空間$V$の部分空間$W$、線形写像$f:V \rightarrow U$が与えられたとする。$W \subseteq \ker f$が成り立つとき、その線形写像$f$によるその部分空間$W$に関する商vector空間からの自然な線形写像$\psi_{W}(f)$は線形写像である。この定理によりその写像$\psi_{W}(f)$を商vector空間からの自然な線形写像と呼んでも混乱は生じなかろう。
\end{thm}
\begin{proof}
体$K$上の$m$次元vector空間$V$、$n$次元vector空間$U$、そのvector空間$V$の部分空間$W$、線形写像$f:V \rightarrow U$が与えられたとする。$W \subseteq \ker f$が成り立つとき、その線形写像$f$によるその部分空間$W$に関する商vector空間からの自然な線形写像$\psi_{W}(f)$について、$\forall k,l \in K\forall\mathbf{v} + W,\mathbf{w} + W \in {V}/{W}$に対し、次のようになる。
\begin{align*}
\psi_{W}(f)\left( k\left( \mathbf{v} + W \right) + l\left( \mathbf{w} + W \right) \right) &= \psi_{W}(f)\left( k\mathbf{v} + l\mathbf{w} + W \right) = f\left( k\mathbf{v} + l\mathbf{w} \right)\\
&= kf\left( \mathbf{v} \right) + lf\left( \mathbf{w} \right) = k\psi_{W}(f)\left( \mathbf{v} + W \right) + l\psi_{W}(f)\left( \mathbf{w} + W \right)
\end{align*}
よって、その商vector空間からの自然な線形写像$\psi_{W}(f)$は線形写像である。
\end{proof}
\begin{thm}\label{2.4.4.4}
体$K$上の$m$次元vector空間$V$、$n$次元vector空間$U$、そのvector空間$V$の部分空間$W$、線形写像$f:V \rightarrow U$が与えられたとし、$W \subseteq \ker f$が成り立つとする。$W = \ker f$が成り立つならそのときに限り、その線形写像$f$によるその部分空間$W$に関する商vector空間からの自然な線形写像$\psi_{W}(f)$は単射である。
\end{thm}
\begin{proof}
体$K$上の$m$次元vector空間$V$、$n$次元vector空間$U$、そのvector空間$V$の部分空間$W$、線形写像$f:V \rightarrow U$が与えられたとし、$W \subseteq \ker f$が成り立つとする。$W = \ker f$が成り立つなら、その線形写像$f$によるその部分空間$W$に関する商vector空間からの自然な線形写像$\psi_{W}(f)$について、$\forall\mathbf{v} + W,\mathbf{w} + W \in {V}/{W}$に対し、$\mathbf{v} + W \neq \mathbf{w} + W$が成り立つなら、$\mathbf{v} \equiv \mathbf{w}\ \mathrm{mod}W$が成り立つ、即ち、$\mathbf{v} - \mathbf{w} \notin W$が成り立つことになり、したがって、$\ker f \subseteq W$が成り立つことと対偶律により$f\left( \mathbf{v} - \mathbf{w} \right) = f\left( \mathbf{v} \right) - f\left( \mathbf{w} \right) \neq \mathbf{0}$が成り立つ。よって、$f\left( \mathbf{v} \right) \neq f\left( \mathbf{w} \right)$が得られ、その線形写像$f$によるその部分空間$W$に関する商vector空間からの自然な線形写像$\psi_{W}(f)$は単射であることが示された。\par
逆に、その線形写像$f$によるその部分空間$W$に関する商vector空間からの自然な線形写像$\psi_{W}(f)$が単射であるなら、$\forall\mathbf{v} + W,\mathbf{w} + W \in {V}/{W}$に対し、$\mathbf{v} + W \neq \mathbf{w} + W$が成り立つなら、$\mathbf{v} - \mathbf{w} \notin W$が成り立つ。そこで、$\mathbf{v} - \mathbf{w} \in \ker f$が成り立つと仮定すると、$f\left( \mathbf{v} - \mathbf{w} \right) = f\left( \mathbf{v} \right) - f\left( \mathbf{w} \right) = \mathbf{0}$が成り立つことになり、したがって、$f\left( \mathbf{v} \right) = f\left( \mathbf{w} \right)$、即ち、$\psi_{W}(f)\left( \mathbf{v} + W \right) = \psi_{W}(f)\left( \mathbf{w} + W \right)$が成り立つことになるが、これはその線形写像$\psi_{W}(f)$が単射であることに矛盾する。したがって、$\mathbf{v} - \mathbf{w} \notin \ker f$が成り立つ。あとは、対偶律より$\ker f \subseteq W$が得られ、仮定よりよって、$W = \ker f$が成り立つ。
\end{proof}
\begin{thm}\label{2.4.4.5}
体$K$上の$m$次元vector空間$V$、$n$次元vector空間$U$、そのvector空間$V$の部分空間$W$、線形写像$f:V \rightarrow U$が与えられたとする。$W = \ker f$が成り立つなら、${V}/{\ker f} \cong V(f)$が成り立つ。
\end{thm}
\begin{proof}
体$K$上の$m$次元vector空間$V$、$n$次元vector空間$U$、そのvector空間$V$の部分空間$W$、線形写像$f:V \rightarrow U$が与えられたとする。$W = \ker f$が成り立つなら、その線形写像$f$によるその部分空間$W$に関する商vector空間からの自然な線形写像$\psi_{W}(f)$は単射である。そこで、その線形写像$\psi_{W}(f)$の終集合を$V\left( \psi_{W}(f) \right)$にしたものは線形同型写像であるので、${V}/{W} \cong V\left( \psi(f) \right)$が成り立つ。そこで、$\forall\mathbf{u} \in U$に対し、$\mathbf{u} \in V\left( \psi_{W}(f) \right)$が成り立つならそのときに限り、あるvector$\mathbf{v} + W$がその商vector空間${V}/{W}$に存在して、$\mathbf{u} = \psi_{W}(f)\left( \mathbf{v} + W \right) = f\left( \mathbf{v} \right)$が成り立つ。これが成り立つならそのときに限り、あるvector$\mathbf{v}$がそのvector空間$V$に存在して、$\mathbf{u} = \psi_{W}(f)\left( \mathbf{v} + W \right) = f\left( \mathbf{v} \right)$が成り立つ。これにより、$\mathbf{u} \in V(f)$が成り立つので、$V\left( \psi_{W}(f) \right) = V(f)$が得られる。よって、次式が成り立つ。
\begin{align*}
{V}/{\ker f} = {V}/{W} \cong V\left( \psi_{W}(f) \right) = V(f)
\end{align*}
\end{proof}
%\hypertarget{ux5546vectorux7a7aux9593ux304bux3089ux306eux81eaux7136ux306aux7ddaux5f62ux5199ux50cfux3092ux8a98ux5c0eux3059ux308bux5199ux50cf}{%
\subsubsection{商vector空間からの自然な線形写像を誘導する写像}%\label{ux5546vectorux7a7aux9593ux304bux3089ux306eux81eaux7136ux306aux7ddaux5f62ux5199ux50cfux3092ux8a98ux5c0eux3059ux308bux5199ux50cf}}
\begin{dfn}
体$K$上の$m$次元vector空間$V$、$n$次元vector空間$U$、そのvector空間$V$の部分空間$W$が与えられたとき、次式のようにおくと、
\begin{align*}
D\left( \psi_{W} \right) = \left\{ f \in L(V,U) \middle| W \subseteq \ker f \right\}
\end{align*}
次式のように写像$\psi_{W}$が定義される。
\begin{align*}
\psi_{W}:D\left( \psi_{W} \right) \rightarrow L\left( {V}/{W},U \right);f \mapsto \left( \psi_{W}(f):{V}/{W} \rightarrow U;\mathbf{v} + W \mapsto f\left( \mathbf{v} \right) \right)
\end{align*}
この写像$\psi_{W}$をそれらのvector空間たち$V$、$U$間のその部分空間$W$に関する商vector空間からの自然な線形写像を誘導する写像ということにする。
\end{dfn}
\begin{thm}\label{2.4.4.6}
体$K$上の$m$次元vector空間$V$、$n$次元vector空間$U$、そのvector空間$V$の部分空間$W$が与えられたとき、その集合$\left\{ f \in L(V,U) \middle| W \subseteq \ker f \right\}$はそのvector空間$L(V,U)$の部分空間をなす。
\end{thm}
\begin{proof}
体$K$上の$m$次元vector空間$V$、$n$次元vector空間$U$、そのvector空間$V$の部分空間$W$が与えられたとき、その集合$\left\{ f \in L(V,U) \middle| W \subseteq \ker f \right\}$において、そのvector空間$L(V,U)$の零vector$0:V \rightarrow U;\mathbf{v} \mapsto \mathbf{0}$はもちろん、$0 \in \left\{ f \in L(V,U) \middle| W \subseteq \ker f \right\}$を満たす。さらに、$\forall k,l \in K\forall f,g \in \left\{ f \in L(V,U) \middle| W \subseteq \ker f \right\}$に対し、写像$kf + lg$は$kf + lg \in L(V,U)$を満たすのであった。このとき、$\forall\mathbf{w} \in W$に対し、次のようになる。
\begin{align*}
(kf + lg)\left( \mathbf{w} \right) = kf\left( \mathbf{w} \right) + lg\left( \mathbf{w} \right) = k\mathbf{0} + l\mathbf{0} = \mathbf{0}
\end{align*}
ゆえに、$kf + lg \in \left\{ f \in L(V,U) \middle| W \subseteq \ker f \right\}$が得られる。よって、定理\ref{2.1.1.9}よりその集合$\left\{ f \in L(V,U) \middle| W \subseteq \ker f \right\}$はそのvector空間$L(V,U)$の部分空間をなす。
\end{proof}
\begin{thm}\label{2.4.4.7}
体$K$上の$m$次元vector空間$V$、$n$次元vector空間$U$、そのvector空間$V$の部分空間$W$が与えられたとき、それらのvector空間たち$V$、$U$間のその部分空間$W$に関する商vector空間からの自然な線形写像を誘導する写像$\psi_{W}$は線形写像である。
\end{thm}
\begin{proof}
体$K$上の$m$次元vector空間$V$、$n$次元vector空間$U$、そのvector空間$V$の部分空間$W$が与えられたとき、それらのvector空間たち$V$、$U$間のその部分空間$W$に関する商vector空間からの自然な線形写像を誘導する写像$\psi_{W}$について、定理\ref{2.4.4.6}に注意すれば、次式のようにおくと、
\begin{align*}
D\left( \psi_{W} \right) = \left\{ f \in L(V,U) \middle| W \subseteq \ker f \right\}
\end{align*}
$\forall k,l \in K\forall f,g \in D\left( \psi_{W} \right)$に対し、線形写像$\psi_{W}(kf + lg)$が定義されて、$\forall\mathbf{v} + W \in {V}/{W}$に対し、次のようになる。
\begin{align*}
\psi_{W}(kf + lg)\left( \mathbf{v} + W \right) &= (kf + lg)\left( \mathbf{v} \right) = kf\left( \mathbf{v} \right) + lg\left( \mathbf{v} \right)\\
&= k\psi_{W}(f)\left( \mathbf{v} + W \right) + l\psi_{W}(g)\left( \mathbf{v} + W \right)\\
&= \left( k\psi_{W}(f) + l\psi_{W}(g) \right)\left( \mathbf{v} + W \right)
\end{align*}
以上より、$\psi_{W}(kf + lg) = k\psi_{W}(f) + l\psi_{W}(g)$が得られたので、それらのvector空間たち$V$、$U$間のその部分空間$W$に関する商vector空間からの自然な線形写像を誘導する写像$\psi_{W}$は線形写像である。
\end{proof}
\begin{thm}\label{2.4.4.8}
体$K$上の$n$次元vector空間$V$の部分空間$W$、双対空間$V^{*}$が与えられたとする。$\forall f \in V^{*}$に対し、$f \in W^{\bot}$が成り立つなら、$W \subseteq \ker f$が成り立つ。
\end{thm}
\begin{proof}
体$K$上の$n$次元vector空間$V$の部分空間$W$、双対空間$V^{*}$が与えられたとする。$\forall f \in V^{*}$に対し、$f \in W^{\bot}$が成り立つなら、$\forall\mathbf{w} \in W$に対し、$f\left( \mathbf{w} \right) = 0$が成り立つ。ゆえに、$\mathbf{w} \in \ker f$が得られ、よって、$W \subseteq \ker f$が成り立つ。
\end{proof}
\begin{thm}\label{2.4.4.9}
体$K$上の$n$次元vector空間$V$の部分空間$W$、双対空間$V^{*}$が与えられたとする。それらのvector空間たち$V$、$U$間のその部分空間$W$に関する商vector空間からの自然な線形写像を誘導する写像$\psi_{W}$は次式のように与えられ、
\begin{align*}
\psi_{W}:W^{\bot} \rightarrow \left( {V}/{W} \right)^{*};f \mapsto \left( \psi_{W}(f):{V}/{W} \rightarrow K;\mathbf{v} + W \mapsto f\left( \mathbf{v} \right) \right)
\end{align*}
しかも、その写像$\psi_{W}$は線形同型写像であり$W^{\bot} \cong \left( {V}/{W} \right)^{*}$が成り立つ。
\end{thm}
\begin{proof}
体$K$上の$n$次元vector空間$V$の部分空間$W$、双対空間$V^{*}$が与えられたとする。それらのvector空間たち$V$、$U$間のその部分空間$W$に関する商vector空間からの自然な線形写像を誘導する写像$\psi_{W}$は、定義より次式のようにおくと、
\begin{align*}
D\left( \psi_{W} \right) = \left\{ f \in L(V,K) \middle| W \subseteq \ker f \right\}
\end{align*}
次式のように与えられる。
\begin{align*}
\psi_{W}:D\left( \psi_{W} \right) \rightarrow L\left( {V}/{W},K \right);f \mapsto \left( \psi_{W}(f):{V}/{W} \rightarrow K;\mathbf{v} + W \mapsto f\left( \mathbf{v} \right) \right)
\end{align*}
そこで、双対空間の定義より$L\left( {V}/{W},K \right) = \left( {V}/{W} \right)^{*}$が成り立つので、次式のようになる。
\begin{align*}
\psi_{W}:D\left( \psi_{W} \right) \rightarrow \left( {V}/{W} \right)^{*};f \mapsto \left( \psi_{W}(f):{V}/{W} \rightarrow K;\mathbf{v} + W \mapsto f\left( \mathbf{v} \right) \right)
\end{align*}
さらに、$L(V,K) = V^{*}$が成り立つことに注意すれば、$\forall f \in V^{*}$に対し、$f \in D\left( \psi_{W} \right)$が成り立つなら、$\forall\mathbf{w} \in W$に対し、$\mathbf{w} \in \ker f$が成り立つことから、$f\left( \mathbf{w} \right) = 0$が成り立つ。ゆえに、$f \in W^{\bot}$が得られる。逆に、$f \in W^{\bot}$が成り立つなら、$\forall\mathbf{w} \in W$に対し、$f\left( \mathbf{w} \right) = 0$が成り立つことから、$\mathbf{w} \in \ker f$が成り立つので、$W \subseteq \ker f$が得られる。以上より、$f \in D\left( \psi_{W} \right)$が成り立つ。これにより、$D\left( \psi_{W} \right) = W^{\bot}$が得られたので、次式のようになる。
\begin{align*}
\psi_{W}:W^{\bot} \rightarrow \left( {V}/{W} \right)^{*};f \mapsto \left( \psi_{W}(f):{V}/{W} \rightarrow K;\mathbf{v} + W \mapsto f\left( \mathbf{v} \right) \right)
\end{align*}\par
定理\ref{2.4.4.7}よりその写像$\psi_{W}$は線形写像であることが分かるので、あとはこれが全単射であることを示せばよい。もちろん、$V\left( \psi_{W} \right) \subseteq \left( {V}/{W} \right)^{*}$が成り立つので、定理\ref{2.4.2.4}、定理\ref{2.4.2.12}より次のようになる。
\begin{align*}
\dim\left( {V}/{W} \right)^{*} &= \dim{V}/{W}\\
&= \dim V - \dim W\\
&= \dim V - \left( \dim V - \dim W^{\bot} \right)\\
&= \dim V - \dim V + \dim W^{\bot}\\
&= \dim W^{\bot}
\end{align*}
そこで、次元公式より次のようになる。
\begin{align*}
\dim\left( {V}/{W} \right)^{*} &= \dim W^{\bot}\\
&= \mathrm{rank} \psi_{W} + \mathrm{nullity} \psi_{W}
\end{align*}
そこで、$\ker\psi_{W} = \left\{ 0:V \rightarrow K;\mathbf{v} \mapsto 0 \right\}$が成り立つことが示されれば、$\mathrm{nullity} \psi_{W} = 0$となり、$\dim\left( {V}/{W} \right)^{*} = \mathrm{rank} \psi_{W}$が示される。$\forall f,g \in W^{\bot}$に対し、$f \neq g$が成り立つなら、$\exists\mathbf{v} \in V$に対し、$f\left( \mathbf{v} \right) \neq g\left( \mathbf{v} \right)$が成り立つので、$f\left( \mathbf{v} \right) = \psi_{W}(f)\left( \mathbf{v} + W \right)$かつ$g\left( \mathbf{v} \right) = \psi_{W}(g)\left( \mathbf{v} + W \right)$が成り立つことから、$\exists\mathbf{v} + W \in {V}/{W}$に対し、$\psi_{W}(f)\left( \mathbf{v} + W \right) \neq \psi_{W}(g)\left( \mathbf{v} + W \right)$が得られる。これにより、$\psi_{W}(f) \neq \psi_{W}(g)$が成り立つので、その線形写像$\psi_{W}$は単射であることが示された\footnote{実はこれとは別の方法があることに注意しよう。次式が成り立つことに注意すれば、
  \begin{align*}
\dim W^{\bot} = \dim V - \dim W = \dim{V}/{W} = \dim\left( {V}/{W} \right)^{*}
\end{align*}
  次の定理\ref{2.1.5.16}より明らかだと思えるであろう。
  \begin{quote}
  体$K$上の$n$次元vector空間たち$V$、$W$の基底の1つをそれぞれ$\alpha$、$\beta$とし、それらの基底たち$\alpha 、\beta$に関する線形写像$f:V \rightarrow W$の$[ f]^{\beta}_{\alpha} \in M_{nn}(K)$なる表現行列$[ f]^{\beta}_{\alpha}$を用いて写像$F_{\alpha \rightarrow \beta}$が次式のように定義されれば、
  \begin{align*}
F_{\alpha \rightarrow \beta}:L(V,W) \rightarrow M_{nn}(K);f \mapsto [ f]^{\beta}_{\alpha}
\end{align*}
  $\forall f \in L(V,W)$に対し、次のことは同値である。
  \begin{itemize}
  \item
    その写像$f$は線形同型写像である。
  \item 
    その写像$f$は全射$f:V \twoheadrightarrow W$である。
  \item
    その写像$f$は単射$f:V \rightarrowtail W$である。
  \item
    $n = \mathrm{rank}f = \dim{V(f)}$が成り立つ。
  \item
    $\mathrm{nullity}f = \dim{\ker f} = 0$が成り立つ。
  \item
    それらの基底たち$\alpha$、$\beta$に関する線形写像$f:V \rightarrow W$の表現行列$[ f]^{\beta}_{\alpha}$は正則行列である、即ち、$[ f]^{\beta}_{\alpha} \in {\mathrm{GL}}_{n}(K)$が成り立つ。
  \item
    その行列$F_{\alpha \rightarrow \beta}(f)$は正則行列である、即ち、$F_{\alpha \rightarrow \beta}(f) \in {\mathrm{GL}}_{n}(K)$が成り立つ。
  \end{itemize}
\end{quote}
}。定理\ref{2.1.3.17}より$\ker\psi_{W} = \left\{ 0:V \rightarrow K;\mathbf{v} \mapsto 0 \right\}$が成り立つことが分かり、したがって、$\mathrm{nullity} \psi_{W} = 0$が得られ、$\dim\left( {V}/{W} \right)^{*} = \mathrm{rank} \psi_{W}$が成り立つ。ゆえに、定理\ref{2.1.1.24}より$V\left( \psi_{W} \right) = \left( {V}/{W} \right)^{*}$が成り立つ。これにより、その線形写像$\psi_{W}$も全射であることが分かり、よって、その線形写像$\psi_{W}$は線形同型写像であり$W^{\bot} \cong \left( {V}/{W} \right)^{*}$が成り立つ。
\end{proof}
\begin{thm}\label{2.4.4.10}
体$K$上の$n$次元vector空間$V$の部分空間$W$、双対空間$V^{*}$が与えられたとする。次式のようなそれらのvector空間たち$V$、$U$間のその部分空間$W$に関する商vector空間からの自然な線形写像を誘導する写像$\psi_{W}$は定理\ref{2.4.4.9}より線形同型写像であった。
\begin{align*}
\psi_{W}:W^{\bot} \rightarrow \left( {V}/{W} \right)^{*};f \mapsto \left( \psi_{W}(f):{V}/{W} \rightarrow K;\mathbf{v} + W \mapsto f\left( \mathbf{v} \right) \right)
\end{align*}
このとき、その集合$V$からその集合${V}/{W}$への商写像$C_{\equiv \ \mathrm{mod}W}$を用いてこれの逆写像$\psi_{W}^{- 1}$は次のように与えられる。
\begin{align*}
\psi_{W}^{- 1}:\left( {V}/{W} \right)^{*} \rightarrow W^{\bot};\varLambda \mapsto \varLambda \circ C_{\equiv \ \mathrm{mod}W}
\end{align*}
\end{thm}
\begin{proof}
体$K$上の$n$次元vector空間$V$の部分空間$W$、双対空間$V^{*}$が与えられたとする。次式のようなそれらのvector空間たち$V$、$U$間のその部分空間$W$に関する商vector空間からの自然な線形写像を誘導する写像$\psi_{W}$は定理\ref{2.4.4.9}より線形同型写像であった。
\begin{align*}
\psi_{W}:W^{\bot} \rightarrow \left( {V}/{W} \right)^{*};f \mapsto \left( \psi_{W}(f):{V}/{W} \rightarrow K;\mathbf{v} + W \mapsto f\left( \mathbf{v} \right) \right)
\end{align*}
このとき、その集合$V$からその集合${V}/{W}$への商写像$C_{\equiv \ \mathrm{mod}W}$を用いて次式のように写像$\psi'$が定義されると、
\begin{align*}
\psi':\left( {V}/{W} \right)^{*} \rightarrow W^{\bot};\varLambda \mapsto \varLambda \circ C_{\equiv \ \mathrm{mod}W}
\end{align*}
$\forall\varLambda \in \left( {V}/{W} \right)^{*}$に対し、定理\ref{2.4.1.10}より次のようになるかつ、
\begin{align*}
\psi_{W} \circ \psi'(\varLambda) &= \psi_{W}\left( \psi'(\varLambda) \right)\\
&= \psi_{W}\left( \varLambda \circ C_{\equiv \ \mathrm{mod}W} \right)\\
&= \psi_{W}\left( \varLambda \circ C_{\equiv \ \mathrm{mod}W}:V \rightarrow K;\mathbf{v} \mapsto \varLambda \circ C_{\equiv \ \mathrm{mod}W}\left( \mathbf{v} \right) \right)\\
&= \psi_{W}\left( \varLambda \circ C_{\equiv \ \mathrm{mod}W}:V \rightarrow K;\mathbf{v} \mapsto \varLambda\left( \mathbf{v} + W \right) \right)\\
&= \psi_{W}\left( \varLambda \circ C_{\equiv \ \mathrm{mod}W} \right):{V}/{W} \rightarrow K;\mathbf{v} + W \mapsto \varLambda\left( \mathbf{v} + W \right)\\
&= \varLambda
\end{align*}
$\forall f \in W^{\bot}$に対し、次のようになる。
\begin{align*}
\psi' \circ \psi_{W}(f) &= \psi'\left( \psi_{W}(f) \right)\\
&= \psi'\left( \psi_{W}\left( f:V \rightarrow K;\mathbf{v} \mapsto f\left( \mathbf{v} \right) \right) \right)\\
&= \psi'\left( \psi_{W}(f):{V}/{W} \rightarrow K;\mathbf{v} + W \mapsto f\left( \mathbf{v} \right) \right)\\
&= \psi'\left( \psi_{W}(f) \right)\\
&= \psi_{W}(f) \circ C_{\equiv \ \mathrm{mod}W}\\
&= \psi_{W}(f) \circ C_{\equiv \ \mathrm{mod}W}:V \rightarrow K;\mathbf{v} \mapsto \psi_{W}(f) \circ C_{\equiv \ \mathrm{mod}W}\left( \mathbf{v} \right)\\
&= \psi_{W}(f) \circ C_{\equiv \ \mathrm{mod}W}:V \rightarrow K;\mathbf{v} \mapsto \psi_{W}(f)\left( \mathbf{v} + W \right)\\
&= \psi_{W}(f) \circ C_{\equiv \ \mathrm{mod}W}:V \rightarrow K;\mathbf{v} \mapsto f\left( \mathbf{v} \right)\\
&= f
\end{align*}\par
よって、$\psi_{W} \circ \psi' = I_{\left( {V}/{W} \right)^{*}}$かつ$\psi' \circ \psi_{W} = I_{W^{\bot}}$が成り立つので、$\psi_{W}^{- 1} = \psi'$が得られる。
\end{proof}
\begin{thm}\label{2.4.4.11}
体$K$上の$n$次元vector空間$V$の部分空間$W$、双対空間$V^{*}$が与えられたとき、${V^{*}}/{W^{\bot}} \cong W^{*}$が成り立つ。
\end{thm}
\begin{proof}
体$K$上の$n$次元vector空間$V$の部分空間$W$、双対空間$V^{*}$が与えられたとき、定理\ref{2.4.2.4}、定理\ref{2.4.2.13}、定理\ref{2.4.4.9}より次のようになる。
\begin{align*}
{V^{*}}/{W^{\bot}} \cong \left( {V^{*}}/{W^{\bot}} \right)^{*} \cong W^{\bot\bot} = W^{**} \cong W^{*}
\end{align*}
\end{proof}
%\hypertarget{ux53ccux5bfeux5199ux50cf}{%
\subsubsection{双対写像}%\label{ux53ccux5bfeux5199ux50cf}}
\begin{dfn}
体$K$上の$m$次元vector空間$V$、$n$次元vector空間$W$が与えられたとき、線形写像$\varphi:V \rightarrow W$に対し、次式のように写像$\varphi^{*}$が定義される。
\begin{align*}
\varphi^{*}:W^{*} \rightarrow V^{*};f \mapsto f \circ \varphi
\end{align*}
その写像$\varphi^{*}$をその線形写像$\varphi$の双対写像という。
\end{dfn}
\begin{thm}\label{2.4.4.12}
体$K$上の$m$次元vector空間$V$、$n$次元vector空間$W$が与えられたとき、線形写像$\varphi:V \rightarrow W$の双対写像$\varphi^{*}$は線形写像である。
\end{thm}
\begin{proof}
体$K$上の$m$次元vector空間$V$、$n$次元vector空間$W$が与えられたとき、線形写像$\varphi:V \rightarrow W$の双対写像$\varphi^{*}$について、$\forall k,l \in K\forall\mathbf{v} \in V\forall f,g \in W^{*}$に対し、次のようになる。
\begin{align*}
\varphi^{*}(kf + lg)\left( \mathbf{v} \right) &= (kf + lg) \circ \varphi\left( \mathbf{v} \right)\\
&= (kf + lg)\left( \varphi\left( \mathbf{v} \right) \right)\\
&= kf\left( \varphi\left( \mathbf{v} \right) \right) + lg\left( \varphi\left( \mathbf{v} \right) \right)\\
&= kf \circ \varphi\left( \mathbf{v} \right) + lg \circ \varphi\left( \mathbf{v} \right)\\
&= k\varphi^{*}(f)\left( \mathbf{v} \right) + l\varphi^{*}(g)\left( \mathbf{v} \right)\\
&= \left( k\varphi^{*}(f) + l\varphi^{*}(g) \right)\left( \mathbf{v} \right)
\end{align*}
よって、$\varphi^{*}(kf + lg) = k\varphi^{*}(f) + l\varphi^{*}(g)$が成り立つ。ゆえに、線形写像$\varphi:V \rightarrow W$の双対写像$\varphi^{*}$は線形写像である。
\end{proof}
\begin{thm}\label{2.4.4.13}
体$K$上の$m$次元vector空間$V$、$n$次元vector空間$W$、線形写像$\varphi:V \rightarrow W$の双対写像$\varphi^{*}$が与えられたとき、そのvector空間$V$とその双対空間$V^{*}$との双対性を表す内積$D_{V}$とそのvector空間$W$とその双対空間$W^{*}$との双対性を表す内積$D_{W}$について、$\forall\mathbf{v} \in V\forall f \in W^{*}$に対し、$D_{W}\left( \varphi\left( \mathbf{v} \right),f \right) = D_{V}\left( \mathbf{v},\varphi^{*}(f) \right)$が成り立つ。
\end{thm}
\begin{proof}
体$K$上の$m$次元vector空間$V$、$n$次元vector空間$W$、線形写像$\varphi:V \rightarrow W$の双対写像$\varphi^{*}$が与えられたとき、そのvector空間$V$とその双対空間$V^{*}$との双対性を表す内積$D_{V}$とそのvector空間$W$とその双対空間$W^{*}$との双対性を表す内積$D_{W}$について、$\forall\mathbf{v} \in V\forall f \in W^{*}$に対し、次のようになる。
\begin{align*}
D_{W}\left( \varphi\left( \mathbf{v} \right),f \right) = f\left( \varphi\left( \mathbf{v} \right) \right) = f \circ \varphi\left( \mathbf{v} \right) = \varphi^{*}(f)\left( \mathbf{v} \right) = D_{V}\left( \mathbf{v},\varphi^{*}(f) \right)
\end{align*}
\end{proof}
\begin{thm}\label{2.4.4.14}
体$K$上の$m$次元vector空間$V$、$n$次元vector空間$W$、線形写像$\varphi:V \rightarrow W$が与えられたとき、再双対空間での自然な線形同型写像$\rho:V \rightarrow V^{**};\mathbf{v} \mapsto \left( \widetilde{\mathbf{v}}:V^{*} \rightarrow K;f \mapsto f\left( \mathbf{v} \right) \right)$、$\sigma:W \rightarrow W^{**};\mathbf{w} \mapsto \left( \widetilde{\mathbf{w}}:W^{*} \rightarrow K;f \mapsto f\left( \mathbf{w} \right) \right)$を用いれば$\sigma^{- 1} \circ \varphi^{**} \circ \rho = \varphi$が成り立つ。
\end{thm}
\begin{proof}
体$K$上の$m$次元vector空間$V$、$n$次元vector空間$W$、線形写像$\varphi:V \rightarrow W$が与えられたとき、再双対空間での自然な線形同型写像たち$\rho:V \rightarrow V^{**};\mathbf{v} \mapsto \left( \widetilde{\mathbf{v}}:V^{*} \rightarrow K;f \mapsto f\left( \mathbf{v} \right) \right)$、$\sigma:W \rightarrow W^{**};\mathbf{w} \mapsto \left( \widetilde{\mathbf{w}}:W^{*} \rightarrow K;f \mapsto f\left( \mathbf{w} \right) \right)$を用いれば、$\forall\mathbf{v} \in V$に対し、次のようになる。
\begin{align*}
\sigma^{- 1} \circ \varphi^{**} \circ \rho\left( \mathbf{v} \right) &= \sigma^{- 1}\left( \varphi^{**}\left( \rho\left( \mathbf{v} \right) \right) \right)\\
&= \sigma^{- 1}\left( \varphi^{**}\left( \widetilde{\mathbf{v}}:V^{*} \rightarrow K;f \mapsto f\left( \mathbf{v} \right) \right) \right)\\
&= \sigma^{- 1}\left( \left( \widetilde{\mathbf{v}}:V^{*} \rightarrow K;f \mapsto f\left( \mathbf{v} \right) \right) \circ \varphi^{*} \right)\\
&= \sigma^{- 1}\left( \widetilde{\mathbf{v}} \circ \varphi^{*}:W^{*} \rightarrow K;f \mapsto \widetilde{\mathbf{v}} \circ \varphi^{*}(f) \right)\\
&= \sigma^{- 1}\left( \widetilde{\mathbf{v}} \circ \varphi^{*}:W^{*} \rightarrow K;f \mapsto \widetilde{\mathbf{v}} \circ (f \circ \varphi) \right)\\
&= \sigma^{- 1}\left( \widetilde{\mathbf{v}} \circ \varphi^{*}:W^{*} \rightarrow K;f \mapsto (f \circ \varphi)\left( \mathbf{v} \right) \right)\\
&= \sigma^{- 1}\left( \widetilde{\mathbf{v}} \circ \varphi^{*}:W^{*} \rightarrow K;f \mapsto f\left( \varphi\left( \mathbf{v} \right) \right) \right)\\
&= \sigma^{- 1}\left( \sigma\left( \varphi\left( \mathbf{v} \right) \right):W^{*} \rightarrow K;f \mapsto \sigma\left( \varphi\left( \mathbf{v} \right) \right)(f) \right)\\
&= \sigma^{- 1}\left( \sigma\left( \varphi\left( \mathbf{v} \right) \right) \right) = \sigma^{- 1} \circ \sigma \circ \varphi\left( \mathbf{v} \right) = \varphi\left( \mathbf{v} \right)
\end{align*}
よって、$\sigma^{- 1} \circ \varphi^{**} \circ \rho = \varphi$が成り立つ。
\end{proof}
\begin{thm}\label{2.4.4.15}
体$K$上の$m$次元vector空間$V$、$n$次元vector空間$W$、線形写像$\varphi:V \rightarrow W$が与えられたとき、次式が成り立つ。
\begin{align*}
\ker\varphi^{*} &= {V(\varphi)}^{\bot}\\
V\left( \varphi^{*} \right)^{\bot\bot} &= {\ker\varphi^{**}}^{\bot}
\end{align*}
\end{thm}
\begin{proof}
体$K$上の$m$次元vector空間$V$、$n$次元vector空間$W$、線形写像$\varphi:V \rightarrow W$が与えられたとき、$\forall f \in W^{*}$に対し、次のようになることから、
\begin{align*}
f \in {V(\varphi)}^{\bot} &\Leftrightarrow \forall\varphi\left( \mathbf{v} \right) \in V(\varphi)\left[ f\left( \varphi\left( \mathbf{v} \right) \right) = 0 \right]\\
&\Leftrightarrow \forall\mathbf{v} \in V\left[ f \circ \varphi\left( \mathbf{v} \right) = \varphi^{*}(f)\left( \mathbf{v} \right) = 0 \right]\\
&\Leftrightarrow \varphi^{*}(f) = 0\\
&\Leftrightarrow f \in \ker\varphi^{*}
\end{align*}
$\ker\varphi^{*} = {V(\varphi)}^{\bot}$が成り立つ。これにより、次式が得られる。
\begin{align*}
{V\left( \varphi^{*} \right)}^{\bot\bot} &= \left( {V\left( \varphi^{*} \right)}^{\bot} \right)^{\bot}\\
&= \left( \ker\left( \varphi^{*} \right)^{*} \right)^{\bot}\\
&= {\ker\varphi^{**}}^{\bot}
\end{align*}
\end{proof}
\begin{thm}\label{2.4.4.16}
体$K$上の$m$次元vector空間$V$、$n$次元vector空間$W$、線形写像$\varphi:V \rightarrow W$が与えられたとき、その線形写像$\varphi:V \rightarrow W$の双対写像$\varphi^{*}$について、$\mathrm{rank} \varphi^{*} = \mathrm{rank} \varphi $が成り立つ。
\end{thm}
\begin{proof}
体$K$上の$m$次元vector空間$V$、$n$次元vector空間$W$、線形写像$\varphi:V \rightarrow W$が与えられたとき、その線形写像$\varphi:V \rightarrow W$の双対写像$\varphi^{*}$について、次元公式と定理\ref{2.4.2.4}、定理\ref{2.4.2.12}、定理\ref{2.4.4.15}より次のようになる。
\begin{align*}
\mathrm{rank} \varphi^{*} &= \dim V^{*} - \mathrm{nullity} \varphi^{*}\\
&= \dim V^{*} - \dim{\ker \varphi^{*}}\\
&= \dim V^{*} - \dim{V(\varphi)}^{\bot}\\
&= \dim V^{*} - \left( \dim V - \dim{V(\varphi)} \right)\\
&= \dim V^{*} - \dim V + \dim{V(\varphi)}\\
&= \dim V - \dim V + \mathrm{rank} \varphi \\
&= \mathrm{rank} \varphi 
\end{align*}
\end{proof}
\begin{thm}\label{2.4.4.17}
体$K$上の$m$次元vector空間$V$、$n$次元vector空間$W$、線形写像$\varphi:V \rightarrow W$が与えられたとき、次のことが成り立つ。
\begin{itemize}
\item
  その線形写像$\varphi$が全射であるならそのときに限り、その双対写像$\varphi^{*}$は単射である。
\item
  その線形写像$\varphi$が単射であるならそのときに限り、その双対写像$\varphi^{*}$は全射である。
\end{itemize}
\end{thm}
\begin{proof}
体$K$上の$m$次元vector空間$V$、$n$次元vector空間$W$、線形写像$\varphi:V \rightarrow W$が与えられたとき、その線形写像$\varphi$が全射であるなら、$V(\varphi) = W$が成り立つので、定理\ref{2.4.2.4}、定理\ref{2.4.4.16}より次のようになる。
\begin{align*}
\mathrm{rank} \varphi^{*}  &= \mathrm{rank} \varphi \\
&= \dim W\\
&= \dim W^{*}
\end{align*}
次元公式より$\mathrm{nullity} \varphi^{*} = 0$が得られるので、$\ker\varphi^{*} = \left\{ 0:W \rightarrow K;\mathbf{w} \mapsto 0 \right\}$が成り立つ。定理\ref{2.1.3.17}よりその双対写像$\varphi^{*}$は単射である。\par
逆に、その双対写像$\varphi^{*}$が単射であるなら、定理\ref{2.1.2.12}より$\ker\varphi^{*} = \left\{ 0:W \rightarrow K;\mathbf{w} \mapsto 0 \right\}$が成り立つ。$V(\varphi) \subseteq W$が成り立つのは明らかであるので、次元公式と定理\ref{2.4.2.4}、定理\ref{2.4.4.16}より次のようになる。
\begin{align*}
\mathrm{rank} \varphi &= \mathrm{rank} \varphi^{*} \\
&= \dim W^{*} - \mathrm{nullity} \varphi^{*} \\
&= \dim W^{*}\\
&= \dim W
\end{align*}
したがって、定理\ref{2.1.1.22}より$V(\varphi) = W$が成り立つので、その線形写像$\varphi$は全射である。\par
また、その線形写像$\varphi$が単射であるなら、定理\ref{2.1.2.12}より$\ker\varphi = \left\{ \mathbf{0} \right\}$が成り立つ。そこで、$V\left( \varphi^{*} \right) \subseteq V^{*}$が成り立つのは明らかであるので、次元公式と定理\ref{2.4.2.4}、定理\ref{2.4.4.16}より次のようになる。
\begin{align*}
\mathrm{rank} \varphi^{*} &= \mathrm{rank} \varphi \\
&= \dim V - \mathrm{nullity} \varphi \\
&= \dim V\\
&= \dim V^{*}
\end{align*}
したがって、定理\ref{2.1.1.22}より$V\left( \varphi^{*} \right) = V^{*}$が成り立つので、その双対写像$\varphi^{*}$は全射である。\par
逆に、その線形写像$\varphi^{*}$が全射であるなら、$V\left( \varphi^{*} \right) = V^{*}$が成り立つので、定理\ref{2.4.2.4}、定理\ref{2.4.4.16}より次のようになる。
\begin{align*}
\dim V = \dim V^{*} = \mathrm{rank} \varphi^{*} = \mathrm{rank} \varphi 
\end{align*}
次元公式より$\mathrm{nullity} \varphi = 0$が得られるので、$\ker\varphi = \left\{ \mathbf{0} \right\}$が成り立つ。定理\ref{2.1.2.12}よりその線形写像$\varphi$は単射である。
\end{proof}
\begin{thm}\label{2.4.4.18}
体$K$上の$m$次元vector空間$V$、$n$次元vector空間$W$、線形写像$\varphi:V \rightarrow W$が与えられたとき、次のことが成り立つ。
\begin{itemize}
\item
  そのvector空間$V$の部分空間$U$が与えられたとき、${V\left( \varphi|U \right)}^{\bot} = V\left( {\varphi^{*}}^{- 1}|U^{\bot} \right)$が成り立つ。
\item
  そのvector空間$W$の部分空間$U$が与えられたとき、${V\left( {\varphi^{**}}^{- 1}|U^{\bot\bot} \right)}^{\bot} = V\left( \varphi^{*}|U^{\bot} \right)^{\bot\bot}$が成り立つ。
\end{itemize}
\end{thm}
\begin{proof}
体$K$上の$m$次元vector空間$V$、$n$次元vector空間$W$、線形写像$\varphi:V \rightarrow W$が与えられたとき、$\forall f \in W^{*}$に対し、次のようになる。
\begin{align*}
f \in {V\left( \varphi|U \right)}^{\bot} &\Leftrightarrow \forall\varphi\left( \mathbf{u} \right) \in V\left( \varphi|U \right)\left[ f\left( \varphi\left( \mathbf{u} \right) \right) = f \circ \varphi\left( \mathbf{u} \right) = 0 \right]\\
&\Leftrightarrow \forall\mathbf{u} \in U\left[ \varphi^{*}(f)\left( \mathbf{u} \right) = 0 \right]\\
&\Leftrightarrow \varphi^{*}(f) \in U^{\bot} \Leftrightarrow \left\{ \varphi^{*}(f) \right\} \subseteq U^{\bot}\\
&\Rightarrow f \in V\left( {\varphi^{*}}^{- 1}|\left\{ \varphi^{*}(f) \right\} \right) \subseteq V\left( {\varphi^{*}}^{- 1}|U^{\bot} \right)\\
&\Rightarrow f \in V\left( {\varphi^{*}}^{- 1}|U^{\bot} \right)
\end{align*}\par
逆に、$f \in V\left( {\varphi^{*}}^{- 1}|U^{\bot} \right)$が成り立つなら、$\varphi^{*}(f) \in V\left( \varphi^{*}|V\left( {\varphi^{*}}^{- 1}|U^{\bot} \right) \right) \subseteq U^{\bot}$、即ち、$\left\{ \varphi^{*}(f) \right\} \subseteq U^{\bot}$が成り立つので、以上の議論により、$\forall f \in W^{*}$に対し、次のようになる。
\begin{align*}
f \in {V\left( \varphi|U \right)}^{\bot} &\Leftrightarrow \left\{ \varphi^{*}(f) \right\} \subseteq U^{\bot}\\
&\Leftrightarrow f \in V\left( {\varphi^{*}}^{- 1}|U^{\bot} \right)
\end{align*}
よって、${V\left( \varphi|U \right)}^{\bot} = V\left( {\varphi^{*}}^{- 1}|U^{\bot} \right)$が成り立つ。\par
一方で、上記の議論により直ちに${V\left( {\varphi^{**}}^{- 1}|U^{\bot\bot} \right)}^{\bot} = V\left( \varphi^{*}|U^{\bot} \right)^{\bot\bot}$が得られる。
\end{proof}
\begin{thebibliography}{50}
  \bibitem{1}
  松坂和夫, 代数系入門, 岩波書店, 1976. 新装版第1刷 p182-191 ISBN978-4-00-029873-5
  \bibitem{2}
  佐武一郎, 線型代数学, 裳華房, 1958. 第53版 p193-197 ISBN4-7853-1301-3
\end{thebibliography}
\end{document}
