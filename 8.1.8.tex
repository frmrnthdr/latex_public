\documentclass[dvipdfmx]{jsarticle}
\setcounter{section}{1}
\setcounter{subsection}{7}
\usepackage{xr}
\externaldocument{8.1.1}
\externaldocument{8.1.2}
\externaldocument{8.1.6}
\usepackage{amsmath,amsfonts,amssymb,array,comment,mathtools,url,docmute}
\usepackage{longtable,booktabs,dcolumn,tabularx,mathtools,multirow,colortbl,xcolor}
\usepackage[dvipdfmx]{graphics}
\usepackage{bmpsize}
\usepackage{amsthm}
\usepackage{enumitem}
\setlistdepth{20}
\renewlist{itemize}{itemize}{20}
\setlist[itemize]{label=•}
\renewlist{enumerate}{enumerate}{20}
\setlist[enumerate]{label=\arabic*.}
\setcounter{MaxMatrixCols}{20}
\setcounter{tocdepth}{3}
\newcommand{\rotin}{\text{\rotatebox[origin=c]{90}{$\in $}}}
\renewcommand{\thesection}{第\arabic{section}部}
\renewcommand{\thesubsection}{\arabic{section}.\arabic{subsection}}
\renewcommand{\thesubsubsection}{\arabic{section}.\arabic{subsection}.\arabic{subsubsection}}
\everymath{\displaystyle}
\allowdisplaybreaks[4]
\usepackage{vtable}
\theoremstyle{definition}
\newtheorem{thm}{定理}[subsection]
\newtheorem*{thm*}{定理}
\newtheorem{dfn}{定義}[subsection]
\newtheorem*{dfn*}{定義}
\newtheorem{axs}[dfn]{公理}
\newtheorem*{axs*}{公理}
\renewcommand{\headfont}{\bfseries}
\makeatletter
  \renewcommand{\section}{%
    \@startsection{section}{1}{\z@}%
    {\Cvs}{\Cvs}%
    {\normalfont\huge\headfont\raggedright}}
\makeatother
\makeatletter
  \renewcommand{\subsection}{%
    \@startsection{subsection}{2}{\z@}%
    {0.5\Cvs}{0.5\Cvs}%
    {\normalfont\LARGE\headfont\raggedright}}
\makeatother
\makeatletter
  \renewcommand{\subsubsection}{%
    \@startsection{subsubsection}{3}{\z@}%
    {0.4\Cvs}{0.4\Cvs}%
    {\normalfont\Large\headfont\raggedright}}
\makeatother
\makeatletter
\renewenvironment{proof}[1][\proofname]{\par
  \pushQED{\qed}%
  \normalfont \topsep6\p@\@plus6\p@\relax
  \trivlist
  \item\relax
  {
  #1\@addpunct{.}}\hspace\labelsep\ignorespaces
}{%
  \popQED\endtrivlist\@endpefalse
}
\makeatother
\renewcommand{\proofname}{\textbf{証明}}
\usepackage{tikz,graphics}
\usepackage[dvipdfmx]{hyperref}
\usepackage{pxjahyper}
\hypersetup{
 setpagesize=false,
 bookmarks=true,
 bookmarksdepth=tocdepth,
 bookmarksnumbered=true,
 colorlinks=false,
 pdftitle={},
 pdfsubject={},
 pdfauthor={},
 pdfkeywords={}}
\begin{document}
%\hypertarget{filter}{%
\subsection{filter}%\label{filter}}
%\hypertarget{filter-1}{%
\subsubsection{filter}%\label{filter-1}}
\begin{dfn}
空集合でない集合$S$の部分集合系$\mathfrak{F}$が次のことを満たすとき、その集合$\mathfrak{F}$をその集合$S$上のfilterという。
\begin{itemize}
\item
  $\emptyset \notin \mathfrak{F}$が成り立つ。
\item
  $\forall F \in \mathfrak{F}\forall G \in \mathfrak{P}(S)$に対し、$F \subseteq G \subseteq S$が成り立つなら、$G \in \mathfrak{F}$が成り立つ。
\item
  $\forall F,G \in \mathfrak{F}$に対し、$F \cap G \in \mathfrak{F}$が成り立つ。
\end{itemize}
\end{dfn}
\begin{thm}\label{8.1.8.1}
集合$S$上のfilter$\mathfrak{F}$が与えられたとき、$S \in \mathfrak{F}$が成り立つ。
\end{thm}
\begin{proof}
集合$S$上のfilter$\mathfrak{F}$が与えられたとき、$\forall F \in \mathfrak{F}$に対し、$F \subseteq S \subseteq S$が成り立つなら、$S \in \mathfrak{F}$が成り立つことにより自明である。
\end{proof}
\begin{thm}\label{8.1.8.2}
集合$S$上のfilter$\mathfrak{F}$が与えられたとき、$\forall F \in \mathfrak{P}(S)$に対し、$F \in \mathfrak{F}$かつ$S \setminus F \in \mathfrak{F}$が成り立つことはない。
\end{thm}
\begin{proof}
その集合$S$上のfilter$\mathfrak{F}$が与えられたとき、$\exists F \in \mathfrak{P}(S)$に対し、$F \in \mathfrak{F}$かつ$S \setminus F \in \mathfrak{F}$が成り立つと仮定すると、filterの定義より$F \cap S \setminus F \in \mathfrak{F}$が成り立ち、したがって、$\emptyset \in \mathfrak{F}$が成り立つことになるが、これはfilterの定義に矛盾する。
\end{proof}
%\hypertarget{filterux306eux53ceux675f}{%
\subsubsection{filterの収束}%\label{filterux306eux53ceux675f}}
\begin{dfn}\label{filterの収束}
位相空間$\left( S,\mathfrak{O} \right)$とその集合$S$上のfilter$\mathfrak{F}$が与えられたとき、その集合$S$の元$a$の全近傍系$\mathbf{V}(a)$が$\mathbf{V}(a)\subseteq \mathfrak{F}$を満たすとき、そのfilter$\mathfrak{F}$はその元$a$に収束するといい、$\mathfrak{F} \rightarrow a$などと書く。
\end{dfn}
\begin{thm}\label{8.1.8.3}
位相空間$\left( S,\mathfrak{O} \right)$が与えられたとき、$\forall M \in \mathfrak{P}(S)\forall a \in S$に対し、次のことが成り立つ。
\begin{itemize}
\item
  その元$a$がその集合$M$の内点である、即ち、$a \in {\mathrm{int}}M$が成り立つならそのときに限り、$\mathfrak{F} \rightarrow a$なるその集合$S$上の任意のfilter$\mathfrak{F}$に対し、$M \in \mathfrak{F}$が成り立つ。
\item
  その元$a$がその集合$M$の触点である、即ち、$a \in {\mathrm{cl}}M$が成り立つならそのときに限り、$\mathfrak{F} \rightarrow a$なるその集合$S$上のあるfilter$\mathfrak{F}$が存在して、$M \in \mathfrak{F}$が成り立つ。
\end{itemize}
\end{thm}
\begin{proof}
位相空間$\left( S,\mathfrak{O} \right)$が与えられたとき、$\forall M \in \mathfrak{P}(S)\forall a \in S$に対し、その元$a$がその集合$M$の内点である、即ち、$a \in {\mathrm{int}}M$が成り立つなら、その元$a$の全近傍系$\mathbf{V}(a)$を用いて$M \in \mathbf{V}(a)$が成り立つ。ここで、$\mathfrak{F} \rightarrow a$なるその集合$S$上の任意のfilter$\mathfrak{F}$に対し、$\mathbf{V}(a)\subseteq \mathfrak{F}$が成り立つので、$M \in \mathfrak{F}$が成り立つ。逆に、$\mathfrak{F} \rightarrow a$なるその集合$S$上の任意のfilter$\mathfrak{F}$に対し、$M \in \mathfrak{F}$が成り立つなら、その全近傍系$\mathbf{V}(a)$について、定理\ref{8.1.1.24}より次のことが成り立つので、
\begin{itemize}
\item
  $\emptyset \notin \mathbf{V}(a)$が成り立つ。
\item
  $\forall V \in \mathbf{V}(a)\forall W \in \mathfrak{P}(S)$に対し、$V \subseteq W \subseteq S$が成り立つなら、$W \in \mathbf{V}(a)$が成り立つ。
\item
  $\forall V,W \in \mathbf{V}(a)$に対し、$V \cap W \in \mathbf{V}(a)$が成り立つ。
\item
  $\mathbf{V}(a) \subseteq \mathbf{V}(a)$が成り立つ。
\end{itemize}
その全近傍系$\mathbf{V}(a)$は$\mathbf{V}(a) \rightarrow a$なるその集合$S$上のfilterでもある。したがって、$M \in \mathbf{V}(a)$が成り立つので、$a \in {\mathrm{int}}M$が成り立ち、よって、その元$a$がその集合$M$の内点である。\par
その元$a$がその集合$M$の触点である、即ち、$a \in {\mathrm{cl}}M$が成り立つなら、その元$a$の全近傍系$\mathbf{V}(a)$を用いて、$\exists V \in \mathbf{V}(a)$に対し、$M \cap V \subseteq F$が成り立つような集合$F$全体の集合$\mathfrak{F}$について、もちろん$\emptyset \notin \mathfrak{F}$が成り立つ。さらに、$\forall F \in \mathfrak{F}\forall G \in \mathfrak{P}(S)$に対し、$F \subseteq G \subseteq S$が成り立つなら、$\exists V \in \mathbf{V}(a)$に対し、$M \cap V \subseteq F \subseteq G$が成り立つので、$G \in \mathfrak{F}$が成り立つ。$\forall F,G \in \mathfrak{F}$に対し、その元$a$の近傍たち$V$、$W$が存在して、$M \cap V \subseteq F$かつ$M \cap W \subseteq G$が成り立つので、$M \cap (V \cap W) \subseteq F \cap G$が成り立ち、定理\ref{8.1.1.24}より$V \cap W \in \mathbf{V}(a)$が成り立つので、$F \cap G \in \mathfrak{F}$が成り立つ。$\forall V \in \mathbf{V}(a)$に対し、明らかに$M \cap V \subseteq V$が成り立つので、$V \in \mathfrak{F}$が成り立つ、即ち、$\mathbf{V}(a)\subseteq \mathfrak{F}$が成り立つ。以上より、次のことが成り立つので、
\begin{itemize}
\item
  $\emptyset \notin \mathfrak{F}$が成り立つ。
\item
  $\forall F \in \mathfrak{F}\forall G \in \mathfrak{P}(S)$に対し、$F \subseteq G \subseteq S$が成り立つなら、$G \in \mathfrak{F}$が成り立つ。
\item
  $\forall F,G \in \mathfrak{F}$に対し、$F \cap G \in \mathfrak{F}$が成り立つ。
\item
  $\mathbf{V}(a)\subseteq \mathfrak{F}$が成り立つ。
\end{itemize}
その集合$\mathfrak{F}$は$\mathfrak{F} \rightarrow a$なるその集合$S$上のfilterでもある。したがって、$\mathfrak{F} \rightarrow a$なるその集合$S$上のあるfilter$\mathfrak{F}$が存在して$M \in \mathfrak{F}$が成り立つ。逆に、$\mathfrak{F} \rightarrow a$なるその集合$S$上のあるfilter$\mathfrak{F}$が存在して$M \in \mathfrak{F}$が成り立つなら、$\mathbf{V}(a)\subseteq \mathfrak{F}$が成り立つので、$\forall V \in \mathbf{V}(a)$に対し、$V \in \mathfrak{F}$が成り立ち、filterの定義より$V \cap M \in \mathfrak{F}$が成り立つかつ、$\emptyset \notin \mathfrak{F}$が成り立たないので、$V \cap M \neq \emptyset$が成り立つ。このことはその元$a$の基本近傍系の元に対しても同じようなことがいえるので、定理\ref{8.1.2.17}よりその元$a$がその集合$M$の触点である、即ち、$a \in {\mathrm{cl}}(M)$が成り立つ。
\end{proof}
\begin{thm}\label{8.1.8.4}
位相空間$\left( S,\mathfrak{O} \right)$が与えられたとき、その集合$S$上の任意のfilter$\mathfrak{F}$に対し、$\mathfrak{F} \rightarrow a$が成り立つなら、$\forall F \in \mathfrak{F}$に対し、$a \in {\mathrm{cl}}F$が成り立つ。
\end{thm}
\begin{proof}
位相空間$\left( S,\mathfrak{O} \right)$が与えられたとき、その集合$S$上の任意のfilter$\mathfrak{F}$に対し、$\mathfrak{F} \rightarrow a$が成り立つなら、その集合$S$の元$a$の全近傍系$\mathbf{V}(a)$が$\mathbf{V}(a)\subseteq \mathfrak{F}$を満たす。したがって、$\forall V \in \mathbf{V}(a)$に対し、$V \in \mathfrak{F}$が成り立つので、$\forall F \in \mathfrak{F}$に対し、filterの定義より$V \cup F \neq \emptyset$が成り立つ。ここで、基本近傍系の定義に注意すれば、定理\ref{8.1.2.15}より$a \in {\mathrm{cl}}M$が成り立つ。
\end{proof}
%\hypertarget{ux6975ux5927filter}{%
\subsubsection{極大filter}%\label{ux6975ux5927filter}}
\begin{dfn*}[定義\ref{有限交叉性}の再掲]
集合$S$の部分集合系$\mathfrak{X}$が与えられたとき、これの任意の空でない有限集合である部分集合に属する集合同士の共通部分が空でないとき、即ち、$\forall\mathfrak{X}'\in \mathfrak{P}\left( \mathfrak{X} \right)$に対し、$0 < {\#}\mathfrak{X}' < \aleph_{0}$が成り立つなら、$\bigcap_{} \mathfrak{X}' \neq \emptyset$が成り立つとき、その集合$\mathfrak{X}$は有限交叉性を持つという。
\end{dfn*}
\begin{dfn}
集合$S$上のfilter$\mathfrak{F}$が与えられたとき、その集合$S$の任意のfilter$\mathfrak{F}'$に対し、$\mathfrak{F \subseteq}\mathfrak{F}'$が成り立つなら、$\mathfrak{F} = \mathfrak{F}'$が成り立つとき、そのfilter$\mathfrak{F}$をその集合$S$上の極大filterという。以下、その集合$S$上の極大filter全体の集合を$\beta(S)$とおく。
\end{dfn}
\begin{thm}\label{8.1.8.5}
集合$S$上のfilter$\mathfrak{F}$が与えられたとき、次のことは同値である。
\begin{itemize}
\item
  そのfilter$\mathfrak{F}$はその集合$S$上の極大filterである、即ち、$\mathfrak{F}\in \beta(S)$が成り立つ。
\item
  その集合$S$の任意のfilter$\mathfrak{F}'$に対し、$\mathfrak{F \subseteq}\mathfrak{F}'$が成り立つなら、$\mathfrak{F} = \mathfrak{F}'$が成り立つ。
\item
  $\forall F \in \mathfrak{P}(S)$に対し、$F \in \mathfrak{F}$または$S \setminus F \in \mathfrak{F}$が成り立つ。
\end{itemize}
\end{thm}
\begin{proof}
集合$S$上のfilter$\mathfrak{F}$が与えられたとき、定義よりそのfilter$\mathfrak{F}$はその集合$S$上の極大filterであるならそのときに限り、その集合$S$の任意のfilter$\mathfrak{F}'$に対し、$\mathfrak{F \subseteq}\mathfrak{F}'$が成り立つなら、$\mathfrak{F} = \mathfrak{F}'$が成り立つ。\par
一方で、その集合$S$の任意のfilter$\mathfrak{F}'$に対し、$\mathfrak{F \subseteq}\mathfrak{F}'$が成り立つなら、$\mathfrak{F} = \mathfrak{F}'$が成り立つとき、$S \in \mathfrak{F}$が成り立つので、$\forall F \in \mathfrak{P}(S)$に対し、次式のように集合$\mathfrak{X}$が定義されれば、
\begin{align*}
\mathfrak{X} =\left\{ X \in \mathfrak{P}(S) \middle| F \cup X \in \mathfrak{F} \right\}
\end{align*}
その集合$\mathfrak{X}$が有限交叉性をもつとき、$\emptyset \in \mathfrak{X}$が成り立つとすれば、その集合$\left\{ \emptyset \right\}$もその集合$\mathfrak{X}$の空でない有限集合である部分集合で$\bigcap_{} \left\{ \emptyset \right\} = \emptyset$が成り立つが、これはその集合$\mathfrak{X}$が有限交叉性をもつことに矛盾する。したがって、$\emptyset \notin \mathfrak{X}$が成り立つ。$\forall X \in \mathfrak{X\forall}Y \in \mathfrak{P}(S)$に対し、$X \subseteq Y \subseteq S$が成り立つなら、$F \cup X \in \mathfrak{F}$かつ$F \cup X \subseteq F \cup Y \subseteq S$が成り立つので、filterの定義より$F \cup Y \in \mathfrak{F}$が成り立つ。したがって、$Y \in \mathfrak{X}$が成り立つ。最後に、$\forall X,Y \in \mathfrak{X}$に対し、$F \cup X,F \cup Y \in \mathfrak{F}$が成り立つので、$(F \cup X) \cap (F \cup Y) = F \cup (X \cap Y)\in \mathfrak{F}$が成り立つ。したがって、$X \cap Y \in \mathfrak{X}$が成り立つ。以上より、その集合$\mathfrak{X}$はその集合$S$上のfilterである。ここで、$\forall G \in \mathfrak{F}$に対し、$G \subseteq F \cup G \subseteq S$が成り立つので、filterの定義より$F \cup G \in \mathfrak{F}$が成り立つ。ゆえに、$G \in \mathfrak{X}$が得られ、したがって、$\mathfrak{F \subseteq X}$が成り立つ。ここで、仮定よりそのfilter$\mathfrak{F}$は$\mathfrak{F = X}$を満たす。このとき、$S \setminus F \cup F = S \in \mathfrak{F}$が成り立つので、$S \setminus F \in \mathfrak{X = F}$が得られる。\par
その集合$\mathfrak{X}$が有限交叉性をもたないとき、これの空でない有限集合であるある部分集合$\mathfrak{X}'$が存在して、これに属する集合同士の共通部分$\bigcap_{} \mathfrak{X}'$が空である。このとき、$\forall X \in \mathfrak{X}'$に対し、$F \cup X \in \mathfrak{F}$が成り立つので、filterの定義より次のようになる。
\begin{align*}
F &= F \cup \emptyset\\
&= F \cup \bigcap_{} \mathfrak{X}'\\
&= F \cup \bigcap_{X \in \mathfrak{X}'} X\\
&= \bigcap_{X \in \mathfrak{X}'} (F \cup X)\in \mathfrak{F}
\end{align*}\par
以上より、その集合$S$の任意のfilter$\mathfrak{F}'$に対し、$\mathfrak{F \subseteq}\mathfrak{F}'$が成り立つなら、$\mathfrak{F} = \mathfrak{F}'$が成り立つなら、$\forall F \in \mathfrak{P}(S)$に対し、$F \in \mathfrak{F}$または$S \setminus F \in \mathfrak{F}$が成り立つことが示された。\par
$\forall F \in \mathfrak{P}(S)$に対し、$F \in \mathfrak{F}$または$S \setminus F \in \mathfrak{F}$が成り立つとき、その集合$S$のあるfilter$\mathfrak{F}'$が存在して、$\mathfrak{F \subset}\mathfrak{F}'$が成り立つと仮定すると、あるそのfilter$\mathfrak{F}'$の元$F$が存在して、$F \notin \mathfrak{F}$が成り立つ。このとき、仮定より$S \setminus F \in \mathfrak{F \subset}\mathfrak{F}'$が成り立つことになるが、次のようになることにより、
\begin{align*}
F \cap S \setminus F = \emptyset \in \mathfrak{F}'
\end{align*}
その集合$\mathfrak{F}'$がfilterであることに矛盾する。よって、その集合$S$の任意のfilter$\mathfrak{F}'$に対し、$\mathfrak{F \subseteq}\mathfrak{F}'$が成り立つなら、$\mathfrak{F} = \mathfrak{F}'$が成り立つ。
\end{proof}
\begin{thm}\label{8.1.8.6}
集合$S$の部分集合系$\mathfrak{X}$が与えられたとき、これが有限交叉性をもつなら、これを含むその集合$S$上の極大filter$\mathfrak{F}$が存在する。
\end{thm}
\begin{proof}
集合$S$の部分集合系$\mathfrak{X}$が与えられたとき、これが有限交叉性をもつとする。このとき、その部分集合系$\mathfrak{X}$の空でない有限集合である部分集合$\mathfrak{X}'$が存在して、この共通部分$\bigcap_{} \mathfrak{X}'$を含むようなその集合$S$の部分集合全体の集合$\mathfrak{F}$が考えられれば、$\forall X \in \mathfrak{X}$に対し、$\left\{ X \right\}\subseteq \mathfrak{X}$かつ$\bigcap_{} \left\{ X \right\} = X \subseteq X$かつ$X \subseteq S$が成り立つので、$X \in \mathfrak{F}$が成り立つ。ゆえに、$\mathfrak{X \subseteq F}$が成り立つ。\par
さらに、$\emptyset \in \mathfrak{F}$が成り立つとすれば、その部分集合系$\mathfrak{X}$の空でない有限な部分集合$\mathfrak{X}'$が存在して、$\bigcap_{} \mathfrak{X}' \subseteq \emptyset$が成り立つので、$\bigcap_{} \mathfrak{X}' = \emptyset$も成り立つ。しかしながら、これはその部分集合系$\mathfrak{X}$が有限交叉性をもつことに矛盾するので、$\emptyset \notin \mathfrak{F}$が成り立つ。$\forall F \in \mathfrak{F}\forall G \in \mathfrak{P}(S)$に対し、$F \subseteq G \subseteq S$が成り立つなら、その部分集合系$\mathfrak{X}$の空でない有限な部分集合$\mathfrak{X}'$が存在して、$\bigcap_{} \mathfrak{X}' \subseteq F$が成り立つので、$\bigcap_{} \mathfrak{X}' \subseteq G$が成り立つ。したがって、$G \in \mathfrak{F}$が成り立つ。$\forall F,G \in \mathfrak{F}$に対し、その部分集合系$\mathfrak{X}$の空でない有限な部分集合たち$\mathfrak{X}'$、$\mathfrak{Y}'$が存在して、$\bigcap_{} \mathfrak{X}' \subseteq F$かつ$\bigcap_{} \mathfrak{Y}' \subseteq G$が成り立つ。このとき、その和集合$\mathfrak{X}' \cup \mathfrak{Y}'$はその部分集合系$\mathfrak{X}$の空でない有限な部分集合であり、次のようになるので、
\begin{align*}
\bigcap_{} \left( \mathfrak{X}' \cup \mathfrak{Y}' \right) &= \bigcap_{X \in \mathfrak{X}' \cup \mathfrak{Y}'} X\\
&= \bigcap_{X \in \mathfrak{X}'} X \cap \bigcap_{Y \in \mathfrak{Y}'} Y\\
&= \bigcap_{} \mathfrak{X}' \cap \bigcap_{} \mathfrak{Y}'\\
&\subseteq F \cap G
\end{align*}
$F \cap G \in \mathfrak{F}$が成り立つ。以上より、その集合$\mathfrak{F}$はその部分集合系$\mathfrak{X}$を含むその集合$S$のfilterとなる。\par
そこで、その部分集合系$\mathfrak{X}$を含むその集合$S$のfilter全体の集合$\varphi$を用いて順序集合$(\varphi, \subseteq )$が考えられれば、その集合$\varphi$の空でない全順序集合となるような部分集合が存在する。実際、$\forall\mathfrak{F}' \in \varphi$に対し、その集合$\left\{ \mathfrak{F}' \right\}$がその集合$\varphi$の空でない全順序集合となるような部分集合である。\par
その集合$\varphi$の空でない全順序集合となるような任意の部分集合$\varphi'$が与えられたとき、$\forall\mathfrak{F}' \in \varphi'$に対し、もちろん、$\mathfrak{X \subseteq}\mathfrak{F}' \subseteq \bigcup_{} \varphi'$が成り立つ。さらに、$\forall\mathfrak{F}' \in \varphi'$に対し、$\emptyset \notin \mathfrak{F}'$が成り立つので、$\emptyset \notin \bigcup_{} \varphi'$が成り立つ。$\forall F \in \bigcup_{} \varphi'\forall G \in \mathfrak{P}(S)$に対し、$F \subseteq G \subseteq S$が成り立つなら、$\exists\mathfrak{F}' \in \varphi'$に対し、$F \in \mathfrak{F}'$が成り立つので、filterの定義より$B \in \mathfrak{F}'$が成り立ち、したがって、$G \in \bigcup_{} \varphi$が成り立つ。$\forall F,G \in \bigcup_{} \varphi'$に対し、あるfilters$\mathfrak{F}'$、$\mathfrak{G}'$がその集合$\varphi'$に存在して、$F \in \mathfrak{F}'$かつ$G \in \mathfrak{G}'$が成り立つ。ここで、その集合$\varphi'$が全順序集合なので、$\mathfrak{F}' \subseteq \mathfrak{G}'$または$\mathfrak{G}' \subseteq \mathfrak{F}'$が成り立つ。ここで、$\mathfrak{G}' \subseteq \mathfrak{F}'$が成り立つとしてもよく、したがって、$F,G \in \mathfrak{F}'$が成り立つ。これにより、$F \cap G \in \mathfrak{F}' \subseteq \bigcup_{} \varphi'$が成り立つ。以上より、その集合$\bigcup_{} \varphi'$はその部分集合系$\mathfrak{X}$を含むその集合$S$上のfilterである。さらに、この集合$\bigcup_{} \varphi'$はその集合$\varphi'$の上界であるから、その集合$\varphi'$は上に有界である。\par
ここで、いっそうよいZornの補題よりその順序集合$(\varphi, \subseteq )$は極大元をもつ、即ち、$\mathfrak{\exists F \in}\varphi\forall\mathfrak{F}' \in \varphi$に対し、$\mathfrak{F \subseteq}\mathfrak{F}'$が成り立つなら、$\mathfrak{F} = \mathfrak{F}'$が成り立つ\footnote{次のような定理です。
\begin{quote}
  順序集合$(A,O)$の任意の全順序な部分順序集合$\left( A',O \right)$が上に有界なら、その集合Aの極大元が存在する。
\end{quote}}。これはその集合$\mathfrak{X}$を含むその集合$S$上の極大filter$\mathfrak{F}$である。
\end{proof}
\begin{thm}\label{8.1.8.7}
集合$S$上の極大filter$\mathfrak{F}$が与えられたとき、$\forall F\in \mathfrak{F}$に対し、$F \cap M \neq \emptyset$なるその集合$S$の部分集合$M$が与えられたらば、$M \in \mathfrak{F}$が成り立つ。
\end{thm}
\begin{proof}
集合$S$上の極大filter$\mathfrak{F}$が与えられたとき、$\forall F \in \mathfrak{F}$に対し、$F \cap M \neq \emptyset$なるその集合$S$の部分集合$M$が与えられたらば、集合$\mathfrak{F \cup}\left\{ M \right\}$の任意の空でない有限集合である部分集合$\mathfrak{X}'$に対し、その集合$\mathfrak{X}'$の任意の元はその集合$\mathfrak{F}$に属するか、その集合$M$であることになる。ここで、$M \notin \mathfrak{X}'$のとき、その集合$\mathfrak{X}'$が有限集合であることに注意してfilterの定義より$\bigcap_{} \mathfrak{X}'\in \mathfrak{F}$が成り立つ。さらに、filterの定義より$\emptyset \neq \bigcap_{} \mathfrak{X}'$が成り立つ。$M \in \mathfrak{X}'$のとき、$\forall F \in \mathfrak{X}' \setminus \left\{ M \right\}$に対し、$F \cap M \neq \emptyset$が成り立つので、次のようになる。
\begin{align*}
\emptyset &\neq \bigcap_{F \in \mathfrak{X}' \setminus \left\{ M \right\}} (F \cap M)\\
&= \bigcap_{F \in \mathfrak{X}' \setminus \left\{ M \right\}} F \cap M\\
&= \bigcap_{F \in \mathfrak{X}'} F\\
&= \bigcap_{} \mathfrak{X}'
\end{align*}
これにより、その集合$\mathfrak{F \cup}\left\{ M \right\}$は有限交叉性をもつ。\par
定理\ref{8.1.8.6}よりその集合$S$上の極大filter$\mathfrak{F}'$が存在して、$\mathfrak{F \cup}\left\{ M \right\} \subseteq \mathfrak{F}'$が成り立つ。ここで、$\mathfrak{F \subseteq F \cup}\left\{ M \right\} \subseteq \mathfrak{F}'$が成り立つかつ、その集合$\mathfrak{F}$もまた極大filterなので、$\mathfrak{F} = \mathfrak{F}'$が成り立つ。したがって、$\mathfrak{F \cup}\left\{ M \right\} \subseteq \mathfrak{F}' = \mathfrak{F}$が成り立つので、$M \in \mathfrak{F}$が得られる。
\end{proof}
\begin{thm}\label{8.1.8.8}
集合$S$が与えられたとき、$\forall a \in S$に対し、次式のような集合$p_{a}$はその集合$S$上の極大filterである、即ち、$p_{a} \in \beta(S)$が成り立つ。
\begin{align*}
p_{a} = \left\{ F \in \mathfrak{P}(S) \middle| a \in F \right\}
\end{align*}
\end{thm}
\begin{dfn}
上の集合$p_{a}$をその集合$S$上のその元$a$から誘導される単項filterという。
\end{dfn}
\begin{proof}
集合$S$が与えられたとき、$\forall a \in S$に対し、次式のような集合$p_{a}$において、
\begin{align*}
p_{a} = \left\{ F \in \mathfrak{P}(S) \middle| a \in F \right\}
\end{align*}
定義より$\emptyset \notin p_{a}$が成り立つかつ、$S \in p_{a}$が成り立つ。ここで、$\forall F \in p_{a}\forall G \in \mathfrak{P}(S)$に対し、$F \subseteq G \subseteq S$が成り立つなら、$a \in G$が成り立つので、$G \in p_{a}$も成り立つ。$\forall F,G \in p_{a}$に対し、$a \in F$かつ$a \in G$が成り立つので、$a \in F \cap G$も成り立ち、したがって、$F \cap G \in \mathfrak{F}$が成り立つ。ゆえに、その集合$p_{a}$はその集合$S$上のfilterである。\par
さらに、$\exists F \in \mathfrak{P}(S)$に対し、$F \notin p_{a}$かつ$S \setminus F \notin p_{a}$が成り立つと仮定すると、$F \notin p_{a}$より$a \notin F$が成り立つので、$a \in S$より$a \in S \setminus F$が成り立ち、したがって、$S \setminus F \in p_{a}$が成り立つが、これは$S \setminus F \notin p_{a}$が成り立つことに矛盾する。したがって、$\forall F \in \mathfrak{P}(S)$に対し、$F \in \mathfrak{F}$または$S \setminus F \in \mathfrak{F}$が成り立つ。
\end{proof}
\begin{thm}\label{8.1.8.9}
集合$S$が与えられたとき、$\forall a \in S\forall\mathfrak{F \in}\beta(S)$に対し、その極大filter$\mathfrak{F}$がその元$a$から誘導される単項filter$p_{a}$に等しいならそのときに限り、その極大filter$\mathfrak{F}$はその集合$\left\{ a \right\}$に属される。
\end{thm}
\begin{proof}
集合$S$が与えられたとき、$\forall a \in S\forall\mathfrak{F \in}\beta(S)$に対し、その極大filter$\mathfrak{F}$がその元$a$から誘導される単項filter$p_{a}$に等しいなら、$a \in \left\{ a \right\}$が成り立つことにより$\left\{ a \right\} \in p_{a} = \mathfrak{F}$が成り立つ。逆に、その極大filter$\mathfrak{F}$はその集合$\left\{ a \right\}$に属される、即ち、$\left\{ a \right\}\in \mathfrak{F}$が成り立つなら、$a \in \left\{ a \right\}$より、$\forall F \in \mathfrak{F}$に対し、$a \notin F$が成り立つと仮定すると、その集合$F$の与え方により$\left\{ a \right\} \cap F = \emptyset$が成り立つかつ、filterの定義より$\left\{ a \right\} \cap F \in \mathfrak{F}$が成り立つことになるが、これは$\emptyset \in \mathfrak{F}$が成り立つことになりfilterの定義に矛盾する。したがって、$a \in F$が成り立つので、$\mathfrak{F \subseteq}p_{a}$が成り立つことになる。ここで、$\mathfrak{F}\in \beta(S)$が成り立つので、極大filterの定義より$\mathfrak{F} = p_{a}$が成り立つ。
\end{proof}
%\hypertarget{ux6975ux5927filterux306bux3088ux3063ux3066ux8a98ux5c0eux3055ux308cux305fux4f4dux76f8ux7a7aux9593}{%
\subsubsection{極大filterによって誘導された位相空間}%\label{ux6975ux5927filterux306bux3088ux3063ux3066ux8a98ux5c0eux3055ux308cux305fux4f4dux76f8ux7a7aux9593}}
\begin{dfn}
集合$S$が与えられたとき、$\forall A \in \mathfrak{P}(S)$に対し、次のように集合$\beta_{A \in}(S)$を定める。
\begin{align*}
\beta_{A \in}(S) = \left\{ \mathfrak{F \in}\beta(S) \middle| A \in \mathfrak{F} \right\}
\end{align*}
\end{dfn}
\begin{thm}\label{8.1.8.10}
集合$S$が与えられたとき、$\forall A,B \in \mathfrak{P}(S)$に対し、次のことが成り立つ。
\begin{itemize}
\item
  $\beta_{S \in}(S) = \beta(S)$が成り立つ。
\item
  $\beta_{\emptyset \in}(S) = \emptyset$が成り立つ。
\item
  $\beta_{A \in}(S) \subseteq \beta_{B \in}(S)$が成り立つならそのときに限り、$A \subseteq B$が成り立つ。
\item
  $\beta_{A \in}(S) = \beta_{B \in}(S)$が成り立つならそのときに限り、$A = B$が成り立つ。
\item
  $\beta_{A \in}(S) \cup \beta_{B \in}(S) = \beta_{A \cup B \in}(S)$が成り立つ。
\item
  $\beta_{A \in}(S) \cap \beta_{B \in}(S) = \beta_{A \cap B \in}(S)$が成り立つ。
\item
  $\beta_{S \setminus A \in}(S) = \beta(S) \setminus \beta_{A \in}(S)$が成り立つ。
\end{itemize}\par
さらに、その集合$\mathfrak{P}(S)$の任意の添数集合$\varLambda$によって添数づけられた族$\left\{ A_{\lambda} \right\}_{\lambda \in \varLambda}$に対し、次のことが成り立つ。
\begin{itemize}
\item
  $\bigcup_{\lambda \in \varLambda} {\beta_{A_{\lambda} \in}(S)} \subseteq \beta_{\bigcup_{\lambda \in \varLambda} A_{\lambda} \in}(S)$が成り立つ。
\item
  $\bigcap_{\lambda \in \varLambda} {\beta_{A_{\lambda} \in}(S)} \supseteq \beta_{\bigcap_{\lambda \in \varLambda} A_{\lambda} \in}(S)$が成り立つ。
\end{itemize}
\end{thm}
\begin{proof}
集合$S$が与えられたとき、もちろん、定義より$\beta_{S \in}(S) \subseteq \beta(S)$が成り立つ。逆に、$\mathfrak{\forall F \in}\beta(S)$に対し、定理\ref{8.1.8.1}より$S \in \mathfrak{F}$が成り立つので、$\mathfrak{F}\in \beta_{S \in}(S)$が成り立つ。したがって、$\beta(S) \subseteq \beta_{S \in}(S)$が成り立つので、$\beta_{S \in}(S) = \beta(S)$が成り立つ。また、$\beta_{\emptyset \in}(S) = \emptyset$が成り立つことはfilterの定義より明らかである。\par
$\beta_{A \in}(S) \subseteq \beta_{B \in}(S)$が成り立つなら、$\forall a \in A$に対し、これから誘導される単項filter$p_{a}$について、$A \in p_{a}$が成り立つので、$p_{a} \in \beta_{A \in}(S)$が成り立ち、仮定よりしたがって、$p_{a} \in \beta_{B \in}(S)$が成り立つ。これにより、$B \in p_{a}$が成り立ち、したがって、$a \in B$が成り立つので、$A \subseteq B$が成り立つ。逆に、$A \subseteq B$が成り立つなら、$\mathfrak{\forall F \in}\beta_{A \in}(S)$に対し、$A \in \mathfrak{F}$が成り立つ。ここで、仮定より$A \subseteq B \subseteq S$が成り立つことから、filterの定義より$B \in \mathfrak{F}$が成り立つので、$\mathfrak{F}\in \beta_{B \in}(S)$が成り立つ。したがって、$\beta_{A \in}(S) \subseteq \beta_{B \in}(S)$が成り立つ。\par
もちろん、上記の議論により直ちに$\beta_{A \in}(S) = \beta_{B \in}(S)$が成り立つならそのときに限り、$A = B$が成り立つ。\par
$\mathfrak{\forall F \in}\beta_{A \in}(S) \cup \beta_{B \in}(S)$に対し、$\mathfrak{F}\in \beta_{A \in}(S)$または$\mathfrak{F}\in \beta_{B \in}(S)$が成り立つので、$A \in \mathfrak{F}$または$B \in \mathfrak{F}$が成り立つ。そこで、いずれの場合でも、$A \subseteq A \cup B \subseteq S$かつ$B \subseteq A \cup B \subseteq S$が成り立つので、filterの定義より$A \cup B \in \mathfrak{F}$が成り立つ。したがって、$\mathfrak{F}\in \beta_{A \cup B \in}(S)$が成り立つ。逆に、$\mathfrak{F}\in \beta_{A \cup B \in}(S)$が成り立つなら、$A \cup B \in \mathfrak{F}$が成り立つ。ここで、$\mathfrak{F \notin}\beta_{A \in}(S)$かつ$\mathfrak{F \notin}\beta_{B \in}(S)$が成り立つと仮定すると、$\mathfrak{F}\in \beta(S)$が成り立つことに注意すれば、定理\ref{8.1.8.4}より$A \in \mathfrak{F}$または$S \setminus A \in \mathfrak{F}$が成り立つかつ、$B \in \mathfrak{F}$または$S \setminus B$が成り立つので、$S \setminus A,S \setminus B \in \mathfrak{F}$が成り立つ。したがって、filterの定義より$S \setminus A \cap S \setminus B = S \setminus (A \cup B)\in \mathfrak{F}$が成り立つ。このとき、定理\ref{8.1.8.2}より$A \cup B \in \mathfrak{F}$かつ$S \setminus (A \cup B)\in \mathfrak{F}$が成り立つことはないのであったので、$A \cup B \notin \mathfrak{F}$が成り立つことになるが、これは仮定に矛盾する。したがって、$\mathfrak{F}\in \beta_{A \in}(S) \cup \beta_{B \in}(S)$が成り立つ。以上より、$\beta_{A \in}(S) \cup \beta_{B \in}(S) = \beta_{A \cup B \in}(S)$が成り立つ。\par
$\mathfrak{\forall F \in}\beta_{A \in}(S) \cap \beta_{B \in}(S)$に対し、$\mathfrak{F}\in \beta_{A \in}(S)$かつ$\mathfrak{F}\in \beta_{B \in}(S)$が成り立つので、$A \in \mathfrak{F}$かつ$B \in \mathfrak{F}$が成り立つ。filterの定義より$A \cap B \in \mathfrak{F}$が成り立つので、$\mathfrak{F}\in \beta_{A \cap B \in}(S)$が成り立つ。逆に、$\mathfrak{\forall F \in}\beta_{A \cap B \in}(S)$に対し、$A \cap B \in \mathfrak{F}$が成り立つかつ、$A \notin \mathfrak{F}$または$B \notin \mathfrak{F}$が成り立つとすると、$A \notin \mathfrak{F}$が成り立つとき、定理\ref{8.1.8.4}より$A \in \mathfrak{F}$または$S \setminus A \in \mathfrak{F}$が成り立つので、$S \setminus A \in \mathfrak{F}$が成り立つ。したがって、filterの定義より$(S \setminus A) \cap A \cap B \in \mathfrak{F}$が成り立つことになるが、$(S \setminus A) \cap A \cap B = \emptyset$が成り立つことにより、$\emptyset \in \mathfrak{F}$が成り立ちこれはfilterの定義に矛盾する。$B \notin \mathfrak{F}$が成り立つときも同様である。したがって、$A \in \mathfrak{F}$かつ$B \in \mathfrak{F}$が成り立つことになり、よって、$\mathfrak{F}\in \beta_{A \in}(S) \cap \beta_{B \in}(S)$が成り立つ。\par
$\mathfrak{\forall F \in}\beta_{S \setminus A \in}(S)$に対し、$S \setminus A \in \mathfrak{F}$が成り立つ。ここで、$\mathfrak{F}\in \beta(S)$が成り立つことに注意すれば、定理\ref{8.1.8.4}より$A \in \mathfrak{F}$または$S \setminus A \in \mathfrak{F}$が成り立つのであった。定理\ref{8.1.8.2}より$A \in \mathfrak{F}$かつ$S \setminus A \in \mathfrak{F}$が成り立つことはないので、$A \notin \mathfrak{F}$が成り立つことになり、したがって、$\mathfrak{F \notin}\beta_{A \in}(S)$が成り立つ。よって、$\beta(S) \setminus \beta_{A \in}(S)$が成り立つ。逆に、$\mathfrak{\forall F \in}\beta(S) \setminus \beta_{A \in}(S)$が成り立つなら、$\mathfrak{F \notin}\beta_{A \in}(S)$が成り立つので、$A \notin \mathfrak{F}$が成り立つ。ここで、定理\ref{8.1.8.2}より$A \in \mathfrak{F}$かつ$S \setminus A \in \mathfrak{F}$が成り立つことはないのであった。$\mathfrak{F}\in \beta(S)$が成り立つことに注意すれば、定理\ref{8.1.8.4}より$A \in \mathfrak{F}$または$S \setminus A \in \mathfrak{F}$が成り立つので、$S \setminus A \in \mathfrak{F}$が成り立ち、したがって、$\mathfrak{F}\in \beta_{S \setminus A \in}(S)$が成り立つ。\par
さらに、その集合$\mathfrak{P}(S)$の任意の添数集合$\varLambda$によって添数づけられた族$\left\{ A_{\lambda} \right\}_{\lambda \in \varLambda}$が与えられたとき、$\mathfrak{\forall F \in}\bigcup_{\lambda \in \varLambda} {\beta_{A_{\lambda} \in}(S)}$に対し、和集合の定義より、$\exists\lambda \in \varLambda$に対し、$\mathfrak{F}\in \beta_{A_{\lambda} \in}(S)$が成り立つので、$A_{\lambda}\in \mathfrak{F}$が成り立つ。そこで、いずれの場合でも、$A_{\lambda} \subseteq \bigcup_{\lambda \in \varLambda} A_{\lambda} \subseteq S$が成り立つので、filterの定義より$\bigcup_{\lambda \in \varLambda} A_{\lambda}\in \mathfrak{F}$が成り立つ。したがって、$\mathfrak{F}\in \beta_{\bigcup_{\lambda \in \varLambda} A_{\lambda} \in}(S)$が成り立つ。\par
$\mathfrak{\forall F \in}\beta_{\bigcap_{\lambda \in \varLambda} A_{\lambda} \in}(S)$に対し、$\bigcap_{\lambda \in \varLambda} A_{\lambda}\in \mathfrak{F}$が成り立つかつ、$\exists\lambda \in \varLambda$に対し、$A_{\lambda}\notin \mathfrak{F}$が成り立つとすると、定理\ref{8.1.8.4}より$A_{\lambda}\in \mathfrak{F}$または$S \setminus A_{\lambda}\in \mathfrak{F}$が成り立つので、$S \setminus A_{\lambda}\in \mathfrak{F}$が成り立つ。したがって、filterの定義より$\left( S \setminus A_{\lambda} \right) \cap \bigcap_{\lambda \in \varLambda} A_{\lambda}\in \mathfrak{F}$が成り立つことになるが、$\left( S \setminus A_{\lambda} \right) \cap \bigcap_{\lambda \in \varLambda} A_{\lambda} = \emptyset$が成り立つことにより、$\emptyset \in \mathfrak{F}$が成り立ちこれはfilterの定義に矛盾する。したがって、$\forall\lambda \in \varLambda$に対し、$A_{\lambda}\in \mathfrak{F}$が成り立つことになり、よって、$\mathfrak{F}\in \bigcap_{\lambda \in \varLambda} {\beta_{A_{\lambda} \in}(S)}$が成り立つ。
\end{proof}
\begin{thm}\label{8.1.8.11}
集合$S$が与えられたとき、その集合$S$上の極大filter全体の集合$\beta(S)$と次式のように定義される集合$\mathfrak{B}$の族の和集合全体の集合$\mathfrak{O}_{\beta}$との組$\left( \beta(S),\mathfrak{O}_{\beta} \right)$は位相空間をなす。
\begin{align*}
\mathfrak{B} = \left\{ \beta_{A \in}(S)\in \mathfrak{P}\left( \beta(S) \right) \middle| A \subseteq S \right\}
\end{align*}
さらに、その集合$\mathfrak{B}$がその位相空間$\left( \beta(S),\mathfrak{O}_{\beta} \right)$の開基となる。
\end{thm}
\begin{dfn}
その位相空間$\left( \beta(S),\mathfrak{O}_{\beta} \right)$はその集合$S$から極大filterによって誘導された位相空間という。
\end{dfn}
\begin{proof}
集合$S$が与えられたとき、その集合$S$上の極大filter全体の集合$\beta(S)$と次式のように定義される集合$\mathfrak{B}$の族の和集合全体の集合$\mathfrak{O}_{\beta}$との組$\left( \beta(S),\mathfrak{O}_{\beta} \right)$において、
\begin{align*}
\mathfrak{B} = \left\{ \beta_{A \in}(S)\in \mathfrak{P}\left( \beta(S) \right) \middle| A \subseteq S \right\}
\end{align*}
filterの定義より$\beta_{\emptyset \in}(S) = \emptyset$が成り立つので、$\emptyset \in \mathfrak{O}_{\beta}$が成り立つ。定理\ref{8.1.8.10}より$\beta_{S \in}(S) = \beta(S)$が成り立つので、$\beta(S) \in \mathfrak{O}_{\beta}$が成り立つ。\par
$\forall O,P \in \mathfrak{O}_{\beta}$に対し、定義より、あるその集合$\mathfrak{P}(S)$の元の族々$\left\{ A_{\mu} \right\}_{\mu \in M }$、$\left\{ B_{\nu} \right\}_{\nu \in N }$が存在して、$O = \bigcup_{\mu \in M} {\beta_{A_{\mu} \in}(S)}$かつ$P = \bigcup_{\nu \in N} {\beta_{B_{\nu} \in}(S)}$が成り立つので、定理\ref{8.1.8.10}より次のようになる。
\begin{align*}
O \cap P &= \bigcup_{\mu \in M} {\beta_{A_{\mu} \in}(S)} \cap \bigcup_{\nu \in N} {\beta_{B_{\nu} \in}(S)}\\
&= \bigcup_{(\mu,\nu) \in M \times N} {\beta_{A_{\mu} \in}(S) \cap \beta_{B_{\nu} \in}(S)}\\
&= \bigcup_{(\mu,\nu) \in M \times N} {\beta_{A_{\mu} \cap B_{\nu} \in}(S)}
\end{align*}
ここで、$A_{\mu} \cap B_{\nu} \subseteq S$が成り立つので、$O \cap P \in \mathfrak{O}_{\beta}$が成り立つ。\par
任意の添数集合$\varLambda$によって添数づけられたその集合$\mathfrak{O}_{\beta}$の元の族$\left\{ O_{\lambda} \right\}_{\lambda \in \varLambda}$が与えられたとき、定義より、$\forall\lambda \in \varLambda$に対し、あるその集合$\mathfrak{P}(S)$の元の族$\left\{ A_{\mu_{\lambda}} \right\}_{\mu_{\lambda} \in M_{\lambda} }$が存在して、$O_{\lambda} = \bigcup_{\mu_{\lambda} \in M_{\lambda}} {\beta_{A_{\mu_{\lambda}} \in}(S)}$が成り立つので、次のようになる。
\begin{align*}
\bigcup_{\lambda \in \varLambda} O_{\lambda} &= \bigcup_{\lambda \in \varLambda} {\bigcup_{\mu_{\lambda} \in M_{\lambda}} {\beta_{A_{\mu_{\lambda}} \in}(S)}}\\
&= \bigcup_{\forall\lambda \in \varLambda\left[ \mu_{\lambda} \in M_{\lambda} \right]} {\beta_{A_{\mu_{\lambda}} \in}(S)}
\end{align*}
ここで、$A_{\mu_{\lambda}} \subseteq S$が成り立つので、$\bigcup_{\lambda \in \varLambda} O_{\lambda} \in \mathfrak{O}_{\beta}$が成り立つ。\par
よって、その組$\left( \beta(S),\mathfrak{O}_{\beta} \right)$は位相空間をなす。
\begin{align*}
\mathfrak{B} = \left\{ \beta_{A \in}(S)\in \mathfrak{P}\left( \beta(S) \right) \middle| A \subseteq S \right\}
\end{align*}
さらに、その集合$\mathfrak{B}$がその位相空間$\left( \beta(S),\mathfrak{O}_{\beta} \right)$の開基となることは開基の定義より直ちに従う。
\end{proof}
\begin{thm}\label{8.1.8.12}
集合$S$が与えられたとき、その集合$S$から極大filterによって誘導された位相空間$\left( \beta(S),\mathfrak{O}_{\beta} \right)$はcompact空間である。
\end{thm}
\begin{proof}
集合$S$が与えられたとき、その集合$S$から極大filterによって誘導された位相空間$\left( \beta(S),\mathfrak{O}_{\beta} \right)$が与えられたとき、その集合$\beta(S)$のある開被覆$\mathfrak{U}$が存在して、これの任意の有限集合である部分集合$\mathfrak{U}'$に対し、$\bigcup_{} \mathfrak{U}' \subset \beta(S)$が成り立つと仮定すると、$\forall O \in \mathfrak{U}'$に対し、その位相$\mathfrak{O}_{\beta}$の定義よりあるその集合$\mathfrak{P}(S)$の元の族$\left\{ A_{\mu_{O}} \right\}_{\mu_{O} \in M_{O} }$が存在して、$O = \bigcup_{\mu_{O} \in M_{O}} {\beta_{A_{\mu_{O}} \in}(S)}$が成り立つので、次のようになる。
\begin{align*}
\bigcup_{} \mathfrak{U}' &= \bigcup_{O \in \mathfrak{U}'} O\\
&= \bigcup_{O \in \mathfrak{U}'} {\bigcup_{\mu_{O} \in M_{O}} {\beta_{A_{\mu_{O}} \in}(S)}}\\
&= \bigcup_{\forall O \in \mathfrak{U}'\left[ \mu_{O} \in M_{O} \right]} {\beta_{A_{\mu_{O}} \in}(S)}
\end{align*}\par
ここで、$\exists O \in \mathfrak{U}'$に対し、$\aleph_{0} \leq {\#}M_{O}$が成り立つと仮定すると、その集合$\beta_{A_{\mu_{O}} \in}(S)$全体の集合$\left\{ \beta_{A_{\mu_{O}} \in}(S) \right\}_{\forall O \in \mathfrak{U}'\left[ \mu_{O} \in M_{O} \right]}$は無限集合であるかつ、$\beta_{A_{\mu_{O}} \in}(S) \in \mathfrak{O}_{\beta}$が成り立つので、集合$\left\{ \beta_{A_{\mu_{O}} \in}(S) \right\}_{\forall O \in \mathfrak{U}'\left[ \mu_{O} \in M_{O} \right]}$はその開被覆$\mathfrak{U}$のその部分集合$\mathfrak{U}'$に一致する。しかしながら、その集合$\mathfrak{U}'$は仮定より有限集合であったので、これにその集合$\left\{ \beta_{A_{\mu_{O}} \in}(S) \right\}_{\forall O \in \mathfrak{U}'\left[ \mu_{O} \in M_{O} \right]}$が無限集合であることが矛盾する。\par
したがって、その集合$\left\{ \beta_{A_{\mu_{O}} \in}(S) \right\}_{\forall O \in \mathfrak{U}'\left[ \mu_{O} \in M_{O} \right]}$は有限集合であるので、$O \in \mathfrak{U}'$なる添数$\mu_{O}$を$\mu \in M'$、$O \in \mathfrak{U}$なる添数$\mu_{O}$を$\mu \in M$とおきなおすと、定理\ref{8.1.8.7}と数学的帰納法により次式が成り立つ。
\begin{align*}
\bigcup_{} \mathfrak{U}' &= \bigcup_{\mu \in M'} {\beta_{A_{\mu} \in}(S)}\\
&= \beta_{\bigcup_{\mu \in M'} A_{\mu} \in}(S)
\end{align*}
定理\ref{8.1.8.10}よりしたがって、次のようになる。
\begin{align*}
\bigcup_{} \mathfrak{U}' \subset \beta(S) &\Leftrightarrow \beta_{\bigcup_{\mu \in M'} A_{\mu} \in}(S) \subset \beta(S)\\
&\Leftrightarrow \beta_{\bigcup_{\mu \in M'} A_{\mu} \in}(S) \neq \beta(S)\\
&\Leftrightarrow \beta_{\bigcup_{\mu \in M'} A_{\mu} \in}(S) \neq \beta_{S \in}(S)\\
&\Leftrightarrow \bigcup_{\mu \in M'} A_{\mu} \neq S\\
&\Leftrightarrow S \setminus \bigcup_{\mu \in M'} A_{\mu} \neq \emptyset\\
&\Leftrightarrow \bigcap_{\mu \in M'} {S \setminus A_{\mu}} \neq \emptyset
\end{align*}\par
このことは集合$\left\{ S \setminus A_{\mu} \right\}_{\mu \in M}$を$\mathfrak{X}$とおいてその集合$\mathfrak{X}$の任意の空集合でない有限集合である部分集合$\mathfrak{X}'$に対し、$\bigcap_{} \mathfrak{X}' \neq \emptyset$が成り立つことを意味している。したがって、その集合$\mathfrak{X}$は有限交叉性をもつことになる。ここで、定理\ref{8.1.8.6}よりその集合$\mathfrak{X}$を含む極大filter$\mathfrak{F}$が存在する。このとき、$\forall\mu \in M$に対し、$S \setminus A_{\mu}\in \mathfrak{X \subseteq F}$が成り立つので、$S \setminus A_{\mu}\in \mathfrak{F}$が得られる。\par
このとき、$\mathfrak{F}\in \beta(S)$が成り立つことに注意すれば、定理\ref{8.1.8.4}より$A_{\mu}\in \mathfrak{F}$または$S \setminus A_{\mu}\in \mathfrak{F}$が成り立つのであった。定理\ref{8.1.8.2}より$A_{\mu}\in \mathfrak{F}$かつ$S \setminus A_{\mu}\in \mathfrak{F}$が成り立つことはないので、$A_{\mu}\notin \mathfrak{F}$が成り立つことになる。\par
一方で、その集合$\beta(S)$の開被覆$\mathfrak{U}$の定義より次式が成り立つので、
\begin{align*}
\beta(S) &= \bigcup_{} \mathfrak{U}\\
&= \bigcup_{O \in \mathfrak{U}} O\\
&= \bigcup_{O \in \mathfrak{U}} {\bigcup_{\mu_{O} \in M_{O}} {\beta_{A_{\mu_{O}} \in}(S)}}\\
&= \bigcup_{\forall O \in \mathfrak{U}\left[ \mu_{O} \in M_{O} \right]} {\beta_{A_{\mu_{O}} \in}(S)}\\
&= \bigcup_{\mu \in M} {\beta_{A_{\mu} \in}(S)}
\end{align*}
$\mathfrak{F}\in \beta(S) = \bigcup_{\mu \in M} {\beta_{A_{\mu} \in}(S)}$が成り立つ、即ち、$\exists\mu \in M$に対し、$\mathfrak{F}\in \beta_{A_{\mu} \in}(S)$が成り立つ、即ち、$\exists\mu \in M$に対し、$A_{\mu}\in \mathfrak{F}$が成り立つことになる。しかしながら、このことは、$\forall\mu \in M$に対し、$A_{\mu}\notin \mathfrak{F}$が成り立つことに矛盾する。\par
以上より、極大filterによって誘導されたその位相空間$\left( \beta(S),\mathfrak{O}_{\beta} \right)$が与えられたとき、その集合$\beta(S)$の任意の開被覆$\mathfrak{U}$に対し、これのある有限集合である部分集合$\mathfrak{U}'$が存在して、$\bigcup_{} \mathfrak{U}' = \beta(S)$が成り立つことになる。よって、その集合$S$から極大filterによって誘導された位相空間$\left( \beta(S),\mathfrak{O}_{\beta} \right)$はcompact空間である。
\end{proof}
\begin{thm}\label{8.1.8.13}
集合$S$が与えられたとき、その集合$S$から極大filterによって誘導された位相空間$\left( \beta(S),\mathfrak{O}_{\beta} \right)$はHausdorff空間である。
\end{thm}
\begin{proof}
集合$S$が与えられたとき、その集合$S$から極大filterによって誘導された位相空間$\left( \beta(S),\mathfrak{O}_{\beta} \right)$が与えられたとき、$\mathfrak{\forall F,G \in}\beta(S)$に対し、$\mathfrak{F \neq G}$が成り立つとする\footnote{分かり切っていることかもしれませんが、このような集合は位相空間の定義より存在します。例えば、台集合と空集合をとればよいです。}。このとき、$\mathfrak{F \setminus G \neq \emptyset}$が成り立つとしてもよいので、$F \in \mathfrak{F \setminus G}$なる元$F$が存在する。このとき、$F \in \mathfrak{F}$かつ$S \setminus F \in \mathfrak{G}$が成り立つので、$\mathfrak{F}\in \beta_{F \in}(S)$かつ$\mathfrak{G \in}\beta_{S \setminus F \in}(S)$が得られる。ここで、その位相空間$\left( \beta(S),\mathfrak{O}_{\beta} \right)$の定義よりそれらの集合たち$\beta_{F \in}(S)$、$\beta_{S \setminus F \in}(S)$は開集合なので、${\mathrm{int}}{\beta_{F \in}(S)} = \beta_{F \in}(S)$かつ${\mathrm{int}}{\beta_{S \setminus F \in}(S)} = \beta_{S \setminus F \in}(S)$が成り立つ。ゆえに、それらの集合たち$\beta_{F \in}(S)$、$\beta_{S \setminus F \in}(S)$はそれぞれ極大filterたち$\mathfrak{F}$、$\mathfrak{G}$の近傍である。さらに、定理\ref{8.1.8.10}より次のようになる。
\begin{align*}
\beta_{F \in}(S) \cap \beta_{S \setminus F \in}(S) &= \beta_{F \cap S \setminus F}(S)\\
&= \beta_{\emptyset}(S) = \emptyset
\end{align*}
よって、その集合$S$から極大filterによって誘導された位相空間$\left( \beta(S),\mathfrak{O}_{\beta} \right)$はHausdorff空間である。
\end{proof}
\begin{thm}\label{8.1.8.14}
集合$S$が与えられたとき、$\forall a \in S$に対し、その集合$S$上のその元$a$から誘導される単項filter$p_{a}$はその集合$S$から極大filterによって誘導された位相空間$\left( \beta(S),\mathfrak{O}_{\beta} \right)$で孤立点である。
\end{thm}
\begin{proof}
集合$S$が与えられたとき、$\forall a \in S$に対し、その集合$S$上のその元$a$から誘導される単項filter$p_{a}$が与えられたとき、$\mathfrak{\forall F \in}\beta(S)$に対し、$\mathfrak{F}\in \beta_{\left\{ a \right\} \in}(S)$が成り立つならそのときに限り、$\left\{ a \right\}\in \mathfrak{F}$が成り立つ。ここで、これが成り立つならそのときに限り、定理\ref{8.1.8.9}より$\mathfrak{F} = p_{a}$が成り立つ。したがって、$\left\{ p_{a} \right\} = \beta_{\left\{ a \right\} \in}(S)$が成り立つことになる。ここで、その集合$S$から極大filterによって誘導された位相空間$\left( \beta(S),\mathfrak{O}_{\beta} \right)$の定義よりその集合$\left\{ p_{a} \right\}$は開集合となる。したがって、次のようになる。
\begin{align*}
p_{a} \in \left\{ p_{a} \right\} &= {\mathrm{int}}\left\{ p_{a} \right\}\\
&= \beta(S) \setminus {\mathrm{cl}}\left( \beta(S) \setminus \left\{ p_{a} \right\} \right)
\end{align*}
したがって、$p_{a} \notin {\mathrm{cl}}\left( \beta(S) \setminus \left\{ p_{a} \right\} \right)$が得られ、よって、その集合$S$上のその元$a$から誘導される単項filter$p_{a}$はその集合$S$から極大filterによって誘導された位相空間$\left( \beta(S),\mathfrak{O}_{\beta} \right)$で孤立点である。
\end{proof}
\begin{thm}\label{8.1.8.15}
集合$S$が与えられたとき、族$\left\{ p_{a} \right\}_{a \in F}$は、その集合$S$から極大filterによって誘導された位相空間$\left( \beta(S),\mathfrak{O}_{\beta} \right)$において、その台集合$\beta(S)$の中で稠密である。
\end{thm}
\begin{proof}
集合$S$が与えられたとき、その集合$S$から極大filterによって誘導された位相空間$\left( \beta(S),\mathfrak{O}_{\beta} \right)$において、$\forall F \in \mathfrak{P}(S)$に対し、集合$\beta_{F \in}(S)$が空集合でないなら、定理\ref{8.1.8.10}より$F \neq \emptyset$が成り立つことになる。したがって、$\forall a \in F$に対し、$F \in p_{a}$が成り立つ、即ち、$p_{a} \in \beta_{F \in}(S)$が成り立つので、$\beta_{F \in}(S) \cap \left\{ p_{a} \right\}_{a \in F} \neq \emptyset$が成り立つ。ゆえに、その位相空間$\left( \beta(S),\mathfrak{O}_{\beta} \right)$の定義より$\forall O \in \mathfrak{O}_{\beta}$に対し、$O \cap \left\{ p_{a} \right\}_{a \in F} \neq \emptyset$が成り立つことが直ちに分かる。定理\ref{8.1.2.17}よりその族$\left\{ p_{a} \right\}_{a \in F}$はその台集合$\beta(S)$の中で稠密である。
\end{proof}
%\hypertarget{filterux3068compactux7a7aux9593hausdorffux7a7aux9593}{%
\subsubsection{filterとcompact空間}%\label{filterux3068compactux7a7aux9593hausdorffux7a7aux9593}}
\begin{thm}\label{8.1.8.16}
位相空間$\left( S,\mathfrak{O} \right)$とその集合$S$上のfilter$\mathfrak{F}$が与えられたとき、$\mathfrak{F} \rightarrow a$が成り立つなら、$\forall F\in \mathfrak{F}$に対し、$a \in {\mathrm{cl}}F$が成り立つ、即ち、$a \in \bigcap_{F\in \mathfrak{F}} {{\mathrm{cl}}F}$が成り立つ。
\end{thm}
\begin{proof}
位相空間$\left( S,\mathfrak{O} \right)$とその集合$S$上のfilter$\mathfrak{F}$が与えられたとき、$\mathfrak{F} \rightarrow a$が成り立つなら、その元$a$の任意の近傍$V$をとると、その元$a$の全近傍系を$\mathbf{V}(a)$として$V \in \mathbf{V}(a)$が成り立つ。したがって、その元$a$の基本近傍系を$\mathbf{V}^{*}(a)$として、$\exists U \in \mathbf{V}^{*}(a)$に対し、$U \subseteq V$が成り立つ。ここで、filterの収束の定義より$\mathbf{V}(a)\subseteq \mathfrak{F}$が成り立つので、基本近傍系の定義より$\mathbf{V}^{*}(a) \subseteq \mathbf{V}(a)$が成り立つことから$U \in \mathfrak{F}$が成り立つことになる。$\forall F \in \mathfrak{F}$に対し、filterの定義より$U \cap F \neq \emptyset$が成り立つので、定理\ref{8.1.1.26}より$a \in {\mathrm{cl}}F$が成り立つ。これが$a \in \bigcap_{F \in \mathfrak{F}} {{\mathrm{cl}}F}$が成り立つことに言い換えられることは明らかであろう。
\end{proof}
\begin{thm}\label{8.1.8.17}
位相空間$\left( S,\mathfrak{O} \right)$とその集合$S$上の極大filter$\mathfrak{F}$が与えられたとき、次のことは同値である。
\begin{itemize}
\item
  $\mathfrak{F} \rightarrow a$が成り立つ。
\item
  $\forall F\in \mathfrak{F}$に対し、$a \in {\mathrm{cl}}F$が成り立つ。
\item
  $a \in \bigcap_{F \in \mathfrak{F}} {{\mathrm{cl}}F}$が成り立つ。
\end{itemize}
\end{thm}
\begin{proof}
位相空間$\left( S,\mathfrak{O} \right)$とその集合$S$上の極大filter$\mathfrak{F}$が与えられたとき、定理\ref{8.1.8.16}より$\mathfrak{F} \rightarrow a$が成り立つなら、$\forall F \in \mathfrak{F}$に対し、$a \in {\mathrm{cl}}F$が成り立つ。これが$a \in \bigcap_{F \in \mathfrak{F}} {{\mathrm{cl}}F}$が成り立つことに言い換えられることは明らかであろう。一方で、$\forall F \in \mathfrak{F}$に対し、$a \in {\mathrm{cl}}F$が成り立つとする。ここで、その元$a$の全近傍系$\mathbf{V}(a)$が$\mathbf{V}(a)\subseteq \mathfrak{F}$を満たさないと仮定すると、$\exists V \in \mathbf{V}(a)$に対し、$V \notin \mathfrak{F}$が成り立つことになる。ここで、そのfilter$\mathfrak{F}$は極大filterなので、定理\ref{8.1.8.5}より$V \in \mathfrak{F}$または$S \setminus V \in \mathfrak{F}$が成り立つ。さらに、定理\ref{8.1.8.2}より$V \in \mathfrak{F}$かつ$S \setminus V \in \mathfrak{F}$が成り立つことはないので、$S \setminus V \in \mathfrak{F}$が成り立つことになる。ここで、仮定より$a \in {\mathrm{cl}}(S \setminus V) = S \setminus {\mathrm{int}}V$が成り立つ。しかしながら、近傍の定義より$a \in {\mathrm{int}}V$が成り立つことに矛盾する。よって、$\forall F \in \mathfrak{F}$に対し、$a \in {\mathrm{cl}}F$が成り立つなら、$\mathfrak{F} \rightarrow a$が成り立つ。
\end{proof}\par
ここで、次の定理が述べられるまえに、次の定義と定理が述べられよう。
\begin{dfn*}[定義\ref{有限交叉性}の再掲]
集合$S$の部分集合系$\mathfrak{X}$が与えられたとき、これの任意の空でない有限集合である部分集合に属する集合同士の共通部分が空でないとき、即ち、$\forall\mathfrak{X}'\in \mathfrak{P}\left( \mathfrak{X} \right)$に対し、$0 < {\#}\mathfrak{X}' < \aleph_{0}$が成り立つなら、$\bigcap_{} \mathfrak{X}' \neq \emptyset$が成り立つとき、その集合$\mathfrak{X}$は有限交叉性を持つという。
\end{dfn*}
\begin{thm*}[定理\ref{8.1.6.1}の再掲]
位相空間$\left( S,\mathfrak{O} \right)$について、次のことは同値である。
\begin{itemize}
\item
  その位相空間$\left( S,\mathfrak{O} \right)$はcompact空間である。
\item
  その位相空間$\left( S,\mathfrak{O} \right)$の閉集合系を$\mathfrak{A}$とおくとき、その台集合$S$の任意の部分集合系$\mathfrak{X}$が$\mathfrak{X \subseteq A}$かつ有限交叉性を持つなら、$\bigcap_{} \mathfrak{X} \neq \emptyset$が成り立つ。
\item
  その台集合$S$の任意の部分集合系$\mathfrak{X}$が有限交叉性を持つなら、$\bigcap_{X \in \mathfrak{X}} {{\mathrm{cl}}X} \neq \emptyset$が成り立つ。
\end{itemize}
\end{thm*}\par
さて本題に戻って、次の定理が掲げられよう。
\begin{thm}\label{8.1.8.19}
位相空間$\left( S,\mathfrak{O} \right)$が与えられたとき、その位相空間$\left( S,\mathfrak{O} \right)$がcompact空間であるならそのときに限り、その集合$S$上の任意の極大filter$\mathfrak{F}$に対し、その極限点は少なくとも1つもつ。
\end{thm}
\begin{proof}
位相空間$\left( S,\mathfrak{O} \right)$が与えられたとき、その位相空間$\left( S,\mathfrak{O} \right)$がcompact空間であるなら、その集合$S$上の任意の極大filter$\mathfrak{F}$に対し、定理\ref{8.1.6.1}よりその台集合$S$の任意の部分集合系$\mathfrak{X}$が有限交叉性を持つなら、$\bigcap_{X \in \mathfrak{X}} {{\mathrm{cl}}X} \neq \emptyset$が成り立つのであった。このとき、$\exists a \in S$に対し、$a \in \bigcap_{X \in \mathfrak{X}} {{\mathrm{cl}}X}$が成り立つので、その集合$S$上の極大filter$\mathfrak{F}$が与えられていることに注意すれば、定理\ref{8.1.8.17}より$\mathfrak{F} \rightarrow a$が成り立つ。\par
逆に、その集合$S$上の任意の極大filter$\mathfrak{F}$に対し、その極限点は少なくとも1つもつとする。その位相空間$\left( S,\mathfrak{O} \right)$において、その位相空間$\left( S,\mathfrak{O} \right)$の閉集合系を$\mathfrak{A}$とおくとき、その台集合$S$の任意の部分集合系$\mathfrak{X}$が$\mathfrak{X \subseteq A}$かつ有限交叉性を持つなら、定理\ref{8.1.8.6}よりその集合$S$の部分集合系$\mathfrak{X}$を含むその集合$S$上の極大filter$\mathfrak{F}$が存在する。ここで、仮定より$\exists a \in S$に対し、$\mathfrak{F} \rightarrow a$が成り立つ。したがって、定理\ref{8.1.8.17}より$\forall F \in \mathfrak{F}$に対し、$a \in {\mathrm{cl}}F$が成り立つことになる。これにより、$\forall X \in \mathfrak{X}$に対し、$a \in {\mathrm{cl}}X = X$が成り立つ、即ち、$a \in \bigcap_{X \in \mathfrak{X}} X = \bigcap_{} \mathfrak{X}$が成り立つ。なお、$\mathfrak{X \subseteq A}$が成り立つことに注意した。これにより、その台集合$S$の任意の部分集合系$\mathfrak{X}$が$\mathfrak{X \subseteq A}$かつ有限交叉性を持つなら、$\bigcap_{} \mathfrak{X} \neq \emptyset$が成り立つことが示され、さらに、定理\ref{8.1.6.1}よりその位相空間$\left( S,\mathfrak{O} \right)$はcompact空間である。
\end{proof}
\subsubsection{filterとHausdorff空間}
\begin{thm}\label{8.1.8.18}
位相空間$\left( S,\mathfrak{O} \right)$が与えられたとき、その位相空間$\left( S,\mathfrak{O} \right)$がHausdorff空間であるならそのときに限り、その集合$S$上の任意のfilter$\mathfrak{F}$に対し、その極限点がたかだか1つである。
\end{thm}
\begin{proof}
位相空間$\left( S,\mathfrak{O} \right)$が与えられたとする。その位相空間$\left( S,\mathfrak{O} \right)$がHausdorff空間であるなら、その集合$S$上の任意のfilter$\mathfrak{F}$のうちある元に収束するものが考えられれば、$\mathfrak{F} \rightarrow a$かつ$\mathfrak{F} \rightarrow b$が成り立つかつ、$a \neq b$が成り立つと仮定すると、仮定よりそれらの元々$a$、$b$の全近傍系$\mathbf{V}(a)$、$\mathbf{V}(b)$を用いて、$\mathbf{V}(a)\subseteq \mathfrak{F}$、$\mathbf{V}(b)\subseteq \mathfrak{F}$が成り立つかつ、$\exists V \in \mathbf{V}(a)\exists W \in \mathbf{V}(b)$に対し、$V \cap W = \emptyset$が成り立つ。このとき、$V,W \in \mathfrak{F}$が成り立つので、$V \cap W = \emptyset \in \mathfrak{F}$が成り立つことになるが、これはfilterの定義に矛盾する。よって、$\mathfrak{F} \rightarrow a$かつ$\mathfrak{F} \rightarrow b$が成り立つなら、$a = b$が成り立つ。\par
逆に、その位相空間$\left( S,\mathfrak{O} \right)$がHausdorff空間でないなら、$\exists a,b \in S$に対し、それらの元々$a$、$b$の全近傍系$\mathbf{V}(a)$、$\mathbf{V}(b)$を用いて、$\forall V \in \mathbf{V}(a)\forall W \in \mathbf{V}(b)$に対し、$V \cap W \neq \emptyset$が成り立つ。ここで、次式のように集合$\mathfrak{F}$が定義されれば、
\begin{align*}
\mathfrak{F} = \left\{ V \cap W \in \mathfrak{P}(S) \middle| V \in \mathbf{V}(a) \land W \in \mathbf{V}(b) \right\}
\end{align*}
その集合$\mathfrak{F}$はfilterになる。実際、$\emptyset \in \mathfrak{F}$が成り立つなら、$\emptyset \in \mathbf{V}(a)$または$\emptyset \in \mathbf{V}(b)$が成り立つことになるが、これは全近傍系の定義に矛盾するので、$\emptyset \notin \mathfrak{F}$が成り立つし、$\forall V \cap W \in \mathfrak{F}\forall A \in \mathfrak{P}(S)$に対し、$V \cap W \subseteq A$が成り立つなら、次式が成り立つので、
\begin{align*}
V = (V \cap W) \sqcup (V \setminus W) \subseteq A \cup V \setminus W,\ \ W = (V \cap W) \sqcup (W \setminus V) \subseteq A \cup W \setminus V
\end{align*}
次のようになり、
\begin{align*}
a \in {\mathrm{int}}V \subseteq {\mathrm{int}}(A \cup V \setminus W),\ \ b \in {\mathrm{int}}W \subseteq {\mathrm{int}}(A \cup W \setminus V)
\end{align*}
したがって、$A \cup V \setminus W \in \mathbf{V}(a)$かつ$A \cup W \setminus V \in \mathbf{V}(b)$が成り立ち、このとき、次のようになるので、
\begin{align*}
(A \cup V \setminus W) \cap (A \cup W \setminus V) &= A \cup (V \setminus W \cap W \setminus V)\\
&= A \cup \emptyset = A
\end{align*}
$A \in \mathfrak{F}$が得られるし、$\forall V \cap W,V' \cap W'\in \mathfrak{F}$に対し、$V,V' \in \mathbf{V}(a)$かつ$W,W' \in \mathbf{V}(b)$とすれば、$V \cap V' \in \mathbf{V}(a)$かつ$W \cap W' \in \mathbf{V}(b)$が成り立つので、$(V \cap W) \cap \left( V' \cap W' \right)\in \mathfrak{F}$が成り立つ。このとき、$\forall V \in \mathbf{V}(a)$に対し、$V = V \cap S$かつ$S \in \mathbf{V}(b)$が成り立つので、$V \in \mathfrak{F}$が成り立ち、したがって、$\mathbf{V}(a)\subseteq \mathfrak{F}$が成り立つ。同様にして、$\mathbf{V}(b)\subseteq \mathfrak{F}$も成り立つので、$\mathfrak{F} \rightarrow a$かつ$\mathfrak{F} \rightarrow b$が成り立つかつ、$a \neq b$が成り立つ。これはその集合$S$上のあるfilter$\mathfrak{F}$が存在して、その極限点がたかだか1つであるとは限らないことを意味する。よって、対偶律によりその集合$S$上の任意のfilter$\mathfrak{F}$に対し、その極限点がたかだか1つであるなら、その位相空間$\left( S,\mathfrak{O} \right)$がHausdorff空間である。
\end{proof}\par
\begin{thebibliography}{50}
\bibitem{1}
  松坂和夫, 集合・位相入門, 岩波書店, 1968. 新装版第2刷 p111,247,274 ISBN978-4-00-029871-1
\bibitem{2}
  福井敏純. "集合と位相空間入門". 埼玉大学. \url{http://www.rimath.saitama-u.ac.jp/lab.jp/Fukui/lectures/Set_Topsp.pdf} (2021-11-16 4:15 取得)
\bibitem{3}
  後藤達哉. "ウルトラフィルターの空間". 筑波大学. \url{https://fujidig.github.io/201905-ultrafilter-stonecech/201905-ultrafilter-stonecech.pdf} (2021-12-8 7:33 取得)
\end{thebibliography}
\end{document}
