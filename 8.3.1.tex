\documentclass[dvipdfmx]{jsarticle}
\setcounter{section}{3}
\setcounter{subsection}{0}
\usepackage{xr}
\externaldocument{8.1.6}
\usepackage{amsmath,amsfonts,amssymb,array,comment,mathtools,url,docmute}
\usepackage{longtable,booktabs,dcolumn,tabularx,mathtools,multirow,colortbl,xcolor}
\usepackage[dvipdfmx]{graphics}
\usepackage{bmpsize}
\usepackage{amsthm}
\usepackage{enumitem}
\setlistdepth{20}
\renewlist{itemize}{itemize}{20}
\setlist[itemize]{label=•}
\renewlist{enumerate}{enumerate}{20}
\setlist[enumerate]{label=\arabic*.}
\setcounter{MaxMatrixCols}{20}
\setcounter{tocdepth}{3}
\newcommand{\rotin}{\text{\rotatebox[origin=c]{90}{$\in $}}}
\renewcommand{\thesection}{第\arabic{section}部}
\renewcommand{\thesubsection}{\arabic{section}.\arabic{subsection}}
\renewcommand{\thesubsubsection}{\arabic{section}.\arabic{subsection}.\arabic{subsubsection}}
\everymath{\displaystyle}
\allowdisplaybreaks[4]
\usepackage{vtable}
\theoremstyle{definition}
\newtheorem{thm}{定理}[subsection]
\newtheorem*{thm*}{定理}
\newtheorem{dfn}{定義}[subsection]
\newtheorem*{dfn*}{定義}
\newtheorem{axs}[dfn]{公理}
\newtheorem*{axs*}{公理}
\renewcommand{\headfont}{\bfseries}
\makeatletter
  \renewcommand{\section}{%
    \@startsection{section}{1}{\z@}%
    {\Cvs}{\Cvs}%
    {\normalfont\huge\headfont\raggedright}}
\makeatother
\makeatletter
  \renewcommand{\subsection}{%
    \@startsection{subsection}{2}{\z@}%
    {0.5\Cvs}{0.5\Cvs}%
    {\normalfont\LARGE\headfont\raggedright}}
\makeatother
\makeatletter
  \renewcommand{\subsubsection}{%
    \@startsection{subsubsection}{3}{\z@}%
    {0.4\Cvs}{0.4\Cvs}%
    {\normalfont\Large\headfont\raggedright}}
\makeatother
\makeatletter
\renewenvironment{proof}[1][\proofname]{\par
  \pushQED{\qed}%
  \normalfont \topsep6\p@\@plus6\p@\relax
  \trivlist
  \item\relax
  {
  #1\@addpunct{.}}\hspace\labelsep\ignorespaces
}{%
  \popQED\endtrivlist\@endpefalse
}
\makeatother
\renewcommand{\proofname}{\textbf{証明}}
\usepackage{tikz,graphics}
\usepackage[dvipdfmx]{hyperref}
\usepackage{pxjahyper}
\hypersetup{
 setpagesize=false,
 bookmarks=true,
 bookmarksdepth=tocdepth,
 bookmarksnumbered=true,
 colorlinks=false,
 pdftitle={},
 pdfsubject={},
 pdfauthor={},
 pdfkeywords={}}
\begin{document}
%\hypertarget{ux591aux69d8ux4f53}{%
\subsection{多様体}%\label{ux591aux69d8ux4f53}}
%\hypertarget{ux4f4dux76f8ux591aux69d8ux4f53}{%
\subsubsection{位相多様体}%\label{ux4f4dux76f8ux591aux69d8ux4f53}}
\begin{dfn}
Hausdorff空間$\left( \mathcal{M},\mathfrak{O} \right)$が与えられたとき、$\forall p\in \mathcal{M}$に対し、$U \in \mathbf{V}(p)\mathfrak{\cap O}$なるある開近傍$U$と\footnote{開集合でもあるような近傍のことです。}$n$次元Euclid空間$E^{n}$における位相空間$\left( \mathbb{R}^{n},\mathfrak{O}_{d_{E^{n}}} \right)$の開集合$E$が存在して、$\left( U,\mathfrak{O}_{U} \right) \approx \left( E,\left( \mathfrak{O}_{d_{E^{n}}} \right)_{E} \right)$が成り立つとき、そのHausdorff空間$\left( \mathcal{M},\mathfrak{O} \right)$を$n$次元位相多様体、$n$-topological manifoldという。
\end{dfn}
\begin{dfn}
$n$次元位相多様体$\left( \mathcal{M},\mathfrak{O} \right)$において、上の定義における開近傍$U$と$\left( U,\mathfrak{O}_{U} \right) \approx \left( E,\left( \mathfrak{O}_{d_{E^{n}}} \right)_{E} \right)$な$n$次元Euclid空間$E^{n}$における位相空間$\left( \mathbb{R}^{n},\mathfrak{O}_{d_{E^{n}}} \right)$の開集合$E$との間の同相写像$\psi$との組$(U,\psi)$をその位相多様体$\left( \mathcal{M},\mathfrak{O} \right)$の座標近傍、coordinate neighborhood、chartという。
\end{dfn}
\begin{dfn}
$n$次元位相多様体$\left( \mathcal{M},\mathfrak{O} \right)$の座標近傍$(U,\psi)$が与えられたとき、$\forall p \in U$に対し、点$\psi(p)$を実数の組としてみなしたとき、その組をその点$p$のその座標近傍$(U,\psi)$における局所座標、local coordinateといい、その同相写像$\psi$を写像の組とみなしてその座標近傍$(U,\psi)$における局所座標系、local coordinate systemという。
\end{dfn}
\begin{thm}\label{8.3.1.1}
$n$次元位相多様体$\left( \mathcal{M},\mathfrak{O} \right)$の座標近傍$(U,\psi)$が与えられたとする。$\psi = \left( \psi_{i} \right)_{i \in \varLambda_{n}}$とおくとき\footnote{つまり、$\forall q \in U$に対し、点$\psi(q)$の第$i$成分を$\psi_{i}(q)$と書くことにしましょうということです。}、$\forall p,q \in U\forall i \in \varLambda_{n}$に対し、$\psi_{i}(p) = \psi_{i}(q)$が成り立つなら、$p = q$が成り立つ。
\end{thm}
\begin{proof}
定義よりその写像$\psi$が全単射であることから明らかである。実際、$n$次元位相多様体$\left( \mathcal{M},\mathfrak{O} \right)$の座標近傍$(U,\psi)$が与えられたとする。$\psi = \left( \psi_{i} \right)_{i \in \varLambda_{n}}$とおくとき、$\forall p,q \in U\forall i \in \varLambda_{n}$に対し、$\psi_{i}(p) = \psi_{i}(q)$が成り立つなら、$\psi(p) = \psi(q)$が成り立ち、そこで、その写像$\psi$は定義より同相写像であるので、全単射である、特に、単射であるので、$p = q$が成り立つ。
\end{proof}
\begin{thm}\label{8.3.1.2}
$n$次元位相多様体$\left( \mathcal{M},\mathfrak{O} \right)$について、ある局所座標系$\psi$が存在して、組$\left( \mathcal{M},\psi \right)$が局所座標をなすならそのときに限り、$n$次元Euclid空間$E^{n}$における位相空間$\left( \mathbb{R}^{n},\mathfrak{O}_{d_{E^{n}}} \right)$の開集合$E$が存在して、$\left( \mathcal{M},\mathfrak{O} \right) \approx \left( E,\left( \mathfrak{O}_{d_{E^{n}}} \right)_{E} \right)$が成り立つ。
\end{thm}
\begin{proof}
局所座標の定義そのものから明らかである。実際、$n$次元位相多様体$\left( \mathcal{M},\mathfrak{O} \right)$について、$\forall p\in \mathcal{M}$に対し、台集合$\mathcal{M}$自身がその点$p$の開近傍でもあることに注意すれば、ある局所座標系$\psi$が存在して、組$\left( \mathcal{M},\psi \right)$が局所座標をなすなら、$n$次元Euclid空間$E^{n}$における位相空間$\left( \mathbb{R}^{n},\mathfrak{O}_{d_{E^{n}}} \right)$の開集合$E$が存在して、$\left( \mathcal{M},\mathfrak{O} \right) \approx \left( E,\left( \mathfrak{O}_{d_{E^{n}}} \right)_{E} \right)$が成り立つ。逆に、これが成り立つなら、$\forall p\in \mathcal{M}$に対し、台集合$\mathcal{M}$自身がその点$p$の開近傍でもあるので、これらの位相空間たちの間の同相写像$\psi$が存在するので、その組$\left( \mathcal{M},\psi \right)$が局所座標となっている。
\end{proof}
\begin{thm}\label{8.3.1.3}
$n$次元位相多様体$\left( \mathcal{M},\mathfrak{O} \right)$について、ある局所座標の族$\left\{ \left( U_{\alpha},\psi_{\alpha} \right) \right\}_{\alpha \in A}$が存在して、次式が成り立つ。
\begin{align*}
\mathcal{M} = \bigcup_{\alpha \in A} U_{\alpha}
\end{align*}
\end{thm}
\begin{dfn}
$n$次元位相多様体$\left( \mathcal{M},\mathfrak{O} \right)$について、$\mathcal{M} = \bigcup_{\alpha \in A} U_{\alpha}$が成り立つような局所座標の族$\left\{ \left( U_{\alpha},\psi_{\alpha} \right) \right\}_{\alpha \in A}$をその$n$次元位相多様体$\left( \mathcal{M},\mathfrak{O} \right)$の座標近傍系、coordinate
neighborhood system、atlasという。
\end{dfn}
\begin{proof}
$n$次元位相多様体$\left( \mathcal{M},\mathfrak{O} \right)$について、$\forall p\in \mathcal{M}$に対し、ある局所座標$\left( U_{p},\psi_{p} \right)$が存在して、$p \in U_{p}$が成り立つので、$\mathcal{M \subseteq}\bigcup_{p\in \mathcal{M}} U_{p}$が成り立つ。そこで、$U_{p} \subseteq \mathcal{M}$なので、$\bigcup_{p\in \mathcal{M}} U_{p} \subseteq \mathcal{M}$も成り立つ。よって、ある局所座標の族$\left\{ \left( U_{p},\psi_{p} \right) \right\}_{p\in \mathcal{M}}$が存在して、次式が成り立つ。
\begin{align*}
\mathcal{M} = \bigcup_{p\in \mathcal{M}} U_{p}
\end{align*}
\end{proof}
\begin{thm}\label{8.3.1.4}
$n$次元位相多様体$\left( \mathcal{M},\mathfrak{O} \right)$ともう1つの位相空間$\left( \mathcal{N},\mathfrak{P} \right)$が与えられたとき、$\left( \mathcal{M},\mathfrak{O} \right) \approx \left( \mathcal{N},\mathfrak{P} \right)$が成り立つなら、その位相空間$\left( \mathcal{N},\mathfrak{P} \right)$も$n$次元位相多様体である。
\end{thm}\par
この定理は$n$次元位相多様体であるという性質が位相的性質である、即ち、同相写像によって保たれる性質であるということを主張している。
\begin{proof}
$n$次元位相多様体$\left( \mathcal{M},\mathfrak{O} \right)$ともう1つの位相空間$\left( \mathcal{N},\mathfrak{P} \right)$が与えられたとき、$\left( \mathcal{M},\mathfrak{O} \right) \approx \left( \mathcal{N},\mathfrak{P} \right)$が成り立つなら、これらの位相空間たちの間に同相写像$\varphi\mathcal{:M \rightarrow N}$が存在することになる。そこで、その位相空間$\left( \mathcal{N},\mathfrak{P} \right)$がHausdorff空間であることを示そう。$\forall p,q \in \mathcal{N}$に対し、$p \neq q$が成り立つなら、その逆写像$\varphi^{- 1}$が特に単射でもあるので、$\varphi^{- 1}(p) \neq \varphi^{- 1}(q)$が成り立つ。その位相空間$\left( \mathcal{M},\mathfrak{O} \right)$がHausdorff空間であるので、それらの点々$\varphi^{- 1}(p)$、$\varphi^{- 1}(q)$の近傍たち$V$、$W$が存在して、$V \cap W = \emptyset$が成り立つ。そこで、その写像$\varphi^{- 1}$が同相写像であるので、$V \in \mathbf{V}\left( \varphi^{- 1}(p) \right)$かつ$W \in \mathbf{V}\left( \varphi^{- 1}(q) \right)$より$V\left( \varphi|V \right) \in \mathbf{V}(p)$かつ$V\left( \varphi|W \right) \in \mathbf{V}(q)$が成り立つかつ、その写像$\varphi$が全単射であることから$V\left( \varphi|V \cap W \right) = V\left( \varphi|V \right) \cap V\left( \varphi|W \right)$より$V\left( \varphi|V \right) \cap V\left( \varphi|W \right) = \emptyset$が成り立つ。ゆえに、その位相空間$\left( \mathcal{N},\mathfrak{P} \right)$はHausdorff空間である。\par
$\forall p \in \mathcal{N}$に対し、$\varphi^{- 1}(p)\in \mathcal{M}$なので、$U \in \mathbf{V}\left( \varphi^{- 1}(p) \right)\mathfrak{\cap O}$なるある開近傍$U$と$n$次元Euclid空間$E^{n}$における位相空間$\left( \mathbb{R}^{n},\mathfrak{O}_{d_{E^{n}}} \right)$の開集合$E$が存在して、$\left( U,\mathfrak{O}_{U} \right) \approx \left( E,\left( \mathfrak{O}_{d_{E^{n}}} \right)_{E} \right)$が成り立つ。その写像$\varphi^{- 1}$が同相写像であるので、$U \in \mathbf{V}\left( \varphi^{- 1}(p) \right)$かつ$U \in \mathfrak{O}$より$V\left( \varphi|U \right) \in \mathbf{V}(p)$かつ$V\left( \varphi|U \right)\in \mathfrak{P}$が成り立つ。ゆえに、その集合$V\left( \varphi|U \right)$はその元$p$の開近傍でもある。そこで、その写像$\varphi|U:U \rightarrow V\left( \varphi|U \right)$はその部分位相空間$\left( U,\mathfrak{O}_{U} \right)$からその部分位相空間$\left( V\left( \varphi|U \right),\ \ \mathfrak{P}_{V\left( \varphi|U \right)} \right)$への同相写像であることを示そう。$\forall P \in \mathfrak{P}_{V\left( \varphi|U \right)}$に対し、その集合$V\left( \varphi|U \right)$も開集合なので、その集合$P$はその位相空間$\left( \mathcal{N},\mathfrak{P} \right)$での開集合でもある。したがって、$V\left( \varphi^{- 1}|P \right)\in \mathfrak{O}$が成り立つ。また、$P \subseteq V\left( \varphi|U \right)$なので、その写像が全単射であることから$V\left( \varphi^{- 1}|P \right) \subseteq U$よりその開集合$V\left( \varphi^{- 1}|P \right)$はその部分位相空間$\left( U,\mathfrak{O}_{U} \right)$における開集合でもある、即ち、$V\left( \varphi^{- 1}|P \right) \in \mathfrak{O}_{U}$が成り立つ。これにより、その写像$\varphi|U$は連続写像である。同様にして考えれば、その写像$\left( \varphi|U \right)^{- 1}$も連続写像であることも分かる。実際、$\forall O \in \mathfrak{O}_{U}$に対し、その集合$U$も開集合なので、その集合$O$はその位相空間$\left( \mathcal{M},\mathfrak{O} \right)$での開集合でもある。したがって、$V\left( \varphi|O \right)\in \mathfrak{P}$が成り立つ。また、$O \subseteq U$なので、$V\left( \varphi|O \right) \subseteq V\left( \varphi|U \right)$よりその開集合$V\left( \varphi|O \right)$はその部分位相空間$\left( V\left( \varphi|U \right),\ \ \mathfrak{P}_{V\left( \varphi|U \right)} \right)$における開集合でもある、即ち、$V\left( \varphi|O \right) \in \mathfrak{P}_{V\left( \varphi|U \right)}$が成り立つ。$\left( \varphi|U \right)^{- 1} = \varphi^{- 1}|V\left( \varphi|U \right)$に注意すれば、その写像$\left( \varphi|U \right)^{- 1}$も連続写像である。以上の議論により、その写像$\varphi|U:U \rightarrow V\left( \varphi|U \right)$はその部分位相空間$\left( U,\mathfrak{O}_{U} \right)$からその部分位相空間$\left( V\left( \varphi|U \right),\ \ \mathfrak{P}_{V\left( \varphi|U \right)} \right)$への同相写像である。\par
最後に、そのHausdorff空間$\left( \mathcal{N},\mathfrak{P} \right)$が$n$次元位相多様体であることを示そう。$\forall p \in \mathcal{N}$に対し、$U \in \mathbf{V}\left( \varphi^{- 1}(p) \right)\mathfrak{\cap O}$なるある開近傍$U$と$n$次元Euclid空間$E^{n}$における位相空間$\left( \mathbb{R}^{n},\mathfrak{O}_{d_{E^{n}}} \right)$の開集合$E$が存在して、$\left( U,\mathfrak{O}_{U} \right) \approx \left( E,\left( \mathfrak{O}_{d_{E^{n}}} \right)_{E} \right)$が成り立つかつ、その集合$V\left( \varphi|U \right)$はその元$p$の開近傍でもあった。そこで上記の議論により、$\left( U,\mathfrak{O}_{U} \right) \approx \left( V\left( \varphi|U \right),\ \ \mathfrak{P}_{V\left( \varphi|U \right)} \right)$が成り立つので、その関係$\approx$が同値関係となっていることから、$\left( V\left( \varphi|U \right),\ \ \mathfrak{P}_{V\left( \varphi|U \right)} \right) \approx \left( E,\left( \mathfrak{O}_{d_{E^{n}}} \right)_{E} \right)$が成り立つ。よって、そのHausdorff空間$\left( \mathcal{N},\mathfrak{P} \right)$は$n$次元位相多様体である。
\end{proof}
\begin{thm}\label{8.3.1.5}
$n$次元位相多様体$\left( \mathcal{M},\mathfrak{O} \right)$の局所座標$(U,\psi)$が与えられたとする。$\forall O \in \mathfrak{O}$に対し、$O \subseteq U$かつ$O \neq \emptyset$が成り立つなら、写像$\psi|O:O \rightarrow V\left( \psi|O \right)$はその部分位相空間$\left( O,\mathfrak{O}_{O} \right)$からその部分位相空間$\left( V\left( \psi|O \right), \left( \mathfrak{O}_{d_{E^{n}}} \right)_{V\left( \psi|O \right)} \right)$への同相写像である。
\end{thm}
\begin{proof}
$n$次元位相多様体$\left( \mathcal{M},\mathfrak{O} \right)$の局所座標$(U,\psi)$が与えられたとする。$\forall O \in \mathfrak{O}$に対し、$O \subseteq U$かつ$O \neq \emptyset$が成り立つとする。\par
$n$次元Euclid空間$E^{n}$における位相空間$\left( \mathbb{R}^{n},\mathfrak{O}_{d_{E^{n}}} \right)$の部分位相空間$\left( V\left( \psi|O \right), \left( \mathfrak{O}_{d_{E^{n}}} \right)_{V\left( \psi|O \right)} \right)$が与えられたとき、写像$\psi|O:O \rightarrow V\left( \psi|O \right)$について、その位相空間$\left( V\left( \psi|O \right), \left( \mathfrak{O}_{d_{E^{n}}} \right)_{V\left( \psi|O \right)} \right)$における任意の開集合$E$に対し、その集合$O$が開集合であることその座標近傍系$\psi$が同相写像であることによりその集合$V\left( \psi|O \right)$も開集合となり、したがって、その集合$E$もその位相空間$\left( \mathbb{R}^{n},\mathfrak{O}_{d_{E^{n}}} \right)$における開集合でもあるので、その値域$V\left( \psi^{- 1}|E \right)$もその位相空間$\left( \mathcal{M},\mathfrak{O} \right)$における開集合となる。さらに、$E \subseteq V\left( \psi|O \right)$が成り立つので、その座標近傍系$\psi$が全単射であることに注意すれば、$V\left( \psi^{- 1}|E \right) \subseteq V\left( \psi^{- 1}|V\left( \psi|O \right) \right) = O$が得られ、その値域$V\left( \psi^{- 1}|E \right)$はその集合$O$の部分集合であるかつ、その位相空間$\left( \mathcal{M},\mathfrak{O} \right)$における開集合でもあるので、その位相空間$\left( O,\mathfrak{O}_{O} \right)$における開集合でもある。ゆえに、その写像$\psi|O$はその位相空間$\left( O,\mathfrak{O}_{O} \right)$からその位相空間$\left( V\left( \psi|O \right),\ \ \left( \mathfrak{O}_{d_{E^{n}}} \right)_{V\left( \psi|O \right)} \right)$への連続写像である。そこで、その座標近傍系$\psi$が全単射なので、その写像$\psi|O$も単射で、その値域に注意すれば、全単射でもある。\par
以上の議論により、逆写像$\left( \psi|O \right)^{- 1}$も存在することになり、同様にして考えれば、その逆写像も連続であることが示される。実際、その位相空間$\left( O,\mathfrak{O}_{O} \right)$における任意の開集合$O'$に対し、その集合$O$も開集合であることにより、その集合$O'$もその位相空間$\left( \mathcal{M},\mathfrak{O} \right)$における開集合でもあるので、その値域$V\left( \psi|O' \right)$もその位相空間$\left( \mathbb{R}^{n},\mathfrak{O}_{d_{E^{n}}} \right)$における開集合となる。さらに、$O' \subseteq O$が成り立つので、$V\left( \psi|O' \right) \subseteq V\left( \psi|O \right)$が得られ、その値域$V\left( \psi|O' \right)$はその位相空間$\left( V\left( \psi|O \right),\ \ \left( \mathfrak{O}_{d_{E^{n}}} \right)_{V\left( \psi|O \right)} \right)$における開集合でもある。ゆえに、その逆写像$\left( \psi|O \right)^{- 1}$はその位相空間$\left( V\left( \psi|O \right), \left( \mathfrak{O}_{d_{E^{n}}} \right)_{V\left( \psi|O \right)} \right)$からその位相空間$\left( O,\mathfrak{O}_{O} \right)$への連続写像である。\par
よって、その写像$\psi|O$は同相写像である。
\end{proof}
\begin{thm}[座標変換]\label{8.3.1.6}
$n$次元位相多様体$\left( \mathcal{M},\mathfrak{O} \right)$の局所座標たち$\left( U_{\alpha},\psi_{\alpha} \right)$、$\left( U_{\beta},\psi_{\beta} \right)$が与えられたとする。$U_{\alpha} \cap U_{\beta} \neq \emptyset$が成り立つなら、次式のように定義される関数$f_{\left( U_{\alpha},\psi_{\alpha} \right) \rightarrow \left( U_{\beta},\psi_{\beta} \right)}$は同相写像であり
\begin{align*}
f_{\left( U_{\alpha},\psi_{\alpha} \right) \rightarrow \left( U_{\beta},\psi_{\beta} \right)} = \psi_{\beta}|U_{\alpha} \cap U_{\beta} \circ \psi_{\alpha}^{- 1}|V\left( \psi_{\alpha}|U_{\alpha} \cap U_{\beta} \right);V\left( \psi_{\alpha}|U_{\alpha} \cap U_{\beta} \right) \rightarrow V\left( \psi_{\beta}|U_{\alpha} \cap U_{\beta} \right)
\end{align*}
その逆関数$f_{\left( U_{\alpha},\psi_{\alpha} \right) \rightarrow \left( U_{\beta},\psi_{\beta} \right)}^{- 1}$は$f_{\left( U_{\beta},\psi_{\beta} \right) \rightarrow \left( U_{\alpha},\psi_{\alpha} \right)}$である。ここでは、その関数$f_{\left( U_{\alpha},\psi_{\alpha} \right) \rightarrow \left( U_{\beta},\psi_{\beta} \right)}$をその局所座標$\left( U_{\alpha},\psi_{\alpha} \right)$からその局所座標$\left( U_{\beta},\psi_{\beta} \right)$への座標変換、変換関数ということにする。
\end{thm}
\begin{proof}
$n$次元位相多様体$\left( \mathcal{M},\mathfrak{O} \right)$の局所座標たち$\left( U_{\alpha},\psi_{\alpha} \right)$、$\left( U_{\beta},\psi_{\beta} \right)$が与えられたとし、$U_{\alpha} \cap U_{\beta} \neq \emptyset$が成り立つとする。その集合$U_{\alpha} \cap U_{\beta}$が開集合であるので、定理\ref{8.3.1.5}より写像たち$\psi_{\alpha}|U_{\alpha} \cap U_{\beta}$、$\psi_{\beta}|U_{\alpha} \cap U_{\beta}$は同相写像であり、その逆写像$\left( \psi_{\alpha}|U_{\alpha} \cap U_{\beta} \right)^{- 1}$も同相写像である。そこで、$\left( \psi_{\alpha}|U_{\alpha} \cap U_{\beta} \right)^{- 1} = \psi_{\alpha}^{- 1}|V\left( \psi_{\alpha}|U_{\alpha} \cap U_{\beta} \right)$が成り立つことに注意すれば、次のようになるので、
\begin{align*}
f_{\left( U_{\alpha},\psi_{\alpha} \right) \rightarrow \left( U_{\beta},\psi_{\beta} \right)} &= \psi_{\beta}\left| U_{\alpha} \cap U_{\beta} \circ \psi_{\alpha}^{- 1} \right|V\left( \psi_{\alpha}|U_{\alpha} \cap U_{\beta} \right)\\
&= \psi_{\beta}|U_{\alpha} \cap U_{\beta} \circ \left( \psi_{\alpha}|U_{\alpha} \cap U_{\beta} \right)^{- 1}
\end{align*}
その写像$f_{\left( U_{\alpha},\psi_{\alpha} \right) \rightarrow \left( U_{\beta},\psi_{\beta} \right)}$も同相写像であり、特に、連続写像である。\par
最後に、その逆関数$f_{\left( U_{\alpha},\psi_{\alpha} \right) \rightarrow \left( U_{\beta},\psi_{\beta} \right)}^{- 1}$は$f_{\left( U_{\beta},\psi_{\beta} \right) \rightarrow \left( U_{\alpha},\psi_{\alpha} \right)}$であることも示そう。実際、次のようになることから
\begin{comment}
\begin{align*}
&\quad f_{\left( U_{\beta},\psi_{\beta} \right) \rightarrow \left( U_{\alpha},\psi_{\alpha} \right)} \circ f_{\left( U_{\alpha},\psi_{\alpha} \right) \rightarrow \left( U_{\beta},\psi_{\beta} \right)}\\
&= \psi_{\alpha}|U_{\alpha} \cap U_{\beta} \circ \psi_{\beta}^{- 1}|V\left( \psi_{\beta}|U_{\alpha} \cap U_{\beta} \right) \circ \psi_{\beta}|U_{\alpha} \cap U_{\beta} \circ \psi_{\alpha}^{- 1}|V\left( \psi_{\alpha}|U_{\alpha} \cap U_{\beta} \right)\\
&= U_{\alpha} \cap U_{\beta}\overset{\psi_{\alpha}}{\rightarrow}V\left( \psi_{\alpha}|U_{\alpha} \cap U_{\beta} \right) \circ V\left( \psi_{\beta}|U_{\alpha} \cap U_{\beta} \right)\overset{\psi_{\beta}^{- 1}}{\rightarrow}U_{\alpha} \cap U_{\beta} \circ U_{\alpha} \cap U_{\beta}\overset{\psi_{\beta}}{\rightarrow}V\left( \psi_{\beta}|U_{\alpha} \cap U_{\beta} \right) \circ V\left( \psi_{\alpha}|U_{\alpha} \cap U_{\beta} \right)\overset{\psi_{\alpha}^{- 1}}{\rightarrow}U_{\alpha} \cap U_{\beta}\\
&= V\left( \psi_{\alpha}|U_{\alpha} \cap U_{\beta} \right)\overset{\psi_{\alpha}^{- 1}}{\rightarrow}U_{\alpha} \cap U_{\beta}\overset{\psi_{\beta}}{\rightarrow}V\left( \psi_{\beta}|U_{\alpha} \cap U_{\beta} \right) \cdot \overset{\psi_{\beta}^{- 1}}{\rightarrow}U_{\alpha} \cap U_{\beta}\overset{\psi_{\alpha}}{\rightarrow}V\left( \psi_{\alpha}|U_{\alpha} \cap U_{\beta} \right)\\
&= V\left( \psi_{\alpha}|U_{\alpha} \cap U_{\beta} \right)\overset{\psi_{\alpha}^{- 1}}{\rightarrow}U_{\alpha} \cap U_{\beta}\overset{I_{U_{\alpha} \cap U_{\beta}}}{\rightarrow}U_{\alpha} \cap U_{\beta}\overset{\psi_{\alpha}}{\rightarrow}V\left( \psi_{\alpha}|U_{\alpha} \cap U_{\beta} \right)\\
&= V\left( \psi_{\alpha}|U_{\alpha} \cap U_{\beta} \right)\overset{\psi_{\alpha}^{- 1}}{\rightarrow}U_{\alpha} \cap U_{\beta}\overset{\psi_{\alpha}}{\rightarrow}V\left( \psi_{\alpha}|U_{\alpha} \cap U_{\beta} \right)\\
&= V\left( \psi_{\alpha}|U_{\alpha} \cap U_{\beta} \right)\overset{I_{V\left( \psi_{\alpha}|U_{\alpha} \cap U_{\beta} \right)}}{\rightarrow}V\left( \psi_{\alpha}|U_{\alpha} \cap U_{\beta} \right)\\
&= I_{V\left( \psi_{\alpha}|U_{\alpha} \cap U_{\beta} \right)}
\end{align*}
\end{comment}
\begin{align*}
f_{\left( U_{\beta},\psi_{\beta} \right) \rightarrow \left( U_{\alpha},\psi_{\alpha} \right)} \circ f_{\left( U_{\alpha},\psi_{\alpha} \right) \rightarrow \left( U_{\beta},\psi_{\beta} \right)} &= \psi_{\alpha}|U_{\alpha} \cap U_{\beta} \circ \psi_{\beta}^{- 1}|V\left( \psi_{\beta}|U_{\alpha} \cap U_{\beta} \right) \circ \psi_{\beta}|U_{\alpha} \cap U_{\beta} \circ \psi_{\alpha}^{- 1}|V\left( \psi_{\alpha}|U_{\alpha} \cap U_{\beta} \right)\\
&= \psi_{\alpha}|U_{\alpha} \cap U_{\beta} \circ \psi_{\alpha}^{- 1}|V\left( \psi_{\alpha}|U_{\alpha} \cap U_{\beta} \right)\\
&= I_{V\left( \psi_{\alpha}|U_{\alpha} \cap U_{\beta} \right)}
\end{align*}  
$f_{\left( U_{\beta},\psi_{\beta} \right) \rightarrow \left( U_{\alpha},\psi_{\alpha} \right)} \circ f_{\left( U_{\alpha},\psi_{\alpha} \right) \rightarrow \left( U_{\beta},\psi_{\beta} \right)} = I_{V\left( \psi_{\alpha}|U_{\alpha} \cap U_{\beta} \right)}$が得られる。同様にして、$f_{\left( U_{\alpha},\psi_{\alpha} \right) \rightarrow \left( U_{\beta},\psi_{\beta} \right)} \circ f_{\left( U_{\beta},\psi_{\beta} \right) \rightarrow \left( U_{\alpha},\psi_{\alpha} \right)} = I_{V\left( \psi_{\beta}|U_{\alpha} \cap U_{\beta} \right)}$も得られる。よって、$f_{\left( U_{\alpha},\psi_{\alpha} \right) \rightarrow \left( U_{\beta},\psi_{\beta} \right)}^{- 1} = f_{\left( U_{\beta},\psi_{\beta} \right) \rightarrow \left( U_{\alpha},\psi_{\alpha} \right)}$が成り立つ。
\end{proof}
\begin{thm}\label{8.3.1.7}
$n$次元位相多様体$\left( \mathcal{M},\mathfrak{O} \right)$の局所座標たち$\left( U_{\alpha},\psi_{\alpha} \right)$、$\left( U_{\beta},\psi_{\beta} \right)$が与えられたとする。$U_{\alpha} \cap U_{\beta} \neq \emptyset$が成り立つなら、その局所座標$\left( U_{\alpha},\psi_{\alpha} \right)$からその局所座標$\left( U_{\beta},\psi_{\beta} \right)$への座標変換$f_{\left( U_{\alpha},\psi_{\alpha} \right) \rightarrow \left( U_{\beta},\psi_{\beta} \right)}$を用いれば次式が成り立つ。
\begin{align*}
\psi_{\beta}|U_{\alpha} \cap U_{\beta} = f_{\left( U_{\alpha},\psi_{\alpha} \right) \rightarrow \left( U_{\beta},\psi_{\beta} \right)} \circ \psi_{\alpha}|U_{\alpha} \cap U_{\beta}
\end{align*}
\end{thm}\par
これは名の通り2つの局所座標たち$\left( U_{\alpha},\psi_{\alpha} \right)$、$\left( U_{\beta},\psi_{\beta} \right)$の間を変換する式になっている。詳しくいえば、$\forall p \in U_{\alpha} \cap U_{\beta}$に対し、$\psi_{\alpha} = \left( \psi_{\alpha i} \right)_{i \in \varLambda_{n}}$、$\psi_{\beta} = \left( \psi_{\beta i} \right)_{i \in \varLambda_{n}}$とおかれれば、次のようになっている。
\begin{align*}
\begin{pmatrix}
\psi_{\beta 1}(p) \\
\psi_{\beta 2}(p) \\
 \vdots \\
\psi_{\beta n}(p) \\
\end{pmatrix} = f_{\left( U_{\alpha},\psi_{\alpha} \right) \rightarrow \left( U_{\beta},\psi_{\beta} \right)}\begin{pmatrix}
\psi_{\alpha 1}(p) \\
\psi_{\alpha 2}(p) \\
 \vdots \\
\psi_{\alpha n}(p) \\
\end{pmatrix}
\end{align*}
\begin{proof}
$n$次元位相多様体$\left( \mathcal{M},\mathfrak{O} \right)$の局所座標たち$\left( U_{\alpha},\psi_{\alpha} \right)$、$\left( U_{\beta},\psi_{\beta} \right)$が与えられたとする。$U_{\alpha} \cap U_{\beta} \neq \emptyset$が成り立つなら、$\left( \psi_{\alpha}|U_{\alpha} \cap U_{\beta} \right)^{- 1} = \psi_{\alpha}^{- 1}|V\left( \psi_{\alpha}|U_{\alpha} \cap U_{\beta} \right)$が成り立つことに注意すれば、その局所座標$\left( U_{\alpha},\psi_{\alpha} \right)$からその局所座標$\left( U_{\beta},\psi_{\beta} \right)$への座標変換$f_{\left( U_{\alpha},\psi_{\alpha} \right) \rightarrow \left( U_{\beta},\psi_{\beta} \right)}$を用いて次のようになる。
\begin{align*}
f_{\left( U_{\alpha},\psi_{\alpha} \right) \rightarrow \left( U_{\beta},\psi_{\beta} \right)} \circ \psi_{\alpha}|U_{\alpha} \cap U_{\beta} &= \psi_{\beta}|U_{\alpha} \cap U_{\beta} \circ \psi_{\alpha}^{- 1}|V\left( \psi_{\alpha}|U_{\alpha} \cap U_{\beta} \right) \circ \psi_{\alpha}|U_{\alpha} \cap U_{\beta}\\
&= \psi_{\beta}|U_{\alpha} \cap U_{\beta} \circ \left( \psi_{\alpha}|U_{\alpha} \cap U_{\beta} \right)^{- 1} \circ \psi_{\alpha}|U_{\alpha} \cap U_{\beta}\\
&= \psi_{\beta}|U_{\alpha} \cap U_{\beta} \circ I_{U_{\alpha} \cap U_{\beta}}\\
&= \psi_{\beta}|U_{\alpha} \cap U_{\beta}
\end{align*}
\end{proof}
%\hypertarget{ux53efux5faeux5206ux591aux69d8ux4f53}{%
\subsubsection{可微分多様体}%\label{ux53efux5faeux5206ux591aux69d8ux4f53}}
\begin{dfn}
$n$次元位相多様体$\left( \mathcal{M},\mathfrak{O} \right)$について、これのある座標近傍系$\left\{ \left( U_{\alpha},\psi_{\alpha} \right) \right\}_{\alpha \in A}$が存在して、$\forall\alpha,\beta \in A$に対し、$U_{\alpha} \cap U_{\beta} \neq \emptyset$が成り立つなら、座標変換たち$f_{\left( U_{\alpha},\psi_{\alpha} \right) \rightarrow \left( U_{\beta},\psi_{\beta} \right)}$、$f_{\left( U_{\beta},\psi_{\beta} \right) \rightarrow \left( U_{\alpha},\psi_{\alpha} \right)}$がそれらの定義域で$C^{r}$級関数であるとき、その$n$次元位相多様体$\left( \mathcal{M},\mathfrak{O} \right)$を$n$次元$C^{r}$級可微分多様体、$n$次元$C^{r}$多様体といい、その座標近傍系$\left\{ \left( U_{\alpha},\psi_{\alpha} \right) \right\}_{\alpha \in A}$をその$n$次元位相多様体$\left( \mathcal{M},\mathfrak{O} \right)$の$C^{r}$級座標近傍系という。また、そのような関係を導入することをその座標近傍系$\left\{ \left( U_{\alpha},\psi_{\alpha} \right) \right\}_{\alpha \in A}$がその$n$次元位相多様体$\left( \mathcal{M},\mathfrak{O} \right)$に$C^{r}$級可微分構造を入れるという。
\end{dfn}\par
なお、位相多様体はまさしく$C^{0}$多様体であることに注意されたい。
\begin{dfn}
$n$次元位相多様体$\left( \mathcal{M},\mathfrak{O} \right)$について、これのある座標近傍系$\left\{ \left( U_{\alpha},\psi_{\alpha} \right) \right\}_{\alpha \in A}$が存在して、$\forall\alpha,\beta \in A$に対し、$U_{\alpha} \cap U_{\beta} \neq \emptyset$が成り立つなら、座標変換たち$f_{\left( U_{\alpha},\psi_{\alpha} \right) \rightarrow \left( U_{\beta},\psi_{\beta} \right)}$、$f_{\left( U_{\beta},\psi_{\beta} \right) \rightarrow \left( U_{\alpha},\psi_{\alpha} \right)}$がそれらの定義域で$C^{\infty}$級関数であるとき、その$n$次元位相多様体$\left( \mathcal{M},\mathfrak{O} \right)$を$n$次元$C^{\infty}$級可微分多様体、$n$次元$C^{\infty}$多様体といい、その座標近傍系$\left\{ \left( U_{\alpha},\psi_{\alpha} \right) \right\}_{\alpha \in A}$をその$n$次元位相多様体$\left( \mathcal{M},\mathfrak{O} \right)$の$C^{\infty}$級座標近傍系という。また、そのような関係を導入することをその座標近傍系$\left\{ \left( U_{\alpha},\psi_{\alpha} \right) \right\}_{\alpha \in A}$がその$n$次元位相多様体$\left( \mathcal{M},\mathfrak{O} \right)$に$C^{\infty}$級可微分構造を入れるという。以下、微分幾何学の流儀に則り$n$次元$C^{\infty}$級可微分多様体を単に$n$次元多様体ということにする。
\end{dfn}
\begin{dfn}
$n$次元位相多様体$\left( \mathcal{M},\mathfrak{O} \right)$について、これのある座標近傍系$\left\{ \left( U_{\alpha},\psi_{\alpha} \right) \right\}_{\alpha \in A}$が存在して、$\forall\alpha,\beta \in A$に対し、$U_{\alpha} \cap U_{\beta} \neq \emptyset$が成り立つなら、座標変換たち$f_{\left( U_{\alpha},\psi_{\alpha} \right) \rightarrow \left( U_{\beta},\psi_{\beta} \right)}$、$f_{\left( U_{\beta},\psi_{\beta} \right) \rightarrow \left( U_{\alpha},\psi_{\alpha} \right)}$がそれらの定義域で解析的であるとき、即ち、定義域上の任意の点のまわりでTaylor展開できるとき、その$n$次元位相多様体$\left( \mathcal{M},\mathfrak{O} \right)$を$n$次元解析多様体、$n$次元$C^{\omega}$級可微分多様体、$n$次元$C^{\omega}$多様体といい、その座標近傍系$\left\{ \left( U_{\alpha},\psi_{\alpha} \right) \right\}_{\alpha \in A}$をその$n$次元位相多様体$\left( \mathcal{M},\mathfrak{O} \right)$の解析的座標近傍系、$C^{\omega}$級座標近傍系という。また、そのような関係を導入することをその座標近傍系$\left\{ \left( U_{\alpha},\psi_{\alpha} \right) \right\}_{\alpha \in A}$がその$n$次元位相多様体$\left( \mathcal{M},\mathfrak{O} \right)$に解析構造を入れる、$C^{\omega}$級可微分構造を入れるという。
\end{dfn}\par
以下、特に断りがなければ、$C^{r}$を$r$を$0$以上の整数としたものか$C^{\infty}$か$C^{\omega}$であるとする。
%\hypertarget{ux958bux90e8ux5206ux591aux69d8ux4f53}{%
\subsubsection{開部分多様体}%\label{ux958bux90e8ux5206ux591aux69d8ux4f53}}
\begin{thm}\label{8.3.1.8}
$n$次元$C^{r}$多様体$\left( \mathcal{M},\mathfrak{O} \right)$、これの$C^{r}$級座標近傍系$\left\{ \left( U_{\alpha},\psi_{\alpha} \right) \right\}_{\alpha \in A}$が与えられたとき、$\forall O \in \mathfrak{O}$に対し、その部分位相空間$\left( O,\mathfrak{O}_{O} \right)$はその族$\left\{ \left( U_{\alpha} \cap O, \psi_{\alpha}|U_{\alpha} \cap O \right) \right\}_{\alpha \in A}$をこれの$C^{r}$級座標近傍系とする$n$次元$C^{r}$多様体となる。
\end{thm}
\begin{dfn}
$n$次元$C^{r}$多様体$\left( \mathcal{M},\mathfrak{O} \right)$、これの$C^{r}$級座標近傍系$\left\{ \left( U_{\alpha},\psi_{\alpha} \right) \right\}_{\alpha \in A}$が与えられたとき、$\forall O \in \mathfrak{O}$に対し、その族$\left\{ \left( U_{\alpha} \cap O,\psi_{\alpha}|U_{\alpha} \cap O \right) \right\}_{\alpha \in A}$を$C^{r}$級座標近傍系とする$n$次元$C^{r}$多様体$\left( O,\mathfrak{O}_{O} \right)$をその$C^{r}$多様体$\left( \mathcal{M},\mathfrak{O} \right)$の開部分多様体という。
\end{dfn}
\begin{proof}
$n$次元$C^{r}$多様体$\left( \mathcal{M},\mathfrak{O} \right)$、これの$C^{r}$級座標近傍系$\left\{ \left( U_{\alpha},\psi_{\alpha} \right) \right\}_{\alpha \in A}$が与えられたとき、$\forall O \in \mathfrak{O}$に対し、その部分位相空間$\left( O,\mathfrak{O}_{O} \right)$は定理\ref{8.1.6.9}よりHausdorff空間となる。次に、その部分位相空間$\left( O,\mathfrak{O}_{O} \right)$が$n$次元位相多様体でその座標近傍系が$\left\{ \left( U_{\alpha} \cap O,\ \ \psi_{\alpha}|U_{\alpha} \cap O \right) \right\}_{\alpha \in A}$と与えられることを示そう。仮定より$\forall p \in O$に対し、$p\in \mathcal{M}$よりその点$p$のその位相空間$\left( \mathcal{M},\mathfrak{O} \right)$の意味でのある開近傍$U$と$n$次元Euclid空間$E^{n}$における位相空間$\left( \mathbb{R}^{n},\mathfrak{O}_{d_{E^{n}}} \right)$の開集合$E$が存在して、$\left( U,\mathfrak{O}_{U} \right) \approx \left( E,\left( \mathfrak{O}_{d_{E^{n}}} \right)_{E} \right)$が成り立つ。その集合$U \cap O$について、その集合$O$がその位相空間$\left( \mathcal{M},\mathfrak{O} \right)$における開集合なので、その集合$U \cap O$もその部分位相空間$\left( O,\mathfrak{O}_{O} \right)$における開集合でもある。さらに、$p \in {\mathrm{int}}U = U$かつ$p \in O$より$p \in U \cap O$もその部分位相空間$\left( O,\mathfrak{O}_{O} \right)$におけるその点$p$の近傍でもある。また、$\left( U,\mathfrak{O}_{U} \right) \approx \left( E,\left( \mathfrak{O}_{d_{E^{n}}} \right)_{E} \right)$よりある同相写像$\psi:U \rightarrow E$が存在しその組$(U,\psi)$が局所座標となるのであった。$U \cap O \subseteq U$かつ$p \in U \cap O$に注意すれば、定理\ref{8.3.1.5}より写像$\psi|U \cap O:U \cap O \rightarrow V\left( \psi|U \cap O \right)$はその部分位相空間$\left( U \cap O,\mathfrak{O}_{U \cap O} \right)$からその部分位相空間$\left( V\left( \psi|U \cap O \right),\ \ \left( \mathfrak{O}_{d_{E^{n}}} \right)_{V\left( \psi|U \cap O \right)} \right)$への同相写像となるので、$\left( U \cap O,\mathfrak{O}_{U \cap O} \right) \approx \left( V\left( \psi|U \cap O \right),\ \ \left( \mathfrak{O}_{d_{E^{n}}} \right)_{V\left( \psi|U \cap O \right)} \right)$が成り立つ。ゆえに、その部分位相空間$\left( O,\mathfrak{O}_{O} \right)$が$n$次元位相多様体であることが分かり局所座標としてその組$\left( U \cap O,\psi|U \cap O \right)$が挙げられる。さらに、次式が成り立つことから、
\begin{align*}
\bigcup_{\alpha \in A} \left( U_{\alpha} \cap O \right) = \bigcup_{\alpha \in A} U_{\alpha} \cap O = \mathcal{M}\cap O = O
\end{align*}
その座標近傍系が$\left\{ \left( U_{\alpha} \cap O,\ \ \psi_{\alpha}|U_{\alpha} \cap O \right) \right\}_{\alpha \in A}$と与えられる。\par
最後に、その部分位相空間$\left( O,\mathfrak{O}_{O} \right)$が$n$次元$C^{r}$多様体でその$C^{r}$級座標近傍系が$\left\{ \left( U_{\alpha} \cap O,\ \ \psi_{\alpha}|U_{\alpha} \cap O \right) \right\}_{\alpha \in A}$と与えられることを示そう。$\forall\alpha,\beta \in A$に対し、$\left( U_{\alpha} \cap O \right) \cap \left( U_{\beta} \cap O \right) \neq \emptyset$が成り立つなら、もちろん、$U_{\alpha} \cap U_{\beta} \neq \emptyset$が成り立つので、仮定よりそれらの座標変換たち$f_{\left( U_{\alpha},\psi_{\alpha} \right) \rightarrow \left( U_{\beta},\psi_{\beta} \right)}$、$f_{\left( U_{\beta},\psi_{\beta} \right) \rightarrow \left( U_{\alpha},\psi_{\alpha} \right)}$がそれらの定義域で$C^{r}$級関数である。このとき、それらの写像たち$f_{\left( U_{\alpha},\psi_{\alpha} \right) \rightarrow \left( U_{\beta},\psi_{\beta} \right)}|V\left( \psi_{\alpha}|U_{\alpha} \cap U_{\beta} \cap O \right)$、$f_{\left( U_{\beta},\psi_{\beta} \right) \rightarrow \left( U_{\alpha},\psi_{\alpha} \right)}|V\left( \psi_{\beta}|U_{\alpha} \cap U_{\beta} \cap O \right)$もそれらの定義域で$C^{r}$級関数である。したがって、次のようになることから、
\begin{comment}
\begin{align*}
&\quad f_{\left( U_{\alpha} \cap O,\psi_{\alpha}|U_{\alpha} \cap O \right) \rightarrow \left( U_{\beta} \cap O,\psi_{\beta}|U_{\beta} \cap O \right)}\\
&= \left( \psi_{\beta}|U_{\beta} \cap O \right)|U_{\alpha} \cap U_{\beta} \cap O \circ \left( \psi_{\alpha}|U_{\alpha} \cap O \right)^{- 1}|V\left( \psi_{\alpha}|U_{\alpha} \cap U_{\beta} \cap O \right)\\
&= U_{\alpha} \cap U_{\beta} \cap O\overset{\psi_{\beta}|U_{\alpha} \cap U_{\beta}}{\rightarrow}V\left( \psi_{\beta}|U_{\alpha} \cap U_{\beta} \cap O \right) \circ V\left( \psi_{\alpha}|U_{\alpha} \cap U_{\beta} \cap O \right)\overset{\psi_{\alpha}^{- 1}|V\left( \psi_{\alpha}|U_{\alpha} \cap U_{\beta} \right)}{\rightarrow}U_{\alpha} \cap U_{\beta} \cap O\\
&= V\left( \psi_{\alpha}|U_{\alpha} \cap U_{\beta} \cap O \right)\overset{\psi_{\alpha}^{- 1}|V\left( \psi_{\alpha}|U_{\alpha} \cap U_{\beta} \right)}{\rightarrow}U_{\alpha} \cap U_{\beta} \cap O\overset{\psi_{\beta}|U_{\alpha} \cap U_{\beta}}{\rightarrow}V\left( \psi_{\beta}|U_{\alpha} \cap U_{\beta} \cap O \right)\\
&= V\left( \psi_{\alpha}|U_{\alpha} \cap U_{\beta} \cap O \right)\overset{\psi_{\beta}|U_{\alpha} \cap U_{\beta} \circ \psi_{\alpha}^{- 1}|V\left( \psi_{\alpha}|U_{\alpha} \cap U_{\beta} \right)}{\rightarrow}V\left( \psi_{\beta}|U_{\alpha} \cap U_{\beta} \cap O \right)\\
&= V\left( \psi_{\alpha}|U_{\alpha} \cap U_{\beta} \cap O \right)\overset{f_{\left( U_{\alpha},\psi_{\alpha} \right) \rightarrow \left( U_{\beta},\psi_{\beta} \right)}}{\rightarrow}V\left( \psi_{\beta}|U_{\alpha} \cap U_{\beta} \cap O \right)\\
&= f_{\left( U_{\alpha},\psi_{\alpha} \right) \rightarrow \left( U_{\beta},\psi_{\beta} \right)}|V\left( \psi_{\alpha}|U_{\alpha} \cap U_{\beta} \cap O \right)
\end{align*}
\end{comment}
\begin{align*}
f_{\left( U_{\alpha} \cap O,\psi_{\alpha}|U_{\alpha} \cap O \right) \rightarrow \left( U_{\beta} \cap O,\psi_{\beta}|U_{\beta} \cap O \right)} &= \left( \psi_{\beta}|U_{\beta} \cap O \right)|U_{\alpha} \cap U_{\beta} \cap O \circ \left( \psi_{\alpha}|U_{\alpha} \cap O \right)^{- 1}|V\left( \psi_{\alpha}|U_{\alpha} \cap U_{\beta} \cap O \right)\\
&= \psi_{\beta}|U_{\beta} \cap O \circ \left( \psi_{\alpha}|U_{\alpha} \cap O \right)^{- 1}|V\left( \psi_{\alpha}|U_{\alpha} \cap U_{\beta} \cap O \right)\\
&= f_{\left( U_{\alpha},\psi_{\alpha} \right) \rightarrow \left( U_{\beta},\psi_{\beta} \right)}|V\left( \psi_{\alpha}|U_{\alpha} \cap U_{\beta} \cap O \right)
\end{align*}
その関数$f_{\left( U_{\alpha} \cap O,\psi_{\alpha}|U_{\alpha} \cap O \right) \rightarrow \left( U_{\beta} \cap O,\psi_{\beta}|U_{\beta} \cap O \right)}$も$C^{r}$級関数である。同様にして、その関数$f_{\left( U_{\beta} \cap O,\psi_{\beta}|U_{\beta} \cap O \right) \rightarrow \left( U_{\alpha} \cap O,\psi_{\alpha}|U_{\alpha} \cap O \right)}$も$C^{r}$級関数であることが示される。よって、その部分位相空間$\left( O,\mathfrak{O}_{O} \right)$が$n$次元$C^{r}$多様体でその$C^{r}$級座標近傍系が$\left\{ \left( U_{\alpha} \cap O,\ \ \psi_{\alpha}|U_{\alpha} \cap O \right) \right\}_{\alpha \in A}$と与えられる。
\end{proof}
%\hypertarget{ux7a4dux591aux69d8ux4f53}{%
\subsubsection{積多様体}%\label{ux7a4dux591aux69d8ux4f53}}
\begin{thm}\label{8.3.1.9}
添数集合$\varLambda_{n}$によって添数づけられた$n_{i}$次元$C^{r}$多様体$\left( \mathcal{M}_{i},\mathfrak{O}_{i} \right)$の族$\left\{ \left( \mathcal{M}_{i},\ \ \mathfrak{O}_{i} \right) \right\}_{i \in \varLambda_{n}}$、これの座標近傍系$\left\{ \left( U_{\alpha_{i}},\ \ \psi_{\alpha_{i}} \right) \right\}_{\alpha_{i} \in A_{i}}$が与えられたとき、次のような写像$\prod_{i \in \varLambda_{n}} \psi_{\alpha_{i}}$が考えられれば、
\begin{align*}
\prod_{i \in \varLambda_{n}} \psi_{\alpha_{i}}:\prod_{i \in \varLambda_{n}} U_{\alpha_{i}} \rightarrow \prod_{i \in \varLambda_{n}} {V\left( \psi_{\alpha_{i}}|U_{\alpha_{i}} \right)};\left( p_{i} \right)_{i \in \varLambda_{n}} \mapsto \left( \psi_{\alpha_{i}}\left( p_{i} \right) \right)_{i \in \varLambda_{n}}
\end{align*}
その直積位相空間$\left( \prod_{i \in \varLambda_{n}} \mathcal{M}_{i},\ \ \mathfrak{O}_{0} \right)$はその族$\left\{ \left( \prod_{i \in \varLambda_{n}} U_{\alpha_{i}},\ \ \prod_{i \in \varLambda_{n}} \psi_{\alpha_{i}} \right) \right\}_{\forall i \in \varLambda_{n}\left[ \alpha_{i} \in A_{i} \right]}$をこれの$C^{r}$級座標近傍系とする$\sum_{i \in \varLambda_{n}} n_{i}$次元$C^{r}$多様体となる。
\end{thm}
\begin{dfn}
$n_{i}$次元$C^{r}$多様体$\left( \mathcal{M}_{i},\mathfrak{O}_{i} \right)$の族$\left\{ \left( \mathcal{M}_{i},\ \ \mathfrak{O}_{i} \right) \right\}_{i \in \varLambda_{n}}$、これの座標近傍系$\left\{ \left( U_{\alpha_{i}},\ \ \psi_{\alpha_{i}} \right) \right\}_{\alpha_{i} \in A_{i}}$が与えられ次のような写像$\prod_{i \in \varLambda_{n}} \psi_{\alpha_{i}}$が考えられたとき、
\begin{align*}
\prod_{i \in \varLambda_{n}} \psi_{\alpha_{i}}:\prod_{i \in \varLambda_{n}} U_{\alpha_{i}} \rightarrow \prod_{i \in \varLambda_{n}} {V\left( \psi_{\alpha_{i}}|U_{\alpha_{i}} \right)};\left( p_{i} \right)_{i \in \varLambda_{n}} \mapsto \left( \psi_{\alpha_{i}}\left( p_{i} \right) \right)_{i \in \varLambda_{n}}
\end{align*}
その族$\left\{ \left( \prod_{i \in \varLambda_{n}} U_{\alpha_{i}},\ \ \prod_{i \in \varLambda_{n}} \psi_{\alpha_{i}} \right) \right\}_{\forall i \in \varLambda_{n}\left[ \alpha_{i} \in A_{i} \right]}$をこれの$C^{r}$級座標近傍系とする$\sum_{i \in \varLambda_{n}} n_{i}$次元$C^{r}$多様体$\left( \prod_{i \in \varLambda_{n}} \mathcal{M}_{i},\ \ \mathfrak{O}_{0} \right)$をその族$\left\{ \left( \mathcal{M}_{i},\ \ \mathfrak{O}_{i} \right) \right\}_{i \in \varLambda_{n}}$の積多様体という。
\end{dfn}
\begin{proof}
添数集合$\varLambda_{n}$によって添数づけられた$n_{i}$次元$C^{r}$多様体$\left( \mathcal{M}_{i},\mathfrak{O}_{i} \right)$の族$\left\{ \left( \mathcal{M}_{i},\ \ \mathfrak{O}_{i} \right) \right\}_{i \in \varLambda_{n}}$、これの座標近傍系$\left\{ \left( U_{\alpha_{i}},\ \ \psi_{\alpha_{i}} \right) \right\}_{\alpha_{i} \in A_{i}}$が与えられたとし次のような写像$\prod_{i \in \varLambda_{n}} \psi_{\alpha_{i}}$が考えられよう。
\begin{align*}
\prod_{i \in \varLambda_{n}} \psi_{\alpha_{i}}:\prod_{i \in \varLambda_{n}} U_{\alpha_{i}} \rightarrow \prod_{i \in \varLambda_{n}} {V\left( \psi_{\alpha_{i}}|U_{\alpha_{i}} \right)};\left( p_{i} \right)_{i \in \varLambda_{n}} \mapsto \left( \psi_{\alpha_{i}}\left( p_{i} \right) \right)_{i \in \varLambda_{n}}
\end{align*}\par
まず、定理\ref{8.1.6.13}よりその直積位相空間$\left( \prod_{i \in \varLambda_{n}} \mathcal{M}_{i},\ \ \mathfrak{O}_{0} \right)$もHausdorff空間となる。次に、その直積位相空間$\left( \prod_{i \in \varLambda_{n}} \mathcal{M}_{i},\ \ \mathfrak{O}_{0} \right)$が$\sum_{i \in \varLambda_{n}} n_{i}$次元位相多様体でその座標近傍系が$\left\{ \left( \prod_{i \in \varLambda_{n}} U_{\alpha_{i}},\ \ \prod_{i \in \varLambda_{n}} \psi_{\alpha_{i}} \right) \right\}_{\forall i \in \varLambda_{n}\left[ \alpha_{i} \in A_{i} \right]}$と与えられることを示そう。仮定より$\forall\left( p_{i} \right)_{i \in \varLambda_{n}} \in \prod_{i \in \varLambda_{n}} \mathcal{M}_{i}\forall i \in \varLambda_{n}$に対し、$p_{i} \in \mathcal{M}_{i}$よりその点$p_{i}$のその位相空間$\left( \mathcal{M}_{i},\mathfrak{O}_{i} \right)$の意味でのある開近傍$U_{i}$と$n_{i}$次元Euclid空間$E^{n_{i}}$における位相空間$\left( \mathbb{R}^{n_{i}},\mathfrak{O}_{d_{E^{n_{i}}}} \right)$の開集合$E_{i}$が存在して、$\left( U_{i},\mathfrak{O}_{U_{i}} \right) \approx \left( E_{i},\left( \mathfrak{O}_{d_{E^{n_{i}}}} \right)_{E_{i}} \right)$が成り立つ。これにより、ある同相写像$\psi_{i}:U_{i} \rightarrow E_{i}$が存在しその組$\left( U_{i},\psi_{i} \right)$が局所座標となるのであった。このとき、次式のような写像$\prod_{i \in \varLambda_{n}} \psi_{i}$が考えられることにし、
\begin{align*}
\prod_{i \in \varLambda_{n}} \psi_{i}:\prod_{i \in \varLambda_{n}} U_{i} \rightarrow \prod_{i \in \varLambda_{n}} {V\left( \psi_{i}|U_{i} \right)};\left( p_{i} \right)_{i \in \varLambda_{n}} \mapsto \left( \psi_{i}\left( p_{i} \right) \right)_{i \in \varLambda_{n}}
\end{align*}
以下、次のようにおくことにする。
\begin{align*}
N = \sum_{i \in \varLambda_{n}} n_{i},\ \ \psi = \prod_{i \in \varLambda_{n}} \psi_{i},\ \ U = \prod_{i \in \varLambda_{n}} U_{i}
\end{align*}
その部分位相空間$\left( V\left( \psi|U \right),\ \ \left( \mathfrak{O}_{d_{E^{N}}} \right)_{V\left( \psi|U \right)} \right)$における初等開集合とその値域$V\left( \psi|U \right)$の積集合全体の集合がその位相$\left( \mathfrak{O}_{d_{E^{N}}} \right)_{V\left( \psi|U \right)}$の1つの開基となるので、$\forall E \in \left( \mathfrak{O}_{d_{E^{N}}} \right)_{V\left( \psi|U \right)}$に対し、任意の添数集合$M$によって添数づけられたある初等開集合とその値域$V\left( \psi|U \right)$の積集合たちの族$\left\{ \prod_{i \in \varLambda_{n}} E_{\mu i} \cap V\left( \psi|U \right) \right\}_{\mu \in M}$を用いて$E = \bigcup_{\mu \in M} \left( \prod_{i \in \varLambda_{n}} E_{\mu i} \cap V\left( \psi|U \right) \right)$が成り立つ。その写像$\psi_{i}$が全単射であることに注意すれば、次のようになることから、
\begin{align*}
V\left( \psi^{- 1}|E \right) &= V\left( \psi^{- 1}|E \right)\\
&= V\left( \psi^{- 1}|\bigcup_{\mu \in M} \left( \prod_{i \in \varLambda_{n}} E_{\mu i} \cap V\left( \psi|U \right) \right) \right)\\
&= V\left( \psi^{- 1}|\bigcup_{\mu \in M} \left( \prod_{i \in \varLambda_{n}} E_{\mu i} \cap \prod_{i \in \varLambda_{n}} {V\left( \psi_{i}|U_{i} \right)} \right) \right)\\
&= \bigcup_{\mu \in M} {V\left( \left( \prod_{i \in \varLambda_{n}} \psi_{i} \right)^{- 1}|\prod_{i \in \varLambda_{n}} \left( E_{\mu i} \cap V\left( \psi_{i}|U_{i} \right) \right) \right)}\\
&= \bigcup_{\mu \in M} {\prod_{i \in \varLambda_{n}} {V\left( \psi_{i}^{- 1}|E_{\mu i} \cap V\left( \psi_{i}|U_{i} \right) \right)}}\\
&= \bigcup_{\mu \in M} {\prod_{i \in \varLambda_{n}} \left( V\left( \psi_{i}^{- 1}|E_{\mu i} \right) \cap V\left( \psi_{i}^{- 1}|V\left( \psi_{i}|U_{i} \right) \right) \right)}\\
&= \bigcup_{\mu \in M} {\prod_{i \in \varLambda_{n}} \left( V\left( \psi_{i}^{- 1}|E_{\mu i} \right) \cap U_{i} \right)}\\
&= \bigcup_{\mu \in M} \left( \prod_{i \in \varLambda_{n}} {V\left( \psi_{i}^{- 1}|E_{\mu i} \right)} \cap \prod_{i \in \varLambda_{n}} U_{i} \right)\\
&= \bigcup_{\mu \in M} \left( \prod_{i \in \varLambda_{n}} {V\left( \psi_{i}^{- 1}|E_{\mu i} \right)} \cap U \right)\\
&= \bigcup_{\mu \in M} {\prod_{i \in \varLambda_{n}} {V\left( \psi_{i}^{- 1}|E_{\mu i} \right)}} \cap U
\end{align*}
$V\left( \psi_{i}^{- 1}|E_{\mu i} \right) \in \mathfrak{O}_{i}$に注意すれば、その直積集合$\prod_{i \in \varLambda_{n}} {V\left( \psi_{i}^{- 1}|E_{\mu i} \right)}$がその直積位相空間$\left( \prod_{i \in \varLambda_{n}} \mathcal{M}_{i},\ \ \mathfrak{O}_{0} \right)$における初等開集合でもあるので、次式が成り立つ。
\begin{align*}
V\left( \psi^{- 1}|E \right) = \bigcup_{\mu \in M} {\prod_{i \in \varLambda_{n}} {V\left( \psi_{i}^{- 1}|E_{\mu i} \right)}} \cap U \in \left( \mathfrak{O}_{0} \right)_{U}
\end{align*}
よって、その写像$\psi$は連続である。次に、次のような写像$\psi'$が考えられよう。
\begin{align*}
\psi':V\left( \psi|U \right) \rightarrow U;\left( a_{i} \right)_{i \in \varLambda_{n}} \mapsto \left( \psi_{i}^{- 1}\left( a_{i} \right) \right)_{i \in \varLambda_{n}}
\end{align*}
このとき、$\forall\left( p_{i} \right)_{i \in \varLambda_{n}} \in U$に対し、次のようになるかつ、
\begin{align*}
\psi' \circ \psi\left( p_{i} \right)_{i \in \varLambda_{n}} &= \psi'\left( \prod_{i \in \varLambda_{n}} \psi_{i}\left( p_{i} \right)_{i \in \varLambda_{n}} \right)\\
&= \psi'\left( \left( \psi_{i}\left( p_{i} \right) \right)_{i \in \varLambda_{n}} \right)\\
&= \left( \psi_{i}^{- 1}\left( \psi_{i}\left( p_{i} \right) \right) \right)_{i \in \varLambda_{n}}\\
&= \left( \psi_{i}^{- 1} \circ \psi_{i}\left( p_{i} \right) \right)_{i \in \varLambda_{n}}\\
&= \left( p_{i} \right)_{i \in \varLambda_{n}}
\end{align*}
$\forall\left( a_{i} \right)_{i \in \varLambda_{n}} \in V\left( \psi|U \right)$に対し、次のようになることから、
\begin{align*}
\psi \circ \psi'\left( a_{i} \right)_{i \in \varLambda_{n}} &= \prod_{i \in \varLambda_{n}} \psi_{i}\left( \psi'\left( a_{i} \right)_{i \in \varLambda_{n}} \right)\\
&= \prod_{i \in \varLambda_{n}} \psi_{i}\left( \psi_{i}^{- 1}\left( a_{i} \right) \right)_{i \in \varLambda_{n}}\\
&= \left( \psi_{i}\left( \psi_{i}^{- 1}\left( a_{i} \right) \right) \right)_{i \in \varLambda_{n}}\\
&= \left( \psi_{i} \circ \psi_{i}^{- 1}\left( a_{i} \right) \right)_{i \in \varLambda_{n}}\\
&= \left( a_{i} \right)_{i \in \varLambda_{n}}
\end{align*}
$\psi' = \psi^{- 1}$が得られる。ゆえに、その写像$\psi$は全単射である。もちろん、その写像$\psi$の逆写像$\psi^{- 1}$が存在する。先ほどと同様にして考えれば、その逆写像$\psi^{- 1}$も連続であることが示される。実際、その直積位相空間$\left( \prod_{i \in \varLambda_{n}} \mathcal{M}_{i},\ \ \mathfrak{O}_{0} \right)$における初等開集合とその集合$U$の積集合全体の集合がその位相$\left( \mathfrak{O}_{0} \right)_{U}$の1つの開基となるので、$\forall O \in \left( \mathfrak{O}_{0} \right)_{U}$に対し、任意の添数集合$\varLambda$によって添数づけられたある初等開集合とその集合$U$の積集合たちの族$\left\{ \prod_{i \in \varLambda_{n}} O_{\lambda i} \cap U \right\}_{\lambda \in \varLambda}$を用いて$O = \bigcup_{\lambda \in \varLambda} \left( \prod_{i \in \varLambda_{n}} O_{\lambda i} \cap U \right)$が成り立つ。これにより、次のようになることから、
\begin{align*}
V\left( \psi|O \right) &= V\left( \psi|O \right)\\
&= V\left( \psi|\bigcup_{\lambda \in \varLambda} \left( \prod_{i \in \varLambda_{n}} O_{\lambda i} \cap U \right) \right)\\
&= V\left( \psi|\bigcup_{\lambda \in \varLambda} {\prod_{i \in \varLambda_{n}} O_{\lambda i}} \cap U \right)\\
&= V\left( \psi|\bigcup_{\lambda \in \varLambda} {\prod_{i \in \varLambda_{n}} O_{\lambda i}} \right) \cap V\left( \psi|U \right)\\
&= \bigcup_{\lambda \in \varLambda} {V\left( \prod_{i \in \varLambda_{n}} \psi_{i}|\prod_{i \in \varLambda_{n}} O_{\lambda i} \right)} \cap V\left( \psi|U \right)\\
&= \bigcup_{\lambda \in \varLambda} {\prod_{i \in \varLambda_{n}} {V\left( \psi_{i}|O_{\lambda i} \right)}} \cap V\left( \psi|U \right)
\end{align*}
$V\left( \psi_{i}|O_{\lambda i} \right) \in \mathfrak{O}_{d_{E^{n_{i}}}}$に注意すれば、その直積集合$\prod_{i \in \varLambda_{n}} {V\left( \psi_{i}|O_{\lambda i} \right)}$がその位相空間$\left( \mathbb{R}^{N},\ \ \mathfrak{O}_{d_{E^{N}}} \right)$における初等開集合でもあるので、次式が成り立つ。
\begin{align*}
V\left( \psi|O \right) = \bigcup_{\lambda \in \varLambda} {\prod_{i \in \varLambda_{n}} {V\left( \psi_{i}|O_{\lambda i} \right)}} \cap V\left( \psi|U \right) \in \left( \mathfrak{O}_{d_{E^{N}}} \right)_{V\left( \psi|U \right)}
\end{align*}
よって、その逆写像$\psi^{- 1}$も連続である。以上の議論により、その写像$\psi$がその部分位相空間$\left( U,\left( \mathfrak{O}_{0} \right)_{U} \right)$からその部分位相空間$\left( V\left( \psi|U \right),\ \ \left( \mathfrak{O}_{d_{E^{N}}} \right)_{V\left( \psi|U \right)} \right)$への同相写像となるので、$\left( U,\left( \mathfrak{O}_{0} \right)_{U} \right) \approx \left( V\left( \psi|U \right),\ \ \left( \mathfrak{O}_{d_{E^{N}}} \right)_{V\left( \psi|U \right)} \right)$が成り立つ。ゆえに、その直積位相空間$\left( \prod_{i \in \varLambda_{n}} \mathcal{M}_{i},\ \ \mathfrak{O}_{0} \right)$が$N$次元位相多様体であることが分かり局所座標としてその組$(U,\psi)$が挙げられる。さらに、次式が成り立つことから、
\begin{align*}
\bigcup_{\forall i \in \varLambda_{n}\left[ \alpha_{i} \in A_{i} \right]} {\prod_{i \in \varLambda_{n}} U_{\alpha_{i}}} = \prod_{i \in \varLambda_{n}} {\bigcup_{\alpha_{i} \in A_{i}} U_{\alpha_{i}}} = \prod_{i \in \varLambda_{n}} \mathcal{M}_{i}
\end{align*}
その座標近傍系が$\left\{ \left( \prod_{i \in \varLambda_{n}} U_{\alpha_{i}},\ \ \prod_{i \in \varLambda_{n}} \psi_{\alpha_{i}} \right) \right\}_{\forall i \in \varLambda_{n}\left[ \alpha_{i} \in A_{i} \right]}$と与えられる。\par
最後に、その直積位相空間$\left( \prod_{i \in \varLambda_{n}} \mathcal{M}_{i},\ \ \mathfrak{O}_{0} \right)がN$次元$C^{r}$多様体でその$C^{r}$級座標近傍系が$\left\{ \left( \prod_{i \in \varLambda_{n}} U_{\alpha_{i}},\ \ \prod_{i \in \varLambda_{n}} \psi_{\alpha_{i}} \right) \right\}_{\forall i \in \varLambda_{n}\left[ \alpha_{i} \in A_{i} \right]}$と与えられることを示そう。$\forall i \in \varLambda_{n}\forall\alpha_{i},\beta_{i} \in A_{i}$に対し、$\prod_{i \in \varLambda_{n}} U_{\alpha_{i}} \cap \prod_{i \in \varLambda_{n}} U_{\beta_{i}} \neq \emptyset$が成り立つなら、次のようになることから、
\begin{align*}
\emptyset \neq \prod_{i \in \varLambda_{n}} U_{\alpha_{i}} \cap \prod_{i \in \varLambda_{n}} U_{\beta_{i}} = \prod_{i \in \varLambda_{n}} \left( U_{\alpha_{i}} \cap U_{\beta_{i}} \right)
\end{align*}
$\forall i \in \varLambda_{n}$に対し、$U_{\alpha_{i}} \cap U_{\beta_{i}} \neq \emptyset$が成り立つので、仮定よりそれらの座標変換たち$f_{\left( U_{\alpha_{i}},\psi_{\alpha_{i}} \right) \rightarrow \left( U_{\beta_{i}},\psi_{\beta_{i}} \right)}$がそれらの定義域で$C^{r}$級関数である。このとき、以下、次のようにおくことにすると、
\begin{align*}
\psi_{\alpha} = \prod_{i \in \varLambda_{n}} \psi_{\alpha_{i}},\ \ \psi_{\beta} = \prod_{i \in \varLambda_{n}} \psi_{\beta_{i}},\ \ U_{\alpha} = \prod_{i \in \varLambda_{n}} U_{\alpha_{i}},\ \ U_{\beta} = \prod_{i \in \varLambda_{n}} U_{\beta_{i}}
\end{align*}
上記の議論と同様にして、次のようになることから、
\begin{align*}
\psi_{\alpha}^{- 1}|V\left( \psi_{\alpha}|U_{\alpha} \cap U_{\beta} \right):V\left( \psi_{\alpha}|U_{\alpha} \cap U_{\beta} \right) \rightarrow U_{\alpha} \cap U_{\beta};\left( \mathbf{p}_{i} \right)_{i \in \varLambda_{n}} \mapsto \left( \psi_{\alpha_{i}}^{- 1}\left( \mathbf{p}_{i} \right) \right)_{i \in \varLambda_{n}}
\end{align*}
$\forall\left( \mathbf{p}_{i} \right)_{i \in \varLambda_{n}} \in V\left( \psi_{\alpha}|U_{\alpha} \cap U_{\beta} \right)$に対し、次のようになる。
\begin{align*}
f_{\left( U_{\alpha},\psi_{\alpha} \right) \rightarrow \left( U_{\beta},\psi_{\beta} \right)}\left( \mathbf{p}_{i} \right)_{i \in \varLambda_{n}} &= \psi_{\beta}|U_{\alpha} \cap U_{\beta} \circ \psi_{\alpha}^{- 1}|V\left( \psi_{\alpha}|U_{\alpha} \cap U_{\beta} \right)\left( \mathbf{p}_{i} \right)_{i \in \varLambda_{n}}\\
&= \psi_{\beta}|U_{\alpha} \cap U_{\beta}\left( \psi_{\alpha}^{- 1}|V\left( \psi_{\alpha}|U_{\alpha} \cap U_{\beta} \right)\left( \mathbf{p}_{i} \right)_{i \in \varLambda_{n}} \right)\\
&= \psi_{\beta}|U_{\alpha} \cap U_{\beta}\left( \psi_{\alpha_{i}}^{- 1}\left( \mathbf{p}_{i} \right) \right)_{i \in \varLambda_{n}}\\
&= \left( \psi_{\beta_{i}} \circ \psi_{\alpha_{i}}^{- 1}\left( \mathbf{p}_{i} \right) \right)_{i \in \varLambda_{n}}\\
&= \left( \psi_{\beta_{i}}|U_{\alpha_{i}} \cap U_{\beta_{i}}\left( \psi_{\alpha_{i}}^{- 1}\left( \mathbf{p}_{i} \right) \right) \right)_{i \in \varLambda_{n}}\\
&= \left( \psi_{\beta_{i}}|U_{\alpha_{i}} \cap U_{\beta_{i}}\left( \psi_{\alpha_{i}}^{- 1}|V\left( \psi_{\alpha_{i}}|U_{\alpha_{i}} \cap U_{\beta_{i}} \right)\left( \mathbf{p}_{i} \right) \right) \right)_{i \in \varLambda_{n}}\\
&= \left( \psi_{\beta_{i}}|U_{\alpha_{i}} \cap U_{\beta_{i}} \circ \psi_{\alpha_{i}}^{- 1}|V\left( \psi_{\alpha_{i}}|U_{\alpha_{i}} \cap U_{\beta_{i}} \right)\left( \mathbf{p}_{i} \right) \right)_{i \in \varLambda_{n}}\\
&= \left( f_{\left( U_{\alpha_{i}},\psi_{\alpha_{i}} \right) \rightarrow \left( U_{\beta_{i}},\psi_{\beta_{i}} \right)}\left( \mathbf{p}_{i} \right) \right)_{i \in \varLambda_{n}}
\end{align*}
したがって、その関数$f_{\left( U_{\alpha},\psi_{\alpha} \right) \rightarrow \left( U_{\beta},\psi_{\beta} \right)}$も$C^{r}$級関数である。同様にして、その関数$f_{\left( U_{\beta},\psi_{\beta} \right) \rightarrow \left( U_{\alpha},\psi_{\alpha} \right)}$も$C^{r}$級関数であることが示される。よって、その直積位相空間$\left( \prod_{i \in \varLambda_{n}} \mathcal{M}_{i},\ \ \mathfrak{O}_{0} \right)がN$次元$C^{r}$多様体でその$C^{r}$級座標近傍系が$\left\{ \left( \prod_{i \in \varLambda_{n}} U_{\alpha_{i}},\ \ \prod_{i \in \varLambda_{n}} \psi_{\alpha_{i}} \right) \right\}_{\forall i \in \varLambda_{n}\left[ \alpha_{i} \in A_{i} \right]}$と与えられる。
\end{proof}
\begin{thebibliography}{50}
\bibitem{1}
  松島与三, 多様体入門, 裳華房, 1965. 第36刷 p24-31 ISBN978-4-7853-1305-0
\bibitem{2}
  松本幸夫, 多様体の基礎, 東京大学出版会, 1988. 第16刷 p37-53 ISBN4-13-062103-3
\bibitem{3}
  松坂和夫, 集合・位相入門, 岩波書店, 1968. 第14刷 p51 ISBN978-4-00-029871-1
\bibitem{4}
  杉浦光夫, 解析入門I, 東京大学出版会, 1980. 第5刷 p128-130 ISBN4-13-062005-3
\end{thebibliography}
\end{document}
