\documentclass[dvipdfmx]{jsarticle}
\setcounter{section}{1}
\setcounter{subsection}{2}
\usepackage{amsmath,amsfonts,amssymb,array,comment,mathtools,url,docmute}
\usepackage{longtable,booktabs,dcolumn,tabularx,mathtools,multirow,colortbl,xcolor}
\usepackage[dvipdfmx]{graphics}
\usepackage{bmpsize}
\usepackage{amsthm}
\usepackage{enumitem}
\setlistdepth{20}
\renewlist{itemize}{itemize}{20}
\setlist[itemize]{label=•}
\renewlist{enumerate}{enumerate}{20}
\setlist[enumerate]{label=\arabic*.}
\setcounter{MaxMatrixCols}{20}
\setcounter{tocdepth}{3}
\newcommand{\rotin}{\text{\rotatebox[origin=c]{90}{$\in $}}}
\renewcommand{\thesection}{第\arabic{section}部}
\renewcommand{\thesubsection}{\arabic{section}.\arabic{subsection}}
\renewcommand{\thesubsubsection}{\arabic{section}.\arabic{subsection}.\arabic{subsubsection}}
\everymath{\displaystyle}
\allowdisplaybreaks[4]
\usepackage{vtable}
\theoremstyle{definition}
\newtheorem{thm}{定理}[subsection]
\newtheorem*{thm*}{定理}
\newtheorem{dfn}{定義}[subsection]
\newtheorem*{dfn*}{定義}
\newtheorem{axs}[dfn]{公理}
\newtheorem*{axs*}{公理}
\renewcommand{\headfont}{\bfseries}
\makeatletter
  \renewcommand{\section}{%
    \@startsection{section}{1}{\z@}%
    {\Cvs}{\Cvs}%
    {\normalfont\huge\headfont\raggedright}}
\makeatother
\makeatletter
  \renewcommand{\subsection}{%
    \@startsection{subsection}{2}{\z@}%
    {0.5\Cvs}{0.5\Cvs}%
    {\normalfont\LARGE\headfont\raggedright}}
\makeatother
\makeatletter
  \renewcommand{\subsubsection}{%
    \@startsection{subsubsection}{3}{\z@}%
    {0.4\Cvs}{0.4\Cvs}%
    {\normalfont\Large\headfont\raggedright}}
\makeatother
\makeatletter
\renewenvironment{proof}[1][\proofname]{\par
  \pushQED{\qed}%
  \normalfont \topsep6\p@\@plus6\p@\relax
  \trivlist
  \item\relax
  {
  #1\@addpunct{.}}\hspace\labelsep\ignorespaces
}{%
  \popQED\endtrivlist\@endpefalse
}
\makeatother
\renewcommand{\proofname}{\textbf{証明}}
\usepackage{tikz,graphics}
\usepackage[dvipdfmx]{hyperref}
\usepackage{pxjahyper}
\hypersetup{
 setpagesize=false,
 bookmarks=true,
 bookmarksdepth=tocdepth,
 bookmarksnumbered=true,
 colorlinks=false,
 pdftitle={},
 pdfsubject={},
 pdfauthor={},
 pdfkeywords={}}
\begin{document}
%\hypertarget{ux884cux5217}{%
\subsection{行列}%\label{ux884cux5217}}
%\hypertarget{ux884cux5217-1}{%
\subsubsection{行列}%\label{ux884cux5217-1}}
\begin{dfn}
1つの可換環を$R$と、2つの空集合でない集合たちを$I$、$J$とおくとき、その2つの集合たち$I$、$J$の直積$I \times J$からその可換環$R$への写像$a:I \times J \rightarrow R;(i,j) \mapsto a_{ij}$によって得られるその可換環$R$の元$a_{ij}$全体の順序付けられた組をその可換環$R$における$(I,J)$型の行列といい、$\left( a_{ij} \right)_{(i,j) \in I \times J}$、または単に、$\left( a_{ij} \right)$と表される。その2つの集合たち$I$、$J$がそれぞれ${\#}I$つ、${\#}J$つの元からなる有限集合であるとき、$(I,J)$型の行列を$\left( {\#}I,{\#}J \right)$型の行列、${\#}I \times {\#}J$型の行列などという。これに対比してその可換環$R$の元をscalar、無向量などという。特に、その2つの集合たち$I$、$J$は、写像$f$を適切に定めれば、体$K$の元$a_{ij}$を定めるのに任意性があるので、自然数$m$、$n$を用いてその2つの集合たち$I$、$J$を次式のようにおいても一般性が失われることはない。
\begin{align*}
\left\{ \begin{matrix}
\varLambda_{m} = \left\{ i \in \mathbb{N} \middle| 1 \leq i \leq m \right\} = I \\
\varLambda_{n} = \left\{ j \in \mathbb{N} \middle| 1 \leq j \leq m \right\} = J \\
\end{matrix} \right.\ 
\end{align*}
このとき、$\left( \varLambda_{m},\varLambda_{n} \right)$型の行列は$(m,n)$型の行列、$m \times n$型の行列にあたり、$\left( a_{ij} \right)_{(i,j) \in \varLambda_{m} \times \varLambda_{n}}$を次式のようにも書き、その元$a_{ij}$が明示的に書かれているものを成分表示された行列などといい、その元$a_{ij}$はその可換環$R$に属し、これを$(i,j)$成分という。また、これが属する集合を$M_{mn}(R)$、$R^{m \times n}$と書き、特に$m = n$ならば単に$M_{n}(R)$、$R^{n^{2}}$とも書く。
\begin{align*}
\left( a_{ij} \right)_{(i,j) \in \varLambda_{m} \times \varLambda_{n}} = A_{mn} = \left( a_{ij} \right)_{1 \leq i \leq m,1 \leq j \leq n} = \begin{pmatrix}
\begin{matrix}
a_{11} & a_{12} \\
a_{21} & a_{22} \\
\end{matrix} & \begin{matrix}
\cdots & a_{1n} \\
\cdots & a_{2n} \\
\end{matrix} \\
\begin{matrix}
 \vdots & \vdots \\
a_{m1} & a_{m2} \\
\end{matrix} & \begin{matrix}
 \ddots & \vdots \\
\cdots & a_{mn} \\
\end{matrix} \\
\end{pmatrix}
\end{align*}
\end{dfn}\par
以下、行列たち$A_{mn}$、$B_{mn}$、$C_{mn}$がそれぞれ$\left( a_{ij} \right)_{(i,j) \in \varLambda_{m} \times \varLambda_{n}}$、$\left( b_{ij} \right)_{(i,j) \in \varLambda_{m} \times \varLambda_{n}}$、$\left( c_{ij} \right)_{(i,j) \in \varLambda_{m} \times \varLambda_{n}}$と成分表示されるとする。このとき、$\forall A_{mn},B_{mn} \in M_{mn}(R)$に対し、$A_{mn} = B_{mn}$が成り立つことは、$\forall(i,j) \in \varLambda_{m} \times \varLambda_{n}$に対し、$a_{ij} = b_{ij}$が成り立つことと同値であることに注意されたい。
\begin{dfn}
任意の2つの$\left( \varLambda_{m},\varLambda_{n} \right)$型の行列$\left( a_{ij} \right)_{(i,j) \in \varLambda_{m} \times \varLambda_{n}}$、$\left( b_{ij} \right)_{(i,j) \in \varLambda_{m} \times \varLambda_{n}}$と可換環$R$の任意の元$k$、$l$を用いて、次式のように定義される。
\begin{align*}
kA_{mn} + lB_{mn} = \left( ka_{ij} + lb_{ij} \right)_{(i,j) \in \varLambda_{m} \times \varLambda_{n}}
\end{align*}
\end{dfn}
\begin{thm}\label{2.1.3.1}
体$K$上の集合$M_{mn}(K)$は体$K$上のvector空間であり行列はvectorとなる。
\end{thm}
\begin{proof}
体$K$上の集合$M_{mn}(K)$について定義より$\forall\left( a_{ij} \right)_{(i,j) \in \varLambda_{m} \times \varLambda_{n}},\left( b_{ij} \right)_{(i,j) \in \varLambda_{m} \times \varLambda_{n}} \in M_{mn}(K)$に対し、$\left( a_{ij} \right)_{(i,j) \in \varLambda_{m} \times \varLambda_{n}} + \left( b_{ij} \right)_{(i,j) \in \varLambda_{m} \times \varLambda_{n}} = \left( a_{ij} + b_{ij} \right)_{(i,j) \in \varLambda_{m} \times \varLambda_{n}}$が成り立つ。このとき、体$K$が加法について群$(K, + )$をなし、$a_{ij} + b_{ij}$のみ着眼して考えると、各成分も加法について可換群$(K, + )$をなしているので、$M_{mn}(K)$は加法について可換群$\left( M_{mn}(K), + \right)$をなす。\par
また、定義より$\forall k \in K\forall\left( a_{ij} \right)_{(i,j) \in \varLambda_{m} \times \varLambda_{n}} \in M_{mn}(K)$に対し、$k\left( a_{ij} \right)_{(i,j) \in \varLambda_{m} \times \varLambda_{n}} = \left( ka_{ij} \right)_{(i,j) \in \varLambda_{m} \times \varLambda_{n}}$が成り立ち${ka}_{ij} \in K$が成り立つので、$\left( ka_{ij} \right)_{(i,j) \in \varLambda_{m} \times \varLambda_{n}} \in M_{mn}(K)$が成り立ち写像$\mu_{2}:K \times M_{mn}(K) \rightarrow M_{mn}(K);\left( k,\left( a_{ij} \right)_{(i,j) \in \varLambda_{m} \times \varLambda_{n}} \right) \mapsto \left( ka_{ij} \right)_{(i,j) \in \varLambda_{m} \times \varLambda_{n}}$が定義される。\par
さらに、$\forall k \in K\forall\left( a_{ij} \right)_{(i,j) \in \varLambda_{m} \times \varLambda_{n}},\left( b_{ij} \right)_{(i,j) \in \varLambda_{m} \times \varLambda_{n}} \in M_{mn}(K)$に対し、次のようになる。
\begin{align*}
k\left( \left( a_{ij} \right)_{(i,j) \in \varLambda_{m} \times \varLambda_{n}} + \left( b_{ij} \right)_{(i,j) \in \varLambda_{m} \times \varLambda_{n}} \right) &= k\left( a_{ij} + b_{ij} \right)_{(i,j) \in \varLambda_{m} \times \varLambda_{n}}\\
&= \left( k\left( a_{ij} + b_{ij} \right) \right)_{(i,j) \in \varLambda_{m} \times \varLambda_{n}}\\
&= \left( ka_{ij} + kb_{ij} \right)_{(i,j) \in \varLambda_{m} \times \varLambda_{n}}\\
&= \left( ka_{ij} \right)_{i \in \varLambda_{n}}\mathbf{+}\left( kb_{ij} \right)_{(i,j) \in \varLambda_{m} \times \varLambda_{n}}\\
&= k\left( a_{ij} \right)_{(i,j) \in \varLambda_{m} \times \varLambda_{n}}\mathbf{+}k\left( b_{ij} \right)_{(i,j) \in \varLambda_{m} \times \varLambda_{n}}
\end{align*}
$\forall k,l \in K\forall\left( a_{ij} \right)_{(i,j) \in \varLambda_{m} \times \varLambda_{n}} \in M_{mn}(K)$に対し、次のようになる。
\begin{align*}
(k + l)\left( a_{ij} \right)_{(i,j) \in \varLambda_{m} \times \varLambda_{n}} &= \left( (k + l)a_{ij} \right)_{(i,j) \in \varLambda_{m} \times \varLambda_{n}}\\
&= \left( ka_{ij} + la_{ij} \right)_{(i,j) \in \varLambda_{m} \times \varLambda_{n}}\\
&= \left( ka_{ij} \right)_{(i,j) \in \varLambda_{m} \times \varLambda_{n}}\mathbf{+}\left( la_{ij} \right)_{(i,j) \in \varLambda_{m} \times \varLambda_{n}}\\
&= k\left( a_{ij} \right)_{(i,j) \in \varLambda_{m} \times \varLambda_{n}}\mathbf{+}l\left( a_{ij} \right)_{(i,j) \in \varLambda_{m} \times \varLambda_{n}}
\end{align*}
$\forall k,l \in K\forall\left( a_{ij} \right)_{(i,j) \in \varLambda_{m} \times \varLambda_{n}} \in M_{mn}(K)$に対し、次のようになる。
\begin{align*}
(kl)\left( a_{ij} \right)_{(i,j) \in \varLambda_{m} \times \varLambda_{n}} &= \left( (kl)a_{ij} \right)_{(i,j) \in \varLambda_{m} \times \varLambda_{n}}\\
&= \left( k\left( la_{ij} \right) \right)_{(i,j) \in \varLambda_{m} \times \varLambda_{n}}\\
&= {k\left( la_{ij} \right)}_{(i,j) \in \varLambda_{m} \times \varLambda_{n}} = k\left( l\left( a_{ij} \right)_{(i,j) \in \varLambda_{m} \times \varLambda_{n}} \right)
\end{align*}
$\exists 1 \in K\forall\left( a_{ij} \right)_{(i,j) \in \varLambda_{m} \times \varLambda_{n}} \in M_{mn}(K)$に対し、次のようになる。
\begin{align*}
1\left( a_{ij} \right)_{(i,j) \in \varLambda_{m} \times \varLambda_{n}} = \left( 1a_{ij} \right)_{(i,j) \in \varLambda_{m} \times \varLambda_{n}} = \left( a_{ij} \right)_{(i,j) \in \varLambda_{m} \times \varLambda_{n}}
\end{align*}\par
以上より、その集合$M_{mn}(K)$はvector空間の定義を満たしているので、vector空間である。
\end{proof}
\begin{dfn}
体$K$上で${\#}\varLambda = 1$なる添数集合$\varLambda$が集合$\left\{ \varLambda_{m},\varLambda_{n} \right\}$に存在すれば、即ち、その2つの集合たち$\varLambda_{m}$、$\varLambda_{n}$どちらかまたは両方とも元の個数が1になっても$\left( a_{ij} \right)_{(i,j) \in \varLambda_{m} \times \varLambda_{n}}$はvectorであり、$\left( \varLambda_{1},\varLambda_{n} \right)$型の行列$\begin{pmatrix}
a_{11} & a_{12} & \cdots & a_{1n} \\
\end{pmatrix}$、$\left( \varLambda_{m},\varLambda_{1} \right)$型の行列$\begin{pmatrix}
\begin{matrix}
a_{11} \\
a_{21} \\
\end{matrix} \\
\begin{matrix}
 \vdots \\
a_{m1} \\
\end{matrix} \\
\end{pmatrix}$をそれぞれ$n$-行vector、$m$-列vectorという。
\end{dfn}
\begin{dfn}
行列$A_{mn}$から取り出されたvector$\left( a_{ij} \right)_{(i,j) \in \left\{ i \right\} \times \varLambda_{n}}$、vector$\left( a_{ij} \right)_{(i,j) \in \varLambda_{n} \times \left\{ j \right\}}$をそれぞれその行列$A_{mn}$の第$i$行、第$j$列という。
\end{dfn}
\begin{dfn}
次式のように定義される行列$O_{mn}$を$\left( \varLambda_{m},\varLambda_{n} \right)$型の零行列、$(m,n)$型の零行列、$m \times n$型の零行列といい、特に$m = n$ならば$n$次零行列といい、単に$O_{n}$とも表す。
\begin{align*}
O_{mn} = (0)_{(i,j) \in \varLambda_{n} \times \varLambda_{n}}
\end{align*}
\end{dfn}
\begin{thm}\label{2.1.3.2}
可換環$R$上で次のことが成り立つ。
\begin{itemize}
\item
  $\forall A_{mn} \in M_{mn}(R)$に対し、$1A_{mn} = A_{mn}$が成り立つ。
\item
  $\forall A_{mn} \in M_{mn}(R)$に対し、$0A_{mn} = O_{mn}$が成り立つ。
\item
  $\forall A_{mn} \in M_{mn}(R)\forall k,l \in R$に対し、$(k + l)A_{mn} = kA_{mn} + lA_{mn}$が成り立つ。
\item
  $\forall A_{mn},B_{mn} \in M_{mn}(R)\forall k \in R$に対し、$k\left( A_{mn} + B_{mn} \right) = kA_{mn} + kB_{mn}$が成り立つ。
\item
  $\forall A_{mn} \in M_{mn}(R)\forall k,l \in R$に対し、$k\left( lA_{mn} \right) = (kl)A_{mn}$が成り立つ。
\item
  $\forall A_{mn},B_{mn} \in M_{mn}(R)$に対し、$A_{mn} + B_{mn} = B_{mn} + A_{mn}$が成り立つ。
\item
  $\forall A_{mn},B_{mn},C_{mn} \in M_{mn}(R)$に対し、$\left( A_{mn} + B_{mn} \right) + C_{mn} = A_{mn} + \left( B_{mn} + C_{mn} \right)$が成り立つ。
\item
  $\forall A_{mn} \in M_{mn}(R)$に対し、$A_{mn} + O_{mn} = A_{mn}$が成り立つ。
\item
  $\forall A_{mn} \in M_{mn}(R)$に対し、$A_{mn} - A_{mn} = O_{mn}$が成り立つ。
\end{itemize}
\end{thm}
\begin{proof}
定義より成分ごとで可換環の元とみなせる。
\end{proof}
\begin{dfn}
2つの行列$A_{lm}$、$B_{mn}$に対して次式のように行列の積を定義する。
\begin{align*}
A_{lm}B_{mn} = \left( \sum_{h \in \varLambda_{m}} {a_{ih}b_{hj}} \right)_{(i,j) \in \varLambda_{l} \times \varLambda_{n}}
\end{align*}
\end{dfn}
\begin{thm}\label{2.1.3.3}
可換環$R$上で次のことが成り立つ。
\begin{itemize}
\item
  $\forall A_{lm} \in M_{lm}(R)\forall B_{mn} \in M_{mn}(R)\forall C_{no} \in M_{no}(R)$に対し、$A_{lm}\left( B_{mn}C_{no} \right) = \left( A_{lm}B_{mn} \right)C_{no}$が成り立つ。
\item
  $\forall A_{lm} \in M_{lm}(R)\forall B_{mn},C_{mn} \in M_{mn}(R)$に対し、$A_{lm}\left( B_{mn} + C_{mn} \right) = A_{lm}B_{mn} + A_{lm}C_{mn}$が成り立つ。
\item
  $\forall A_{lm},B_{lm} \in M_{lm}(R)\forall C_{mn} \in M_{mn}(R)$に対し、$\left( A_{lm} + B_{lm} \right)C_{mn} = A_{lm}C_{mn} + A_{lm}C_{mn}$が成り立つ。
\item
  $\forall A_{lm} \in M_{lm}(R)\forall B_{mn} \in M_{mn}(R)\forall k \in R$に対し、$A_{lm}\left( kB_{mn} \right) = \left( kA_{lm} \right)B_{mn} = k\left( A_{lm}B_{mn} \right)$が成り立つ。
\end{itemize}\par
以上より、3つの行列同士の積、2つの行列同士の積の$k$倍は結合的であり、行列同士では積が和に対して右からも左からも分配的であるが、2つの行列同士の積は可換的でないことに注意されたい。
\end{thm}
\begin{proof}
可換環$R$上で$\forall A_{lm} \in M_{lm}(R)\forall B_{mn} \in M_{mn}(R)\forall C_{no} \in M_{no}(R)$に対し、次のようになる。
\begin{align*}
A_{lm}\left( B_{mn}C_{no} \right) &= \left( a_{ij} \right)_{(i,j) \in \varLambda_{l} \times \varLambda_{m}}\left( \left( b_{ij} \right)_{(i,j) \in \varLambda_{m} \times \varLambda_{n}}\left( c_{ij} \right)_{(i,j) \in \varLambda_{n} \times \varLambda_{o}} \right)\\
&= \left( a_{ij} \right)_{(i,j) \in \varLambda_{l} \times \varLambda_{m}}\left( \sum_{h \in \varLambda_{n}} {b_{ih}c_{hj}} \right)_{(i,j) \in \varLambda_{m} \times \varLambda_{o}}\\
&= \left( \sum_{g \in \varLambda_{m}} a_{ig}\sum_{h \in \varLambda_{n}} {b_{gh}c_{hj}} \right)_{(i,j) \in \varLambda_{l} \times \varLambda_{o}}\\
&= \left( \sum_{g \in \varLambda_{m}} {\sum_{h \in \varLambda_{n}} {a_{ig}b_{gh}c_{hj}}} \right)_{(i,j) \in \varLambda_{l} \times \varLambda_{o}}\\
&= \left( \sum_{h \in \varLambda_{n}} {\sum_{g \in \varLambda_{m}} {a_{ig}b_{gh}c_{hj}}} \right)_{(i,j) \in \varLambda_{l} \times \varLambda_{o}}\\
&= \left( \sum_{g \in \varLambda_{m}} {a_{ig}b_{gj}} \right)_{(i,j) \in \varLambda_{l} \times \varLambda_{n}}\left( c_{ij} \right)_{(i,j) \in \varLambda_{n} \times \varLambda_{o}}\\
&= \left( \left( a_{ij} \right)_{(i,j) \in \varLambda_{l} \times \varLambda_{m}}\left( b_{ij} \right)_{(i,j) \in \varLambda_{m} \times \varLambda_{n}} \right)\left( c_{ij} \right)_{(i,j) \in \varLambda_{n} \times \varLambda_{o}} \\
&= \left( A_{lm}B_{mn} \right)C_{no}
\end{align*}\par
$\forall A_{lm} \in M_{lm}(R)\forall B_{mn},C_{mn} \in M_{mn}(R)$に対し、次のようになる。
\begin{align*}
A_{lm}\left( B_{mn} + C_{mn} \right) &= \left( a_{ij} \right)_{(i,j) \in \varLambda_{l} \times \varLambda_{m}}\left( \left( b_{ij} \right)_{(i,j) \in \varLambda_{m} \times \varLambda_{n}} + \left( c_{ij} \right)_{(i,j) \in \varLambda_{m} \times \varLambda_{n}} \right)\\
&= \left( a_{ij} \right)_{(i,j) \in \varLambda_{l} \times \varLambda_{m}}\left( b_{ij} + c_{ij} \right)_{(i,j) \in \varLambda_{m} \times \varLambda_{n}}\\
&= \left( \sum_{h \in \varLambda_{m}} {a_{ih}\left( b_{hj} + c_{hj} \right)} \right)_{(i,j) \in \varLambda_{l} \times \varLambda_{n}}\\
&= \left( \sum_{h \in \varLambda_{m}} \left( a_{ih}b_{hj} + a_{ih}c_{hj} \right) \right)_{(i,j) \in \varLambda_{l} \times \varLambda_{n}}\\
&= \left( \sum_{h \in \varLambda_{m}} {a_{ih}b_{hj}} + \sum_{h \in \varLambda_{m}} {a_{ih}c_{hj}} \right)_{(i,j) \in \varLambda_{l} \times \varLambda_{n}}\\
&= \left( \sum_{h \in \varLambda_{m}} {a_{ih}b_{hj}} \right)_{(i,j) \in \varLambda_{l} \times \varLambda_{n}} + \left( \sum_{h \in \varLambda_{m}} {a_{ih}c_{hj}} \right)_{(i,j) \in \varLambda_{l} \times \varLambda_{n}}\\
&= \left( a_{ij} \right)_{(i,j) \in \varLambda_{l} \times \varLambda_{m}}\left( b_{ij} \right)_{(i,j) \in \varLambda_{m} \times \varLambda_{n}} + \left( a_{ij} \right)_{(i,j) \in \varLambda_{l} \times \varLambda_{m}}\left( c_{ij} \right)_{(i,j) \in \varLambda_{m} \times \varLambda_{n}}\\
&= A_{lm}B_{mn} + A_{lm}C_{mn}
\end{align*}\par
$\forall A_{lm},B_{lm} \in M_{lm}(R)\forall C_{mn} \in M_{mn}(R)$に対し、次のようになる。
\begin{align*}
\left( A_{lm} + B_{lm} \right)C_{mn} &= \left( \left( a_{ij} \right)_{(i,j) \in \varLambda_{l} \times \varLambda_{m}} + \left( b_{ij} \right)_{(i,j) \in \varLambda_{l} \times \varLambda_{m}} \right)\left( c_{ij} \right)_{(i,j) \in \varLambda_{m} \times \varLambda_{n}}\\
&= \left( a_{ij} + b_{ij} \right)_{(i,j) \in \varLambda_{l} \times \varLambda_{m}}\left( c_{ij} \right)_{(i,j) \in \varLambda_{m} \times \varLambda_{n}}\\
&= \left( \sum_{h \in \varLambda_{m}} {\left( a_{ih} + b_{ih} \right)c_{hj}} \right)_{(i,j) \in \varLambda_{l} \times \varLambda_{n}}\\
&= \left( \sum_{h \in \varLambda_{m}} \left( a_{ih}c_{hj} + b_{ih}c_{hj} \right) \right)_{(i,j) \in \varLambda_{l} \times \varLambda_{n}}\\
&= \left( \sum_{h \in \varLambda_{m}} {a_{ih}c_{hj}} + \sum_{h \in \varLambda_{m}} {b_{ih}c_{hj}} \right)_{(i,j) \in \varLambda_{l} \times \varLambda_{n}}\\
&= \left( \sum_{h \in \varLambda_{m}} {a_{ih}c_{hj}} \right)_{(i,j) \in \varLambda_{l} \times \varLambda_{n}} + \left( \sum_{h \in \varLambda_{m}} {b_{ih}c_{hj}} \right)_{(i,j) \in \varLambda_{l} \times \varLambda_{n}}\\
&= \left( a_{ij} \right)_{(i,j) \in \varLambda_{l} \times \varLambda_{m}}\left( c_{ij} \right)_{(i,j) \in \varLambda_{m} \times \varLambda_{n}} + \left( a_{ij} \right)_{(i,j) \in \varLambda_{l} \times \varLambda_{m}}\left( c_{ij} \right)_{(i,j) \in \varLambda_{m} \times \varLambda_{n}}\\
&= A_{lm}C_{mn} + A_{lm}C_{mn}
\end{align*}\par
$\forall A_{lm} \in M_{lm}(R)\forall B_{mn} \in M_{mn}(R)\forall k \in R$に対し、次のようになる。
\begin{align*}
A_{lm}\left( kB_{mn} \right) &= \left( a_{ij} \right)_{(i,j) \in \varLambda_{l} \times \varLambda_{m}}\left( k\left( b_{ij} \right)_{(i,j) \in \varLambda_{m} \times \varLambda_{n}} \right)\\
&= \left( a_{ij} \right)_{(i,j) \in \varLambda_{l} \times \varLambda_{m}}\left( kb_{ij} \right)_{(i,j) \in \varLambda_{m} \times \varLambda_{n}}\\
&= \left( \sum_{h \in \varLambda_{m}} {a_{ih}kb_{hj}} \right)_{(i,j) \in \varLambda_{l} \times \varLambda_{n}}\\
&= \left( \sum_{h \in \varLambda_{m}} {ka_{ih}b_{hj}} \right)_{(i,j) \in \varLambda_{l} \times \varLambda_{n}}\\
&= \left( ka_{ij} \right)_{(i,j) \in \varLambda_{l} \times \varLambda_{m}}\left( b_{ij} \right)_{(i,j) \in \varLambda_{m} \times \varLambda_{n}}\\
&= \left( k\left( a_{ij} \right)_{(i,j) \in \varLambda_{l} \times \varLambda_{m}} \right)\left( b_{ij} \right)_{(i,j) \in \varLambda_{m} \times \varLambda_{n}} \\
&= \left( kA_{lm} \right)B_{mn}\\
A_{lm}\left( kB_{mn} \right) &= \left( a_{ij} \right)_{(i,j) \in \varLambda_{l} \times \varLambda_{m}}\left( k\left( b_{ij} \right)_{(i,j) \in \varLambda_{m} \times \varLambda_{n}} \right)\\
&= \left( a_{ij} \right)_{(i,j) \in \varLambda_{l} \times \varLambda_{m}}\left( kb_{ij} \right)_{(i,j) \in \varLambda_{m} \times \varLambda_{n}}\\
&= \left( \sum_{h \in \varLambda_{m}} {a_{ih}kb_{hj}} \right)_{(i,j) \in \varLambda_{l} \times \varLambda_{n}}\\
&= \left( \sum_{h \in \varLambda_{m}} {ka_{ih}b_{hj}} \right)_{(i,j) \in \varLambda_{l} \times \varLambda_{n}}\\
&= k\left( \sum_{h \in \varLambda_{m}} {a_{ih}b_{hj}} \right)_{(i,j) \in \varLambda_{l} \times \varLambda_{n}}\\
&= k\left( \left( a_{ij} \right)_{(i,j) \in \varLambda_{l} \times \varLambda_{m}}\left( b_{ij} \right)_{(i,j) \in \varLambda_{m} \times \varLambda_{n}} \right) \\
&= k\left( A_{lm}B_{mn} \right)
\end{align*}
\end{proof}
%\hypertarget{ux3055ux307eux3056ux307eux306aux884cux5217}{%
\subsubsection{さまざまな行列}%\label{ux3055ux307eux3056ux307eux306aux884cux5217}}
\begin{dfn}
$\left( \varLambda_{m},\varLambda_{n} \right)$型の行列のうち$m = n$であるようなものを$n$次正方行列といい、$A_{nn}$を単に$A_{n}$とも表す。
\begin{align*}
A_{nn} = A_{n} = \left( a_{ij} \right)_{(i,j) \in \varLambda_{n}^{2}} = \begin{pmatrix}
a_{11} & a_{12} & \cdots & a_{1n} \\
a_{21} & a_{22} & \cdots & a_{2n} \\
 \vdots & \vdots & \ddots & \vdots \\
a_{n1} & a_{n2} & \cdots & a_{nn} \\
\end{pmatrix}
\end{align*}
この行列の成分たち$a_{ij}$のうち添数$i$と$j$が等しい成分たちのことを対角成分という。
\end{dfn}
\begin{dfn}
次式で定義される写像$\delta$の組$(i,j)$による像$\delta(i,j)$をKroneckerのdeltaといい$\delta_{ij}$と書く。
\begin{align*}
\delta:\varLambda_{m} \times \varLambda_{n} \rightarrow R;(i,j) \mapsto \left\{ \begin{matrix}
1 & \mathrm{if} & i = j \\
0 & \mathrm{if} & i \neq j \\
\end{matrix} \right.\ 
\end{align*}
\end{dfn}
\begin{dfn}
$n$次正方行列のうち対角成分$a_{ij}$以外の成分たちが全て0であるようなもの$\left( a_{in}\delta_{ij} \right)_{(i,j) \in \varLambda_{n}^{2}}$を$n$次対角行列という。
\begin{align*}
\left( a_{in}\delta_{ij} \right)_{(i,j) \in \varLambda_{n}^{2}} = \left( \left\{ \begin{matrix}
a_{ij} & \mathrm{if} & i = j \\
0 & \mathrm{if} & i \neq j \\
\end{matrix} \right.\  \right)_{(i,j) \in \varLambda_{n}^{2}} = \begin{pmatrix}
a_{11} & \  & \  & O \\
\  & a_{22} & \  & \  \\
\  & \  & \ddots & \  \\
O & \  & \  & a_{nn} \\
\end{pmatrix}
\end{align*}
\end{dfn}
\begin{dfn}
$n$次正方行列の成分たち$a_{ij}$のうち$i \geq j$なるもの以外の成分たちが全て0であるようなもの$\left( \left\{ \begin{matrix}
a_{ij} & \mathrm{if} & i \geq j \\
0 & \mathrm{if} & i < j \\
\end{matrix} \right.\  \right)_{(i,j) \in \varLambda_{n}^{2}}$を$n$次上三角行列という。
\begin{align*}
\left( \left\{ \begin{matrix}
a_{ij} & \mathrm{if} & i \geq j \\
0 & \mathrm{if} & i < j \\
\end{matrix} \right.\  \right)_{(i,j) \in \varLambda_{n}^{2}} = \begin{pmatrix}
a_{11} & \  & \  & * \\
\  & a_{22} & \  & \  \\
\  & \  & \ddots & \  \\
O & \  & \  & a_{nn} \\
\end{pmatrix}
\end{align*}
同様に$n$次正方行列の成分たち$a_{ij}$のうち$i \leq j$なるもの以外の成分たちが全て0であるようなもの$\left( \left\{ \begin{matrix}
a_{ij} & \mathrm{if} & i \leq j \\
0 & \mathrm{if} & i > j \\
\end{matrix} \right.\  \right)_{(i,j) \in \varLambda_{n}^{2}}$を$n$次下三角行列という。
\begin{align*}
\left( \left\{ \begin{matrix}
a_{ij} & \mathrm{if} & i \leq j \\
0 & \mathrm{if} & i > j \\
\end{matrix} \right.\  \right)_{(i,j) \in \varLambda_{n}^{2}} = \begin{pmatrix}
a_{11} & \  & \  & O \\
\  & a_{22} & \  & \  \\
\  & \  & \ddots & \  \\
* & \  & \  & a_{nn} \\
\end{pmatrix}
\end{align*}
\end{dfn}
\begin{dfn}
Kroneckerのdelta$\delta_{ij}$を用いて次式のように表される$n$次対角行列のうち対角成分たちが全て1であるような行列を$n$次単位行列といい$I_{n}$などと書く。
\begin{align*}
I_{n} = \left( \delta_{ij} \right)_{(i,j) \in \varLambda_{n}^{2}} = \begin{pmatrix}
1 & \  & \  & O \\
\  & 1 & \  & \  \\
\  & \  & \ddots & \  \\
O & \  & \  & 1 \\
\end{pmatrix}
\end{align*}
\end{dfn}
\begin{thm}\label{2.1.3.4}
$\left( \varLambda_{m},\varLambda_{n} \right)$型の行列$A_{mn}$、$m$次単位行列$I_{m}$、$n$次単位行列$I_{n}$において、次式が成り立つ。
\begin{align*}
I_{m}A_{mn} = A_{mn}I_{n} = A_{mn}
\end{align*}
\end{thm}
\begin{proof}
$\left( \varLambda_{m},\varLambda_{n} \right)$型の行列$A_{mn}$、$m$次単位行列$I_{m}$、$n$次単位行列$I_{n}$において、その$\left( \varLambda_{m},\varLambda_{n} \right)$型の行列$A_{m,n}$、$m$次単位行列$I_{m}$、$n$次単位行列$I_{n}$をKroneckerのdelta$\delta_{ij}$を用いてそれぞれ$\left( a_{ij} \right)_{(i,j) \in \varLambda_{m} \times \varLambda_{n}}$、$\left( \delta_{ij} \right)_{(i,j) \in \varLambda_{m}^{2}}$、$\left( \delta_{ij} \right)_{(i,j) \in \varLambda_{n}^{2}}$と成分表示されるとする。このとき、次のようになる。
\begin{align*}
I_{m}A_{mn} &= \left( \delta_{ij} \right)_{(i,j) \in \varLambda_{m}^{2}}\left( a_{ij} \right)_{(i,j) \in \varLambda_{m} \times \varLambda_{n}}\\
&= \left( \sum_{k \in \varLambda_{m}} {\delta_{ik}a_{kj}} \right)_{(i,j) \in \varLambda_{m} \times \varLambda_{n}}\\
&= \left( \delta_{ii}a_{ij} + \sum_{k \in \varLambda_{m} \setminus \left\{ i \right\}} {\delta_{ik}a_{kj}} \right)_{(i,j) \in \varLambda_{m} \times \varLambda_{n}}\\
&= \left( 1 \cdot a_{ij} + \sum_{k \in \varLambda_{m} \setminus \left\{ i \right\}} {0 \cdot a_{kj}} \right)_{(i,j) \in \varLambda_{m} \times \varLambda_{n}}\\
&= \left( a_{ij} + 0 \right)_{(i,j) \in \varLambda_{m} \times \varLambda_{n}}\\
&= \left( a_{ij} \right)_{(i,j) \in \varLambda_{m} \times \varLambda_{n}} = A_{mn}\\
A_{mn}I_{n} &= \left( a_{ij} \right)_{(i,j) \in \varLambda_{m} \times \varLambda_{n}}\left( \delta_{ij} \right)_{(i,j) \in \varLambda_{n}^{2}}\\
&= \left( \sum_{k \in \varLambda_{n}} {a_{ik}\delta_{kj}} \right)_{(i,j) \in \varLambda_{m} \times \varLambda_{n}}\\
&= \left( a_{ij}\delta_{jj} + \sum_{k \in \varLambda_{n} \setminus \left\{ j \right\}} {a_{ik}\delta_{kj}} \right)_{(i,j) \in \varLambda_{m} \times \varLambda_{n}}\\
&= \left( a_{ij} \cdot 1 + \sum_{k \in \varLambda_{n} \setminus \left\{ j \right\}} {a_{ik} \cdot 0} \right)_{(i,j) \in \varLambda_{m} \times \varLambda_{n}}\\
&= \left( a_{ij} + 0 \right)_{(i,j) \in \varLambda_{m} \times \varLambda_{n}}\\
&= \left( a_{ij} \right)_{(i,j) \in \varLambda_{m} \times \varLambda_{n}} = A_{mn}
\end{align*}
\end{proof}
\begin{dfn}
$n$次正方行列$A_{nn}$が与えられたとき、次式を満たすような行列$X_{nn}$がその集合$M_{nn}(R)$に存在するとき、
\begin{align*}
X_{nn}A_{nn} = A_{nn}X_{nn} = I_{n}
\end{align*}
その行列$A_{nn}$を可逆行列といい、その行列$X_{nn}$をその行列$A_{nn}$の逆行列という。また、可逆行列$A_{nn}$全体の集合を${\mathrm{GL}}_{n}(R)$と書く。とくに、体$K$上での可逆行列を正則行列ともいう。なお、後に述べるように存在するならこれは一意的なので、その行列$A_{nn}$の逆行列を$A_{nn}^{- 1}$と書く。
\end{dfn}
\begin{thm}\label{2.1.3.5}
1つの$n$次可逆行列$A_{nn}$の逆行列は一意的に存在する。
\end{thm}\par
このことは背理法によって示される。これにより集合${\mathrm{GL}}_{n}(R)$に属する1つの$n$次正方行列$A_{nn}$の逆行列を$A_{nn}^{- 1}$と書くことができる。
\begin{proof}
集合${\mathrm{GL}}_{n}(R)$に属する1つの$n$次可逆行列$A_{nn}$の互いの異なる2つの逆行列があると仮定しこれらを$X_{nn}$、$Y_{nn}$とおこう。したがって、次のようになる。
\begin{align*}
X_{nn} = X_{nn}I_{n} = X_{nn}\left( A_{nn}Y_{nn} \right) = \left( X_{nn}A_{nn} \right)Y_{nn} = I_{n}Y_{nn} = Y_{nn}
\end{align*}
これは仮定に矛盾する。よって、集合${\mathrm{GL}}_{n}(R)$に属する1つの$n$次可逆行列$A_{nn}$の逆行列が存在するなら、それは一意的に存在する。
\end{proof}
\begin{thm}\label{2.1.3.6}
集合${\mathrm{GL}}_{n}(R)$に属する2つの$n$次正方行列たち$A_{nn}$、$B_{nn}$において次のことが成り立つ。
\begin{itemize}
\item
  $n$次可逆行列$A_{nn}$の逆行列$A_{nn}^{- 1}$も可逆行列であり、$\left( A_{nn}^{- 1} \right)^{- 1} = A_{nn}$が成り立つ。
\item
  2つの$n$次可逆行列たち$A_{nn}$、$B_{nn}$の積$A_{nn}B_{nn}$も可逆行列であり、$\left( A_{nn}B_{nn} \right)^{- 1} = B_{nn}^{- 1}A_{nn}^{- 1}$が成り立つ。
\end{itemize}
\end{thm}
\begin{proof}
集合${\mathrm{GL}}_{n}(K)$に属する2つの$n$次正方行列たち$A_{nn}$、$B_{nn}$において、逆行列の定義より明らかに、その$n$次正方行列$A_{nn}$の逆行列$A_{nn}^{- 1}$も可逆行列であり、$\left( A_{nn}^{- 1} \right)^{- 1} = A_{nn}$である。\par
また、$B_{nn}^{- 1}A_{nn}^{- 1}$が存在し、次のようになるので、
\begin{align*}
\left( B_{nn}^{- 1}A_{nn}^{- 1} \right)\left( A_{nn}B_{nn} \right) &= \left( B_{nn}^{- 1}\left( A_{nn}^{- 1}A_{nn} \right) \right)B_{nn}\\
&= \left( B_{nn}^{- 1}I_{n} \right)B_{nn}\\
&= B_{nn}^{- 1}B_{nn} = I_{n}\\
\left( A_{nn}B_{nn} \right)\left( B_{nn}^{- 1}A_{nn}^{- 1} \right) &= \left( A_{nn}\left( B_{nn}B_{nn}^{- 1} \right) \right)A_{nn}^{- 1}\\
&= \left( A_{nn}I_{n} \right)A_{nn}^{- 1}{A_{nn}}^{- 1}\\
&= A_{nn}A_{nn}^{- 1} = I_{n}
\end{align*}
逆行列の定義より2つの$n$次可逆行列たち$A_{nn}$、$B_{nn}$の積$A_{nn}B_{nn}$も可逆行列であり、$\left( A_{nn}B_{nn} \right)^{- 1} = B_{nn}^{- 1}A_{nn}^{- 1}$である。
\end{proof}
%\hypertarget{ux884cux5217ux306eux8de1}{%
\subsubsection{行列の跡}%\label{ux884cux5217ux306eux8de1}}
\begin{dfn}
$A_{nn} \in M_{nn}(R)$なる可換環$R$上の行列$A_{nn}$が$A_{nn} = \left( a_{ij} \right)_{(i,j) \in \varLambda_{n}^{2}}$と与えられたとするとき、次式のような写像$\mathrm{tr} $によるその行列$A_{nn}$の像${\mathrm{tr} }A_{nn}$をその行列$A_{nn}$の跡、traceという。
\begin{align*}
\mathrm{tr} :M_{nn}(R) \rightarrow R;A_{nn} = \left( a_{ij} \right)_{(i,j) \in \varLambda_{n}^{2}} \mapsto \sum_{i \in \varLambda_{n}} a_{ii}
\end{align*}
\end{dfn}
\begin{thm}\label{2.1.3.7}
上記の写像$\mathrm{tr} $について次のことが成り立つ。
\begin{itemize}
\item
  $\forall k,l \in R\forall A_{nn},B_{nn} \in M_{nn}(R)$に対し、${\mathrm{tr} }\left( kA_{nn} + lB_{nn} \right) = k{\mathrm{tr} }A_{nn} + l{\mathrm{tr} }B_{nn}$が成り立つ。
\item
  $\forall A_{nn},B_{nn}\text{∈}M_{nn}(R)$に対し、${\mathrm{tr} }{A_{nn}B_{nn}} = {\mathrm{tr} }{B_{nn}A_{nn}}$が成り立つ。
\item
  $\forall A_{nn} \in M_{nn}(R)\forall P_{nn} \in {\mathrm{GL}}_{n}(R)$に対し、${\mathrm{tr} }{P_{nn}^{- 1}A_{nn}P_{nn}} = {\mathrm{tr} }A_{nn}$が成り立つ。
\end{itemize}
\end{thm}
\begin{proof}
可換環$R$上の行列の跡$\mathrm{tr} $について、$\forall k,l \in R\forall A_{nn},B_{nn} \in M_{nn}(R)$に対し、次のようになる。
\begin{align*}
{\mathrm{tr} }\left( kA_{nn} + lB_{nn} \right) &= {\mathrm{tr} }\left( k\left( a_{ij} \right)_{(i,j) \in \varLambda_{n}^{2}} + l\left( b_{ij} \right)_{(i,j) \in \varLambda_{n}^{2}} \right)\\
&= {\mathrm{tr} }\left( ka_{ij} + lb_{ij} \right)_{(i,j) \in \varLambda_{n}^{2}}\\
&= \sum_{i \in \varLambda_{n}} \left( ka_{ii} + lb_{ii} \right)\\
&= k\sum_{i \in \varLambda_{n}} a_{ii} + l\sum_{i \in \varLambda_{n}} b_{ii}\\
&= k{\mathrm{tr} }\left( a_{ij} \right)_{(i,j) \in \varLambda_{n}^{2}} + l{\mathrm{tr} }\left( b_{ij} \right)_{(i,j) \in \varLambda_{n}^{2}}\\
&= k{\mathrm{tr} }A_{nn} + l{\mathrm{tr} }B_{nn}
\end{align*}\par
$\forall A_{nn},B_{nn}\in M_{nn}(R)$に対し、次のようになる。
\begin{align*}
{\mathrm{tr} }{A_{nn}B_{nn}} &= {\mathrm{tr} }\left( \left( a_{ij} \right)_{(i,j) \in \varLambda_{n}^{2}}\left( b_{ij} \right)_{(i,j) \in \varLambda_{n}^{2}} \right)\\
&= {\mathrm{tr} }\left( \sum_{k \in \varLambda_{n}} {a_{ik}b_{kj}} \right)_{(i,j) \in \varLambda_{n}^{2}}\\
&= \sum_{i \in \varLambda_{n}} {\sum_{k \in \varLambda_{n}} {a_{ik}b_{ki}}}\\
&= \sum_{i \in \varLambda_{n}} {\sum_{k \in \varLambda_{n}} {b_{ik}a_{ki}}}\\
&= {\mathrm{tr} }\left( \sum_{k \in \varLambda_{n}} {b_{ik}a_{kj}} \right)_{(i,j) \in \varLambda_{n}^{2}}\\
&= {\mathrm{tr} }\left( \left( b_{ij} \right)_{(i,j) \in \varLambda_{n}^{2}}\left( a_{ij} \right)_{(i,j) \in \varLambda_{n}^{2}} \right)\\
&= {\mathrm{tr} }{B_{nn}A_{nn}}
\end{align*}\par
$\forall A_{nn} \in M_{nn}(R)\forall P_{nn} \in {\mathrm{GL}}_{n}(R)$に対し、$n$次単位行列$I_{n}$を用いれば次のようになる。
\begin{align*}
{\mathrm{tr} }{P_{nn}^{- 1}A_{nn}P_{nn}} &= {\mathrm{tr} }{P_{nn}^{- 1}P_{nn}A_{nn}}\\
&= {\mathrm{tr} }{I_{n}A_{nn}} = {\mathrm{tr} }A_{nn}
\end{align*}
\end{proof}
%\hypertarget{ux8ee2ux7f6eux884cux5217}{%
\subsubsection{転置行列}%\label{ux8ee2ux7f6eux884cux5217}}
\begin{dfn}
$A_{mn} \in M_{mn}(R)$なる可換環$R$上の行列$A_{mn}$が$A_{mn} = \left( a_{ij} \right)_{(i,j) \in \varLambda_{m} \times \varLambda_{n}}$と与えられたとするとき、次式のような写像${}^{t}$によるその行列$A_{mn}$の像${}^{t}A_{mn}$をその行列$A_{nn}$の転置行列といい$A_{mn}^{T}$、$A_{mn}^{\mathrm{tr} }$などとも書く。
\begin{align*}
{}^{t}:M_{mn}(R) \rightarrow M_{nm}(R);\left( a_{ij} \right)_{(i,j) \in \varLambda_{m} \times \varLambda_{n}} \mapsto \left( a_{ji} \right)_{(i,j) \in \varLambda_{n} \times \varLambda_{m}} = \begin{pmatrix}
a_{11} & a_{21} & \cdots & a_{m1} \\
a_{12} & a_{22} & \cdots & a_{m2} \\
 \vdots & \vdots & \ddots & \vdots \\
a_{1n} & a_{2n} & \cdots & a_{mn} \\
\end{pmatrix}
\end{align*}
\end{dfn}
\begin{thm}\label{2.1.3.8}
上記の写像${}^{t}$について、次のことが成り立つ。
\begin{itemize}
\item
  $\forall A_{mn} \in M_{mn}(R)$に対し、${}^{t}{}^{t}A_{mn} = A_{mn}$が成り立つ。
\item
  $\forall k,l \in R\forall A_{mn},B_{mn} \in M_{mn}(R)$に対し、${}^{t}\left( kA_{mn} + lB_{mn} \right) = k\ {}^{t}A_{mn} + l\ {}^{t}B_{mn}$が成り立つ。
\item
  $\forall A_{mn} \in M_{mn}(R)\forall B_{no} \in M_{no}(R)$に対し、${}^{t}\left( A_{mn}B_{no} \right) ={}^{t}B_{no}{}^{t}A_{mn}$が成り立つ。
\end{itemize}
\end{thm}
\begin{proof}
上記の写像${}^{t}$について、$\forall A_{mn} \in M_{mn}(R)$に対し、$A_{mn} = \left( a_{ij} \right)_{(i,j) \in \varLambda_{m} \times \varLambda_{n}}$とおくと、次のようになる。
\begin{align*}
{}^{t}{}^{t}A_{mn} &={}^{t}{}^{t}\left( a_{ij} \right)_{(i,j) \in \varLambda_{m} \times \varLambda_{n}}\\
&={}^{t}\left( a_{ji} \right)_{(i,j) \in \varLambda_{n} \times \varLambda_{m}}\\
&= \left( a_{ij} \right)_{(i,j) \in \varLambda_{m} \times \varLambda_{n}} = A_{mn}
\end{align*}\par
$\forall k,l \in R\forall A_{mn},B_{mn} \in M_{mn}(R)$に対し、$A_{mn} = \left( a_{ij} \right)_{(i,j) \in \varLambda_{m} \times \varLambda_{n}}$、$B_{mn} = \left( b_{ij} \right)_{(i,j) \in \varLambda_{m} \times \varLambda_{n}}$とおくと、次のようになる。
\begin{align*}
{}^{t}\left( kA_{mn} + lB_{mn} \right) &={}^{t}\left( k\left( a_{ij} \right)_{(i,j) \in \varLambda_{m} \times \varLambda_{n}} + l\left( b_{ij} \right)_{(i,j) \in \varLambda_{m} \times \varLambda_{n}} \right)\\
&={}^{t}\left( ka_{ij} + lb_{ij} \right)_{(i,j) \in \varLambda_{m} \times \varLambda_{n}}\\
&= \left( ka_{ji} + lb_{ji} \right)_{(i,j) \in \varLambda_{n} \times \varLambda_{m}}\\
&= k\left( a_{ji} \right)_{(i,j) \in \varLambda_{n} \times \varLambda_{m}} + l\left( b_{ji} \right)_{(i,j) \in \varLambda_{n} \times \varLambda_{m}}\\
&= k\ {}^{t}\left( a_{ij} \right)_{(i,j) \in \varLambda_{m} \times \varLambda_{n}} + l\ {}^{t}\left( b_{ij} \right)_{(i,j) \in \varLambda_{m} \times \varLambda_{n}}\\
&= k\ {}^{t}A_{mn} + l\ {}^{t}B_{mn}
\end{align*}\par
$\forall A_{mn} \in M_{mn}(R)\forall B_{no} \in M_{no}(R)$に対し、$A_{mn} = \left( a_{ij} \right)_{(i,j) \in \varLambda_{m} \times \varLambda_{n}}$、$B_{no} = \left( b_{ij} \right)_{(i,j) \in \varLambda_{n} \times \varLambda_{o}}$とおくと、次のようになる。
\begin{align*}
{}^{t}\left( A_{mn}B_{no} \right) &={}^{t}\left( \left( a_{ij} \right)_{(i,j) \in \varLambda_{m} \times \varLambda_{n}}\left( b_{ij} \right)_{(i,j) \in \varLambda_{n} \times \varLambda_{o}} \right)\\
&={}^{t}\left( \sum_{h \in \varLambda_{n}} {a_{ih}b_{hj}} \right)_{(i,j) \in \varLambda_{m} \times \varLambda_{o}}\\
&= \left( \sum_{h \in \varLambda_{n}} {a_{jh}b_{hi}} \right)_{(i,j) \in \varLambda_{o} \times \varLambda_{m}}\\
&= \left( \sum_{h \in \varLambda_{n}} {b_{hi}a_{jh}} \right)_{(i,j) \in \varLambda_{o} \times \varLambda_{m}}\\
&= \left( b_{ji} \right)_{(i,j) \in \varLambda_{o} \times \varLambda_{n}}\left( a_{ji} \right)_{(i,j) \in \varLambda_{n} \times \varLambda_{m}}\\
&={}^{t}\left( b_{ij} \right)_{(i,j) \in \varLambda_{n} \times \varLambda_{o}}{}^{t}\left( a_{ij} \right)_{(i,j) \in \varLambda_{m} \times \varLambda_{n}}\\
&={}^{t}B_{no}{}^{t}A_{mn}
\end{align*}
\end{proof}
\begin{dfn}
$A_{nn} \in M_{nn}(R)$なる行列$A_{nn}$の転置行列${}^{t}A_{nn}$が元のその行列$A_{nn}$に等しいとき、即ち、${}^{t}A_{nn} = A_{nn}$が成り立つとき、その行列$A_{mn}$を対称行列という。可換環$R$上の$n$次正方行列のうち対称行列であるもの全体の集合を${\mathrm{Sym} }_{n}(R)$などと書く。
\end{dfn}
\begin{dfn}
$A_{nn} \in M_{nn}(R)$なる行列$A_{nn}$の転置行列${}^{t}A_{nn}$が元のその行列$- A_{nn}$に等しいとき、即ち、${}^{t}A_{nn} = - A_{nn}$が成り立つとき、その行列$A_{mn}$を交代行列、歪対称行列、反対称行列などという。可換環$R$上の$n$次正方行列のうち交代行列全体の集合を${\mathrm{skew}}{{\mathrm{Sym} }_{n}(R)}$などと書く。
\end{dfn}
\begin{dfn}
$A_{nn} \in M_{nn}(R)$なる行列$A_{nn}$の転置行列${}^{t}A_{nn}$が元のその行列$A_{nn}$の逆行列$A_{nn}^{- 1}$に等しいとき、即ち、${}^{t}A_{nn} = A_{nn}^{- 1}$が成り立つとき、その行列$A_{nn}$を直交行列という。可換環$R$上の$n$次正方行列のうち直交行列全体の集合を$\mathrm{O} (n,R)$などと書く。
\end{dfn}
%\hypertarget{ux968fux4f34ux884cux5217}{%
\subsubsection{随伴行列}%\label{ux968fux4f34ux884cux5217}}
\begin{dfn}
$A_{mn} \in M_{mn}\left( \mathbf{C} \right)$なる体$K$上の行列$A_{mn}$が$A_{mn} = \left( a_{ij} \right)_{(i,j) \in \varLambda_{m} \times \varLambda_{n}}$と与えられたとするとき、次式のような写像$c$によるその行列$A_{mn}$の像$c\left( A_{mn} \right)$をその行列$A_{nn}$の複素共役行列といい$\overline{A_{mn}}$などと書く。
\begin{align*}
c:M_{mn}\left( \mathbf{C} \right) \rightarrow M_{mn}\left( \mathbf{C} \right);\left( a_{ij} \right)_{(i,j) \in \varLambda_{m} \times \varLambda_{n}} \mapsto \left( \overline{a_{ij}} \right)_{(i,j) \in \varLambda_{m} \times \varLambda_{n}}
\end{align*}
\end{dfn}
\begin{thm}\label{2.1.3.9}
上記の写像$c$について、次のことが成り立つ。
\begin{itemize}
\item
  $\forall A_{mn} \in M_{mn}\left( \mathbf{C} \right)$に対し、$\overline{\overline{A_{mn}}} = A_{mn}$が成り立つ。
\item
  $\forall k,l \in \mathbf{C}\forall A_{mn},B_{mn} \in M_{mn}\left( \mathbf{C} \right)$に対し、$\overline{kA_{mn} + lB_{mn}} = \overline{k}\ \overline{A_{mn}} + \overline{l}\ \overline{B_{mn}}$が成り立つ。
\item
  $\forall A_{mn} \in M_{mn}\left( \mathbf{C} \right)\forall B_{no} \in M_{no}\left( \mathbf{C} \right)$に対し、$\overline{A_{mn}B_{no}} = \overline{A_{mn}}\ \overline{B_{no}}$が成り立つ。
\end{itemize}
\end{thm}
\begin{proof}
上記の写像${}^{t}$について、$\forall A_{mn} \in M_{mn}\left( \mathbf{C} \right)$に対し、$A_{mn} = \left( a_{ij} \right)_{(i,j) \in \varLambda_{m} \times \varLambda_{n}}$とおくと、次のようになる。
\begin{align*}
\overline{\overline{A_{mn}}} &= \overline{\overline{\left( a_{ij} \right)_{(i,j) \in \varLambda_{m} \times \varLambda_{n}}}}\\
&= \overline{\left( \overline{a_{ij}} \right)_{(i,j) \in \varLambda_{m} \times \varLambda_{n}}}\\
&= \left( \overline{\overline{a_{ij}}} \right)_{(i,j) \in \varLambda_{m} \times \varLambda_{n}}\\
&= \left( a_{ij} \right)_{(i,j) \in \varLambda_{m} \times \varLambda_{n}} = A_{mn}
\end{align*}\par
$\forall k,l \in \mathbf{C}\forall A_{mn},B_{mn} \in M_{mn}\left( \mathbf{C} \right)$に対し、$A_{mn} = \left( a_{ij} \right)_{(i,j) \in \varLambda_{m} \times \varLambda_{n}}$、$B_{mn} = \left( b_{ij} \right)_{(i,j) \in \varLambda_{m} \times \varLambda_{n}}$とおくと、次のようになる。
\begin{align*}
\overline{kA_{mn} + lB_{mn}} &= \overline{k\left( a_{ij} \right)_{(i,j) \in \varLambda_{m} \times \varLambda_{n}} + l\left( b_{ij} \right)_{(i,j) \in \varLambda_{m} \times \varLambda_{n}}}\\
&= \overline{\left( ka_{ij} + lb_{ij} \right)_{(i,j) \in \varLambda_{m} \times \varLambda_{n}}}\\
&= \left( \overline{ka_{ij} + lb_{ij}} \right)_{(i,j) \in \varLambda_{m} \times \varLambda_{n}}\\
&= \left( \overline{k}\overline{a_{ij}} + \overline{l}\ \overline{b_{ij}} \right)_{(i,j) \in \varLambda_{m} \times \varLambda_{n}}\\
&= \overline{k}\left( \overline{a_{ij}} \right)_{(i,j) \in \varLambda_{m} \times \varLambda_{n}} + \overline{l}\left( \overline{b_{ij}} \right)_{(i,j) \in \varLambda_{m} \times \varLambda_{n}}\\
&= \overline{k}\ \overline{\left( a_{ij} \right)_{(i,j) \in \varLambda_{m} \times \varLambda_{n}}} + \overline{l}\ \overline{\left( b_{ij} \right)_{(i,j) \in \varLambda_{m} \times \varLambda_{n}}}\\
&= \overline{k}\ \overline{A_{mn}} + \overline{l}\ \overline{B_{mn}}
\end{align*}\par
$\forall A_{mn} \in M_{mn}\left( \mathbf{C} \right)\forall B_{no} \in M_{no}\left( \mathbf{C} \right)$に対し、$A_{mn} = \left( a_{ij} \right)_{(i,j) \in \varLambda_{m} \times \varLambda_{n}}$、$B_{no} = \left( b_{ij} \right)_{(i,j) \in \varLambda_{n} \times \varLambda_{o}}$とおくと、次のようになる。
\begin{align*}
\overline{A_{mn}B_{no}} &= \overline{\left( a_{ij} \right)_{(i,j) \in \varLambda_{m} \times \varLambda_{n}}\left( b_{ij} \right)_{(i,j) \in \varLambda_{n} \times \varLambda_{o}}}\\
&= \overline{\left( \sum_{h \in \varLambda_{n}} {a_{ih}b_{hj}} \right)_{(i,j) \in \varLambda_{m} \times \varLambda_{n}}}\\
&= \left( \overline{\sum_{h \in \varLambda_{n}} {a_{ih}b_{hj}}} \right)_{(i,j) \in \varLambda_{m} \times \varLambda_{n}}\\
&= \left( \sum_{h \in \varLambda_{n}} {\overline{a_{ih}}\overline{b_{hj}}} \right)_{(i,j) \in \varLambda_{m} \times \varLambda_{n}}\\
&= \left( \overline{a_{ij}} \right)_{(i,j) \in \varLambda_{m} \times \varLambda_{n}}\left( \overline{b_{ij}} \right)_{(i,j) \in \varLambda_{n} \times \varLambda_{o}}\\
&= \overline{\left( a_{ij} \right)_{(i,j) \in \varLambda_{m} \times \varLambda_{n}}}\ \overline{\left( b_{ij} \right)_{(i,j) \in \varLambda_{n} \times \varLambda_{o}}}\\
&= \overline{A_{mn}}\ \overline{B_{no}}
\end{align*}
\end{proof}
\begin{dfn}
$A_{mn} \in M_{mn}\left( \mathbf{C} \right)$なる体$K$上の行列$A_{mn}$が$A_{mn} = \left( a_{ij} \right)_{(i,j) \in \varLambda_{m} \times \varLambda_{n}}$と与えられたとするとき、次式のような写像$a$によるその行列$A_{mn}$の像$a\left( A_{mn} \right)$をその行列$A_{nn}$の随伴行列、Hermite転置行列、Hermite共軛行列、Hermite随伴行列などといい$A_{mn}^{*}$、$A_{mn}^{\dagger}$、$A_{mn}^{H}$などと書く。
\begin{align*}
a:M_{mn}\left( \mathbf{C} \right) \rightarrow M_{mn}\left( \mathbf{C} \right);\left( a_{ij} \right)_{(i,j) \in \varLambda_{m} \times \varLambda_{n}} \mapsto \left( \overline{a_{ji}} \right)_{(i,j) \in \varLambda_{n} \times \varLambda_{m}}
\end{align*}
\end{dfn}
\begin{thm}\label{2.1.3.10}
上記の写像$a$について、次のことが成り立つ。
\begin{itemize}
\item
  $\forall A_{mn} \in M_{mn}\left( \mathbf{C} \right)$に対し、$A_{mn}^{**} = A_{mn}$が成り立つ。
\item
  $\forall k,l \in \mathbf{C}\forall A_{mn},B_{mn} \in M_{mn}\left( \mathbf{C} \right)$に対し、$\left( kA_{mn} + lB_{mn} \right)^{*} = \overline{k}A_{mn}^{*} + \overline{l}B_{mn}^{*}$が成り立つ。
\item
  $\forall A_{mn} \in M_{mn}\left( \mathbf{C} \right)\forall B_{no} \in M_{no}\left( \mathbf{C} \right)$に対し、$\left( A_{mn}B_{no} \right)^{*} = B_{no}^{*}A_{mn}^{*}$が成り立つ。
\end{itemize}
\end{thm}
\begin{proof}
上記の写像${}^{t}$について、$\forall A_{mn} \in M_{mn}\left( \mathbf{C} \right)$に対し、$A_{mn} = \left( a_{ij} \right)_{(i,j) \in \varLambda_{m} \times \varLambda_{n}}$とおくと、次のようになる。
\begin{align*}
A_{mn}^{**} &= \left( a_{ij} \right)_{(i,j) \in \varLambda_{m} \times \varLambda_{n}}^{**}\\
&= \left( \overline{a_{ji}} \right)_{(i,j) \in \varLambda_{n} \times \varLambda_{m}}^{*}\\
&= \left( \overline{\overline{a_{ij}}} \right)_{(i,j) \in \varLambda_{m} \times \varLambda_{n}}\\
&= \left( a_{ij} \right)_{(i,j) \in \varLambda_{m} \times \varLambda_{n}} = A_{mn} \\
\left( kA_{mn} + lB_{mn} \right)^{*} &= \left( k\left( a_{ij} \right)_{(i,j) \in \varLambda_{m} \times \varLambda_{n}} + l\left( b_{ij} \right)_{(i,j) \in \varLambda_{m} \times \varLambda_{n}} \right)^{*}\\
&= \left( ka_{ij} + lb_{ij} \right)_{(i,j) \in \varLambda_{m} \times \varLambda_{n}}^{*}\\
&= \left( \overline{ka_{ji} + lb_{ji}} \right)_{(i,j) \in \varLambda_{n} \times \varLambda_{m}}\\
&= \left( \overline{k}\overline{a_{ji}} + \overline{l}\ \overline{b_{ji}} \right)_{(i,j) \in \varLambda_{n} \times \varLambda_{m}}\\
&= \overline{k}\left( \overline{a_{ji}} \right)_{(i,j) \in \varLambda_{n} \times \varLambda_{m}} + \overline{l}\left( \overline{b_{ji}} \right)_{(i,j) \in \varLambda_{n} \times \varLambda_{m}}\\
&= \overline{k}\left( a_{ij} \right)_{(i,j) \in \varLambda_{m} \times \varLambda_{n}}^{*} + \overline{l}\left( b_{ij} \right)_{(i,j) \in \varLambda_{m} \times \varLambda_{n}}^{*}\\
&= \overline{k}A_{mn}^{*} + \overline{l}B_{mn}^{*} \\
\left( A_{mn}B_{no} \right)^{*} &= \left( \left( a_{ij} \right)_{(i,j) \in \varLambda_{m} \times \varLambda_{n}}\left( b_{ij} \right)_{(i,j) \in \varLambda_{n} \times \varLambda_{o}} \right)^{*}\\
&= \left( \sum_{h \in \varLambda_{n}} {a_{ih}b_{hj}} \right)_{(i,j) \in \varLambda_{m} \times \varLambda_{o}}^{*}\\
&= \left( \overline{\sum_{h \in \varLambda_{n}} {a_{jh}b_{hi}}} \right)_{(i,j) \in \varLambda_{o} \times \varLambda_{m}}\\
&= \left( \sum_{h \in \varLambda_{n}} {\overline{a_{jh}}\overline{b_{hi}}} \right)_{(i,j) \in \varLambda_{o} \times \varLambda_{m}}\\
&= \left( \sum_{h \in \varLambda_{n}} {\overline{b_{hi}}\overline{a_{jh}}} \right)_{(i,j) \in \varLambda_{o} \times \varLambda_{m}}\\
&= \left( \overline{b_{ji}} \right)_{(i,j) \in \varLambda_{o} \times \varLambda_{n}}\left( \overline{a_{ji}} \right)_{(i,j) \in \varLambda_{n} \times \varLambda_{m}} = B_{no}^{*}A_{mn}^{*}
\end{align*}
\end{proof}
\begin{dfn}
$A_{nn} \in M_{nn}\left( \mathbf{C} \right)$なる行列$A_{nn}$の随伴行列$A_{nn}^{*}$が元のその行列$A_{nn}$に等しいとき、即ち、$A_{nn}^{*} = A_{nn}$が成り立つとき、その行列$A_{mn}$を自己随伴行列、Hermite行列という。体$\mathbf{C}$上の$n$次正方行列のうち自己随伴行列であるもの全体の集合を$H_{n}$などと書く。
\end{dfn}
\begin{dfn}
$A_{nn} \in M_{nn}\left( \mathbf{C} \right)$なる行列$A_{nn}$の随伴行列$A_{nn}^{*}$が元のその行列$- A_{nn}$に等しいとき、即ち、$A_{nn}^{*} = - A_{nn}$が成り立つとき、その行列$A_{nn}$を歪自己随伴行列、歪Hermite行列などという。体$\mathbf{C}$上の$n$次正方行列のうち歪自己随伴行列全体の集合を${\mathrm{skew}}H_{n}$などと書く。
\end{dfn}
\begin{dfn}
$A_{nn} \in {\mathrm{GL}}_{n}\left( \mathbf{C} \right)$なる行列$A_{nn}$の転置行列$A_{nn}^{*}$が元のその行列$A_{nn}$の逆行列$A_{nn}^{- 1}$に等しいとき、即ち、$A_{nn}^{*} = A_{nn}^{- 1}$が成り立つとき、その行列$A_{nn}$をunitary行列という。体$\mathbf{C}$上の$n$次正方行列のうちunitary行列全体の集合を$U_{n}$などと書く。
\end{dfn}
%\hypertarget{ux884cux5217ux306eux4ee3ux6570ux5b66ux7684ux306aux69cbux9020}{%
\subsubsection{行列の代数学的な構造}%\label{ux884cux5217ux306eux4ee3ux6570ux5b66ux7684ux306aux69cbux9020}}
\begin{thm}\label{2.1.3.11}
可換環$R$上の組$\left( M_{mn}(R), + \right)$は可換群をなす。なお、このときの単位元は零行列$O_{mn}$、行列$A_{mn}$の逆元は$- A_{mn}$である。
\end{thm}
\begin{proof}
可換環$R$上の組$\left( M_{mn}(R), + \right)$について、このときの単位元を零行列$O_{mn}$、行列$A_{mn}$の逆元を$- A_{mn}$とおくと、定理\ref{2.1.3.2}より明らかにその組$\left( M_{mn}(R), + \right)$は可換群をなす。
\end{proof}
\begin{thm}\label{2.1.3.12}
換環$R$上の組$\left( {\mathrm{GL}}_{n}(R), \cdot \right)$は群をなす。なお、$\cdot$は行列の積で、このときの単位元は単位行列$I_{n}$、行列$A_{nn}$の逆元は逆行列$A_{nn}^{- 1}$である。
\end{thm}
\begin{proof}
可換環$R$上の組$\left( {\mathrm{GL}}_{n}(R), \cdot \right)$について、このときの単位元を単位行列$I_{n}$、行列$A_{nn}$の逆元を逆行列$A_{nn}^{- 1}$とおくと、定理\ref{2.1.3.3}より明らかにその組$\left( {\mathrm{GL}}_{n}(R), \cdot \right)$は可換群をなす。
\end{proof}
\begin{thm}\label{2.1.3.13}
可換環$R$が与えられたとき、その集合${\mathrm{GL}}_{n}(R) \cup \left\{ O_{nn} \right\}$は斜体をなす。なお、このときの単位元は単位行列$I_{nn}$、零元は零行列$O_{nn}$、行列$A_{nn}$の逆元は$A_{nn}^{- 1}$である。
\end{thm}
\begin{proof}
可換環$R$が与えられたとき、その集合${\mathrm{GL}}_{n}(R)$について、このときの単位元を単位行列$I_{nn}$、零元を零行列$O_{nn}$、行列$A_{nn}$の逆元を$A_{nn}^{- 1}$とおくと、定理\ref{2.1.3.2}、定理\ref{2.1.3.3}より明らかにその集合${\mathrm{GL}}_{n}(R) \cup \left\{ O_{nn} \right\}$は環をなす。
\end{proof}
\begin{thm}\label{2.1.3.14}
次式のように写像$i$をおくと、
\begin{align*}
i:{\mathrm{GL}}_{n}(R) \rightarrow {\mathrm{GL}}_{n}(R);A_{nn} \mapsto A_{nn}^{- 1}
\end{align*}
恒等写像を1として組$\left( \left\{ 1,i,{}^{t},c,a \right\}, \circ \right)$は写像の合成が定義可能な限り可換群をなす。このときの単位元は恒等写像$1$で、写像$f$の逆元はその写像自身である。
\end{thm}
\begin{proof}
次式のように写像$i$をおく。
\begin{align*}
i:{\mathrm{GL}}_{n}(R) \rightarrow {\mathrm{GL}}_{n}(R);A_{nn} \mapsto A_{nn}^{- 1}
\end{align*}
恒等写像を1として、単位元を恒等写像$1$、写像$f$の逆元をその写像自身とおくと、定理\ref{2.1.3.6}、定理\ref{2.1.3.8}、定理\ref{2.1.3.9}、定理\ref{2.1.3.10}よりその組$\left( \left\{ 1,i,{}^{t},c,a \right\}, \circ \right)$は写像の合成が定義可能な限り群をなす。\par
また、$\forall A_{nn} \in {\mathrm{GL}}_{n}(R)$に対し、次のようになるので、
\begin{align*}
{}^{t}A_{nn}{}^{t}\left( A_{nn}^{- 1} \right) &={}^{t}\left( A_{nn}^{- 1}A_{nn} \right) ={}^{t}I_{n} = I_{n}\\
{}^{t}\left( A_{nn}^{- 1} \right){}^{t}A_{nn} &={}^{t}\left( A_{nn}A_{nn}^{- 1} \right) ={}^{t}I_{n} = I_{n}
\end{align*}
${}^{t}\left( A_{nn}^{- 1} \right) = \left({}^{t}A_{nn} \right)^{- 1}$が成り立ち、したがって、$i \circ{}^{t} ={}^{t} \circ i$が成り立つ。$\forall A_{nn} \in {\mathrm{GL}}_{n}\left( \mathbf{C} \right)$に対し、次のようになるので、
\begin{align*}
\overline{A_{nn}}\ \overline{A_{nn}^{- 1}} &= \overline{A_{nn}A_{nn}^{- 1}} = \overline{I_{n}} = I_{n}\\
\overline{A_{nn}^{- 1}}\ \overline{A_{nn}} &= \overline{A_{nn}^{- 1}A_{nn}} = \overline{I_{n}} = I_{n}
\end{align*}
$\overline{A_{nn}^{- 1}} = {\overline{A_{nn}}}^{- 1}$が成り立ち、したがって、$i \circ c = c \circ i$が成り立つ。$\forall A_{nn} \in {\mathrm{GL}}_{n}\left( \mathbf{C} \right)$に対し、次のようになるので、
\begin{align*}
A_{nn}^{*}{A_{nn}^{- 1}}^{*} &= \left( A_{nn}^{- 1}A_{nn} \right)^{*} = I_{n}^{*} = I_{n}\\
{A_{nn}^{- 1}}^{*}A_{nn}^{*} &= \left( A_{nn}A_{nn}^{- 1} \right)^{*} = I_{n}^{*} = I_{n}
\end{align*}
${A_{nn}^{- 1}}^{*} = {A_{nn}^{*}}^{- 1}$が成り立ち、したがって、$i \circ a = a \circ i$が成り立つ。$\forall A_{mn} \in M_{mn}\left( \mathbf{C} \right)$に対し、$A_{mn} = \left( a_{ij} \right)_{(i,j) \in \varLambda_{m} \times \varLambda_{n}}$とおくと、次のようになるので、
\begin{align*}
{}^{t}\overline{A_{mn}} &={}^{t}\overline{\left( a_{ij} \right)_{(i,j) \in \varLambda_{m} \times \varLambda_{n}}}\\
&={}^{t}\left( \overline{a_{ij}} \right)_{(i,j) \in \varLambda_{m} \times \varLambda_{n}}\\
&= \left( \overline{a_{ji}} \right)_{(i,j) \in \varLambda_{n} \times \varLambda_{m}}\\
&= \overline{\left( a_{ji} \right)_{(i,j) \in \varLambda_{n} \times \varLambda_{m}}}\\
&= \overline{{}^{t}\left( a_{ij} \right)_{(i,j) \in \varLambda_{m} \times \varLambda_{n}}} = \overline{{}^{t}A_{nn}}
\end{align*}
${}^{t}\overline{A_{mn}} = \overline{{}^{t}A_{nn}}$が成り立ち、したがって、$c \circ{}^{t} ={}^{t} \circ c$が成り立つ。$\forall A_{mn} \in M_{mn}\left( \mathbf{C} \right)$に対し、$A_{mn} = \left( a_{ij} \right)_{(i,j) \in \varLambda_{m} \times \varLambda_{n}}$とおくと、次のようになるので、
\begin{align*}
{\overline{A_{mn}}}^{*} &= {\overline{\left( a_{ij} \right)_{(i,j) \in \varLambda_{m} \times \varLambda_{n}}}}^{*}\\
&= {\left( \overline{a_{ij}} \right)_{(i,j) \in \varLambda_{m} \times \varLambda_{n}}}^{*}\\
&= \left( \overline{\overline{a_{ji}}} \right)_{(i,j) \in \varLambda_{n} \times \varLambda_{m}}\\
&= \overline{\left( \overline{a_{ji}} \right)_{(i,j) \in \varLambda_{n} \times \varLambda_{m}}}\\
&= \overline{\left( a_{ij} \right)_{(i,j) \in \varLambda_{m} \times \varLambda_{n}}^{*}} = \overline{A_{nn}^{*}}
\end{align*}
${\overline{A_{mn}}}^{*} = \overline{A_{nn}^{*}}$が成り立ち、したがって、$c \circ a = a \circ c$が成り立つ。以上よりその群$\left( \left\{ 1,i,{}^{t},c,a \right\}, \circ \right)$は可換群をなす。
\end{proof}
\begin{thm}\label{2.1.3.15}
$a ={}^{t} \circ c$が成り立つ。
\end{thm}
\begin{proof}
$\forall A_{nn} \in M_{mn}\left( \mathbf{C} \right)$に対し、$A_{mn} = \left( a_{ij} \right)_{(i,j) \in \varLambda_{m} \times \varLambda_{n}}$とおくと、次のようになるので、
\begin{align*}
a\left( A_{nn} \right) = A_{nn}^{*} = \left( \overline{a_{ji}} \right)_{(i,j) \in \varLambda_{n} \times \varLambda_{m}} ={}^{t}\left( \overline{a_{ij}} \right)_{(i,j) \in \varLambda_{m} \times \varLambda_{n}} ={}^{t}\overline{\left( a_{ij} \right)_{(i,j) \in \varLambda_{m} \times \varLambda_{n}}} ={}^{t} \circ c\left( A_{mn} \right)
\end{align*}
$a ={}^{t} \circ c$が成り立つ。
\end{proof}
%\hypertarget{blockux884cux5217}{%
\subsubsection{block行列}%\label{blockux884cux5217}}
可換環$R$上に
\begin{align*}
S=\sum_{I \in \varLambda_{M}} m_{I} ,\ \ T=\sum_{J \in \varLambda_{N}} n_{J}
\end{align*}
として$A_{ST} =\left( a_{ij} \right)_{(i,j) \in \varLambda_{S} \times \varLambda_{T}} \in M_{ST}(R)$なる行列$A_{ST} $が与えられたとき、次のようにおくと、
\begin{align*}
A_{IJ} =\left( a_{\sum_{I' \in \varLambda_{I - 1}} m_{I'} + i,\sum_{J' \in \varLambda_{J - 1}} n_{J'} + j} \right)_{(i,j) \in \varLambda_{m_{I}} \times \varLambda_{n_{J}}} \in M_{m_{I}n_{J}}(R)
\end{align*}
次のようになることから、
\begin{align*}
  A_{ST} &= \left( a_{ij} \right)_{(i,j) \in \varLambda_{S} \times \varLambda_{T}} = \begin{pmatrix}
  a_{11} & \cdots & a_{1T} \\
   \vdots & \ddots & \vdots \\
  a_{S1} & \cdots & a_{ST} \\
  \end{pmatrix} \\
  &= \begin{pmatrix}
  a_{11} & \cdots & a_{1,\sum_{J' \in \varLambda_{N}} n_{J'}} \\
   \vdots & \ddots & \vdots \\
  a_{\sum_{I' \in \varLambda_{M}} m_{I'},1} & \cdots & a_{\sum_{I' \in \varLambda_{M}} m_{I'},\sum_{J' \in \varLambda_{N}} n_{J'}} \\
  \end{pmatrix} \\
  &= \left( \begin{matrix}
  a_{11} & \cdots & a_{1n_{1}} & \cdots \\
   \vdots & \ddots & \vdots & \ddots \\
  a_{m_{1}1} & \cdots & a_{m_{1}n_{1}} & \cdots \\
   \vdots & \ddots & \vdots & \ddots \\
  a_{\sum_{I' \in \varLambda_{M - 1}} m_{I'} + 1,1} & \cdots & a_{\sum_{I' \in \varLambda_{M - 1}} m_{I'} + 1,n_{1}} & \cdots \\
   \vdots & \ddots & \vdots & \ddots \\
  a_{\sum_{I' \in \varLambda_{M}} m_{I'},1} & \cdots & a_{\sum_{I' \in \varLambda_{M}} m_{I'},n_{1}} & \cdots \\
  \end{matrix} \right. \\
  &\quad \left. \begin{matrix}
  a_{1,\sum_{J' \in \varLambda_{N - 1}} n_{J'} + 1} & \cdots & a_{1,\sum_{J' \in \varLambda_{N}} n_{J'}} \\
    \vdots & \ddots & \vdots \\
  a_{m_{1},\sum_{J' \in \varLambda_{N - 1}} n_{J'} + 1} & \cdots & a_{m_{1},\sum_{J' \in \varLambda_{N}} n_{J'}} \\
    \vdots & \ddots & \vdots \\
  a_{\sum_{I' \in \varLambda_{M - 1}} m_{I'} + 1,\sum_{J' \in \varLambda_{N - 1}} n_{J'} + 1} & \cdots & a_{\sum_{I' \in \varLambda_{M - 1}} m_{I'} + 1,\sum_{J' \in \varLambda_{N}} n_{J'}} \\
    \vdots & \ddots & \vdots \\
  a_{\sum_{I' \in \varLambda_{M}} m_{I'},\sum_{J' \in \varLambda_{N - 1}} n_{J'} + 1} & \cdots & a_{\sum_{I' \in \varLambda_{M}} m_{I'},\sum_{J' \in \varLambda_{N}} n_{J'}} \\
  \end{matrix} \right) \\
  &= \left( \begin{matrix}
  \begin{pmatrix}
  a_{11} & \cdots & a_{1n_{1}} \\
   \vdots & \ddots & \vdots \\
  a_{m_{1}1} & \cdots & a_{m_{1}n_{1}} \\
  \end{pmatrix} & \cdots \\
   \vdots & \ddots \\
  \begin{pmatrix}
  a_{\sum_{I' \in \varLambda_{M - 1}} m_{I'} + 1,1} & \cdots & a_{\sum_{I' \in \varLambda_{M - 1}} m_{I'} + 1,n_{1}} \\
   \vdots & \ddots & \vdots \\
  a_{\sum_{I' \in \varLambda_{M}} m_{I'},1} & \cdots & a_{\sum_{I' \in \varLambda_{M}} m_{I'},n_{1}} \\
  \end{pmatrix} & \cdots\\
  \end{matrix} \right. \\
  &\quad \left. \begin{matrix}
  \begin{pmatrix}
  a_{1,\sum_{J' \in \varLambda_{N - 1}} n_{J'} + 1} & \cdots & a_{1,\sum_{J' \in \varLambda_{N}} n_{J'}} \\
    \vdots & \ddots & \vdots \\
  a_{m_{1},\sum_{J' \in \varLambda_{N - 1}} n_{J'} + 1} & \cdots & a_{m_{1},\sum_{J' \in \varLambda_{N}} n_{J'}} \\
  \end{pmatrix} \\
    \vdots \\
  \begin{pmatrix}
  a_{\sum_{I' \in \varLambda_{M - 1}} m_{I'} + 1,\sum_{J' \in \varLambda_{N - 1}} n_{J'} + 1} & \cdots & a_{\sum_{I' \in \varLambda_{M - 1}} m_{I'} + 1,\sum_{J' \in \varLambda_{N}} n_{J'}} \\
    \vdots & \ddots & \vdots \\
  a_{\sum_{I' \in \varLambda_{M}} m_{I'},\sum_{J' \in \varLambda_{N - 1}} n_{J'} + 1} & \cdots & a_{\sum_{I' \in \varLambda_{M}} m_{I'},\sum_{J' \in \varLambda_{N}} n_{J'}} \\
  \end{pmatrix} \\
  \end{matrix} \right) \\
  &= \begin{pmatrix}
  A_{11} & \cdots & A_{1N} \\
    \vdots & \ddots & \vdots \\
  A_{M1} & \cdots & A_{MN} \\
  \end{pmatrix} 
\end{align*}
次のように書かれることもできる。
\begin{align*}
  A_{ST} =\begin{pmatrix}
  A_{11} & \cdots & A_{1N} \\
    \vdots & \ddots & \vdots \\
  A_{M1} & \cdots & A_{MN} \\
  \end{pmatrix} 
\end{align*}
\begin{dfn}
その右辺の形をした行列をblock行列などといい、行列をそのblock行列に書き換えることをその行列をblock化するなどという。また、行列たち$A_{IJ}$のことをその行列$A_{ST} $の小行列たちという。
\end{dfn}
\begin{thm}\label{2.1.3.17}
可換環$R$上に
\begin{align*}
S=\sum_{I \in \varLambda_{M}} m_{I} ,\ \ T=\sum_{J \in \varLambda_{N}} n_{J}
\end{align*}
として$\forall \left( a_{ij} \right)_{(i,j) \in \varLambda_{S} \times \varLambda_{T}} ,\left( b_{ij} \right)_{(i,j) \in \varLambda_{S} \times \varLambda_{T}} \in M_{ST}(R)$ $\forall k,l\in R$に対し、次のようにおくと、
\begin{align*}
  A_{IJ} &=\left( a_{\sum_{I' \in \varLambda_{I - 1}} m_{I'} + i,\sum_{J' \in \varLambda_{J - 1}} n_{J'} + j} \right)_{(i,j) \in \varLambda_{m_{I}} \times \varLambda_{n_{J}}} \in M_{m_{I}n_{J}}(R) \\
  B_{IJ} &=\left( b_{\sum_{I' \in \varLambda_{I - 1}} m_{I'} + i,\sum_{J' \in \varLambda_{J - 1}} n_{J'} + j} \right)_{(i,j) \in \varLambda_{m_{I}} \times \varLambda_{n_{J}}} \in M_{m_{I}n_{J}}(R) 
\end{align*}
次式が成り立つ。
\begin{align*}
k\left( A_{IJ} \right)_{(I,J) \in \varLambda_{M} \times \varLambda_{N}} + l\left( B_{IJ} \right)_{(I,J) \in \varLambda_{M} \times \varLambda_{N}} &= \left( kA_{IJ} + lB_{IJ} \right)_{(I,J) \in \varLambda_{M} \times \varLambda_{N}}
\end{align*}
なお、その条件は上に述べた通り2つの行列たち$\left( a_{ij} \right)_{(i,j) \in \varLambda_{S} \times \varLambda_{T}} $、$\left( b_{ij} \right)_{(i,j) \in \varLambda_{S} \times \varLambda_{T}} $の行数と列数が等しく分割の仕方も等しいこととなっている\footnote{誰か分かりやすい説明方法を教えてくれぇ――――――!! }。
\end{thm}
\begin{proof}
上のblock行列の議論に注意すれば、これは単に成分ごとで計算しているだけにすぎない。
\end{proof}
\begin{thm}\label{2.1.3.19}
可換環$R$上に
\begin{align*}
S=\sum_{I \in \varLambda_{M}} m_{I} ,\ \ T=\sum_{J \in \varLambda_{N}} n_{J},\ \ U=\sum_{K \in \varLambda_{O}} o_{K}
\end{align*}
として$\forall \left( a_{ij} \right)_{(i,j) \in \varLambda_{S} \times \varLambda_{T}} \in M_{ST}(R)$ $\forall \left( b_{jk} \right)_{(j,k) \in \varLambda_{T} \times \varLambda_{U}} \in M_{TU}(R)$に対し、次のようにおくと、
\begin{align*}
A_{IJ} &=\left( a_{\sum_{I' \in \varLambda_{I - 1}} m_{I'} + i,\sum_{J' \in \varLambda_{J - 1}} n_{J'} + j} \right)_{(i,j) \in \varLambda_{m_{I}} \times \varLambda_{n_{J}}} \in M_{m_{I}n_{J}}(R) \\
B_{JK} &=\left( b_{\sum_{J' \in \varLambda_{J - 1}} n_{J'} + j,\sum_{K' \in \varLambda_{K - 1}} o_{K'} + k} \right)_{(j,k) \in \varLambda_{n_{J}} \times \varLambda_{o_{K}}} \in M_{n_{J}o_{K}}(R) 
\end{align*}
次式が成り立つ。
\begin{align*}
\left( A_{IJ} \right)_{(I,J) \in \varLambda_{M} \times \varLambda_{N}} \left( B_{JK} \right)_{(J,K) \in \varLambda_{N} \times \varLambda_{O}} = \left( \sum_{J\in \varLambda_{N} } A_{IJ} B_{JK} \right)_{(I,K) \in \varLambda_{M} \times \varLambda_{O}}
\end{align*}
なお、その条件は上に述べた通りその行列$\left( a_{ij} \right)_{(i,j) \in \varLambda_{S} \times \varLambda_{T}}$の列数とその行列$\left( b_{ij} \right)_{(i,j) \in \varLambda_{T} \times \varLambda_{U}}$の行数が等しくその行列$\left( a_{ij} \right)_{(i,j) \in \varLambda_{S} \times \varLambda_{T}}$の分割された列のかたまりの個数とその行列$\left( b_{ij} \right)_{(i,j) \in \varLambda_{T} \times \varLambda_{U}}$の分割された行のかたまりの個数も等しく、$\forall I\in \varLambda_{M} \forall J\in \varLambda_{N} \forall K\in \varLambda_{O} $に対し、その小行列$A_{IJ}$の列数とその小行列$B_{JK}$の行数も等しいこととなっている\footnote{ここも誰か分かりやすい説明方法を教えてくれぇ――――――!! }。
\end{thm}
\begin{proof}
可換環$R$上に
\begin{align*}
S=\sum_{I \in \varLambda_{M}} m_{I} ,\ \ T=\sum_{J \in \varLambda_{N}} n_{J},\ \ U=\sum_{K \in \varLambda_{O}} o_{K}
\end{align*}
として$\forall \left( a_{ij} \right)_{(i,j) \in \varLambda_{S} \times \varLambda_{T}} \in M_{ST}(R)$ $\forall \left( b_{jk} \right)_{(j,k) \in \varLambda_{T} \times \varLambda_{U}} \in M_{TU}(R)$に対し、次のようにおくと、
\begin{align*}
A_{IJ} &=\left( a_{\sum_{I' \in \varLambda_{I - 1}} m_{I'} + i,\sum_{J' \in \varLambda_{J - 1}} n_{J'} + j} \right)_{(i,j) \in \varLambda_{m_{I}} \times \varLambda_{n_{J}}} \in M_{m_{I}n_{J}}(R) \\
B_{JK} &=\left( b_{\sum_{J' \in \varLambda_{J - 1}} n_{J'} + j,\sum_{K' \in \varLambda_{K - 1}} o_{K'} + k} \right)_{(j,k) \in \varLambda_{n_{J}} \times \varLambda_{o_{K}}} \in M_{n_{J}o_{K}}(R) \\
\end{align*}
このとき、次のようになる。
\begin{align*}
&\quad \left( A_{IJ} \right)_{(I,J) \in \varLambda_{M} \times \varLambda_{N}}\left( B_{IJ} \right)_{(J,K) \in \varLambda_{N} \times \varLambda_{O}} \\
&= \left( \left( a_{\sum_{I' \in \varLambda_{I - 1}} m_{I'} + i,\sum_{J' \in \varLambda_{J - 1}} n_{J'} + j} \right)_{(i,j) \in \varLambda_{m_{I}} \times \varLambda_{n_{J}}} \right)_{(I,J) \in \varLambda_{M} \times \varLambda_{N}} \\
&\quad \left( \left( b_{\sum_{J' \in \varLambda_{J - 1}} n_{J'} + j,\sum_{K' \in \varLambda_{K - 1}} o_{K'} + k} \right)_{(j,k) \in \varLambda_{n_{J}} \times \varLambda_{o_{K}}} \right)_{(J,K) \in \varLambda_{N} \times \varLambda_{O}}\\
&= \left( a_{ij} \right)_{(i,j) \in \varLambda_{\sum_{I \in \varLambda_{M}} m_{I}} \times \varLambda_{\sum_{J \in \varLambda_{N}} n_{J}}}\left( b_{jk} \right)_{(j,k) \in \varLambda_{\sum_{J \in \varLambda_{N}} n_{J}} \times \varLambda_{\sum_{K \in \varLambda_{O}} o_{K}}}\\
&= \left( \sum_{j \in \varLambda_{\sum_{J \in \varLambda_{N}} n_{J}}} {a_{ij}b_{jk}} \right)_{(i,k) \in \varLambda_{\sum_{I \in \varLambda_{M}} m_{I}} \times \varLambda_{\sum_{K \in \varLambda_{O}} o_{K}}}\\
&= \left( \left( \sum_{j \in \varLambda_{\sum_{J \in \varLambda_{N}} n_{J}}} {a_{\sum_{I' \in \varLambda_{I - 1}} m_{I'} + i,j}b_{j,\sum_{K' \in \varLambda_{K - 1}} o_{K'} + k}} \right)_{(i,k) \in \varLambda_{m_{I}} \times \varLambda_{o_{K}}} \right)_{(I,K) \in \varLambda_{M} \times \varLambda_{O}}\\
&= \left( \left( \sum_{J \in \varLambda_{N}} \sum_{j \in \varLambda_{n_{J}}}  a_{\sum_{I' \in \varLambda_{I - 1}} m_{I'} + i,\sum_{J' \in \varLambda_{J - 1}} n_{J'} + j} \right. \right. \\
&\quad \left. \left. b_{\sum_{J' \in \varLambda_{J - 1}} n_{J'} + j,\sum_{K' \in \varLambda_{K - 1}} o_{K'} + k} \right)_{(i,k) \in \varLambda_{m_{I}} \times \varLambda_{o_{K}}} \right)_{(I,K) \in \varLambda_{M} \times \varLambda_{O}}\\
&= \left( \sum_{J \in \varLambda_{N}}  \sum_{j \in \varLambda_{n_{J}}} \left( a_{\sum_{I' \in \varLambda_{I - 1}} m_{I'} + i,\sum_{J' \in \varLambda_{J - 1}} n_{J'} + j} \right. \right. \\
&\quad \left. \left.  b_{\sum_{J' \in \varLambda_{J - 1}} n_{J'} + j,\sum_{K' \in \varLambda_{K - 1}} o_{K'} + k} \right)_{(i,k) \in \varLambda_{m_{I}} \times \varLambda_{o_{K}}} \right)_{(I,K) \in \varLambda_{M} \times \varLambda_{O}}\\
&= \left( \sum_{J \in \varLambda_{N}}  \left( a_{\sum_{I' \in \varLambda_{I - 1}} m_{I'} + i,\sum_{J' \in \varLambda_{J - 1}} n_{J'} + j} \right)_{(i,j) \in \varLambda_{m_{I}} \times \varLambda_{n_{J}}} \right. \\
&\quad \left. \left( b_{\sum_{J' \in \varLambda_{J - 1}} n_{J'} + j,\sum_{K' \in \varLambda_{K - 1}} o_{K'} + j} \right)_{(j,k) \in \varLambda_{n_{J}} \times \varLambda_{o_{K}}} \right)_{(I,K) \in \varLambda_{M} \times \varLambda_{O}}\\
&= \left( \sum_{J \in \varLambda_{N}} {A_{IJ}B_{JK}} \right)_{(I,K) \in \varLambda_{M} \times \varLambda_{O}}
\end{align*}
\end{proof}
\begin{comment}
\begin{thm}\label{2.1.3.16}
可換環$R$上に$\left( a_{ij} \right)_{(i,j) \in \varLambda_{\sum_{I \in \varLambda_{M}} m_{I}} \times \varLambda_{\sum_{J \in \varLambda_{N}} n_{J}}} \in M_{\sum_{I \in \varLambda_{M}} m_{I},\sum_{J \in \varLambda_{N}} n_{J}}(R)$なる行列$\left( a_{ij} \right)_{(i,j) \in \varLambda_{\sum_{I \in \varLambda_{M}} m_{I}} \times \varLambda_{\sum_{J \in \varLambda_{N}} n_{J}}}$について次式が成り立つ。
\begin{align*}
\left( a_{ij} \right)_{(i,j) \in \varLambda_{\sum_{I \in \varLambda_{M}} m_{I}} \times \varLambda_{\sum_{J \in \varLambda_{N}} n_{J}}} = \left( \left( a_{\sum_{I' \in \varLambda_{I - 1}} m_{I'} + i,\sum_{J' \in \varLambda_{J - 1}} n_{J'} + j} \right)_{(i,j) \in \varLambda_{m_{I}} \times \varLambda_{n_{J}}} \right)_{(I,J) \in \varLambda_{M} \times \varLambda_{N}}
\end{align*}
\end{thm}
\begin{dfn}
その右辺の形をした行列をblock行列などといい、行列をそのblock行列に書き換えることをその行列をblock化するなどという。また、行列たち$\left( a_{\sum_{I' \in \varLambda_{I - 1}} m_{I'} + i,\sum_{J' \in \varLambda_{J - 1}} n_{J'} + j} \right)_{(i,j) \in \varLambda_{m_{I}} \times \varLambda_{n_{J}}}$のことをその行列の小行列たちという。
\end{dfn}
\begin{proof}
これは、その行列$\left( a_{ij} \right)_{(i,j) \in \varLambda_{\sum_{I \in \varLambda_{M}} m_{I}} \times \varLambda_{\sum_{J \in \varLambda_{N}} n_{J}}}$が成分表示されると、次式のようになることから、明らかであろう。
\begin{align*}
&\quad \begin{pmatrix}
a_{11} & \cdots & a_{1,\sum_{J' \in \varLambda_{N}} n_{J'}} \\
 \vdots & \ddots & \vdots \\
a_{\sum_{I' \in \varLambda_{M}} m_{I'},1} & \cdots & a_{\sum_{I' \in \varLambda_{M}} m_{I'},\sum_{J' \in \varLambda_{N}} n_{J'}} \\
\end{pmatrix}
&= \left( \begin{matrix}
a_{11} & \cdots & a_{1n_{1}} & \cdots \\
 \vdots & \ddots & \vdots & \ddots \\
a_{m_{1}1} & \cdots & a_{m_{1}n_{1}} & \cdots \\
 \vdots & \ddots & \vdots & \ddots \\
a_{\sum_{I' \in \varLambda_{M - 1}} m_{I'} + 1,1} & \cdots & a_{\sum_{I' \in \varLambda_{M - 1}} m_{I'} + 1,n_{1}} & \cdots \\
 \vdots & \ddots & \vdots & \ddots \\
a_{\sum_{I' \in \varLambda_{M}} m_{I'},1} & \cdots & a_{\sum_{I' \in \varLambda_{M}} m_{I'},n_{1}} & \cdots \\
\end{matrix} \right. \\
&\quad \left. \begin{matrix}
  a_{1,\sum_{J' \in \varLambda_{N - 1}} n_{J'} + 1} & \cdots & a_{1,\sum_{J' \in \varLambda_{N}} n_{J'}} \\
  \vdots & \ddots & \vdots \\
  a_{m_{1},\sum_{J' \in \varLambda_{N - 1}} n_{J'} + 1} & \cdots & a_{m_{1},\sum_{J' \in \varLambda_{N}} n_{J'}} \\
  \vdots & \ddots & \vdots \\
  a_{\sum_{I' \in \varLambda_{M - 1}} m_{I'} + 1,\sum_{J' \in \varLambda_{N - 1}} n_{J'} + 1} & \cdots & a_{\sum_{I' \in \varLambda_{M - 1}} m_{I'} + 1,\sum_{J' \in \varLambda_{N}} n_{J'}} \\
  \vdots & \ddots & \vdots \\
  a_{\sum_{I' \in \varLambda_{M}} m_{I'},\sum_{J' \in \varLambda_{N - 1}} n_{J'} + 1} & \cdots & a_{\sum_{I' \in \varLambda_{M}} m_{I'},\sum_{J' \in \varLambda_{N}} n_{J'}} \\
  \end{matrix} \right) \\
&= \left( \begin{matrix}
\begin{pmatrix}
a_{11} & \cdots & a_{1n_{1}} \\
 \vdots & \ddots & \vdots \\
a_{m_{1}1} & \cdots & a_{m_{1}n_{1}} \\
\end{pmatrix} & \cdots \\
 \vdots & \ddots \\
\begin{pmatrix}
a_{\sum_{I' \in \varLambda_{M - 1}} m_{I'} + 1,1} & \cdots & a_{\sum_{I' \in \varLambda_{M - 1}} m_{I'} + 1,n_{1}} \\
 \vdots & \ddots & \vdots \\
a_{\sum_{I' \in \varLambda_{M}} m_{I'},1} & \cdots & a_{\sum_{I' \in \varLambda_{M}} m_{I'},n_{1}} \\
\end{pmatrix} & \cdots\\
\end{matrix} \right.
&\quad \left. \begin{matrix}
  \begin{pmatrix}
  a_{1,\sum_{J' \in \varLambda_{N - 1}} n_{J'} + 1} & \cdots & a_{1,\sum_{J' \in \varLambda_{N}} n_{J'}} \\
   \vdots & \ddots & \vdots \\
  a_{m_{1},\sum_{J' \in \varLambda_{N - 1}} n_{J'} + 1} & \cdots & a_{m_{1},\sum_{J' \in \varLambda_{N}} n_{J'}} \\
  \end{pmatrix} \\
  \vdots \\
  \begin{pmatrix}
  a_{\sum_{I' \in \varLambda_{M - 1}} m_{I'} + 1,\sum_{J' \in \varLambda_{N - 1}} n_{J'} + 1} & \cdots & a_{\sum_{I' \in \varLambda_{M - 1}} m_{I'} + 1,\sum_{J' \in \varLambda_{N}} n_{J'}} \\
   \vdots & \ddots & \vdots \\
  a_{\sum_{I' \in \varLambda_{M}} m_{I'},\sum_{J' \in \varLambda_{N - 1}} n_{J'} + 1} & \cdots & a_{\sum_{I' \in \varLambda_{M}} m_{I'},\sum_{J' \in \varLambda_{N}} n_{J'}} \\
  \end{pmatrix} \\
  \end{matrix} \right)
\end{align*}
\end{proof}
\begin{thm}\label{2.1.3.17}
このとき、$\forall\left( a_{ij} \right)_{(i,j) \in \varLambda_{\sum_{I \in \varLambda_{M}} m_{I}} \times \varLambda_{\sum_{J \in \varLambda_{N}} n_{J}}},\left( b_{ij} \right)_{(i,j) \in \varLambda_{\sum_{I \in \varLambda_{M}} m_{I}} \times \varLambda_{\sum_{J \in \varLambda_{N}} n_{J}}} \in M_{\sum_{I \in \varLambda_{M}} m_{I},\sum_{J \in \varLambda_{N}} n_{J}}(R)$ $\forall k \in R$に対し$\forall(I,J) \in \varLambda_{M} \times \varLambda_{N}$に対し、次式のように行列たち$A_{IJ}$、$B_{IJ}$を定めると、
\begin{align*}
A_{IJ} &= \left( a_{\sum_{I' \in \varLambda_{I - 1}} m_{I'} + i,\sum_{J' \in \varLambda_{J - 1}} n_{J'} + j} \right)_{(i,j) \in \varLambda_{m_{I}} \times \varLambda_{n_{J}}} \in M_{m_{I}n_{J}}(R)\\
B_{IJ} &= \left( b_{\sum_{I' \in \varLambda_{I - 1}} m_{I'} + i,\sum_{J' \in \varLambda_{J - 1}} n_{J'} + j} \right)_{(i,j) \in \varLambda_{m_{I}} \times \varLambda_{n_{J}}} \in M_{m_{I}n_{J}}(R)
\end{align*}
次式が成り立つ。
\begin{align*}
\left( A_{IJ} \right)_{(I,J) \in \varLambda_{M} \times \varLambda_{N}} + \left( B_{IJ} \right)_{(I,J) \in \varLambda_{M} \times \varLambda_{N}} &= \left( A_{IJ} + B_{IJ} \right)_{(I,J) \in \varLambda_{M} \times \varLambda_{N}}\\
k\left( A_{IJ} \right)_{(I,J) \in \varLambda_{M} \times \varLambda_{N}} &= \left( kA_{IJ} \right)_{(I,J) \in \varLambda_{M} \times \varLambda_{N}}
\end{align*}
なお、その条件は上に述べた通り2つの行列たち$\left( a_{ij} \right)_{(i,j) \in \varLambda_{\sum_{I \in \varLambda_{M}} m_{I}} \times \varLambda_{\sum_{J \in \varLambda_{N}} n_{J}}}$、$\left( b_{ij} \right)_{(i,j) \in \varLambda_{\sum_{I \in \varLambda_{M}} m_{I}} \times \varLambda_{\sum_{J \in \varLambda_{N}} n_{J}}}$の行数と列数が等しく分割の仕方も等しいこととなっている\footnote{誰か分かりやすい説明方法を教えてくれぇ――――――!! }。
\end{thm}
\begin{proof}
可換環$R$上に$\forall\left( a_{ij} \right)_{(i,j) \in \varLambda_{\sum_{I \in \varLambda_{M}} m_{I}} \times \varLambda_{\sum_{J \in \varLambda_{N}} n_{J}}},\left( b_{ij} \right)_{(i,j) \in \varLambda_{\sum_{I \in \varLambda_{M}} m_{I}} \times \varLambda_{\sum_{J \in \varLambda_{N}} n_{J}}} \in M_{\sum_{I \in \varLambda_{M}} m_{I},\sum_{J \in \varLambda_{N}} n_{J}}(R)$ $\forall k \in R$に対し$\forall(I,J) \in \varLambda_{M} \times \varLambda_{N}$に対し次式のように行列たち$A_{IJ}$、$B_{IJ}$を定める。
\begin{align*}
A_{IJ} &= \left( a_{\sum_{I' \in \varLambda_{I - 1}} m_{I'} + i,\sum_{J' \in \varLambda_{J - 1}} n_{J'} + j} \right)_{(i,j) \in \varLambda_{m_{I}} \times \varLambda_{n_{J}}} \in M_{m_{I}n_{J}}(R)\\
B_{IJ} &= \left( b_{\sum_{I' \in \varLambda_{I - 1}} m_{I'} + i,\sum_{J' \in \varLambda_{J - 1}} n_{J'} + j} \right)_{(i,j) \in \varLambda_{m_{I}} \times \varLambda_{n_{J}}} \in M_{m_{I}n_{J}}(R)
\end{align*}
このとき、任意の元の列$\left( a_{i} \right)_{i \in \mathbb{N}}$に対し、次式が成り立つことに注意すれば、
\begin{align*}
\sum_{i \in \varLambda_{0}} a_{i} = \sum_{i \in \varnothing} a_{i} = \sum_{\bot} a_{i} = 0
\end{align*}
したがって、次のようになる。
\begin{align*}
&\quad \left( A_{IJ} \right)_{(I,J) \in \varLambda_{M} \times \varLambda_{N}} + \left( B_{IJ} \right)_{(I,J) \in \varLambda_{M} \times \varLambda_{N}}
&= \left( \left( a_{\sum_{I' \in \varLambda_{I - 1}} m_{I'} + i,\sum_{J' \in \varLambda_{J - 1}} n_{J'} + j} \right)_{(i,j) \in \varLambda_{m_{I}} \times \varLambda_{n_{J}}} \right)_{(I,J) \in \varLambda_{M} \times \varLambda_{N}} \\
&\quad + \left( \left( b_{\sum_{I' \in \varLambda_{I - 1}} m_{I'} + i,\sum_{J' \in \varLambda_{J - 1}} n_{J'} + j} \right)_{(i,j) \in \varLambda_{m_{I}} \times \varLambda_{n_{J}}} \right)_{(I,J) \in \varLambda_{M} \times \varLambda_{N}}\\
&= \left( a_{ij} \right)_{(i,j) \in \varLambda_{\sum_{I \in \varLambda_{M}} m_{I}} \times \varLambda_{\sum_{J \in \varLambda_{N}} n_{J}}} + \left( b_{ij} \right)_{(i,j) \in \varLambda_{\sum_{I \in \varLambda_{M}} m_{I}} \times \varLambda_{\sum_{J \in \varLambda_{N}} n_{J}}}\\
&= \left( a_{ij} + b_{ij} \right)_{(i,j) \in \varLambda_{\sum_{I \in \varLambda_{M}} m_{I}} \times \varLambda_{\sum_{J \in \varLambda_{N}} n_{J}}}\\
&= \left( \left( a_{\sum_{I' \in \varLambda_{I - 1}} m_{I'} + i,\sum_{J' \in \varLambda_{J - 1}} n_{J'} + j} + b_{\sum_{I' \in \varLambda_{I - 1}} m_{I'} + i,\sum_{J' \in \varLambda_{J - 1}} n_{J'} + j} \right)_{(i,j) \in \varLambda_{m_{I}} \times \varLambda_{n_{J}}} \right)_{(I,J) \in \varLambda_{M} \times \varLambda_{N}}\\
&= \left( \left( a_{\sum_{I' \in \varLambda_{I - 1}} m_{I'} + i,\sum_{J' \in \varLambda_{J - 1}} n_{J'} + j} \right)_{(i,j) \in \varLambda_{m_{I}} \times \varLambda_{n_{J}}} \right. \\
&\quad \left.+ \left( b_{\sum_{I' \in \varLambda_{I - 1}} m_{I'} + i,\sum_{J' \in \varLambda_{J - 1}} n_{J'} + j} \right)_{(i,j) \in \varLambda_{m_{I}} \times \varLambda_{n_{J}}} \right)_{(I,J) \in \varLambda_{M} \times \varLambda_{N}}\\
&= \left( A_{IJ} + B_{IJ} \right)_{(I,J) \in \varLambda_{M} \times \varLambda_{N}}
\end{align*}\par
また、次のようになる。
\begin{align*}
\quad k\left( A_{IJ} \right)_{(I,J) \in \varLambda_{M} \times \varLambda_{N}} &= k\left( \left( a_{\sum_{I' \in \varLambda_{I - 1}} m_{I'} + i,\sum_{J' \in \varLambda_{J - 1}} n_{J'} + j} \right)_{(i,j) \in \varLambda_{m_{I}} \times \varLambda_{n_{J}}} \right)_{(I,J) \in \varLambda_{M} \times \varLambda_{N}}\\
&= k\left( a_{ij} \right)_{(i,j) \in \varLambda_{\sum_{I \in \varLambda_{M}} m_{I}} \times \varLambda_{\sum_{J \in \varLambda_{N}} n_{J}}}\\
&= \left( ka_{ij} \right)_{(i,j) \in \varLambda_{\sum_{I \in \varLambda_{M}} m_{I}} \times \varLambda_{\sum_{J \in \varLambda_{N}} n_{J}}}\\
&= \left( \left( ka_{\sum_{I' \in \varLambda_{I - 1}} m_{I'} + i,\sum_{J' \in \varLambda_{J - 1}} n_{J'} + j} \right)_{(i,j) \in \varLambda_{m_{I}} \times \varLambda_{n_{J}}} \right)_{(I,J) \in \varLambda_{M} \times \varLambda_{N}}\\
&= \left( k\left( a_{\sum_{I' \in \varLambda_{I - 1}} m_{I'} + i,\sum_{J' \in \varLambda_{J - 1}} n_{J'} + j} \right)_{(i,j) \in \varLambda_{m_{I}} \times \varLambda_{n_{J}}} \right)_{(I,J) \in \varLambda_{M} \times \varLambda_{N}}\\
&= \left( kA_{IJ} \right)_{(I,J) \in \varLambda_{M} \times \varLambda_{N}}
\end{align*}
\end{proof}
\begin{thm}\label{2.1.3.18}
$\forall\left( a_{ij} \right)_{(i,j) \in \varLambda_{\sum_{I \in \varLambda_{M}} m_{I}} \times \varLambda_{\sum_{J \in \varLambda_{N}} n_{J}}} \in M_{\sum_{I \in \varLambda_{M}} m_{I},\sum_{J \in \varLambda_{N}} n_{J}}(R)$ $\forall\left( b_{ij} \right)_{(i,j) \in \varLambda_{\sum_{I \in \varLambda_{N}} n_{I}} \times \varLambda_{\sum_{J \in \varLambda_{O}} o_{J}}} \in M_{\sum_{I \in \varLambda_{N}} n_{I},\sum_{J \in \varLambda_{O}} o_{J}}(R)\forall k \in R$に対し、$\forall(I,J) \in \varLambda_{M} \times \varLambda_{N}$に対し、次式のように小行列たち$A_{IJ}$を
\begin{align*}
A_{IJ} = \left( a_{\sum_{I' \in \varLambda_{I - 1}} m_{I'} + i,\sum_{J' \in \varLambda_{J - 1}} n_{J'} + j} \right)_{(i,j) \in \varLambda_{m_{I}} \times \varLambda_{n_{J}}} \in M_{m_{I}n_{J}}(R)
\end{align*}
$\forall(I,J) \in \varLambda_{N} \times \varLambda_{O}$に対し、次式のように小行列たち$B_{IJ}$を定めると、
\begin{align*}
B_{IJ} = \left( b_{\sum_{I' \in \varLambda_{I - 1}} n_{I'} + i,\sum_{J' \in \varLambda_{J - 1}} o_{J'} + j} \right)_{(i,j) \in \varLambda_{n_{I}} \times \varLambda_{o_{J}}} \in M_{n_{I}o_{J}}(R)
\end{align*}
次式が成り立つ。
\begin{align*}
\left( A_{IJ} \right)_{(I,J) \in \varLambda_{M} \times \varLambda_{N}}\left( B_{IJ} \right)_{(I,J) \in \varLambda_{N} \times \varLambda_{O}} = \left( \sum_{K \in \varLambda_{N}} {A_{IK}B_{KJ}} \right)_{(I,J) \in \varLambda_{M} \times \varLambda_{O}}
\end{align*}
なお、その条件は上に述べた通りその行列$\left( a_{ij} \right)_{(i,j) \in \varLambda_{\sum_{I \in \varLambda_{M}} m_{I}} \times \varLambda_{\sum_{J \in \varLambda_{N}} n_{J}}}$の列数とその行列$\left( b_{ij} \right)_{(i,j) \in \varLambda_{\sum_{I \in \varLambda_{N}} n_{I}} \times \varLambda_{\sum_{J \in \varLambda_{O}} o_{J}}}$の行数が等しくその行列$\left( a_{ij} \right)_{(i,j) \in \varLambda_{\sum_{I \in \varLambda_{M}} m_{I}} \times \varLambda_{\sum_{J \in \varLambda_{N}} n_{J}}}$の分割された列のかたまりの個数とその行列$\left( b_{ij} \right)_{(i,j) \in \varLambda_{\sum_{I \in \varLambda_{N}} n_{I}} \times \varLambda_{\sum_{J \in \varLambda_{O}} o_{J}}}$の分割された行のかたまりの個数も等しく、$\forall\left( I_{1},J_{1} \right) \in \varLambda_{M} \times \varLambda_{N}\forall\left( I_{2},J_{2} \right) \in \varLambda_{N} \times \varLambda_{O}$に対し、その小行列$A_{I_{1}J_{1}}$の列数とその小行列$B_{I_{2}J_{2}}$の行数も等しいこととなっている\footnote{ここも誰か分かりやすい説明方法を教えてくれぇ――――――!! }。
\end{thm}
\begin{proof}
可換環$R$上に$\forall\left( a_{ij} \right)_{(i,j) \in \varLambda_{\sum_{I \in \varLambda_{M}} m_{I}} \times \varLambda_{\sum_{J \in \varLambda_{N}} n_{J}}} \in M_{\sum_{I \in \varLambda_{M}} m_{I},\sum_{J \in \varLambda_{N}} n_{J}}(R)$ $\forall\left( b_{ij} \right)_{(i,j) \in \varLambda_{\sum_{I \in \varLambda_{N}} n_{I}} \times \varLambda_{\sum_{J \in \varLambda_{O}} o_{J}}} \in M_{\sum_{I \in \varLambda_{N}} n_{I},\sum_{J \in \varLambda_{O}} o_{J}}(R)\forall k \in R$に対し、$\forall(I,J) \in \varLambda_{M} \times \varLambda_{N}$に対し、次式のように行列たち$A_{IJ}$を
\begin{align*}
A_{IJ} = \left( a_{\sum_{I' \in \varLambda_{I - 1}} m_{I'} + i,\sum_{J' \in \varLambda_{J - 1}} n_{J'} + j} \right)_{(i,j) \in \varLambda_{m_{I}} \times \varLambda_{n_{J}}} \in M_{m_{I}n_{J}}(R)
\end{align*}
$\forall(I,J) \in \varLambda_{N} \times \varLambda_{O}$に対し、次式のように行列たち$B_{IJ}$を定める。
\begin{align*}
B_{IJ} = \left( b_{\sum_{I' \in \varLambda_{I - 1}} n_{I'} + i,\sum_{J' \in \varLambda_{J - 1}} o_{J'} + j} \right)_{(i,j) \in \varLambda_{n_{I}} \times \varLambda_{o_{J}}} \in M_{n_{I}o_{J}}(R)
\end{align*}\par
このとき、次のようになる。
\begin{align*}
&\quad \left( A_{IJ} \right)_{(I,J) \in \varLambda_{M} \times \varLambda_{N}}\left( B_{IJ} \right)_{(I,J) \in \varLambda_{N} \times \varLambda_{O}} \\
&= \left( \left( a_{\sum_{I' \in \varLambda_{I - 1}} m_{I'} + i,\sum_{J' \in \varLambda_{J - 1}} n_{J'} + j} \right)_{(i,j) \in \varLambda_{m_{I}} \times \varLambda_{n_{J}}} \right)_{(I,J) \in \varLambda_{M} \times \varLambda_{N}} \\
&\quad \left( \left( b_{\sum_{I' \in \varLambda_{I - 1}} n_{I'} + i,\sum_{J' \in \varLambda_{J - 1}} o_{J'} + j} \right)_{(i,j) \in \varLambda_{n_{I}} \times \varLambda_{o_{J}}} \right)_{(I,J) \in \varLambda_{N} \times \varLambda_{O}}\\
&= \left( a_{ij} \right)_{(i,j) \in \varLambda_{\sum_{I \in \varLambda_{M}} m_{I}} \times \varLambda_{\sum_{J \in \varLambda_{N}} n_{J}}}\left( b_{ij} \right)_{(i,j) \in \varLambda_{\sum_{I \in \varLambda_{N}} n_{I}} \times \varLambda_{\sum_{J \in \varLambda_{O}} o_{J}}}\\
&= \left( \sum_{k \in \varLambda_{\sum_{K \in \varLambda_{N}} n_{K}}} {a_{ik}b_{kj}} \right)_{(i,j) \in \varLambda_{\sum_{I \in \varLambda_{M}} m_{I}} \times \varLambda_{\sum_{J \in \varLambda_{O}} o_{J}}}\\
&= \left( \left( \sum_{k \in \varLambda_{\sum_{K \in \varLambda_{N}} n_{K}}} {a_{\sum_{I' \in \varLambda_{I - 1}} m_{I'} + i,k}b_{k,\sum_{J' \in \varLambda_{J - 1}} o_{J'} + j}} \right)_{(i,j) \in \varLambda_{m_{I}} \times \varLambda_{o_{J}}} \right)_{(I,J) \in \varLambda_{M} \times \varLambda_{O}}\\
&= \left( \left( \sum_{K \in \varLambda_{N}} \sum_{k \in \varLambda_{n_{K}}}  a_{\sum_{I' \in \varLambda_{I - 1}} m_{I'} + i,\sum_{J' \in \varLambda_{J - 1}} n_{J'} + k} \right. \right. \\
&\quad \left. \left. b_{\sum_{I' \in \varLambda_{I - 1}} n_{I'} + k,\sum_{J' \in \varLambda_{J - 1}} o_{J'} + j} \right)_{(i,j) \in \varLambda_{m_{I}} \times \varLambda_{o_{J}}} \right)_{(I,J) \in \varLambda_{M} \times \varLambda_{O}}\\
&= \left( \sum_{K \in \varLambda_{N}}  \sum_{k \in \varLambda_{n_{K}}} \left( a_{\sum_{I' \in \varLambda_{I - 1}} m_{I'} + i,\sum_{J' \in \varLambda_{J - 1}} n_{J'} + k} \right. \right. \\
&\quad \left. \left.  b_{\sum_{I' \in \varLambda_{I - 1}} n_{I'} + k,\sum_{J' \in \varLambda_{J - 1}} o_{J'} + j} \right)_{(i,j) \in \varLambda_{m_{I}} \times \varLambda_{o_{J}}} \right)_{(I,J) \in \varLambda_{M} \times \varLambda_{O}}\\
&= \left( \sum_{K \in \varLambda_{N}}  \left( a_{\sum_{I' \in \varLambda_{I - 1}} m_{I'} + i,\sum_{J' \in \varLambda_{J - 1}} n_{J'} + j} \right)_{(i,j) \in \varLambda_{m_{I}} \times \varLambda_{n_{J}}} \right. \\
&\quad \left. \left( b_{\sum_{I' \in \varLambda_{I - 1}} n_{I'} + i,\sum_{J' \in \varLambda_{J - 1}} o_{J'} + j} \right)_{(i,j) \in \varLambda_{n_{I}} \times \varLambda_{o_{J}}} \right)_{(I,J) \in \varLambda_{M} \times \varLambda_{O}}\\
&= \left( \sum_{K \in \varLambda_{N}} {A_{IK}B_{KJ}} \right)_{(I,J) \in \varLambda_{M} \times \varLambda_{O}}
\end{align*}
\end{proof}
\end{comment}
\begin{thebibliography}{50}
  \bibitem{1}
    松坂和夫, 線型代数入門, 岩波書店, 1980. 新装版第2刷 p41-73 ISBN978-4-00-029872-8
  \bibitem{2}
    対馬龍司, 線形代数学講義, 共立出版, 2007. 改訂版8刷 p33-38,53-58,94-104 ISBN978-4-320-11097-7
  \bibitem{3}
    おぐえもん. "行列のブロック分割". OGUEMON.com. \url{https://oguemon.com/study/linear-algebra/matrix-block/} (2021-2-13 0:05 閲覧)
  \bibitem{4}
    中西敏浩. "数学基礎 IV (線形代数学) 講義ノート". 島根大学. \url{http://www.math.shimane-u.ac.jp/~tosihiro/skks4main.pdf} (2020-9-7 取得)
\end{thebibliography}
\end{document}
