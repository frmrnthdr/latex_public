\documentclass[a4paper]{jsarticle}
\setcounter{section}{1}
\setcounter{subsection}{0}
\usepackage{amsmath,amsfonts,amssymb,array,comment,mathtools,url,docmute}
\usepackage{longtable,booktabs,dcolumn,tabularx,mathtools,multirow,colortbl,xcolor}
\usepackage[dvipdfmx]{graphics}
\usepackage{bmpsize}
\usepackage{amsthm}
\usepackage{enumitem}
\setlistdepth{20}
\renewlist{itemize}{itemize}{20}
\setlist[itemize]{label=•}
\renewlist{enumerate}{enumerate}{20}
\setlist[enumerate]{label=\arabic*.}
\setcounter{MaxMatrixCols}{20}
\setcounter{tocdepth}{3}
\newcommand{\rotin}{\text{\rotatebox[origin=c]{90}{$\in $}}}
\newcommand{\amap}[6]{\text{\raisebox{-0.7cm}{\begin{tikzpicture} 
  \node (a) at (0, 1) {$\textstyle{#2}$};
  \node (b) at (#6, 1) {$\textstyle{#3}$};
  \node (c) at (0, 0) {$\textstyle{#4}$};
  \node (d) at (#6, 0) {$\textstyle{#5}$};
  \node (x) at (0, 0.5) {$\rotin $};
  \node (x) at (#6, 0.5) {$\rotin $};
  \draw[->] (a) to node[xshift=0pt, yshift=7pt] {$\textstyle{\scriptstyle{#1}}$} (b);
  \draw[|->] (c) to node[xshift=0pt, yshift=7pt] {$\textstyle{\scriptstyle{#1}}$} (d);
\end{tikzpicture}}}}
\newcommand{\twomaps}[9]{\text{\raisebox{-0.7cm}{\begin{tikzpicture} 
  \node (a) at (0, 1) {$\textstyle{#3}$};
  \node (b) at (#9, 1) {$\textstyle{#4}$};
  \node (c) at (#9+#9, 1) {$\textstyle{#5}$};
  \node (d) at (0, 0) {$\textstyle{#6}$};
  \node (e) at (#9, 0) {$\textstyle{#7}$};
  \node (f) at (#9+#9, 0) {$\textstyle{#8}$};
  \node (x) at (0, 0.5) {$\rotin $};
  \node (x) at (#9, 0.5) {$\rotin $};
  \node (x) at (#9+#9, 0.5) {$\rotin $};
  \draw[->] (a) to node[xshift=0pt, yshift=7pt] {$\textstyle{\scriptstyle{#1}}$} (b);
  \draw[|->] (d) to node[xshift=0pt, yshift=7pt] {$\textstyle{\scriptstyle{#2}}$} (e);
  \draw[->] (b) to node[xshift=0pt, yshift=7pt] {$\textstyle{\scriptstyle{#1}}$} (c);
  \draw[|->] (e) to node[xshift=0pt, yshift=7pt] {$\textstyle{\scriptstyle{#2}}$} (f);
\end{tikzpicture}}}}
\renewcommand{\thesection}{第\arabic{section}部}
\renewcommand{\thesubsection}{\arabic{section}.\arabic{subsection}}
\renewcommand{\thesubsubsection}{\arabic{section}.\arabic{subsection}.\arabic{subsubsection}}
\everymath{\displaystyle}
\allowdisplaybreaks[4]
\usepackage{vtable}
\theoremstyle{definition}
\newtheorem{thm}{定理}[subsection]
\newtheorem*{thm*}{定理}
\newtheorem{dfn}{定義}[subsection]
\newtheorem*{dfn*}{定義}
\newtheorem{axs}[dfn]{公理}
\newtheorem*{axs*}{公理}
\renewcommand{\headfont}{\bfseries}
\makeatletter
  \renewcommand{\section}{%
    \@startsection{section}{1}{\z@}%
    {\Cvs}{\Cvs}%
    {\normalfont\huge\headfont\raggedright}}
\makeatother
\makeatletter
  \renewcommand{\subsection}{%
    \@startsection{subsection}{2}{\z@}%
    {0.5\Cvs}{0.5\Cvs}%
    {\normalfont\LARGE\headfont\raggedright}}
\makeatother
\makeatletter
  \renewcommand{\subsubsection}{%
    \@startsection{subsubsection}{3}{\z@}%
    {0.4\Cvs}{0.4\Cvs}%
    {\normalfont\Large\headfont\raggedright}}
\makeatother
\makeatletter
\renewenvironment{proof}[1][\proofname]{\par
  \pushQED{\qed}%
  \normalfont \topsep6\p@\@plus6\p@\relax
  \trivlist
  \item\relax
  {
  #1\@addpunct{.}}\hspace\labelsep\ignorespaces
}{%
  \popQED\endtrivlist\@endpefalse
}
\makeatother
\renewcommand{\proofname}{\textbf{証明}}
\usepackage{tikz,graphics}
\usepackage[dvipdfmx]{hyperref}
\usepackage{pxjahyper}
\hypersetup{
 setpagesize=false,
 bookmarks=true,
 bookmarksdepth=tocdepth,
 bookmarksnumbered=true,
 colorlinks=false,
 pdftitle={},
 pdfsubject={},
 pdfauthor={},
 pdfkeywords={}}
\begin{document}
\subsection{記号論理学の初歩}
%\hypertarget{ux547dux984c}{%
\subsubsection{命題}%\label{ux547dux984c}}
\begin{dfn}
物事の判断について述べた文で客観的に判断されることができるものを命題という。
\end{dfn}
ここでは基本となる正しいか否か2択のみで判断される二値論理を扱おう\footnote{なお、多値論理というものも存在しますが、先に集合を厳密に構成し数などを導入しておく必要があり内容が膨大な量となりすごくめんどくさいので、ここでは省略させていただきます。}。
\begin{dfn}
ある命題があり、これの内容が正しいとき、この命題は真であるといい、それの内容が正しくないとき、この命題は偽であるという。
\end{dfn}
\begin{dfn}
このようにして考えると、命題は真偽を決める変数とみなせ、この変数を命題変数という。特に一定の値をとるものを命題定数という。
\end{dfn}
\begin{dfn}
真らしさを決め1つ以上の命題変数に作用させる規則を論理演算子、論理結合子という。次のように与えられた論理演算子のことをここでは基本論理演算子ということにする。
\begin{longtable}[c]{|c|l|}
\hline
表記 & 名称 \\
\hline \hline
$\neg p$, $\overline{p}$, $\sim p$ & $p$が成り立たない $p$の否定 $p$が否定される \\
\hline
\hspace{-0.5em}\begin{tabular}{c}
  $p \rightarrow q$, $p \supset q$, \\
  $q \leftarrow p$, $q \subset p$
\end{tabular} & \hspace{-0.5em}\begin{tabular}{l}
  $p$の$q$をの含意 $p$は$q$を含意する \\
  $p$が成り立つなら、$q$が成り立つ
\end{tabular} \\
\hline
$\forall x\in X[p(x)]$ & \hspace{-0.5em}\begin{tabular}{l}
  $x\in X$なる任意の$x$に対し、$p(x)$が成り立つ \\ 
  $x\in X$なる全ての$x$に対し、$p(x)$が成り立つ
\end{tabular} \\
\hline
\end{longtable}
\end{dfn}
\begin{dfn}\label{dfn1.1.1.5}
このような命題の真らしさを値として直感的に議論されることができる。このときの値を真理値、真偽値という。真、偽は真理値でそれぞれT、Fや1、0で書かれることが多い。1つ以上の命題変数があり、これらの入力としての全ての真理値とこれらに対する出力としての真理値を表にしたものを真理表、真理値表という。
\end{dfn}
命題の真らしさが直感的に理解されるのに便利である。しかしながら、ここでは、この内容が後述する量化に対応しきれないため、その定義\refeq{dfn1.1.1.5}が無視されることにする\footnote{でも、書籍によってはやっぱり正式な議論の対象としているものもありますよ~。}。
\begin{longtable}[c]{cccc}
\hline
$p$ & $q$ & $\neg p$ & $p\rightarrow q$ \\
\hline \hline
T & T & F & T \\
T & F & F & T \\
F & T & T & T \\
F & F & T & F \\
\hline
\end{longtable}
%\hypertarget{ux5f62ux5f0fux7684ux4f53ux7cfb}{%
\subsubsection{形式的体系}%\label{ux5f62ux5f0fux7684ux4f53ux7cfb}}
\begin{dfn}
次のように記号、表現、公理、推論規則が定められる\footnote{定義は公理として扱ってもよく(?)、ここでは、これらを同一視することにします。なお、定義にはある語句に対する意味付け、公理には議論で用いてもよい手続きというnuanceがあるときもあります。}。これらの組$S$を形式的体系という。
\begin{itemize}
\item
  記号$A$が可算的な集まりをなす1つのものとして与えられる。
\item
  表現$p$が記号の有限な列として与えられる。
\item
  公理$\alpha$が議論の出発となる特定の表現たちとして与えられ推論規則$R$が表現たちを関係づける。
\end{itemize}
このとき、推論規則$R$によって表現たちの集まりの個数が決まり任意の表現$q$とそれらの表現たち$p_{1}$、$p_{2}$、$\cdots$、$p_{n}$がその推論規則$R$の関係にあるかどうかが決められることができるとする。このとき、その表現$q$はそれらの表現たち$p_{1}$、$p_{2}$、$\cdots$、$p_{n}$からその推論規則$R$によって直接導かれるといい、$p_{1},p_{2},\cdots,p_{n} \vdash_{S}q\ \ \ (R)$や次のように書く。
\begin{align*}
\frac{p_{1},p_{2},\cdots,p_{n}}{q}R_{S}
\end{align*}
\end{dfn}
\begin{dfn}
表現たち$p_{1}$、$p_{2}$、$\cdots$、$p_{m}$を含む表現たち$\alpha_{1}$、$\alpha_{2}$、$\cdots$、$\alpha_{n}$が次のことのうちいづれか成り立つとき、これらの表現たち$p_{1}$、$p_{2}$、$\cdots$、$p_{m}$をその表現$\alpha_{n}$の仮定、前提、その表現$\alpha_{n}$を結論という。特に、このような表現たち$\alpha_{1}$、$\alpha_{2}$、$\cdots$、$\alpha_{n}$が存在するなら、その表現$q$は形式的体系$S$でそれらの表現たち$p_{1}$、$p_{2}$、$\cdots$、$p_{n}$から証明可能である、導かれるなどと、その表現たち$\alpha_{1}$、$\alpha_{2}$、$\cdots$、$\alpha_{n}$をその表現$\alpha_{n}$の演繹、証明と、その表現$\alpha_{n}$を定理という。このことを$\alpha_{1},\alpha_{2},\cdots,\alpha_{n - 1} \vDash_{S}\alpha_{n}$と書く。
\begin{itemize}
\item
  表現$\alpha_{i}$は仮定に含まれる。
\item
  表現$\alpha_{i}$は公理である。
\item
  表現$\alpha_{i}$がある1つの推論規則でこれより前の有限個数の表現たちから直接導かれる。
\end{itemize}
また、$\alpha_{1},\alpha_{2},\cdots,\alpha_{n - 1} \nvDash_{S}\alpha_{n}$が成り立つような例を反例という。
\end{dfn}
\begin{dfn}
形式的体系$S$の全ての表現$q$が定理なのかどうかが有限的な議論で決まるとき、その形式的体系$S$は決定可能であるという。表現$q$が形式的体系$S$で証明可能なことをその表現$q$はその形式的体系$S$で恒真であるという\footnote{書籍によっては真理値を正式な議論の対象とするときがあり、この場合、定理となり、これを健全性などといったりします。}。また、表現たち$q$、$\neg q$がいづれも形式的体系$S$で証明可能なとき、その形式的体系$S$は矛盾するという。そうでないとき、その形式的体系$S$は無矛盾であるという。
\end{dfn}
%\hypertarget{ux8ad6ux7406ux5f0f}{%
\subsubsection{論理式}%\label{ux8ad6ux7406ux5f0f}}
\begin{dfn}
未知の値、対象を表す文字や記号を変数といい、これを有限個数だけ含みこれら全てが具体的な値になることで命題になるような文や式を命題関数といい、そのときの変数。変数$x$の取りうる値の範囲をその変数$x$の定義域といい、$X$のように書くことが多く、これらを合わせて$x \in X$、$X \ni x$などと書く。その範囲$X$の中にはその変数$x$の取りうる値全て含まれ、これらの各々の値$\overline{x}$を対称定数、定数、値などという。1つの命題関数につき、対象となる変数たち全てとこれらの定義域たちを合わせてその命題関数の議論領域といい、$D$などと書く。議論領域$D$において変数$x$とその定義域$X$が定義されているとき、その変数$x$をその議論領域$D$における変数という。以下、その命題関数を命題の最小単位として考えるので、その命題関数を原子論理式というときがある。
\end{dfn}
例えば、変数たち$x$、$y$の定義域がどちらも$\mathbf{Z}$であり、以上を議論領域$D$とする。このとき、次の命題たちはいずれもその議論領域$D$で議論の対象となる。
\begin{quote}
$p(x)$:$x$は偶数である。\\
$q(x,y)$:$x + y$は奇数である。
\end{quote}
一方、次の命題は変数たち$a$、$b$がその議論領域$D$のもとで定義されていないので、その議論領域$D$では議論の対象外となる。
\begin{quote}
$r(a,b)$:$a$の人口は$b$のものより小さい。
\end{quote}
\begin{dfn}
これらの記号たち$\exists$、$\exists!$、$\forall$をそれぞれ存在記号、一意的存在記号、全称記号、存在量化記号、一意的存在量化記号、全称量化記号などといい、これらをまとめて量化記号、限定記号などといい、命題変数$p$に量化記号$Q$と定義域$X$の変数$x$を用いて$Qx \in X\lbrack p\rbrack$と書き換えることを量化という。
\end{dfn}
\begin{axs}
以下のように定義されるものを論理式という。
\begin{itemize}
\item
  命題変数は議論領域$D$の論理式である。
\item
  命題定数は議論領域$D$の論理式である。
\item
  $p$が議論領域$D$の論理式であるなら、$\neg p$も議論領域$D$の論理式である。
\item
  $p$、$q$どちらも議論領域$D$の論理式であるなら、$p \rightarrow q$も議論領域$D$の論理式である。
\item
  量化記号$Q$を用いて$p$が議論領域$D$の論理式で変数$x$の定義域が$X$のその変数$x$が議論領域$D$の変数であるなら、$Qx \in X\lbrack A\rbrack$も議論領域$D$の論理式である。
\item
  これ以外のものは論理式でない。
\end{itemize}
\end{axs}
\begin{axs}
以下のように定義されるものを部分論理式という。
\begin{itemize}
\item
  論理式$p$はその論理式$p$の部分論理式である。
\item
  論理式$p$の部分論理式は全て論理式$\neg p$の部分論理式である。
\item
  論理式たち$p$、$q$の部分論理式は全て論理式$p \rightarrow q$の部分論理式である。
\item
  量化記号$Q$と定義域$X$の変数$x$、論理式$p$を用いて論理式$p$の部分論理式は論理式$Qx \in X\lbrack p\rbrack$の部分論理式である。
\item
  これ以外のものは部分論理式ではない。
\end{itemize}
\end{axs}
\begin{dfn}
量化記号$Q$、定義域$X$の変数$x$、論理式$A$を用いて$Qx \in X\lbrack A\rbrack$の形で表されるとき、その論理式$A$をその量化記号$Q$の作用域という。その量化記号$Q$の直後の変数$x$を除く変数$x$が現れるとき、それぞれの箇所を論理式$Qx \in X\lbrack A\rbrack$におけるその変数$x$の現れという。論理式が$Qx \in X$を含むとき、その量化記号$Q$はその変数$x$を束縛するという。その変数$x$のそれぞれの現れが変数$x$を束縛するその量化記号$Q$の作用域に属するとき、その現れは束縛されているといい、このときの変数$x$を束縛変数という。一方で、属さないとき、その現れは自由であるといい、このときの変数$x$を自由変数という。
\end{dfn}
例えば、定義域$X$の変数$x$を含む命題関数たち$p(x)$、$q(x)$を含む論理式$\forall x \in X\left[ \neg p(x) \right] \land q(x)$において変数$x$の現れは$p(x)$と$q(x)$の2か所で全称記号$\forall$がその変数$x$を束縛しその作用域は$\neg p(x)$となる。したがって、$p(x)$中での$x$の現れは束縛されており、$q(x)$の$x$の現れは自由である。
\begin{dfn}
論理式$p$が変数たち$x_{1}$、$x_{2}$、$\cdots$、$x_{n}$の自由な現れをもつとき、この論理式を$p\left( x_{1},x_{2},\cdots x_{n} \right)$と書くことがある。変数の自由な現れを少なくとも1つもつ論理式を開論理式、変数の自由な現れをもたない論理式を閉論理式という。
\end{dfn}
例えば、開論理式$p\left( x_{1},x_{2},\cdots x_{n} \right)$の一部の自由な現れの変数たち$x_{1}$、$x_{2}$、$\cdots$、$x_{m}$にそれぞれ値々$\overline{x_{1}}$、$\overline{x_{2}}$、$\cdots$、$\overline{x_{m}}$と代入したとき、その開論理式$p\left( x_{1},x_{2},\cdots x_{n} \right)$は変数$x_{m + 1}$、$x_{m + 2}$、$\cdots$、$x_{n}$の自由な現れをまだもつので、その開論理式はまだ開論理式である。一方、開論理式$p\left( x_{1},x_{2},\cdots x_{n} \right)$の全ての自由な現れの変数たち$x_{1}$、$x_{2}$、$\cdots$、$x_{n}$にそれぞれ値々$\overline{x_{1}}$、$\overline{x_{2}}$、$\cdots$、$\overline{x_{m}}$と代入したとき、その開論理式$p\left( x_{1},x_{2},\cdots x_{n} \right)$は変数の自由な現れをもたないので、その開論理式は閉論理式である。
\begin{dfn}
変数たち$x_{1}$、$x_{2}$、$\cdots$、$x_{m}$の自由な現れをもつ開論理式$p\left( x_{1},x_{2},\cdots x_{n} \right)$がある命題定数となるように定めるとき、議論領域$D$、その命題関数$p$の形状、それらの変数たち$x_{1}$、$x_{2}$、$\cdots$、$x_{n}$に代入する値々$\overline{x_{1}}$、$\overline{x_{2}}$、$\cdots$、$\overline{x_{n}}$3つを具体的に定める必要があり、その3つのことからなる組をその命題関数$p\left( x_{1},x_{2},\cdots x_{n} \right)$の論理解釈、解釈などという。
\end{dfn}
その命題関数$p\left( x_{1},x_{2},\cdots x_{n} \right)$に何らかの論理解釈が与えられれば、その開論理式$p\left( x_{1},x_{2},\cdots x_{n} \right)$は命題定数になる。
\begin{dfn}
ある開論理式$p\left( x_{1},x_{2},\cdots x_{n} \right)$について議論領域$D$とその命題関数$p$の形状が与えられたとき、その命題定数$p\left( \overline{x_{1}},\overline{x_{2}},\cdots,\overline{x_{n}} \right)$が証明可能となるような値々$\overline{x_{1}}$、$\overline{x_{2}}$、$\cdots$、$\overline{x_{n}}$のとりうる範囲をその命題関数$p$の真理集合といい$\varphi(p)$と書く。
\end{dfn}
その命題関数$p\left( x_{1},x_{2},\cdots x_{n} \right)$の一部の自由な現れの変数たちにそれぞれ値々を代入したときのその命題関数を$p'$とおくと、これは残りの変数たちの自由な現れをもつ開論理式である。一方、全ての変数たちに値々を代入したとき、その論理式$p\left( \overline{x_{1}},\overline{x_{2}},\cdots,\overline{x_{n}} \right)$は閉論理式であり、$\left( \overline{x_{1}},\overline{x_{2}},\cdots,\overline{x_{n}} \right) \in \varphi(p)$ならその論理式$p\left( \overline{x_{1}},\overline{x_{2}},\cdots,\overline{x_{n}} \right)$は証明可能で、$\neg\left( \overline{x_{1}},\overline{x_{2}},\cdots,\overline{x_{n}} \right) \in \varphi(p)$ならその論理式$p\left( \overline{x_{1}},\overline{x_{2}},\cdots,\overline{x_{n}} \right)$は証明可能でない。\par
例えば、命題関数$p(x,y)$は変数たち$x$、$y$の自由な現れをもつ開論理式であり、議論領域$D$はこれらの変数たち$x$、$y$の定義域$X$、$Y$どちらもA村に市民権をもつ人々で$p(x,y)$が人$x$と人$y$互いに知り合いであることとする。$\forall x \in X\left[ p(x,y) \right]$は人々$x$全て人$y$の知り合いであるという意味になり、一方、$\exists y \in Y\left[ p(x,y) \right]$は人$x$の知り合いである人$y$が存在するという意味になる。次に、$\forall x \in X\forall y \in Y\left[ p(x,y) \right]$はどの組$(x,y)$でも互いに知り合いであるという意味であり、実際、どの組$(x,y)$でも互いに知り合いであるなら、この論理式は真であり、互いに知り合いでない組$(x,y)$が存在するなら、この論理式は偽である。$\exists x \in X\exists y \in Y\left[ p(x,y) \right]$は互いに知り合いである組$(x,y)$が存在するという意味であり、実際、互いに知り合いである組$(x,y)$が存在するなら、この論理式は真であり、どの組$(x,y)$でも互いに知り合いでないなら、この論理式は偽である。さらに、具体的に、A村の村民たち全員が$x_{1}$、$x_{2}$、$y_{1}$、$y_{2}$で組々$\left( x_{1},x_{2} \right)$、$\left( x_{1},y_{1} \right)$、$\left( x_{2},y_{2} \right)$、$\left( y_{1},y_{2} \right)$が互いに知り合いであるとすると、組々$\left( x_{1},x_{2} \right)$、$\left( x_{1},y_{1} \right)$、$\left( x_{2},x_{1} \right)$、$\left( x_{2},y_{2} \right)$、$\left( y_{1},x_{1} \right)$、$\left( y_{1},y_{2} \right)$、$\left( y_{2},x_{2} \right)$、$\left( y_{2},y_{1} \right)$が互いに知り合いであることになり、いずれも真理集合$\varphi(P)$に属する。
\begin{dfn}
上記の論理演算子に基づいて命題たち$p$、$q$を用いて次の論理演算子たちを定めよう。
\begin{longtable}[c]{|c|c|l|}
\hline
表記 & 定義 & 名称 \\
\hline \hline
$p \vee q$, $p\ \mathrm{or} \ q$ & $\neg p \rightarrow q$ & \hspace{-0.5em}\begin{tabular}{l}
  $p$が成り立つ、または、$q$が成り立つ \\
  $p$と$q$の論理和 $p$または$q$が成り立つ 
\end{tabular}\\
\hline
\hspace{-0.5em}\begin{tabular}{c}
  $p \land q$, $p\ \mathrm{and} \ q$, \\
  $p\ \& q$, $\left\{ \begin{matrix} p \\ q \end{matrix} \right. $ 
\end{tabular} & $\neg(p \rightarrow \neg q)$ & \hspace{-0.5em}\begin{tabular}{l}
  $p$が成り立つかつ、$q$が成り立つ \\
  $p$と$q$の論理積 $pかつq$が成り立つ 
\end{tabular} \\
\hline
$p \leftrightarrow q$, $p \equiv q$, $p\ \mathrm{xnor}\ q$ & $\neg\left( (p \rightarrow q) \rightarrow \neg(q \rightarrow p) \right)$ & \hspace{-0.5em}\begin{tabular}{l}
  $p$が成り立つならそのときに限り、$q$が成り立つ \\ 
  $p$の$q$との同値 $p$が$q$と同値である
\end{tabular}\\
\hline
$p\underline{\vee}q$, $p \nleftrightarrow q$, $p\ \mathrm{xor}\ q$ & $\neg(p \rightarrow q) \rightarrow (q\rightarrow p)$ & $p$と$q$の排他的論理和 \\
\hline
$p \downarrow q$, $p|q$, $p\ \mathrm{nor}\ q$ & $\neg(\neg p \rightarrow q)$ & $p$と$q$の否定論理和 \\
\hline
$p \uparrow q$, $p|q$, $p\ \mathrm{nand}\ q$ & $p \rightarrow \neg q$ & $p$と$q$の否定論理積 \\
\hline
$p \nrightarrow q$, $q \nleftarrow p$ & $\neg(p \rightarrow q)$ & $p$の$q$をの非含意 \\
\hline
$\top$ & $p \rightarrow p$ & 恒真式 \\
\hline
$\bot$ & $\neg(p \rightarrow p)$ & 恒偽式 \\
\hline
$\exists x \in X[p(x)]$ & $\neg\forall x \in X[\neg p(x)]$ & \hspace{-0.5em}\begin{tabular}{l}
  $p(x)$なる$x$が$X$に存在する \\
  ある$x$が$X$に存在して$p(x)$が成り立つ \\ 
  ある$x$に対し、$p(x)$が成り立つような \\ 
  その$x$が$X$に存在する
\end{tabular} \\
\hline
\end{longtable}
\begin{longtable}[c]{ccccccccccc}
\hline
$p$ & $q$ & $p\vee q$ & $p\land q$ & $p \leftrightarrow q$ & $p\underline{\vee}q$ & $p \downarrow q$ & $p \uparrow q$ & $p \nrightarrow q$ & $\top $ & $\bot $ \\
\hline \hline
T & T & T & T & T & F & F & F & F & T & F \\
T & F & F & F & F & T & F & T & T & T & F \\
F & T & T & F & F & T & F & T & F & T & F \\
F & F & T & F & T & F & T & T & F & T & F \\
\hline
\end{longtable}
\end{dfn}
\begin{dfn}
特に、論理式たち$p \rightarrow q$、$p \leftrightarrow q$が証明可能なとき、これらを次のようにそれぞれ$p \Rightarrow q$、$p \Leftrightarrow q$と書く。
\begin{longtable}[c]{|c|c|l|}
\hline 
表記 & もとの命題 & 名称 \\
\hline \hline 
$p \Rightarrow q$, $q \Leftarrow p$, $q\ \mathrm{if}\ p$ & $p \rightarrow q$ & \hspace{-0.5em}\begin{tabular}{l}
  $p$の$q$への十分条件 $p$が成り立つことは \\
  $q$が成り立つことへの十分条件である \\
  $q$の$p$への必要条件 $q$が成り立つことは\\
  $p$が成り立つことへの必要条件である
\end{tabular}\\
\hline
$p \Leftrightarrow q$, $p\ \mathrm{iff}\ q$ & $p \leftrightarrow q$ & \hspace{-0.5em}\begin{tabular}{l}
  $p$の$q$への必要十分条件\\
  $p$が成り立つことは$q$が成り立つことへの必要十分条件である
\end{tabular} \\
\hline
\end{longtable}
\begin{longtable}[c]{cccccccc}
\hline
$p$ & $q$ & $p \rightarrow q$ & $q \rightarrow p$ & $p \leftrightarrow q$ & $p \Rightarrow q$ & $q \Rightarrow p$ & $p \Leftrightarrow q$ \\
\hline \hline
T & T & T & T & T & T & T & T \\
T & F & F & T & F & & T & \\
F & T & T & F & F & T & & \\
F & F & T & T & T & T & T & T \\
\hline
\end{longtable}
\end{dfn}
\begin{dfn}
ここで、論理式のかっこの省略するときの規則を次に示そう。なお、これらの規則は論理演算子が否定$\neg$、選言$\vee$、連言$\land$、含意$\rightarrow$、同値$\leftrightarrow$のみであるときに適用されることができる。このことは、冪乗の次に積、商、これらの次に和、差の関係に似て否定、量化記号の次に選言、連言、これらの次に含意、これの次に同値の関係に似ている。
\begin{itemize}
\item
  論理式の一番外側のかっこは省略されることができる。
\item
  論理式$A$を用いて$\neg A$または量化記号$Q$を用いて$QA$の外側にあるかっこが省略されても必要十分条件であるなら、そのかっこは省略されることができる。
\item
  論理式$A$、$B$を用いて$A \vee B$または$A \land B$の外側にあるかっこが省略されても必要十分条件であるなら、そのかっこは省略されることができる。
\item
  論理式$A$、$B$を用いて$A \rightarrow B$の外側にあるかっこが省略されても必要十分条件であるなら、そのかっこは省略されることができる。
\item
  論理式$A$、$B$を用いて$A \leftrightarrow B$の外側にあるかっこが省略されても必要十分条件であるなら、そのかっこは省略されることができる。
\end{itemize}
\end{dfn}
%\hypertarget{hilbertux6d41ux5f62ux5f0fux7684ux4f53ux7cfb}{%
\subsubsection{Hilbert流形式的体系}%\label{hilbertux6d41ux5f62ux5f0fux7684ux4f53ux7cfb}}
\begin{axs}
次のように定められる形式的体系$H_{p}$をHilbert流形式的体系という\footnote{これ以外に自然演繹系やKleene流形式的体系、sequent計算LK、sequent計算LJなどがあるそうです。}。
\begin{itemize}
\item
  後述する論理式以外のその形式的体系$H_{p}$の記号として$\neg$、$\Rightarrow$、$\forall$、$\exists$がある。
\item
  形式的体系$H_{p}$の表現は論理式であるとする。
\item
  公理として$p \leftrightarrow q$が成り立つなら、論理式$p$を論理式$q$に代入してもよく変数$x$の定義域が$X$のその変数$x$、論理式たち$p$、$q$、$r$を用いた次のことが成り立つ。
\end{itemize}
\begin{align}
p &\rightarrow (q \rightarrow p)\tag*{(A1)} \label{(A1)} \\
\left( p \rightarrow (q \rightarrow r) \right) &\rightarrow \left( (p \rightarrow q) \rightarrow (p \rightarrow r) \right)\tag*{(A2)} \label{(A2)} \\
(\neg q \rightarrow \neg p) &\rightarrow \left( (\neg q \rightarrow p) \rightarrow q \right)\tag*{(A3)} \label{(A3)} \\
\forall x \in X[p(x)] &\rightarrow p(a)\tag*{(A4)} \label{(A4)} \\
\forall x \in X[p \rightarrow q(x) ] &\rightarrow \left( p \rightarrow \forall x \in X[q(x)] \right)\tag*{(A5)} \label{(A5)}
\end{align}
\begin{itemize}
\item
  推論規則として変数$x$の定義域が$X$のその変数$x$、論理式たち$p$、$q$、$r$を用いた次のことが成り立つ\footnote{\refeq{(MP)}はmodus ponensの略です。}。なお、変数$a$はこれの定義域が$X$で公理を除く仮定をなす論理式の中に自由変数として現れていない自由変数である。
\end{itemize}
\begin{align}
p,p \rightarrow q &\vdash q\tag*{(MP)} \label{(MP)} \\
p(a) &\vdash \forall x \in X[p(x)]\tag*{(Gen)} \label{(Gen)}
\end{align}
\end{axs}
\begin{thm}
\label{1.1.1.1}
次のことが成り立つ。
\begin{align*}
\vDash p \rightarrow p
\end{align*}
\end{thm}
\begin{proof}
公理\refeq{(A1)}より$p \rightarrow \left( (p \rightarrow p) \rightarrow p \right)$が、公理\refeq{(A2)}より$\left( p \rightarrow \left( (p \rightarrow p) \rightarrow p \right) \right) \rightarrow \left( \left( p \rightarrow (p \rightarrow p) \right) \rightarrow (p \rightarrow p) \right)$が成り立つので、推論規則\refeq{(MP)}より$\left( p \rightarrow (p \rightarrow p) \right) \rightarrow (p \rightarrow p)$が成り立つ。また、公理\refeq{(A1)}より$p \rightarrow (p \rightarrow p)$が成り立つので、これと先ほどの推論規則\refeq{(MP)}の結論に推論規則\refeq{(MP)}を適用させて$p \rightarrow p$が得られる。
\end{proof}
\begin{thm}[演繹定理]
\label{1.1.1.2}
Hilbert流形式的体系$H_{p}$において、次のことが成り立つ\footnote{$\left( \alpha_{1},\alpha_{2},\cdots,\alpha_{n - 2} \vDash \alpha_{n - 1} \rightarrow \alpha_{n} \right) \vDash \left( \alpha_{1},\alpha_{2},\cdots,\alpha_{n - 1} \vDash \alpha_{n} \right)$が成り立つ方の定理を逆演繹定理とかいったりする書籍とかもあります。}。
\begin{align*}
\left( \alpha_{1},\alpha_{2},\cdots,\alpha_{n - 1} \vDash \alpha_{n} \right) &\vDash \left( \alpha_{1},\alpha_{2},\cdots,\alpha_{n - 2} \vDash \alpha_{n - 1} \rightarrow \alpha_{n} \right) \\
\left( \alpha_{1},\alpha_{2},\cdots,\alpha_{n - 2} \vDash \alpha_{n - 1} \rightarrow \alpha_{n} \right) &\vDash \left( \alpha_{1},\alpha_{2},\cdots,\alpha_{n - 1} \vDash \alpha_{n} \right)
\end{align*}
\end{thm}
\begin{proof}
前者について仮定より論理式$\alpha_{n}$が論理式たち$\alpha_{1}$、$\alpha_{2}$、$\cdots$、$\alpha_{n - 1}$から証明可能である。$n = 2$のとき、証明の定義よりその論理式$\alpha_{2}$が仮定に含まれるか、公理であるか、それらの論理式たち$\alpha_{1}$、$\alpha_{2}$が同じものであるので、論理式$\alpha_{2}$が仮定に含まれるか公理であるとき、公理\refeq{(A1)}より$\alpha_{2} \rightarrow \left( \alpha_{1} \rightarrow \alpha_{2} \right)$が成り立ち、したがって、推論規則\refeq{(MP)}より$\alpha_{1} \rightarrow \alpha_{2}$が成り立つ。それらの論理式たち$\alpha_{1}$、$\alpha_{2}$が同じものであるとき、定理\refeq{1.1.1.1}より明らかに$\alpha_{1} \rightarrow \alpha_{2}$が成り立つ。$n = k$のときに成り立つものとして、$n = k + 1$のとき、証明の定義よりその論理式$\alpha_{k + 1}$が仮定に含まれるか、公理であるときでは$n = 1$のときと同様にして示される。一方で、仮定に含まれるもののうち1つ$\alpha_{i}$とその論理式$\alpha_{k + 1}$が同じものであるとき、定理\refeq{1.1.1.1}より$\alpha_{i} \rightarrow \alpha_{k + 1}$が成り立つので、公理\refeq{(A1)}より$\left( \alpha_{i} \rightarrow \alpha_{k + 1} \right) \rightarrow \left( \alpha_{k} \rightarrow \left( \alpha_{i} \rightarrow \alpha_{k + 1} \right) \right)$が成り立ち推論規則\refeq{(MP)}より$\alpha_{k} \rightarrow \left( \alpha_{i} \rightarrow \alpha_{k + 1} \right)$が成り立つ。これが、公理\refeq{(A2)}より$\left( \alpha_{k} \rightarrow \left( \alpha_{i} \rightarrow \alpha_{k + 1} \right) \right) \rightarrow \left( \left( \alpha_{k} \rightarrow \alpha_{i} \right) \rightarrow \left( \alpha_{k} \rightarrow \alpha_{k + 1} \right) \right)$が成り立つので、これに推論規則\refeq{(MP)}を適用されると、$\left( \alpha_{k} \rightarrow \alpha_{i} \right) \rightarrow \left( \alpha_{k} \rightarrow \alpha_{k + 1} \right)$が成り立つ。これが、公理\refeq{(A1)}より$\alpha_{i} \rightarrow \left( \alpha_{k} \rightarrow \alpha_{i} \right)$が成り立ち推論規則\refeq{(MP)}より$\alpha_{k} \rightarrow \alpha_{i}$が成り立つので、これに推論規則\refeq{(MP)}を適用されると、$\alpha_{k} \rightarrow \alpha_{k + 1}$が成り立つ。このことを$k = 1$から繰り返せばよい。\par
後者について仮定より論理式$\alpha_{n - 1} \rightarrow \alpha_{n}$が論理式たち$\alpha_{1}$、$\alpha_{2}$、$\cdots$、$\alpha_{n - 2}$から証明可能である。ここで、仮定としてその論理式$\alpha_{n - 1}$を加えると、推論規則\refeq{(MP)}より結論$\alpha_{n}$が得られる。以上より、その論理式$\alpha_{n}$はそれらの論理式たち$\alpha_{1}$、$\alpha_{2}$、$\cdots$、$\alpha_{n - 1}$から証明可能である。
\end{proof}
\begin{thm}
\label{1.1.1.3}
次のことが成り立つ。
\begin{align*}
p \rightarrow q,q \rightarrow r &\vDash p \rightarrow r \\
p \rightarrow (q \rightarrow r),q &\vDash p \rightarrow r
\end{align*}
\end{thm}
\begin{proof}
仮定より$p$と$p \rightarrow q$が成り立つので、推論規則\refeq{(MP)}より$q$が得られる。これと仮定より$q \rightarrow r$が成り立つので、推論規則\refeq{(MP)}より$r$が得られる。以上より、$p \rightarrow q,q \rightarrow r,p \vDash r$が成り立つ。演繹定理よりよって、結論$p \rightarrow r$は仮定たち$p \rightarrow q$、$q \rightarrow r$から証明可能である。\par
仮定より$p$と$p \rightarrow (q \rightarrow r)$が成り立つので、推論規則\refeq{(MP)}より$q \rightarrow r$が得られる。これと仮定より$q$が成り立つので、推論規則\refeq{(MP)}より$r$が得られる。以上より、$p \rightarrow (q \rightarrow r),q,p \vDash r$が成り立つ。演繹定理よりよって、結論$p \rightarrow r$は仮定$p \rightarrow (q \rightarrow r)$、$q$から証明可能である。
\end{proof}
\begin{thm}
\label{1.1.1.4}
次のことが成り立つ。
\begin{align*}
&\vDash \neg\neg p \rightarrow p \\
&\vDash p \rightarrow \neg\neg p \\
&\vDash \neg p \rightarrow (p \rightarrow q) \\
&\vDash (\neg q \rightarrow \neg p) \rightarrow (p \rightarrow q) \\
&\vDash (p \rightarrow q) \rightarrow (\neg q \rightarrow \neg p) \\
&\vDash p \rightarrow \left( \neg q \rightarrow \neg(p \rightarrow q) \right) \\
&\vDash (p \rightarrow q) \rightarrow \left( (\neg p \rightarrow q) \rightarrow q \right) 
\end{align*}
\end{thm}
\begin{proof}
公理\refeq{(A3)}より$(\neg p \rightarrow \neg\neg p) \rightarrow \left( (\neg p \rightarrow \neg p) \rightarrow p \right)$が成り立ち、定理\refeq{1.1.1.1}より$\neg p \rightarrow \neg p$が成り立つので、定理\refeq{1.1.1.3}より$(\neg p \rightarrow \neg\neg p) \rightarrow p$が得られる。ここで、公理$(A1)$より$\neg\neg p \rightarrow (\neg p \rightarrow \neg\neg p)$が成り立ち定理\refeq{1.1.1.3}より$\neg\neg p \rightarrow p$が得られる。\par
上記の議論により$\neg\neg\neg p \rightarrow \neg p$が成り立ち、公理\refeq{(A3)}より$(\neg\neg\neg p \rightarrow \neg p) \rightarrow \left( (\neg\neg\neg p \rightarrow p) \rightarrow \neg\neg p \right)$が成り立つ。以上、推論規則\refeq{(MP)}より$(\neg\neg\neg p \rightarrow \neg p) \rightarrow \neg\neg p$が得られる。また、公理\refeq{(A1)}より$p \rightarrow (\neg\neg\neg p \rightarrow p)$が成り立つことから、定理\refeq{1.1.1.3}より$p \rightarrow \neg\neg p$が得られる。\par
仮定として論理式たち$\neg p$、$p$が与えられたとき、公理\refeq{(A1)}より$\neg p \rightarrow (\neg q \rightarrow \neg p)$、$p \rightarrow (\neg q \rightarrow p)$が成り立ち、推論規則\refeq{(MP)}よりそれぞれ$\neg q \rightarrow \neg p$、$\neg q \rightarrow p$が得られ、$\neg q \rightarrow \neg p$と公理\refeq{(A3)}の$(\neg q \rightarrow \neg p) \rightarrow \left( (\neg q \rightarrow p) \rightarrow q \right)$が推論規則\refeq{(MP)}に適用されることで$(\neg q \rightarrow p) \rightarrow q$が得られ、これと$\neg q \rightarrow p$が推論規則\refeq{(MP)}に適用されることで$q$が得られる。演繹定理より$\neg p \rightarrow (p \rightarrow q)$が成り立つ。\par
仮定より$\neg q \rightarrow \neg p$が成り立ち、これと公理\refeq{(A3)}の$(\neg q \rightarrow \neg p) \rightarrow \left( (\neg q \rightarrow p) \rightarrow q \right)$が推論規則\refeq{(MP)}に適用されることで、$(\neg q \rightarrow p) \rightarrow q$が得られ、公理\refeq{(A1)}の$p \rightarrow (\neg q \rightarrow p)$が成り立つことにより、したがって、定理\refeq{1.1.1.3}より$p \rightarrow q$が得られる。よって、演繹定理より$(\neg q \rightarrow \neg p) \rightarrow (p \rightarrow q)$が成り立つ。\par
仮定より$p \rightarrow q$が成り立ち、上記の議論により$\neg p \rightarrow p$が成り立つので、定理\refeq{1.1.1.3}より$\neg\neg p \rightarrow q$が成り立つ。ここで、上記の議論により$q \rightarrow \neg\neg q$が成り立つので、定理\refeq{1.1.1.3}より$\neg\neg p \rightarrow \neg\neg q$が成り立つ。ここで、上記の議論により$(\neg\neg p \rightarrow \neg\neg q) \rightarrow (\neg q \rightarrow \neg p)$が成り立つので、推論規則\refeq{(MP)}より$\neg q \rightarrow \neg p$が成り立つ。よって、演繹定理より$(p \rightarrow q) \rightarrow (\neg q \rightarrow \neg p)$が成り立つ。\par
仮定として論理式たち$p$、$p \rightarrow q$が与えられたとき、推論規則\refeq{(MP)}より$q$が得られ演繹定理より$p \vDash (p \rightarrow q) \rightarrow q$が得られる。ここで、上記の議論と演繹定理より$(p \rightarrow q) \rightarrow q \vDash \neg q \rightarrow \neg(p \rightarrow q)$が成り立つので、演繹定理より$p \rightarrow \left( \neg q \rightarrow \neg(p \rightarrow q) \right)$が成り立つ。\par
仮定として論理式たち$p \rightarrow q$、$\neg p \rightarrow q$が与えられたとき、上記の議論により$(p \rightarrow q) \rightarrow (\neg q \rightarrow \neg p)$が成り立ち、これと仮定の$p \rightarrow q$に推論規則\refeq{(MP)}が適用されることで、$\neg q \rightarrow \neg p$が成り立つ。仮定の$\neg p \rightarrow q$において上記の議論より$(\neg p \rightarrow q) \rightarrow (\neg q \rightarrow \neg\neg p)$が成り立つので、推論規則\refeq{(MP)}より$\neg q \rightarrow \neg\neg p$が成り立つ。ここで、公理\refeq{(A3)}より$(\neg q \rightarrow \neg\neg p) \rightarrow \left( (\neg q \rightarrow \neg p) \rightarrow q \right)$が成り立つので、推論規則\refeq{(MP)}により$(\neg q \rightarrow \neg p) \rightarrow q$が成り立ち、これと先ほどの$\neg q \rightarrow \neg p$が推論規則\refeq{(MP)}に適用されることで、$q$が得られる。以上、演繹定理より$(p \rightarrow q) \rightarrow \left( (\neg p \rightarrow q) \rightarrow q \right)$が成り立つ。
\end{proof}
\begin{thm}
\label{1.1.1.5}
次のことが成り立つ。
\begin{align*}
p,q &\vDash p\\
p,q &\vDash \neg(p \rightarrow \neg q)\\
p \rightarrow q,q \rightarrow p &\vDash p \leftrightarrow q\\
\neg(p \rightarrow \neg q) &\vDash p\\
\neg(p \rightarrow \neg q) &\vDash q\\
\bot &\vDash p\\
p &\vDash \neg p \rightarrow q\\
q &\vDash \neg p \rightarrow q
\end{align*}
\end{thm}
\begin{proof}
定理\refeq{1.1.1.4}より$p \rightarrow \left( \neg\neg q \rightarrow \neg(p \rightarrow \neg q) \right)$が成り立つので、これが演繹定理に2回適用されることで、$\neg\neg q \vDash p \rightarrow \neg(p \rightarrow \neg q)$が成り立つ。ここで、定理\refeq{1.1.1.4}より$q \rightarrow \neg\neg q$が成り立つので、定理\refeq{1.1.1.3}より$q \vDash p \rightarrow \neg(p \rightarrow \neg q)$が成り立つ。よって、演繹定理より$p,q \vDash \neg(p \rightarrow \neg q)$が成り立つ。\par
上記の議論により$p \rightarrow q$、$q \rightarrow p$が成り立つので、$\neg\left( (p \rightarrow q) \rightarrow \neg(q \rightarrow p) \right)$、即ち、$p \leftrightarrow q$が得られる。\par
定理\refeq{1.1.1.4}より$\neg p \rightarrow (p \rightarrow \neg q)$と$\left( \neg p \rightarrow (p \rightarrow \neg q) \right) \rightarrow \left( \neg(p \rightarrow \neg q) \rightarrow \neg\neg p \right)$が成り立つので、推論規則\refeq{(MP)}より$\neg(p \rightarrow \neg q) \rightarrow \neg\neg p$が成り立ち、定理\refeq{1.1.1.4}より$\neg\neg p \rightarrow p$が成り立つので、定理\refeq{1.1.1.3}より$\neg(p \rightarrow \neg q) \rightarrow p$が成り立ち、演繹定理よりよって、$\neg(p \rightarrow \neg q) \vDash p$が成り立つ。\par
公理\refeq{(A1)}より$\neg q \rightarrow (p \rightarrow \neg q)$と定理\refeq{1.1.1.4}より$\left( \neg q \rightarrow (p \rightarrow \neg q) \right) \rightarrow \left( \neg(p \rightarrow \neg q) \rightarrow \neg\neg q \right)$が成り立つので、推論規則\refeq{(MP)}より$\neg(p \rightarrow \neg q) \rightarrow \neg\neg q$が成り立ち、定理\refeq{1.1.1.4}より$\neg\neg q \rightarrow q$が成り立つので、定理\refeq{1.1.1.3}より$\neg(p \rightarrow \neg q) \rightarrow q$が成り立ち、演繹定理よりよって、$\neg(p \rightarrow \neg q) \vDash q$が成り立つ。\par
上記の議論により$q \leftrightarrow \neg\neg q$が成り立ち、公理より$\bot$は$\neg(q \rightarrow \neg\neg q)$と同じものとみなせるので、上記の議論により$q$、$\neg q$が得られる。$\neg q$と公理\refeq{(A1)}の$\neg q \rightarrow (\neg p \rightarrow \neg q)$が推論規則\refeq{(MP)}に適用されることで、$\neg p \rightarrow \neg q$が成り立ち、定理\refeq{1.1.1.4}より$(\neg p \rightarrow \neg q) \rightarrow (q \rightarrow p)$が成り立つので、推論規則\refeq{(MP)}より$q \rightarrow p$が成り立つ。これと$q$が推論規則\refeq{(MP)}に適用されることで、$p$が成り立つ。\par
仮定として論理式たち$p$、$\neg p$が与えられたとき、$\neg(p \rightarrow \neg\neg p)$が成り立ち、上記の議論と同様にして、これは$\bot$と同じものとみなせる。したがって、$q$が得られる。以上、$p,\neg p \vDash \neg(p \rightarrow \neg\neg p)$と$\neg(p \rightarrow \neg\neg p) \rightarrow q$が得られているので、推論規則\refeq{(MP)}より$p,\neg p \vDash q$が得られる。演繹定理よりよって、$p \vDash \neg p \rightarrow q$が成り立つ。\par
公理\refeq{(A1)}より$q \rightarrow (\neg p \rightarrow q)$が成り立つので、演繹定理より$q \vDash \neg p \rightarrow q$が成り立つ。
\end{proof}
\begin{thm}
\label{1.1.1.6}
次のことが成り立つ。
\begin{align*}
p(a) &\vDash \exists x \in X\left[ p(x) \right] \\
\forall x \in X\left[ p(x) \right] &\vDash \exists x \in X\left[ p(x) \right]
\end{align*}
\end{thm}
\begin{proof}
公理\refeq{(A4)}より$\forall x \in X\left[ \neg p(x) \right] \rightarrow \neg p(a)$が成り立つ。ここで、定理\refeq{1.1.1.4}より$p(a) \rightarrow \neg\forall x \in X\left[ \neg p(x) \right]$が成り立つので、演繹定理よりしたがって、$p(a) \vDash \exists x \in X\left[ p(x) \right]$が得られる。\par
公理\refeq{(A4)}より$\forall x \in X\left[ p(x) \right] \rightarrow p(a)$が成り立つかつ、上記の議論により$p(a) \rightarrow \exists x \in X\left[ p(x) \right]$が成り立つので、定理\refeq{1.1.1.3}より$\forall x \in X\left[ p(x) \right] \rightarrow \exists x \in X\left[ p(x) \right]$が得られる。あとは演繹定理よりしたがって、$\forall x \in X\left[ p(x) \right] \vDash \exists x \in X\left[ p(x) \right]$が成り立つ。
\end{proof}
%\hypertarget{ux8ad6ux7406ux5f0fux306eux8a08ux7b97}{%
\subsubsection{論理式の計算}%\label{ux8ad6ux7406ux5f0fux306eux8a08ux7b97}}
\begin{thm}
\label{1.1.1.7}
次の論理式たちは全て恒真である。なお、$p$、$q$、$r$いずれも命題であるとする。この定理は論理式の計算がされるのに便利である。
\begin{longtable}[c]{|cc|c|}
\hline
\multicolumn{2}{|c|}{名称} & 論理式 \\
\hline \hline
\multicolumn{2}{|c|}{交換律} & \hspace{-0.5em}\begin{tabular}{c}
  $p \vee q \Leftrightarrow q \vee p $ \\
  $p \land q \Leftrightarrow q \land p$ 
\end{tabular}\\
\hline
\multicolumn{2}{|c|}{結合律} & \hspace{-0.5em}\begin{tabular}{c}
  $(p \vee q) \vee r \Leftrightarrow p \vee (q \vee r) $ \\
  $(p \land q) \land r \Leftrightarrow p \land (q \land r) $ \\
  $\left( (p \leftrightarrow q) \leftrightarrow r \right) \Leftrightarrow \left( p \leftrightarrow (q \leftrightarrow r) \right)$ 
\end{tabular}\\
\hline
\multicolumn{2}{|c|}{分配律} & \hspace{-0.5em}\begin{tabular}{c}
  $p \vee (q \land r) \Leftrightarrow (p \vee q) \land (p \vee r) $ \\
  $p \land (q \vee r) \Leftrightarrow (p \land q) \vee (p \land r) $ 
\end{tabular}\\
\hline
\multicolumn{2}{|c|}{二重否定律} & $\neg\neg p \Leftrightarrow p$ \\
\hline
\multicolumn{2}{|c|}{de Morgan律} & \hspace{-0.5em}\begin{tabular}{c}
  $\neg(p \vee q) \Leftrightarrow \neg p \land \neg q $ \\
  $\neg(p \land q) \Leftrightarrow \neg p \vee \neg q$ 
\end{tabular}\\
\hline
\multicolumn{2}{|c|}{対偶律} & $p \rightarrow q \Leftrightarrow \neg q \rightarrow \neg p$ \\
\hline
\multicolumn{2}{|c|}{省略律} & \hspace{-0.5em}\begin{tabular}{c}
  $p \vee p \Leftrightarrow p$ \\
  $p \land p \Leftrightarrow p$ 
\end{tabular}\\
\hline
\multicolumn{2}{|c|}{選言と連言の分解} & \hspace{-0.5em}\begin{tabular}{c}
  $p \vee (\neg p \land q) \Leftrightarrow p \vee q$ \\
  $p \land (\neg p \vee q) \Leftrightarrow p \land q$ \\
  $p \vee (p \land \neg p) \Leftrightarrow p$ \\
  $p \land (p \vee \neg p) \Leftrightarrow p$ 
\end{tabular}\\
\hline
\multicolumn{2}{|c|}{移出律と移入律} & $p \land q \rightarrow r \Leftrightarrow p \rightarrow (q \rightarrow r)$ \\
\hline
\multicolumn{2}{|c|}{含意の分解} & \hspace{-0.5em}\begin{tabular}{c}
  $p \rightarrow q \vee r \Leftrightarrow (p \rightarrow q) \vee (p \rightarrow r) $ \\
  $p \rightarrow q \land r \Leftrightarrow (p \rightarrow q) \land (p \rightarrow r) $ \\
  $p \vee q \rightarrow r \Leftrightarrow (p \rightarrow r) \land (q \rightarrow r)$ 
\end{tabular}\\
\hline
\multicolumn{2}{|c|}{反射律} & $p \Leftrightarrow p$ \\
\hline
\multicolumn{2}{|c|}{対称律} & $(p \leftrightarrow q) \Leftrightarrow (q \leftrightarrow p)$ \\
\hline
\multicolumn{2}{|c|}{推移律 三段論法} & $(p \rightarrow q) \land (q \rightarrow r) \Rightarrow p \rightarrow r$ \\
\hline
\multicolumn{2}{|c|}{恒等律} & \hspace{-0.5em}\begin{tabular}{c}
  $\top \Leftrightarrow \neg\bot $ \\
  $\bot \Leftrightarrow \neg\top$ 
\end{tabular}\\
\hline
\multicolumn{2}{|c|}{恒真式と恒偽式による吸収律} & \hspace{-0.5em}\begin{tabular}{c}
  $\top \Leftrightarrow p \vee \top $ \\
  $\bot \Leftrightarrow p \land \bot $ \\
  $p \Leftrightarrow p \vee \bot $ \\
  $p \Leftrightarrow p \land \top$ 
\end{tabular}\\
\hline
\multicolumn{1}{|c}{\multirow{2}{*}{排中律}} & \multicolumn{1}{|c|}{非矛盾律} & $\top \Leftrightarrow p \vee \neg p$ \\ \cline{2-3}
& \multicolumn{1}{|c|}{矛盾律} & $\bot \Leftrightarrow p \land \neg p$ \\
\hline
\multicolumn{2}{|c|}{選言の導入} & $p \Rightarrow p \vee q$ \\
\hline
\multicolumn{2}{|c|}{連言の除去} & $p \land q \Rightarrow p$ \\
\hline
\end{longtable}
\end{thm}
\begin{proof}
二重否定律、対偶律、反射律、推移律、選言の導入、連言の除去はすでに示されている。排中律は定義そのものである。\par
まず、交換律を示そう。論理式$p \vee q$は$\neg p \rightarrow q$と同じものであり、二重否定律より$p \leftrightarrow \neg\neg p$が成り立つので、公理よりこれは$\neg q \rightarrow p$と同じものとみなされる。したがって、$q \vee p$が得られ、演繹定理より$p \vee q \rightarrow q \vee p$が得られる。同様にして、$q \vee p \rightarrow p \vee q$が得られるので、よって、$p \vee q \leftrightarrow q \vee p$が得られる。また、論理式$p \land q$は$\neg(p \rightarrow \neg q)$と同じものであり、対偶律と二重否定律より$(p \rightarrow \neg q) \leftrightarrow (\neg\neg q \rightarrow \neg p)$と$q \leftrightarrow \neg\neg q$が成り立つので、公理よりこれは$\neg(q \rightarrow \neg p)$と同じものとみなされる。したがって、$q \land p$が得られ、演繹定理より$p \land q \rightarrow q \land p$が得られる。同様にして、$q \land p \rightarrow p \land q$が得られるので、よって、$p \land q \leftrightarrow q \land p$が得られる。\par
次に、de Morgan律を示そう。論理式$\neg(p \vee q)$は$\neg(\neg p \rightarrow q)$と同じものであり、対偶律より$(\neg p \rightarrow q) \leftrightarrow (\neg q \rightarrow \neg\neg p)$が成り立つので、公理よりこれは$\neg(\neg q \rightarrow \neg\neg p)$と同じものとみなされる。したがって、$\neg p \land \neg q$が得られ、演繹定理より$\neg(p \vee q) \rightarrow \neg p \land \neg q$が得られる。同様にして、$\neg p \land \neg q \rightarrow \neg(p \vee q)$が得られるので、よって、$\neg(p \vee q) \leftrightarrow \neg p \land \neg q$が得られる。また、論理式$\neg(p \land q)$は$\neg\neg(p \rightarrow \neg q)$と同じものであり、二重否定律より$\neg\neg(p \rightarrow \neg q) \leftrightarrow (p \rightarrow \neg q)$と$p \leftrightarrow \neg\neg p$が成り立つので、公理よりこれは$\neg\neg p \rightarrow \neg q$と同じものとみなされる。したがって、$\neg p \vee \neg q$が得られ、演繹定理より$\neg(p \land q) \rightarrow \neg p \vee \neg q$が得られる。同様にして、$\neg p \vee \neg q \rightarrow \neg(p \land q)$が得られるので、よって、$\neg(p \land q) \leftrightarrow \neg p \vee \neg q$が得られる。\par
次に、結合律を示そう。論理式$(p \land q) \land r$が与えられたとき、定理\refeq{1.1.1.5}より$p$、$q$、$r$がいづれも成り立つので、証明の定義に注意すれば、再び定理\refeq{1.1.1.5}より$p \land (q \land r)$が得られる。演繹定理より$(p \land q) \land r \rightarrow p \land (q \land r)$が成り立つ。同様にして、$p \land (q \land r) \rightarrow (p \land q) \land r$が得られるので、よって、$(p \land q) \land r \leftrightarrow p \land (q \land r)$が得られる。また、上記の議論により、$\neg p \land (\neg q \land \neg r) \rightarrow (\neg p \land \neg q) \land \neg r$が成り立つのであった。対偶律とde Morgan律よりこれは$(p \land q) \land r \rightarrow p \land (q \land r)$と同じものである。演繹定理より$(p \vee q) \vee r \rightarrow p \vee (q \vee r)$が成り立つ。同様にして、$p \vee (q \vee r) \rightarrow (p \vee q) \vee r$が得られるので、よって、$(p \vee q) \vee r \leftrightarrow p \vee (q \vee r)$が得られる。また、論理積の結合律より明らかに$\left( (p \leftrightarrow q) \leftrightarrow r \right) \leftrightarrow \left( p \leftrightarrow (q \leftrightarrow r) \right)$が成り立つ。\par
次に、省略律を示そう。$p \land p \leftrightarrow p$が成り立つことは定理\refeq{1.1.1.5}と演繹定理より明らかにである。また、$\neg p \land \neg p \leftrightarrow \neg p$が成り立つことと対偶律より明らかに$p \vee p \leftrightarrow p$が成り立つ。\par
次に、移出律と移入律を示そう。論理式$p \land q \rightarrow r$は選言を用いて書き換えられると、$(\neg p \vee \neg q) \vee r$と与えられる。結合律によりこれは$\neg p \vee (\neg q \vee r)$と与えられるので、選言の定義より$p \rightarrow (q \rightarrow r)$が成り立つ。演繹定理より$(p \land q \rightarrow r) \rightarrow \left( p \rightarrow (q \rightarrow r) \right)$が成り立ち、同様にして、$\left( p \rightarrow (q \rightarrow r) \right) \rightarrow (p \land q \rightarrow r)$が得られるので、よって、$(p \land q \rightarrow r) \leftrightarrow \left( p \rightarrow (q \rightarrow r) \right)$が得られる。\par
次に、含意の分解を示そう。論理式$p \rightarrow q \vee r$は選言を用いて書き換えられると、$\neg p \vee (q \vee r)$が与えられる。ここで、結合律と吸収律より$(\neg p \vee q) \vee (\neg p \vee r)$が与えられるので、選言の定義より$(p \rightarrow q) \vee (p \rightarrow r)$が成り立つ。演繹定理より$(p \rightarrow q \vee r) \rightarrow (p \rightarrow q) \vee (p \rightarrow r)$が成り立ち、同様にして、$(p \rightarrow q) \vee (p \rightarrow r) \rightarrow (p \rightarrow q \vee r)$が得られるので、よって、$(p \rightarrow q \vee r) \leftrightarrow (p \rightarrow q) \vee (p \rightarrow r)$が得られる。また、仮定として$(p \rightarrow q) \land (p \rightarrow r)$と$p$が与えられたとする。定理\refeq{1.1.1.5}より$p \rightarrow q$、$p$、$p \rightarrow r$、$p$が与えられ、推論規則$(MP)$より$q$と$r$が成り立ち、定理\refeq{1.1.1.5}よりしたがって、$q \land r$が成り立つ。演繹定理より$(p \rightarrow q) \land (p \rightarrow r) \rightarrow (p \rightarrow q \land r)$が成り立つ。逆に、論理式$p \rightarrow q \land r$が与えられたとする。ここで、論理式$\neg(p \rightarrow q)$が与えられたとき、二重否定律と定理\refeq{1.1.1.5}より$p$と$\neg q$が得られる。ここで、$\neg q$について、定理\refeq{1.1.1.5}と二重否定律より$q \rightarrow \neg r$が成り立つので、再び定理\refeq{1.1.1.5}より$\neg\left( p \rightarrow \neg(q \rightarrow \neg r) \right)$が得られる。演繹定理より$\neg(p \rightarrow q) \rightarrow \neg\left( p \rightarrow \neg(q \rightarrow \neg r) \right)$が成り立つので、二重否定律と対偶律、de Morgan律より$(p \rightarrow q \land r) \rightarrow (p \rightarrow q)$が成り立つ。同様にして、$(p \rightarrow q \land r) \rightarrow (p \rightarrow r)$が成り立つので、定理\refeq{1.1.1.5}より$(p \rightarrow q) \land (p \rightarrow r)$が得られるので、演繹定理より$(p \rightarrow q \land r) \rightarrow (p \rightarrow q) \land (p \rightarrow r)$が成り立つ。以上より、$(p \rightarrow q \land r) \leftrightarrow (p \rightarrow q) \land (p \rightarrow r)$が得られた。また、論理式$p \vee q \rightarrow r$が与えられたとき、対偶律とde Morgan律より$\neg r \rightarrow \neg p \vee \neg q$が成り立ち、上記の議論により$(\neg r \rightarrow \neg p) \land (\neg r \rightarrow \neg q)$が成り立ち、再び対偶律より$(p \rightarrow r) \land (q \rightarrow r)$が成り立つ。演繹定理より$(p \vee q \rightarrow r) \rightarrow (p \rightarrow r) \land (q \rightarrow r)$が得られる。同様にして、$(p \rightarrow r) \land (q \rightarrow r) \rightarrow (p \vee q \rightarrow r)$が得られるので、よって、$p \vee q \rightarrow r \Leftrightarrow (p \rightarrow r) \land (q \rightarrow r)$が成り立つ。\par
次に、分配律を示そう。論理式$p \vee (q \land r)$は$\neg p \rightarrow q \land r$のことであるから、含意の分解より$(\neg p \rightarrow q) \land (\neg p \rightarrow r)$が成り立つ。これはまさしく$(p \vee q) \land (p \vee r)$のことである。演繹定理より$p \vee (q \land r) \rightarrow (p \vee q) \land (p \vee r)$が得られ、同様にして、$(p \vee q) \land (p \vee r) \rightarrow p \vee (q \land r)$も示される。論理式$p \land (q \vee r)$はde Morgan律より$\neg\left( \neg p \vee (\neg q \land \neg r) \right)$と同じものであるから、上記の議論と対偶律より$\neg\left( \neg p \vee (\neg q \land \neg r) \right) \leftrightarrow \neg\left( (\neg p \vee \neg q) \land (\neg p \vee \neg r) \right)$が得られ、再びde Morgan律より$(p \land q) \vee (p \land r)$が得られる。以上、演繹定理より$p \land (q \vee r) \Leftrightarrow (p \land q) \vee (p \land r)$が得られる。\par
対称律は交換律より明らかである。\par
恒等律について、恒真式の定義より$\top$は$p \vee \neg p$と同じものとみなされる。二重否定律とde Morgan律、交換律よりこれは$\neg(p \land \neg p)$と同じものであるから、恒偽式の定義と演繹定理より$\top \rightarrow \neg\bot$が得られる。同様にして、$\neg\bot \rightarrow \top$が得られるので、$\top \leftrightarrow \neg\bot$が成り立つ。対偶律より$\bot \leftrightarrow \neg\top$も示される。\par
次に、恒真式と恒偽式による吸収律について、恒真式の定義と吸収律より明らかに$\top \leftrightarrow p \vee \top$が成り立つ。同様に$\bot \leftrightarrow p \land \bot$も得られる。一方で、論理式$p$が与えられたとき、定理\refeq{1.1.1.5}より$p \vee \bot$が得られる。逆に、論理式$p \land \top$が与えられたとき、定理\refeq{1.1.1.5}より$p$が得られる。逆に、論理式$p$が与えられたとき、定理\refeq{1.1.1.5}より$p \vee \bot$が得られ、恒偽式の定義と分配律、吸収律より$p \land (p \vee \neg p)$が得られ、恒真式の定義より$p \land \top$が得られる。以上、$p \leftrightarrow p \land \top$が成り立つ。対偶律より$p \leftrightarrow p \vee \bot$も示される。
\end{proof}
\begin{thm}
\label{1.1.1.8}
次の論理式たちは全て恒真である。なお、$p(x)$、$q(x)$いずれも定義域$X$の変数$x$に依存する命題関数、$p$、$q$いずれもその変数$x$に依存しない命題、$p(x,y)$いずれも定義域それぞれ$X$、$Y$の変数たち$x$、$y$に依存する命題関数、$\left\{ x',x''\right\}$を$x'$、$x''$のみとりうる定義域、$X \setminus \left\{ x' \right\}$をその定義域$X$から$x'$を除いた範囲であるとする。
\begin{longtable}[c]{|c|c|}
\hline
名称 & 論理式 \\
\hline \hline 
特殊化 & \hspace{-0.5em}\begin{tabular}{c}
  $\exists x \in \left\{ x',x'' \right\} \left[ p(x)\right] \Leftrightarrow p\left( x' \right) \vee p\left( x'' \right) $ \\
  $\forall x \in \left\{ x',x'' \right\}\left[ p(x) \right] \Leftrightarrow p\left( x' \right) \land p\left( x'' \right) $ 
\end{tabular}\\
\hline
分配律 & \hspace{-0.5em}\begin{tabular}{c}
  $\exists x \in X\left[ p(x) \land q \right] \Leftrightarrow \exists x \in X\left[ p(x) \right] \land q $ \\
  $\forall x \in X\left[ p(x) \vee q \right] \Leftrightarrow \forall x \in X\left[ p(x) \right] \vee q$ 
\end{tabular}\\
\hline
de Morgan律 & \hspace{-0.5em}\begin{tabular}{c}
  $\neg\left( \exists x \in X\left[ p(x) \right] \right) \Leftrightarrow \forall x \in X\left[ \neg p(x) \right] $ \\
  $\neg\left( \forall x \in X\left[ p(x) \right] \right) \Leftrightarrow \exists x \in X\left[ \neg p(x) \right]$ 
\end{tabular}\\
\hline
省略律 & \hspace{-0.5em}\begin{tabular}{c}
  $\exists x \in X\lbrack p\rbrack \Leftrightarrow p$ \\
  $\forall x \in X\lbrack p\rbrack \Leftrightarrow p$ 
\end{tabular}\\
\hline
移出律と移入律 & $\forall x \in X\left[ p(x) \right] \rightarrow q \Leftrightarrow \forall x \in X \setminus \left\{ x' \right\}\left[ p(x) \right] \rightarrow \left( p\left( x' \right) \rightarrow q \right)$ \\
\hline
量化記号と選言、連言の関係 & \hspace{-0.5em}\begin{tabular}{c}
  $\exists x \in X\left[ p(x) \vee q(x) \right] \Leftrightarrow \exists x \in X\left[ p(x) \right] \vee \exists x \in X\left[ q(x) \right] $ \\
  $\exists x \in X\left[ p(x) \land q(x) \right] \Rightarrow \exists x \in X\left[ p(x) \right] \land \exists x \in X\left[ q(x) \right] $ \\
  $\forall x \in X\left[ p(x) \vee q(x) \right] \Leftarrow \forall x \in X\left[ p(x) \right] \vee \forall x \in X\left[ q(x) \right] $ \\
  $\forall x \in X\left[ p(x) \land q(x) \right] \Leftrightarrow \forall x \in X\left[ p(x) \right] \land \forall x \in X\left[ q(x) \right] $ 
\end{tabular}\\
\hline
量化記号と含意の関係 & \hspace{-0.5em}\begin{tabular}{c}
  $\exists x \in X\left[ p \rightarrow q(x) \right] \Leftrightarrow p \rightarrow \exists x \in X\left[ q(x) \right] $ \\
  $\exists x \in X\left[ p(x) \rightarrow q \right] \Leftrightarrow \forall x \in X\left[ p(x) \right] \rightarrow q $ \\
  $\forall x \in X\left[ p \rightarrow q(x) \right] \Leftrightarrow p \rightarrow \forall x \in X\left[ q(x) \right] $ \\
  $\forall x \in X\left[ p(x) \rightarrow q \right] \Leftrightarrow \exists x \in X\left[ p(x) \right] \rightarrow q$ 
\end{tabular}\\
\hline
量化記号の交換 & \hspace{-0.5em}\begin{tabular}{c}
  $\forall x \in X\forall y \in Y\left[ p(x,y) \right] \Leftrightarrow \forall y \in Y\forall x \in X\left[ p(x,y) \right] $ \\
  $\exists x \in X\exists y \in Y\left[ p(x,y) \right] \Leftrightarrow \exists y \in Y\exists x \in X\left[ p(x,y) \right] $ \\
  $\exists x \in X\forall y \in Y\left[ p(x,y) \right] \Rightarrow \forall y \in Y\exists x \in X\left[ p(x,y) \right] $ 
\end{tabular}\\
\hline
\end{longtable}
\end{thm}
この定理は論理式の計算がされるのに便利である。ただし、$\exists x \in X\forall y \in Y\left[ p(x,y) \right] \Leftarrow \forall y \in Y\exists x \in X\left[ p(x,y) \right]$は成り立つとは限らないことに注意されたい。
\begin{proof}
特殊化を示そう。$\forall x \in \left\{ x',x'' \right\}\left[ p(x) \right]$が成り立つなら、公理\refeq{(A4)}より明らかに$p\left( x' \right)$、$p\left( x'' \right)$が成り立ち、定理\refeq{1.1.1.5}より$p\left( x' \right) \vee p\left( x'' \right)$が成り立つので、$\forall x \in \left\{ x',x'' \right\}\left[ p(x) \right] \rightarrow p\left( x' \right) \land p\left( x'' \right)$が得られる。また、演繹定理と推論規則\refeq{(Gen)}より$p\left( x' \right) \land p\left( x'' \right) \rightarrow \forall x \in \left\{ x',x'' \right\}\left[ p(x) \right]$が成り立つので、$\forall x \in \left\{ x',x'' \right\}\left[ p(x) \right] \Leftrightarrow p\left( x' \right) \land p\left( x'' \right)$が得られる。対偶律より明らかに$\exists x \in \left\{ x',x'' \right\}\left[ p(x) \right] \Leftrightarrow p\left( x' \right) \vee p\left( x'' \right)$が成り立つ。\par
de Morgan律は二重否定律より明らかである。また、省略律は公理\refeq{(A4)}と推論規則\refeq{(Gen)}より明らかである。移出律と移入律は特殊化により$\forall x \in X\left[ p(x) \right] \leftrightarrow \forall x \in X \setminus \left\{ x' \right\}\left[ p(x) \right] \land p\left( x' \right)$が成り立つことに注意すれば、定理\refeq{1.1.1.7}より明らかである。\par
次に、量化記号と選言、連言の関係を示そう。$\forall x \in X\left[ p(x) \land q(x) \right]$が成り立つなら、演繹定理と公理\refeq{(A4)}より$p(a) \land q(a)$が得られ、定理\refeq{1.1.1.5}より$p(a)$、$q(a)$が成り立つ。このとき、仮定に注意すれば、推論規則\refeq{(Gen)}に適用されることができるので、そうすることで$\forall x \in X\left[ p(x) \right]$、$\forall x \in X\left[ q(x) \right]$が得られ、再び定理\refeq{1.1.1.5}と演繹定理より$\forall x \in X\left[ p(x) \land q(x) \right] \rightarrow \forall x \in X\left[ p(x) \right] \land \forall x \in X\left[ q(x) \right]$が得られる。同様にして、$\forall x \in X\left[ p(x) \right] \land \forall x \in X\left[ q(x) \right] \rightarrow \forall x \in X\left[ p(x) \land q(x) \right]$が得られるので、$\forall x \in X\left[ p(x) \land q(x) \right] \Leftrightarrow \forall x \in X\left[ p(x) \right] \land \forall x \in X\left[ q(x) \right]$が成り立つ。対偶律より明らかに$\exists x \in X\left[ p(x) \vee q(x) \right] \Leftrightarrow \exists x \in X\left[ p(x) \right] \vee \exists x \in X\left[ q(x) \right]$が成り立つ。$\exists x \in X\left[ p(x) \land q(x) \right]$が成り立つとすれば、演繹定理と推論規則\refeq{(Gen)}より$p(a) \land q(a)$となる変数$a$が存在して定理\refeq{1.1.1.5}より$p(a)$、$q(a)$が成り立つ。また、定理\refeq{1.1.1.6}より$\exists x \in X\left[ p(x) \right]$、$\exists x \in X\left[ q(x) \right]$が得られ、再び定理\refeq{1.1.1.5}と演繹定理より$\exists x \in X\left[ p(x) \land q(x) \right] \rightarrow \exists x \in X\left[ p(x) \right] \land \exists x \in X\left[ q(x) \right]$が得られる。あとは、対偶律より明らかに$\forall x \in X\left[ p(x) \vee q(x) \right] \Leftarrow \forall x \in X\left[ p(x) \right] \vee \forall x \in X\left[ q(x) \right]$も得られる。\par
分配律と量化記号と含意の関係について、公理\refeq{(A5)}より明らかに$\forall x \in X\left[ p \rightarrow q(x) \right] \rightarrow \left( p \rightarrow \forall x \in X\left[ q(x) \right] \right)$が成り立つ。また、仮定として$p$、$p \rightarrow \forall x \in X\left[ q(x) \right]$が与えられたなら、推論規則\refeq{(MP)}より$\forall x \in X\left[ q(x) \right]$が成り立ち公理\refeq{(A4)}より$q(a)$が得られる。したがって、演繹定理より$p \rightarrow \forall x \in X\left[ q(x) \right]$が成り立つなら、$p \rightarrow q(a)$が得られる。このとき、仮定に注意すれば、推論規則\refeq{(Gen)}に適用されることができるので、そうすることで演繹定理より$\left( p \rightarrow \forall x \in X\left[ q(x) \right] \right) \rightarrow \forall x \in X\left[ p \rightarrow q(x) \right]$が成り立つ。以上より、$\forall x \in X\left[ p \rightarrow q(x) \right] \Leftrightarrow p \rightarrow \forall x \in X\left[ q(x) \right]$が成り立つ。これにより、明らかに次のことが成り立つ。
\begin{align*}
\exists x \in X\left[ p(x) \land q \right] &\Leftrightarrow \exists x \in X\left[ p(x) \right] \land q \\
\forall x \in X\left[ p(x) \vee q \right] &\Leftrightarrow \forall x \in X\left[ p(x) \right] \vee q 
\end{align*}
また、量化記号と選言、連言の関係と対偶律より明らかに次のことも成り立つ。
\begin{align*}
\exists x \in X\left[ p \rightarrow q(x) \right] \Leftrightarrow p \rightarrow \exists x \in X\left[ q(x) \right]
\end{align*}
上記の議論に定理\refeq{1.1.1.8}の対偶律が2回適用されることで、次のことも成り立つ。
\begin{align*}
\exists x \in X\left[ p(x) \rightarrow q \right] &\Leftrightarrow \forall x \in X\left[ p(x) \right] \rightarrow q \\
\forall x \in X\left[ p(x) \rightarrow q \right] &\Leftrightarrow \exists x \in X\left[ p(x) \right] \rightarrow q
\end{align*}\par
最後に、量化記号の交換について、公理\refeq{(A4)}と変数の条件に十分注意すれば、推論規則\refeq{(Gen)}により$\forall x \in X\forall y \in Y\left[ p(x,y) \right] \Leftrightarrow \forall y \in Y\forall x \in X\left[ p(x,y) \right]$が成り立つ。また、二重否定律と対偶律より明らかに$\exists x \in X\exists y \in Y\left[ p(x,y) \right] \Leftrightarrow \exists y \in Y\exists x \in X\left[ p(x,y) \right]$も成り立つ。最後のは変数$y$が変数$x$に依存するかどうかで考えられれば、推論規則\refeq{(Gen)}、演繹定理、対偶律より$\forall y \in Y\left[ p(x,a) \right]$がいえるので、量化記号と含意の関係に注意すれば、同じようにして$\forall y \in Y\exists x \in X\left[ p(x,y) \right]$が得られる。
\end{proof}
%\hypertarget{ux4ee3ux8868ux7684ux306aux63a8ux8ad6ux898fux5247}{%
\subsubsection{代表的な推論規則}%\label{ux4ee3ux8868ux7684ux306aux63a8ux8ad6ux898fux5247}}
\begin{thm}
\label{1.1.1.9}
代表的な推論規則として、いくつかの推論規則が次の表のとおりに成り立つ。ただし、$\alpha$、$\alpha_{i}$、$\beta$、$\gamma$、$\delta$はいずれも論理式であるとし$x \in X$なる変数$x$を含む開論理式を$\alpha(x)$、$\beta(x)$とおいて、$\overline{x} \in X$なる値$\overline{x}$を代入したときのそれらの論理式$\alpha\left( \overline{x} \right)$、$\beta\left( \overline{x} \right)$が閉論理式であるとする。
\begin{longtable}[c]{|c|c|}
\hline
名称 & 推論規則 \\
\hline \hline
含意除去 & $\alpha,\alpha \rightarrow \beta \vDash \beta$ \\
\hline
含意導入 & $\alpha_{1},\alpha_{2},\cdots,\alpha_{n} \vDash \beta \rightarrow \gamma\ \mathrm{if}\ \alpha_{1},\alpha_{2},\cdots,\alpha_{n},\beta \vDash \gamma$ \\
\hline
連言除去 & $\alpha_{1},\alpha_{2},\cdots,\alpha_{n} \vDash \alpha_{i}$ \\
\hline
連言導入 & $\alpha_{1},\alpha_{2},\cdots,\alpha_{n} \vDash \alpha_{1} \land \alpha_{2} \land \cdots \land \alpha_{n}$ \\
\hline
選言除去 & $\alpha_{1} \rightarrow \beta,\alpha_{2} \rightarrow \beta,\cdots,\alpha_{n} \rightarrow \beta,\alpha_{1} \vee \alpha_{2} \vee \cdots \vee \alpha_{n} \vDash \beta$ \\
\hline
選言導入 & $\alpha_{i} \vDash \alpha_{1} \vee \alpha_{2} \vee \cdots \vee \alpha_{n}$ \\
\hline
二重否定除去 & $\neg\neg\alpha \vDash \alpha$ \\
\hline
二重否定導入 & $\alpha \vDash \neg\neg\alpha$ \\
\hline
否定除去 & $\alpha,\neg\alpha \vDash \bot$ \\
\hline
否定導入 & $\alpha \rightarrow \bot \vDash \neg\alpha$ \\
\hline
後件否定 & $\alpha \rightarrow \beta,\neg\beta \vDash \neg\alpha$ \\
\hline
選言三段論法 & $\alpha \vee \beta,\neg\beta \vDash \alpha$ \\
\hline
選言肯定 & $\left( (\alpha \land \neg\beta) \vee (\neg\alpha \land \beta) \right),\alpha \vDash \neg\beta$ \\
\hline
仮言三段論法 & $\alpha \rightarrow \beta,\beta \rightarrow \gamma \vDash \alpha \rightarrow \gamma$ \\
\hline
構成的両刀論法 & $\alpha \rightarrow \beta,\gamma \rightarrow \delta,\alpha \vee \gamma \vDash \beta \vee \delta$ \\
\hline
破壊的両刀論法 & $\alpha \rightarrow \beta,\gamma \rightarrow \delta.\neg\beta \vee \neg\delta \vDash \neg\alpha \vee \neg\gamma$ \\
\hline
存在導入、存在汎化 & $\alpha\left( \overline{x} \right),\overline{x} \in X \vDash \exists x \in X\left[ A(x) \right]$ \\
\hline
存在除去、存在例化 & $\exists x \in X\left[ \alpha(x) \right] \vDash \alpha\left( \overline{x} \right) \land \overline{x} \in X$ \\
\hline
全称導入、全称汎化 & $\alpha(c),c \in X \vDash \forall x \in X\left[ \alpha(x) \right]$ \\
\hline
全称除去、全称例化 & $\forall x \in X\left[ \alpha(x) \right],c \in X \vDash \alpha(c)$ \\
\hline
全称含意除去 & $\forall x \in X\left[ \alpha(x) \rightarrow \beta(x) \right],\alpha\left( \overline{x} \right) \vDash \beta\left( \overline{x} \right)$ \\
\hline
全称後件否定 & $\forall x \in X\left[ \alpha(x) \rightarrow \beta(x) \right],\neg\beta\left( \overline{x} \right) \vDash \neg\alpha\left( \overline{x} \right)$ \\
\hline
\end{longtable}
\end{thm}
\begin{proof}
含意導入は推論規則\refeq{(MP)}そのものである。仮言三段論法は定理\refeq{1.1.1.3}より明らかである。含意除去は演繹定理そのものである。二重否定除去、二重否定導入は定理\refeq{1.1.1.4}より明らかである。連言除去、連言導入、選言導入は定理\refeq{1.1.1.5}より明らかである。否定除去について、定理\refeq{1.1.1.5}と恒偽式の定義より直ちに示される。存在導入、全称除去は定理\refeq{1.1.1.6}より明らかである。存在除去、全称導入は推論規則\refeq{(Gen)}そのものである。\par
選言除去について、定理\refeq{1.1.1.5}と定理\refeq{1.1.1.8}の分配律より$\alpha_{1} \rightarrow \beta,\alpha_{2} \rightarrow \beta,\cdots,\alpha_{n} \rightarrow \beta,\alpha_{i}$の論理和となり推論規則\refeq{(MP)}より$\alpha_{1} \rightarrow \beta,\alpha_{2} \rightarrow \beta,\cdots,\alpha_{i - 1} \rightarrow \beta,\beta,\alpha_{i + 1} \rightarrow \beta,\cdots,\alpha_{n} \rightarrow \beta$が得られ連言除去により$\beta$の論理和が得られる。したがって、定理\refeq{1.1.1.8}の省略律より結論として$\beta$が得られる。\par
次に、否定導入を示そう。恒真式の定義より次のようになる。
\begin{align*}
\top &\Leftrightarrow \alpha \vee \neg\alpha \\
&\Leftrightarrow (\alpha \vee \neg\alpha) \land (\alpha \vee \neg\alpha \vee \neg\alpha) \\
&\Leftrightarrow \left( \alpha \land (\alpha \vee \neg\alpha) \right) \vee \neg\alpha \\ 
&\Leftrightarrow \neg\left( \neg\alpha \vee (\alpha \land \neg\alpha) \right) \vee \neg\alpha \\
&\Leftrightarrow (\alpha \rightarrow \alpha \land \neg\alpha) \rightarrow \neg\alpha \\
&\Leftrightarrow (\alpha \rightarrow \bot) \rightarrow \neg\alpha
\end{align*}
演繹定理よりよって、$\alpha \rightarrow \bot \vDash \neg\alpha$が得られる。\par
次に、後件否定を示そう。定理\refeq{1.1.1.5}より次のようになる。
\begin{align*}
(\alpha \rightarrow \beta) \land \neg\beta &\Leftrightarrow (\neg\alpha \vee \beta) \land \neg\beta \\
&\Leftrightarrow (\neg\alpha \land \neg\beta) \vee (\beta \land \neg\beta) \\
&\Leftrightarrow (\neg\alpha \land \neg\beta) \vee \bot \\ 
&\Leftrightarrow \neg\alpha \land \neg\beta \\ 
&\Rightarrow \neg\alpha
\end{align*}
演繹定理よりよって、$\alpha \rightarrow \beta,\neg\beta \vDash \neg\alpha$が得られる。選言三段論法は後件否定より明らかである。\par
次に、選言肯定を示そう。定理\refeq{1.1.1.5}より次のようになる。
\begin{align*}
\left( (\alpha \land \neg\beta) \vee (\neg\alpha \land \beta) \right) \land \alpha &\Leftrightarrow (\alpha \land \alpha \land \neg\beta) \vee (\alpha \land \neg\alpha \land \beta) \\
&\Leftrightarrow (\alpha \land \neg\beta) \vee (\bot \land \beta) \\ 
&\Leftrightarrow (\alpha \land \neg\beta) \vee \bot \\
&\Leftrightarrow \alpha \land \neg\beta \\
&\Rightarrow \neg\beta
\end{align*}
演繹定理よりよって、$\left( (\alpha \land \neg\beta) \vee (\neg\alpha \land \beta) \right),\alpha \vDash \neg\beta$が得られる。\par
次に、構成的両刀論法を示そう。このとき、次のようになる。
\begin{align*}
(\alpha \rightarrow \beta) \land (\gamma \rightarrow \delta) \land (\alpha \vee \gamma) \rightarrow \beta \vee \delta &\Leftrightarrow \neg\left( (\neg\alpha \vee \beta) \land (\neg\gamma \vee \delta) \land (\alpha \vee \gamma) \right) \vee \beta \vee \delta \\
&\Leftrightarrow \neg(\neg\alpha \vee \beta) \vee \neg(\neg\gamma \vee \delta) \vee \neg(\alpha \vee \gamma) \vee \beta \vee \delta \\
&\Leftrightarrow (\alpha \land \neg\beta) \vee (\gamma \land \neg\delta) \vee (\neg\alpha \land \neg\gamma) \vee \beta \vee \delta \\
&\Leftrightarrow \left( (\alpha \land \neg\beta) \vee \beta \right) \vee \left( (\gamma \land \neg\delta) \vee \delta \right) \vee (\neg\alpha \land \neg\gamma) \\
&\Leftrightarrow \left( (\alpha \vee \beta) \land (\neg\beta \vee \beta) \right) \vee \left( (\gamma \vee \delta) \land (\delta \vee \neg\delta) \right) \vee (\neg\alpha \land \neg\gamma) \\
&\Leftrightarrow \left( (\alpha \vee \beta) \land \top \right) \vee \left( (\gamma \vee \delta) \land \top \right) \vee (\neg\alpha \land \neg\gamma) \\
&\Leftrightarrow (\alpha \vee \beta \vee \gamma \vee \delta) \vee (\neg\alpha \land \neg\gamma) \\
&\Leftrightarrow (\alpha \vee \neg\alpha \vee \beta \vee \gamma \vee \delta) \land (\alpha \vee \beta \vee \gamma \vee \neg\gamma \vee \delta) \\
&\Leftrightarrow (\top \vee \beta \vee \gamma \vee \delta) \land (\alpha \vee \beta \vee \top \vee \delta) \\
&\Leftrightarrow \top \land \top \Leftrightarrow \top
\end{align*}
定理\ref{1.1.1.5}と演繹定理よりよって、$\alpha \rightarrow \beta,\gamma \rightarrow \delta,\alpha \vee \gamma \vDash \beta \vee \delta$が得られる。対偶律より明らかに破壊的両刀論法が示される。\par
次に、全称含意除去を示そう。このとき、次のようになる。
\begin{align*}
\forall x \in X\left[ \alpha(x) \rightarrow \beta(x) \right] \land \alpha\left( \overline{x} \right) \rightarrow \beta\left( \overline{x} \right) &\Leftrightarrow \forall x \in X\left[ \neg\alpha(x) \vee \beta(x) \right] \land \alpha\left( \overline{x} \right) \rightarrow \beta\left( \overline{x} \right) \\
&\Leftrightarrow \forall x \in X\left[ \neg\alpha(x) \vee \beta(x) \right] \land \left( \neg\alpha\left( \overline{x} \right) \vee \beta\left( \overline{x} \right) \right) \land \alpha\left( \overline{x} \right) \rightarrow \beta\left( \overline{x} \right) \\
&\Leftrightarrow \forall x \in X\left[ \neg\alpha(x) \vee \beta(x) \right] \land \left( \left( \alpha\left( \overline{x} \right) \land \neg\alpha\left( \overline{x} \right) \right) \vee \left( \alpha\left( \overline{x} \right) \land \beta\left( \overline{x} \right) \right) \right) \rightarrow \beta\left( \overline{x} \right) \\
&\Leftrightarrow \forall x \in X\left[ \neg\alpha(x) \vee \beta(x) \right] \land \left( \bot \vee \left( \alpha\left( \overline{x} \right) \land \beta\left( \overline{x} \right) \right) \right) \rightarrow \beta\left( \overline{x} \right) \\
&\Leftrightarrow \forall x \in X\left[ \neg\alpha(x) \vee \beta(x) \right] \land \alpha\left( \overline{x} \right) \land \beta\left( \overline{x} \right) \rightarrow \beta\left( \overline{x} \right) \\
&\Leftrightarrow \neg\left( \forall x \in X\left[ \neg\alpha(x) \vee \beta(x) \right] \land \alpha\left( \overline{x} \right) \land \beta\left( \overline{x} \right) \right) \vee \beta\left( \overline{x} \right) \\
&\Leftrightarrow \neg\forall x \in X\left[ \neg\alpha(x) \vee \beta(x) \right] \vee \neg\alpha\left( \overline{x} \right) \vee \left( \neg\beta\left( \overline{x} \right) \vee \beta\left( \overline{x} \right) \right) \\
&\Leftrightarrow \neg\forall x \in X\left[ \neg\alpha(x) \vee \beta(x) \right] \vee \neg\alpha\left( \overline{x} \right) \vee \top \Leftrightarrow \top 
\end{align*}
\ref{1.1.1.5}と演繹定理よりよって、$\forall x \in X\left[ \alpha(x) \rightarrow \beta(x) \right],\alpha\left( \overline{x} \right) \vDash \beta\left( \overline{x} \right)$が得られる。対偶律より明らかに全称後件否定が示される。
\end{proof}
%\hypertarget{ux8a3cux660e}{%
\subsubsection{証明}%\label{ux8a3cux660e}}
証明には以下6つの代表的な手法がある。ここでは、手法の内容とその手法の証明を述べよう。なお、$p$、$q$、$r$、$p(x)$いづれも論理式であるとし$x$を取りうる範囲が$X$の変数であるとする。
\begin{thm}[条件付き証明]
\label{1.1.1.10}
1つ目は$p \vDash q \rightarrow r$の代わりに$p \land q \vDash r$を示す手法である。
\end{thm}
\begin{proof}
$p \land q \vDash r$が成り立つとき、演繹定理より次のようになる。
\begin{align*}
\top \Leftrightarrow p \land q \rightarrow r &\Leftrightarrow \neg(p \land q) \vee r \\ 
&\Leftrightarrow \neg p \vee \neg q \vee r \\
&\Leftrightarrow p \rightarrow (\neg q \vee r) \\ 
&\Leftrightarrow p \rightarrow (q \rightarrow r)
\end{align*}
したがって、$p \vDash q \rightarrow r$が成り立つ。よって、条件付き証明が示された。
\end{proof}
\begin{thm}[背理法]
\label{1.1.1.11}
2つ目は$\vDash p$の代わりに$\neg p \vDash \bot$を示す手法である\footnote{帰謬法ともいったりする。}。
\end{thm}
\begin{proof}
$\neg p \vDash \bot$が成り立つとき、演繹定理より次のようになる。
\begin{align*}
\top &\Leftrightarrow \neg p \rightarrow \bot \\ 
&\Leftrightarrow \neg\neg p \vee \bot \\ 
&\Leftrightarrow p \vee \bot \\
&\Leftrightarrow (p \land \neg p) \vee p \\
&\Leftrightarrow (p \vee p) \land (p \vee \neg p) \\
&\Leftrightarrow p \land \top \Leftrightarrow p
\end{align*}
したがって、$\vDash p$が成り立つ。よって、背理法が示された。
\end{proof}
\begin{thm}[対偶法]
\label{1.1.1.12}
3つ目は$p \vDash q$の代わりに$\neg q \vDash \neg p$を示す手法である。
\end{thm}
\begin{proof}
これは演繹定理と対偶律より明らかにである。
\end{proof}
\begin{thm}[消去法]
\label{1.1.1.13}
4つ目は$p \vDash q \vee r$の代わりに$p \land \neg q \vDash r$を示す手法である。
\end{thm}
\begin{proof}
$p \land \neg q \vDash r$が成り立つとき、演繹定理より次のようになる。
\begin{align*}
\top &\Leftrightarrow p \land \neg q \rightarrow r \\
&\Leftrightarrow \neg(p \land \neg q) \vee r \\
&\Leftrightarrow \neg p \vee \neg\neg q \vee r \\
&\Leftrightarrow \neg p \vee (q \vee r) \\
&\Leftrightarrow p \rightarrow q \vee r
\end{align*}
したがって、$p \vDash q \vee r$が成り立つ。よって、消去法が示された。
\end{proof}
\begin{thm}[場合分け論法]
\label{1.1.1.14}
5つ目は$p\left( x_{1} \right) \vee p\left( x_{2} \right) \vee \cdots \vee p\left( x_{n} \right),q \vDash r$の代わりに$\exists x \in \left\{ x_{1},\ \ x_{2},\ \ \cdots,\ \ x_{n} \right\}\left[ p(x) \right],\ \ q \vDash r$を示す手法である。
\end{thm}
\begin{proof}
これは演繹定理と定理\refeq{1.1.1.8}の特殊化より明らかである。
\end{proof}
\begin{thm}[総当たり法]
\label{1.1.1.15}
6つ目は$p\left( x_{1} \right) \vee p\left( x_{2} \right) \vee \cdots \vee p\left( x_{n} \right) \vDash q$の代わりに$\vDash p\left( x_{1} \right) \rightarrow q \vee p\left( x_{2} \right) \rightarrow q \vee \cdots \vee p\left( x_{n} \right) \rightarrow q$を示す手法である。
\end{thm}
\begin{proof}
これは選言除去と演繹定理より明らかである。
\end{proof}
\begin{thebibliography}{50}
\bibitem{1}
  WIIS. "論理-数学-ワイズ". WIIS. \url{https://wiis.info/math/logic/}, (2020-11-21 閲覧)
\bibitem{2}
  田崎晴明. "数学:物理を学び楽しむために". 学習院大学. \url{https://www.gakushuin.ac.jp/~881791/mathbook/index.html}, (2020-11-14 取得)
\bibitem{3}
  長岡亮介, 論理学で学ぶ数学, 旺文社, 2017. 初版 p16-85 ISBN978-4-01-037704-8
\bibitem{4}
  松本幸夫, 数理論理学, 共立出版, 1970. 復刊1刷 ISBN4-320-01682-3
\bibitem{5}
  高崎金久. "VI. 論理の形式化". 京都大学. \url{http://www2.yukawa.kyoto-u.ac.jp/~kanehisa.takasaki/edu/logic/logic6.html}, (2021-8-7 15:30 閲覧)
\end{thebibliography}
\end{document}
